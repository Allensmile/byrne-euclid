\input preamble.tex
\input preamble_be.tex

%\def\mpPre{textLabels := true;}

% to turn text labels on uncomment next line
% \def\mpPre{textLabels := true;}

\starttext

\setuplayout[title]

\setupheader [state=stop]

\startalignment[middle]

{\tfb \WORD{The first six books of}}

\vskip 0.25\baselineskip

{\tfc \WORD{The elements of Euclid}}

\vskip 0.5\baselineskip

{\WORD{in which coloured diagrams and symbols are used instead of letters for the greater ease of learners}}

\vskip 0.75\baselineskip

{\tfb \WORD{By Oliver Byrne}}

%{\WORD{SURVEYOR OF HER MAJESTY'S SETTLEMENTS IN THE FALKLAND ISLANDS AND AUTHOR OF NUMEROUS MATHEMATICAL WORKS}}

\defineNewPicture{
scaleFactor := 7/6;
angleScale := 4/3;
pair A, B, C, D, E, F, G, H, I, J, K, L, M, d[];
A := (0, 0);
B := A shifted (-7/10u, -8/7u);
C = whatever[A, A shifted ((A-B) rotated 90)] = whatever[B, B shifted dir(0)];
d1 := (B-A) rotated -90;
D := A shifted d1;
E := B shifted d1;
d2 := (A-C) rotated -90;
F := C shifted d2;
G := A shifted d2;
d3 := (C-B) rotated -90;
H := B shifted d3;
I := C shifted d3;
J = whatever[A, A shifted dir(90)];
J = whatever[B, C];
K = whatever[A, A shifted dir(90)];
K = whatever[H, I];
L = whatever[B, F];
L = whatever[A, C];
M = whatever[A, I];
M = whatever[B, C];
draw byPolygon(A,B,E,D)(black);
draw byPolygon(L,A,G,F)(byred);
draw byPolygon(C,L,F)(byred);
draw byPolygon(J,M,I,K)(byyellow);
draw byPolygon (M,C,I)(byyellow);
draw byPolygon(B,J,K,H)(byblue);
draw byAngle(F, C, A, byyellow, 0);
draw byAngle(B, C, I, byblue, 0);
draw byAngle(A, C, B, black, 0);
draw byLine(A, K, byred, 1, 0);
draw byLineFull(B, F, black, 0, 0)(G, G, 1, 1, -1);
draw byLineFull(A, I, black, 0, 0)(K, K, 1, 1, 1);
byLineDefine(C, F, byblue, 1, 0);
byLineDefine(C, I, black, 1, 0);
draw byNamedLineSeq(0)(CF,CI);
byLineDefine(A, B, byyellow, 0, 0);
byLineDefine(B, C, byred, 0, 0);
byLineDefine(C, A, byblue, 0, 0);
draw byNamedLineSeq(-1)(AB,BC,CA);
byLineDefineWithName (C, A, black, 0, 0)(CAb);
byLineStylize (M, M, 1, 0, -1) (CAb);
byLineDefineWithName (A, M, black, 0, 0)(AMb);
byLineStylize (C, C, 0, 1, -1) (AMb);
byLineDefineWithName (B, C, black, 0, 0)(BCb);
byLineStylize (L, L, 0, 1, -1) (BCb);
byLineDefineWithName (L, B, black, 0, 0)(BLb);
byLineStylize (C, C, 1, 0, -1) (BLb);
draw byLabelsOnPolygon(B, E, D, A, G, F, C, I, K, H)(0, 1);
draw byLabelsOnPolygon(A, J, C)(2, -1);
}

\vfill\vfill

~\hfill\reuseMPgraphic{\currentInstance::currentPicture}\hfill~

\vfill\vfill\vfill

{\tfa github.com/jemmybutton}
\vskip 0.25\baselineskip

{\tfb 2017 ed.\,0.4}
\vskip \baselineskip

\symbol[cc][cc] \symbol[cc][by] \symbol[cc][sa]

\startnarrower
\setuplocalinterlinespace[line=2ex]

{\tfxx This rendition of Oliver Byrne's \quotation{The first six books of the Elements of Euclid} is made by Slyusarev Sergey and  is distributed under CC-BY-SA 4.0 license}
\stopnarrower

\vskip -\baselineskip

\stopalignment

\pagebreak\ \pagebreak

\setupheader[state=start]
\setuplayout[reset]

\startintro[title={Introduction}]

\regularLettrine{T}\kerncharacters[-0.01]{he arts and sciences have become so extensive, that to facilitate their acquirement is of as much importance as to extend their boundaries. Illustration, if it does not shorten the time of study, will at least make it more agreeable. This Work has a greater aim than mere illustration; we do not introduce colors for the purpose of entertainment, or to amuse \emph{by certain combinations of tint and form}, but to assist the mind in its researches after truth, to increase the facilities of introduction, and to diffuse permanent knowledge. If we wanted authorities to prove the importance and usefulness of geometry, we might quote every philosopher since the day of Plato. Among the Greeks, in ancient, as in the school of Pestalozzi and others in recent times, geometry was adopted as the best gymnastic of the mind. In fact, Euclid's Elements have become, by common consent, the basis of mathematical science all over the civilized globe. But this will not appear extraordinary, if we consider that this sublime science is not only better calculated than any other to call forth the spirit of inquiry, the elevate the mind, and to strengthen the reasoning faculties, but also it forms the best introduction to most of the useful and important vocations of human life. Arithmetic, land-surveying, hydrostatics, pneumatics, optics, physical astronomy, \&c. are all dependent on the propositions of geometry.}

\kerncharacters[-0.01]{Much however depends on the first communication of any science to a learner, though the best and most easy methods are seldom adopted. Propositions are placed before a student, who though having a sufficient understanding, is told just as much about them on entering at the very threshold of the science, as given him a prepossession most unfavorable to his future study of this delightful subject; or \quotation{the formalities and paraphrenalia of rigour are so ostentatiously put forward, as almost to hide the reality. Endless and perplexing repetitions, which do not confer greater exactitude on the reasoning, render the demonstrations involved and obscure, and conceal from the view of the student the consecution of evidence.} Thus an aversion is created in the mind of the pupil, and a subject so calculated to improve the reasoning powers, and give the habit of close thinking, is degraded by a dry and rigid course of instruction into an uninteresting exercise of the memory. To rise the curiosity, and to awaken the listless and dormant powers of younger minds should be the aim of every teacher; but where examples of excellence are wanting, the attempts to attain it are but few, while eminence excites attention and produces imitation. The object of this Work is to introduce a method of teaching geometry, which has been much approved of by many scientific men in this country, as well as in France and America. The plan here adopted forcibly appeals to the eye, the most sensitive and the most comprehensive of our external organs, and its pre-eminence to important subject on the mind is supported by the incontrovertible maxim expressed in the well known words of Horace:—}

\vskip 0.5\baselineskip

\startalignment[middle]
\emph{Segnius irritant animos demissa per aurem\\
Quam quae sunt oculis subjecta fidelibus}\\
A feebler impress through the ear is made,\\
Than what is by the faithful eye conveyed.
\stopalignment

\vskip 0.5\baselineskip

\kerncharacters[-0.01]{All language consists of representative signs, and those signs are the best which effect their purposes with the greatest precision and dispatch. Such for all common purposes are the audible signs called words, which are still considered as audible, whether addressed immediately to the ear, or through the medium of letters to the eye. Geometrical diagrams are not signs, but the materials of geometrical science, the object of which is to show the relative quantities of their parts by a process of reasoning called Demonstration. This reasoning has been generally carried on by words, letters, and black or uncoloured diagrams; but as the use of coloured symbols, signs, and diagrams in the linear arts and sciences, renders the process of reasoning more precise, and the attainment more expeditious, they have been in this instance accordingly adopted.}

\kerncharacters[0.01]{Such is the expedition of this enticing mode of communicating knowledge, that the Elements of Euclid can be acquired in less that one third of the time usually employed, and the retention by the memory is much more permanent; these facts have been ascertained by numerous experiments made by the inventor, and several others who have adopted his plans. The particulars of which are few and obvious; the letters annexed to points, lines, or other parts of a diagram are in fact but arbitrary names, and represent them in the demonstration; instead of these, the parts being differently coloured, are made to name themselves, for their forms in corresponding colours represent them in the demonstration.}

\defineNewPicture{
	pair A, B, C;
	B := (0, 0);
	A := B shifted (dir(-145)*3u);
	C = whatever[A, A shifted (1,0)] = whatever[B, B shifted dir(-145+90)];
		draw byAngleWithName(A, B, C, byyellow, 0)(B);
		draw byAngleWithName(B, C, A, byblue, 0)(C);
		draw byAngleWithName(C, A, B, byred, 0)(A);
		byLineDefine(A, B, byblue, 0, 0);
		byLineDefine(B, C, byred, 0, 0);
		byLineDefine(C, A, byyellow, 0, 0);
		draw byNamedLineSeq(0)(AB,BC,CA);
	label.top(\sometxt{B}, B);
		label.rt(\sometxt{C}, C);
		label.lft(\sometxt{A}, A);
}

\drawCurrentPictureInMargin
In order to give a better idea of this system, and of advantages gained by its adoption, let us take a right angled triangle, and express some of its properties both by colours and the method generally employed.

\vskip \baselineskip

\startalignment[middle]
\emph{Some of the properties of the right angled triangle ABC, expressed by the method generally employed:}
\stopalignment

\startitemize[m,joinedup,nowhite]
\item The angle BAC, together with the angles BCA and ABC are equal to two right angles, or twice the angle ABC.
\item The angle CAB added to the angle ACB will be equal to the angle ABC.
\item The angle ABC is greater than either of the angles BAC or BCA.
\item The angle BCA or the angle CAB is less than the angle ABC.
\item If from the angle ABC, there be taken the angle BAC, the remainder will be equal to the angle ACB.
\item The square of AC is equal to the sum of the squares of AB and BC.
\stopitemize

\vskip \baselineskip

\startalignment[middle]
\emph{The same properties expressed by colouring the different parts:}
\stopalignment

\startitemize[m,joinedup,nowhite]
\item $\drawAngle{A} + \drawAngle{B} + \drawAngle{C} = 2 \drawAngle{B} = \drawTwoRightAngles$. \\ That is, the red angle added to the yellow angle added to the blue angle, equal twice the yellow angle, equal two right angles.
\item $\drawAngle{A} + \drawAngle{C} = \drawAngle{B}$. \\ Or in words, the red angle added to the blue angle, equal the yellow angle.
\item $\drawAngle{B} > \drawAngle{A} \mbox{ or } > \drawAngle{C}$. \\ The yellow angle is greater than either the red or blue angle.
\item $\drawAngle{A} \mbox{ or } \drawAngle{C} < \drawAngle{B}$. \\ Either the red or blue angle is less that the yellow angle.
\item $\drawAngle{B} \mbox{ minus } \drawAngle{C} = \drawAngle{A}$. \\ In other terms, the yellow angle made less be the blue angle equal red angle.
\item $\drawUnitLine{CA}^2 = \drawUnitLine{AB}^2 + \drawUnitLine{BC}^2$. \\ That is, the square of the yellow line is equal to the sum of the squares of the blue and red lines.
\stopitemize

\vfill\pagebreak

In oral demonstrations we gain with colours this important advantage, the eye and the ear can be addressed at the same moment, so that for teaching geometry, and other linear arts and sciences, in classes, the system is best ever proposed, this is apparent from the examples given.

\kerncharacters[-0.015]{Whence it is evident that a reference from the test to the diagram is more rapid and sure, by giving the forms and colours of the parts, or by naming the parts and their colours, than naming the parts and letters on the diagram. Besides the superior simplicity, this system is likewise conspicuous for concentration, and wholly excludes the injurious though prevalent practice of allowing the student to commit the demonstration to memory; until reason, and fact, and proof only make impressions of the understanding.}

Again, when lecturing on the principles or properties of figures, if we mention the colour of the part or parts referred to, as in saying, the red angle, the blue line, or lines, \&c, the part or parts thus named will be immediately seen by all the class at the same instant; not so if we say the angle ABC, the triangle PFQ, the figure EGKt, and so on; for the letters must be traced one by one before students arrange in their minds the particular magnitude referred to, which often occasions confusion and error, as well as loss of time. Also if the parts which are given as equal, have the same colours in any diagram, the mind will not wander from the object before it; that is, such an arrangement presents an ocular demonstration of the parts to be proved equal, and the learner retains the data throughout the whole of reasoning. But whatever may be the advantages of the present plan, if it be not substituted for, it can always be made a powerful auxiliary to the other methods, for the purpose of introduction, or of a more speedy reminiscence, or of more permanent retention by the memory.

The experience of all who have formed systems to impress facts on the understanding, agree in proving that coloured representations, as pictures, cuts, diagrams, \&c. are more easily fixed in the mind than mere sentences unmarked by any peculiarity. Curious as it may appear, poets seem to be aware of this fact more than mathematicians; many modern poets allude to this visible system of communicating knowledge, one of them has thus expressed himself:

\vskip 0.5\baselineskip

\startalignment[middle]
Sounds which address the ear are lost and die\\
In one short hour, but these which strike the eye,\\
Live long upon the mind, the faithful sight\\
Engraves the knowledge with a beam of light.
\stopalignment

\vskip 0.5\baselineskip

This perhaps may be reckoned the only improvement which plain geometry has received since the days of Euclid, and if there were any geometers of note before that time, Euclid's success has quite eclipsed their memory, and even occasioned all good things of that kind to be assigned to him; like \AE sop among the writers of Fables. It may also be worthy of remark, as tangible diagrams afford the only medium through which geometry and other linear arts can be taught to the blind, the visible system is no less adapted to the exigencies of the deaf and dumb.

Care must be taken to show that colour has nothing to do with the lines, angles, or magnitudes, except merely to name them. A mathematical line, which is length without breadth, cannot possess colour, yet the junction of two colours on the same plane gives a good idea of what is meant by a mathematical line; recollect we are speaking familiarly, such a junction is to be understood and not the colour, when we say the black line, the red line or lines, \&c.

Colours and coloured diagrams may at first appear a clumsy method to convey proper notions of the properties and parts of mathematical figures and magnitudes, however they will be found to afford a means more refined and extensive than any that has hitherto proposed.

We shall here define a point a line, and a surface, and demonstrate a proposition in order to show the truth of this assertion.

A point is that which has position, but not magnitude; or a point is position only, abstracted from the consideration of length, breadth, and thickness. Perhaps the following description is better calculated to explain the nature of mathematical point to those who have not acquired the idea, than the above specious definition.

\defineNewPicture{
	angleScale := 2;
	pair O, A, B, C;
	O := (0, 0);
	A := dir(0) scaled 3u;
	B := dir(120) scaled 3u;
	C := dir(240) scaled 3u;
		draw byAngle(A, O, B, byred, 0);
		draw byAngle(B, O, C, byblue, 0);
		draw byAngle(C, O, A, byyellow, 0);
}
Let three colours \drawCurrentPictureInMargin meet and cover a portion of the paper, where they meet is not blue, nor is it yellow, nor is it red, as it occupies no portion of the plane, for if it did, it would belong to the blue, the red, or the yellow part; yet it exists, and has position without magnitude, so that with a little reflection, this junction of three colours on a plane, gives a good idea of a mathematical point.

A line is length without breadth. With the assistance of colours, nearly in the same manner as before, an idea of a line may be thus given:—

\defineNewPicture{
	pair A, B, C, D, E, F;
	A := (0, 0);
	B := (5/2u, ypart(A));
	C := (xpart(A), -2u);
	D := (xpart(B), ypart(C));
	E := 1/2[A, C];
	F := 1/2[B, D];
		draw byPolygon(A,B,F,E)(byred);
		draw byPolygon(C,D,F,E)(byblue);
}
\drawCurrentPictureInMargin
Let two colours meet and cover a portion of paper; where they meet is not red, nor is it blue; therefore the junction occupies no portion of the plane, and therefore it cannot have breadth, but only length: from which we can readily form an idea of what is meant by a mathematical line. For the purpose of illustration, one colour differing from the colour of the paper, or plane upon which it is drawn, would have been sufficient; hence in future, if we say the red line, the blue line or lines, \&c. it is the junctions with the plane upon which they are drawn are to be understood.

\defineNewPicture{
	pair A', A'', A''', B', B'', B''', C', C'', C''', D', D'', D''', d[];
	d1 := (3/2u, 0);
	d2 := (-3/4u, -2/3u);
	d3 := (0, -3/2u);
	A' := (0, 0);
	B' := A' shifted d1;
	C' := A' shifted d2;
	D' := C' shifted d1;
	A'' := A' shifted d3;
	B'' := B' shifted d3;
	C'' := C' shifted d3;
	D'' := D' shifted d3;
	A''' := A'' shifted d3;
	B''' := B'' shifted d3;
	C''' := C'' shifted d3;
	D''' := D'' shifted d3;
		draw byPolygon(A',B',B'',A'',C'',C')(byred);
		draw byPolygon(A'',B'',D'',C'')(byblue);
		draw byPolygon(C'',D'',B'',B''',D''',C''')(byyellow);
		label.lft(\sometxt{P}, C');
		label.lft(\sometxt{R}, C'');
		label.rt(\sometxt{S}, B'');
		label.rt(\sometxt{Q}, B''');
}
\drawCurrentPictureInMargin
Surface if that which has length and breadth without thickness.

\kerncharacters[-0.02]{When we consider a solid body (PQ), we perceive at once that it has three dimensions, namely :— length, breadth, and thickness; suppose one part of this solid (PS) to be red, and the other part (QR) yellow, and that the colours be distinct without commingling, the blue surface (RS) which separates these parts, or which is the same thing, that which divides the solid without loss of material, must be without thickness, and only possesses length and breadth; this plainly appears from reasoning, similar to that just employed in defining, or rather describing a point and a line.}

The proposition which we have selected to elucidate the manner in which the principles are applied, is the fifth of the first Book.

\defineNewPicture[1/4]{
angleScale := 5/6;
pair A, B, C, D, E;
A := (0, 0);
B := A shifted (u, -2u);
C := B xscaled -1;
D := 9/5[A,B];
E := 9/5[A,C];
draw byAngle(B, A, C, black, 0);
draw byAngle(A, B, C, byblue, 0);
draw byAngle(B, C, A, byblue, 0);
draw byAngle(C, B, E, byyellow, 0);
draw byAngle(D, C, B, byyellow, 0);
draw byAngle(B, D, C, byred, 0);
draw byAngle(C, E, B, byred, 0);
byAngleDefine(E, B, D, black, 1);
byAngleDefine(D, C, E, black, 1);
byLineDefine(B, D, byyellow, 0, 0);
byLineDefine(C, E, byyellow, 0, 0);
byLineDefine(B, E, byblue, 0, 0);
byLineDefine(C, D, byblue, 0, 0);
byLineDefine(A, B, byred, 0, 0);
byLineDefine(A, C, byred, 0, 0);
byLineDefine(B, C, black, 0, 0);
draw byNamedLineSeq(0)(CD,noLine,BC,noLine,BE,CE,AC,AB,BD);
label.top(\sometxt{A}, A);
label.lft(\sometxt{C}, C);
label.rt(\sometxt{B}, B);
label.lft(\sometxt{E}, E);
label.rt(\sometxt{D}, D);
}
\drawCurrentPictureInMargin
In an isosceles triangle ABC. the internal angles at the base ABC, ACB are equal, and when the sides AB, AC are produced, the external angles at the base BCE, CBD are also equal.

\startCenterAlign
Produce \drawUnitLine{AB} and \drawUnitLine{AC},\\
make $\drawUnitLine{BD} = \drawUnitLine{CE}$, draw \drawUnitLine{BE} and \drawUnitLine{CD}.\\
In
\drawFromCurrentPicture{
draw byNamedAngle(BAC);
startAutoLabeling;
draw byNamedLineSeq(0)(BE,CE,AC,AB);
stopAutoLabeling;
}
and
\drawFromCurrentPicture{
draw byNamedAngle(BAC);
startAutoLabeling;
draw byNamedLineSeq(0)(BD,CD,AC,AB);
stopAutoLabeling;
}\\
we have $\drawUnitLine{AB,BD} = \drawUnitLine{AC,CE}$,\\
\drawAngle{BAC} common and $\drawUnitLine{AB} = \drawUnitLine{AC}$:\\
$\therefore \drawAngle{BCA,DCB} = \drawAngle{ABC,CBE}$, $\drawUnitLine{BE} = \drawUnitLine{CD}$\\
and $\drawAngle{CEB} = \drawAngle{BDC}$ \inprop[prop:I.IV].\\
Again in \drawLine{BC,BE,CE} and \drawLine{BC,BD,CD},\\
$\drawUnitLine{BD} = \drawUnitLine{CE}$, $\drawAngle{CEB} = \drawAngle{BDC}$\\
and $\drawUnitLine{BE} = \drawUnitLine{CD}$;\\
$\therefore \drawAngle{DCE,DCB} = \drawAngle{EBD,CBE}$
and $\drawAngle{DCB} = \drawAngle{CBE}$ \inprop[prop:I.IV]\\
But $\drawAngle{BCA,DCB} = \drawAngle{ABC,CBE}$, $\therefore \drawAngle{BCA} = \drawAngle{ABC}$.
\stopCenterAlign

\qedNB

\startalignment[middle]
\emph{By annexing Letters to the Diagram.}
\stopalignment

Let the equal sides AB and AC be produced through the extremities BC, of the third side, and in the produced part BD of either, let any point D be assumed, and from the other let AE be cut off equal to AD \inprop[prop:I.III]. Let points E and D, so taken in the produced sides, be connected by straight lines DC and BE with the alternate extremities of the third side of the triangle.

In the triangles DAC and EAB the sides DA and AC are respectively equal to EA and AB, and the included angle A is common to both triangles. Hence \inprop[prop:I.IV] the line DC is equal to BE, the angle ADC to the angle AEB, and the angle ACD to the angle ABE; if from the equal lines AD and AE the equal sides AB and AC be taken, the remainders BD and CE will be equal. Hence in the triangles BDC and CEB, the sides BD and DC are respectively equal to CE and EB, and the angles D and E included by those sides are also equal. Hence \inprop[prop:I.IV] the angles DBC and ECB, which are those included by the third side BC and the productions of the equal sides AB and AC are equal. Also the angles DCB and EBC are equal if those equals be taken from the angles DCA and EBA before proved equal, the remainders, which are the angles ABC and ACD opposite to the equal sides, will be equal.

\emph{Therefore in an isosceles triangle,} \&c.

\qedNB

Our object in this place being to introduce system rather than to teach any particular set of propositions, we have therefore selected the foregoing out of the regular course. For schools and other public places of instruction, dyed chalks will answer to describe the diagrams, \&c. for private use coloured pencils will be found very convenient.

We are happy to find that the Elements of Mathematics now forms a considerable part of every sound female education, therefore we call the attention to those interested or engaged in the education of ladies to this very attractive mode of communicating knowledge, and to the succeeding work for its future developement.

We shall for the present conclude by observing, as the senses of sight and hearing can be so forcibly and instantaneously addressed alike with one thousand as with one, \emph{the million} might be taught geometry and other branches of mathematics with great ease, this would advance the purpose of education more than any thing \emph{might} be named, for it would teach the people how to think, and not what to think; it is in this particular the great error of education originates.

\vfill\pagebreak

\startsupersection[title={Definitions}]

\startDefinitionOnlyNumber[reference=def:I]
\startalignment[last]
A \emph{point} is that which has no parts.
\stopalignment
\stopDefinitionOnlyNumber

\startDefinitionOnlyNumber[reference=def:II]
\startalignment[last]
A \emph{line} is length without breadth.
\stopalignment
\stopDefinitionOnlyNumber

\startDefinitionOnlyNumber[reference=def:III]
\startalignment[last]
The extremities of a line are points.
\stopalignment
\stopDefinitionOnlyNumber

\startDefinitionOnlyNumber[reference=def:IV]
\startalignment[last]
The straight or right line is that which lies evenly between its extremities.
\stopalignment
\stopDefinitionOnlyNumber

\startDefinitionOnlyNumber[reference=def:V]
\startalignment[last]
A surface is that which has length and breadth only.
\stopalignment
\stopDefinitionOnlyNumber

\startDefinitionOnlyNumber[reference=def:VI]
\startalignment[last]
The extramities of a surface are lines.
\stopalignment
\stopDefinitionOnlyNumber

\startDefinitionOnlyNumber[reference=def:VII]
\startalignment[last]
A plane surface is that which lies evenly between its extremities.
\stopalignment
\stopDefinitionOnlyNumber

\startDefinitionOnlyNumber[reference=def:VIII]
\startalignment[last]
A plane angle is the inclination of two lines to one another, in a plane, which meet together, but are not in the same direction.
\stopalignment
\stopDefinitionOnlyNumber

\defineNewPicture{
	pair A, B, C;
	A := (0, 0);
	B := (u, u);
	C := (3/2u, ypart(A));
		draw byAngleWithName(B, A, C, byyellow, 0)(A);
		byLineDefine(A, B, byblue, 0, 0);
		byLineDefine(A, C, byred, 0, 0);
		draw byNamedLineSeq(0)(AC,AB);
}
\startDefinitionOnlyNumber[reference=def:IX]
\drawCurrentPictureInMargin[inside]
\startalignment[last]
A plane rectilinear angle is the inclination of two straight lines to one another, which meet together, but are not in the same straight line.
\stopalignment
\stopDefinitionOnlyNumber

\defineNewPicture{
	pair A, B, C, D;
	A := (0, 0);
	B := (4/3u, 0);
	C := (0, u);
	D := (-4/3u, 0);
		draw byAngle(B, A, C, black, 1);
		draw byAngle(D, A, C, black, 1);
		draw byLine(D, B, black, 0, 0);
		draw byLine(A, C, black, 0, 0);
}
\startDefinitionOnlyNumber[reference=def:X]
\drawCurrentPictureInMargin[inside]
\startalignment[last]
When one straight line standing on another straight line makes the adjacent angles equal, each of these angles is called a \emph{right angle}, and each of these lines is said to be \emph{perpendicular} to one another
\stopalignment
\stopDefinitionOnlyNumber

\defineNewPicture{
	pair A, B, C;
	A := (0, 0);
	B := (-u, u);
	C := (3/2u, ypart(A));
		draw byAngleWithName(B, A, C, byred, 0)(A);
		byLineDefine(A, B, byyellow, 0, 0);
		byLineDefine(A, C, byblue, 0, 0);
		draw byNamedLineSeq(0)(AC,AB);
}
\drawCurrentPictureInMargin[inside]
\startDefinitionOnlyNumber[reference=def:XI]
\startalignment[last]
An obtuse angle is an angle greater than a right angle
\stopalignment
\stopDefinitionOnlyNumber

\defineNewPicture{
	pair A, B, C;
	A := (0, 0);
	B := (u, u);
	C := (3/2u, ypart(A));
		draw byAngleWithName(B, A, C, byblue, 0)(A);
		byLineDefine(A, B, byyellow, 0, 0);
		byLineDefine(A, C, byred, 0, 0);
		draw byNamedLineSeq(0)(AC,AB);
}
\drawCurrentPictureInMargin[inside]
\startDefinitionOnlyNumber[reference=def:XII]
\startalignment[last]
An acute angle is an angle less than a right angle.
\stopalignment
\stopDefinitionOnlyNumber

\startDefinitionOnlyNumber[reference=def:XII]
\startalignment[last]
A term or boundary is the extremity of any thing.
\stopalignment
\stopDefinitionOnlyNumber

\startDefinitionOnlyNumber[reference=def:XII]
\startalignment[last]
A figure is a surface enclosed on all sides by a line or lines.
\stopalignment
\stopDefinitionOnlyNumber

\defineNewPicture{
	pair O, A, B, C, D, E;
	numeric r;
	r := 2/3u;
	O := (0, 0);
	A := dir(0) scaled r;
	B := dir(60) scaled r;
	C := dir(130) scaled r;
	D := dir(180) scaled r;
	E := dir(-60) scaled r;
		draw byLine(O, B)(black, 0, 0);
		draw byLine(O, C)(byred, 0, 0);
		draw byLine(O, E)(byyellow, 0, 0);
		draw byLine(D, A)(byblue, 0, 0);
		draw byCircleR(O, r, byred, 0, 0, 0)(O);
}
\startDefinitionOnlyNumber[reference=def:XV]
\drawCurrentPictureInMargin[inside]
\startalignment[last]
A circle is a plane figure, bounded by one continued line, called its circumference or periphery; and having a certain point within it, from which all straight lines drawn to its circumference are equal.
\stopalignment
\stopDefinitionOnlyNumber

\startDefinitionOnlyNumber[reference=def:XVI]
\startalignment[last]
This point (from which the equal lines are drawn) is called the centre of the circle.
\stopalignment
\stopDefinitionOnlyNumber

\defineNewPicture{
	pair O, A, B;
	numeric r;
	r := 3/4u;
	O := (0, 0);
	A := dir(0) scaled r;
	B := dir(180) scaled r;
		draw byLine(A, B)(byyellow, 0, 0);
		draw byCircleR(O, r, byred, 0, 0, 0)(O);
}
\startDefinitionOnlyNumber[reference=def:XVII]
\drawCurrentPictureInMargin[inside]
\startalignment[last]
A diameter of a circle is a straight line drawn through the centre, terminated both ways in the circumference.
\stopalignment
\stopDefinitionOnlyNumber

\defineNewPicture{
	pair O, A, B;
	numeric r;
	r := 3/4u;
	O := (0, 0);
	A := dir(0) scaled r;
	B := dir(180) scaled r;
		draw byLine(A, B)(byblue, 0, 0);
		draw byArc(O, A, B)(r, byyellow, 0, 0, 0, 0)(O);
		draw byArc(O, B, A)(r, byyellow, 1, 0, 0, 0)(O);
}
\startDefinitionOnlyNumber[reference=def:XVIII]
\drawCurrentPictureInMargin[inside]
\startalignment[last]
A semicircle is the figure contained by the diameter, and the part of the circle cut off by the diameter.
\stopalignment
\stopDefinitionOnlyNumber

\defineNewPicture{
	pair O, A, B;
	path P;
	numeric r;
	r := 3/4u;
	P := fullcircle scaled 2r;
	O := (0, 0);
	A := point 1 of P;
	B := point 3 of P;
		draw byLine(A, B)(byred, 0, 0);
		draw byArc(O, A, B)(r, byblue, 0, 0, 0, 0)(O);
		draw byArc(O, B, A)(r, byblue, 1, 0, 0, 0)(O);
}
\startDefinitionOnlyNumber[reference=def:XIX]
\drawCurrentPictureInMargin[inside]
\startalignment[last]
A segment of a circle is a figure contained by straight line and the part of the circumference which it cuts off.
\stopalignment
\stopDefinitionOnlyNumber

\startDefinitionOnlyNumber[reference=def:XX]
\startalignment[last]
A figure contained by straight lines only, is called a rectilinear figure.
\stopalignment
\stopDefinitionOnlyNumber

\startDefinitionOnlyNumber[reference=def:XXI]
\startalignment[last]
A triangle is a rectilinear figure included by three sides.
\stopalignment
\stopDefinitionOnlyNumber

\defineNewPicture{
	pair A, B, C, D;
	A := (0, 0);
	B := (u, 1/2u);
	C := (-1/2u, -4/3u);
	D := (4/3u, -u);
		draw byLine(C, B)(byred, 0, 0);
		draw byLine(A, D)(byblue, 0, 0);
		byLineDefine(A, B, byyellow, 0, 0);
		byLineDefine(A, C, byyellow, 0, 0);
		byLineDefine(B, D, byyellow, 0, 0);
		byLineDefine(C, D, black, 0, 0);
		draw byNamedLineSeq(0)(BD,CD,AC,AB);
}
\startDefinitionOnlyNumber[reference=def:XXII]
\drawCurrentPictureInMargin[inside]
\startalignment[last]
A quadrilateral figure is one which is bounded by four sides. The straight lines \drawUnitLine{AD} and \drawUnitLine{CB} connecting the vertices of the opposite angles of a quadrilateral figure, are called its diagonals.
\stopalignment
\stopDefinitionOnlyNumber

\startDefinitionOnlyNumber[reference=def:XXIII]
\startalignment[last]
A polygon is a rectilinear figure bounded by more than four sides.
\stopalignment
\stopDefinitionOnlyNumber

\defineNewPicture{
	pair A, B, C;
	A := dir(-30) scaled 1/2u;
	B := dir(-150) scaled 1/2u;
	C := dir(90) scaled 1/2u;
		byLineDefine(A, B, byblue, 0, 0);
		byLineDefine(B, C, byred, 0, 0);
		byLineDefine(C, A, byyellow, 0, 0);
		draw byNamedLineSeq(0)(AB,BC,CA);
}
\startDefinitionOnlyNumber[reference=def:XXIV]
\drawCurrentPictureInMargin[inside]
\startalignment[last]
A triangle whose sides are equal, is said to be equilateral.
\stopalignment
\stopDefinitionOnlyNumber

\defineNewPicture{
	pair A, B, C;
	A := dir(-60) scaled 1/2u;
	B := dir(-120) scaled 1/2u;
	C := dir(90) scaled 1/2u;
		byLineDefine(A, B, byblue, 0, 0);
		byLineDefine(B, C, byred, 0, 0);
		byLineDefine(C, A, byred, 0, 0);
		draw byNamedLineSeq(0)(AB,BC,CA);
}
\startDefinitionOnlyNumber[reference=def:XXV]
\drawCurrentPictureInMargin[inside]
\startalignment[last]
A triangle which has only two sides equal is called an isosceles triangles.
\stopalignment
\stopDefinitionOnlyNumber

\startDefinitionOnlyNumber[reference=def:XXVI]
\startalignment[last]
A scalene triangle is one which has no two sides equal.
\stopalignment
\stopDefinitionOnlyNumber

\defineNewPicture{
	pair A, B, C;
	A := (0, 0);
	B := (-u, 0);
	C := (0, 3/4u);
		byLineDefine(A, B, byred, 0, 0);
		byLineDefine(B, C, byyellow, 0, 0);
		byLineDefine(C, A, byblue, 0, 0);
		draw byNamedLineSeq(0)(AB,BC,CA);
}
\startDefinitionOnlyNumber[reference=def:XXVII]
\drawCurrentPictureInMargin[inside]
\startalignment[last]
A right angled triangle is that which has a right angle.
\stopalignment
\stopDefinitionOnlyNumber

\defineNewPicture{
	pair A, B, C;
	A := (-1/4u, 0);
	B := (-u, 0);
	C := (0, 3/4u);
		byLineDefine(A, B, byred, 0, 0);
		byLineDefine(B, C, byblue, 0, 0);
		byLineDefine(C, A, byyellow, 0, 0);
		draw byNamedLineSeq(0)(AB,BC,CA);
}
\startDefinitionOnlyNumber[reference=def:XXVIII]
\drawCurrentPictureInMargin[inside]
\startalignment[last]
An obtuse angled triangle is that which has an obtuse angle.
\stopalignment
\stopDefinitionOnlyNumber

\defineNewPicture{
	pair A, B, C;
	A := (0, 0);
	B := (-u, 0);
	C := (-1/4u, 3/4u);
		byLineDefine(A, B, byblue, 0, 0);
		byLineDefine(B, C, byyellow, 0, 0);
		byLineDefine(C, A, byred, 0, 0);
		draw byNamedLineSeq(0)(AB,BC,CA);
}
\startDefinitionOnlyNumber[reference=def:XXIX]
\drawCurrentPictureInMargin[inside]
\startalignment[last]
An acute angled triangle is that which has three acute angles.
\stopalignment
\stopDefinitionOnlyNumber

\defineNewPicture{
	pair A, B, C, D;
	numeric s;
	s := u;
	A := (0, 0);
	B := (s, 0);
	C := (0, s);
	D := (s, s);
		byLineDefine(A, B, byred, 0, 0);
		byLineDefine(A, C, byblue, 0, 0);
		byLineDefine(B, D, byyellow, 0, 0);
		byLineDefine(C, D, black, 0, 0);
		draw byNamedLineSeq(0)(AB,AC,CD,BD);
}
\startDefinitionOnlyNumber[reference=def:XXX]
\drawCurrentPictureInMargin[inside]
\startalignment[last]
Of four-sided figures, a square is that which has all its sides equal, and all its angles right angles.
\stopalignment
\stopDefinitionOnlyNumber

\defineNewPicture{
	pair A, B, C, D;
	numeric s;
	s := u;
	A := (0, 0);
	B := (s, 0);
	C := A shifted (dir(80) scaled s);
	D := B shifted (dir(80) scaled s);
		byLineDefine(A, B, byred, 0, 0);
		byLineDefine(A, C, byblue, 0, 0);
		byLineDefine(B, D, byyellow, 0, 0);
		byLineDefine(C, D, black, 0, 0);
		draw byNamedLineSeq(0)(AB,AC,CD,BD);
}
\startDefinitionOnlyNumber[reference=def:XXXI]
\drawCurrentPictureInMargin[inside]
\startalignment[last]
A rhombus is that which has all its sides equal, but its angles are not right angles.
\stopalignment
\stopDefinitionOnlyNumber

\defineNewPicture{
	pair A,B,C,D;
	numeric s;
	s := u;
	A := (0, 0);
	B := (4/3s, 0);
	C := (0, 3/4s);
	D := (4/3s, 3/4s);
		byLineDefine(A, B, byblue, 0, 0);
		byLineDefine(A, C, byred, 0, 0);
		byLineDefine(B, D, byred, 0, 0);
		byLineDefine(C, D, byblue, 0, 0);
		draw byNamedLineSeq(0)(AB,AC,CD,BD);
}
\startDefinitionOnlyNumber[reference=def:XXXII]
\drawCurrentPictureInMargin[inside]
\startalignment[last]
An oblong is that which has all its angles right angles, but has not all its sides equal.
\stopalignment
\stopDefinitionOnlyNumber

\defineNewPicture{
	pair A, B, C, D;
	numeric s;
	s := u;
	A := (0, 0);
	B := (s, 0);
	C := (1/4s, 3/4s);
	D := (s + 1/4s, 3/4s);
		byLineDefine(A, B, byblue, 0, 0);
		byLineDefine(A, C, byred, 0, 0);
		byLineDefine(B, D, byred, 0, 0);
		byLineDefine(C, D, byblue, 0, 0);
		draw byNamedLineSeq(0)(AB,AC,CD,BD);
}
\startDefinitionOnlyNumber[reference=def:XXXIII]
\drawCurrentPictureInMargin[inside]
\startalignment[last]
A rhomboid is that which has its opposite sides equal to one another, but all its sides are not equal nor its angles right angles.
\stopalignment
\stopDefinitionOnlyNumber

\startDefinitionOnlyNumber[reference=def:XXXIV]
\startalignment[last]
All other quadrilateral figures are called trapezimus.
\stopalignment
\stopDefinitionOnlyNumber

\defineNewPicture{
	pair A, B, C, D;
	numeric s;
	s := u;
	A := (0, 0);
	B := (4/3s, 0);
	C := (0, 1/2s);
	D := (4/3s, 1/2s);
		draw byLine(A, B, byred, 0, 0);
		draw byLine(C, D, byyellow, 0, 0);
}
\startDefinitionOnlyNumber[reference=def:XXXV]
\drawCurrentPictureInMargin[inside]
\startalignment[last]
Parallel straight lines are such as are in the same plane, and which being produced continually in both directions would never meet.
\stopalignment
\stopDefinitionOnlyNumber
\stopsupersection

\startsupersection[title={Postulates}]

\startPostulateOnlyNumber[reference=post:I]
Let it be granted that a straight line may be drawn from any one point to any other point.
\stopPostulateOnlyNumber

\startPostulateOnlyNumber[reference=post:II]
Let it be granted that a finite straight line may be produced to any length in a straight line.
\stopPostulateOnlyNumber

\startPostulateOnlyNumber[reference=post:III]
Let it be granted that a circle may be described with any centre at any distance from that centre.
\stopPostulateOnlyNumber
\stopsupersection

\vfill\pagebreak

\startsupersection[title={Axioms}]

\startAxiomOnlyNumber[reference=ax:I]
Magnitudes which are equal to the same are equal to each other.
\stopAxiomOnlyNumber

\startAxiomOnlyNumber[reference=ax:II]
If equals be added to equals the sums will be equal.
\stopAxiomOnlyNumber

\startAxiomOnlyNumber[reference=ax:III]
If equals be taken away from equals the remainders will be equal.
\stopAxiomOnlyNumber

\startAxiomOnlyNumber[reference=ax:IV]
If equals be added to unequals the sums will be unequal.
\stopAxiomOnlyNumber

\startAxiomOnlyNumber[reference=ax:V]
If equals be taken away from unequals the remainders will be unequal.
\stopAxiomOnlyNumber

\startAxiomOnlyNumber[reference=ax:VI]
The doubles of the same or equal magnitudes are equal.
\stopAxiomOnlyNumber

\startAxiomOnlyNumber[reference=ax:VII]
The halves of the same or equal magnitudes are equal.
\stopAxiomOnlyNumber

\startAxiomOnlyNumber[reference=ax:VIII]
The magnitudes which coincide with one another, or exactly fill the same space, are equal.
\stopAxiomOnlyNumber

\startAxiomOnlyNumber[reference=ax:IX]
The whole is greater than its part.
\stopAxiomOnlyNumber

\startAxiomOnlyNumber[reference=ax:X]
Two straight lines cannot include a space.
\stopAxiomOnlyNumber

\startAxiomOnlyNumber[reference=ax:XI]
All right angles are equal.
\stopAxiomOnlyNumber

\startAxiomOnlyNumber[reference=ax:XII]
\defineNewPicture{
	pair A, B, C, D, E, F, G, H;
	numeric s;
	s := 3/2u;
	A := (0, 0);
	B := (4/3s, 0);
	C := (0, s);
	D := (4/3s, s);
	E := (1/3s, 7/6s);
	F := (xpart(E), -1/6s);
	G = whatever[A, B] = whatever[E, F];
	H = whatever[C, D] = whatever[E, F];
		draw byAngleWithName(B, G, E, byred, 0)(G);
		draw byAngleWithName(D, H, F, byyellow, 0)(H);
		draw byLine(A, B, byblue, 0, 0);
		draw byLine(C, D, byred, 0, 0);
		draw byLine(E, F, black, 0, 0);
}
\drawCurrentPictureInMargin[inside]
If two straight lines $\left(\vcenter{\nointerlineskip\hbox{\drawUnitLine{AB}}\nointerlineskip\hbox{\drawUnitLine{CD}}}\right)$ meet a third straight line (\drawUnitLine{EF}) so as to make the two interior angles (\drawAngle{H} and \drawAngle{G}) on the same side less than two straight angles, these two straight lines will meet if they be produced on that side on which the angles are less than two right angles.
\stopAxiomOnlyNumber
\stopsupersection

\vfill\pagebreak

\startsupersection[title={Euclidations}]

The twelfth axiom may be expressed in any of the following ways:
\startitemize[m,joinedup,nowhite]
\item Two diverging straight lines cannot be both parallel to the same straight line.
\item If a straight line intersect one of the two parallel straight lines it must also intersect the other.
\item Only one straight line can be drawn through a given point, parallel to a given straight line.
\stopitemize
Geometry has for its principal objects the exposition and explanation of the properties of \emph{figure}, and figure is defined to be the relation which subsists between the boundaries of space. Space or magnitude is of three kinds, \emph{linear}, \emph{superficial}, and \emph{solid}.

\defineNewPicture{
	pair A, B, C;
	numeric s;
	s := 3/2u;
	A := (0, s);
	B := (1/2s, 0);
	C := B xscaled -1;
		draw byAngleWithName(B, A, C, byyellow, 0)(A);
		byLineDefine(B, A, byblue, 0, 0);
		byLineDefine(C, A, byred, 0, 0);
		draw byNamedLineSeq(0)(CA,BA);
		label.urt(\sometxt{A}, A);
}\drawCurrentPictureInMargin
Angles might properly be considered as a fourth species of magnitude. Angular magnitude evidently consists of parts, and must therefore be admitted to be a species of quantity. The student must not suppose that the magnitude of an angle is affected by the length of the straight lines which include it and of whose mutual divergence it is the measure. The \emph{vertex} of an angle is the point the \emph{sides} of the \emph{legs} of the angle meet, as A.

\defineNewPicture{
	pair B, C, D, E, F, G, H;
	numeric s;
	s := 5/4u;
	C := (0, 0);
	B := dir(0)*s;
	D := dir(50)*s;
	E := dir(-30)*s;
	F := E scaled -1;
	G := D scaled -1;
	H := B scaled -1;
	angleScale := 4/3;
		draw byAngle(E, C, B, byyellow, 0);
		draw byAngle(B, C, D, black, 0);
		draw byAngle(D, C, F, byblue, 0);
		draw byAngle(F, C, H, byred, 0);
		draw byAngle(H, C, G, byyellow, 1);
		draw byAngle(G, C, E, byblue, 1);
		draw byLine(B, H, byblue, 0, 0);
		draw byLine(D, G, byred, 0, 0);
		draw byLine(E, F, black, 0, 0);
		label.bot(\sometxt{C}, C shifted (0, -3pt));
		label.bot(\sometxt{B}, B);
		label.lrt(\sometxt{D}, D);
		label.llft(\sometxt{F}, F);
		label.bot(\sometxt{H}, H);
		label.lrt(\sometxt{G}, G);
		label.llft(\sometxt{E}, E);
}
\drawCurrentPictureInMargin
An angle is often designated by a single letter when its legs are the only lines which meet together at its vertex. Thus the red and blue lines form the yellow angle, which in other systems would be called angle A. But when more than two lines meet in the same point, it was necessary by former methods, in order to avoid confusion, to employ three letters to designate an angle about that point, the letter which marked the vertex of the angle being always placed in the middle. Thus the black and red lines meeting together at C, form the blue angle, and has been usually denominated the angle FCD or DCF. The lines FC and CD are the legs of the angle; the point C is its vertex. In like manner the black angle would be designated the angle DCB or BCD. The red and blue angles added together, or the angle HCD added to FCD, make the angle HCD; and so of other angles.

When the legs of an angle are produced or prolonged beyond its vertex, the angles made by them on both sides of the vertex are said to be \emph{vertically opposite} to each other: thus the red and yellow angles are said to be vertically opposite angles.

\emph{Superposition} is the process by which one magnitude may be conceived to be placed upon another, so as exactly to cover it, or so that every part of each shall exactly coincide.

A line is said to be \emph{produced}, when it is extended, prolonged, or it has length increased, and the increase of length which it receives is called \emph{produced part}, or its \emph{production}.

The entire length of the line or lines which enclose a figure, is called its \emph{perimeter}. The first six books of Euclid treat of plain figures only. A line drawn from the centre of a circle to its circumference, is called a \emph{radius}. That side of a right angled triangle, which is opposite to the right angle, is called the \emph{hypotenuse}. An oblong is defined in the second book, and called a \emph{rectangle}. All lines which considered in the first six books of the Elements are supposed to be in the same plane.

The \emph{straight-edge} and \emph{compasses} are the only instruments, the use of which is permitted in Euclid, or plain Geometry. To declare this restriction is the object of the postulated.

The \emph{Axioms} of geometry are certain general propositions, the truth of which is taken to be self-evident and incapable of being established by demonstration.

\emph{Propositions} are those results which are obtained in geometry by a process of reasoning. There are two species of propositions in geometry, \emph{problems} and \emph{theorems}.

A \emph{Problem} is a proposition in which something is proposed to be done; as a line to be drawn under some given conditions, a circle to be described, some figure to be constructed, \&c.

The \emph{solution} of the problem consists in showing how the thing required may be done by the aid of the rule or straight-edge and compasses.

The \emph{demonstration} consists in proving that the process indicated in the solution attains the required end.

A \emph{Theorem} is a proposition in which the truth of some principle is asserted. This principle must be deduced from the axioms and definitions, or other truths previously and independently established. To show this is the object of demonstration.

A \emph{Problem} is analogous to a postulate.

A \emph{Theorem} resembles an axiom.

A \emph{Postulate} is a problem, the solution to which is assumed.

An \emph{Axiom} is a theorem, the truth of which is granted without demonstration.

A \emph{Corollary} is an inference deduced immediately from a proposition.

A \emph{Scholium} is a note or observation on a proposition not containing an inference of sufficient importance to entitle it to the name of \emph{corollary}.

A \emph{Lemma} is a proposition merely introduced for the purpose of establishing some more important proposition.

\stopsupersection

\vfill\pagebreak

\startsupersection[title={Symbols and abbreviations}]

\symb{$\therefore$}
expresses the word \emph{therefore}.

\symb{$\because$}
 expresses the word \emph{because}.

\symb{$=$}
 expresses the word \emph{equal}. This sign of equality may be read \emph{equal to}, or \emph{is equal to}, or \emph{are equal to}; but the discrepancy in regard to the introduction of the auxiliary verbs \emph{is}, \emph{are}, \&c. cannot affect the geometrical rigour.

\symb{$\neq$}
 means the same as if the words \emph{`not equal'} were written.

\symb{$>$}
 signifies \emph{greater than}.

\symb{$<$}
 signifies \emph{less than}.

\symb{$\ngtr$}
 signifies \emph{not greater than}.

\symb{$\nless$}
 signifies \emph{not less than}.

\symb{$+$}
 is read \emph{plus} (\emph{more}), the sign of addition; when interposed between two or more magnitudes, signifies their sum.

\symb{$-$}
 is read \emph{minus} (\emph{less}), signifies subtraction; and when placed between two quantities denotes that the latter is taken from the former.

\symb{$\times$}
 this sign expresses the product of two or more numbers when placed between them in arithmetic and algebra; but in geometry it is generally used to express a \emph{rectangle}, when placed between \quotation{two straight lines which contain one if its right angles.} A \emph{rectangle} may also be represented by placing a point between two of its conterminous sides.

\symb{$:\ ::\ :$}
 expresses an \emph{analogy} or \emph{proportion}; thus if A, B, C and D represent four magnitudes, and A has to B the same ratio that C has to D, the proportion is thus briefly written

$A : B :: C : D$, $A : B = C : D$, or $\dfrac{A}{B} = \dfrac{C}{D}$.

This equality or sameness of ratio is read,

as A is to B, so is C to D;

or A is to B, as C is to D.

\symb{$\parallel$}
 signifies \emph{parallel to}.

\symb{$\perp$}
 signifies \emph{perpendicular to}.

\defineNewPicture{
	pair A, B, C, D;
	numeric s;
	s := 3/2u;
	A := (0, 0);
	B := dir(0)*s;
	C := dir(50)*s;
	D := dir(90)*s;
		draw byAngle(B, A, C, white, 0);
		draw byAngle(B, A, D, white, 0);
	byPointLabelDefine(A, "");
	byPointLabelDefine(B, "");
	byPointLabelDefine(C, "");
	byPointLabelDefine(D, "");
}

\symb{\drawAngle{BAC}}
 signifies \emph{angle}.

\symb{\drawAngle{BAD}}
 signifies \emph{right angle}.

\symb{\drawTwoRightAngles}
 signifies \emph{two right angles}.


\defineNewPicture{
	pair A, B, C, D;
	A := (0, -1/4u);
	B := (u, 0);
	C := (-u, 0);
	D := (0, u);
		byLineDefine (A, D, black, 0, 0);
		byLineDefine (B, D, black, 0, 0);
		byLineDefine (C, D, black, 0, 0);
	byPointLabelDefine(A, "");
	byPointLabelDefine(D, "");
}

\symb{\drawFromCurrentPicture{
draw byNamedLine(AD);
draw byNamedLineSeq(0)(BD,CD);
}
or
\drawFromCurrentPicture{
draw byNamedLineSeq(0)(AD,BD);
}}
briefly designates a \emph{point}.

A square is described on a line is concisely written thus, $\drawUnitLine{AD}^2$.

In the same manner twice the square of, is expressed by $2 \cdot \drawUnitLine{AD}^2$.

\symb{def.}
 signifies \emph{definition}.

\symb{pos.}
 signifies \emph{postulate}.

\symb{ax.}
 signifies \emph{axiom}.

\symb{hyp.}
 signifies \emph{hypothesis}. It may be necessary here to remark, that \emph{hypothesis} is the condition assumed or taken for granted. Thus, the hypothesis of the proposition given in the Introduction, is that the triangle is isosceles, or that its legs are equal.

\symb{const.}
 signifies \emph{construction}. The \emph{construction} is the change made in the original figure, by drawing lines, making angles, describing circles, \&c. in order to adapt it to the argument of the demonstration or the solution of the problem. The conditions under which these changes are made, are as indisputable as those contained in the hypothesis. For instance, if we make an angle equal to a given angle, these two angles are equal by construction.

\symb{Q. E. D.}
signifies \emph{Quod erat demonstrandum}. Which was to be demonstrated.
\stopsupersection

\stopintro

\startbook[title={Book I}]
\startVerboseProposition[title={Prop. I. Prob.}, reference=prop:I.I]

\defineNewPicture[1/2]{
	pair A, B, C;
	path P[];
	numeric r;
	r := 3/2u;
	A := (0, 0);
	B := (r, 0);
	P1 := fullcircle scaled 2r;
	P2 := fullcircle scaled 2r shifted B;
	C := P1 intersectionpoint P2;
		byLineDefine(A, B, black, 0, 0);
		byLineDefine(B, C, byred, 0, 0);
		byLineDefine(C, A, byyellow, 0, 0);
		draw byNamedLineSeq(1)(AB,CA,BC);
		draw byCircle(A, B, byblue, 0, 0, 1/2)(A);
		draw byCircle(B, A, byred, 0, 0, 1/2)(B);
		draw byLabelsOnPolygon(A, C, B)(0, -1);
}
\drawCurrentPictureInMargin
\problemNP{O}{n}{a given finite straight line (\drawUnitLine{AB}) to describe an equilateral triangle.}

\startCenterAlign
Describe \offsetPicture{15pt}{0pt}{\drawFromCurrentPicture{
draw byNamedLine(AB);
draw byNamedCircle(A);
draw byLabelLineEnd(A, B, 0);
draw byLabelLineEnd(B, A, 1);
}} and \offsetPicture{15pt}{0pt}{\drawFromCurrentPicture{
draw byNamedLine(AB); 
draw byNamedCircle(B);
draw byLabelLineEnd(A, B, 1);
draw byLabelLineEnd(B, A, 0);
}}
\inpost[post:III];\\
draw \drawUnitLine{CA} and \drawUnitLine{BC} \inpost[post:I].\\
Then will \drawLine[bottom][triangleABC]{AB,CA,BC} be equilateral.

For $\drawUnitLine{AB} = \drawUnitLine{CA}$ \indef[def:XV];\\
and $\drawUnitLine{AB} = \drawUnitLine{BC}$ \indef[def:XV];\\
$\therefore \drawUnitLine{CA} = \drawUnitLine{BC}$ \inax[ax:I];\\
and therefore \triangleABC\ is the equilateral triangle required.
\stopCenterAlign

\qed
\stopVerboseProposition

\startProposition[title={Prop. II. Prob.}, reference=prop:I.II]
\defineNewPicture{
pair A, B, C, D, E, F;
path P[];
numeric r[];
A := (0, 0);
B := (-3/5u, -3/5u);
C := (-2u, -1/3u);
r1 := abs(A-B);
D := (fullcircle scaled 2r1 shifted A) intersectionpoint (fullcircle scaled 2r1 shifted B);
r2 := abs(B-C);
r3 := r1 + r2;
P1 := fullcircle scaled 2r2 shifted B;
P2 := fullcircle scaled 2r3 shifted D;
E := (D -- 10[D, B]) intersectionpoint P1;
F := (D -- 10[D, A]) intersectionpoint P2;
byLineDefine(A, B, black, 1, 0);
byLineDefine(B, C, black, 0, 0);
byLineDefine(B, D, byred, 0, 0); 
byLineDefine(D, A, byred, 0, 0); % improvement: change style of either BD or DA
byLineDefine(B, E, byyellow, 0, 0);
byLineDefine(A, F, byblue, 0, 0);
draw byNamedLineSeq(0)(AF,DA,BD,BE);
draw byNamedLineSeq(0)(AB,BC);
draw byCircle(D, E, byred, 0, 0, 1/2)(A);
draw byCircle(B, C, byblue, 0, 0, -1/2)(B);
draw byLabelsOnPolygon(E, D, A, F)(2, -1);
draw byLabelsOnPolygon(E, B, C)(2, -1);
draw byLabelLineEnd(C, B, 0);
draw byLabelLineEnd(E, B, 1);
draw byLabelLineEnd(F, A, 1);
}
\drawCurrentPictureInMargin
\problemNP{F}{rom}{a given point (\drawFromCurrentPicture{
startGlobalRotation(-lineAngle.DA);
draw byNamedLineSeq(0)(DA,AF);
draw byLabelPoint(A, 90, 1);
stopGlobalRotation;
}) to draw a straight line equal to a given straight line (\drawUnitLine{BC}).}

\startCenterAlign
Draw \drawUnitLine{AB} \inpost[post:I], describe \drawFromCurrentPicture[bottom]{
startAutoLabeling;
startTempScale(scaleFactor*3);
startGlobalRotation(180-lineAngle.AB);
draw byNamedLineSeq(0)(AB,BD,DA);
stopGlobalRotation;
stopTempScale;
stopAutoLabeling;
} \inprop[prop:I.I],\\
produce \drawUnitLine{BD} \inpost[post:II],\\
describe
\drawFromCurrentPicture{
draw byNamedLine (BC); 
draw byNamedCircle(B); 
draw byLabelLineEnd(B, C, 0); 
draw byLabelLineEnd(C, B, 0);
}
\inpost[post:III], and
\drawFromCurrentPicture{
draw byNamedLine (BD, BE);
draw byNamedCircle(A);
draw byLabelLineEnd(D, E, 0); 
draw byLabelLineEnd(E, D, 1);
}
\inpost[post:III];\\
produce \drawUnitLine{DA} \inpost[post:II],\\
then \drawUnitLine{AF} is the line required.

For $\drawUnitLine{BE,BD} = \drawUnitLine{DA,AF}$ \indef[def:XV],\\
and $\drawUnitLine{BD} = \drawUnitLine{DA}$ (const.),\\
$\therefore \drawUnitLine{BE} = \drawUnitLine{AF}$ \inax[post:III],\\
but \indef[def:XV] $\drawUnitLine{BC} = \drawUnitLine{BE} = \drawUnitLine{AF}$;

$\therefore \drawUnitLine{AF}$ drawn from the given point (\drawUnitLine{DA,AF}), is equal to the given line \drawUnitLine{BC}.
\stopCenterAlign

\qed
\stopProposition

\startProposition[title={Prop. III. Prob.}, reference=prop:I.III]
\defineNewPicture{
pair A, B, C, D, E, F;
path P;
numeric r;
A := (0, 0);
r := 7/4u;
B := A shifted (r, 0);
C := A shifted (4/3r, 0);
D := A shifted dir(30)*r;
E := A shifted (7/6r, -1/6r);
F := A shifted (7/6r, -7/6r);
byLineDefine(A, B, black, 0, 0);
byLineDefine(B, C, black, 1, 0);
byLineDefine(A, D, byred, 0, 0);
draw byNamedLineSeq(0)(BC,AB,AD);
draw byLine(E, F, byblue, 0, 0);
draw byCircle(A, D, byblue, 0, 0, 0)(A);
draw byLabelsOnPolygon(B, A, D)(2, -1);
draw byLabelLineEnd(D, A, 0);
draw byLabelLineEnd(C, A, 0);
draw byLabelPoint(B, angle(B-A) + 45, 2);
draw byLabelsOnPolygon(E, F)(0, 0);
}
\drawCurrentPictureInMargin
\problemNP{F}{rom}{the greater (\drawUnitLine{AB,BC}) of two given straight lines, to cut off a part equal to the less (\drawUnitLine{EF}).}

\startCenterAlign
Draw $\drawUnitLine{AD} = \drawUnitLine{EF}$ \inprop[prop:I.II];\\
describe 
\drawFromCurrentPicture{
draw byNamedLine (AD); 
draw byNamedCircle(A);
draw byLabelLineEnd(D, A, 0);
draw byLabelLineEnd(A, D, 0);
} \inpost[post:III],\\
then $\drawUnitLine{EF} = \drawUnitLine{AB}$

For $\drawUnitLine{AD} = \drawUnitLine{AB}$ \indef[def:XV],\\
and $\drawUnitLine{EF} = \drawUnitLine{AD}$ (const.);

$\therefore \drawUnitLine{EF} = \drawUnitLine{AB}$ \inax[ax:I];
\stopCenterAlign

\qed
\stopProposition

\startProposition[title={Prop. IV. Theor.}, reference=prop:I.IV]
\defineNewPicture{
pair A, B, C, D, E, F, d;
A := (0, 0);
B := A shifted (-5/2u, -7/2u);
C := A shifted (1/3u, -5/2u);
d := (0, -4u);
D := A shifted d;
E := B shifted d;
F := C shifted d;
draw byAngleWithName(B, A, C, byyellow, 0)(A);
draw byAngleWithName(A, B, C, byblue, 0)(B);
draw byAngleWithName(B, C, A, byred, 0)(C);
byLineDefine(A, B, byred, 0, 0);
byLineDefine(B, C, black, 0, 0);
byLineDefine(C, A, byblue, 0, 0);
draw byNamedLineSeq(0)(CA,BC,AB);
draw byAngleWithName(E, D, F, byyellow, 0)(D);
draw byAngleWithName(D, E, F, byblue, 0)(E);
draw byAngleWithName(E, F, D, byred, 0)(F);
byLineDefine(D, E, byred, 0, 1);
byLineDefine(E, F, black, 0, 1);
byLineDefine(F, D, byblue, 0, 1);
draw byNamedLineSeq(0)(FD,EF,DE);
draw byLabelsOnPolygon(F, E, D)(0, 0);
draw byLabelsOnPolygon(B, A, C)(0, -1);
}
\drawCurrentPictureInMargin
\problemNP{I}{f}{two triangles have two sides of the one respectively equal to two sides of the other, (\drawUnitLine{AB} to \drawUnitLine{DE} and \drawUnitLine{CA} to \drawUnitLine{FD}) and the angles (\drawAngle{A} and \drawAngle{D}) contained by those equal sides also equal; then their bases or their sides (\drawUnitLine{BC} and \drawUnitLine{EF}) are also equal: and the remaining angles opposite to equal sides are respectively equal ($\drawAngle{B} = \drawAngle{E}$ and $\drawAngle{C} = \drawAngle{F}$): and the triangles are equal in every respect.}

Let two triangles be conceived, to be so placed, that the vertex of one of the equal angles, \drawAngle{A} or \drawAngle{D}; shall fall upon that of the other, and \drawUnitLine{AB} to coincide with \drawUnitLine{DE}, then will \drawUnitLine{CA} coincide with \drawUnitLine{FD} if applied: consequently \drawUnitLine{BC} will coincide with \drawUnitLine{EF}, or two straight lines will enclose a space, which is impossible \inax[ax:X], therefore $\drawUnitLine{BC} = \drawUnitLine{EF}$, $\drawAngle{B} = \drawAngle{E}$ and $\drawAngle{C} = \drawAngle{F}$, and as the triangles \drawLine{CA,BC,AB} and \drawLine{FD,EF,DE} coincide, when applied, they are equal in every respect.

\qed
\stopProposition

\startProposition[title={Prop. V. Theor.}, reference=prop:I.V]
\defineNewPicture{
pair A, B, C, D, E;
picture q;
A := (0, 0);
B := A shifted (u, -2u);
C := B xscaled -1;
D := 9/5[A,B];
E := 9/5[A,C];
draw byAngle(B, A, C, black, 0);
draw byAngle(A, B, C, byblue, 0);
draw byAngle(B, C, A, byblue, 0);
draw byAngle(C, B, E, byyellow, 0);
draw byAngle(D, C, B, byyellow, 0);
draw byAngle(B, D, C, byred, 0);
draw byAngle(C, E, B, byred, 0);
byAngleDefine(E, B, D, black, 1);
byAngleDefine(D, C, E, black, 1);
byLineDefine(B, D, byyellow, 0, 0);
byLineDefine(C, E, byyellow, 0, 0);
byLineDefine(B, E, byblue, 0, 0);
byLineDefine(C, D, byblue, 0, 0);
byLineDefine(A, B, byred, 0, 0);
byLineDefine(A, C, byred, 0, 0);
byLineDefine(B, C, black, 0, 0);
draw byNamedLineSeq(0)(CD,noLine,BC,noLine,BE,CE,AC,AB,BD);
draw byLabelsOnPolygon(E, C, A, B, D, C, B)(0, 0);
}
\drawCurrentPictureInMargin
\problemNP{I}{n}{any isosceles triangle \drawLine[bottom]{BC,AC,AB} if the equal sides be produced, the external angles at the base are equal, and the internal angles at the base are also equal.}

\startCenterAlign
Produce \drawUnitLine{AB} and \drawUnitLine{AC} \inpost[post:II],\\
take $\drawUnitLine{BD} = \drawUnitLine{CE}$ \inprop[prop:I.III];\\
draw \drawUnitLine{BE} and \drawUnitLine{CD}.

Then in
\drawFromCurrentPicture{
startAutoLabeling;
draw byNamedAngle(BAC);
draw byNamedLineSeq(0)(BE,CE,AC,AB);
stopAutoLabeling;
}
and
\drawFromCurrentPicture{
startAutoLabeling;
draw byNamedAngle(BAC);
draw byNamedLineSeq(0)(BD,CD,AC,AB);
stopAutoLabeling;
}\\
we have $\drawUnitLine{AB,BD} = \drawUnitLine{AC,CE}$ (const.),\\
\drawAngle{BAC} common to both,\\
and $\drawUnitLine{AB} = \drawUnitLine{AC}$ (hyp.)\\
$\therefore \drawAngle{BCA,DCB} = \drawAngle{ABC,CBE}$, $\drawUnitLine{BE} = \drawUnitLine{CD}$ and $\drawAngle{CEB} = \drawAngle{BDC}$ \inprop[prop:I.IV].

Again in \drawLine{BE,CE,BC} and \drawLine{BD,CD,BC}\\
we have $\drawUnitLine{BD} = \drawUnitLine{CE}$, $\drawAngle{CEB} = \drawAngle{BDC}$ and $\drawUnitLine{BE} = \drawUnitLine{CD}$,\\
$\therefore \drawAngle{DCE,DCB} = \drawAngle{EBD,CBE}$ and $\drawAngle{DCB} = \drawAngle{CBE}$ \inprop[prop:I.IV]\\
but $\drawAngle{BCA,DCB} = \drawAngle{ABC,CBE}$, $\therefore \drawAngle{BCA} = \drawAngle{ABC}$.
\stopCenterAlign

\qed
\stopProposition

\startProposition[title={Prop VI. Theor.}, reference=prop:I.VI]
\defineNewPicture[1/4]{
pair A, B, C, D;
A := (0, 0);
B := A shifted (7/2u, 0);
D := A shifted (7/4u, 3u);
C := 2/3[A, D];
draw byAngleWithName(B, A, D, byyellow, 0)(A);
draw byAngleWithName(A, B, D, black, 0)(B);
byLineDefine(B, C, byyellow, 0, 0);
byLineDefine(A, B, byred, 0, 0);
byLineDefine(B, D, byblue, 0, 0);
byLineDefine(C, A, black, 0, 0);
byLineDefine(C, D, black, 1, 0);
draw byNamedLine(BC);
draw byNamedLineSeq(0)(CA,CD,BD,AB);
draw byLabelsOnPolygon(A, C, D, B)(0, 0);
}
\drawCurrentPictureInMargin
\problemNP{I}{n}{any triangle (\drawLine[bottom][triangleABD]{CA,CD,BD,AB}) if two angles (\drawAngle{A} and \drawAngle{B}) are equal, the sides (\drawUnitLine{CA,CD} and \drawUnitLine{BD}) opposite to them are also equal.}

For if the sides be not equal, let one of them \drawUnitLine{CA,CD} be greater than the other \drawUnitLine{BD}, and from it to cut off $\drawUnitLine{CA} = \drawUnitLine{BD}$ \inprop[prop:I.III], draw \drawUnitLine{BC}.

\startCenterAlign
Then in \drawLine[bottom]{BC,AB,CA} and \triangleABD,\\
$\drawUnitLine{CA} = \drawUnitLine{BD}$ (const.),\\
$\drawAngle{A} = \drawAngle{B}$ (hyp.)\\
and \drawUnitLine{AB} common,\\
$\therefore$ the triangles are equal \inprop[prop:I.IV]\\
a part equal to the whole, which is absurd;\\
$\therefore$ neither of the sides \drawUnitLine{CA,CD} or \drawUnitLine{BD} is greater than the other,\\
$\therefore$ hence they are equal.
\stopCenterAlign

\qed
\stopProposition

\startProposition[title={Prop VII. Theor.}, reference=prop:I.VII]
\defineNewPicture{
pair A, B, C, D, E, F, G, H;
A := (0, 0);
B := A shifted (4u, 0);
C := A shifted (u, 3u);
D := C shifted (7/4u, 0);
E := 1/2[C, D] yscaled -0.7;
F := E shifted (0, -2u);
G := 5/4[A, E];
H := 5/4[A, F];
draw byAngleWithName(B, C, A, black, 0)(C);
draw byAngle(D, C, B, byred, 0);
draw byAngleWithName(A, D, B, byyellow, 0)(D);
draw byAngle(C, D, A, byblue, 0);
draw byAngle(B, F, H, black, 0);
draw byAngle(B, F, E, byred, 0);
draw byAngle(B, E, G, byyellow, 0);
draw byAngle(G, E, F, byblue, 0);
draw byLine(C, D, black, 1, 0);
draw byLine(E, F, black, 1, 0);
draw byLine(A, B, black, 0, 0);
byLineDefine(B, C, byblue, 0, 0);
byLineDefine(C, A, byred, 0, 0);
byLineDefine(B, D, byblue, 0, 0);
byLineDefine(D, A, byred, 0, 0);
byLineDefine(B, E, byblue, 0, 0);
byLineDefine(E, A, byred, 0, 0);
byLineDefine(B, F, byblue, 0, 0);
byLineDefine(F, A, byred, 0, 0);
byLineDefine(E, G, byred, 1, 0);
byLineDefine(F, H, byred, 1, 0);
draw byNamedLine(EG,FH);
draw byNamedLineSeq(0)(BC,CA,EA,BE);
draw byNamedLineSeq(0)(BD,DA,FA,BF);
byPointLabelDefine(F, "C");
byPointLabelDefine(E, "D");
draw byLabelsOnPolygon(F, A, C, D, B, F, noPoint)(2, 0);
draw byLabelsOnPolygon(A, E, B)(2, 0);
draw byLabelsOnPolygon(H, F, A)(2, 0);
}
\drawCurrentPictureInMargin
\problemNP{O}{n}{the same base (\drawUnitLine{AB}), and on the same side of it there can not be two triangles having their conterminous sides (\drawUnitLine{CA} and \drawUnitLine{DA}, \drawUnitLine{BC} and \drawUnitLine{BD}) at both extremities of the base, equal to each other.}

When two triangles stand on the same base, and on the same side of it, the vertex of the one shall either fall outside of the other triangle, or within it; or, lastly, on one of its sides.

If it be possible let the two triangles be constructed so that $\left\{\eqalign{\drawUnitLine{CA}&=\drawUnitLine{BC}\cr \drawUnitLine{DA}&=\drawUnitLine{BD}\cr}\right\}$, then draw \drawUnitLine{CD} and,

\startCenterAlign
$\drawAngle{C,DCB} = \drawAngle{CDA}$ \inprop[prop:I.V]

$\therefore \drawAngle{DCB} < \drawAngle{CDA}$ and

$\left.
\eqalign{
\therefore\drawAngle{DCB} &< \drawAngle{CDA,D}\cr
\mbox{but \inprop[prop:I.V]} \drawAngle{DCB} &= \drawAngle{CDA,D}
}\right\}\mbox{which is absurd,}$
\stopCenterAlign

\noindent therefore the two triangles cannot have their conterminous sides equal at both extremities of the base.

\qed
\stopProposition

\startProposition[title={Prop VIII. Theor.}, reference=prop:I.VIII]
\defineNewPicture{
pair A, B, C, D, E, F, d;
A := (0, 0);
B := A shifted (-u, -4u);
C := A shifted (3/2u, -3u);
d := (0, -9/2u);
D := A shifted d;
E := B shifted d;
F := C shifted d;
draw byAngleWithName(F, D, E, black, 0)(D);
draw byAngleWithName(C, A, B, black, 0)(A);
byLineDefine(A, B, byred, 0, 0);
byLineDefine(B, C, black, 0, 0);
byLineDefine(C, A, byblue, 0, 0);
byLineDefine(D, E, byred, 0, 1);
byLineDefine(E, F, black, 0, 1);
byLineDefine(F, D, byblue, 0, 1);
draw byNamedLineSeq(0)(CA,BC,AB);
draw byNamedLineSeq(0)(FD,EF,DE);
draw byLabelsOnPolygon(C, B, A)(0, 0);
draw byLabelsOnPolygon(F, E, D)(0, 0);
}
\drawCurrentPictureInMargin
\problemNP{I}{f}{two triangles have two sides of the one respectively equal to two sides of the other ($\drawUnitLine{CA} = \drawUnitLine{FD}$ and $\drawUnitLine{AB} = \drawUnitLine{DE}$) and also their bases ($\drawUnitLine{BC} = \drawUnitLine{EF}$), equal; then the angles
(\drawFromCurrentPicture{
startAutoLabeling;
startGlobalRotation(-angleDirection.A);
draw byNamedAngle(A);
draw byNamedAngleDummySides(A);
stopGlobalRotation;
stopAutoLabeling;
} and
\drawFromCurrentPicture{
startAutoLabeling;
startGlobalRotation(-angleDirection.D);
draw byNamedAngle(D);
draw byNamedAngleDummySides(D);
stopGlobalRotation;
stopAutoLabeling;
}) contained by their equal sides are also equal.}

If the equal bases \drawUnitLine{BC} and \drawUnitLine{EF} be conceived to be placed one upon the other, so that the triangles shall lie at the same side of them, and that equal sides \drawUnitLine{AB} and \drawUnitLine{DE}, \drawUnitLine{CA} and \drawUnitLine{FD} be conterminous, the vertex of the one must fall on the vertex of the other; for to suppose them not coincident would contradict the last proposition.

Therefore sides \drawUnitLine{AB} and \drawUnitLine{CA}, being coincident with \drawUnitLine{DE} and \drawUnitLine{FD}, $\therefore \drawAngle{A} = \drawAngle{D}$.

\qed
\stopProposition

\startProposition[title={Prop IX. Prob.}, reference=prop:I.IX]
\defineNewPicture{
pair A, B, C, D, E, F;
A := (0, 2u);
B := (-4/3u, 0);
C := B xscaled -1;
D := A yscaled -1;
E := 5/4[A, B];
F := 5/4[A, C];
draw byAngle(B, A, D, byblue, 0);
draw byAngle(C, A, D, byyellow, 0);
byLineDefine(B, C, byyellow, 0, 0);
byLineDefine(A, D, black, 0, 0);
byLineDefine(D, B, byblue, 0, 0);
byLineDefine(C, D, byblue, 0, 0);
byLineDefine(A, B, byred, 0, 0);
byLineDefine(C, A, byred, 0, 0);
byLineDefine(B, E, byred, 1, 0);
byLineDefine(C, F, byred, 1, 0);
draw byNamedLine(BC,AD);
draw byNamedLineSeq(0)(DB,CD);
draw byNamedLineSeq(0)(BE,AB,CA,CF);
draw byLabelsOnPolygon(D, B, A, C)(0, 0);
}
\drawCurrentPictureInMargin
\problemNP{T}{o}{bisect a given rectilinear angle (\drawAngle{BAD,CAD}).}

\startCenterAlign
Take $\drawUnitLine{AB} = \drawUnitLine{CA}$ \inprop[prop:I.III]

draw \drawUnitLine{BC} upon which describe \drawLine{CD,DB,BC} \inprop[prop:I.I],\\
draw \drawUnitLine{AD}.

Because $\drawUnitLine{AB} = \drawUnitLine{CA}$ (const.)\\
and \drawUnitLine{AD} common to the two triangles\\
and $\drawUnitLine{CD} = \drawUnitLine{DB}$ (const.),

$\therefore \drawAngle{BAD} = \drawAngle{CAD}$ \inprop[prop:I.VIII].
\stopCenterAlign

\qed
\stopProposition

\startProposition[title={Prop X. Prob.}, reference=prop:I.X]
\defineNewPicture{
pair A, B, C, D;
A := (0, 3u);
B := (-7/4u, 0);
C := B xscaled -1;
D := 1/2[B, C];
draw byAngle(B, A, D, byblue, 0);
draw byAngle(C, A, D, byyellow, 0);
draw byLine(A, D, byred, 0, 0);
byLineDefine(D, B, black, 0, 0);
byLineDefine(C, D, black, 1, 0);
byLineDefine(A, B, byyellow, 0, 0);
byLineDefine(C, A, byblue, 0, 0);
draw byNamedLineSeq(0)(AB,CA,CD,DB);
draw byLabelsOnPolygon(B, A, C, D)(0, 0);
}
\drawCurrentPictureInMargin
\problemNP{T}{o}{bisect a given finite straight line (\drawUnitLine{DB,CD}).}

\startCenterAlign
Construct \drawLine[bottom]{AB,CA,CD,DB} \inprop[prop:I.I],\\
draw \drawUnitLine{AD}, making $\drawAngle{BAD} = \drawAngle{CAD}$ \inprop[prop:I.IX],

Then $\drawUnitLine{DB} = \drawUnitLine{CD}$ by \inprop[prop:I.IV],

for~$\drawUnitLine{AB} = \drawUnitLine{CA}$ (const.) $\drawAngle{BAD} = \drawAngle{CAD}$\\
and \drawUnitLine{AD} common to the two triangles.

Therefore the given line is bisected.
\stopCenterAlign

\qed
\stopProposition

\startProposition[title={Prop XI. Prob.}, reference=prop:I.XI]
\defineNewPicture[1/4]{
pair A, B, C, D, E, F;
A := (0, 3u);
B := (-7/4u, 0);
C := B xscaled -1;
D := 1/2[B, C];
E := 3/2[D, B];
F := 3/2[D, C];
draw byAngle(A, D, B, byred, 0);
draw byAngle(C, D, A, byblue, 0);
draw byLine(A, D, byyellow, 0, 0);
byLineDefine(A, B, byblue, 0, 0);
byLineDefine(C, A, byblue, 0, 0);
draw byNamedLineSeq(0)(AB,CA);
draw byLine(D, B, black, 0, 0);
draw byLine(B, E, black, 1, 0);
draw byLine(C, D, byred, 0, 0);
draw byLine(F, C, byred, 1, 0);
draw byLabelsOnPolygon(F, C, D, B, E)(2, 0);
draw byLabelsOnPolygon(B, A, C)(2, 0);
}
\drawCurrentPictureInMargin
\problemNP{F}{rom}{a given point 
(\drawFromCurrentPicture{
draw byNamedLineSeq(0)(DB,CD);
draw byLabelPoint(D, 90, 1);
}) in a given straight line (\drawUnitLine[3/2cm]{DB,CD}), to draw a perpendicular.}

\startCenterAlign
Take any point 
(\drawFromCurrentPicture{
draw byNamedLineSeq(0)(CD,FC);
draw byLabelPoint(C, 90, 1);
}) in the given line,\\
cut off $\drawUnitLine{DB} = \drawUnitLine{CD}$ \inprop[prop:I.III],\\
construct \drawLine[bottom]{AB,CA,CD,DB} \inprop[prop:I.I],\\
draw \drawUnitLine{AD} and it shall be perpendicular to the given line.

for $\drawUnitLine{AB} = \drawUnitLine{CA}$ (const.)\\
$\drawUnitLine{CD} = \drawUnitLine{DB}$ (const.)\\
and \drawUnitLine{AD} common to the two triangles.

Therefore $\drawAngle{ADB} = \drawAngle{CDA}$ \inprop[prop:I.VIII]

$\therefore \drawUnitLine{AD} \perp \drawUnitLine{DB,CD}$ \indef[def:X]
\stopCenterAlign

\qed
\stopProposition

\startProposition[title={Prop XII. Prob.}, reference=prop:I.XII]
\defineNewPicture{
pair A, B, C, D, E, F;
path c;
numeric r, a[];
A := (0, 3u);
B := (-7/4u, 0);
C := B xscaled -1;
D := 1/2[B, C];
E := 4/3[D, B];
F := 4/3[D, C];
r := abs(A-B);
c := fullcircle scaled 2r shifted A;
a1 := xpart(c intersectiontimes (F--1/2[B, C]));
a2 := xpart(c intersectiontimes (E--1/2[B, C]));
draw byAngle(A, D, B, byyellow, 0);
draw byAngle(C, D, A, byblue, 0);
draw byLine(A, D, byred, 0, 0);
byLineDefine(A, B, byblue, 0, 0);
byLineDefine(C, A, byblue, 0, 0);
draw byNamedLineSeq(0)(AB,CA);
draw byArc(A, B, C)(r, byred, 0, 0, 0, 0)(O);
draw byArcBE(A, a2-1/4, a2, r, byred, 1, 0, 0, 0)(Ol);
draw byArcBE(A, a1, a1+1/4, r, byred, 1, 0, 0, 0)(Or);
draw byLine(D, B, black, 0, 0);
draw byLine(B, E, black, 1, 0);
draw byLine(C, D, byyellow, 0, 0);
draw byLine(F, C, byyellow, 1, 0);
draw byLabelLineEnd(B, A, 0);
draw byLabelLineEnd(D, A, 0);
draw byLabelLineEnd(C, A, 0);
}
\drawCurrentPictureInMargin
\problemNP{T}{o}{draw a straight line perpendicular to a given indefinite straight line (\drawUnitLine[1.2cm]{DB,CD}) from a~given point (\drawFromCurrentPicture[middle][pointA]{
startTempScale(scaleFactor/2);
draw byNamedLine(AD);
draw byNamedLineSeq(0)(AB,CA);
draw byLabelsOnPolygon(B, A, C)(2, 0);
stopTempScale;
}) without.}

\startCenterAlign
With the given point \pointA\ as centre, at one side of the line, and any distance \drawUnitLine{DB} capable of extending to the other side, describe \drawArc{O}. % improvement: the distance mentioned here seems not to be in the original drawing, it should either be drawn, or not be referenced graphically as DB

Make $\drawUnitLine{DB} = \drawUnitLine{CD}$ \inprop[prop:I.X],\\
draw \drawUnitLine{AB}, \drawUnitLine{CA} and \drawUnitLine{AD}.

Then $\drawUnitLine{AD} \perp \drawUnitLine{DB,CD}$.

For \inprop[prop:I.VIII] since $\drawUnitLine{DB} = \drawUnitLine{CD}$ (const.),\\
\drawUnitLine{AD} common to both,\\
and $\drawUnitLine{AB} = \drawUnitLine{CA}$ \indef[def:XV],

$\therefore \drawAngle{ADB} = \drawAngle{CDA}$, and

$\therefore \drawUnitLine{AD} \perp \drawUnitLine{DB,CD}$ \indef[def:X]
\stopCenterAlign

\qed
\stopProposition

\startProposition[title={Prop XIII. Theor.}, reference=prop:I.XIII]
\defineNewPicture{
pair A, B, C, D, E;
A := (0, 5/2u);
B := (-7/4u, 0);
C := B xscaled -1;
D := (xpart(A), ypart(B));
E := (2/3xpart(C), 2/3ypart(A));
draw byAngle(A, D, B, byyellow, 0);
draw byAngle(E, D, A, byred, 0);
draw byAngle(C, D, E, byblue, 0);
draw byLine(A, D, black, 0, 0);
draw byLine(E, D, byyellow, 0, 0);
draw byLine(B, C, byred, 0, 0);
draw byLabelsOnPolygon(C, D, B, noPoint)(0, 0);
draw byLabelLineEnd(E, D, 0);
draw byLabelLineEnd(A, D, 0);
}
\drawCurrentPictureInMargin
\problemNP{W}{hen}{a straight line (\drawUnitLine{ED}) standing upon another straight line (\drawUnitLine{BC}) makes angles with it; they are either two right angles or together equal to two right angles.}

\startCenterAlign
If \drawUnitLine{ED} be $\perp$ to \drawUnitLine{BC} then,\\
\drawAngle{ADB,EDA} and $\drawAngle{CDE} = \drawTwoRightAngles$ \indef[def:X],

but if \drawUnitLine{ED} be not $\perp$ to \drawUnitLine{BC},\\
draw $\drawUnitLine{AD} \perp \drawUnitLine{BC}$; \inprop[prop:I.XI]\\
$\drawAngle{ADB} +\drawAngle{CDE,EDA} = \drawTwoRightAngles$ (const.),\\
$\drawAngle{ADB} = \drawAngle{CDE,EDA} = \drawAngle{EDA} + \drawAngle{CDE}$

$\therefore \drawAngle{ADB} + \drawAngle{CDE,EDA} = \drawAngle{ADB} + \drawAngle{EDA} + \drawAngle{CDE}$ \inax[ax:II]

$= \drawAngle{ADB,EDA} + \drawAngle{CDE} = \drawTwoRightAngles$.
\stopCenterAlign

\qed
\stopProposition

\startProposition[title={Prop XIV. Theor.}, reference=prop:I.XIV]
\defineNewPicture[1/4]{
pair A, B, C, D, E;
A := (u, 5/2u);
B := (-7/4u, 0);
C := B xscaled -1;
D := (0, 0);
E := (xpart(C), -1/2ypart(A));
draw byAngle(B, D, A, byyellow, 0);
draw byAngle(C, D, A, byblue, 0);
draw byAngle(E, D, C, byred, 0);
draw byLine(A, D, byred, 0, 0);
draw byLine(E, D, byyellow, 0, 0);
draw byLine(B, D, byblue, 0, 0);
draw byLine(C, D, black, 0, 0);
draw byLabelsOnPolygon(E, D, B, noPoint)(0, 0);
draw byLabelLineEnd(A, D, 0);
}
\drawCurrentPictureInMargin
\problemNP{I}{f}{two straight lines (\drawUnitLine{BD} and \drawUnitLine{CD}) meeting a third straight line (\drawUnitLine{AD}), at the same point, and at opposite sides of it, make with it adjacent angles (\offsetPicture{0pt}{15pt}{\drawAngle{BDA}} and \drawAngle{CDA}) equal to two right angles; these straight lines lie in one continuous straight line.}

\startCenterAlign
For, if possible let \drawUnitLine{ED}, and not \drawUnitLine{CD}, be the continuation of \drawUnitLine{BD},\\
then $\drawAngle{BDA} + \drawAngle{CDA,EDC} = \drawTwoRightAngles$

but by the hypothesis $\drawAngle{BDA} + \drawAngle{CDA} = \drawTwoRightAngles$

$\therefore\drawAngle{CDA,EDC} = \drawAngle{CDA}$, \inax[ax:III]; which is absurd \inax[ax:IX].

$\therefore \drawUnitLine{ED}$, is not the continuation of \drawUnitLine{BD}, and the like may be demonstrated of any other straight line except \drawUnitLine{CD}, $\therefore \drawUnitLine{CD}$ is the continuation of \drawUnitLine{BD}.
\stopCenterAlign

\qed
\stopProposition

\startProposition[title={Prop XV. Theor.}, reference=prop:I.XV]
\defineNewPicture{
pair A, B, C, D, E;
A := (7/4u, 3/2u);
B := A scaled -1;
C := A xscaled -1;
D := C scaled -1;
E := (A--B) intersectionpoint (C--D);
draw byAngle(B, E, C, byyellow, 0);
draw byAngle(C, E, A, byred, 0);
draw byAngle(A, E, D, black, 0);
draw byAngle(D, E, B, byblue, 0);
draw byLine(A, B, byred, 0, 0);
draw byLine(C, D, black, 0, 0);
draw byLabelsOnPolygon(C, E, A, noPoint)(0, 0);
draw byLabelPoint(B, lineAngle.AB + 90, 1);
draw byLabelPoint(D, lineAngle.CD - 90, 1);
}
\drawCurrentPictureInMargin
\problemNP{I}{f}{two straight lines (\drawUnitLine{AB} and \drawUnitLine{CD}) intersect on another, the vertical angles \drawAngle{BEC} and \drawAngle{AED}, \drawAngle{CEA} and \drawAngle{DEB} are equal.}

\startCenterAlign
$\drawAngle{BEC} + \drawAngle{CEA} = \drawTwoRightAngles$

$\drawAngle{AED} + \drawAngle{CEA} = \drawTwoRightAngles$

$\therefore \drawAngle{BEC} = \drawAngle{AED}$.

In the same manner it may be shown that\\
$\drawAngle{CEA} = \drawAngle{DEB}$
\stopCenterAlign

\qed
\stopProposition

\startProposition[title={Prop XVI. Theor.}, reference=prop:I.XVI]
\defineNewPicture[1/4]{
pair A, B, C, D, E, F, G;
A := (0, 0);
B := A shifted (3/2u, 7/2u);
C := A shifted (3u, 0);
D := B shifted (3u, 0);
E = whatever[A, D] = whatever[B, C];
F := (xpart(D), ypart(A));
G := 4/3[B, C];
draw byAngleWithName(B, A, C, byblue, 0)(A);
draw byAngleWithName(C, B, A, black, 0)(B);
draw byAngle(A, E, B, byyellow, 0);
draw byAngle(D, E, C, byyellow, 0);
draw byAngle(E, C, D, black, 0);
draw byAngle(G, C, A, byred, 0);
draw byAngle(D, C, F, black, 1);
byLineDefine(C, F, black, 1, 0);
byLineDefine(C, G, black, 0, 0);
byLineDefine(B, E, byblue, 0, 0);
byLineDefine(E, C, byblue, 1, 0);
byLineDefine(A, E, byred, 0, 0);
byLineDefine(E, D, byred, 1, 0);
byLineDefine(A, B, byyellow, 1, 0);
byLineDefine(A, C, black, 0, 0);
byLineDefine(C, D, byyellow, 0, 0);
draw byNamedLineSeq(0)(AE,ED,CD);
draw byNamedLineSeq(0)(EC,CG,noLine,CF,AC,AB,BE);
draw byLabelsOnPolygon(F, A, B, E, D, C)(2, 0);
draw byLabelsOnPolygon(F, C, G, noPoint)(0, 0);
}
\drawCurrentPictureInMargin
\problemNP{I}{f}{a side of a triangle (\drawLine[bottom]{BE,EC,AC,AB}) is produced, the external angle (\drawFromCurrentPicture[bottom][anglesECDpDCF]{
startAutoLabeling;
draw byNamedAngleSides(ECD,DCF)(CF);
stopAutoLabeling;
}) is greater than either of the internal remote angles (\drawAngle{B} or \drawAngle{A})
}

\startCenterAlign
Make $\drawUnitLine{BE} = \drawUnitLine{EC}$ \inprop[prop:I.X];\\
Draw \drawUnitLine{AE} and produce it until $\drawUnitLine{ED} = \drawUnitLine{AE}$;\\
draw \drawUnitLine{CD}. In \drawLine{BE,AE,AB} and \drawLine{EC,ED,CD};\\
$\drawUnitLine{BE} = \drawUnitLine{EC}$, $\drawAngle{AEB} = \drawAngle{DEC}$ \inprop[prop:I.XV] and $\drawUnitLine{AE} = \drawUnitLine{ED}$ (const.),\\
$\therefore \drawAngle{B} = \drawAngle{ECD}$ \inprop[prop:I.IV],\\
$\therefore \anglesECDpDCF\ > \drawAngle{ECD}$.

In like manner it can be shown, that if \drawUnitLine{AC,CF} be produced, $\drawAngle{GCA} > \drawAngle{A}$ \\
and therefore \anglesECDpDCF\ which is $= \drawAngle{GCA}$ is $> \drawAngle{A}$.
\stopCenterAlign

\qed
\stopProposition

\startProposition[title={Prop XVII. Theor.}, reference=prop:I.XVII]
\defineNewPicture{
pair A, B, C, D;
A := (0, 0);
B := A shifted (3/2u, 5/2u);
C := A shifted (9/4u, 0);
D := C shifted (u, 0);
draw byAngleWithName(B, A, C, byblue, 0)(A);
draw byAngleWithName(A, B, C, black, 0)(B);
draw byAngle(A, C, B, byred, 0);
draw byAngle(B, C, D, byyellow, 0);
byLineDefine(A, B, byred, 0, 0);
byLineDefine(B, C, byblue, 0, 0);
byLineDefine(A, C, black, 0, 0);
byLineDefine(C, D, black, 0, 0);
draw byNamedLineSeq(0)(noLine,BC,AB,AC,CD);
draw byLabelsOnPolygon(D, C, A, B)(0, 0);
}
\drawCurrentPictureInMargin
\problemNP{A}{ny}{two angles of a triangle \drawLine[bottom]{AB,BC,AC} are together less than two right angles.}

\startCenterAlign
Produce \drawUnitLine{AC}, then will\\
$\drawAngle{ACB} + \drawAngle{BCD} = \drawTwoRightAngles$

But $\drawAngle{BCD} > \drawAngle{A}$ \inprop[prop:I.XVI]

$\therefore \drawAngle{ACB} + \drawAngle{A} < \drawTwoRightAngles$,
\stopCenterAlign

\noindent and in the same manner it may be shown that any other two angles of the triangle taken together are less than two right angles.

\qed
\stopProposition

\startProposition[title={Prop XVIII. Theor.}, reference=prop:I.XVIII]
\defineNewPicture[1/4]{
pair A, B, C, D;
A := (0, 0);
B := A shifted (5/2u, -1/2u);
C := A shifted (3/2u, 2u);
D := 2[C, A];
draw byAngleWithName(C, A, B, byblue, 0)(A);
draw byAngle(A, B, C, black, 0);
draw byAngle(D, B, A, byred, 0);
draw byAngleWithName(B, D, A, byyellow, 0)(D);
draw byLine(A, B, byyellow, 0, 0);
byLineDefine(A, C, byred, 0, 0);
byLineDefine(B, C, byblue, 0, 0);
byLineDefine(B, D, black, 0, 0);
byLineDefine(D, A, byred, 1, 0);
draw byNamedLineSeq(0)(AC,BC,BD,DA);
draw byLabelsOnPolygon(A, C, B, D)(0, 0);
}
\drawCurrentPictureInMargin
\problemNP{I}{n}{any triangle \drawLine{AC,BC,BD,DA} if one side \drawUnitLine{DA,AC} be greater than another \drawUnitLine[0.5cm]{BC}, the angle opposite to the greater side is greater than the angle opposite to the less. i. e. $\drawAngle{ABC,DBA} > \drawAngle{D}$}

\startCenterAlign
Make $\drawUnitLine{AC} = \drawUnitLine{BC}$ \inprop[prop:I.III], draw \drawUnitLine{AB},

Then will $\drawAngle{A} = \drawAngle{ABC}$ \inprop[prop:I.V];

but $\drawAngle{ABC} > \drawAngle{D}$ \inprop[prop:I.XVI]

$\therefore \drawAngle{ABC} > \drawAngle{D}$ and much more\\
is $\drawAngle{ABC,DBA} > \drawAngle{D}$.
\stopCenterAlign

\qed
\stopProposition

\startProposition[title={Prop XIX. Theor.}, reference=prop:I.XIX]
\defineNewPicture[1/4]{
pair A, B, C;
A := (0, 0);
B := A shifted (7/2u, 0);
C := A shifted (u, 3u);
draw byAngleWithName(C, A, B, byblue, 0)(A);
draw byAngleWithName(A, B, C, byred, 0)(B);
byLineDefine(A, B, black, 0, 0);
byLineDefine(B, C, byblue, 0, 0);
byLineDefine(C, A, byred, 0, 0);
draw byNamedLineSeq(0)(CA,BC,AB);
draw byLabelsOnPolygon(B, A, C)(0, 0);
}
\drawCurrentPictureInMargin
\problemNP{I}{f}{in any triangle \drawLine[bottom]{CA,BC,AB} one angle \drawAngle{A} be greater than another \drawAngle{B} the side \drawUnitLine{BC} which is opposite to the greater angle, is greater than the side \drawUnitLine{CA} opposite the less.}

\startCenterAlign
If \drawUnitLine{BC} be not greater than \drawUnitLine{CA} then must\\
$\drawUnitLine{BC} =$ or $< \drawUnitLine{CA}$.

If $\drawUnitLine{BC} = \drawUnitLine{CA}$ then\\
$\drawAngle{A} = \drawAngle{B}$ \inprop[prop:I.V];\\
which is contrary to the hypothesis.

\drawUnitLine{BC} is not less than \drawUnitLine{CA}; for if it were,\\
$\drawAngle{A} < \drawAngle{B}$ \inprop[prop:I.XVIII]\\
which is contrary to the hypothesis:

$\therefore \drawUnitLine{BC} > \drawUnitLine{CA}$.
\stopCenterAlign

\qed
\stopProposition

\startProposition[title={Prop XX. Theor.}, reference=prop:I.XX]
\defineNewPicture{
pair A, B, C, D;
A := (0, 0);
B := A shifted (7/2u, 0);
D := A shifted (4/3u, 3/2u);
C := ((fullcircle scaled 2arclength(D--B)) shifted D) intersectionpoint (D--10[A, D]);
draw byAngleWithName(B, C, A, byred, 0)(C);
draw byAngle(C, B, D, byblue, 0);
draw byAngle(D, B, A, byyellow, 0);
byLineDefine(B, D, byred, 0, 0);
byLineDefine(A, B, black, 0, 0);
byLineDefine(B, C, byyellow, 0, 0);
byLineDefine(C, D, byblue, 1, 0);
byLineDefine(D, A, byblue, 0, 0);
draw byNamedLineSeq(0)(BD);
draw byNamedLineSeq(0)(DA,CD,BC,AB);
draw byLabelsOnPolygon(D, C, B, A)(0, 0);
}
\drawCurrentPictureInMargin
\problemNP{A}{ny}{two sides \drawUnitLine{DA} and \drawUnitLine{BD} of a triangle \drawLine[bottom]{DA,BD,AB} taken together are greater than the third side (\drawUnitLine{AB}).}

\startCenterAlign
Produce \drawUnitLine{DA}, and\\
make $\drawUnitLine{CD} = \drawUnitLine{BD}$ \inprop[prop:I.III];\\
draw \drawUnitLine{BC}.

Then because $\drawUnitLine{CD} = \drawUnitLine{BD}$ (const.),\\
$\drawAngle{CBD} = \drawAngle{C}$ \inprop[prop:I.V]

$\therefore \drawAngle{CBD,DBA} > \drawAngle{C}$ \inax[ax:IX]

$\therefore \drawUnitLine{DA} + \drawUnitLine{CD} > \drawUnitLine{AB}$ \inprop[prop:I.XIX]

and $\therefore \drawUnitLine{DA} + \drawUnitLine{BD} > \drawUnitLine{AB}$
\stopCenterAlign

\qed
\stopProposition

\startProposition[title={Prop XXI. Theor.}, reference=prop:I.XXI]
\defineNewPicture{
pair A, B, C, D, E;
A := (0, 0);
B := A shifted (7/2u, 0);
C := A shifted (3u, 4u);
D := 1/2[1/2[A, B], C];
E = whatever[A, D] = whatever[B, C];
draw byAngleWithName(B, D, A, byred, 0)(D);
draw byAngleWithName(B, E, D, byblue, 0)(E);
draw byAngleWithName(B, C, A, byyellow, 0)(C);
byLineDefine(B, D, byyellow, 0, 0);
byLineDefine(A, D, black, 0, 0);
byLineDefine(D, E, black, 1, 0);
byLineDefine(A, B, byblue, 1, 0);
byLineDefine(B, E, byred, 1, 0);
byLineDefine(E, C, byred, 0, 0);
byLineDefine(C, A, byblue, 0, 0);
draw byNamedLine(BD);
draw byNamedLineSeq(0)(AD,DE);
draw byNamedLineSeq(0)(CA,EC,BE,AB);
draw byLabelsOnPolygon(A, C, E, B)(0, 0);
draw byLabelsOnPolygon(A, D, E)(2, 0);
}
\drawCurrentPictureInMargin
\problemNP[2]{I}{f}{from any point (\drawFromCurrentPicture{
startTempScale(1/5);
draw byNamedLineSeq(0)(AD,BD);
draw byLabelsOnPolygon(A, D, B)(2, 0);
stopTempScale;
}) within a triangle \drawFromCurrentPicture[bottom]{
startAutoLabeling;
startTempScale(1/5);
draw byNamedLineSeq(0)(CA,EC,BE,AB);
stopTempScale;
stopAutoLabeling;
} straight lines be drawn to the extremities of one side (\drawSizedLine{AB}), these lines must be together less than the two other sides, but must contain a greater angle.}

\startCenterAlign
Produce \drawSizedLine{AD},\\
$\drawSizedLine{CA} + \drawSizedLine{EC} > \drawSizedLine{AD,DE}$ \inprop[prop:I.XX],\\
add \drawSizedLine{BE} to each,\\
$\drawSizedLine{CA} + \drawSizedLine{EC,BE} > \drawSizedLine{AD,DE} + \drawSizedLine{BE}$ \inax[ax:IV]

in the same manner it may be shown that\\
$\drawSizedLine{AD,DE} + \drawSizedLine{BE} > \drawSizedLine{AD} + \drawSizedLine{BD}$,\\
$\therefore \drawSizedLine{CA} + \drawSizedLine{EC,BE} > \drawSizedLine{AD} + \drawSizedLine{BD}$,\\
which was to be proved.

Again $\drawAngle{E} > \drawAngle{C}$ \inprop[prop:I.XVI],\\
and also $\drawAngle{D} > \drawAngle{E}$ \inprop[prop:I.XVI],

$\therefore \drawAngle{D} > \drawAngle{C}$.
\stopCenterAlign

\qed
\stopProposition

\startProposition[title={Prop XXII. Prob.}, reference=prop:I.XXII]
\defineNewPicture[1/2]{
numeric r[], d;
pair A, B, C, D, E, LI, LII, LIII, LIV, LV, LVI;
path q[];
r1 := 3/2u;
r2 := 4/3u;
r3 := (2/3)*(r1+r2);
d := 1/3u;
A := (0, 0);
B := A shifted (r3, 0);
q1 := (fullcircle scaled 2r1) shifted A;
q2 := (fullcircle scaled 2r2) shifted B;
C := q1 intersectionpoint q2;
D := point 11/2 of q1;
E := point 3/4 of q2;
LI := (xpart(point 0 of q2), ypart(point 6 of q1) - 1/2d);
LII := LI shifted (-r3, 0);
LIII := LI shifted (0, -d);
LIV := LIII shifted (-r2, 0);
LV := LIII shifted (0, -d);
LVI := LV shifted (-r1, 0);
draw byCircle(A, D, byblue, 0, 0, 0)(A);
byLineDefine(A, D, byblue, 0, 0);
byLineDefine(B, E, byred, 0, 0);
byLineDefine(A, B, black, 0, 0);
byLineDefine(B, C, byyellow, 0, 0);
byLineDefine(C, A, byyellow, 1, 0);
draw byNamedLineSeq(0)(BC,CA);
draw byNamedLineSeq(0)(AD,AB,BE);
draw byLineWithName (LI, LII, black, 1, 0)(L');
draw byLineWithName (LIII, LIV, byred, 1, 0)(L'');
draw byLineWithName (LV, LVI, byblue, 1, 0)(L''');
draw byCircle(B, E, byred, 0, 0, 0)(B);
draw byLabelsOnPolygon(D, A, C)(2, 0);
draw byLabelsOnPolygon(E, B, A)(2, 0);
draw byLabelsOnPolygon(A, C, B)(2, 0);
draw byLabelLineEnd(D, A, 0);
draw byLabelLineEnd(E, B, 0);
draw byLabelLine(L', L'', L''');
}
\drawCurrentPictureInMargin
\problemNP{G}{iven}{three right lines $\left\{\vcenter{
\nointerlineskip\hbox{\drawSizedLine{L'}}
\nointerlineskip\hbox{\drawSizedLine{L''}}
\nointerlineskip\hbox{\drawSizedLine{L'''}}}\right.$
the sum of any two greater than the third, to construct a triangle whose sides shall be respectively equal to the given lines.}

\startCenterAlign
Assume $\drawSizedLine{AB} = \drawSizedLine{L'}$ \inprop[prop:I.III].

$\left.\eqalign{
\mbox{Draw } \drawSizedLine{AD} &= \drawSizedLine{L'''}\cr
\mbox{and } \drawSizedLine{BE} &= \drawSizedLine{L''}
}\right\}\mbox{\inprop[prop:I.II].}$

With \drawSizedLine{AD} and \drawSizedLine{BE} as radii, describe
\drawFromCurrentPicture{
draw byNamedLine(AD); draw byNamedCircle(A);
draw byLabelLineEnd(A, D, 0);
draw byLabelLineEnd(D, A, 0);
} and
\offsetPicture{12pt}{0pt}{\drawFromCurrentPicture{
draw byNamedLine(BE); draw byNamedCircle(B);
draw byLabelLineEnd(B, E, 0);
draw byLabelLineEnd(E, B, 0);
}} \inpost[post:III];\\
draw \drawSizedLine{CA} and \drawSizedLine{BC},\\
then will \drawLine[bottom]{CA,BC,AB} be the triangle required.

$\left.\eqalign{
\mbox{For } \drawSizedLine{AB} &= \drawSizedLine{L'} \mbox{,} \cr
\drawSizedLine{BC} &= \drawSizedLine{BE} = \drawSizedLine{L''} \cr
\mbox{and } \drawSizedLine{CA} &= \drawSizedLine{AD} = \drawSizedLine{L'''} \cr
}\right\}\mbox{(const.)}$
\stopCenterAlign

\qed
\stopProposition

\startProposition[title={Prop XXIII. Prob.}, reference=prop:I.XXIII]
\defineNewPicture{
angleScale := 3/2;
pair A, B, C, D, E, F, G, H, J, d;
A := (0, 0);
B := A shifted (7/2u, 0);
C := A shifted (3u, 11/5u);
D := 5/4[A, B];
E := 7/6[A, C];
d := (0, -3u);
F := A shifted d;
G := B shifted d;
H := C shifted d;
J := D shifted d;
draw byAngleWithName(B, A, C, byred, 0)(A);
draw byAngleWithName(G, F, H, byblue, 0)(F);
byLineDefine(B, D, black, 1, 1);
byLineDefine(C, E, byblue, 1, 1);
byLineDefine(A, B, black, 0, 1);
byLineDefine(C, A, byblue, 0, 1);
draw byLine(B, C, byred, 0, 1);
draw byNamedLineSeq(0)(CE,CA,AB,BD);
byLineDefine(G, J, black, 1, 0);
byLineDefine(F, G, black, 0, 0);
byLineDefine(G, H, byred, 0, 0);
byLineDefine(H, F, byyellow, 0, 0);
draw byNamedLineSeq(0)(noLine,GH,HF,FG,GJ);
draw byLabelsOnPolygon(D, B, A, C, E, noPoint)(0, 0);
draw byLabelsOnPolygon(noPoint,J, G, F, H, G)(2, 0);
}
\drawCurrentPictureInMargin
\problemNP{A}{t}{a given point (\drawFromCurrentPicture{
startTempScale(scaleFactor/2);
draw byNamedLineSeq(0)(FG,HF);
draw byLabelsOnPolygon(G, F, H)(2, 0);
stopTempScale;
}) in a given straight line (\drawUnitLine{FG,GJ}), to make an angle equal to a given rectilinear angle (\drawAngle{A}).}

Draw \drawUnitLine{BC} between any two points in the legs of the given angle.

\startCenterAlign
Construct \drawLine[bottom]{HF,GH,FG} \inprop[prop:I.XXII]\\
so that $\drawUnitLine{FG} = \drawUnitLine{AB}$, $\drawUnitLine{HF} = \drawUnitLine{CA}$\\
and $\drawUnitLine{GH} = \drawUnitLine{BC}$.

Then $\drawAngle{A} = \drawAngle{F}$ \inprop[prop:I.VIII].
\stopCenterAlign

\qed
\stopProposition

\startProposition[title={Prop XXIV. Theor.}, reference=prop:I.XXIV]
\defineNewPicture{
pair A, B, C, D, E, F, G, d;
A := (0, 0);
B := A shifted (u, -5/2u);
C := A shifted (-u, -7/2u);
D := (xpart(C) - 3/2u, ypart(B));
d := (0, -4u);
E := A shifted d;
F := B shifted d;
G := C shifted d;
draw byAngle(B, A, C, black, 2);
draw byAngle(C, A, D, black, 3);
draw byAngle(F, E, G, black, 2);
draw byAngle(B, D, A, byblue, 0);
draw byAngle(C, D, B, byred, 0);
draw byAngle(D, C, A, byyellow, 0);
draw byAngle(A, C, B, black, 0);
byLineDefine(A, B, byblue, 0, 0);
byLineDefine(B, C, black, 1, 0);
draw byLine(C, A, byred, 0, 0);
draw byLine(B, D, black, 0, 0);
byLineDefine(A, D, byred, 1, 0);
byLineDefine(C, D, byblue, 1, 0);
draw byNamedLineSeq(0)(AB,AD,CD,BC);
byLineDefine(E, F, byblue, 0, 1);
byLineDefine(F, G, byyellow, 0, 1);
byLineDefine(G, E, byred, 0, 1);
draw byNamedLineSeq(0)(EF,FG,GE);
draw byLabelsOnPolygon(D, A, B, C)(0, 0);
draw byLabelsOnPolygon(G, E, F)(0, 0);
}
\drawCurrentPictureInMargin
\problemNP{I}{f}{two triangles have two sides of the one respectively equal to two sides of the other (\drawUnitLine{AB} to \drawUnitLine{EF} and \drawUnitLine{AD} to \drawUnitLine{GE}), and if one of the angles
(\drawFromCurrentPicture[bottom]{
startAutoLabeling;
draw byNamedAngleSides(BAC,CAD)(AB,CA,AD);
stopAutoLabeling;
})
contained by the equal sides be greater than the other
(\drawFromCurrentPicture[bottom][angleFEG]{
startAutoLabeling;
draw byNamedAngleSides(FEG)(EF, GE);
stopAutoLabeling;
}),
the side (\drawUnitLine{BD}) which is opposite to the greater angle is greater than the side (\drawUnitLine{FG}) which is opposite to the less angle.
}

\startCenterAlign
Make $\drawFromCurrentPicture[bottom][angleBAC]{
startAutoLabeling;
draw byNamedAngleSides(BAC)(AB, CA);
stopAutoLabeling;
} = \angleFEG$ \inprop[prop:I.XXIII],\\
and $\drawUnitLine{CA} = \drawUnitLine{GE}$ \inprop[prop:I.III],\\
draw \drawUnitLine{CD} and \drawUnitLine{BC}.

Because $\drawUnitLine{CA} = \drawUnitLine{AD}$ (\inaxL[ax:I]. hyp. const.)\\
$\therefore \drawAngle{BDA,CDB} = \drawAngle{DCA}$ \inprop[prop:I.V]
but $\drawAngle{CDB} < \drawAngle{DCA}$,\\
and $\therefore \drawAngle{CDB} < \drawAngle{DCA,ACB}$,

$\therefore \drawUnitLine{BD} > \drawUnitLine{BC}$ \inprop[prop:I.XIX]

but $\drawUnitLine{BC} = \drawUnitLine{FG}$ \inprop[prop:I.IV]

$\therefore \drawUnitLine{BD} > \drawUnitLine{FG}$.
\stopCenterAlign

\qed
\stopProposition

\startProposition[title={Prop XXV. Theor.}, reference=prop:I.XXV]
\defineNewPicture{
pair A, B, C, D, E, F, d;
A := (0, 0);
B := A shifted (u, -3u);
C := A shifted (-7/4u, -4u);
d := (0, -9/2u);
D := A shifted d;
E := ((B shifted -A) rotated -10) shifted d;
F := C shifted d;
draw byAngleWithName(B, A, C, byyellow, 0)(A);
draw byAngleWithName(E, D, F, black, 0)(D);
byLineDefine(A, B, byblue, 0, 0);
byLineDefine(B, C, black, 0, 0);
byLineDefine(C, A, byred, 0, 0);
draw byNamedLineSeq(0)(AB,BC,CA);
byLineDefine(D, E, byblue, 0, 1);
byLineDefine(E, F, byyellow, 0, 1);
byLineDefine(F, D, byred, 0, 1);
draw byNamedLineSeq(0)(DE,EF,FD);
draw byLabelsOnPolygon(A, B, C)(0, 0);
draw byLabelsOnPolygon(D, E, F)(0, 0);
}
\drawCurrentPictureInMargin
\problemNP{I}{f}{two triangles have two sides (\drawUnitLine[0.7cm]{AB} and \drawUnitLine[0.7cm]{CA}) respectively equal to two sides (\drawUnitLine{DE} and \drawUnitLine{FD}) of the other, but their bases unequal, the angle subtended by the greater base (\drawUnitLine{BC}) of the one, must be greater than the angle subtended by the less base (\drawUnitLine{EF}) of the other.}

\startCenterAlign
$\drawAngle{A} =\mbox{, } > \mbox{ or } < \drawAngle{D}$

\drawAngle{A} is not equal to \drawAngle{D}\\
for if $\drawAngle{A} = \drawAngle{D}$ then $\drawUnitLine{BC} = \drawUnitLine{EF}$ \inprop[prop:I.IV]\\
which is contrary to the hypothesis;

\drawAngle{A} is not less than \drawAngle{D}\\
for if $\drawAngle{A} < \drawAngle{D}$\\
then $\drawUnitLine{BC} < \drawUnitLine{EF}$ \inprop[prop:I.XXIV],\\
which is also contrary to the hypothesis:

$\therefore \drawAngle{A} > \drawAngle{D}$.
\stopCenterAlign

\qed
\stopProposition

\startProposition[title={Prop XXVI. Theor.}, reference=prop:I.XXVI]
\defineNewPicture{
pair A, B, C, D, E, F, G, d;
A := (0, 0);
B := A shifted (3u, 0);
C := A shifted (2u, 3u);
d := (0, -4u);
D := A shifted d;
E := B shifted d;
F := C shifted d;
G := 3/4[D, F];
draw byAngleWithName(B, A, C, byyellow, 0)(A);
draw byAngleWithName(C, B, A, byred, 0)(B);
draw byAngleWithName(A, C, B, black, 1)(C);
byLineDefine(A, B, byblue, 0, 0);
byLineDefine(B, C, black, 0, 0);
byLineDefine(C, A, byred, 0, 0);
draw byNamedLineSeq(0)(CA,BC,AB);
draw byAngleWithName(E, D, F, byyellow, 0)(D);
draw byAngle(G, E, D, black, 0);
draw byAngle(F, E, G, byblue, 0);
draw byAngleWithName(D, F, E, black, 1)(F);
draw byLine(E, G, byyellow, 0, 1);
byLineDefine(D, E, byblue, 0, 1);
byLineDefine(E, F, black, 0, 1);
byLineDefine(F, G, byred, 1, 1);
byLineDefine(G, D, byred, 0, 1);
draw byNamedLineSeq(0)(GD,FG,EF,DE);
draw byLabelsOnPolygon(C, B, A)(0, 0);
draw byLabelsOnPolygon(D, G, F, E)(0, 0);
}
\problemNP{I}{f}{two triangles have two angles of the one respectively equal to two angles of the other ($\drawAngle{A} = \drawAngle{D}$ and $\drawAngle{B} = \drawAngle{GED,FEG}$), and a side of the one equal to a side of the other similarly placed with respect to the equal angles, the remaining sides and angles are respectively equal to one another.}
\drawCurrentPictureInMargin
\startsubproposition[title={Case I.}]
\startCenterAlign
Let \drawUnitLine{AB} and \drawUnitLine{DE} which lie between the equal angles be equal,\\
then $\drawUnitLine{CA} = \drawUnitLine{GD,FG}$.

For if it be possible, let one of them \drawUnitLine{GD,FG} be greater than the other;\\
make $\drawUnitLine{CA} = \drawUnitLine{GD}$, draw \drawUnitLine{EG}.

In \drawLine[bottom]{CA,BC,AB} and
\drawLine[bottom]{GD,EG,DE} we have\\
$\drawUnitLine{CA} = \drawUnitLine{GD}$, $\drawAngle{A} = \drawAngle{D}$, $\drawUnitLine{AB} = \drawUnitLine{DE}$;\\
$\therefore \drawAngle{B} = \drawAngle{GED}$ (pr. 4.)\\
but $\drawAngle{B} = \drawAngle{GED,FEG}$ (hyp.)

and therefore $\drawAngle{GED} = \drawAngle{GED,FEG}$, which is absurd; hence neither of the sides \drawUnitLine{CA} and \drawUnitLine{GD,FG} is greater than the other; and $\therefore$ they are equal;

$\therefore \drawUnitLine{BC} = \drawUnitLine{EF}$, and $\drawAngle{C} = \drawAngle{F}$, \inprop[prop:I.IV].
\stopCenterAlign
\stopsubproposition

\vfill\pagebreak

\defineNewPicture{
pair A, B, C, D, E, F, G, d;
d := (0, -4u);
A := (0, 0);
B := A shifted (3u, 0);
C := A shifted (1u, 3u);
D := A shifted d;
E := B shifted d;
F := C shifted d;
G := 3/4[D, E];
draw byAngleWithName(B, A, C, byyellow, 0)(A);
draw byAngleWithName(C, B, A, byred, 0)(B);
byLineDefine(A, B, byblue, 0, 0);
byLineDefine(B, C, black, 0, 0);
byLineDefine(C, A, byred, 0, 0);
draw byNamedLineSeq(0)(CA,AB,BC);
draw byAngleWithName(F, D, E, byyellow, 0)(D);
draw byAngleWithName(F, G, D, black, 0)(G);
draw byAngleWithName(F, E, D, byred, 0)(E);
draw byLine(F, G, byyellow, 0, 1);
byLineDefine(D, G, byblue, 0, 1);
byLineDefine(G, E, byblue, 1, 1);
byLineDefine(E, F, black, 0, 1);
byLineDefine(F, D, byred, 0, 1);
draw byNamedLineSeq(0)(FD,EF,GE,DG);
draw byLabelsOnPolygon(C, B, A)(0, 0);
draw byLabelsOnPolygon(D, F, E, G)(0, 0);
}
\drawCurrentPictureInMargin
\startsubproposition[title={Case II.}]
\startCenterAlign
Again, let $\drawUnitLine{CA} = \drawUnitLine{FD}$, which lie opposite the equal angles \drawAngle{B} and \drawAngle{E}. If it be possible, let $\drawUnitLine{DG,GE} > \drawUnitLine{AB}$, then take $\drawUnitLine{DG} = \drawUnitLine{AB}$, draw \drawUnitLine{FG}.

Then in \drawLine[bottom]{CA,BC,AB} and \drawLine[bottom]{FD,FG,DG} we have $\drawUnitLine{CA} = \drawUnitLine{FD}$, $\drawUnitLine{AB} = \drawUnitLine{DG}$ and $\drawAngle{A} = \drawAngle{D}$,

$\therefore \drawAngle{B} = \drawAngle{G}$ \inprop[prop:I.IV]\\
but $\drawAngle{B} = \drawAngle{E}$ (hyp.)

$\therefore \drawAngle{G} = \drawAngle{E}$ which is absurd \inprop[prop:I.XVI]

Consequently, neither of the sides \drawUnitLine{AB} or \drawUnitLine{DG,GE} is greater than the other, hence they must be equal. It follows (by \inpropL[prop:I.IV]) that the triangles are equal in all respects.
\stopCenterAlign
\stopsubproposition

\qed
\stopProposition

\startProposition[title={Prop XXVII. Theor.}, reference=prop:I.XXVII]
\defineNewPicture{
pair A, B, C, D, E, F, G, H, I, d;
A := (0, 0);
B := A shifted (8/3u, 0);
d := (0, -7/4u);
C := A shifted d;
D := B shifted d;
E := 1/3[A, B];
F := 2/3[C, D];
G := 3/2[F, E];
H := 3/2[E, F];
I := 1/2[A, C] shifted (-2u, 0);
draw byAngle(A, E, F, byyellow, 0);
draw byAngle(F, E, B, byred, 0);
draw byAngle(C, F, E, byblue, 0);
draw byAngle(E, F, D, byyellow, 0);
byLineDefine(I, A, byblue, 0, 0);
byLineDefine(A, B, byblue, 0, 0);
byLineDefine(I, C, byred, 0, 0);
byLineDefine(C, D, byred, 0, 0);
draw byNamedLineSeq(0)(CD,IC,IA,AB);
draw byLine(G, H, black, 0, 0);
draw byLabelLine(AB, CD, GH);
}
\drawCurrentPictureInMargin
\problemNP{I}{f}{a straight line (\drawUnitLine{GH}) meeting two other straight lines (\drawUnitLine{CD} and \drawUnitLine{AB}) makes with them the alternate angles (\drawAngle{CFE} and \drawAngle{FEB}; \drawAngle{EFD} and \drawAngle{AEF}) equal, these two straight lines are parallel.}

If \drawUnitLine{CD} be not parallel to \drawUnitLine{AB} they shall meet when produced.

If it be possible, let those lines be not parallel, but meet when produced; then the external angle \drawAngle{FEB} is greater than \drawAngle{CFE} \inprop[prop:I.XVI], but they are also equal (hyp.), which is absurd: in the same manner it may be shown that they cannot meet on the other side; $\therefore$ they are parallel.

\qed
\stopProposition

\startProposition[title={Prop XXVIII. Theor.}, reference=prop:I.XXVIII]
\defineNewPicture{
pair A, B, C, D, E, F, G, H, d;
A := (0, 0);
B := A shifted (7/2u, 0);
d := (0, -3/2u);
C := A shifted d;
D := B shifted d;
E := 9/20[A, B];
F := 11/20[C, D];
G := 7/4[F, E];
H := 4/3[E, F];
draw byAngle(G, E, A, black, 0);
draw byAngle(B, E, G, byyellow, 0);
draw byAngle(A, E, F, byred, 0);
draw byAngle(F, E, B, byblue, 0);
draw byAngle(C, F, E, byblue, 0);
draw byAngle(E, F, D, byred, 0);
draw byLine(A, B, byred, 0, 0);
draw byLine(C, D, byyellow, 0, 0);
draw byLine(G, H, black, 0, 0);
draw byLabelLine(AB, CD, GH);
}
\drawCurrentPictureInMargin
\problemNP{I}{f}{a straight line (\drawUnitLine{GH}) meeting two other straight lines (\drawUnitLine{AB} and \drawUnitLine{CD}) makes the external equal to the internal and opposite angle, at the same side of the cutting line (namely $\drawAngle{GEA} = \drawAngle{CFE}$ or $\drawAngle{BEG} = \drawAngle{EFD}$), or if it makes two internal angles at the same side (\drawAngle{EFD} and \drawAngle{FEB}, or \drawAngle{CFE} and \drawAngle{AEF}) together equal to two right angles, those two straight lines are parallel.}

\startCenterAlign
First, if $\drawAngle{GEA} = \drawAngle{CFE}$, then $\drawAngle{GEA} = \drawAngle{FEB}$ \inprop[prop:I.XV],\\
$\therefore \drawAngle{CFE} = \drawAngle{FEB} \therefore \drawUnitLine{AB} \parallel \drawUnitLine{CD}$ \inprop[prop:I.XXVII].

Secondly, if $\drawAngle{CFE} + \drawAngle{AEF} = \drawTwoRightAngles$,\\
then $\drawAngle{AEF} + \drawAngle{FEB} = \drawTwoRightAngles$ \inprop[prop:I.XIII],\\
$\therefore \drawAngle{CFE} + \drawAngle{AEF} = \drawAngle{AEF} + \drawAngle{FEB}$ \inax[ax:III]

$\therefore \drawAngle{CFE} = \drawAngle{FEB}$

$\therefore \drawUnitLine{AB} \parallel \drawUnitLine{CD}$
\stopCenterAlign

\qed
\stopProposition

\startProposition[title={Prop XXIX. Theor.}, reference=prop:I.XXIX]
\defineNewPicture{
pair A, B, C, D, E, F, G, H, I, J, d[];
A := (0, 0);
B := A shifted (7/2u, 0);
d1 := (0, -2u);
C := A shifted d1;
D := B shifted d1;
E := 11/20[A, B];
F := 9/20[C, D];
G := 7/4[F, E];
H := 4/3[E, F];
d2 := (3/2u, 1/2u);
I := E shifted -d2;
J := E shifted d2;
draw byAngle(I, E, A, byblue, 0);
draw byAngle(F, E, I, byyellow, 0);
draw byAngle(B, E, F, black, 0);
draw byAngle(G, E, B, byred, 0);
draw byAngle(E, F, D, black, 0);
draw byLine(I, E, black, 0, 0);
draw byLine(E, J, black, 1, 0);
draw byLine(A, B, byyellow, 0, 0);
draw byLine(C, D, byred, 0, 0);
draw byLine(G, H, byblue, 0, 0);
draw byLabelsOnPolygon(A, E, G)(2, 0);
draw byLabelPoint(I, lineAngle.IE - 90, 1);
draw byLabelPoint(J, lineAngle.EJ + 90, 1);
}
\drawCurrentPictureInMargin
\problemNP{A}{ straight}{line (\drawUnitLine{GH}) falling on two parallel straight lines (\drawUnitLine{AB} and \drawUnitLine{CD}), makes the alternate angles equal to one another; and also the external equal to the internal and opposite angle on the same side; and the two internal angles on the same side together equal to two right angles.}

For if the alternate angles \drawAngle{IEA,FEI} and \drawAngle{EFD} be not equal, draw \drawUnitLine{IE}, making $\drawAngle{FEI} = \drawAngle{EFD}$ \inprop[prop:I.XXIII].

Therefore $\drawUnitLine{IE,EJ} \parallel \drawUnitLine{CD}$ \inprop[prop:I.XXVII] and therefore two straight lines which intersect are parallel to the same straight line, which is impossible \inax[ax:XII].

Hence \drawAngle{IEA,FEI} and \drawAngle{EFD} are not unequal, that is, they are equal: $\drawAngle{IEA,FEI} = \drawAngle{GEB}$ \inprop[prop:I.XV]; $\therefore \drawAngle{GEB} = \drawAngle{EFD}$, the external angle equal to the internal and opposite on the same side: if \drawAngle{BEF} be added to both, then $\drawAngle{EFD} + \drawAngle{BEF} = \drawAngle{BEF,GEB} = \drawTwoRightAngles$ \inprop[prop:I.XIII]. That is to say, the two internal angles at the same side of the cutting line are equal to two right angles.

\qed
\stopProposition

\startProposition[title={Prop XXX. Theor.}, reference=prop:I.XXX]
\defineNewPicture{
pair A, B, C, D, E, F, G, H, I, J, K, d;
A := (0, 0);
B := A shifted (7/2u, 0);
d := (0, -u);
C := A shifted d;
D := B shifted d;
E := C shifted d;
F := D shifted d;
G := 13/20[A, B];
H := 7/20[E, F];
I := (G--H) intersectionpoint (C--D);
J := 3/2[H, G];
K := 5/4[G, H];
draw byAngleWithName(B, G, J, byyellow, 0)(G);
draw byAngleWithName(D, I, J, byblue, 0)(I);
draw byAngleWithName(F, H, J, byred, 0)(H);
draw byLine(A, B, byred, 0, 0);
draw byLine(C, D, byyellow, 0, 0);
draw byLine(E, F, byblue, 0, 0);
draw byLine(J, K, black, 0, 0);
draw byLabelLine(AB, CD, EF, JK);
draw byLabelsOnPolygon(J, G, A)(2, 0);
draw byLabelsOnPolygon(J, I, C)(2, 0);
draw byLabelsOnPolygon(J, H, E)(2, 0);
}
\drawCurrentPictureInMargin
\problemNP{S}{traight}{lines (\drawUnitLine{AB} and \drawUnitLine{EF}) which are parallel to the same straight line (\drawUnitLine{CD}), are parallel to one another.}

\startCenterAlign
Let \drawUnitLine{JK} intersect $\left\{\vcenter{\nointerlineskip\hbox{\drawUnitLine{AB}}\nointerlineskip\hbox{\drawUnitLine{CD}}\nointerlineskip\hbox{\drawUnitLine{EF}}}\right\}$;

Then, $\drawAngle{G} = \drawAngle{I} = \drawAngle{H}$ \inprop[prop:I.XXIX],

$\therefore \drawAngle{G} = \drawAngle{H}$

$\therefore \drawUnitLine{AB} \parallel \drawUnitLine{EF}$ \inprop[prop:I.XXVII].
\stopCenterAlign

\qed
\stopProposition

\startProposition[title={Prop XXXI. Prob.}, reference=prop:I.XXXI]
\defineNewPicture[1/6]{
pair A, B, C, D, E, F, d;
A := (0, 0);
B := A shifted (7/2u, 0);
d := (0, -3u);
C := A shifted d;
D := B shifted d;
E := 4/5[A, B];
F := 1/5[C, D];
draw byAngleWithName(F, E, A, byyellow, 0)(E);
draw byAngleWithName(E, F, D, byred, 0)(F);
draw byLine(E, F, black, 0, 0);
draw byLine(A, E, byred, 0, 0);
draw byLine(E, B, byred, 1, 0);
draw byLine(C, F, byblue, 0, 0);
draw byLine(F, D, byblue, 0, 0);
draw byLabelsOnPolygon(A, E, B, noPoint, D, F, C, noPoint)(0, 0);
}
\drawCurrentPictureInMargin
\problemNP{F}{rom}{a given point
\drawFromCurrentPicture[middle][pointE]{
draw byNamedLineSeq(0)(AE,EF);
draw byLabelsOnPolygon(A, E, F)(2, 0);
}
to draw a straight line parallel to a given straight line (\drawUnitLine{CF,FD}).}

\startCenterAlign
Draw \drawUnitLine{EF} from the point \pointE to any point 
\drawFromCurrentPicture[middle][pointF]{
draw byNamedLineSeq(0)(FD,EF);
draw byLabelsOnPolygon(D, F, E)(2, 0);
}
in \drawUnitLine{CF,FD},

make $\drawAngle{E} = \drawAngle{F}$ \inprop[prop:I.XXIII],

then $\drawUnitLine{AE,EB} \parallel \drawUnitLine{CF,FD}$ \inprop[prop:I.XXVII].
\stopCenterAlign

\qed
\stopProposition

\startProposition[title={Prop XXXII. Theor.}, reference=prop:I.XXXII]
\defineNewPicture[1/6]{
pair A, B, C, D, E;
A := (0, 0);
B := A shifted (-7/4u, -3u);
C := A shifted (7/4u, -3u);
D := 4/3[B, A];
E := A shifted (unitvector(C-B) scaled 3/2u);
draw byAngle(D, A, E, byred, 0);
draw byAngle(E, A, C, black, 0);
draw byAngle(C, A, B, byblue, 0);
draw byAngleWithName(A, B, C, byyellow, 0)(B);
draw byAngleWithName(B, C, A, black, 0)(C);
draw byLine(A, E, byblue, 0, 0);
byLineDefine(A, D, black, 1, 0);
byLineDefine(A, B, black, 0, 0);
byLineDefine(B, C, byred, 0, 0);
byLineDefine(C, A, byyellow, 0, 0);
draw byNamedLineSeq(0)(CA,noLine,AD,AB,BC);
draw byLabelsOnPolygon(C, B, A, D, noPoint, E, A)(0, 0);
}
\drawCurrentPictureInMargin
\problemNP[2]{I}{f}{any side (\drawUnitLine{AB}) of a triangle by produced, the external angle (\drawAngle{DAE,EAC}) is equal to the sum of two internal and opposite angles (\drawAngle{B} and \drawAngle{C}), and the three internal angles of any triangle taken together are equal to two right angles.}

\startCenterAlign
Through the point 
\drawFromCurrentPicture{
draw byNamedLineSeq(0)(AB,CA);
draw byLabelsOnPolygon(B, A, C)(2, 0);
}
draw\\
$\drawUnitLine{AE} \parallel \drawUnitLine{BC}$ \inprop[prop:I.XXXI].

Then $\left\{\eqalign{\drawAngle{DAE} &= \drawAngle{B}\cr \drawAngle{EAC} &= \drawAngle{C}\cr}\right\}$ \inprop[prop:I.XXIX],

$\therefore \drawAngle{B} + \drawAngle{C} = \drawAngle{DAE,EAC}$ \inax[ax:II],

and therefore\\
$\drawAngle{B} + \drawAngle{CAB} + \drawAngle{C} = \drawAngle{DAE,EAC,CAB} = \drawTwoRightAngles$ \inprop[prop:I.XIII]
\stopCenterAlign

\qed
\stopProposition

\startProposition[title={Prop XXXIII. Theor.}, reference=prop:I.XXXIII]
\defineNewPicture[1/4]{
pair A, B, C, D, d[];
d1 := (5/2u, 0);
d2 := (-7/8u, -3u);
A := (0, 0);
B := A shifted d1;
C := A shifted d2;
D := C shifted d1;
draw byAngle(B, A, D, byyellow, 0);
draw byAngle(D, A, C, byred, 0);
draw byAngle(C, D, A, black, 0);
draw byAngle(A, D, B, byblue, 0);
draw byLine(A, D, black, 0, 0);
byLineDefine(A, B, byred, 0, 0);
byLineDefine(C, D, byred, 1, 0);
byLineDefine(A, C, byblue, 0, 0);
byLineDefine(B, D, byyellow, 0, 0);
draw byNamedLineSeq(0)(AB,BD,CD,AC);
draw byLabelsOnPolygon(A, B, D, C)(0, 0);
}
\drawCurrentPictureInMargin
\problemNP{S}{traight}{lines (\drawUnitLine{AC} and \drawUnitLine{BD}) which join the adjacent extremities of two equal and parallel straight lines (\drawUnitLine{AB} and \drawUnitLine{CD}), are themselves parallel.}

\startCenterAlign
Draw \drawUnitLine{AD} the diagonal.

$\drawUnitLine{AB} = \drawUnitLine{CD}$ (hyp.)\\
$\drawAngle{BAD} = \drawAngle{CDA}$ \inprop[prop:I.XXIX]\\
and \drawUnitLine{AD} common to the two triangles;

$\therefore \drawUnitLine{AC} = \drawUnitLine{BD}$, and $\drawAngle{ADB} = \drawAngle{DAC}$ \inprop[prop:I.IV];

and $\therefore \drawUnitLine{AC} \parallel \drawUnitLine{BD}$ \inprop[prop:I.XXVII].
\stopCenterAlign

\qed
\stopProposition

\startProposition[title={Prop XXXIV. Theor.}, reference=prop:I.XXXIV]
\defineNewPicture{
pair A, B, C, D, d[];
d1 := (5/2u, 0);
d2 := (-7/8u, -3u);
A := (0, 0);
B := A shifted d1;
C := A shifted d2;
D := C shifted d1;
draw byAngle(B, A, D, byblue, 0);
draw byAngle(D, A, C, byred, 0);
draw byAngle(C, D, A, byyellow, 0);
draw byAngle(A, D, B, byred, 0);
draw byAngleWithName(A, C, D, black, 0)(C);
draw byAngleWithName(D, B, A, black, 0)(B);
draw byLine(A, D, black, 0, 0);
byLineDefine(A, B, byred, 0, 0);
byLineDefine(C, D, byred, 1, 0);
byLineDefine(A, C, byyellow, 0, 0);
byLineDefine(B, D, byblue, 0, 0);
draw byNamedLineSeq(0)(AB,BD,CD,AC);
draw byLabelsOnPolygon(A, B, D, C)(0, 0);
}
\drawCurrentPictureInMargin
\problemNP{T}{he}{opposite sides and angles of any parallelogram are equal, and the diagonal (\drawUnitLine{AD}) divides it into two equal parts.}

\startCenterAlign
Since $\left\{\eqalign{\drawAngle{BAD} &= \drawAngle{CDA}\cr\drawAngle{DAC} &= \drawAngle{ADB}\cr}\right\}$ \inprop[prop:I.XXIX]\\
and \drawUnitLine{AD} common to the two triangles.

$\therefore \left\{\eqalign{\drawUnitLine{AB} &= \drawUnitLine{CD}\cr \drawUnitLine{AC} &= \drawUnitLine{BD}\cr \drawAngle{B} &= \drawAngle{C}\cr}\right\}$ \inprop[prop:I.XXVI]\\
and $\drawAngle{BAD,DAC} = \drawAngle{CDA,ADB}$ \inax[ax:II]:
\stopCenterAlign

Therefore the opposite sides and angles of the parallelogram are equal: and as the triangles \drawLine{AD,CD,AC} and \drawLine{AB,BD,AD} are equal in every respect \inprop[prop:I.IV], the diagonal divides the parallelogram into two equal parts.

\qed
\stopProposition

\startProposition[title={Prop XXXV. Theor.}, reference=prop:I.XXXV]
\defineNewPicture{
pair A, B, C, D, E, F, G, d[];
d1 := (7/4u, 0);
d2 := (u, -3u);
d3 := (-2u, -3u);
A := (0, 0);
B := A shifted d1;
C := A shifted d2;
D := C shifted d1;
E := C shifted -d3;
F := D shifted -d3;
G := (B--D) intersectionpoint (C--E);
draw byPolygon(A,B,G,C)(byyellow);
draw byPolygon(E,F,D,G)(byyellow);
draw byPolygon(B,E,G)(byyellow);
draw byPolygon(C,D,G)(byblue);
draw byAngleWithName(B, A, C, byred, 0)(A);
draw byAngleWithName(E, B, D, byblue, 0)(B);
draw byAngleWithName(A, E, C, black, 0)(E);
draw byAngleWithName(A, F, D, white, 0)(F);
byLineDefine(A, C, byblue, 0, 0);
byLineDefine(B, D, byred, 0, 0);
byLineDefine(C, D, black, 0, 0);
draw byNamedLineSeq(0)(AC,CD,BD);
draw byLabelsOnPolygon(C, A, B, E, F, D)(0, 0);
}
\drawCurrentPictureInMargin
\problemNP{P}{arallelograms}{on the same base, and between the same parallels, are (in area) equal.}

\startCenterAlign
On account of the parallels,\\
$\left.\eqalign{
\drawAngle{A} &= \drawAngle{B};\cr
\drawAngle{E} &= \drawAngle{F};\cr
\mbox{and } \drawUnitLine{AC} &= \drawUnitLine{BD}\cr
}\right\}
\eqalign{
&\mbox{\inprop[prop:I.XXIX]}\cr
&\mbox{\inprop[prop:I.XXIX]}\cr
&\mbox{\inprop[prop:I.XXXIV]}\cr
}$

But
$\drawFromCurrentPicture[middle][polygonABC]{
startAutoLabeling;
draw byNamedPolygon (ABGC, BEG);
stopAutoLabeling;
draw byNamedLine (AC);
}
=
\drawFromCurrentPicture[middle][polygonEFD]{
startAutoLabeling;
draw byNamedPolygon (EFDG, BEG);
stopAutoLabeling;
draw byNamedLine (BD);
}$ \inprop[prop:I.VIII]

$\therefore
\drawFromCurrentPicture[middle][polygonAFDC]{
startAutoLabeling;
draw byNamedPolygon (ABGC, EFDG, BEG, CDG);
stopAutoLabeling;
draw byNamedLine (AC);
} - \polygonEFD =
\drawPolygon{ABGC, CDG}$,\\
and $\drawPolygon{ABGC, CDG} - \polygonABC =
\drawPolygon{EFDG, CDG}$;

$\therefore \drawPolygon{ABGC, CDG} = \drawPolygon{EFDG, CDG}$.
\stopCenterAlign

\qed
\stopProposition

\startProposition[title={Prop XXXVI. Theor.}, reference=prop:I.XXXVI]
\defineNewPicture{
pair A, B, C, D, E, F, G, H, J, I, d[];
numeric h;
h := 3u;
d1 := (3/2u, 0);
d2 := (2/3u, -h);
d3 := (-8/3u, -h);
d4 := (-1/2u, -h);
A := (0, 0);
B := A shifted d1;
C := A shifted d2;
D := C shifted d1;
E := C shifted -d3;
F := D shifted -d3;
G := E shifted d4;
H := F shifted d4;
I := (B--D) intersectionpoint (C--E);
J := (E--G) intersectionpoint (D--F);
draw byPolygon(A,B,I,C)(byred);
draw byPolygon(C,D,I)(byred);
draw byPolygon(I,D,J,E)(byblue);
draw byPolygon(E,F,J)(byyellow);
draw byPolygon(J,F,H,G)(byyellow);
byLineDefine(C, E, byyellow, 0, 0);
byLineDefine(D, F, black, 1, 0);
byLineDefine(C, D, black, 0, 0);
byLineDefine(E, F, byred, 0, 0);
draw byNamedLineSeq(0)(CE,EF,DF,CD);
draw byLineFull(G, H, byblue, 0, 0)(E, F, 1, 1, 0);
draw byLabelsOnPolygon(A, B, D, C)(0, 0);
draw byLabelsOnPolygon(E, F, H, G)(0, 0);
}
\drawCurrentPictureInMargin
\problemNP{P}{arallelograms}{(\drawPolygon[bottom][polygonABDC]{ABIC, CDI}~and~\drawPolygon[bottom][polygonEFHG]{EFJ, JFHG}) on equal bases, and between the same parallels are equal.}

\startCenterAlign
Draw \drawUnitLine{CE} and \drawUnitLine{DF}\\
$\drawUnitLine{CD} = \drawUnitLine{GH} = \drawUnitLine{EF}$ by (\inpropL[prop:I.XXXIV], and hyp.);

$\therefore \drawUnitLine{CD} = \mbox{ and } \parallel \drawUnitLine{EF}$;

$\therefore \drawUnitLine{CE} = \mbox{ and } \parallel \drawUnitLine{DF}$ \inprop[prop:I.XXXV]

And therefore
\drawPolygon[bottom][polygonCDFE]{CDI, IDJE, EFJ}
is a parallelogram:

but $\polygonABDC = \polygonCDFE = \polygonEFHG$ \inprop[prop:I.XXXV]

$\therefore \polygonABDC = \polygonEFHG$ \inax[ax:I].
\stopCenterAlign

\qed
\stopProposition

\startProposition[title={Prop XXXVII. Theor.}, reference=prop:I.XXXVII]
\defineNewPicture{
pair A, B, C, D, E, F, G, H, I, d[];
d1 := (3/2u, 0);
d2 := (3/4u, -3u);
d3 := (-7/4u, -3u);
A := (0, 0);
B := A shifted d1;
C := A shifted d2;
D := C shifted d1;
E := C shifted -d3;
F := D shifted -d3;
G := (B--D) intersectionpoint (C--E);
H := 11/10[F, A];
I := 11/10[A, F];
draw byPolygon(A,B,C)(byblue);
draw byPolygon(B,C,G)(byyellow);
draw byPolygon(C,D,G)(byyellow);
draw byPolygon(D,G,E)(black);
draw byPolygon(E,F,D)(byred);
draw byLine(B, D, byred, 0, 0);
draw byLine(E, C, byblue, 0, 0);
byLineDefine(A, C, byred, 1, 0);
byLineDefine(F, D, byblue, 1, 0);
byLineDefine(C, D, black, 0, 0);
draw byNamedLineSeq(0)(AC,CD,FD);
draw byLine(H, I, black, 1, 0);
draw byLabelsOnPolygon(D, C, A, B, E, F)(0, 0);
draw byLabelLine(HI);
}
\drawCurrentPictureInMargin
\problemNP{T}{riangles}{
\drawPolygon[bottom][polygonBCD]{BCG, CDG} and~\drawPolygon[bottom][polygonCDE]{DGE, CDG}
on the same base (\drawUnitLine{CD}) and between the same parallels are equal.
}

\startCenterAlign
$\left.\eqalign{\mbox{Draw} \drawUnitLine{AC} &\parallel \drawUnitLine{BD}\cr
\drawUnitLine{FD} &\parallel \drawUnitLine{EC}}\right\}\mbox{\inprop[prop:I.XXXI]}$

Produce \drawUnitLine{HI}.

\drawPolygon[bottom][polygonABDC]{ABC, BCG, CDG}
and
\drawPolygon[bottom][polygonEFDC]{DGE, CDG, EFD}
are parallelograms on the same base and between the same parallels, and therefore equal. \inprop[prop:I.XXXV]

$\therefore \left\{\eqalign{\polygonABDC &= \mbox{ twice } \polygonBCD\cr\polygonEFDC &= \mbox{ twice } \polygonCDE\cr}\right\}$ \inprop[prop:I.XXXIV]

$\therefore \polygonBCD = \polygonCDE$.
\stopCenterAlign

\qed
\stopProposition

\startProposition[title={Prop XXXVIII. Theor.}, reference=prop:I.XXXVIII]
\defineNewPicture{
pair A, B, C, D, E, F, G, H, J, I, d[];
numeric h;
h := 3u;
d1 := (3/2u, 0);
d2 := (2/3u, -h);
d3 := (-8/3u, -h);
d4 := (-1/2u, -h);
A := (0, 0);
B := A shifted d1;
C := A shifted d2;
D := C shifted d1;
E := C shifted -d3;
F := D shifted -d3;
G := E shifted d4;
H := F shifted d4;
I := (xpart(A), ypart(C));
J := (xpart(F), ypart(C));
draw byPolygon(A,B,C)(byyellow);
draw byPolygon(B,C,D)(byred);
draw byPolygon(E,F,H)(black);
draw byPolygon(E,G,H)(byblue);
draw byLine(B, D, byblue, 0, 0);
draw byLine(E, G, byred, 0, 0);
byLineDefine(A, C, byblue, 1, 0);
byLineDefine(F, H, byred, 1, 0);
byLineDefine(A, F, black, 1, 0);
draw byNamedLineSeq(0)(AC,AF,FH);
draw byLine(I, J, black, 1, 0);
draw byLabelsOnPolygon(A, B, D, C)(0, 0);
draw byLabelsOnPolygon(E, F, H, G)(0, 0);
}
\drawCurrentPictureInMargin
\problemNP{T}{riangles}{(\drawPolygon[bottom][polygonBCD]{BCD} and \drawPolygon[bottom][polygonEGH]{EGH}) on equal bases and between the same parallels are equal.}

\startCenterAlign
$\left.\eqalign{\mbox{Draw } \drawUnitLine{AC} &\parallel \drawUnitLine{BD}\cr
\mbox{and } \drawUnitLine{FH} &\parallel \drawUnitLine{EG}}\right\}\mbox{\inprop[prop:I.XXXI]}$\\
$\drawPolygon[bottom][polygonABDC]{ABC, BCD} = \drawPolygon[bottom][polygonEFHG]{EFH, EGH}$ \inprop[prop:I.XXXVI];

but $\polygonABDC = \mbox{ twice } \polygonBCD$ \inprop[prop:I.XXXIV],\\
and $\polygonEFHG = \mbox{ twice } \polygonEGH$ \inprop[prop:I.XXXIV],

$\therefore \polygonBCD = \polygonEGH$ \inax[ax:VII].
\stopCenterAlign

\qed
\stopProposition

\startProposition[title={Prop XXXIX. Theor.}, reference=prop:I.XXXIX]
\defineNewPicture{
pair A, B, C, D, E, F, G;
A := (0, 0);
B := A shifted (5/2u, 0);
C := A shifted (u, -5/2u);
D := C shifted (3u, 0);
E = whatever[A, D] = whatever[B, C];
F := 5/4[C, B];
G := 6/4[C, B];
draw byPolygon(A,B,E)(byred);
draw byPolygon(A,E,C)(byyellow);
draw byPolygon(B,E,D)(black);
draw byPolygon(E,C,D)(byyellow);
draw byPolygon(F,B,D)(byblue);
byLineDefine(A, F, byred, 0, 0);
byLineDefine(D, F, byyellow, 0, 0);
byLineDefine(A, B, byblue, 0, 0);
byLineDefine(C, D, black, 0, 0);
byLineDefine(C, G, black, 1, 0);
draw byNamedLineSeq(0)(AB,AF,DF,CD,CG);
draw byLabelsOnPolygon(F, D, C, A)(2, 0);
draw byLabelsOnPolygon(A, B, F)(2, 0);
draw byLabelsOnPolygon(A, F, G, noPoint)(0, 0);
}
\drawCurrentPictureInMargin
\problemNP{E}{qual}{triangles
\drawPolygon[bottom][polygonADC]{AEC, ECD} and~\drawPolygon[bottom][polygonBDC]{BED, ECD}
on the same base (\drawUnitLine{CD}) and on the same side of it, are between the same parallels.}

\startCenterAlign
If \drawUnitLine{AB}, which joins the vertices of the triangles, be not $\parallel \drawUnitLine{CD}$, draw $\drawUnitLine{AF} \parallel \drawUnitLine{CD}$ \inprop[prop:I.XXXI], meeting \drawUnitLine{CG}.

Draw \drawUnitLine{DF}.

Because $\drawUnitLine{AF} \parallel \drawUnitLine{CD}$ (const.)\\
$\polygonADC =
\drawPolygon[bottom][polygonFDC]{BED, ECD, FBD}$ \inprop[prop:I.XXXVII];\\
but $\polygonADC = \polygonBDC$ (hyp.);

$\therefore \polygonBDC = \polygonFDC$, a part equal to the whole, which is absurd.

$\therefore \drawUnitLine{AF} \nparallel \drawUnitLine{CD}$; and in the same manner it can be demonstrated, that no other line except \drawUnitLine{AB} is $\parallel \drawUnitLine{CD}$; $\therefore \drawUnitLine{AB} \parallel \drawUnitLine{CD}$.
\stopCenterAlign

\qed
\stopProposition

\startProposition[title={Prop XL. Theor.}, reference=prop:I.XL]
\defineNewPicture{
pair A, B, C, D, E, F, G, H, d;
A := (0, 0);
B := A shifted (5/2u, 0);
C := A shifted (-u, -9/4u);
d := (2u, 0);
D := C shifted d;
E := B shifted (-1/2u, -9/4u);
F := E shifted d;
G := 5/4[E, B];
H := 2[B, G];
draw byPolygon(A,C,D)(byyellow);
draw byPolygon(B,E,F)(byred);
draw byPolygon(G,B,F)(byblue);
draw byLine(E, H, black, 1, 0);
byLineDefine(A, G, byred, 0, 0);
byLineDefine(F, G, byyellow, 0, 0);
byLineDefine(A, B, byblue, 0, 0);
byLineDefine(C, D, black, 0, 0);
byLineDefine(E, F, black, 0, 0);
byLineDefine(D, E, byblue, 1, 0);
draw byNamedLineSeq(0)(AB,AG,FG,EF,DE,CD);
draw byLabelsOnPolygon(G, F, E, D, C, A)(2, 0);
draw byLabelsOnPolygon(A, B, G)(2, 0);
draw byLabelsOnPolygon(A, G, H, noPoint)(0, 0);
}
\drawCurrentPictureInMargin
\problemNP{E}{qual}{triangles
(\drawFromCurrentPicture[bottom][polygonACD]{
startAutoLabeling;
draw byNamedPolygon (ACD);
stopAutoLabeling;
draw byNamedLineFull(A, A, 1, 1, 0)(CD);
}
and
\drawFromCurrentPicture[bottom][polygonBEF]{
startAutoLabeling;
draw byNamedPolygon (BEF);
stopAutoLabeling;
draw byNamedLineFull(B, B, 1, 1, 0)(EF);
}) on equal bases, and on the same side, are between the same parallels.}

\startCenterAlign
If \drawSizedLine{AB} which joins the vertices of triangles be not $\parallel \drawSizedLine{CD,DE,EF}$,\\
draw \drawSizedLine{AG} $\parallel \drawSizedLine{CD,DE,EF}$ \inprop[prop:I.XXXI], meeting \drawSizedLine{EH}.\\
Draw \drawSizedLine{FG}.

Because $\drawSizedLine{AG} \parallel \drawSizedLine{CD,DE,EF}$ (const.)\\
$\polygonACD =
\drawFromCurrentPicture[bottom][polygonGEF]{
startAutoLabeling;
draw byNamedPolygon (BEF, GBF);
stopAutoLabeling;
draw byNamedLineFull(G, G, 1, 1, 0)(EF);
}$ but $\polygonACD = \polygonBEF$

$\therefore \polygonBEF = \polygonGEF$, a part equal to the whole, which is absurd.

$\therefore \drawSizedLine{AG} \nparallel \drawSizedLine{CD,DE,EF}$: and in the same manner it can be demonstrated, that no other line except \drawSizedLine{AB} is $\parallel \drawSizedLine{CD,DE,EF}$: $\therefore \drawSizedLine{AB} \parallel \drawSizedLine{CD,DE,EF}$.
\stopCenterAlign

\qed
\stopProposition

\startProposition[title={Prop XLI. Theor.}, reference=prop:I.XLI]
\defineNewPicture{
pair A, B, C, D, E, F, G, d;
A := (0, 0);
d := (2u, 0);
B := A shifted d;
C := B shifted (2u, 0);
D := A shifted (4/3u, -3u);
E := D shifted d;
F = whatever[B, E] = whatever[D, C];
G = whatever[A, E] = whatever[D, C];
draw byPolygon(A,B,F,G)(byyellow);
draw byPolygon(G,F,E)(byyellow);
draw byPolygon(A,G,D)(byblue);
draw byPolygon(D,E,G)(byblue);
draw byPolygon(C,F,E)(byred);
draw byLine(A, E, byred, 0, 0);
draw byLineFull(A, C, black, 1, 0)(D, E, 1, 1, 0);
draw byLineFull(D, E, black, 0, 0)(A, C, 1, 1, 0);
draw byLabelsOnPolygon(E, D, A, B, C)(0, 0);
}
\drawCurrentPictureInMargin
\problemNP[2]{I}{f}{a parallelogram
\drawPolygon[bottom][polygonABED]{ABFG,GFE,AGD,DEG}
and a triangle
\drawPolygon[bottom][polygonCED]{GFE,DEG,CFE}
are upon the same base \drawUnitLine{DE} and between the same parallels \drawUnitLine{AC} and \drawUnitLine{DE}, the parallelogram is double the triangle.}

\startCenterAlign
Draw \drawUnitLine{AE} the diagonal;

Then
$\drawPolygon[bottom][polygonAED]{AGD,DEG} = \polygonCED$ \inprop[prop:I.XXXVII]\\
$\polygonABED = \mbox{ twice } \polygonAED$ \inprop[prop:I.XXXIV]

$\therefore \polygonABED = \mbox{ twice } \polygonCED$.
\stopCenterAlign

\qed
\stopProposition

\startProposition[title={Prop XLII. Prob.}, reference=prop:I.XLII]
\defineNewPicture{
pair A, B, C, D, E, F, G, H, I, J, d;
A := (0, 0);
d := (2u, 0);
B := A shifted d;
C := B shifted (2u, 0);
D := A shifted (u, -3u);
E := D shifted d;
F := (B--E) intersectionpoint (D--C);
G := 2[D, E];
H := B shifted (xpart(G)-xpart(E), -ypart(D)-u);
I := E shifted (xpart(G)-xpart(E), -ypart(D)-u);
J := D shifted (xpart(G)-xpart(E), -ypart(D)-u);
draw byPolygon(A,B,F,D)(byyellow);
draw byPolygon(D,E,F)(byyellow);
draw byPolygon(C,F,E)(byblue);
draw byPolygon(E,C,G)(black);
draw byAngleWithName(B, E, D, byblue, 0)(E);
draw byAngleWithName(H, I, J, byyellow, 0)(I);
angleStandalone.I := true;
draw byLine(A, D, byred, 1, 0);
draw byLine(B, E, byred, 0, 0);
draw byLine(C, E, byyellow, 0, 0);
draw byLine(A, C, byblue, 0, 0);
draw byLine(D, E, black, 0, 0);
draw byLine(E, G, black, 1, 0);
draw byLabelsOnPolygon(G, E, D, A, B, C)(0, 0);
startAutoLabeling;
draw byNamedAngleSidesFull(I)();
stopAutoLabeling;
}
\drawCurrentPicture
\problemNP[2]{T}{o}{construct a parallelogram equal to a given triangle
\drawPolygon[bottom][polygonCDG]{DEF,CFE,ECG} and having an angle equal to a given rectilinear angle \drawAngle{I}.}

\startCenterAlign
Make $\drawUnitLine{DE} = \drawUnitLine{EG}$ \inprop[prop:I.X]\\
Draw \drawUnitLine{CE}.

Make $\drawAngle{E} = \drawAngle{I}$ \inprop[prop:I.XXIII]\\
Draw $\left\{\eqalign{\drawUnitLine{AD} &\parallel \drawUnitLine{BE}\cr\drawUnitLine{AC} &\parallel \drawUnitLine{DE}\cr}\right\}$ \inprop[prop:I.XXXI]

$\drawPolygon[bottom][polygonABED]{ABFD,DEF}
= \mbox{ twice }
\drawPolygon[bottom][polygonCED]{DEF,CFE}$ \inprop[prop:I.XLI]\\
but $\polygonCED =
\drawPolygon[bottom][polygonDCG]{ECG}$ \inprop[prop:I.XXXVIII]

$\therefore \polygonABED = \polygonCDG$.
\stopCenterAlign

\qed
\stopProposition

\startProposition[title={Prop XLIII. Theor.}, reference=prop:I.XLIII]
\defineNewPicture[1/2]{
pair A, B, C, D, E, F, G, H, I, d[];
path q[];
d1 := (5/2u, 0);
d2 := (-u, -3u);
A := (0, 0);
B := A shifted d1;
C := A shifted d2;
D := C shifted d1;
E := 2/5[A, D];
q1 := (E shifted d1) -- (E shifted -d1);
q2 := (E shifted d2) -- (E shifted -d2);
F := q1 intersectionpoint (A--C);
G := q1 intersectionpoint (B--D);
H := q2 intersectionpoint (A--B);
I := q2 intersectionpoint (C--D);
draw byPolygon(A,E,H)(byyellow);
draw byPolygon(A,E,F)(byyellow);
draw byPolygon(H,B,G,E)(byblue);
draw byPolygon(F,C,I,E)(black);
draw byPolygon(I,D,E)(byred);
draw byPolygon(G,D,E)(byred);
draw byLabelsOnPolygon(C, F, A, H, B, G, D, I)(0, 0);
draw byLabelsOnPolygon(F, E, H)(2, 0);
}
\drawCurrentPictureInMargin
\problemNP{T}{he}{complements
\drawPolygon[bottom][polygonHBGE]{HBGE}
and
\drawPolygon[bottom][polygonFCIE]{FCIE}
of the parallelograms which are about the diagonal of a parallelogram are equal.}

\startCenterAlign
$\drawPolygon[bottom][polygonADC]{AEF,FCIE,IDE} = \drawPolygon[bottom][polygonABD]{AEH,HBGE,GDE}$ \inprop[prop:I.XXXIV]\\
and $\drawPolygon[bottom][polygonAEFpIDE]{AEF,IDE} = \drawPolygon[bottom][polygonAEHpGDE]{AEH,GDE}$ \inprop[prop:I.XXXIV]

$\therefore \polygonFCIE = \polygonHBGE$ \inax[ax:III]
\stopCenterAlign

\qed
\stopProposition

\startProposition[title={Prop XLIV. Prob.}, reference=prop:I.XLIV]
\defineNewPicture{
pair A, B, C, D, E, F, G, H, I, J, K, L, M, N, O, d[];
path q[];
d1 := (3u, 0);
d2 := (-3/2u, -3u);
d3 := (3/2u, 5/2u);
d4 := -d2 +1/2d1;
A := (0, 0);
B := A shifted d1;
C := A shifted d2;
D := C shifted d1;
E := 2/5[C, B];
q1 := (E shifted d1) -- (E shifted -d1);
q2 := (E shifted d2) -- (E shifted -d2);
F := q1 intersectionpoint (A--C);
G := q1 intersectionpoint (B--D);
H := q2 intersectionpoint (A--B);
I := q2 intersectionpoint (C--D);
J := A shifted d3;
K := J shifted (2(xpart(A)-xpart(H)), 0);
L := (xpart(1/3[J, K]), ypart(F)-ypart(A)+ypart(J));
M := A shifted d4;
N := F shifted d4;
O := E shifted d4;
draw byPolygon(J,K,L)(byred);
draw byPolygon(A,H,E,F)(byyellow);
draw byPolygon(E,G,D,I)(byblue);
draw byAngleWithName(A, F, E, byblue, 0)(F);
draw byAngleWithName(F, E, I, byred, 0)(E);
draw byAngleWithName(E, I, D, black, 0)(I);
draw byAngleWithName(M, N, O, byyellow, 0)(N);
angleStandalone.N := true;
draw byLine(B, E, byred, 0, 0);
draw byLine(E, C, black, 0, 1);
byLineDefine(A, F, byred, 1, 0);
byLineDefine(F, C, black, 0, 1);
draw byLine(H, E, byblue, 1, 0);
byLineDefine(B, G, byyellow, 0, 0);
byLineDefine(A, H, byblue, 0, 0);
byLineDefine(H, B, black, 0, 0);
byLineDefine(F, E, black, 1, 0);
byLineDefine(E, G, black, 0, 0);
byLineDefine(C, D, byyellow, 1, 0);
draw byNamedLineSeq(0)(FE,EG,BG,HB,AH,AF,FC,CD);
draw byLabelsOnPolygon(K, J, L)(0, 0);
draw byLabelsOnPolygon(F, A, H, B, G, D, I, C)(0, 0);
draw byLabelsOnPolygon(F, E, H)(2, 0);
startAutoLabeling;
draw byNamedAngleSidesFull(N)();
stopAutoLabeling;
}
\drawCurrentPicture
\problemNP{T}{o}{a given straight line (\drawUnitLine{EG}) to apply a parallelogram equal to a given triangle (\drawPolygon[bottom][polygonJKL]{JKL}), and having an angle equal to a given rectilinear angle (\drawAngle{N}).}

\startCenterAlign
Make $\drawPolygon[bottom][polygonAHEF]{AHEF} = \polygonJKL$ with $\drawAngle{F} = \drawAngle{N}$ \inprop[prop:I.XLII]\\
and having one of its sides \drawUnitLine{FE} conterminous with and in continuation of \drawUnitLine{EG}.

Produce \drawUnitLine{AH} till it meets $\drawUnitLine{BG} \parallel \drawUnitLine{HE}$ draw \drawUnitLine{BE} produce it till it meets \drawUnitLine{AF} continued; draw $\drawUnitLine{CD} \parallel \drawUnitLine{FE,EG}$ meeting \drawUnitLine{BG} produced and produce \drawUnitLine{HE}.

$\polygonAHEF = \drawPolygon[bottom][polygonEGDI]{EGDI}$ \inprop[prop:I.XLIII]\\
but $\polygonAHEF = \polygonJKL$ (const.)

$\therefore \polygonEGDI = \polygonJKL$;

and $\drawAngle{F} = \drawAngle{E} =\drawAngle{I} = \drawAngle{N}$ (\inpropL[prop:I.XXIX] and const.)
\stopCenterAlign

\qed
\stopProposition

\startProposition[title={Prop XLV. Prob.}, reference=prop:I.XLV]
\defineNewPicture{
pair A, B, C, D, E, F, G, H, I, J, K, L, M, N, O, P, d[];
numeric a, h[], b[], s[];
a := 15;
A := (0, 0);
B := A shifted (0, 2u);
C := A shifted (4/3u, u);
D := A shifted (2u, -3/2u);
E := A shifted (-6/5u, -u);
b1 := arclength(B--C);
h1 := distanceToLine(A, B--C);
s1 := (b1 * h1)/2;
b2 := arclength(C--D);
h2 := distanceToLine(A, C--D);
s2 := (b2 * h2)/2;
b3 := arclength(D--E);
h3 := distanceToLine(A, D--E);
s3 := (b3 * h3)/2;
d1 := (0, ypart(D)-u);
d2 := (0, -b3/2) rotated -a;
d3 := (h3*(1/cosd(a)), 0);
d6 := (u, 0);
F := (-u, 0) shifted d1;
G := F shifted d3;
H := F shifted d2;
I := G shifted d2;
d4 := (2*(s2/b3)*(1/cosd(a)), 0);
J := G shifted d4;
K := J shifted d2;
d5 := (2*(s1/b3)*(1/cosd(a)), 0);
L := J shifted d5;
M := L shifted d2;
N := L shifted d6;
O := M shifted d6;
P := K shifted d6;
draw byPolygon(A,B,C)(byred);
draw byPolygon(A,C,D)(byyellow);
draw byPolygon(A,D,E)(byblue);
byLineDefine(A, C, byblue, 0, 0);
byLineDefine(A, D, byred, 0, 0);
draw byNamedLineSeq(0)(AC,AD);
draw byPolygon(F,G,I,H)(byblue);
draw byPolygon(G,J,K,I)(byyellow);
draw byPolygon(J,L,M,K)(byred);
draw byAngleWithName(G, I, H, byyellow, 0)(I);
draw byAngleWithName(J, K, I, black, 0)(K);
draw byAngleWithName(L, M, K, byblue, 0)(M);
draw byAngleWithName(N, O, P, byred, 0)(O);
angleStandalone.O := true;
draw byLine(G, I, byred, 0, 0);
draw byLine(J, K, byblue, 0, 0);
draw byLabelsOnPolygon(A, B, C, D, E)(0, 0);
draw byLabelsOnPolygon(F, G, J, L, M, K, I, H)(0, 0);
startAutoLabeling;
draw byNamedAngleSidesFull(O)();
stopAutoLabeling;
}
\drawCurrentPictureInMargin
\problemNP[2]{T}{o}{construct a parallelogram equal to a given rectilinear figure (\drawPolygon[middle][polygonABCDE]{ABC,ACD,ADE}) on an angle, equal to a given rectilinear angle (\drawAngle{O}).}

\startCenterAlign
Draw \drawUnitLine{AD} and \drawUnitLine{AC} dividing the rectilinear figure into triangles.

Construct $\drawPolygon{FGIH} = \drawPolygon{ADE}$\\
having $\drawAngle{I} = \drawAngle{O}$ \inprop[prop:I.XLII]

to \drawUnitLine{GI} apply $\drawPolygon{GJKI} = \drawPolygon{ACD}$\\
having $\drawAngle{K} = \drawAngle{O}$ \inprop[prop:I.XLIV]

to \drawUnitLine{JK} apply $\drawPolygon{JLMK} = \drawPolygon{ABC}$\\
having $\drawAngle{M} = \drawAngle{O}$ \inprop[prop:I.XLIV]

$\therefore \drawPolygon[middle][polygonFLMH]{FGIH,GJKI,JLMK} = \polygonABCDE$

and \polygonFLMH is a parallelogram. (\inpropL[prop:I.XXIX], \inpropL[prop:I.XIV], \inpropL[prop:I.XXX])\\
having $\drawAngle{M} = \drawAngle{O}$.
\stopCenterAlign

\qed
\stopProposition

\startProposition[title={Prop XLVI. Prob.}, reference=prop:I.XLVI]
\defineNewPicture{
pair A, B, C, D;
numeric d;
d := 7/2u;
A := (0, 0);
B := A shifted (d, 0);
C := A shifted (0, -d);
D := A shifted (d, -d);
draw byAngleWithName(B, A, C, black, 0)(A);
draw byAngleWithName(D, B, A, byblue, 0)(B);
draw byAngleWithName(C, D, B, byred, 0)(D);
draw byAngleWithName(A, C, D, byyellow, 0)(C);
byLineDefine(A, B, byred, 0, 0);
byLineDefine(B, D, byyellow, 0, 0);
byLineDefine(D, C, black, 0, 0);
byLineDefine(C, A, byblue, 0, 0);
draw byNamedLineSeq(0)(AB,BD,DC,CA);
draw byLabelsOnPolygon(A, B, D, C)(0, 0);
}
\drawCurrentPictureInMargin
\problemNP{U}{pon}{a given straight line (\drawUnitLine{DC}) to construct a square.}

\startCenterAlign
Draw $\drawUnitLine{CA} \perp \mbox{ and } = \drawUnitLine{DC}$ (\inpropL[prop:I.XI], \inpropL[prop:I.III])

Draw $\drawUnitLine{AB} \parallel \drawUnitLine{DC}$, and meeting \drawUnitLine{BD} drawn $\parallel \drawUnitLine{CA}$.

In
\drawFromCurrentPicture[bottom][polygonABDC]{
startTempAngleScale(angleScale*3/4);
draw byNamedAngle(A,B,C,D);
draw byNamedLineSeq(0)(AB,BD,DC,CA);
draw byLabelsOnPolygon(A, B, D, C)(0, 0);
stopTempAngleScale;
}
$\drawUnitLine{CA} = \drawUnitLine{DC}$ (const.)\\
$\drawAngle{C} = \mbox{right angle}$ (const.)

$\therefore \drawAngle{D} = \drawAngle{C} = \mbox{a right angle}$ \inprop[prop:I.XXIX], and the remaining sides and angles must be equal, \inprop[prop:I.XXXIV]

And $\therefore \polygonABDC$ is a square. \indef[def:XXVII]
\stopCenterAlign

\qed
\stopProposition

\startProposition[title={Prop XLVII. Theor.}, reference=prop:I.XLVII]
\defineNewPicture[1/2]{
pair A, B, C, D, E, F, G, H, I, J, K, L, M, d[];
%byPointLabelDefine(A, "α");
%byPointLabelDefine(B, "β");
%byPointLabelDefine(C, "γ");
%byPointLabelDefine(D, "δ");
%byPointLabelDefine(E, "ε");
%byPointLabelDefine(F, "ζ");
%byPointLabelDefine(G, "η");
%byPointLabelDefine(H, "θ");
%byPointLabelDefine(I, "ι");
%byPointLabelDefine(J, "κ");
%byPointLabelDefine(K, "λ");
%byPointLabelDefine(L, "μ");
%byPointLabelDefine(M, "ν");
A := (0, 0);
B := A shifted (-7/10u, -8/7u);
C = whatever[A, A shifted ((A-B) rotated 90)] = whatever[B, B shifted dir(0)];
d1 := (B-A) rotated -90;
D := A shifted d1;
E := B shifted d1;
d2 := (A-C) rotated -90;
F := C shifted d2;
G := A shifted d2;
d3 := (C-B) rotated -90;
H := B shifted d3;
I := C shifted d3;
J = whatever[A, A shifted dir(90)];
J = whatever[B, C];
K = whatever[A, A shifted dir(90)];
K = whatever[H, I];
L = whatever[B, F];
L = whatever[A, C];
M = whatever[A, I];
M = whatever[B, C];
draw byPolygon(A,B,E,D)(black);
draw byPolygon(L,A,G,F)(byred);
draw byPolygon(C,L,F)(byred);
draw byPolygon(J,M,I,K)(byblue);
draw byPolygon (M,C,I)(byblue);
draw byPolygon(B,J,K,H)(byyellow);
draw byAngle(F, C, A, byyellow, 0);
draw byAngle(B, C, I, byyellow, 0);
draw byAngle(A, C, B, black, 0);
draw byLine(A, K, black, 1, 0);
draw byLineFull(B, F, black, 0, 0)(G, G, 1, 1, -1);
draw byLineFull(A, I, black, 0, 0)(K, K, 1, 1, 1);
byLineDefine(C, F, byblue, 1, 0);
byLineDefine(C, I, byred, 1, 0);
draw byNamedLineSeq(0)(CF,CI);
byLineDefine(A, B, byyellow, 0, 0);
byLineDefine(B, C, byred, 0, 0);
byLineDefine(C, A, byblue, 0, 0);
draw byNamedLineSeq(-1)(AB,BC,CA);
byLineDefineWithName (C, A, black, 0, 0)(CAb);
byLineStylize (M, M, 1, 0, -1) (CAb);
byLineDefineWithName (A, M, black, 0, 0)(AMb);
byLineStylize (C, C, 0, 1, -1) (AMb);
byLineDefineWithName (B, C, black, 0, 0)(BCb);
byLineStylize (L, L, 0, 1, -1) (BCb);
byLineDefineWithName (L, B, black, 0, 0)(BLb);
byLineStylize (C, C, 1, 0, -1) (BLb);
draw byLabelsOnPolygon(B, E, D, A, G, F, C, I, K, H)(0, 1);
draw byLabelsOnPolygon(A, J, C)(2, -1);
}
\drawCurrentPictureInMargin
\problemNP{I}{n}{a right angled triangle \drawLine[bottom][triangleABC]{CA,BC,AB} the square on the hypotenuse \drawUnitLine{BC} is equal to the sum of the squares of the sides (\drawUnitLine{CA} and \drawUnitLine{AB}).}

\startCenterAlign
On \drawUnitLine{BC}, \drawUnitLine{CA}, \drawUnitLine{AB} describe squares, \inprop[prop:I.XLVI]

Draw $\drawUnitLine{AK} \parallel \drawUnitLine{CI}$ \inprop[prop:I.XXXI]\\
also draw \drawUnitLine{BF} and \drawUnitLine{AI}.\\
$\drawAngle{BCI} = \drawAngle{FCA}$,

To each add \drawAngle{ACB} $\therefore \drawAngle{BCI,ACB} = \drawAngle{FCA,ACB}$,\\
$\drawUnitLine{BC} = \drawUnitLine{CI}$ and $\drawUnitLine{CA} = \drawUnitLine{CF}$;

$\therefore
\drawFromCurrentPicture[middle][polygonAFC]{
draw byNamedPolygon(MCI);
draw byNamedAngle(ACB);
draw byNamedLine(CAb,AMb);
draw byLabelsOnPolygon(I,A,C)(1, 1);
}
=
\drawFromCurrentPicture[middle][polygonBLC]{
draw byNamedPolygon(CLF);
draw byNamedAngle(ACB);
draw byNamedLine(BCb,BLb);
draw byLabelsOnPolygon(B,F,C)(1, 1);
}
$.

Again, because $\drawUnitLine{AB} \parallel \drawUnitLine{CF}$\\
$\drawPolygon[middle][polygonACFG]{LAGF,CLF} = \mbox{ twice } \polygonBLC$,\\
and $\drawPolygon[middle][polygonJMCK]{JMIK,MCI} = \mbox{ twice } \polygonAFC$;

$\therefore \polygonACFG = \polygonJMCK$.

In the same manner it can be shown that $\drawPolygon[middle][polygonABED]{ABED} = \drawPolygon[middle][polygonBJKH]{BJKH}$;

hence $\drawPolygon[middle][polygonABEDpACFG]{ABED,LAGF,CLF} = \drawPolygon[middle][polygonBCIH]{JMIK,MCI,BJKH}$.

\stopCenterAlign

\qed
\stopProposition

\startProposition[title={Prop XLVIII. Theor.}, reference=prop:I.XLVIII]
\defineNewPicture{
pair A, B, C, D;
numeric d;
d := 7/4u;
A := (0, 0);
B := A shifted (0, 3u);
C := A shifted (d, 0);
D := A shifted (-d, 0);
draw byAngle(B, A, C, byred, 0);
draw byAngle(D, A, B, byyellow, 0);
draw byLine(A, B, byblue, 0, 0);
byLineDefine(A, C, black, 0, 0);
byLineDefine(A, D, black, 1, 0);
byLineDefine(B, C, byred, 0, 0);
byLineDefine(B, D, byred, 1, 0);
draw byNamedLineSeq(0)(AC,AD,BD,BC);
draw byLabelsOnPolygon(D, B, C, A)(0, 0);
}
\drawCurrentPictureInMargin
\problemNP{I}{f}{the square of one side (\drawUnitLine{BC}) of a triangle is equal to the squares of the other two sides (\drawUnitLine{AB} and \drawUnitLine{AC}), the angle (\drawAngle{DAB}) subtended by that side is a right angle.}

\startCenterAlign
Draw $\drawUnitLine{AD} \perp \drawUnitLine{AB}$ and $= \drawUnitLine{AC}$ (\inpropL[prop:I.XI], \inpropL[prop:I.III])\\
and draw \drawUnitLine{BD} also.

Since $\drawUnitLine{AD} = \drawUnitLine{AC}$ (const.)\\
$\drawUnitLine{AD}^2 = \drawUnitLine{AC}^2$;

$\therefore \drawUnitLine{AD}^2 + \drawUnitLine{AB}^2 = \drawUnitLine{AC}^2 + \drawUnitLine{AB}^2$

but $\drawUnitLine{AD}^2 + \drawUnitLine{AB}^2 = \drawUnitLine{BD}^2$ \inprop[prop:I.XLVII],\\
and $\drawUnitLine{AC}^2 + \drawUnitLine{AB}^2 = \drawUnitLine{BC}^2$ (hyp.)

$\therefore \drawUnitLine{BD}^2 = \drawUnitLine{BC}^2$,

$\therefore \drawUnitLine{BD} = \drawUnitLine{BC}$;

and $\therefore \drawAngle{DAB} = \drawAngle{BAC}$ \inprop[prop:I.VIII],

consequently \drawAngle{BAC} is a right angle.
\stopCenterAlign

\qed

\stopProposition
\stopbook

\startbook[title={Book 2}]
\startDefinition[title={Definition I},reference=def:II.I]

\defineNewPicture{
pair A, B, C, D;
numeric w, h;
w := 7/2u;
h := 3u;
A := (0, 0);
B := (w, 0);
C := (0, h);
D := (w, h);
draw byPolygon(A,B,D,C)(byblue);
byLineDefine(A, B, black, 0, 0);
byLineDefine(A, C, byred, 0, 0);
draw byNamedLineSeq(0)(AB,AC);
draw byLabelsOnPolygon(A, C, D, B)(0, 0);
}
\drawCurrentPictureInMargin
\problemNP{A}{rectangle}{or a right angled parallelogram is said to be contained by any two of its adjacent or conterminous sides.}

Thus: the right angled parallelogram \drawPolygon{ABDC} is said to be contained by the sides \drawUnitLine{AB} and \drawUnitLine{AC}; or it may be briefly designated by $\drawUnitLine{AB} \cdot \drawUnitLine{AC}$.

If the adjacent sides are equal; i. e. $\drawUnitLine{AB} = \drawUnitLine{AC}$, then $\drawUnitLine{AB} \cdot \drawUnitLine{AC}$ which is the expression for the rectangle under \drawUnitLine{AB} and \drawUnitLine{AC} is a square, and

is equal to $\left\{\eqalign{
\drawUnitLine{AB} \cdot \drawUnitLine{AC}&\mbox{ or } \drawUnitLine{AC}^2\cr
\drawUnitLine{AB} \cdot \drawUnitLine{AC}&\mbox{ or } \drawUnitLine{AC}^2}\right.$

\stopDefinition

\vfill\pagebreak

\startDefinition[title={Definition II},reference=def:II.II]

\defineNewPicture{
pair A, B, C, D, E, F, G, H, I, d[];
d1 := (3u, 0);
d2 := (1/2u, 3u);
A := (0, 0);
B := A shifted d1;
C := A shifted d2;
D := A shifted d1 shifted d2;
E := 2/3[A, B];
F := 2/3[C, D];
G := 2/3[A, C];
H := 2/3[B, D];
I = whatever[E, F] = whatever[G, H];
draw byPolygon(A,E,I,G)(byblue);
draw byPolygon(E,B,H,I)(byyellow);
draw byPolygon(G,I,F,C)(byyellow);
draw byPolygon(I,H,D,F)(byred);
draw byLabelsOnPolygon(C, F, D, H, B, E, A, G)(0,0);
draw byLabelsOnPolygon(G, I, F)(2,0);
}
\drawCurrentPictureInMargin
\problemNP[4]{I}{n}{a parallelogram, the figure composed of one of the parallelograms about the diagonal, together with the two complements, is called a Gnomon.}

Thus \drawPolygon{AEIG,EBHI,GIFC} and \drawPolygon{EBHI,GIFC,IHDF} are called Gnomons.
\stopDefinition

\vfill\pagebreak

\startProposition[title={Prop. I. theor.},reference=prop:II.I]
\defineNewPicture{
pair B, C, D, E, G, H, K, L, M, N;
numeric w, h;
w := 7/2u;
h := 3u;
G := (0, 0);
H := (w, 0);
B := (0, h);
C := (w, h);
K := 2/5[G, H];
D := 2/5[B, C];
L := 3/4[G, H];
E := 3/4[B, C];
M := G shifted (0, -2/3u);
N := M shifted (h, 0);
draw byPolygon(G,K,D,B)(byyellow);
draw byPolygon(K,L,E,D)(byblue);
draw byPolygon(L,H,C,E)(byred);
draw byLine(K, D, black, 1, 0);
draw byLine(L, E, black, 1, 0);
byLineDefine(G, B, black, 0, 0);
draw byLineWithName(M, N, black, 0, 0)(A); % improvement: the line wasn't there, but it should've been
byLineDefine(H, C, black, 1, 0);
byLineDefine(G, K, byblue, 0, 0);
byLineDefine(K, L, byred, 0, 0);
byLineDefine(L, H, byyellow, 0, 0);
byLineDefine(B, D, byblue, 1, 0);
byLineDefine(D, E, byred, 1, 0);
byLineDefine(E, C, byyellow, 1, 0);
draw byNamedLineSeq(0)(GK,KL,LH,HC,EC,DE,BD,GB);
draw byLabelsOnPolygon(B, D, E, C, H, L, K, G)(0, 0);
draw byLabelLine(A);
}
\drawCurrentPictureInMargin
\problemNP[2]{T}{he}{rectangle contained by two straight lines, one of which is divided into any number of parts,
$\drawProportionalLine{GK,KL,LH} \cdot \drawProportionalLine{A} = \left\{\eqalign{
&\drawProportionalLine{A} \cdot \drawProportionalLine{GK}\cr
+&\drawProportionalLine{A} \cdot \drawProportionalLine{KL}\cr
+&\drawProportionalLine{A} \cdot \drawProportionalLine{LH}}\right.$\\
is equal to the sum of the rectangles contained by the undivided line, and several parts of the divided line.}

\startCenterAlign
Draw $\drawProportionalLine{GB} \perp \drawProportionalLine{GK,KL,LH} \mbox{ and} = \drawProportionalLine{GB}$ (\inpropL[prop:I.II], \inpropL[prop:I.III]); complete the parallelograms, that is to say,

Draw $\left\{\eqalign{
\drawProportionalLine{BD,DE,EC} &\parallel \drawProportionalLine{GK,KL,LH} \cr
\vcenter{
\nointerlineskip\hbox{\drawProportionalLine{KD}}
\nointerlineskip\hbox{\drawProportionalLine{LE}}
\nointerlineskip\hbox{\drawProportionalLine{HC}}} &\parallel \drawProportionalLine{GB}
}\right\}$ \inprop[prop:I.XXXI]

$\drawPolygon[bottom]{GKDB,KLED,LHCE} =
\drawPolygon[bottom]{GKDB} +
\drawPolygon[bottom]{KLED} +
\drawPolygon[bottom]{LHCE}$

$\drawPolygon[bottom]{GKDB,KLED,LHCE} = \drawProportionalLine{GK,KL,LH} \cdot \drawProportionalLine{GB}$

$\polygonGKDB = \drawProportionalLine{GK} \cdot \drawProportionalLine{GB}$,
$\polygonKLED = \drawProportionalLine{KL} \cdot \drawProportionalLine{GB}$,

$\polygonLHCE = \drawProportionalLine{LH} \cdot \drawProportionalLine{GB}$

$\therefore \drawProportionalLine{GK,KL,LH} \cdot \drawProportionalLine{A} = \drawProportionalLine{GK} \cdot \drawProportionalLine{A} + \drawProportionalLine{KL} \cdot \drawProportionalLine{A} + \drawProportionalLine{LH} \cdot \drawProportionalLine{A}$.
\stopCenterAlign

\qed
\stopProposition

\startProposition[title={Prop. II. theor.},reference=prop:II.II]
\defineNewPicture{
pair A, B, C, D, E, F;
numeric w;
w := 7/2u;
A := (0, w);
B := (w, w);
C := 2/3[A, B];
D := (0, 0);
E := (w, 0);
F := 2/3[D, E];
draw byPolygon(A,C,F,D)(byred);
draw byPolygon(C,B,E,F)(byyellow);
draw byLine(C, F, black, 0, 0);
byLineDefine(A, D, black, 1, 0);
byLineDefine(B, E, black, 1, 0);
byLineDefine(A, C, byblue, 0, 0);
byLineDefine(C, B, byred, 0, 0);
draw byNamedLineSeq(0)(AD,AC,CB,BE);
draw byLabelsOnPolygon(A, C, B, E, F, D)(0, 0);
}
\drawCurrentPictureInMargin
\problemNP{I}{f}{a straight line be divided into any two parts \drawProportionalLine{AC,CB}, the square of the whole line is equal to the sum of the rectangles contained by the whole line and each of its parts.\\
$\drawProportionalLine{AC,CB}^2 = \left\{\eqalign{
& \drawProportionalLine{AC,CB} \cdot \drawProportionalLine{AC} \cr
+ & \drawProportionalLine{AC,CB} \cdot \drawProportionalLine{CB}
}\right.$
}

\startCenterAlign
Describe \drawPolygon[bottom][polygonABED]{ACFD,CBEF} \inprop[prop:I.XLVI]

Draw \drawProportionalLine{CF} parallel to \drawProportionalLine{AD} \inprop[prop:I.XXXI]

$\polygonABED = \drawProportionalLine{AC,CB}^2$

$\drawPolygon[bottom]{ACFD} = \drawProportionalLine{CF} \cdot \drawProportionalLine{AC} = \drawProportionalLine{AC,CB} \cdot \drawProportionalLine{AC}$

$\drawPolygon[bottom]{CBEF} = \drawProportionalLine{CF} \cdot \drawProportionalLine{CB} = \drawProportionalLine{AC,CB} \cdot \drawProportionalLine{CB}$

$\polygonABED = \drawPolygon[bottom]{ACFD} + \drawPolygon[bottom]{CBEF}$

$\therefore \drawProportionalLine{AC,CB}^2 = \drawProportionalLine{AC,CB} \cdot \drawProportionalLine{AC} + \drawProportionalLine{AC,CB} \cdot \drawProportionalLine{CB}$.
\stopCenterAlign

\qed
\stopProposition

\startProposition[title={Prop. III. theor.},reference=prop:II.III]
\defineNewPicture{
pair A, B, C, D, E, F;
numeric w, h;
w := -4u;
h := 11/4u;
A := (0, h);
B := (w, h);
C := (w+h, h);
D := (w+h, 0);
E := (w, 0);
F := (0, 0);
draw byPolygon(A,C,D,F)(byyellow);
draw byPolygon(C,B,E,D)(byred);
byLineDefine(D, F, byred, 0, 0);
byLineDefine(B, C, byblue, 0, 0);
byLineDefine(C, D, byblue, 0, 0);
byLineDefine(D, E, byblue, 0, 0);
byLineDefine(E, B, byblue, 0, 0);
draw byNamedLineSeq(0)(CD,noLine,DF,DE,EB,BC);
draw byLabelsOnPolygon(B, C, A, F, D, E)(0, 0);
}
\drawCurrentPictureInMargin
\problemNP{I}{f}{a straight line be divided into any two parts \drawProportionalLine{DE,DF}, the rectangle contained by the whole line and either of its parts, is equal to the square of that part, together with the rectangle under the parts.\\
$\drawProportionalLine{DE,DF} \cdot \drawProportionalLine{DE} = \drawProportionalLine{DE}^2 + \drawProportionalLine{DE} \cdot \drawProportionalLine{DF}$, or \\
$\drawProportionalLine{DE,DF} \cdot \drawProportionalLine{DF} = \drawProportionalLine{DF}^2 + \drawProportionalLine{DE} \cdot \drawProportionalLine{DF}$.}

\startCenterAlign
Describe \drawPolygon[bottom]{CBED} \inprop[prop:I.XLVI]

Describe \drawPolygon[bottom]{ACDF} \inprop[prop:I.XXXI]

Then $\drawPolygon[bottom][polygonABEF]{ACDF,CBED} = \polygonCBED + \polygonACDF$, but\\
$\polygonABEF = \drawProportionalLine{DE,DF} \cdot \drawProportionalLine{DE}$ and\\
$\polygonCBED = \drawProportionalLine{DE}^2$, $\polygonACDF = \drawProportionalLine{DE} \cdot \drawProportionalLine{DF}$,

$\therefore \drawProportionalLine{DE,DF} \cdot \drawProportionalLine{DE} = \drawProportionalLine{DE}^2 + \drawProportionalLine{DE} \cdot \drawProportionalLine{DF}$:

In a similar manner it may be readily shown that $\drawProportionalLine{DE,DF} \cdot \drawProportionalLine{DF} = \drawProportionalLine{DF}^2 + \drawProportionalLine{DE} \cdot \drawProportionalLine{DF}$.
\stopCenterAlign

\qed
\stopProposition

\startProposition[title={Prop. IV. theor.},reference=prop:II.IV]
\defineNewPicture[1/4]{
pair A, B, C, D, E, F, G, H, K;
numeric w;
w := 9/2u;
A := (0, w);
B := (w, w);
C :=2/3[A, B];
D := (0, 0);
E := (w, 0);
F := 2/3[D, E];
H := 2/3[D, A];
K := 2/3[E, B];
G = whatever[H, K] = whatever[F, C];
draw byPolygon(A,C,G,H)(byyellow);
draw byPolygon(G,K,E,F)(byyellow);
draw byPolygon(D,H,G)(byblue);
draw byPolygon(G,B,C)(byred);
draw byAngle(F, D, G, byyellow, 0);
draw byAngle(F, G, D, byred, 0);
draw byAngle(K, G, B, black, 0);
draw byAngle(G, B, K, byblue, 0);
draw byLine(H, G, byred, 1, 0);
draw byLine(G, K, byred, 0, 0);
draw byLine(C, G, byblue, 1, 0);
draw byLine(F, G, byblue, 0, 0);
byLineDefine(E, K, byblue, 0, 0);
byLineDefine(F, E, byred, 0, 0);
byLineDefine(D, F, byblue, 0, 0);
byLineDefine(K, B, byred, 0, 0);
byLineDefine(G, D, black, 0, 0);
byLineDefine(B, G, black, 1, 0);
draw byNamedLineSeq(-1)(BG,GD,DF,FE,EK,KB);
byLineDefine(A, D, black, 0, 0);
byLineStylize(B, E, 0, 0, -1)(AD);
byLineDefine(B, A, black, 0, 0);
byLineStylize(E, D, 0, 0, -1)(BA);
draw byLabelsOnPolygon(A, C, B, K, E, F, D, H)(0, 0);
draw byLabelsOnPolygon(H, G, C)(2, 0);
}
\drawCurrentPictureInMargin
\problemNP{I}{f}{a straight line be divided into any two parts \drawProportionalLine{DF,FE}, the square of the whole line is equal to the squares of the parts, together with twice the rectangle contained by the parts.\\
$\drawProportionalLine{DF,FE}^2 = \drawProportionalLine{DF}^2 + \drawProportionalLine{FE}^2 + \mbox{twice} \drawProportionalLine{DF} \cdot \drawProportionalLine{FE}$
}

\startCenterAlign
Describe \drawLine[bottom][squareABED]{AD,BA,KB,EK,FE,DF} \inprop[prop:I.XLVI]

draw \drawProportionalLine{BG,GD} \inpost[post:I]\\
and $\left\{\eqalign{
\drawProportionalLine{FG,CG} &\parallel \drawProportionalLine{EK,KB} \cr
\drawProportionalLine{HG,GK} &\parallel \drawProportionalLine{DF,FE}
}\right\}$ \inprop[prop:I.XXXI]\\
$\drawAngle{GBK} = \drawAngle{FDG}$ \inprop[prop:I.V],\\
$\drawAngle{GBK} = \drawAngle{FGD}$ \inprop[prop:I.XXIX],

$\therefore \drawAngle{FDG} = \drawAngle{FGD}$

$\therefore$ by (\inpropL[prop:I.VI], \inpropL[prop:I.XXIX], \inpropL[prop:I.XXXIV])\\ $\drawFromCurrentPicture[bottom][squareFGHD]{
draw byNamedPolygon(DHG);
draw byNamedLineFull(G, G, 1, 0, -1)(DF);
draw byNamedLineFull(D, D, 0, 1, -1)(FG);
draw byLabelsOnPolygon(D, H, G, F)(0, 0);
} \mbox{ is a square } = \drawProportionalLine{DF}^2$.

For the same reasons \drawFromCurrentPicture[bottom][squareKBCG]{
draw byNamedPolygon(GBC);
draw byNamedLineFull(B, B, 1, 0, -1)(GK);
draw byNamedLineFull(G, G, 0, 1, -1)(KB);
draw byLabelsOnPolygon(G, C, B, K)(0, 0);
} is a square $= \drawProportionalLine{GK}^2$,

$\drawPolygon[bottom]{ACGH} = \drawPolygon[middle]{GKEF} = \drawProportionalLine{DF} \cdot \drawProportionalLine{GK}$ \inprop[prop:I.XLIII]

But $\drawFromCurrentPicture[bottom][squareABEDf]{
draw byNamedPolygon(GBC,DHG,ACGH,GKEF);
draw byNamedLineFull(G, G, 1, 0, -1)(DF);
draw byNamedLineFull(D, D, 0, 1, -1)(FG);
draw byNamedLineFull(B, B, 1, 0, -1)(GK);
draw byNamedLineFull(G, G, 0, 1, -1)(KB);
draw byLabelsOnPolygon(A, B, E, D)(0, 0);
} = \squareFGHD + \drawPolygon{ACGH} + \drawPolygon{GKEF} + \squareKBCG$,

$\therefore \drawProportionalLine{DF,FE}^2 = \drawProportionalLine{DF}^2 + \drawProportionalLine{FE}^2 + \mbox{ twice } \drawProportionalLine{DF} \cdot \drawProportionalLine{FE}$
\stopCenterAlign

\qed
\stopProposition

\startProposition[title={Prop. V. theor.},reference=prop:II.V]
\defineNewPicture[1/4]{
pair A, B, C, D, E, F, G, H, K, L, M;
numeric h;
h := 6u;
A := (0, -h);
B := (0, 0);
C := 1/2[A, B];
D := 2/5[B, C];
E := (1/2h, -1/2h);
F := (1/2h, 0);
G := (xpart(F), ypart(D));
H = whatever[D, G] = whatever[B, E];
K := (xpart(H), ypart(A));
L := (xpart(H), ypart(C));
M := (xpart(H), ypart(B));
draw byPolygon(B,D,H,M)(byblue);
draw byPolygon(D,C,L,H)(byyellow);
draw byPolygon(C,L,K,A)(black);
draw byPolygon(M,H,G,F)(byyellow);
draw byPolygon(H,L,E,G)(byred);
draw byLine(H, G, black, 1, 0);
draw byLine(D, H, byred, 0, 0);
byLineDefine(L, M, black, 1, 0);
byLineDefine(C, L, byred, 0, 0);
byLineDefine(B, D, byred, 0, 0);
byLineDefine(D, C, byblue, 0, 0);
byLineDefine(C, A, byyellow, 0, 0);
byLineDefine(E, B, black, 0, 0);
byLineDefine(A, K, byred, 1, 0);
byLineDefine(K, L, byyellow, 0, 0);
byLineDefine(L, E, byblue, 1, 0);
draw byNamedLineSeq(-1)(BD,DC,CA,AK,KL,LM,CL,LE,EB);
draw byLabelsOnPolygon(A, C, D, B, M, F, G, E, L, K)(0, 0);
draw byLabelsOnPolygon(M, H, G)(2, 0);
}
\drawCurrentPictureInMargin
\problemNP{I}{f}{a straight line be divided \drawProportionalLine{BD,DC} \drawProportionalLine{CA} into two equal parts and also \drawProportionalLine{BD} \drawProportionalLine{DC,CA} into two unequal parts, the rectangle contained by the unequal parts, together with the square of the line between the points of section, is equal to the square of half that line \\
$\drawProportionalLine{BD} \cdot \drawProportionalLine{DC,CA} + \drawProportionalLine{DC}^2 = \drawProportionalLine{CA}^2 = \drawProportionalLine{BD,DC}^2$.
}

\startCenterAlign
Describe \drawPolygon[bottom][squareCBFE]{BDHM,DCLH,MHGF,HLEG} \inprop[prop:I.XLVI],

draw \drawProportionalLine{EB}\\
and $\left\{\eqalign{
\drawProportionalLine{DH,HG} & \parallel \drawProportionalLine{CL,LE} \cr
\drawProportionalLine{LM,KL} & \parallel \drawProportionalLine{BD,DC,CA} \cr
\drawProportionalLine{AK} & \parallel \drawProportionalLine{CL,LE}
}\right\}$ \inprop[prop:I.XXXI]

$\drawPolygon[bottom]{CLKA} = \drawPolygon[bottom]{DCLH,BDHM}$ \inprop[prop:I.XXXVI]

$\drawPolygon[bottom]{MHGF} = \drawPolygon[bottom]{DCLH}$ \inprop[prop:I.XLIII]

$\therefore \mbox{\inax[ax:II] } \drawPolygon[bottom]{DCLH,BDHM,MHGF} = \drawFromCurrentPicture[bottom]{
startGlobalRotation(90);
startAutoLabeling;
draw byNamedPolygon(DCLH,CLKA);
stopAutoLabeling;
stopGlobalRotation;
} = \drawProportionalLine{BD} \cdot \drawProportionalLine{DC,CA}$

but $\drawPolygon[bottom]{HLEG} = \drawProportionalLine{DC}^2$ \inprop[prop:II.IV]\\
and $\squareCBFE = \drawProportionalLine{BD,DC}^2$ (const.)

$\therefore \mbox{\inax[ax:II] } \squareCBFE = \drawFromCurrentPicture[bottom]{
startGlobalRotation(90);
startAutoLabeling;
draw byNamedPolygon(DCLH,CLKA,HLEG);
stopAutoLabeling;
stopGlobalRotation;
}$

$\therefore \drawProportionalLine{BD} \cdot \drawProportionalLine{DC,CA} + \drawProportionalLine{DC}^2 = \drawProportionalLine{CA}^2 = \drawProportionalLine{BD,DC}^2$
\stopCenterAlign

\qed
\stopProposition

\startProposition[title={Prop. VI. theor.},reference=prop:II.VI]
\defineNewPicture{
pair A, B, C, D, E, F, G, H, K, L, M;
numeric h, s;
h := 5u;
s := 2/5h;
A := (0, h);
B := (0, 0);
C := A shifted (0, -s);
D := A shifted (0, -2s);
E := (-h+s, h-s);
F := (-h+s, 0);
G := (xpart(F), ypart(D));
H = whatever[D, G] = whatever[B, E];
K := (xpart(H), ypart(A));
L := (xpart(H), ypart(C));
M := (xpart(H), ypart(B));
draw byPolygon(B,D,H,M)(byblue);
draw byPolygon(D,C,L,H)(byyellow);
draw byPolygon(C,L,K,A)(black);
draw byPolygon(M,H,G,F)(byyellow);
draw byPolygon(H,L,E,G)(byred);
draw byLine(H, G, byblue, 1, 0);
draw byLine(D, H, byred, 0, 0);
byLineDefine(L, M, black, 1, 0);
byLineDefine(C, L, byred, 0, 0);
byLineDefine(B, D, byred, 0, 0);
byLineDefine(D, C, byblue, 0, 0);
byLineDefine(C, A, byyellow, 0, 0);
byLineDefine(E, B, black, 0, 0);
byLineDefine(A, K, byred, 1, 0);
byLineDefine(K, L, byyellow, 0, 0);
byLineDefine(L, E, black, 1, 0);
draw byNamedLineSeq(-1)(BD,DC,CA,AK,KL,LM,CL,LE,EB);
draw byLabelsOnPolygon(F, G, E, L, K, A, C, D, B, M)(0, 0);
draw byLabelsOnPolygon(L, H, D)(2, -1);
}
\drawCurrentPictureInMargin
\problemNP{I}{f}{a straight line be bisected \drawProportionalLine{DC,CA} and produced to any point \drawProportionalLine{BD,DC,CA}, the rectangle contained by the whole line so increased, and the part produced, together with the square of half the line, is equal to the square of the line made up of the half, and the produced part.\\
$\drawProportionalLine{BD,DC,CA} \cdot \drawProportionalLine{BD} + \drawProportionalLine{DC}^2 = \drawProportionalLine{BD,DC}^2$
}

\startCenterAlign
Describe \drawPolygon[bottom][squareCBFE]{BDHM,DCLH,MHGF,HLEG} \inprop[prop:I.XLVI], draw \drawProportionalLine{EB}\\
and $\left\{\eqalign{
\drawProportionalLine{HG,DH} & \parallel \drawProportionalLine{CL,LE} \cr
\drawProportionalLine{LM,KL} & \parallel \drawProportionalLine{BD,DC,CA} \cr
\drawProportionalLine{AK} & \parallel \drawProportionalLine{CL,LE}
}\right\}$ \inprop[prop:I.XXXI]

$\drawPolygon[bottom]{MHGF} = \drawPolygon[bottom]{DCLH} = \drawPolygon[bottom]{CLKA}$ (\inpropL[prop:I.XXXVI], \inpropL[prop:I.XLIII])

$\therefore \drawPolygon[bottom]{BDHM,DCLH,MHGF} = \drawPolygon[bottom]{BDHM,DCLH,CLKA} = \drawProportionalLine{BD} \cdot \drawProportionalLine{BD,DC,CA}$;\\
but $\drawPolygon[bottom]{HLEG} = \drawProportionalLine{DC}^2$ \inprop[prop:II.IV]

$\therefore \squareCBFE = \drawProportionalLine{HG,DH}^2 = \drawPolygon[bottom]{BDHM,DCLH,CLKA,HLEG}$ (const., \inaxL[ax:II])

$\therefore \drawProportionalLine{BD,DC,CA} \cdot \drawProportionalLine{BD} + \drawProportionalLine{DC}^2 = \drawProportionalLine{BD,DC}^2$
\stopCenterAlign

\qed
\stopProposition

\startProposition[title={Prop. VII. theor.},reference=prop:II.VII]
\defineNewPicture{
pair A, B, C, D, E, F, G, H, N;
numeric w;
w := 7/2u;
A := (0, w);
B := (w, w);
C := 3/5[A, B];
D := (0, 0);
E := (w, 0);
N := 3/5[D, E];
F := 3/5[E, B];
G = whatever[D, B] = whatever[N, C];
H := whatever[A, D] = whatever[F, G];
draw byPolygon(D,N,G,H)(byred);
draw byPolygon(N,E,F,G)(black);
draw byPolygon(H,G,C,A)(byyellow);
draw byPolygon(G,F,B,C)(byblue);
draw byLine(G, N, byblue, 0, 0);
draw byLine(G, F, byred, 0, 0);
draw byLine(G, H, black, 1, 0);
draw byLine(G, C, black, 1, 0);
byLineDefine(B, D, black, 0, 0);
byLineDefine(D, N, byblue, 0, 0);
byLineDefine(N, E, byred, 0, 0);
byLineDefine(E, B, byyellow, 0, 0);
draw byNamedLineSeq(-1)(BD,DN,NE,EB);
draw byLabelsOnPolygon(D, H, A, C, B, F, E, N)(0, 0);
draw byLabelsOnPolygon(H, G, C)(2, 0);
}
\drawCurrentPictureInMargin
\problemNP{I}{f}{a straight line be divided into any two parts \drawProportionalLine{DN,NE}, the squares of the whole line and one of the parts are equal to twice the rectangle contained by the whole line and that part, together with the square of the other parts.\\
$\drawProportionalLine{DN,NE}^2 + \drawProportionalLine{NE}^2 = 2\drawProportionalLine{DN,NE} \cdot \drawProportionalLine{NE} + \drawProportionalLine{DN}^2$
}

\startCenterAlign
Describe \drawPolygon[bottom][squareABED]{DNGH,NEFG,HGCA,GFBC} \inprop[prop:I.XLVI], draw \drawProportionalLine{BD} \inpost[post:I],\\
and $\left\{\eqalign{
\drawProportionalLine{GN,GC} &\parallel \drawProportionalLine{EB} \cr
\drawProportionalLine{GH,GF} &\parallel \drawProportionalLine{DN,NE}
}\right\}$\\
$\drawPolygon[bottom]{HGCA} = \drawPolygon[bottom]{NEFG}$ \inprop[prop:I.XLIII],\\
add $\drawPolygon[bottom]{GFBC} = \drawProportionalLine{NE}^2$ to both \inprop[prop:II.IV]

$\drawPolygon[bottom]{HGCA,GFBC} = \drawPolygon[bottom]{NEFG,GFBC} = \drawProportionalLine{DN,NE} \cdot \drawProportionalLine{NE}$\\
$\drawPolygon[bottom]{DNGH} = \drawProportionalLine{DN}^2$ \inprop[prop:II.IV]\\
$\drawPolygon[bottom]{HGCA,GFBC} + \drawPolygon[bottom]{NEFG,GFBC} + \drawPolygon[bottom]{DNGH} = 2\drawProportionalLine{DN,NE} \cdot \drawProportionalLine{NE} + \drawProportionalLine{DN}^2 = \squareABED + \drawPolygon[bottom]{GFBC}$;

$\drawProportionalLine{DN,NE}^2 + \drawProportionalLine{NE}^2 = 2\drawProportionalLine{DN,NE} \cdot \drawProportionalLine{NE} + \drawProportionalLine{DN}^2$
\stopCenterAlign

\qed
\stopProposition

\startProposition[title={Prop. VIII. theor.},reference=prop:II.VIII]
\defineNewPicture{
pair A, B, C, D, E, F, G, H, K, L, M, N, O, P, Q, R;
numeric w, d;
w := 7/2u;
d := u;
A := (0, w + d);
B := (w, w + d);
C := (w - d, w + d);
D := (w + d, w + d);
E := (0, 0);
F := (w + d, 0);
G := (w - d, w);
H := (w - d, 0);
K := (w, w);
L := (w, 0);
M := (0, w);
N := (w + d, w);
O := (0, w - d);
P := (w + d, w - d);
Q := (w - d, w - d);
R := (w, w - d);
draw byLine(C, Q, byblue, 1, 0);
draw byLine(B, R, black, 1, 0);
draw byLine(Q, H, byblue, 0, 0);
draw byLine(R, L, byblue, 0, 0);
draw byLine(M, G, black, 1, 0);
draw byLine(O, Q, byred, 1, 0);
draw byLine(G, N, byred, 0, 0);
draw byLine(Q, P, byred, 0, 0);
draw byLine(D, E, black, 0, 0);
byLineDefine(E, H, byblue, 0, 0);
byLineDefine(H, L, byred, 0, 0);
byLineDefine(L, F, byyellow, 0, 0);
byLineDefine(F, P, byblue, 0, 0);
byLineDefine(P, N, byyellow, 0, 0);
byLineDefine(N, D, byred, 0, 0);
byLineDefine(A, D, black, 0, 0);
byLineDefine(E, A, black, 0, 0);
draw byNamedLineSeq(0)(EH,HL,LF,FP,PN,ND,AD,EA);
byLineDefine(D, F, black, 0, 0);
byLineDefine(B, L, black, 0, 1);
byLineDefine(C, H, black, 0, 1);
byLineDefine(M, N, black, 0, 1);
byLineDefine(O, P, black, 0, 1);
draw byLabelsOnPolygon(A, C, B, D, N, P, F, L, H, E, O, M)(0, 0);
}
\drawCurrentPicture
\problemNP{I}{f}{a straight line be divided into any two parts \drawProportionalLine{EH,HL}, the square of the sum of the whole line and any of its parts is equal to four times the rectangle contained by the whole line, and that part together with the square of the other part. \\
$\drawProportionalLine{EH,HL,LF}^2 = 4 \cdot \drawProportionalLine{EH,HL} \cdot \drawProportionalLine{HL} + \drawProportionalLine{EH}^2$
}

\startCenterAlign
Produce \drawProportionalLine{EH,HL} and make $\drawProportionalLine{LF} = \drawProportionalLine{HL}$

Construct \drawFromCurrentPicture[bottom]{
draw byNamedLine(BL,CH,MN,OP);
draw byNamedLineSeq(0)(LF,HL,EH,EA,AD,DF);
} \inprop[prop:I.XLVI];\\
draw \drawProportionalLine{DE},

$\left.\eqalign{
\left.\eqalign{
\vcenter{
\nointerlineskip\hbox{\drawProportionalLine{CQ,QH}}
\nointerlineskip\hbox{\drawProportionalLine{BR,RL}}
}
}\right\} & \parallel \drawProportionalLine{FP,PN,ND}\cr
\left.\eqalign{
\vcenter{
\nointerlineskip\hbox{\drawProportionalLine{OQ,QP}}
\nointerlineskip\hbox{\drawProportionalLine{MG,GN}}
}
}\right\} & \parallel \drawProportionalLine{EH,HL,LF}\cr
}\right\}$ \inprop[prop:I.XXXI]

$\drawProportionalLine{EH,HL,LF}^2 = \drawProportionalLine{LF}^2 + \drawProportionalLine{EH,HL}^2 + 2 \cdot \drawProportionalLine{EH,HL} \cdot \drawProportionalLine{LF}$ \inprop[prop:II.IV]\\
but $\drawProportionalLine{HL}^2 + \drawProportionalLine{EH,HL}^2 = 2 \cdot \drawProportionalLine{EH,HL} \cdot \drawProportionalLine{HL} + \drawProportionalLine{EH}^2$ \inprop[prop:II.VII]

$\therefore \drawProportionalLine{EH,HL,LF}^2 = 4 \cdot \drawProportionalLine{EH,HL} \cdot \drawProportionalLine{HL} + \drawProportionalLine{EH}^2$
\stopCenterAlign

\qed
\stopProposition

\startProposition[title={Prop. IX. theor.},reference=prop:II.IX]
\defineNewPicture{
pair A, B, C, D, E, F, G, H;
numeric w;
w := 5u;
A := (-1/2w, 0);
B := (1/2w, 0);
C := (0, 0);
D := (1/5w, 0);
E := (0, 1/2w);
F = whatever[E, B] = (xpart(D), whatever);
G = whatever[E, C] = (whatever, ypart(F));
H := 1/2[E, F];
draw byAngleWithName(E, A, B, byyellow, 0)(A);
draw byAngleWithName(A, B, E, byblue, 0)(B);
draw byAngle(A, E, C, byyellow, 0);
draw byAngle(C, E, B, byred, 0);
draw byAngle(E, F, G, byred, 0);
draw byAngle(D, F, B, black, 0);
draw byLine(A, F, black, 0, 0);
draw byLine(F, D, byred, 1, 0);
draw byLine(G, F, byyellow, 0, -1);
draw byLine(C, G, byblue, 0, 0);
draw byLine(G, E, byblue, 1, 0);
byLineDefine(A, C, byblue, 0, 0);
byLineDefine(C, D, byyellow, 0, 0);
byLineDefine(D, B, byred, 0, 0);
byLineDefine(F, B, byyellow, 1, 0);
byLineDefine(H, F, black, 0, 0);
byLineDefine(E, H, black, 1, 0);
byLineDefine(A, E, black, 1, 0);
draw byNamedLineSeq(0)(AC,CD,DB,FB,HF,EH,AE);
draw byLabelsOnPolygon(E, F, B, D, C, A)(0, 0);
draw byLabelsOnPolygon(C, G, E)(2, 0);
}
\drawCurrentPictureInMargin
\problemNP{I}{f}{a straight line be divided into two equal parts \drawProportionalLine{AC} \drawProportionalLine{CD,DB} and also into two unequal parts \drawProportionalLine{AC,CD} \drawProportionalLine{DB}, the squares of the unequal parts are together double the squares of half the line, and of the part between the points of section.\\
$\drawProportionalLine{AC,CD}^2 + \drawProportionalLine{DB}^2 = 2 \cdot \drawProportionalLine{AC}^2 + 2 \cdot \drawProportionalLine{CD}^2$
}

\startCenterAlign
Make $\drawProportionalLine{CG,GE} \perp \mbox{ and } = \drawProportionalLine{AC} \mbox{ or } \drawProportionalLine{CD,DB}$,\\
Draw \drawProportionalLine{AE} and \drawProportionalLine{EH,HF,FB},\\
$\drawProportionalLine{FD} \parallel \drawProportionalLine{CG,GE}$, $ \drawProportionalLine{GF} \parallel \drawProportionalLine{CD,DB}$ and draw \drawProportionalLine{AF}.

$\drawAngle{A} = \drawAngle{AEC}$ \inprop[prop:I.V] $=$ half a right angle \inprop[prop:I.XXXII]\\
$\drawAngle{B} = \drawAngle{DFB}$ \inprop[prop:I.V] $=$ half a right angle \inprop[prop:I.XXXII]\\
$\therefore \drawAngle{AEC,CEB} =$ a right angle.

$\drawAngle{B} = \drawAngle{CEB} = \drawAngle{EFG} = \drawAngle{DFB}$ (\inpropL[prop:I.V], \inpropL[prop:I.XXIX]).\\
hence $\drawProportionalLine{FD} = \drawProportionalLine{DB}$, $\drawProportionalLine{GE} = \drawProportionalLine{GF} = \drawProportionalLine{CD}$ (\inpropL[prop:I.VI], \inpropL[prop:I.XXXIV])

$\drawProportionalLine{AF}^2 = \left\{\eqalign{
&\drawProportionalLine{AC,CD}^2 + \drawProportionalLine{FD}^2 \mbox{, or } + \drawProportionalLine{DB}^2 \cr
&= \left\{\eqalign{
&= \drawProportionalLine{AE}^2 + \drawProportionalLine{EH,HF}^2 \cr
&= 2 \cdot \drawProportionalLine{AC}^2 + 2 \cdot \drawProportionalLine{CD}^2
}\right. \mbox{ \inprop[prop:I.XLVII]}
}\right.$

$\therefore \drawProportionalLine{AC,CD}^2 + \drawProportionalLine{DB}^2 = 2 \cdot \drawProportionalLine{AC}^2 + 2 \cdot \drawProportionalLine{CD}^2$
\stopCenterAlign

\qed
\stopProposition

\startProposition[title={Prop. X. theor.},reference=prop:II.X]
\defineNewPicture{
pair A, B, C, D, E, F, G;
numeric w;
w := 7/2u;
A := (0, 0);
B := (w, 0);
C := 1/2[A, B];
D := 4/3[A, B];
E := (w/2, w/2);
F := (xpart(D), ypart(E));
G = whatever[E, B] = whatever[F, D];
draw byAngleWithName(E, A, C, black, 0)(A);
draw byAngle(C, E, A, byyellow, 0);
draw byAngle(B, E, C, byyellow, 0);
draw byAngle(F, E, B, byblue, 0);
draw byAngle(C, B, E, byred, 0);
draw byAngle(D, B, G, byred, 0);
draw byAngleWithName(B, G, D, byblue, 0)(G);
draw byLine(C, E, byred, 0, -1);
draw byLine(A, C, byred, 0, 0);
draw byLine(C, B, byyellow, 0, 0);
draw byLine(B, D, byblue, 0, 0);
draw byLine(E, B, black, 0, 0);
draw byLine(B, G, black, 1, 0);
byLineDefine(E, F, byyellow, 1, 0);
byLineDefine(F, D, byred, 0, 0);
byLineDefine(D, G, byred, 1, 0);
byLineDefine(A, G, black, 0, 0);
byLineDefine(A, E, byblue, 1, 0);
draw byNamedLineSeq(0)(EF,FD,DG,AG,AE);
draw byLabelsOnPolygon(A, E, F, D, G)(0, 0);
draw byLabelsOnPolygon(G, B, C, A)(2, 0);
}
\drawCurrentPictureInMargin
\problemNP{I}{f}{a straight line \drawProportionalLine{AC,CB} be bisected and produced to any point \drawProportionalLine{AC,CB,BD}, the squares of the whole produced line, and of the produced part, are together double of the squares of half line, and of the line made up of the half and produced part.\\
$\drawProportionalLine{AC,CB,BD}^2 + \drawProportionalLine{BD}^2 = 2 \cdot \drawProportionalLine{CB}^2 + 2 \cdot \drawProportionalLine{CB,BD}^2$
}

\startCenterAlign
Make $\drawProportionalLine{CE} \perp \mbox{ and } = \mbox{ to } \drawProportionalLine{AC} \mbox{ or } \drawProportionalLine{CB}$,\\
draw \drawProportionalLine{AE} and \drawProportionalLine{EB,BG},\\
and $\left\{\eqalign{
\drawProportionalLine{FD,DG} & \parallel \drawProportionalLine{CE} \cr
\drawProportionalLine{EF} & \parallel \drawProportionalLine{CB,BD}
}\right\}$ \inprop[prop:I.XXXI]\\
draw \drawProportionalLine{AG} also.

$\drawAngle{A} = \drawAngle{CEA}$ \inprop[prop:I.V] $=$ half right angle \inprop[prop:I.XXXII]\\
$\drawAngle{CBE} = \drawAngle{BEC}$ \inprop[prop:I.V] $=$ half right angle \inprop[prop:I.XXXII]\\
$\therefore \drawAngle{CEA,BEC} = $ a right angle.

$\drawAngle{A} = \drawAngle{CBE} = \drawAngle{BEC} = \drawAngle{FEB} = \drawAngle{G} =$ half a right angle (\inpropL[prop:I.V], \inpropL[prop:I.XXXII], \inpropL[prop:I.XXIX], \inpropL[prop:I.XXXIV]),\\
and $\drawProportionalLine{BD} = \drawProportionalLine{DG}$, $\drawProportionalLine{CB,BD} = \drawProportionalLine{EF} = \drawProportionalLine{FD,DG}$, (\inpropL[prop:I.VI], \inpropL[prop:I.XXXIV]).

Hence by \inprop[prop:I.XLVII]

$\drawProportionalLine{AG}^2 = \left\{\eqalign{
& \drawProportionalLine{AC,CB,BD}^2 + \drawProportionalLine{DG}^2 \mbox{ or } \drawProportionalLine{BD}^2 \cr
& \left\{\eqalign{
& + \drawProportionalLine{AE}^2 = 2 \cdot \drawProportionalLine{AC}^2 \cr
& + \drawProportionalLine{EB,BG}^2 = 2\cdot \drawProportionalLine{EF}^2
}\right.
}\right.$\\
$\therefore \drawProportionalLine{AC,CB,BD}^2 + \drawProportionalLine{BD}^2 = 2 \cdot \drawProportionalLine{CB}^2 + 2 \cdot \drawProportionalLine{CB,BD}^2$
\stopCenterAlign

\qed
\stopProposition

\startProposition[title={Prop. XI. prob.},reference=prop:II.XI]
\defineNewPicture{
pair A, B, C, D, E, F, G, H, K;
numeric w;
w := 7/2u;
A := (0, 0);
B := (w, 0);
C := (0, w);
D := (w, w);
E := 1/2[A, C];
F := E shifted (0, -abs(E-B));
G := F shifted (abs(F-A), 0);
H = whatever[A, B] = (xpart(G), whatever);
K = whatever[G, H] = whatever[C, D];
draw byPolygon(A,H,K,C)(byyellow);
draw byPolygon(H,B,D,K)(byblue);
draw byPolygon(A,H,G,F)(byblue);
byLineDefine(K, G, black, 1, 0);
byLineDefine(A, H, byred, 0, 0);
byLineDefine(H, B, byred, 1, 0);
byLineDefine(B, E, black, 0, 0);
byLineDefine(C, E, byblue, 1, 0);
byLineDefine(E, A, byblue, 0, 0);
byLineDefine(A, F, byyellow, 0, 0);
draw byNamedLineSeq(1)(BE,HB,AH);
draw byNamedLine(CE,EA,AF,KG);
draw byLabelsOnPolygon(F, A, E, C, K, D, B, H, G)(0, 0);
}
\drawCurrentPictureInMargin
\problemNP{T}{o}{divide a given straight line \drawProportionalLine{AH,HB} in such a manner, that the rectangle contained by the whole line and one of its parts may be equal to the square of the other.\\
$\drawProportionalLine{AH,HB} \cdot \drawProportionalLine{HB} = \drawProportionalLine{AH}^2$
}

\startCenterAlign
Describe \drawPolygon[bottom]{AHKC,HBDK} \inprop[prop:I.XLVI],\\
make $\drawProportionalLine{EA} = \drawProportionalLine{CE}$ \inprop[prop:I.X],\\
draw \drawProportionalLine{BE},\\
take $\drawProportionalLine{EA,AF} = \drawProportionalLine{BE}$ \inprop[prop:I.III],\\
on \drawProportionalLine{AF} describe \drawPolygon[bottom]{AHGF} \inprop[prop:I.XLVI],\\
Produce \drawProportionalLine{KG} \inpost[post:II].

Then, \inprop[prop:II.VI] $\drawProportionalLine{CE,EA,AF} \cdot \drawProportionalLine{AF} + \drawProportionalLine{EA}^2 = \drawProportionalLine{EA,AF}^2 = \drawProportionalLine{BE}^2 = \drawProportionalLine{AH,HB}^2 + \drawProportionalLine{EA}^2 \therefore \drawProportionalLine{CE,EA,AF} \cdot \drawProportionalLine{AF} = \drawProportionalLine{AH,HB}^2$, or,\\
$\drawPolygon[bottom]{AHKC,AHGF} = \drawPolygon[bottom]{AHKC,HBDK} \therefore \drawPolygon[bottom]{AHGF} = \drawPolygon[bottom]{HBDK}$

$\therefore \drawProportionalLine{AH,HB} \cdot \drawProportionalLine{HB} = \drawProportionalLine{AH}^2$

\stopCenterAlign

\qed
\stopProposition

\startProposition[title={Prop. XII. theor.},reference=prop:II.XII]
\defineNewPicture{
pair A, B, C, D;
numeric w, h;
w := 7/2u;
h := 3u;
A := (3/5w, 0);
B := (w, h);
C := (0, 0);
D := (w, 0);
draw byLine(B, A, byred, 0, 0);
byLineDefine(C, A, black, 0, 0);
byLineDefine(A, D, black, 1, 0);
byLineDefine(D, B, byyellow, 0, 0);
byLineDefine(B, C, byblue, 0, 0);
draw byNamedLineSeq(0)(DB,AD,CA,BC);
draw byLabelsOnPolygon(C, B, D, A)(0, 0);
}
\drawCurrentPictureInMargin
\problemNP{I}{n}{any obtuse angled triangle, the square of the side subtending the obtuse angle exceeds the sum of the squares of the sides containing the obtuse angle, by twice the rectangle contained by either of these sides and the produced parts of the same from the obtuse angle to the perpendicular let fall on it from the opposite acute angle.\\
$\drawProportionalLine{BC}^2 > \drawProportionalLine{CA}^2 + \drawProportionalLine{BA}^2$ by $2 \cdot \drawProportionalLine{CA} \cdot \drawProportionalLine{AD}$
}

\startCenterAlign
By \inpropL[prop:II.IV]\\
$\drawProportionalLine{CA,AD}^2 = \drawProportionalLine{CA}^2 + \drawProportionalLine{AD}^2 + 2 \cdot \drawProportionalLine{CA} \cdot \drawProportionalLine{AD}^2$:

add $\drawProportionalLine{DB}^2$ to both\\
$\drawProportionalLine{CA,AD}^2 + \drawProportionalLine{DB}^2 = \drawProportionalLine{BC}^2$ \inprop[prop:I.XLVII]\\
$= 2 \cdot \drawProportionalLine{CA} \cdot \drawProportionalLine{AD} + \drawProportionalLine{CA}^2 + \left\{\eqalign{
&\drawProportionalLine{AD}^2 \cr
&\drawProportionalLine{DB}^2
}\right\}$ or\\
$+ \drawProportionalLine{BA}^2$ \inprop[prop:I.XLVII].

$\therefore \drawProportionalLine{BC}^2 = 2 \cdot \drawProportionalLine{CA} \cdot \drawProportionalLine{AD} + \drawProportionalLine{CA}^2 + \drawProportionalLine{BA}^2$: hence $ \drawProportionalLine{BC}^2 > \drawProportionalLine{CA}^2 + \drawProportionalLine{BA}^2$ by $2 \cdot \drawProportionalLine{CA} \cdot \drawProportionalLine{AD}^2$
\stopCenterAlign

\qed
\stopProposition

\startProposition[title={Prop. XIII. theor.},reference=prop:II.XIII]
\defineNewPicture[1/4]{
pair A,B,C,D,E,F,G,H, d;
numeric w, h;
w := 3u;
h := 3u;
A := (2/5w, h);
B:= (0, 0);
C := (w, 0);
D = whatever[B, C] = (xpart(A), whatever);
d := (0, -h -4/3u);
E := (w, h) shifted d;
F := (0, 0) shifted d;
G := (2/5w, 0) shifted d;
H = whatever[F, G] = (xpart(E), whatever);
draw byLine(A, D, byyellow, 0, 0);
byLineDefine(A, B, byred, 0, 0);
byLineDefine(B, D, black, 0, 0);
byLineDefine(D, C, black, 1, 0);
byLineDefine(C, A, byblue, 0, 0);
draw byNamedLineSeq(0)(AB,BD,DC,CA);
draw byLine(E, G, byblue, 0, 0);
byLineDefine(E, F, byred, 0, 0);
byLineDefine(F, G, black, 0, 0);
byLineDefine(G, H, black, 1, 0);
byLineDefine(H, E, byyellow, 0, 0);
draw byNamedLineSeq(0)(EF,FG,GH,HE);
label.top("First", (xpart(1/2[B, C]), ypart(A) + 1/4u));
label.top("Second", (xpart(1/2[F, H]), ypart(E)));
draw byLabelsOnPolygon(B, A, C, D)(0, 0);
draw byLabelsOnPolygon(F, E, H, G)(0, 0);
}
\drawCurrentPictureInMargin
\problemNP{I}{n}{any triangle, the square of the side subtending an acute angle, is less than the sum of the squares of the sides containing that angle, by twice the rectangle contained by either of these sides, and the part of it intercepted between the foot of the perpendicular let fall on it from the opposite angle, and the angular point of the acute angle.\\
First.\\
$\drawProportionalLine{CA}^2 < \drawProportionalLine{BD,DC}^2 + \drawProportionalLine{AB}^2$ by $2 \cdot \drawProportionalLine{BD,DC} \cdot \drawProportionalLine{BD}$.\\
Second.\\
$\drawProportionalLine{EG}^2 < \drawProportionalLine{EF}^2 + \drawProportionalLine{FG}^2$ by $2 \cdot \drawProportionalLine{FG} \cdot \drawProportionalLine{FG,GH}$.
}

\startCenterAlign
First, suppose the perpendicular to fall within the triangle, then \inprop[prop:II.VII]\\
$\drawProportionalLine{BD,DC}^2 + \drawProportionalLine{BD}^2 = 2 \cdot \drawProportionalLine{BD,DC} \cdot \drawProportionalLine{BD} + \drawProportionalLine{DC}^2$,\\
add to each $\drawProportionalLine{AD}^2$ then,\\
$\drawProportionalLine{BD,DC}^2 + \drawProportionalLine{BD}^2 + \drawProportionalLine{AD}^2 = 2 \cdot \drawProportionalLine{BD,DC} \cdot \drawProportionalLine{BD} + \drawProportionalLine{DC}^2 + \drawProportionalLine{AD}^2$\\
$\therefore$ \inprop[prop:I.XLVII]\\
$\drawProportionalLine{BD,DC}^2 + \drawProportionalLine{AB}^2 = 2 \cdot \drawProportionalLine{BD,DC} \cdot \drawProportionalLine{BD} + \drawProportionalLine{CA}^2$,\\
and $\therefore \drawProportionalLine{CA}^2 < \drawProportionalLine{BD,DC}^2 + \drawProportionalLine{AB}^2$ by $2 \cdot \drawProportionalLine{BD,DC} \cdot \drawProportionalLine{DC}$

Next suppose the perpendicular to fall without the triangle, then \inprop[prop:II.VII]\\
$\drawProportionalLine{FG,GH}^2 + \drawProportionalLine{FG}^2 = 2 \cdot \drawProportionalLine{FG,GH} \cdot \drawProportionalLine{FG} + \drawProportionalLine{GH}^2$,\\
add to each $\drawProportionalLine{HE}^2$ then\\
$\drawProportionalLine{FG,GH}^2 + \drawProportionalLine{FG}^2 + \drawProportionalLine{HE}^2= 2 \cdot \drawProportionalLine{FG,GH} \cdot \drawProportionalLine{FG} + \drawProportionalLine{GH}^2 + \drawProportionalLine{HE}^2$\\
$\therefore$ \inprop[prop:I.XLVII]\\
$\drawProportionalLine{EF} + \drawProportionalLine{FG}^2 = 2 \cdot \drawProportionalLine{FG,GH} \cdot \drawProportionalLine{FG}^2 + \drawProportionalLine{EG}^2$,\\
$\therefore \drawProportionalLine{EG}^2 < \drawProportionalLine{EF}^2 + \drawProportionalLine{FG}^2$ by $2 \cdot \drawProportionalLine{FG,GH} \cdot \drawProportionalLine{FG}$.
\stopCenterAlign

\qed
\stopProposition

\startProposition[title={Prop. XIV. prob.},reference=prop:II.XIV]
\defineNewPicture[1/3]{
path A;
pair a, b, c, d, e, f;
pair B, C, D, E, F, G, H;
numeric w, h, s, r;
a := (0, 0);
b := (u, 1/4u);
c := (2u, 1/5u);
d := (11/5u, -u);
e := (7/8u, -2u);
f := (1/8u, -3/4u);
A := a--b--c--d--e--f--cycle;
s := 0;
for i := 1 step 1 until length(A) - 1:
	s := s + 1/2(
		abs((point 0 of A) - (point i of A))
		*distanceToLine((point i + 1 of A), (point 0 of A)--(point i of A))
		);
endfor;
w := 5/2u;
h := s/w;
B := (w, 0);
C := (w, -h);
D := (0, -h);
E := (0, 0);
F := (-h, 0);
G := 1/2[B, F];
r := abs(B - G);
H := (D--(E shifted ((E-D)*10))) intersectionpoint ((fullcircle scaled 2r) shifted G);
forsuffixes i=a,b,c,d,e,f:
	i := i shifted (xpart(G)-xpart(1/2[urcorner(A),ulcorner(A)]), r - ypart(lrcorner(A)) + 1/2u);
	byPointLabelDefine(i, "");
endfor;
draw byPolygon(a,b,c,d,e,f)(byyellow);
draw byPolygon(B,C,D,E)(byred);
draw byLine(H, G, byred, 0, 0);
draw byLine(H, E, byblue, 0, 0);
draw byLine(E, D, byyellow, 0, 0);
byLineDefine(H, F, black, 0, 1);
byLineDefine(H, B, black, 0, 1);
byLineDefine(F, E, black, 1, 0);
byLineDefine(E, G, byblue, 1, 0);
byLineDefine(G, B, black, 0, 0);
draw byNamedLineSeq(-1)(FE,EG,GB,HB,HF);
draw byArc(G, B, F, r, byred, 0, 0, 0, 0)(G);
byLineDefine(B, F, black, 0, 0);
draw byLabelsOnPolygon(B, C, D, E, F, H)(0, 0);
draw byLabelsOnPolygon(H, G, B)(2, 0);
}
\drawCurrentPictureInMargin
\problemNP[4]{T}{o}{draw a right line of which the square shall be equal to a given rectilinear figure.\\
To draw \drawSizedLine{HE} such that, $\drawSizedLine{HE}^2 = \drawPolygon{abcdef}$
}

\startCenterAlign
Make $\drawPolygon{BCDE} = \drawPolygon{abcdef}$ \inprop[prop:I.XLV],

Produce \drawSizedLine{EG,GB} until $\drawSizedLine{FE} = \drawSizedLine{ED}$;\\
take $\drawSizedLine{FE,EG} = \drawSizedLine{GB}$ \inprop[prop:I.X],

Describe
\drawFromCurrentPicture[bottom]{
startTempScale(1/3);
draw byNamedLineFull(B, F, 0, 0, 1)(BF);
startAutoLabeling;
draw byNamedArc(G);
stopAutoLabeling;
stopTempScale;
}
\inpost[post:III],\\
and produce \drawSizedLine{ED} to meet it: draw \drawSizedLine{HG}.

$\drawSizedLine{HF}^2 \mbox{ or } \drawSizedLine{HG}^2 = \drawSizedLine{FE} \cdot \drawSizedLine{EG,GB} + \drawSizedLine{EG}^2$ \inprop[prop:II.V],\\
but $\drawSizedLine{HG}^2 = \drawSizedLine{HE} ^2 + \drawSizedLine{EG}^2$ \inprop[prop:I.XLVII];

$\therefore \drawSizedLine{HE}^2 + \drawSizedLine{EG}^2 = \drawSizedLine{FE} \cdot \drawSizedLine{EG,GB} + \drawSizedLine{EG}^2$

$\therefore \drawSizedLine{HE}^2 = \drawSizedLine{FE} \cdot \drawSizedLine{EG,GB}$, and

$\therefore \drawSizedLine{HE}^2 = \drawPolygon{BCDE} = \drawPolygon{abcdef}$
\stopCenterAlign

\qed
\stopProposition
\stopbook

\startbook[title={Book III}]

\startsupersection[title={Definitions}]

\startDefinitionOnlyNumber[reference=def:III.I]
Equal circles are those whose diameters are equal.
\stopDefinitionOnlyNumber

\startDefinitionOnlyNumber[reference=def:III.II]
\defineNewPicture{
	pair O, A, B;
	numeric r;
	r := 3/4u;
	O := (0, 0);
	A := (-r, -r);
	B := (r, -r);
	draw byCircleR(O, r, black, 0, 0, -1)(O);
	draw byLine(A, B, black, 0, 0);
}\drawCurrentPictureInMargin
A right line is said to touch a circle when it meets the circle, and being produced does not cut it.
\stopDefinitionOnlyNumber

\startDefinitionOnlyNumber[reference=def:III.III]
\defineNewPicture{
	pair A, B, C;
	numeric r[];
	r1 := 2/3u;
	r2 := 1/2r1;
	r3 := 2/5r1;
	A := (0, 0);
	B := A shifted (dir(110) scaled (r1-r2));
	C := A shifted (dir(-130) scaled (r1+r3));
	fill (fullcircle scaled 2r1) shifted A withcolor byyellow;
	draw byCircleR(A, r1, byyellow, 0, 0, 0)(A);
	fill (fullcircle scaled 2r2) shifted B withcolor byred;
	draw byCircleR(B, r2, black, 0, 0, -1/2)(B);
	fill (fullcircle scaled 2r3) shifted C withcolor byblue;
	draw byCircleR(C, r3, byblue, 0, 0, -1/2)(C);
}\drawCurrentPictureInMargin
Circles are said to touch one another which meet, but do not cut one another.
\stopDefinitionOnlyNumber

\startDefinitionOnlyNumber[reference=def:III.IV]
\defineNewPicture{
	pair O, A, B, C, D, E, F;
	numeric r;
	r := 3/4u;
	O := (0, 0);
	A := dir(170) scaled r;
	B := dir(-110) scaled r;
	C := A xscaled -1;
	D := B xscaled -1;
	E := 1/2[A, B];
	F := 1/2[C, D];
	draw byLine(A, B, black, 0, 0);
	draw byLine(C, D, black, 0, 0);
	byLineDefine(O, E, byred, 0, 0);
	byLineDefine(O, F, byblue, 0, 0);
	draw byNamedLineSeq(0)(OE,OF);
	draw byCircleR(O, r, black, 0, 0, 0)(O);
}\drawCurrentPictureInMargin
Right lines are said to be equally distant from the centre of a circle when the perpendiculars drawn to them from the centre are equal.
\stopDefinitionOnlyNumber

\startDefinitionOnlyNumber[reference=def:III.V]
And the straight line on which the greater perpendicular falls is said to be farther from the centre.
\stopDefinitionOnlyNumber

\vfill\pagebreak

\startDefinitionOnlyNumber[reference=def:III.VI]
\defineNewPicture{
	pair A, B;
	numeric r;
	r := 3/4u;
	A := (0, 0);
	B := (0, -1/8u);
	draw byFilledCircleSegment(A, r, 1/2, 4 - 1/2, byred)(A);
	draw byFilledCircleSegment(B, r, 4 - 1/2, 8 + 1/2, byblue)(B);
}\drawCurrentPictureInMargin
A segment of a circle is the figure contained by a straight line and the part of the circumference it cuts off.
\stopDefinitionOnlyNumber

\startDefinitionOnlyNumber[reference=def:III.VII]
An angle of a segment is that contained by a straight line and a circumference of a circle. 
\stopDefinitionOnlyNumber

% improvement: originally, seventh definition was missed

\startDefinitionOnlyNumber[reference=def:III.VIII]
\defineNewPicture{
	pair O, A, B, C, D;
	numeric r, b, e;
	r := u;
	b := -1;
	e := 5;
	O := (0, 0);
	A := dir(100) scaled r;
	B := dir(30) scaled r;
	C := point b of fullcircle scaled 2r;
	D := point e of fullcircle scaled 2r;
	draw byAngleWithName(C, A, D, byblue, 0)(A);
	draw byAngleWithName(C, B, D, byyellow, 0)(B);
	draw byArcBE(O, b, e, r, black, 0, 0, 0, 0)(O);
	draw byLine(C, A, black, 0, 0);
	draw byLine(C, B, black, 0, 0);
	draw byLine(D, A, black, 0, 0);
	draw byLine(D, B, black, 0, 0);
	draw byLine(C, D, black, 0, 0);
}\drawCurrentPictureInMargin
An angle in a segment is the angle contained by two straight lines drawn from any point in the circumference of the segment to the extremities of the straight line which is the base of the segment.
\stopDefinitionOnlyNumber

\startDefinitionOnlyNumber[reference=def:III.IX]
\defineNewPicture{
	pair O, A, B, C;
	numeric r, b, e;
	r := u;
	b := -3/2;
	e := 9/2;
	O := (0, 0);
	A := dir(80) scaled r;
	B := point b of fullcircle scaled 2r;
	C := point e of fullcircle scaled 2r;
	draw byAngleWithName(C, A, B, byblue, 0)(A);
	draw byArcBE(O, b, e, r, black, 1, 0, 0, 0)(Op);
	draw byArcBE(O, e, b + 8, r, black, 0, 0, 0, 0)(Om);
	draw byLine(C, A, black, 0, 0);
	draw byLine(B, A, black, 0, 0);
}\drawCurrentPictureInMargin
An angle is said to stand on the part of the circumference, or the arch, intercepted between the right lines that contain the angle.
\stopDefinitionOnlyNumber

\startDefinitionOnlyNumber[reference=def:III.X]
\defineNewPicture{
	pair O;
	numeric r, b, e;
	r := 2/3u;
	b := 1;
	e := 3;
	O := (0, 0);
	draw byFilledCircleSector(O, r, b, e, byyellow)(O);
	draw byArcBE(O, e, b + 8, r, black, 0, 0, -1, 0)(O);
}\drawCurrentPictureInMargin
A sector of a circle is the figure contained by two radii and the arch between them.
\stopDefinitionOnlyNumber

\startDefinitionOnlyNumber[reference=def:III.XI]
\defineNewPicture{
	pair M, N, A, B, C, D, E, F;
	numeric r[], b, e;
	r1 := 3/2u;
	r2 := u;
	b := 1;
	e := 3;
	M := (0, 0);
	N := (0, -1/3u);
	A := (point b of fullcircle scaled 2r1) shifted M;
	B := (point 1/3[b,e] of fullcircle scaled 2r1) shifted M;
	C := (point e of fullcircle scaled 2r1) shifted M;
	D := (point b of fullcircle scaled 2r2) shifted N;
	E := (point 1/3[b,e] of fullcircle scaled 2r2) shifted N;
	F := (point e of fullcircle scaled 2r2) shifted N;
	draw byPolygon(A,B,C)(byred);
	draw byPolygon(D,E,F)(byred);
	draw byArcBE(M, b, e, r1, black, 0, 0, 0, 1)(M);
	draw byArcBE(N, b, e, r2, black, 0, 0, 0, 1)(N);
}\drawCurrentPictureInMargin
Similar segments of circles are those, which contain equal angles.
\stopDefinitionOnlyNumber

\startDefinitionOnlyNumber[reference=def:III.XII]
\defineNewPicture{
	pair O;
	numeric r[];
	r1 := 1/3u;
	r2 := 1/2u;
	r3 := 3/4u;
	O := (0, 0);
	draw byFilledCircleSegment(O, r3, 0, 8, byred)(OI);
	draw byFilledCircleSegment(O, r2, 0, 8, white)(OII);
	draw byCircleR(O, r2, black, 0, 0, 0)(OII);
	draw byFilledCircleSegment(O, r1, 0, 8, byblue)(OIII);
}\drawCurrentPictureInMargin
Circles which have the same centre are called \emph{concentric circles}.
\stopDefinitionOnlyNumber
\stopsupersection

% improvement: this last definition was not marked as definition in Byrne's book, and is not present in conventional editions of the Elements, but since it sort of is, let it be

\vfill\pagebreak

\startProposition[title={Prop. I. prob.},reference=prop:III.I]
\defineNewPicture{
pair A, B, C, D, E, F, G;
numeric r, a;
r := 9/4u;
F := (0, 0);
A := F shifted (dir(170)*r);
B := F shifted (dir(-95)*r);
D := 1/2[A, B];
C := F shifted (dir(angle(A-B) - 90)*r);
E := F shifted (dir(angle(A-B) +90)*r);
G := F shifted (dir(-45)*1/2r);
a := -angle(G-D);
forsuffixes i=A, B, D, C, E, F, G:
	i := i rotated a;
endfor;
draw byAngle(A, D, F, byblue, 0);
draw byAngle(F, D, G, byyellow, 0);
draw byAngle(G, D, B, black, 0);
draw byLine(D, G)(byblue, 1, 0);
draw byLine(A, D)(byred, 0, 0);
draw byLine(D, B)(byred, 1, 0);
draw byLine(E, C)(black, 0, 0);
draw byMarkLine(1/2, black)(EC);
byLineDefine(A, G, byblue, 0, 0);
byLineDefine(B, G, black, 1, 0);
draw byNamedLineSeq(0)(AG,BG);
draw byCircleR(F, r, byblue, 0, 0, 0)(F);
draw byLabelsOnPolygon(A, G, B)(0, 0);
draw byLabelsOnPolygon(E, D, A)(2, 0);
draw byLabelsOnPolygon(E, F, C)(2, -4);
draw byLabelLineEnd(E, C, 0);
draw byLabelLineEnd(C, E, 0);
}
\drawCurrentPictureInMargin
\problemNP{T}{o}{find the centre of a given circle \drawCircle[middle][1/4]{F}.}

Draw within the circle any straight line \drawUnitLine{AD,DB}, make $\drawUnitLine{AD} = \drawUnitLine{DB}$, draw $\drawUnitLine{EC} \perp \drawUnitLine{AD,DB}$; bisect \drawUnitLine{EC}, and the point of bisection is the centre.

For, if it be possible, let any other point as the point of concourse of \drawUnitLine{AG}, \drawUnitLine{DG} and \drawUnitLine{BG} be the centre.

Because in \drawLine[bottom][triangleADG]{AG,DG,AD} and \drawLine[middle][triangleDGB]{DB,DG,BG} $\drawUnitLine{AG} = \drawUnitLine{BG}$ (hyp. and \indefL[def:XV]), $\drawUnitLine{AD} = \drawUnitLine{DB}$ (const.) and \drawUnitLine{DG} common, $\drawAngle{ADF,FDG} = \drawAngle{GDB}$ \inprop[prop:I.VIII], and therefore right angles; but $\drawAngle{FDG,GDB} = \drawRightAngle$ (const.) $\drawAngle{GDB} = \drawAngle{FDG,GDB}$ \inax[ax:XI] which is absurd; therefore the assumed point is not the centre of the circle; and in the same manner it can be proved that no other point which is not on \drawUnitLine{EC} is the centre, therefore the centre is in \drawUnitLine{EC}, and therefore the point where \drawUnitLine{EC} is bisected is the centre.

\qed
\stopProposition

\startProposition[title={Prop. II. theor.},reference=prop:III.II]
\defineNewPicture{
pair A, B, D, E, F;
numeric r;
r := 9/4u;
D := (0, 0);
A := (dir(185) scaled r) shifted D;
B := (dir(-70) scaled r) shifted D;
E := 3/5[A, B];
F := (dir(angle(E-D)) scaled r) shifted D;
draw byAngleWithName(D, A, B, byblue, 0)(A);
draw byAngleWithName(A, E, D, byyellow, 0)(E);
draw byAngleWithName(A, B, D, black, 0)(B);
draw byLine(D, E, black, 0, 0);
draw byLine(E, F, black, 1, 0);
draw byLine(A, B, byred, 0, 0);
byLineDefine(A, D, byyellow, 0, 0);
byLineDefine(B, D, byblue, 0, 0);
draw byNamedLineSeq(0)(AD,BD);
draw byCircleR(D, r, byred, 0, 0, 0)(D);
draw byLabelsOnPolygon(B, F, A, D)(0, 0);
draw byLabelsOnPolygon(A, E, F)(2, 0);
}
\drawCurrentPictureInMargin
\problemNP[2]{A}{straight}{line (\drawSizedLine{AB}) joining two points in the circumference of a circle \drawCircle[middle][1/4]{D}, lies wholly within the circle.}

\startCenterAlign
Find the centre of \circleD\ \inprop[prop:III.I];\\
from the centre draw \drawSizedLine{DE} to any point in \drawSizedLine{AB}, meeting the circumference from the centre;\\
draw \drawSizedLine{AD} and \drawSizedLine{BD}.

Then $\drawAngle{A} = \drawAngle{B}$ \inprop[prop:I.V]\\
but $\drawAngle{E} > \drawAngle{A} \mbox{ or } > \drawAngle{B}$ \inprop[prop:I.XVI]

$\therefore \drawSizedLine{AD} > \drawSizedLine{DE}$ \inprop[prop:I.XIX]\\
but $\drawSizedLine{AD} = \drawSizedLine{DE,EF}$,

$\therefore \drawSizedLine{DE,EF} > \drawSizedLine{DE}$;

$\therefore \drawSizedLine{DE} < \drawSizedLine{DE,EF}$;

$\therefore$ every point in \drawSizedLine{AB} lies within the circle.
\stopCenterAlign

\qed
\stopProposition

\startProposition[title={Prop. III. theor.},reference=prop:III.III]
\defineNewPicture[1/2]{
pair A, B, C, D, E, F;
numeric r;
r := 9/4u;
E := (0, 0);
A := (dir(-90 - 60) scaled r) shifted E;
B := (dir(-90 + 60) scaled r) shifted E;
C := (dir(90) scaled r) shifted E;
D := (dir(-90) scaled r) shifted E;
F = whatever[A, B] = whatever[C, D];
draw byAngleWithName(E, A, F, byblue, 0)(A);
draw byAngle(A, F, E, black, 0);
draw byAngle(E, F, B, byyellow, 0);
draw byAngleWithName(F, B, E, byred, 0)(B);
draw byLine(C, E, black, 1, 0);
draw byLine(E, F, black, 0, 0);
draw byLine(F, D, black, 1, 0);
draw byLine(A, F, byred, 0, 0);
draw byLine(F, B, byred, 1, 0);
byLineDefine(A, E, byyellow, 0, 0);
byLineDefine(E, B, byblue, 0, 0);
draw byNamedLineSeq(0)(AE,EB);
draw byCircleR(E, r, byblue, 0, 0, 0)(E);
draw byLabelLineEnd(A, E, 0);
draw byLabelLineEnd(B, E, 0);
draw byLabelsOnPolygon(D, F, A)(2, 0);
draw byLabelsOnPolygon(A, E, C)(2, 0);
}
\drawCurrentPictureInMargin
\problemNP{I}{f}{a straight line (\drawUnitLine{EF}) drawn through the centre of a circle \drawCircle[middle][1/6]{E} bisect a chord (\drawUnitLine{AF,FB}) which does not pass through the centre, it is perpendicular to it; or, if perpendicular to it, it bisects it.}

\startCenterAlign
Draw \drawUnitLine{AE} and \drawUnitLine{EB} to the centre of the circle.

In \drawLine[bottom][triangleAEF]{AE,EF,AF} and
\drawLine[bottom][triangleFEB]{EF,EB,FB}\\
$\drawUnitLine{AE} = \drawUnitLine{EB}$, \drawUnitLine{EF} common,\\
and $\drawUnitLine{AF} = \drawUnitLine{FB} \therefore \drawAngle{AFE} = \drawAngle{EFB}$ \inprop[prop:I.VIII]\\
and $\therefore \drawUnitLine{EF} \perp \drawUnitLine{AF,FB}$ \inprop[prop:I.X]

Again let $\drawUnitLine{EF} \perp \drawUnitLine{AF,FB}$\\
Then in \triangleAEF\ and \triangleFEB\\
$\drawAngle{A} = \drawAngle{B}$ \inprop[prop:I.V]\\
$\drawAngle{AFE} = \drawAngle{EFB}$ (hyp.)\\
and $\drawUnitLine{AE} = \drawUnitLine{EB}$

$\therefore \drawUnitLine{AF} = \drawUnitLine{FB}$ \inprop[prop:I.XXVI]

and $\therefore \drawUnitLine{EF}$ bisects \drawUnitLine{AF,FB}.
\stopCenterAlign

\qed
\stopProposition

\startProposition[title={Prop. IV. theor.},reference=prop:III.IV]
\defineNewPicture{
pair A, B, C, D, E, F;
numeric r;
r := 9/4u;
F := (0, 0);
A := (dir(-175)*r) shifted F;
B := (dir(-140)*r) shifted F;
C := (dir(-50)*r) shifted F;
D := (dir(-10)*r) shifted F;
E = whatever[A, C] = whatever[B, D];
draw byAngle(F, E, D, byblue, 0);
draw byAngle(D, E, C, byyellow, 0);
draw byLine(E, F, black, 1, 0);
draw byLine(B, D, byred, 0, 0);
draw byLine(A, C, black, 0, 0);
draw byCircleR(F, r, byblue, 0, 0, 0)(F);
draw byLabelLineEnd(A, F, 0);
draw byLabelLineEnd(B, F, 0);
draw byLabelLineEnd(C, F, 0);
draw byLabelLineEnd(D, F, 0);
draw byLabelLineEnd(F, E, 0);
draw byLabelsOnPolygon(C, E, B)(2, 0);
}
\drawCurrentPictureInMargin
\problemNP{I}{f}{in a circle two straight lines cut one another, which do not both pass through the centre, they do not bisect one another.}

If one of the lines pass through the centre, it is evident that it cannot be bisected by the other, which does not pass through the centre.

But if neither of the lines \drawUnitLine{AC} or \drawUnitLine{BD} pass through the centre, draw \drawUnitLine{EF} from the centre to their intersection.

\startCenterAlign
If \drawUnitLine{AC} be bisected, \drawUnitLine{EF} $\perp$ to it \inprop[prop:III.III]\\
$\therefore \drawAngle{FED,DEC} = \drawRightAngle$\\
and if \drawUnitLine{BD} be bisected, $\drawUnitLine{EF} \perp \drawUnitLine{BD}$ \inprop[prop:III.III]\\
$\therefore \drawAngle{FED} = \drawRightAngle$;

and $\therefore \drawAngle{FED} = \drawAngle{FED,DEC}$;\\
a part equal to whole, which is absurd:

$\therefore$ \drawUnitLine{AC} and \drawUnitLine{BD} do not bisect one another.
\stopCenterAlign

\qed
\stopProposition

\startProposition[title={Prop. V. theor.},reference=prop:III.V]
\defineNewPicture{
pair M, N, E, F, G, C;
numeric r[], s;
path c[];
r1 := 2u;
r2 := 2u;
s := u;
M := (1/2s, 0);
N := (-1/2s, 0);
c1 := (fullcircle scaled 2r1) shifted M;
c2 := (fullcircle scaled 2r2) shifted N;
E := 1/2[M, N];
C := (subpath(0, 4) of c1) intersectionpoint (subpath(0, 4) of c2);
G := point 7/2 of c2;
F := c1 intersectionpoint (E--G);
byLineDefine(C, E, byyellow, 0, 0);
byLineDefine(E, F, black, 0, 0);
byLineDefine(F, G, black, 1, 0);
draw byNamedLineSeq(0)(CE,EF,FG);
draw byArcBE(M, 4, 0, r1, byred, 0, 0, 0, 0)(Ma);
draw byArcBE(N, 4, 0, r2, byblue, 0, 0, 0, 0)(Na);
draw byArcBE(N, 4, 8, r2, byblue, 0, 0, 0, 0)(Nb);
draw byArcBE(M, 4, 8, r1, byred, 0, 0, 0, 0)(Mb);
draw byLabelLineEnd(C, E, 0);
draw byLabelLineEnd(G, E, 0);
draw byLabelsOnPolygon(C, E, F)(2, 0);
draw byLabelsOnPolygon(C, F, E)(2, -1);
}
\drawCurrentPictureInMargin
\problemNP{I}{f}{two circles intersect \drawArc{Ma,Na,Nb,Mb} they have not the same centre.}

Suppose it possible that two intersecting circles have a common centre; from such supposed centre draw \drawUnitLine{CE} to the intersecting point, and \drawUnitLine{EF,FG} meeting the circumferences of the circles.

\startCenterAlign
Then $\drawUnitLine{CE} = \drawUnitLine{EF}$ \inprop[prop:I.XV]\\
and $\drawUnitLine{CE} = \drawUnitLine{EF,FG}$ \inprop[prop:I.XV]

$\therefore \drawUnitLine{EF} = \drawUnitLine{EF,FG}$\\
a part equal to the whole, which is absurd:

$\therefore$ circles supposed to intersect in any point cannot have the same centre.
\stopCenterAlign

\qed
\stopProposition

\startProposition[title={Prop. VI. theor.},reference=prop:III.VI]
\defineNewPicture{
pair M, N, B, C, E, F;
numeric r[], a;
path c[];
a := 80;
r1 := 9/4u;
r2 := 7/4u;
M := (0, 0);
N := M shifted (dir(a)*(r1-r2));
c1 := (fullcircle scaled 2r1) shifted M;
c2 := (fullcircle scaled 2r2) shifted N;
F := 1/2[M, N];
C :=c1 intersectionpoint (M--(M shifted (dir(a)*2r1)));
B := point -3/2 of c1;
E := c2 intersectionpoint (F--B);
byLineDefine(C, F, byyellow, 0, 0);
byLineDefine(F, E, byblue, 1, 0);
byLineDefine(E, B, byblue, 0, 0);
draw byNamedLineSeq(0)(CF,FE,EB);
draw byCircle(M, C, byred, 0, 0, 0)(M);
draw byCircle(N, C, black, 0, 0, -1)(N);
draw byLabelLineEnd(C, F, 0);
draw byLabelLineEnd(B, F, 0);
draw byLabelsOnPolygon(E, F, C)(2, 0);
draw byLabelPoint(E, angle(B-F)+45, 2);
}
\drawCurrentPictureInMargin
\problemNP{I}{f}{two circles \drawFromCurrentPicture{draw byNamedCircle(M,N);} touch one another internally, they have not the same centre.}

For, if it be possible, let both circles have the same centre; from such a supposed centre draw \drawUnitLine{FE,EB}, and \drawUnitLine{CF} to the point of contact.

\startCenterAlign
Then $\drawUnitLine{CF} = \drawUnitLine{FE}$ \inprop[prop:I.XV]\\
and $\drawUnitLine{CF} = \drawUnitLine{FE,EB}$ \inprop[prop:I.XV]

$\therefore \drawUnitLine{FE} = \drawUnitLine{FE,EB}$;
\stopCenterAlign
\noindent a part equal to the whole, which is absurd; therefore the assumed point is not the centre of both circles, and in the same manner it can be demonstrated that no other point is.

\qed
\stopProposition

\startProposition[title={Prop. VII. theor.},reference=prop:III.VII]
\defineNewPicture[1/2]{
pair A, B, C, D, E, F, G, H, K, M, N, d;
numeric r;
r := 2u;
E := (0, 0);
A := E shifted (dir(90)*r);
D := E shifted (dir(-90)*r);
F := 2/3[E, D];
B := E shifted (dir(20)*r);
C := E shifted (dir(-5)*r);
G := E shifted (dir(15)*r);
H := E shifted (dir(-170)*r);
K := E shifted (dir(175)*r);
M = whatever[E, H] = whatever[F, K];
N := D;
draw byAngle(B, E, C, black, 0);
draw byAngle(C, E, F, byyellow, 0);
draw byLine(F, B, byred, 0, 0);
draw byLine(E, C, byblue, 1, 0);
draw byLine(F, C, byblue, 0, 0);
draw byLine(E, B, byred, 1, 0);
draw byLine(A, E, black, 1, 0);
draw byLine(E, F, black, 0, 0);
draw byLine(F, D, byyellow, 0, 0);
draw byCircleR(E, r, byblue, 0, 0, 0)(E);
draw byLabelLineEnd(A, E, 0);
draw byLabelLineEnd(B, E, 0);
draw byLabelLineEnd(C, E, 0);
draw byLabelLineEnd(D, E, 0);
draw byLabelsOnPolygon(D, F, E, A)(2, 0);
label.top("Fig. 1", (0, r + 1/4u));
d := (0, -2r -u);
draw image(
draw byAngle(G, F, E, byyellow, 0);
draw byAngle(K, F, E, byred, 0);
draw byLine(F, G, byred, 0, 0);
draw byLine(E, G, byred, 1, 0);
draw byLine(E, M, byyellow, 0, 0);
draw byLine(M, H, byyellow, 1, 0);
draw byLine(F, M, byblue, 0, 0);
draw byLine(M, K, byblue, 1, 0);
draw byLine(A, E, black, 1, 0);
draw byLine(E, F, black, 0, 0);
draw byLine(F, N, black, 0, 0);
draw byCircleR(E, r, byblue, 0, 0, 0)(E);
draw byLabelLineEnd(G, E, 0);
draw byLabelLineEnd(K, E, 0);
draw byLabelLineEnd(H, E, 0);
draw byLabelsOnPolygon(D, F, M, H)(2, 0);
draw byLabelsOnPolygon(M, E, A)(2, 0);
label.top("Fig. 2", (0, r));
) shifted d;
}
\drawCurrentPictureInMargin
\problemNP[2]{I}{f}{from any point within a circle \drawFromCurrentPicture{
startTempScale(1/3);
draw byNamedCircle(E);
draw byLabelPoint(F, 0, 0);
stopTempScale;
} which is not the centre, lines
$\left\{\vcenter{
\nointerlineskip\hbox{\drawUnitLine{EF,AE}}
\nointerlineskip\hbox{\drawUnitLine{FB}}
\nointerlineskip\hbox{\drawUnitLine{FC}}}\right.$
are drawn to the circumference; the greatest of those lines is that (\drawUnitLine{EF,AE}) which passes through the centre, and the least is the remaining part (\drawUnitLine{FD}) of the diameter.\\
Of the others, that (\drawUnitLine{FB}) which is nearer to the line passing through the centre, is greater than that (\drawUnitLine{FC}) which is more remote.\\
Fig. 2. The two lines (\drawUnitLine{FM,MK} and \drawUnitLine{FG}) which make equal angles with that passing through the centre, on opposite sides of it, are equal to each other; and there cannot be drawn a third line equal to them, from the same point to the circumference.}

\startsubproposition[title={Figure I.}]
To the centre of the circle draw \drawUnitLine{EB} and \drawUnitLine{EC}; then $\drawUnitLine{AE} = \drawUnitLine{EB}$ \inprop[prop:I.XV] $\drawUnitLine{EF,AE} = \drawUnitLine{EF} + \drawUnitLine{EB} > \drawUnitLine{FB}$ \inprop[prop:I.XX] in like manner \drawUnitLine{EF,AE} may be shown to be greater than \drawUnitLine{FC}, or any other line drawn from the same point to the circumference. Again, by \inprop[prop:I.XX] $\drawUnitLine{EF} + \drawUnitLine{FC} > \drawUnitLine{EC} = \drawUnitLine{FD} + \drawUnitLine{EF}$, take \drawUnitLine{EF} from both; $\therefore \drawUnitLine{FC} > \drawUnitLine{FD}$ (ax.), and in like manner it may be shown that \drawUnitLine{FD} is less than any other line drawn from the same point to the circumference. Again, in \drawLine[middle][triangleEFB]{FB,EF,EB} and \drawLine[middle][triangleEFC]{FC,EF,EC}, \drawUnitLine{EF} common, $\drawAngle{BEC,CEF} > \drawAngle{CEF}$, and $\drawUnitLine{EB} > \drawUnitLine{EC} \therefore \drawUnitLine{FB} > \drawUnitLine{FC}$ \inprop[prop:I.XXIV] and \drawUnitLine{FB} may in like manner to be proved greater than any other line drawn from the same point to the circumference more remote from \drawUnitLine{EF,AE}.
\stopsubproposition

\startsubproposition[title={Figure II.}]
\startCenterAlign
If $\drawAngle{KFE} = \drawAngle{GFE}$ then $\drawUnitLine{FM,MK} = \drawUnitLine{FG}$,\\
if not take $\drawUnitLine{FM} = \drawUnitLine{FG}$ draw \drawUnitLine{EM,MH}\\
then in \drawLine[middle][triangleEFM]{EM,EF,FM} and \drawLine[middle][triangleEFG]{FG,EF,EG}, \drawUnitLine{EF} common,\\
$\drawAngle{KFE} = \drawAngle{GFE}$ and $\drawUnitLine{FG} = \drawUnitLine{FM}$

$\therefore \drawUnitLine{EG} = \drawUnitLine{EM}$ \inprop[prop:I.IV]

$\therefore \drawUnitLine{EG} = \drawUnitLine{EM,MH} = \drawUnitLine{EM}$\\
a part equal to whole, which is absurd:
\stopCenterAlign

$\therefore \drawUnitLine{FG} = \drawUnitLine{FM,MK}$; and no other line is equal to \drawUnitLine{FG} drawn from the same point to the circumference; for if it were nearer to the one passing through the centre it would be greater, and if it were more remote it would be less.

\stopsubproposition

\qed
\stopProposition

\startProposition[title={Prop. VIII. theor.},reference=prop:III.VIII]
\problemNP{T}{he}{original text of this proposition is here divided into three parts.}

\startsubproposition[title={I.}]
\defineNewPicture[1/4]{
pair M, D, A, E, F;
numeric r;
r := 7/4u;
M := (0, 0);
D := M shifted (dir(90)*3/2r);
A := (dir(-90)*r) shifted M;
E := (dir(-140)*r) shifted M;
F := (dir(-170)*r) shifted M;
draw byAngle(D, M, F, byyellow, 0);
draw byAngle(F, M, E, black, 0);
draw byLine(D, E, byred, 0, 0);
draw byLine(M, A, black, 1, 0);
draw byLine(M, E, byred, 1, 0);
draw byLine(M, F, byblue, 1, 0);
byLineDefine(D, M, black, 0, 0);
byLineDefine(D, F, byblue, 0, 0);
draw byNamedLineSeq(0)(DM,DF);
draw byCircle(M, E, black, 0, 0, 0)(M);
draw byLabelLineEnd(F, M, 0);
draw byLabelLineEnd(E, M, 0);
draw byLabelLineEnd(A, M, 0);
draw byLabelsOnPolygon(F, D, M, A)(2, 0);
}
\drawCurrentPictureInMargin
If from a point without a circle, straight lines
$\left\{\vcenter{
\nointerlineskip\hbox{\drawProportionalLine{DM,MA}}
\nointerlineskip\hbox{\drawProportionalLine{DE}}
\nointerlineskip\hbox{\drawProportionalLine{DF}}}\right\}$
are drawn to the circumference; of those falling upon the concave circumference the greatest is that (\drawUnitLine{DM,MA}) which passes through the centre, and the line (\drawUnitLine{DE}) which is nearer the greatest is greater than that (\drawUnitLine{DF}) which is more remote.

\startCenterAlign
Draw \drawUnitLine{MF} and \drawUnitLine{ME} to the centre.\\
Then, \drawUnitLine{DM,MA} which passes through the centre, is greatest;\\
for since $\drawUnitLine{MA} = \drawUnitLine{ME}$, if \drawUnitLine{DM} be added to both $\drawUnitLine{DM,MA} = \drawUnitLine{DM} + \drawUnitLine{ME}$;\\
but $> \drawUnitLine{DE}$ \inprop[prop:I.XX]\\
$\therefore$ \drawUnitLine{DM,MA} is greater than any other line drawn from the same point to the concave circumference.\\
Again in \drawLine[middle][triangleDFM]{DM,MF,DF} and \drawLine[middle][triangleDEM]{DM,ME,DE}, $\drawUnitLine{MF} = \drawUnitLine{ME}$, and \drawUnitLine{DM} common,\\
but $\drawAngle{DMF,FME} > \drawAngle{DMF}$, $\therefore \drawUnitLine{DE} > \drawUnitLine{DF}$ \inprop[prop:I.XXIV];\\
and in like manner \drawUnitLine{DE} may be shown $>$ than any other line more remote from \drawUnitLine{DM,MA}.
\stopCenterAlign
\stopsubproposition

\vfill\pagebreak

\startsubproposition[title={II.}]
\defineNewPicture{
pair M, D, G, H, K;
numeric r;
r := 7/4u;
M := (0, 0);
D := M shifted (dir(90)*2r);
G := (dir(90)*r) shifted M;
H := (dir(130)*r) shifted M;
K := (dir(110)*r) shifted M;
draw byLine(G, M, black, 0, 0);
draw byLine(H, M, byblue, 0, 0);
draw byLine(D, K, byred, 1, 0);
draw byLine(K, M, byred, 0, 0);
byLineDefine(D, G, black, 1, 0);
byLineDefine(D, H, byblue, 1, 0);
draw byNamedLineSeq(0)(DG,DH);
draw byCircle(M, G, black, 0, 0, 0)(M);
draw byLabelsOnPolygon(K, H, M)(2, -2);
draw byLabelsOnPolygon(G, K, M)(2, -2);
draw byLabelsOnPolygon(D, G, K)(2, -2);
draw byLabelsOnPolygon(H, D, G)(2, 0);
draw byLabelsOnPolygon(G, M, H)(2, 0);
}
\drawCurrentPictureInMargin
Of those lines falling on the convex circumference the least (\drawUnitLine{DG}) which being produced would pass through the centre, and the line which is nearer to least is less than that which is more remote.

\startCenterAlign
For, since $\drawUnitLine{KM} + \drawUnitLine{DK} > \drawUnitLine{DG,GM}$ \inprop[prop:I.XX]\\
and $\drawUnitLine{KM} = \drawUnitLine{GM}$,\\
$\therefore \drawUnitLine{DK} > \drawUnitLine{DG}$ \inax[ax:V]

Again, since $\drawUnitLine{HM} + \drawUnitLine{DH} > \drawUnitLine{KM} + \drawUnitLine{DK}$ \inprop[prop:I.XXI],\\
and $\drawUnitLine{HM} = \drawUnitLine{KM}$,

$\therefore \drawUnitLine{DK} < \drawUnitLine{DH}$. And so of others.
\stopCenterAlign
\stopsubproposition

\startsubproposition[title={III.}]
\defineNewPicture{
pair M, D, B, G, H, N, O;
numeric r;
r := 7/4u;
M := (0, 0);
D := M shifted (dir(90)*2r);
G := (dir(90)*r) shifted M;
B := (dir(90 - 20)*r) shifted M;
H := (dir(90 + 45)*r) shifted M;
N := (dir(90 - 45)*r) shifted M;
O = whatever[D, N] = whatever[M, B];
draw byAngle(H, D, M, byyellow, 0);
draw byAngle(M, D, N, byblue, 0);
draw byLine(B, O, byred, 0, 0);
draw byLine(O, N, byyellow, 0, 0);
draw byLine(D, M, black, 0, 0);
draw byLine(H, M, byblue, 0, 0);
draw byLine(B, M, black, 1, 0);
byLineDefine(N, M, byyellow, 1, 0);
byLineDefine(D, H, byblue, 1, 0);
byLineDefine(D, O, byred, 1, 0);
draw byNamedLineSeq(0)(DH,DO,NM);
draw byCircle(M, H, black, 0, 0, 0)(M);
draw byLabelsOnPolygon(H, D, O, N)(2, 0);
draw byLabelsOnPolygon(N, M, H)(2, 0);
draw byLabelsOnPolygon(G, H, M)(2, -2);
draw byLabelsOnPolygon(M, B, G)(2, -2);
draw byLabelsOnPolygon(M, N, B)(2, -2);
}
\drawCurrentPictureInMargin
Also the lines making equal angles with that which passes through the centre are equal, whether falling on the concave or convex circumference; and no third line can be drawn equal to them from the same point to the circumference.

\startCenterAlign
For if $\drawUnitLine{DO,ON} > \drawUnitLine{DH}$, but making $\drawAngle{HDM} = \drawAngle{MDN}$;\\
make $\drawUnitLine{DO} = \drawUnitLine{DH}$, and draw \drawUnitLine{BM,BO}.

Then in \drawLine[middle][triangleDMO]{DM,DO,BO,BM} and \drawLine[middle][triangleDMH]{HM,DH,DM} we have $\drawUnitLine{DO} = \drawUnitLine{DH}$\\
and \drawUnitLine{DM} common, and also $\drawAngle{MDN} = \drawAngle{HDM}$,

$\therefore \drawUnitLine{BM,BO} = \drawUnitLine{HM}$ \inprop[prop:I.IV]

But $\drawUnitLine{HM} = \drawUnitLine{BM}$;

$\therefore \drawUnitLine{BM} = \drawUnitLine{BM,BO}$, which is absurd.

$\therefore \drawUnitLine{DH} \mbox{ is not } = \drawUnitLine{DO}$, nor to any part of \drawUnitLine{DO,ON}, $\therefore \drawUnitLine{DO,ON} \mbox{ is not } > \drawUnitLine{DH}$.

Neither is $\drawUnitLine{DH} > \drawUnitLine{DO,ON}$, they are $\therefore =$ to each other.
\stopCenterAlign
And any other line drawn from the same point to the circumference must lie at the same side with one of these lines, and be more or less remote than it from the line passing through the centre, and cannot therefore be equal to it.

\qed
\stopsubproposition
\stopProposition

\startProposition[title={Prop. IX. theor.},reference=prop:III.IX]
\defineNewPicture[1/4]{
pair D, A, B, C, F, L, H;
numeric r;
r := 7/4u;
D := (0, 0);
A := (dir(170)*r) shifted D;
B := (dir(-90)*r) shifted D;
C := (dir(-45)*r) shifted D;
L := (dir(45)*r) shifted D;
H := (dir(45 + 180)*r) shifted D;
F := 2/4[D, L];
draw byLine(D, A, byyellow, 1, 0);
draw byLine(D, B, byyellow, 0, 0);
draw byLine(D, C, byblue, 0, 0);
draw byLine(D, F, black, 0, 0);
draw byLine(F, L, byred, 1, 0);
draw byLine(D, H, byred, 0, 0);
draw byCircle(D, A, byblue, 0, 0, 0)(D);
draw byLabelLineEnd(A, D, 0);
draw byLabelLineEnd(B, D, 0);
draw byLabelLineEnd(C, D, 0);
draw byLabelLineEnd(H, D, 0);
draw byLabelLineEnd(L, D, 0);
draw byLabelsOnPolygon(A, D, F, L)(2, 0);
}
\drawCurrentPictureInMargin
\problemNP{I}{f}{a point be taken within a circle \drawCircle[middle]{D}, from which more than two equal straight lines (\drawUnitLine{DA}, \drawUnitLine{DB}, \drawUnitLine{DC}) can be drawn to the circumference, that point must be the centre of the circle.}

For if it be supposed that the point \drawFromCurrentPicture{
draw byNamedLineSeq(0)(DB,DC);
draw byLabelsOnPolygon(B, D, C)(2, 0);
} in which more than two equal straight lines meet is not the centre, some other point \drawUnitLine{DF,FL} must be; join these two points by \drawUnitLine{DF} and produce it both ways to the circumference.

Then since more than two equal straight lines are drawn from a point which is not the centre, to the circumference, two of them at least must lie at the same side of the diameter \drawUnitLine{DH,DF,FL}; and since from a point \drawFromCurrentPicture[middle][pointD]{
draw byNamedLine(DB);
draw byNamedLineSeq(0)(DH,DC);
draw byLabelsOnPolygon(H, D, C)(2, 0);
}, which is not the centre, straight lines are drawn to the circumference; the greatest is \drawUnitLine{DF,FL}, which passes through the centre: and \drawUnitLine{DC} which is nearer to \drawUnitLine{DF,FL}, $> \drawUnitLine{DB}$ which is more remote \inprop[prop:III.VIII]; but $\drawUnitLine{DC} = \drawUnitLine{DB}$ (hyp.) which is absurd.

The same may be demonstrated of any other point, different from \pointD, which must be the centre of the circle.

\qed
\stopProposition

\startProposition[title={Prop. X. theor.},reference=prop:III.X]
\defineNewPicture[1/5]{
pair P, G, H, B, d, dd;
pair Pd, Gd, Hd, Bd, Pdd;
numeric r, t[];
path cr[], crd[];
r := 7/4u;
d := (0, -9/4r);
P := (0, 0);
cr1 := ((fullcircle scaled 7/3r xscaled 4/5) rotated 45) shifted P;
cr2 := ((fullcircle scaled 7/3r xscaled 4/5) rotated -45) shifted P;
H := (subpath (0, 2) of cr1) intersectionpoint cr2;
B := (subpath (0, -2) of cr1) intersectionpoint cr2;
G := (subpath (-2, -4) of cr1) intersectionpoint cr2;
Pd := P shifted d;
crd1 := (fullcircle scaled 2r) shifted Pd;
dd := (0, -3/2r);
crd2 := crd1 shifted dd;
Pdd := Pd shifted dd;
t1 := xpart(crd2 intersectiontimes (subpath (-2, -4) of crd1));
t2 := xpart(crd2 intersectiontimes (subpath (0, -2) of crd1));
Bd := point t1 of crd2;
Hd := point t2 of crd2;
Gd := point -2 of crd1;
crd2 := subpath (t1, t2 + 8) of crd2;
crd2 := crd2 .. (point -2 of crd1) .. cycle;
draw byLine(P, B, byyellow, 0, 0);
draw byLine(P, G, black, 0, 0);
draw byLine(P, H, byblue, 0, 0);
draw byArbitraryFigure(cr1, byred, 0, 0)(fI);
draw byArbitraryFigure(cr2, byblue, 0, 0)(fII);
draw byLine(Pdd, Gd, black, 0, 0);
byLineDefine(Pdd, Bd, byyellow, 0, 0);
byLineDefine(Pdd, Hd, byblue, 0, 0);
draw byNamedLineSeq(0)(PddBd,PddHd);
draw byArbitraryFigure(crd1, byred, 0, 0)(fdI);
draw byArbitraryFigure(crd2, byblue, 0, 0)(fdII);
byCircleDefineR(P, r, byred, 0, 0, 0)(PI);
byCircleDefineR(P, r, byblue, 0, 0, 0)(PII);
draw byLabelLineEnd(G, P, 0);
draw byLabelLineEnd(B, P, 0);
draw byLabelLineEnd(H, P, 0);
draw byLabelsOnPolygon(G, P, H)(2, 0);
byPointLabelDefine(Gd, "G");
byPointLabelDefine(Hd, "H");
byPointLabelDefine(Bd, "B");
byPointLabelDefine(Dd, "D");
byPointLabelDefine(Pdd, "P");
draw byLabelLineEnd(Gd, Pdd, 0);
draw byLabelPoint(Bd, 180, 2);
draw byLabelPoint(Hd, 0, 2);
draw byLabelsOnPolygon(Hd, Pdd, Bd)(2, 0);
}
\drawCurrentPictureInMargin
\problemNP{O}{ne}{circle \drawFromCurrentPicture{draw byNamedCircle(PII);} cannot intersect another \drawFromCurrentPicture[middle][circlePI]{draw byNamedCircle(PI);} in more points than two.}

For if it be possible, let it intersect in three points; from the centre of \drawCircle{PII} draw \drawUnitLine{PG}, \drawUnitLine{PB} and \drawUnitLine{PH} to the points of intersection;

$\therefore \drawUnitLine{PG} = \drawUnitLine{PB} = \drawUnitLine{PH}$ \indef[def:XV], but as the circles intersect, they have not the same centre \inprop[prop:III.V]:

$\therefore$ the assumed point is not the centre of \circlePI , and $\therefore$ as \drawUnitLine{PG}, \drawUnitLine{PB} and \drawUnitLine{PH} from a point not the centre, they are not equal (\inpropL[prop:III.VII], \inpropL[prop:III.VIII]); but it was shown before that they were equal, which is absurd; the circles therefore do not intersect in three points.

\qed
\stopProposition

\startProposition[title={Prop. XI. theor.},reference=prop:III.XI]
\defineNewPicture[1/2]{
pair M, N, A, D, F, G, H, K;
numeric r[];
path cr[];
r1 := 9/4u;
r2 := 2/3r1;
M := (0, 0);
N := M shifted (0, +r1-r2);
cr1 := (fullcircle scaled 2r1) shifted M;
cr2 := (fullcircle scaled 2r2) shifted N;
A := (0, r1) shifted M;
F := 1/2[M,N] shifted (dir(-20)*1/3r2);
G := 1/2[M,N] shifted (dir(-20 + 180)*1/3r2);
D := cr2 intersectionpoint (F--10[F, G]);
H := cr1 intersectionpoint (F--10[F, G]);
K := cr1 intersectionpoint (F--10[G, F]);
draw byPolygon(A,F,G)(byyellow);
draw byLine(A, G, byred, 0, 0);
draw byLine(A, F, byblue, 1, 0);
draw byLine(H, D, byyellow, 0, 0);
draw byLine(D, G, byyellow, 1, 0);
draw byLine(G, F, black, 0, 0);
draw byLine(F, K, byblue, 0, 0);
byPointLabelDefine(M, "F");
byPointLabelDefine(N, "G");
draw byCircle(M, A, black, 0, 0, 0)(M);
draw byCircle(N, A, byblue, 0, 0, -1)(N);
draw byLabelsOnPolygon(K, F, G, D)(2, 0);
draw byLabelsOnPolygon(A, D, N)(2, 0); 
draw byLabelLineEnd(A, G, 0);
draw byLabelLineEnd(H, F, 0);
}
\drawCurrentPictureInMargin
\problemNP{I}{f}{two circles \drawCircle[middle][1/4]{N} and \drawCircle[middle][1/5]{M} touch one another internally, the right line joining their centres, being produced, shall pass through a point of contact.}

\startCenterAlign
For, if it be possible, let \drawSizedLine{GF} join their centres, and produce it both ways;\\
from a point of contact draw \drawSizedLine{AG} to the centre of \circleN,\\
and from the same point of contact draw \drawSizedLine{AF} to the centre of \circleM.

Because in
\drawFromCurrentPicture{
draw byNamedPolygon(AFG);
draw byNamedLineFull(A, A, 1, 1, 0)(GF);
};
$\drawSizedLine{GF} + \drawSizedLine{AG} > \drawSizedLine{AF}$ \inprop[prop:I.XX],\\
and $\drawSizedLine{AF} = \drawSizedLine{HD,DG,GF}$ as they are radii of \circleM,\\
but $\drawSizedLine{GF} + \drawSizedLine{AG} > \drawSizedLine{HD,DG,GF}$;\\
take away \drawSizedLine{GF} which is common,\\
and $\drawSizedLine{AG} > \drawSizedLine{HD,DG}$;\\
but $\drawSizedLine{AG} = \drawSizedLine{DG}$,\\
because they are radii of \circleN,\\
and $\therefore \drawSizedLine{DG} > \drawSizedLine{HD,DG}$ a part greater than the whole, which is absurd.
\stopCenterAlign

The centres are not therefore so placed, that a line joining them can pass through any point but a point of contact.

\qed
\stopProposition

\startProposition[title={Prop. XII. theor.},reference=prop:III.XII]
\defineNewPicture[1/4]{
pair M, N, A, C, D, F, G;
numeric r[];
path cr[];
r1 := 3/2u;
r2 := 2u;
M := (0, 0);
N := (0, -r1-r2);
cr1 := (fullcircle scaled 2r1) shifted M;
cr2 := (fullcircle scaled 2r2) shifted N;
A := M shifted (0, -r1);
F := M shifted (dir(185)*1/2r1);
G := N shifted (dir(175)*1/2r2);
C := cr1 intersectionpoint (F--G);
D := cr2 intersectionpoint (F--G);
byLineDefine(F, C, byred, 0, 0);
byLineDefine(C, D, black, 0, 0);
byLineDefine(D, G, byblue, 0, 0);
byLineDefine(A, F, byyellow, 1, 0);
byLineDefine(A, G, byyellow, 0, 0);
draw byNamedLineSeq(0)(FC,CD,DG,AG,AF);
draw byCircle(M, A, byblue, 0, 0, -1/2)(M);
draw byCircle(N, A, byred, 0, 0, -1/2)(N);
byPointLabelDefine(M, "F");
byPointLabelDefine(N, "G");
draw byLabelsOnPolygon(C, F, A)(2, 0);
draw byLabelsOnPolygon(A, G, D)(2, 0);
draw byLabelsOnPolygon(A, C, F)(2, 0);
draw byLabelsOnPolygon(G, D, D)(2, 0);
draw byLabelPoint(A, angle(F-A)-45, 2);
}
\drawCurrentPictureInMargin
\problemNP{I}{f}{two circle \drawCircle{M} and \drawCircle{N} touch one another externally, the straight line \drawUnitLine{FC,CD,DG} joining their centres, passes through the point of contact.}

If it be possible, let \drawUnitLine{FC,CD,DG} joining the centres, and not pass through a point of contact; then from a point of contact draw \drawUnitLine{AF} and \drawUnitLine{AG} to the centres.

\startCenterAlign
Because $\drawUnitLine{AF} + \drawUnitLine{AG} > \drawUnitLine{FC,CD,DG}$ \inprop[prop:I.XX]\\
and $\drawUnitLine{FC} = \drawUnitLine{AF}$ \inprop[prop:I.XV],\\
and $\drawUnitLine{DG} = \drawUnitLine{AG}$ \inprop[prop:I.XV],

$\therefore \drawUnitLine{FC} + \drawUnitLine{DG} > \drawUnitLine{FC,CD,DG}$, a part greater than whole, which is absurd.
\stopCenterAlign

The centres are not therefore so placed that the line joining them can pass through any point but the point of contact.

\qed
\stopProposition

\startProposition[title={Prop. XIII. theor.},reference=prop:III.XIII]
\problemNP{O}{ne}{circle cannot touch another, either externally or internally, in more points than one}

\defineNewPicture{
pair M, N, F, G, H, B, D;
numeric r[];
path cr[];
r1 := 7/4u;
r2 := 3/4r1;
M := (0, 0);
N := (dir(120)*(r1-r2)) shifted M;
cr1 := (fullcircle scaled 2r1) shifted M;
cr2 := (fullcircle scaled 2r2) shifted N;
t1 := xpart(cr1 intersectiontimes (M--10[M, N]));
t2 := xpart(cr2 intersectiontimes (M--10[M, N]));
cr2 := (subpath (t2 + 2/3, t2 - 2/3 + 8) of cr2) .. tension 3/2 .. cycle;
B := point (t1 - 1/2) of cr1;
D := point (t1 + 1/2) of cr1;
G := 3/4[B, 1/2[M, N]];
H := 5/4[B, 1/2[M, N]];
draw byLine(D, G, black, 0, 0);
byLineDefine(D, H, byred, 0, 0);
byLineDefine(B, G, byblue, 1, 0);
byLineDefine(G, H, byblue, 0, 0);
draw byNamedLineSeq(0)(BG,GH,DH);
draw byCircleR(M, r1, byyellow, 0, 0, 1)(M);
draw byArbitraryFigure(cr2, byblue, 0, 0)(fI);
byCircleDefineR(M, r1, byyellow, 0, 0, 0)(M);
byCircleDefineR(N, r2, byblue, 0, 0, 0)(N);
}

\startsubproposition[title={Figure I.}]\drawCurrentPictureInMargin
For if it possible, let \drawCircle{M} and \drawCircle{N} touch one another internally in two points; draw \drawUnitLine{GH} joining their centres, and produce it until it pass through one of the points of contact \inprop[prop:III.XI];

\startCenterAlign
draw \drawUnitLine{DH} and \drawUnitLine{DG},

But $\drawUnitLine{BG} = \drawUnitLine{DG}$ \indef[def:XV],\\
$\therefore$ if \drawUnitLine{GH} be added to both,\\
$\drawUnitLine{GH,BG} = \drawUnitLine{GH} + \drawUnitLine{DG}$;

but $\drawUnitLine{GH,BG} = \drawUnitLine{DH}$ \indef[def:XV],\\
and $\therefore \drawUnitLine{GH} + \drawUnitLine{DG} = \drawUnitLine{DH}$;

but $\therefore \drawUnitLine{GH} + \drawUnitLine{DG} > \drawUnitLine{DH}$ \inprop[prop:I.XX],\\
which is absurd.
\stopCenterAlign
\stopsubproposition

\vfill\pagebreak

\defineNewPicture{
pair P, Q, A, C;
numeric r[], t[];
path cr[];
r1 := 7/4u;
P := (0, 0);
Q := P shifted (1/4r1, 0);
cr3 := (fullcircle scaled 2r1) shifted P;
cr4 := (fullcircle scaled 2r1) shifted Q;
t3 := xpart(cr3 intersectiontimes (subpath (0, 4) of cr4));
t4 := xpart(cr3 intersectiontimes (subpath (4, 8) of cr4));
t5 := xpart(cr4 intersectiontimes (subpath (0, 4) of cr3));
t6 := xpart(cr4 intersectiontimes (subpath (4, 8) of cr3));
A := point t3 of cr3;
C := point t4 of cr3;
draw byLine(A, C, byred, 0, 0);
draw byMarkLine(3/7, black)(AC);
draw byMarkLine(4/7, black)(AC);
draw byArcBE(P, t3, t4, r1, byblue, 0, 0, 0, 0)(PI);
draw byArcBE(P, t4, t3 + 8, r1, black, 0, 0, 0, 0)(PII);
draw byArcBE(Q, t5, t6 + 8, r1, byblue, 0, 0, 0, 0)(QI);
draw byArcBE(Q, t5, t6, r1, black, 0, 0, 0, 0)(QII);
}


\startsubproposition[title={Figure II.}]\drawCurrentPictureInMargin
But if the points of contact be the extremities of the right line joining the centres, this straight line must be bisected in two different points for the two centres; because it is the diameter of both circles, which is absurd.
\stopsubproposition

\defineNewPicture{
pair S, T, Ad, Cd;
numeric r[];
path cr[];
r1 := 7/4u;
r2 := 3/4r1;
S := (0, 0);
T := S shifted (0, r1 + r2);
cr5 := (fullcircle scaled 2r1) shifted S;
cr6 := (fullcircle scaled 2r2) shifted T;
Ad := 1/2[point 2 of cr5, point 6 of cr6];
Cd := Ad shifted (1/3r2, 0);
cr5 := (subpath (2 + 2/4, 2 - 2/4 + 8) of cr5) .. tension 2 .. cycle;
cr6 := (subpath (6 + 2/4, 6 - 2/4 + 8) of cr6) .. tension 2 .. cycle;
byLineDefine(S, Ad, byred, 0, 0);
byLineDefine(Ad, T, byred, 1, 0);
byLineDefine(Cd, S, byblue, 1, 0);
byLineDefine(Cd, T, black, 0, 0);
draw byNamedLineSeq(0)(SAd,AdT,CdT,CdS);
draw byArbitraryFigure(cr5, byyellow, 0, 0)(fII);
draw byArbitraryFigure(cr6, byblue, 0, 0)(fIII);
byCircleDefineR(S, r1, byyellow, 0, 0, 0)(S);
byCircleDefineR(T, r2, byblue, 0, 0, 0)(T);
}
\startsubproposition[title={Figure III.}]\drawCurrentPictureInMargin
Next, if it be possible, let \drawCircle{S} and \drawCircle{T} touch externally in two points; draw \drawUnitLine{SAd,AdT} joining the centres of the circles, and passing through one of the points of contact, and draw \drawUnitLine{CdS} and \drawUnitLine{CdT}.

\startCenterAlign
$\drawUnitLine{CdS} = \drawUnitLine{SAd}$ \indef[def:XV];\\
and $\drawUnitLine{AdT} = \drawUnitLine{CdT}$ \indef[def:XV];

$\therefore \drawUnitLine{CdT} + \drawUnitLine{CdS} = \drawUnitLine{SAd,AdT}$;

but $\drawUnitLine{CdT} + \drawUnitLine{CdS} > \drawUnitLine{SAd,AdT}$ \inprop[prop:I.XX], which is absurd.
\stopCenterAlign
\stopsubproposition

There is therefore no case in which two circles can touch one another in two points.

\qed
\stopProposition

\startProposition[title={Prop. XIV. theor.},reference=prop:III.XIV]
\defineNewPicture{
pair A, B, C, D, E, F, G;
numeric r;
r := 9/4u;
E := (0, 0);
A := (dir(90-20)*r) shifted E;
B := (dir(90-130)*r) shifted E;
C := (dir(90+20)*r) shifted E;
D := (dir(90+130)*r) shifted E;
F = whatever[A, B] = whatever[E, E shifted ((A-B) rotated 90)];
G = whatever[C, D] = whatever[E, E shifted ((C-D) rotated 90)];
draw byAngleWithName(E, F, A, byyellow, 0)(F);
draw byAngleWithName(E, G, C, black, 1)(G);
draw byLine(E, A, black, 0, 0);
draw byLine(E, C, byblue, 0, 0);
byLineDefine(E, F, black, 1, 0);
byLineDefine(E, G, byblue, 1, 0);
draw byNamedLineSeq(0)(EF,EG);
draw byLine(A, F, byred, 0, 0);
draw byLine(F, B, byred, 1, 0);
draw byLine(C, G, byyellow, 0, 0);
draw byLine(G, D, byyellow, 1, 0);
draw byCircleR(E, r, byblue, 0, 0, 0)(E);
}
\drawCurrentPictureInMargin
\problemNP{E}{qual}{straight lines
$\left(\vcenter{\nointerlineskip\hbox{\drawProportionalLine{AF,FB}}\nointerlineskip\hbox{\drawProportionalLine{CG,GD}}}\right)$
inscribed in a circle are equally distant from the centre; and also, straight lines equally distant from the centre are equal.}

\startCenterAlign
From the centre of \drawCircle[middle][1/4]{E} draw \drawProportionalLine{EF} $\perp$ to \drawProportionalLine{AF,FB} and $\drawProportionalLine{EG} \perp \drawProportionalLine{CG,GD}$, join \drawProportionalLine{EA} and \drawProportionalLine{EC}.

Then $\drawProportionalLine{CG} = \mbox{ half } \drawProportionalLine{CG,GD}$ \inprop[prop:III.III]

Then $\drawProportionalLine{AF} = \frac{1}{2} \drawProportionalLine{AF,FB}$ \inprop[prop:III.III]\\
since $\drawProportionalLine{CG,GD} = \drawProportionalLine{AF,FB}$ (hyp.)\\
$\therefore \drawProportionalLine{CG} = \drawProportionalLine{AF}$,\\
and $\drawProportionalLine{EA} = \drawProportionalLine{EC}$ \indef[def:XV]\\
$\therefore \drawProportionalLine{EA}^2 = \drawProportionalLine{EC}^2$;

but since \drawAngle{F} is a right angle\\
$\drawProportionalLine{EA}^2 = \drawProportionalLine{EF}^2 + \drawProportionalLine{AF}^2$ \inprop[prop:I.XLVII]\\
and $\drawProportionalLine{EC}^2 = \drawProportionalLine{EG}^2 + \drawProportionalLine{CG}^2$ for the same reason,\\
$\therefore \drawProportionalLine{EF}^2 + \drawProportionalLine{AF}^2 = \drawProportionalLine{EG}^2 + \drawProportionalLine{CG}^2$

$\therefore \drawProportionalLine{EF}^2 = \drawProportionalLine{EG}^2$

$\therefore \drawProportionalLine{EF} = \drawProportionalLine{EG}$
\stopCenterAlign

Also, if the lines \drawProportionalLine{AF,FB} and \drawProportionalLine{CG,GD} be equally distant from the centre; that is to say, if the perpendiculars \drawProportionalLine{EF} and \drawProportionalLine{EG} be given equal, then $\drawProportionalLine{AF,FB} = \drawProportionalLine{CG,GD}$.

\startCenterAlign
For, as in the preceding case,\\
$\drawProportionalLine{EG}^2 + \drawProportionalLine{CG}^2 = \drawProportionalLine{AF}^2 + \drawProportionalLine{EF}^2$\\
but $\drawProportionalLine{EG}^2 = \drawProportionalLine{EF}^2$

$\therefore \drawProportionalLine{CG} = \drawProportionalLine{AF}$, and the doubles of these $\drawProportionalLine{AF,FB} = \drawProportionalLine{CG,GD}$ are also equal.
\stopCenterAlign

\qed
\stopProposition

\startProposition[title={Prop. XV. theor.},reference=prop:III.XV]
\problemNP{T}{he}{diameter is the greatest straight line in a circle: and, of all others, that which is nearest to the centre is greater than the more remote.}

\startsubproposition[title={Figure I.}]
\defineNewPicture[2/3]{
pair A, D, E, F, G, M, N;
numeric r;
r := 9/4u;
E := (0, 0);
A := (r, 0) shifted E;
D := (-r, 0) shifted E;
F := (dir(90-30)*r) shifted E;
G := (dir(90+30)*r) shifted E;
M := (dir(90-60)*r) shifted E;
N := (dir(90+60)*r) shifted E;
draw byAngle(N, E, G, byred, 0);
draw byAngle(G, E, F, byyellow, 0);
draw byAngle(F, E, M, byred, 0);
draw byLine(E, M, byyellow, 1, 0);
draw byLine(E, N, byyellow, 0, 0);
draw byLine(E, F, byblue, 1, 0);
draw byLine(E, G, black, 1, 0);
draw byLine(D, E, byred, 0, 0);
draw byLine(E, A, black, 0, 0);
draw byLine(F, G, byred, 1, 0);
draw byLine(M, N, byblue, 0, 0);
draw byCircleR(E, r, black, 0, 0, 0)(E);
}
\drawCurrentPictureInMargin
\startCenterAlign
The diameter \drawUnitLine{DE,EA} is $>$ any line \drawUnitLine{MN}.

For draw \drawUnitLine{EN} an \drawUnitLine{EM}.

Then $\drawUnitLine{EM} = \drawUnitLine{EA}$\\
and $\drawUnitLine{EN} = \drawUnitLine{DE}$,\\
$\therefore \drawUnitLine{EN} + \drawUnitLine{EM} = \drawUnitLine{DE,EA}$\\
but $\drawUnitLine{EN} + \drawUnitLine{EM} > \drawUnitLine{MN}$ \inprop[prop:I.XX]

$\therefore \drawUnitLine{DE,EA} > \drawUnitLine{MN}$.
\stopCenterAlign

Again, the line which is nearer the centre is greater than the one more remote.

First, let the given lines be \drawUnitLine{MN} and \drawUnitLine{FG}, which are at the same side of the centre and do not intersect;

\startCenterAlign
draw $\left\{
\vcenter{
\nointerlineskip\hbox{\drawUnitLine{EN},}
\nointerlineskip\hbox{\drawUnitLine{EM},}
\nointerlineskip\hbox{\drawUnitLine{EG},}
\nointerlineskip\hbox{\drawUnitLine{EF}.}
}
\right.$

In
\drawFromCurrentPicture{
draw byNamedAngle(NEG,GEF,FEM);
draw byNamedLineSeq(0)(MN,EM,EN);
}
and
\drawFromCurrentPicture{
draw byNamedAngle(GEF);
draw byNamedLineSeq(0)(FG,EF,EG);
},\\
$\drawUnitLine{EN} \mbox{ and } \drawUnitLine{EM} = \drawUnitLine{EG} \mbox{ and } \drawUnitLine{EF}$;\\
but $\drawAngle{NEG,GEF,FEM} > \drawAngle{GEF}$,

$\therefore \drawUnitLine{MN} > \drawUnitLine{EF}$ \inprop[prop:I.XXIV]
\stopCenterAlign
\stopsubproposition

\vfill\pagebreak

\startsubproposition[title={Figure II.}]
\defineNewPicture{
pair A, B, C, D, E, F, G, M, N, K, L, H;
numeric r;
r := 9/4u;
E := (0, 0);
A := (r, 0) shifted E;
D := (-r, 0) shifted E;
F := (dir(90-30)*r) shifted E;
G := (dir(90+30)*r) shifted E;
B := (dir(-90-60)*r) shifted E;
C := (dir(-90+60)*r) shifted E;
M := (dir(90-60)*r) shifted E;
N := (dir(90+60)*r) shifted E;
H := 1/2[B, C];
K := 1/2[F, G];
L := 1/2[M, N];
draw byLine(E, L, byyellow, 1, 0);
draw byLine(L, K, byred, 1, 0);
draw byLine(E, H, byblue, 1, 0);
draw byLine(F, G, byyellow, 0, 0);
draw byLine(M, N, black, 0, 0);
draw byLine(B, C, byred, 0, 0);
draw byLine(D, A, black, 0, 0);
draw byCircleR(E, r, black, 0, 0, 0)(E);
}
\drawCurrentPictureInMargin
Let the given lines be \drawUnitLine{BC} and \drawUnitLine{FG} which either are at different sides of the centre, or intersect; from the centre draw \drawUnitLine{EL,LK} and $\drawUnitLine{EH} \perp \drawUnitLine{FG} \mbox{ and } \drawUnitLine{BC}$,

\startCenterAlign
make $\drawUnitLine{EH} = \drawUnitLine{EL}$,\\
and draw $\drawUnitLine{MN} \perp \drawUnitLine{EL,LK}$.

Since \drawUnitLine{BC} and \drawUnitLine{MN} are equally distant from the centre, $\drawUnitLine{BC} = \drawUnitLine{MN}$ \inprop[prop:III.XIV];\\
but $\drawUnitLine{MN} > \drawUnitLine{FG}$ \inprop[prop:III.XV],

$\therefore \drawUnitLine{BC} > \drawUnitLine{FG}$.
\stopCenterAlign
\stopsubproposition

\qed
\stopProposition

\startProposition[title={Prop. XVI. theor.},reference=prop:III.XVI]
\defineNewPicture[1/5]{
pair A, B, C, D, E, F, G, H, K;
numeric r;
r :=2u;
D := (0, 0);
A := (0, -r);
B := (0, r);
C := (dir(190)*r) shifted D;
E := (4/3r, -r);
F := (4/3r, -1/3r);
G := 11/12[A, F];
H := (dir(angle(G-D))*r) shifted D;
K := (-r, -r);
draw byAngle(C, A, D, byyellow, 0);
draw byAngle(D, A, G, byblue, 0);
draw byAngle(G, A, E, byred, 0);
draw byAngleWithName(D, C, A, black, 0)(C);
draw byAngleWithName(A, G, D, black, 1)(G);
draw byLine(A, C, byred, 0, 0);
draw byLine(D, C, byblue, 0, 0);
draw byLine(D, H, byblue, 1, 0);
draw byLine(H, G, black, 1, 0);
draw byLine(A, G, byred, 1, 0);
draw byLine(G, F, black, 1, 0);
draw byLine(B, D, byyellow, 1, 0);
draw byLine(D, A, black, 0, 0);
draw byLineFull(E, K, byyellow, 0, 0)(B, D, 0, 0, -1);
draw byCircle(D, A, byblue, 0, 0, 0)(D);
}
\drawCurrentPictureInMargin
\problemNP{T}{he}{straight line \drawUnitLine{EK} drawn from the extremity of the diameter \drawUnitLine{BD,DA} of the circle perpendicular to it falls without the circle\\
And if any straight line \drawUnitLine{AG} by drawn from a point within that perpendicular to the point of contact, it cuts the circle.}

\startsubproposition[title={Part I.}]
If it be possible, let \drawUnitLine{AC}, which meets the circle again, be $\perp \drawUnitLine{DA}$, and draw \drawUnitLine{DC}.

\startCenterAlign
Then, because $\drawUnitLine{DA} = \drawUnitLine{DC}$,\\
$\drawAngle{CAD} = \drawAngle{C}$ \inprop[prop:I.V],\\
and $\therefore$ each of these angles is acute \inprop[prop:I.XVII]\\
but $\drawAngle{CAD} = \drawRightAngle$ (hyp.), which is absurd,

therefore \drawUnitLine{AC} drawn $\perp \drawUnitLine{DA}$ does not meet the circle again.
\stopCenterAlign
\stopsubproposition

\startsubproposition[title={Part II.}]
Let \drawUnitLine{EK} be $\perp \drawUnitLine{DA}$ and let \drawUnitLine{AG} be drawn from a point \drawLine{AG,HG,DH} between \drawUnitLine{EK} and the circle, which, if it be possible, does not cut the circle.

\startCenterAlign
Because $\drawAngle{DAG,GAE} = \drawRightAngle$,\\
$\therefore \drawAngle{DAG}$ is an acute angle;\\
suppose $\drawUnitLine{DH,HG} \perp \drawUnitLine{AG}$, drawn from the centre of the circle, it must fall at the side of \drawAngle{DAG} the acute angle.

$\therefore$ \drawAngle{G} which is supposed to be a right angle is $> \drawAngle{DAG}$,

$\therefore \drawUnitLine{DA} > \drawUnitLine{DH,HG}$;

but $\drawUnitLine{DH} = \drawUnitLine{DA}$,
\stopCenterAlign

and $\therefore \drawUnitLine{DH} > \drawUnitLine{DH,HG}$, a part greater than the whole, which is absurd. Therefore the point does not fall outside the circle, and therefore the straight line \drawUnitLine{AG} cuts the circle.
\stopsubproposition

\qed
\stopProposition

\startProposition[title={Prop. XVII. prob.},reference=prop:III.XVII]
\defineNewPicture[1/2]{
pair A, B, D, E, F;
numeric r[], a;
path cr[];
r1 := 6/4u;
r2 := 9/4u;
E := (0, 0);
cr1 := (fullcircle scaled 2r1) shifted E;
cr2 := (fullcircle scaled 2r2) shifted E;
A := (dir(50)*r2) shifted E;
D := (dir(50)*r1) shifted E;
F := cr2 intersectionpoint (D--D shifted (dir(angle(A-E) - 90)*r2));
B := cr1 intersectionpoint (E--F);
a := angle(B-E);
forsuffixes i=A, B, D, F:
i := ((i shifted -E) rotated -a) shifted E;
endfor;
draw byAngleWithName(A, B, E, byyellow, 0)(B);
draw byAngleWithName(F, D, E, byyellow, 0)(D);
draw byAngleWithName(F, E, A, byblue, 0)(E);
draw byLine(A, B, byblue, 0, 0);
draw byLine(F, D, byblue, 1, 0);
byLineDefine(A, D, byred, 0, 0);
byLineDefine(D, E, byred, 1, 0);
byLineDefine(F, B, black, 0, 0);
byLineDefine(B, E, black, 1, 0);
draw byNamedLineSeq(0)(AD,DE,BE,FB);
draw byCircleR(E, r1, byred, 0, 0, -1)(EI);
draw byCircleR(E, r2, byyellow, 0, 0, 0)(EII);
}
\drawCurrentPictureInMargin
\problemNP{T}{o}{draw a tangent to a given circle \drawCircle[middle][1/5]{EI} from a given point, either in or outside of its circumference.}

If a given point be in the circumference, as at \drawFromCurrentPicture[bottom]{
startTempScale(1/3);
draw byNamedLineSeq(0)(AB,BE);
stopTempScale;
}, it is plain that the straight line $\drawUnitLine{AB} \perp \drawUnitLine{BE}$ the radius, will be required tangent \inprop[prop:III.XVI].

But if the given point\drawFromCurrentPicture[bottom]{
startTempScale(1/3);
draw byNamedLineSeq(0)(DE,AD,AB);
stopTempScale;
} be outside of the circumference,

\startCenterAlign
draw \drawUnitLine{DE,AD} from it to the centre, cutting \circleEI;\\
and draw $\drawUnitLine{FD} \perp \drawUnitLine{DE}$,\\
describe \drawCircle[middle][1/6]{EII} concentric with \circleEI\ radius $= \drawUnitLine{AD,DE}$,\\ % improvement: These two lines were not originally there, but without them this proof has no sense
draw \drawUnitLine{BE,FB} to the centre from the point where \drawUnitLine{FD} falls on \circleEII\ circumference,\\
draw \drawUnitLine{AB} from the point where \drawUnitLine{BE,FB} cuts \circleEI,\\
%
Then \drawUnitLine{AB} will be the tangent required.\\
For in
\drawFromCurrentPicture[bottom]{
scaleFactor := 1/2;
draw byNamedLineSeq(0)(FD,FB,BE,DE);
}
and \drawLine[bottom]{BE,DE,AD,AB}
$\drawUnitLine{AD,DE} = \drawUnitLine{FB,BE}$, \drawAngle{E} common,\\
and $\drawUnitLine{DE} = \drawUnitLine{BE}$,\\
$\therefore \mbox{ \inprop[prop:I.IV] } \drawAngle{B} = \drawAngle{D} = \drawRightAngle$,\\
$\therefore \drawUnitLine{FD}$ is a tangent to \circleEI.
\stopCenterAlign

\qed
\stopProposition

\startProposition[title={Prop. XVIII. theor.},reference=prop:III.XVIII]
\defineNewPicture{
pair B, C, D, F, G;
numeric r;
r := 7/4u;
F := (0, 0);
C := (r, 0);
G := (r, 6/5r);
D := 6/5[C, G];
B := (dir(angle(G-F))*r) shifted F;
draw byCircle(F, C, byyellow, 0, 0, 0)(F);
draw byAngleWithName(F, C, G, byred, 0)(C);
draw byAngleWithName(C, G, F, byyellow, 0)(G);
byLineDefine(F, B, byred, 0, 0);
byLineDefine(B, G, byred, 1, 0);
byLineDefine(C, D, byblue, 1, 0);
byLineDefine(F, C, byblue, 0, 0);
draw byNamedLineSeq(0)(BG,FB,FC,CD);
}
\drawCurrentPictureInMargin
\problemNP[4]{I}{f}{a straight line \drawUnitLine{CD} be a tangent to a circle, the straight line \drawUnitLine{FC} drawn from the centre to the point of contact, is perpendicular to it.}

\startCenterAlign
For, if it be possible, let \drawUnitLine{FB,BG} be $\perp \drawUnitLine{CD}$,\\
then because $\drawAngle{G} = \drawRightAngle$, \drawAngle{C} is acute \inprop[prop:I.XVII]

$\therefore \drawUnitLine{FC} > \drawUnitLine{FB,BG}$ \inprop[prop:I.XIX];

but $\drawUnitLine{FC} = \drawUnitLine{FB}$,\\
and $\therefore \drawUnitLine{FB} > \drawUnitLine{FB,BG}$, a part greater than the whole, which is absurd.

$\therefore \drawUnitLine{FB,BG}$ is not $\perp \drawUnitLine{CD}$;
\stopCenterAlign

and in the same manner it can be demonstrated, that no other line except \drawUnitLine{FC} is perpendicular to \drawUnitLine{CD}.

\qed
\stopProposition

\startProposition[title={Prop. XIX. theor.},reference=prop:III.XIX]
\defineNewPicture{
pair A, C, E, F, G;
numeric r;
r := 7/4u;
G := (0, 0);
A := (-r, 0) shifted G;
C := (r, 0) shifted G;
E := (r, 6/5r) shifted G;
F := (-1/7r, 1/2r) shifted G;
draw byAngle(A, C, F, byblue, 0);
draw byAngle(F, C, E, byyellow, 0);
draw byLine(A, C, byyellow, 0, 0);
draw byLine(C, E, byblue, 0, 0);
draw byLine(C, F, byred, 1, 0);
draw byCircleR(G, r, byred, 0, 0, 0)(G);
}
\drawCurrentPictureInMargin
\problemNP{I}{f}{a straight line \drawUnitLine{CE} be a tangent to a circle, the straight line \drawUnitLine{AC}, drawn perpendicular to it from point of the contact, passes through the centre of the circle.}

For, if it be possible, let the centre be without \drawUnitLine{AC}, and draw \drawUnitLine{CF} from the supposed centre to the point of contact.

\startCenterAlign
Because $\drawUnitLine{CF} \perp \drawUnitLine{CE}$ \inprop[prop:III.XVIII]\\
$\therefore \drawAngle{FCE} = \drawRightAngle$, a right angle;\\
but $\drawAngle{ACF,FCE} = \drawRightAngle$ (hyp.),

and $\therefore \drawAngle{FCE} = \drawAngle{ACF,FCE}$, a part equal to the whole, which is absurd.
\stopCenterAlign

Therefore the assumed point is not the centre; and in the same manner it can be demonstrated, that no other point without \drawUnitLine{AC} is the centre.

\qed
\stopProposition

\startProposition[title={Prop. XX. theor.},reference=prop:III.XX]
\problemNP[4]{T}{he}{angle at the centre of a circle is double the angle at the circumference, when they have the same part of the circumference for their base.}\unskip

\defineNewPicture{
pair A, E, F, C;
r := 7/4u;
E := (0, 0);
A := (dir(80)*r) shifted E;
F := (dir(80 + 180)*r) shifted E;
C := (dir(-30)*r) shifted E;
draw byAngle(C, A, F, byyellow, 0);
draw byAngle(C, E, F, byblue, 0);
draw byAngleWithName(E, C, A, byred, 0)(C);
draw byLine(A, C, black, 0, 1);
draw byLine(E, C, black, 0, 0);
draw byLine(E, F, byred, 1, 0);
draw byLine(E, A, byred, 0, 0);
draw byCircleR(E, r, byblue, 0, 0, 0)(E);
}
\drawCurrentPictureInMargin
\startsubproposition[title={Figure I.}]
\startCenterAlign
Let the centre of the circle be on \drawUnitLine{EF,EA}\\
a side of \drawAngle{CAF}.

Because $\drawUnitLine{EC} = \drawUnitLine{EA}$,\\
$\drawAngle{CAF} = \drawAngle{C}$ \inprop[prop:I.V].

But $\drawAngle{CEF} = \drawAngle{CAF} + \drawAngle{C}$,\\
of $\drawAngle{CEF} = \mbox{ twice } \drawAngle{CAF}$ \inprop[prop:I.XXXII].
\stopCenterAlign
\stopsubproposition

\defineNewPicture{
pair A, E, F, C, B;
r := 7/4u;
E := (0, 0);
A := (dir(80)*r) shifted E;
F := (dir(80 + 180)*r) shifted E;
B := (dir(185)*r) shifted E;
C := (dir(-30)*r) shifted E;
draw byAngle(C, A, F, byyellow, 0);
draw byAngle(C, E, F, byblue, 0);
draw byAngleWithName(E, C, A, byyellow, 0)(C);
draw byAngle(B, A, F, byred, 0);
draw byAngle(B, E, F, black, 0);
draw byAngleWithName(E, B, A, byred, 0)(B);
draw byLine(A, C, black, 0, 1);
draw byLine(E, C, black, 0, 1);
draw byLine(A, B, black, 0, 1);
draw byLine(E, B, black, 0, 1);
draw byLine(A, F, black, 0, 0);
draw byCircleR(E, r, byblue, 0, 0, 0)(E);
}
\drawCurrentPictureInMargin
\startsubproposition[title={Figure II.}]
\startCenterAlign
Let the centre be within \drawAngle{BAF,CAF}, the angle at the circumference;

draw \drawUnitLine{AF} from the angular point through the centre of the circle;\\
then $\drawAngle{B} = \drawAngle{BAF}$, and $\drawAngle{C} = \drawAngle{CAF}$, because of the equality of the sides \inprop[prop:I.V].

Hence $\drawAngle{BAF} + \drawAngle{B} + \drawAngle{CAF} + \drawAngle{C} = \mbox{ twice } \drawAngle{BAF,CAF}$.

But $\drawAngle{BEF} = \drawAngle{BAF} + \drawAngle{B}$,\\
and $\drawAngle{CEF} = \drawAngle{CAF} + \drawAngle{C}$,\\
$\therefore \drawAngle{BEF,CEF} = \mbox{ twice } \drawAngle{BAF,CAF}$.
\stopCenterAlign
\stopsubproposition

\vfill\pagebreak

\defineNewPicture{
pair E, C, F, G, D;
r := 7/4u;
E := (0, 0);
F := (dir(80 + 180)*r) shifted E;
C := (dir(-30)*r) shifted E;
D := (dir(30)*r) shifted E;
G := (dir(30 + 180)*r) shifted E;
draw byAngle(G, E, F, byblue, 0);
draw byAngle(F, E, C, byyellow, 0);
draw byAngle(G, D, F, black, 0);
draw byAngle(F, D, C, byred, 0);
draw byLine(D, C, black, 0, 1);
draw byLine(D, F, black, 0, 1);
draw byLine(E, C, black, 0, 1);
draw byLine(E, F, black, 0, 1);
draw byLine(D, G, byred, 0, 0);
draw byCircleR(E, r, byblue, 0, 0, 0)(E);
}
\drawCurrentPictureInMargin
\startsubproposition[title={Figure III.}]
\startCenterAlign
Let the centre be without \drawAngle{FDC}\\
and draw \drawUnitLine{DG}, the diameter.

Because $\drawAngle{GEF,FEC} = \mbox{ twice } \drawAngle{GDF,FDC}$;\\
and $\drawAngle{GEF} = \mbox{ twice } \drawAngle{GDF}$ (case I.);

$\therefore \drawAngle{FEC} = \mbox{ twice } \drawAngle{FDC}$.
\stopCenterAlign
\stopsubproposition

\qed
\stopProposition

\startProposition[title={Prop. XXI. theor.},reference=prop:III.XXI]
\problemNP{T}{he}{angles in the same segment of a circle are equal.}

\defineNewPicture{
pair A, B, D, E, F;
numeric r;
r := 2u;
F := (0, 0);
B := (dir(-90 - 50)*r) shifted F;
D := (dir(-90 + 50)*r) shifted F;
A := (dir(90 + 25)*r) shifted F;
E := (dir(90 - 35)*r) shifted F;
draw byAngleWithName(B, A, D, byred, 0)(A);
draw byAngleWithName(B, E, D, byblue, 0)(E);
draw byAngleWithName(B, F, D, byyellow, 0)(F);
byLineDefine(B, F, byblue, 0, 0);
byLineDefine(D, F, byred, 0, 0);
draw byNamedLineSeq(0)(BF,DF);
draw byLine(B, D, black, 1, 0);
draw byLine(B, A, black, 0, 1);
draw byLine(B, E, black, 0, 1);
draw byLine(D, A, black, 0, 1);
draw byLine(D, E, black, 0, 1);
draw byCircleR(F, r, byblue, 0, 0, 0)(F);
}\drawCurrentPictureInMargin
\startsubproposition[title={Figure I.}]
Let the segment be greater than a semicircle, and draw \drawUnitLine{DF} and \drawUnitLine{BF} to the centre.

\startCenterAlign
$\drawAngle{E} = \mbox{ twice } \drawAngle{A} \mbox{ or twice } = \drawAngle{E}$ \inprop[prop:III.XX];

$\therefore \drawAngle{A} = \drawAngle{E}$
\stopCenterAlign
\stopsubproposition

\defineNewPicture{
pair A, B, D, E, F, G;
numeric r;
path cr;
r := 2u;
F := (0, 0);
cr := (fullcircle scaled 2r) shifted F;
B := (dir(90 + 85)*r) shifted F;
D := (dir(90 - 85)*r) shifted F;
A := (dir(90 + 25)*r) shifted F;
E := (dir(90 - 35)*r) shifted F;
G := (dir(-90 + 20)*r) shifted F;
draw byAngle(B, A, G, byyellow, 0);
draw byAngle(G, A, D, byred, 0);
draw byAngle(B, E, G, byblue, 0);
draw byAngle(G, E, D, black, 0);
draw byFilledCircleSegment(F, r, xpart(cr intersectiontimes (F -- 2[F, B])), xpart(cr intersectiontimes (F -- 2[F, G])), byblue)(BG);
draw byFilledCircleSegment(F, r, xpart(cr intersectiontimes (F -- 2[F, G])), 8 + xpart(cr intersectiontimes (F -- 2[F, D])), byyellow)(GD);
draw byLine(G, A, byblue, 0, 0);
draw byLine(G, E, byred, 0, 0);
draw byLine(B, D, black, 1, 0);
draw byLine(B, A, black, 0, 1);
draw byLine(B, E, black, 0, 1);
draw byLine(D, A, black, 0, 1);
draw byLine(D, E, black, 0, 1);
draw byCircleR(F, r, byblue, 0, 0, 0)(F);
}\drawCurrentPictureInMargin
\startsubproposition[title={Figure II.}]
Let the segment be a semicircle, or less than a semicircle, draw \drawUnitLine{GA} the diameter, also draw \drawUnitLine{GE}.

\startCenterAlign
$\drawAngle{BAG} = \drawAngle{BEG}$ and $\drawAngle{GAD} = \drawAngle{GED}$ (case I.)

$\therefore \drawAngle{BAG,GAD} = \drawAngle{BEG,GED}$.
\stopCenterAlign
\stopsubproposition


\qed
\stopProposition

\startProposition[title={Prop. XXII. theor.},reference=prop:III.XXII]
\defineNewPicture{
pair A, B, C, D, E;
numeric r;
r := 7/4u;
E := (0, 0);
A := (dir(80)*r) shifted E;
B := (dir(10)*r) shifted E;
C := (dir(-100)*r) shifted E;
D := (dir(150)*r) shifted E;
draw byAngle(D, A, C, byred, 0);
draw byAngle(C, A, B, byblue, 0);
draw byAngle(A, B, D, byyellow, 0);
draw byAngle(D, B, C, byred, 0);
draw byAngle(B, C, A, black, 0);
draw byAngle(A, C, D, byyellow, 0);
draw byAngle(C, D, B, byblue, 0);
draw byAngle(B, D, A, black, 0);
draw byLine(A, B, black, 0, 1);
draw byLine(B, C, black, 0, 1);
draw byLine(C, D, black, 0, 1);
draw byLine(D, A, black, 0, 1);
draw byLine(A, C, byred, 0, 0);
draw byLine(B, D, black, 0, 0);
draw byCircleR(E, r, byred, 0, 0, 1/2)(E);
}
\drawCurrentPictureInMargin
\problemNP{T}{he}{opposite angles \drawAngle{DAC,CAB} and \drawAngle{BCA,ACD}, \drawAngle{CDB,BDA} and \drawAngle{ABD,DBC} of any quadrilateral figure inscribed in a circle, are together equal to two right angles.}

\startCenterAlign
Draw \drawUnitLine{AC} and \drawUnitLine{BD} the diagonals;\\
and because angles in the same segment are equal\\
$\drawAngle{CDB} = \drawAngle{CAB}$,\\
and $\drawAngle{DAC} = \drawAngle{DBC}$;

add \drawAngle{BCA,ACD} to both.\\
$\drawAngle{DAC,CAB} + \drawAngle{BCA,ACD} = \drawAngle{BCA,ACD} + \drawAngle{CDB} + \drawAngle{DBC} = \mbox{ two right angles }$ \inprop[prop:I.XXXII].

In like manner it may be shown that,\\
$\drawAngle{CDB,BDA} + \drawAngle{ABD,DBC} = \drawTwoRightAngles$.
\stopCenterAlign

\qed
\stopProposition

\startProposition[title={Prop. XXIII. theor.},reference=prop:III.XXIII]
\defineNewPicture{
pair A, B, C, D, M, N;
numeric r[], t[];
path cr[];
r1 := 14/6u;
r2 := 15/6u;
M := (0, 0);
N := (0, 1/3r1);
cr1 := (fullcircle scaled 2r1) shifted M;
cr2 := (fullcircle scaled 2r2) shifted N;
t1 := xpart(cr1 intersectiontimes (subpath (-2, 2) of cr2));
t2 := xpart(cr1 intersectiontimes (subpath (2, 6) of cr2));
t3 := xpart(cr2 intersectiontimes (subpath (-2, 2) of cr1));
t4 := xpart(cr2 intersectiontimes (subpath (2, 6) of cr1));
A := point t1 of cr1;
B := point t2 of cr1;
C := point 3/2 of cr1;
D := cr2 intersectionpoint (1/2[A,C]--2[A,C]);
draw byAngleWithName(A, C, B, byyellow, 0)(C);
draw byAngleWithName(A, D, B, byblue, 0)(D);
draw byLineFull(A, D, byred, 0, 0)(B, D, 1, 0, 0);
draw byLineFull(B, C, byblue, 0, 0)(A, C, 1, 0, 0);
draw byLineFull(B, D, byyellow, 0, 0)(A, D, 1, 0, 0);
draw byLineFull(A, B, black, 0, 0)(A, B, 0, 0, 1);
draw byArc(M, A, B, r1, byred, 0, 0, 0, 1)(M);
draw byArc(N, A, B, r2, byblue, 0, 0, 0, 1)(N);
}
\drawCurrentPictureInMargin
\problemNP{U}{pon}{the same straight line, and upon the same side of it, two similar segments of circles cannot be constructed, which do not coincide.}

\startCenterAlign
For if it be possible, let two similar segments\\
\drawFromCurrentPicture[bottom]{
draw byNamedLine(AB);
draw byNamedArcExact(N);
}
and
\drawFromCurrentPicture[bottom]{
draw byNamedLine(AB);
draw byNamedArcExact(M);
}
be constructed;

draw any right line \drawUnitLine{AD} cutting both the segments,

draw \drawUnitLine{BC} and \drawUnitLine{BD}.

Because the segments are similar,\\
$\drawAngle{C} = \drawAngle{D}$ \inprop[prop:III.X],

but $\drawAngle{C} > \drawAngle{D}$ \inprop[prop:III.XVI]\\
which is absurd;

therefore no point in either of the segments falls without the other, and therefore the segments coincide.
\stopCenterAlign

\qed
\stopProposition

\startProposition[title={Prop. XXIV. theor.},reference=prop:III.XXIV]
\defineNewPicture{
pair A, B, C, D, M, N, d;
numeric r, t[];
path cr[];
r := 5/2u;
t1 := 2-1;
t2 := 2+1;
d := (0, -3/2u);
M := (0, 0);
N := M shifted d;
cr1 := ((subpath (t1, t2) of fullcircle) scaled 2r) shifted M;
cr2 := ((subpath (t1, t2) of fullcircle) scaled 2r) shifted N;
A := point length(cr1) of cr1;
B := point 0 of cr1;
C := point length(cr2) of cr2;
D := point 0 of cr2;
draw byFilledCircleSegment (M, r, t1, t2, byred)(M);
draw byLineFull(A, B, black, 0, 0)(A, B, 0, 0, -1);
draw byArc(M, B, A, r, byblue, 0, 0, 1/2, 1)(M);
draw byFilledCircleSegment (N, r, t1, t2, byyellow)(N);
draw byLineFull(C, D, byblue, 0, 0)(C, D, 0, 0, -1);
draw byArc(N, D, C, r, byred, 0, 0, 1/2, 1)(N);
}
\drawCurrentPictureInMargin
\problemNP{S}{imilar}{segments
\drawFromCurrentPicture[bottom]{
draw byNamedFilledCircleSegment(M);
draw byNamedLine(AB);
draw byNamedArcExact(M);
}
and
\drawFromCurrentPicture[bottom]{
draw byNamedFilledCircleSegment(N);
draw byNamedLine(CD);
draw byNamedArcExact(N);
}
, of circles upon equal straight lines (\drawUnitLine{AB} and \drawUnitLine{CD}) are each equal to the other.}

\startCenterAlign
For, if
\drawFromCurrentPicture[bottom]{draw byNamedFilledCircleSegment(N);}
be so applied to
\drawFromCurrentPicture[bottom]{draw byNamedFilledCircleSegment(M);},\\
that \drawUnitLine{CD} may fall on \drawUnitLine{AB},\\
the extremities of \drawUnitLine{CD} may be on the extremities \drawUnitLine{AB}\\
and \drawArc{M} at the same side as \drawArc{N};\\
because $\drawUnitLine{CD} = \drawUnitLine{AB}$,\\
\drawUnitLine{CD} must wholly coincide with \drawUnitLine{AB};
\stopCenterAlign

\noindent and the similar being then upon the same straight line and at the same side of it, must also coincide \inprop[prop:III.XXIII], and therefore equal.

\qed
\stopProposition

\startProposition[title={Prop. XXV. prob.},reference=prop:III.XXV]
\defineNewPicture{
pair A, B, C, D, E, F, O;
numeric r;
r := 7/4u;
O := (0, 0);
A := (dir(-20)*r) shifted O;
B := (dir(85)*r) shifted O;
C := (dir(180)*r) shifted O;
D := 1/2[A, B];
E := 1/2[B, C];
F = whatever[D, D shifted ((A-B) rotated 90)] = whatever[E, E shifted ((B-C) rotated 90)];
byLineDefine(D, F, byred, 0, 0);
byLineDefine(E, F, byyellow, 0, 0);
draw byNamedLineSeq(0)(DF,EF);
draw byLine(A, B, black, 0, 0);
draw byLine(B, C, byblue, 0, 0);
draw byArcBE(O, -6/5, 5, r, byblue, 0, 0, 1/2, 0)(O);
}
\drawCurrentPictureInMargin
\problemNP{A}{segment}{of a circle being given, to describe the circle of which it is the segment.}

\startCenterAlign
From any point in the segment draw \drawUnitLine{BC} and \drawUnitLine{AB},\\
bisect them, and from the points of bisection\\
draw $\drawUnitLine{EF} \perp \drawUnitLine{BC}$\\
and $\drawUnitLine{DF} \perp \drawUnitLine{AB}$\\
where they meet is the centre of the circle.
\stopCenterAlign

Because \drawUnitLine{BC} terminated in the circle is bisected perpendicularly by \drawUnitLine{EF}, it passes through the centre \inprop[prop:III.I], likewise \drawUnitLine{DF} passes through the centre, therefore the centre is in the intersection of these perpendiculars.

\qed
\stopProposition

\startProposition[title={Prop. XXVI. theor.},reference=prop:III.XXVI]
\defineNewPicture{
pair A, B, C, D, E, F, G, H, d;
numeric r, t[];
path cr[];
r := 7/4u;
t1 := 5;
t2 := 7;
G := (0, 0);
cr1 := (fullcircle scaled 2r) shifted G;
A := point 3/2 of cr1;
B := point t1 of cr1;
C := point t2 of cr1;
d := (0, -5/2r);
H := G shifted d;
cr2 := (fullcircle scaled 2r) shifted H;
D := point 5/2 of cr2;
E := point t1 of cr2;
F := point t2 of cr2;
draw byAngleWithName(B, A, C, byred, 0)(A);
draw byAngleWithName(B, G, C, byyellow, 0)(G);
draw byFilledCircleSegment (G, r, t1, t2, byyellow)(G);
draw byLine(B, C, black, 0, 0);
draw byLine(A, B, black, 0, 1);
draw byLine(A, C, black, 0, 1);
byLineDefine(B, G, byblue, 0, 0);
byLineDefine(C, G, byred, 0, 0);
draw byNamedLineSeq(0)(BG,CG);
draw byArc(G, C, B, r, byblue, 0, 0, 0, 0)(G);
draw byArc(G, B, C, r, black, 0, 0, 0, 0)(Gb);
byCircleDefineR(G, r, byblue, 0, 0, 0)(G);
draw byAngleWithName(E, D, F, byblue, 0)(D);
draw byAngleWithName(E, H, F, black, 0)(H);
draw byFilledCircleSegment (H, r, t1, t2, byyellow)(H);
draw byLine(E, F, black, 1, 0);
draw byLine(D, E, black, 0, 1);
draw byLine(D, F, black, 0, 1);
byLineDefine(E, H, byblue, 1, 0);
byLineDefine(F, H, byred, 1, 0);
draw byNamedLineSeq(0)(EH,FH);
draw byArc(H, F, E, r, byred, 0, 0, 0, 0)(H);
draw byArc(H, E, F, r, black, 0, 0, 0, 0)(Hb);
byCircleDefineR(H, r, byred, 0, 0, 0)(H);
}
\drawCurrentPictureInMargin
\problemNP[4]{I}{n}{equal circles \drawCircle[middle][1/5]{G} and \drawCircle[middle][1/5]{H} the arcs \drawArc{Gb}, \drawArc{Hb} on which stand equal angles, whether at the centre of circumference, are equal.}

\startCenterAlign
For, let $\drawAngle{G} = \drawAngle{H}$ at the centre,\\
draw \drawUnitLine{BC} and \drawUnitLine{EF}.\\
Then since $\circleG = \circleH$,\\
\drawLine[bottom]{BG,CG,BC}
and
\drawLine[bottom]{EH,FH,EF}
have\\
$\drawUnitLine{BG} = \drawUnitLine{CG} = \drawUnitLine{EH} = \drawUnitLine{FH}$,\\
and $\drawAngle{G} = \drawAngle{H}$,\\
$\therefore \drawUnitLine{BC} = \drawUnitLine{EF}$ \inprop[prop:I.IV].\\
But $\drawAngle{A} = \drawAngle{D}$ \inprop[prop:III.XX];\\
$\therefore$
\drawFromCurrentPicture{
startTempScale(1/5);
draw byNamedArc(G);
draw byNamedLine(BC);
stopTempScale;
}
and
\drawFromCurrentPicture{
startTempScale(1/5);
draw byNamedArc(H);
draw byNamedLine(EF);
stopTempScale;
}
are similar \indef[def:III.XI];\\
they are also equal \inprop[prop:III.XXIV]
\stopCenterAlign

If therefore the equal segments be taken from the equal circles, the remaining segments will be equal;

\startCenterAlign
hence $
\drawFromCurrentPicture{
draw byNamedFilledCircleSegment(G);
}
=
\drawFromCurrentPicture{
draw byNamedFilledCircleSegment(H);
}
$ \inax[ax:III];\\
and $\therefore \drawArc{Gb} = \drawArc{Hb}$.
\stopCenterAlign

But if the given equal angles be at the circumference, it is evident that the angles at the centre, being double of those at the circumference, are also equal, and therefore the arcs on which they stand are equal.

\qed
\stopProposition

\startProposition[title={Prop. XXVII. theor.},reference=prop:III.XXVII]
\defineNewPicture[1/2]{
pair A, B, C, D, E, F, G, H, K, d;
numeric r, t[];
path cr[];
r := 7/4u;
t1 := 5;
t2 := 7;
t3 := 13/2;
G := (0, 0);
cr1 := (fullcircle scaled 2r) shifted G;
A := point 3/2 of cr1;
B := point t1 of cr1;
C := point t2 of cr1;
K := point t3 of cr1;
d := (0, -5/2r);
H := G shifted d;
cr2 := (fullcircle scaled 2r) shifted H;
D := point 3/2 of cr2;
E := point t1 of cr2;
F := point t3 of cr2;
draw byAngle(B, A, K, byyellow, 0);
draw byAngle(B, G, K, byyellow, 0);
draw byAngle(K, A, C, byblue, 0);
draw byAngle(K, G, C, byblue, 0);
draw byLine(A, K, black, 0, 1);
draw byLine(A, B, black, 0, 1);
draw byLine(A, C, black, 0, 1);
byLineDefine(G, B, black, 0, 1);
byLineDefine(G, C, black, 0, 1);
draw byNamedLineSeq(0)(GB,GC);
draw byLine(G, K, black, 0, 1);
draw byArc(G, C, B, r, byred, 0, 0, 0, 0)(G);
draw byArc(G, B, K, r, black, 0, 0, 0, 0)(GbI);
draw byArc(G, K, C, r, byred, 1, 0, 0, 0)(GbII);
byCircleDefineR(G, r, byred, 0, 0, 0)(G);
draw byAngleWithName(E, D, F, byred, 0)(D);
draw byAngleWithName(E, H, F, byred, 0)(H);
draw byLine(D, F, black, 0, 1);
draw byLine(D, E, black, 0, 1);
byLineDefine(H, E, black, 0, 1);
byLineDefine(H, F, black, 0, 1);
draw byNamedLineSeq(0)(HE,HF);
draw byArc(H, F, E, r, byblue, 0, 0, 0, 0)(H);
draw byArc(H, E, F, r, black, 1, 0, 0, 0)(Hb);
byCircleDefineR(H, r, byblue, 0, 0, 0)(H);
}
\drawCurrentPictureInMargin
\problemNP[2]{I}{n}{equal circles \drawCircle[middle][1/3]{G} and \drawCircle[middle][1/3]{H} the angles \drawAngle{BAK} and \drawAngle{D} which stand upon equal arches are equal, whether they be at the centres or at circumferences.}

\startCenterAlign
For if it be possible, let one of them\\
\drawAngle{D} be greater than the other \drawAngle{BAK}\\
and make $\drawAngle{BAK,KAC} = \drawAngle{D}$

$\therefore \drawArc{GbI,GbII} = \drawArc{Hb}$ \inprop[prop:III.XXVI]

but $\drawArc{GbI} = \drawArc{Hb}$ (hyp.)\\
$\therefore \drawArc{GbI} = \drawArc{GbI,GbII}$ a part equal to the whole, which is absurd;

$\therefore$ neither angle is greater than the other,

and $\therefore$ they are equal.
\stopCenterAlign

\qed
\stopProposition

\startProposition[title={Prop. XXVIII. theor.},reference=prop:III.XXVIII]
\defineNewPicture[1/2]{
pair A, B, D, E, K, L, d;
numeric r, t[];
path cr[];
r := 7/4u;
t1 := 5;
t2 := 7;
K := (0, 0);
cr1 := (fullcircle scaled 2r) shifted K;
A := point t1 of cr1;
B := point t2 of cr1;
d := (0, -5/2r);
L := K shifted d;
cr2 := (fullcircle scaled 2r) shifted L;
D := point t1 of cr2;
E := point t2 of cr2;
draw byAngleWithName(A, K, B, byred, 0)(K);
draw byLine(A, B, byred, 0, 0);
byLineDefine(A, K, black, 0, 0);
byLineDefine(B, K, byblue, 0, 0);
draw byNamedLineSeq(0)(AK,BK);
draw byArc(K, B, A, r, byyellow, 0, 0, 0, 0)(K);
draw byArc(K, A, B, r, byblue, 0, 0, 0, 0)(Kb);
byCircleDefineR(K, r, byyellow, 0, 0, 0)(K);
draw byAngleWithName(D, L, E, byyellow, 0)(L);
draw byLine(D, E, byred, 1, 0);
byLineDefine(D, L, black, 1, 0);
byLineDefine(E, L, byblue, 1, 0);
draw byNamedLineSeq(0)(DL,EL);
draw byArc(L, E, D, r, black, 0, 0, 0, 0)(L);
draw byArc(L, D, E, r, byred, 0, 0, 0, 0)(Lb);
byCircleDefineR(L, r, black, 0, 0, 0)(L);
}
\drawCurrentPictureInMargin
\problemNP{I}{n}{equal circles \drawCircle[middle][1/3]{K} and \drawCircle[middle][1/3]{L} equal chords \drawUnitLine{AB}, \drawUnitLine{DE} cut off equal arches.}

\startCenterAlign
From the centres of the equal circles,\\
draw \drawUnitLine{AK}, \drawUnitLine{BK} and \drawUnitLine{DL}, \drawUnitLine{EL};

and because $\circleK = \circleL$\\
$\drawUnitLine{AK}, \drawUnitLine{BK} = \drawUnitLine{DL}, \drawUnitLine{EL}$\\
also $\drawUnitLine{AB} = \drawUnitLine{DE}$ (hyp.)

$\therefore \drawAngle{K} = \drawAngle{L}$

$\therefore \drawArc{Kb} = \drawArc{Lb}$ \inprop[prop:III.XXVI]

and $\therefore \drawArc[bottom][1/3]{K} = \drawArc[bottom][1/3]{L}$ \inax[ax:III]
\stopCenterAlign

\qed
\stopProposition

\startProposition[title={Prop. XXIX. theor.},reference=prop:III.XXIX]
\defineNewPicture[1/2]{
pair A, B, D, E, K, L, d;
numeric r, t[];
path cr[];
r := 7/4u;
t1 := 5;
t2 := 7;
K := (0, 0);
cr1 := (fullcircle scaled 2r) shifted K;
A := point t1 of cr1;
B := point t2 of cr1;
d := (0, -5/2r);
L := K shifted d;
cr2 := (fullcircle scaled 2r) shifted L;
D := point t1 of cr2;
E := point t2 of cr2;
draw byAngleWithName(A, K, B, byred, 0)(K);
draw byLine(A, B, byred, 0, 0);
byLineDefine(A, K, black, 0, 0);
byLineDefine(B, K, byblue, 0, 0);
draw byNamedLineSeq(0)(AK,BK);
draw byArc(K, B, A, r, byyellow, 0, 0, 0, 0)(K);
draw byArc(K, A, B, r, byblue, 0, 0, 0, 0)(Kb);
byCircleDefineR(K, r, byyellow, 0, 0, 0)(K);
draw byAngleWithName(D, L, E, byyellow, 0)(L);
draw byLine(D, E, byred, 1, 0);
byLineDefine(D, L, black, 1, 0);
byLineDefine(E, L, byblue, 1, 0);
draw byNamedLineSeq(0)(DL,EL);
draw byArc(L, E, D, r, black, 0, 0, 0, 0)(L);
draw byArc(L, D, E, r, byred, 0, 0, 0, 0)(Lb);
byCircleDefineR(L, r, black, 0, 0, 0)(L);
}
\drawCurrentPictureInMargin
\problemNP{I}{n}{equal circles \drawCircle[middle][1/3]{K} and \drawCircle[middle][1/3]{L} the chords \drawUnitLine{AB}, \drawUnitLine{DE} which subtend equal arcs are equal.}

\startCenterAlign
If the equal arcs be semicircles the proposition is evident. But if not,

Let \drawUnitLine{AK}, \drawUnitLine{BK} and \drawUnitLine{DL}, \drawUnitLine{EL} be drawn to the centres;

because $\drawArc{Kb} = \drawArc{Lb}$ (hyp.)\\
and $\drawAngle{K} = \drawAngle{L}$ \inprop[prop:III.XXVII];

but $\drawUnitLine{AK} \mbox{ and } \drawUnitLine{BK} = \drawUnitLine{DL} \mbox{ and } \drawUnitLine{EL}$

$\therefore \drawUnitLine{AB} = \drawUnitLine{DE}$ \inprop[prop:III.XXVII];

but these are the chords subtending the equal arcs.
\stopCenterAlign

\qed
\stopProposition

\startProposition[title={Prop. XXX. prob.},reference=prop:III.XXX]
\defineNewPicture{
pair A, B, C, D, E;
numeric r, t[];
path cr;
r := 9/4u;
t1 := -1/2;
t2 := 4 + 1/2;
t3 := 2;
E := (0, 0);
cr := (fullcircle scaled 2r) shifted E;
A := point t2 of cr;
B := point t1 of cr;
D := point t3 of cr;
C := 1/2[A, B];
draw byAngle(A, C, D, byblue, 0);
draw byAngle(D, C, B, byred, 0);
draw byLine(D, A, byblue, 0, 0);
draw byLine(D, C, byyellow, 0, 0);
draw byLine(D, B, byblue, 1, 0);
draw byLine(A, C, black, 0, 0);
draw byLine(C, B, black, 1, 0);
draw byArc(E, B, D, r, byred, 1, 0, 0, 0)(Er);
draw byArc(E, D, A, r, byred, 0, 0, 0, 0)(El);
}
\drawCurrentPictureInMargin
\problemNP{T}{o}{bisect a given arc \drawArc[middle][1/5]{El,Er}.}

\startCenterAlign
Draw \drawUnitLine{AC,CB};\\
make $\drawUnitLine{AC} = \drawUnitLine{CB}$,\\
draw $\drawUnitLine{DC} \perp \drawUnitLine{AC,CB}$, and it bisects the arc.

Draw \drawUnitLine{DA} and \drawUnitLine{DB}.

$\drawUnitLine{AC} = \drawUnitLine{CB}$ (const.)\\
\drawUnitLine{DC} is common,\\
and $\drawAngle{ACD} = \drawAngle{DCB}$ (const.)

$\therefore \drawUnitLine{DA} = \drawUnitLine{DB}$ \inprop[prop:I.IV]\\
$
\drawFromCurrentPicture{
startGlobalRotation(-arcAngle.El);
draw byNamedArc(El);
stopGlobalRotation;
}
=
\drawFromCurrentPicture{
startGlobalRotation(-arcAngle.Er);
draw byNamedArc(Er);
stopGlobalRotation;
}
$ \inprop[prop:III.XXVII],\\
and therefore the given arc is bisected.
\stopCenterAlign

\qed
\stopProposition

\startProposition[title={Prop. XXXI. theor.},reference=prop:III.XXXI]
\problemNP{I}{n}{a circle the angle in a semicircle is a right angle, the angle in a segment greater than a semicircle is acute, and the angle in a segment less than a semicircle is obtuse.}

\defineNewPicture{
pair A, B, C, E;
numeric r;
r := 7/4u;
E := (0, 0);
A := (dir(110)*r) shifted E;
B := (dir(180)*r) shifted E;
C := (dir(0)*r) shifted E;
draw byAngleWithName(A, B, C, byred, 0)(B);
draw byAngleWithName(B, C, A, byblue, 0)(C);
draw byAngle(E, A, B, byyellow, 0);
draw byAngle(C, A, E, black, 0);
draw byLine(A, E, byred, 0, 0);
draw byLine(A, B, black, 0, 1);
draw byLine(A, C, black, 0, 1);
draw byLine(B, E, byblue, 0, 0);
draw byLine(E, C, black, 0, 0);
draw byCircleR(E, r, black, 0, 0, 0)(E);
}
\drawCurrentPictureInMargin
\startsubproposition[title={Figure I.}]
\startCenterAlign
The angle \drawAngle{EAB,CAE} in a semicircle is a right angle.

Draw \drawUnitLine{AE} and \drawUnitLine{BE,EC}\\
$\drawAngle{B}=\drawAngle{EAB}$ and $\drawAngle{C} = \drawAngle{CAE}$ \inprop[prop:I.V]\\
$\drawAngle{C} + \drawAngle{B} = \drawAngle{EAB,CAE} = \mbox{ the half of } \drawTwoRightAngles= \drawRightAngle$ \inprop[prop:I.XXXII]
\stopCenterAlign
\stopsubproposition

\defineNewPicture{
pair A, B, C, E, D;
numeric r;
r := 7/4u;
E := (0, 0);
A := (dir(110)*r) shifted E;
B := (dir(180 + 30)*r) shifted E;
C := (dir(0 + 30)*r) shifted E;
D := (dir(0 - 30)*r) shifted E;
draw byAngle(B, A, D, byblue, 0);
draw byAngle(D, A, C, byred, 0);
draw byLine(A, D, black, 0, 1);
draw byLine(A, B, black, 0, 1);
draw byLine(B, D, black, 0, 1);
draw byLine(A, C, byblue, 0, 0);
draw byLine(B, C, byred, 0, 0);
draw byCircleR(E, r, byblue, 0, 0, 0)(E);
}
\drawCurrentPictureInMargin
\startsubproposition[title={Figure II.}]
\startCenterAlign
The angle \drawAngle{BAD} in a segment greater than a semicircle is acute

Draw \drawUnitLine{BC} the diameter, and \drawUnitLine{AC}\\
$\therefore \drawAngle{BAD,DAC} = \mbox{ a right angle}$\\
$\therefore$ \drawAngle{BAD} is acute.
\stopCenterAlign
\stopsubproposition

\vfill\pagebreak

\defineNewPicture{
pair A, B, C, E, D;
numeric r;
r := 7/4u;
E := (0, 0);
A := (dir(140)*r) shifted E;
B := (dir(-110)*r) shifted E;
C := (dir(5)*r) shifted E;
D := (dir(80)*r) shifted E;
draw byAngleWithName(A, D, C, byred, 0)(D);
draw byAngleWithName(C, B, A, byyellow, 0)(B);
draw byLine(A, B, byred, 0, 0);
draw byLine(B, C, byblue, 0, 0);
draw byLine(C, D, black, 0, 1);
draw byLine(D, A, black, 0, 1);
draw byCircleR(E, r, byblue, 0, 0, 0)(E);
}
\drawCurrentPictureInMargin
\startsubproposition[title={Figure III.}]
\startCenterAlign
The angle \drawAngle{D} in a segment less than a semicircle is obtuse.

Take in the opposite circumference any point, to which draw \drawUnitLine{BC} and \drawUnitLine{AB}.\\
Because $\drawAngle{B} + \drawAngle{D} = \drawTwoRightAngles$ \inprop[prop:III.XXII]\\
but $\drawAngle{B} < \drawRightAngle$ (figure II.),\\
$\therefore$ \drawAngle{D} is obtuse.
\stopCenterAlign
\stopsubproposition

\qed
\stopProposition

\startProposition[title={Prop. XXXII. theor.},reference=prop:III.XXXII]
\defineNewPicture{
pair A, B, C, D, E, F, O;
numeric r;
r := 9/4u;
O := (0, 0);
A := (0, r) shifted O;
B := (0, -r) shifted O;
C := (dir(-20)*r) shifted O;
D := (dir(30)*r) shifted O;
E := (-r, -r) shifted O;
F := (r, -r) shifted O;
draw byAngle(A, B, E, black, 1);
draw byAngle(D, B, A, byblue, 0);
draw byAngle(F, B, D, byyellow, 0);
draw byAngleWithName(B, A, D, byyellow, 0)(A);
draw byAngle(A, D, B, black, 0);
draw byAngle(C, D, B, black, 1);
draw byAngleWithName(D, C, B, byred, 0)(C);
draw byLine(A, D, black, 0, 1);
draw byLine(D, C, black, 0, 1);
draw byLine(C, B, black, 0, 1);
draw byLine(D, B, byred, 0, 0);
draw byLineFull(E, F, byblue, 0, 0)(E, F, 0, 0, 1);
draw byLine(A, B, black, 0, 0);
draw byCircleR(O, r, byred, 0, 0, 0)(O);
}
\drawCurrentPictureInMargin
\problemNP{I}{f}{a straight line \drawUnitLine{EF} be a tangent to a circle, and from the point of contact a right line \drawUnitLine{DB} be drawn cutting the circle, the angle \drawAngle{FBD} made by this line with the tangent is equal to the angle \drawAngle{A} in the alterate segment of the circle.}

If the chord should pass through the centre, it is evident the angles are equal, for each of them is a right angle. (\inpropL[prop:III.XVI], \inpropL[prop:III.XIX])

But if not, draw $\drawUnitLine{AB} \perp \drawUnitLine{EF}$ from the point of contact, it must pass through the centre of the circle, \inprop[prop:III.XIX]

\startCenterAlign
$\therefore \drawAngle{ADB} = \drawAngle{ABE} $ \inprop[prop:III.XXXI]

$\drawAngle{A} + \drawAngle{DBA} = \drawAngle{ABE}= \drawAngle{DBA,FBD}$ \inprop[prop:I.XXXII]

$\therefore \drawAngle{A} = \drawAngle{FBD}$ (ax.).

Again $\drawAngle{ABE,DBA,FBD} = \drawTwoRightAngles = \drawAngle{A} + \drawAngle{C}$ \inprop[prop:III.XXII]

$\therefore \drawAngle{ABE,DBA} = \drawAngle{C}$, (ax.), which is the angle in the alternate segment.
\stopCenterAlign

\qed
\stopProposition

\startProposition[title={Prop. XXXIII. prob.},reference=prop:III.XXXIII]
\defineNewPicture{
pair A, B, D, E, G, K;
numeric r;
r := 7/4u;
G := (0, 0);
A := (0, -r) shifted G;
B := (dir(10)*r) shifted G;
E := (0, r) shifted G;
D := (r, -r) shifted G;
K := (-r, -r) shifted G;
draw byAngle(E, A, K, black, 1);
draw byAngle(B, A, E, byred, 0);
draw byAngle(D, A, B, byyellow, 0);
draw byAngleWithName(G, B, A, byred, 0)(B);
draw byLine(A, B, black, 0, 0);
draw byLine(G, B, byyellow, 0, 0);
draw byLine(A, G, byblue, 0, 0);
draw byLine(G, E, byblue, 1, 0);
draw byLineFull(K, D, byred, 0, 0)(K, D, 0, 0, 1);
draw byCircle(G, B, byblue, 0, 0, 0)(G);
byAngleDefineWithName(E, A, K, black, 1)(givenRight);
byAngleDefineWithName(B, A, K, black, 1)(givenObtuse);
byAngleDefineWithName(D, A, B, black, 1)(givenAcute);
}
\drawCurrentPictureInMargin
\problemNP{O}{n}{a given straight line \drawUnitLine{AB} to describe a segment of a circle that shall contain an angle equal to a given angle \drawAngle{givenRight}, \drawAngle{givenObtuse}, \drawAngle{givenAcute}.}

If a given angle be a right angle, bisect the given line, and describe a semicircle on it, this will evidently contain a right angle. \inprop[prop:III.XXXI]

If the given angle be acute or obtuse, make with the given line, at its extremity,

\startCenterAlign
$\drawAngle{DAB} = \drawAngle{givenAcute}$,

draw $\drawUnitLine{AG} \perp \drawUnitLine{KD}$\\
and make $\drawAngle{B} = \drawAngle{BAE}$,

describe \drawCircle[middle][1/5]{G} with \drawUnitLine{AG} or \drawUnitLine{GB} as radius, for they are equal.

\drawUnitLine{KD} is tangent to \circleG\ \inprop[prop:III.XVI]

$\therefore$ \drawUnitLine{AB} divides the circle into two segments capable of containing angles equal to \drawAngle{EAK,BAE} and \drawAngle{DAB} which were made respectively equal to \drawAngle{givenObtuse} and \drawAngle{givenAcute} \inprop[prop:III.XXXII]
\stopCenterAlign

\qed
\stopProposition

\startProposition[title={Prop. XXXIV. prob.},reference=prop:III.XXXIV]
\defineNewPicture{
pair A, B, F, E, O;
numeric r;
path cr;
r := 7/4u;
O := (0, 0);
cr := (fullcircle scaled 2r) shifted O;
A := point 1 of cr;
B := point -2 of cr;
E := (r, -r) shifted O;
F := (-r, -r) shifted O;
draw byFilledCircleSegment(O, r, -2, 1, byyellow)(O);
draw byAngleWithName(F, B, A, byblue, 0)(B);
draw byLine(A, B, black, 0, 0);
draw byLineFull(E,F, byred, 0, 0)(E, F, 0, 0, -1);
draw byCircleR(O, r, byblue, 0, 0, 0)(O);
byAngleDefineWithName(F, B, A, byred, 0)(givenAngle);
}
\drawCurrentPictureInMargin
\problemNP{T}{o}{cut off from a given circle \drawCircle{O} a segment which shall contain an angle equal to a given angle \drawAngle{givenAngle}.}

\startCenterAlign
Draw \drawUnitLine{EF} \inprop[prop:III.XVII], a tangent to the circle at any point;

at the point of contact make $\drawAngle{B} = \drawAngle{givenAngle}$ the given angle;

and \drawFromCurrentPicture[middle][segmentO]{draw byNamedFilledCircleSegment(O);} contains an angle $=$ the given angle.

Because \drawUnitLine{EF} is a tangent,\\
and \drawUnitLine{AB} cuts it,\\
the angle in $\segmentO\ = \drawAngle{B}$ \inprop[prop:III.XXXII],

but $\drawAngle{B} = \drawAngle{givenAngle}$ (const.)
\stopCenterAlign

\qed
\stopProposition

\startProposition[title={Prop. XXXV. theor.},reference=prop:III.XXXV]
\defineNewPicture{
pair A, B, C, D, E;
numeric r;
r := 3/2u;
E := (0, 0);
A := (dir(50)*r) shifted E;
B := (dir(100)*r) shifted E;
C := (dir(50 + 180)*r) shifted E;
D := (dir(100 + 180)*r) shifted E;
draw byLine(A, E, black, 0, 0);
draw byLine(E, C, black, 1, 0);
draw byLine(D, E, byblue, 0, 0);
draw byLine(E, B, byblue, 1, 0);
draw byCircleR(E, r, byyellow, 0, 0, 0)(E);
}
\problemNP{I}{f}{two chords
$\left\{\vcenter{
\nointerlineskip\hbox{\drawSizedLine{AE,EC}}
\nointerlineskip\hbox{\drawSizedLine{DE,EB}}}\right\}$
in a circle intersect each other, the rectangle contained by the segments of the one is equal to the rectangle contained by the segments of the other.}
\drawCurrentPictureInMargin
\startsubproposition[title={Figure I.}]
If the given right lines pass through the centre, they are bisected in the point of intersection, hence the rectangles under their segments are squares of their halves, and therefore equal.
\stopsubproposition

\defineNewPicture{
pair A, B, C, D, E, H;
numeric r;
r := 3/2u;
E := (0, 0);
A := (dir(30)*r) shifted E;
B := (dir(170)*r) shifted E;
C := (dir(30 + 180)*r) shifted E;
D := (dir(-50)*r) shifted E;
H = whatever[A, C] = whatever[B, D];
draw byLine(D, E, byred, 0, 0);
draw byLine(E, B, byyellow, 0, 0);
draw byLine(B, H, byblue, 0, 0);
draw byLine(H, D, byblue, 1, 0);
draw byLine(A, E, black, 0, 0);
draw byLine(E, H, byred, 1, 0);
draw byLine(H, C, black, 1, 0);
draw byCircleR(E, r, byred, 0, 0, 0)(E);
}
\drawCurrentPictureInMargin
\startsubproposition[title={Figure II.}]
\startCenterAlign
Let \drawSizedLine{HC,EH,AE} pass through the centre, and \drawSizedLine{BH,HD} not; draw \drawSizedLine{EB} and \drawSizedLine{DE}.\\
Then $\drawSizedLine{BH} \times \drawSizedLine{HD} = \drawSizedLine{EB}^2 - \drawSizedLine{EH}^2$ \inprop[prop:II.VI], or $\drawSizedLine{BH} \times \drawSizedLine{HD} = \drawSizedLine{HC,EH}^2 - \drawSizedLine{EH}^2$.\\
$\therefore \drawSizedLine{BH} \times \drawSizedLine{HD} = \drawSizedLine{EH,AE} \times \drawSizedLine{AE}$ \inprop[prop:II.V].
\stopCenterAlign
\stopsubproposition

\defineNewPicture{
pair A, B, C, D, E, F, K, L;
numeric r;
path cr;
r := 3/2u;
F := (0, 0);
cr := (fullcircle scaled 2r) shifted F;
A := (dir(00)*r) shifted F;
B := (dir(170)*r) shifted F;
C := (dir(-150)*r) shifted F;
D := (dir(-50)*r) shifted F;
E = whatever[A, C] = whatever[B, D];
K := cr intersectionpoint (F -- 10[E, F]);
L := cr intersectionpoint (F -- 10[F, E]);
draw byLine(K, E, byred, 1, 0);
draw byLine(E, L, byred, 0, 0);
draw byLine(B, E, byblue, 0, 0);
draw byLine(E, D, byblue, 1, 0);
draw byLine(A, E, black, 0, 0);
draw byLine(E, C, black, 1, 0);
draw byCircleR(F, r, byblue, 0, 0, 0)(F);
}
\drawCurrentPictureInMargin
\startsubproposition[title={Figure III.}]
Let neither of the given lines pass through the centre, draw through their intersection a diameter \drawSizedLine{KE,EL},

\startCenterAlign
and $\drawSizedLine{KE} \times \drawSizedLine{EL} = \drawSizedLine{BE} \times \drawSizedLine{ED}$ (part 2.),\\
also $\drawSizedLine{KE} \times \drawSizedLine{EL} = \drawSizedLine{AE} \times \drawSizedLine{EC}$ (part 2.);\\
$\therefore \drawSizedLine{BE} \times \drawSizedLine{ED} = \drawSizedLine{AE} \times \drawSizedLine{EC}$.
\stopCenterAlign
\stopsubproposition

\qed
\stopProposition

\startProposition[title={Prop. XXXVI. theor.},reference=prop:III.XXXVI]
\defineNewPicture{
pair A, B, C, D, F;
numeric r;
path cr[];
r := 7/4u;
F := (0, 0);
cr1 := (fullcircle scaled 2r) shifted F;
D := (r, 3/2r) shifted F;
C := (dir(angle(D-F)))*r;
A := (dir(angle(D-F) + 180))*r;
cr2 := ((fullcircle scaled abs(D-F)) rotated angle(D-F)) shifted 1/2[D, F];
B := cr1 intersectionpoint (subpath (0, 4) of cr2);
draw byLine(B, F, byyellow, 0, 0);
draw byLine(C, F, byred, 1, 0);
draw byLine(F, A, black, 0, 0);
byLineDefine(D, B, byblue, 0, 0);
byLineDefine(D, C, byred, 0, 0);
draw byNamedLineSeq(0)(DB,DC);
draw byCircleR(F, r, black, 0, 0, 0)(F);
}
\problemNP{I}{f}{from a point without a circle two straight lines be drawn to it, one of which \drawSizedLine{DB} is a tangent to the circle, and the other \drawSizedLine{FA,CF,DC} cuts it; the rectangle under the whole cutting line \drawSizedLine{FA,CF,DC} and the external segment \drawSizedLine{DC} is equal to the square of the tangent \drawSizedLine{DB}.}\unskip
\drawCurrentPictureInMargin
\startsubproposition[title={Figure I.}]
\startCenterAlign
Let \drawSizedLine{FA,CF,DC} pass through the centre;

draw \drawSizedLine{BF} from the centre to the point of contact;

$\drawSizedLine{DB}^2 = \drawSizedLine{CF,DC}^2 \mbox{ minus } \drawSizedLine{BF}^2$ \inprop[prop:I.XLVII],

or $\drawSizedLine{DB}^2 = \drawSizedLine{CF,DC}^2 \mbox{ minus } \drawSizedLine{CF}^2$,

$\therefore \drawSizedLine{DB}^2 = \drawSizedLine{FA,CF,DC} \times \drawSizedLine{DC}$ \inprop[prop:II.VI].
\stopCenterAlign
\stopsubproposition

\defineNewPicture{
pair A, B, C, D, E, F;
numeric r;
path cr[];
r := 7/4u;
E := (0, 0);
cr1 := (fullcircle scaled 2r) shifted E;
D := (r, 3/2r) shifted E;
A := (dir(angle(D-E) + 120))*r;
C := cr1 intersectionpoint (A--D);
cr2 := ((fullcircle scaled abs(D-E)) rotated angle(D-E)) shifted 1/2[D, E];
B := cr1 intersectionpoint (subpath (0, 4) of cr2);
draw byLine(E, A, byyellow, 1, 0);
draw byLine(E, B, byyellow, 0, 0);
draw byLine(E, C, byblue, 1, 0);
draw byLine(D, C, byred, 0, 0);
draw byLine(C, A, byred, 1, 0);
draw byLine(E, B, byyellow, 0, 0);
byLineDefine(D, B, byblue, 0, 0);
byLineDefine(D, E, black, 0, 0);
draw byNamedLineSeq(0)(DB,DE);
draw byCircleR(E, r, black, 0, 0, 0)(E);
}\drawCurrentPictureInMargin
\startsubproposition[title={Figure II.}]
\startCenterAlign
If \drawSizedLine{CA,DC} do not pass through the centre,

draw \drawSizedLine{EA} and \drawSizedLine{EC}.

Then $\drawSizedLine{CA,DC} \times \drawSizedLine{CA} = \drawSizedLine{DE}^2 \mbox{ minus } \drawSizedLine{EC}^2$ \inprop[prop:II.VI],

that is, $\drawSizedLine{CA,DC} \times \drawSizedLine{CA} = \drawSizedLine{DE}^2 \mbox{ minus } \drawSizedLine{EB}^2$,

$\therefore \drawSizedLine{CA,DC} \times \drawSizedLine{CA} = \drawSizedLine{DB}^2$ \inprop[prop:III.XVIII].
\stopCenterAlign
\stopsubproposition

\qed
\stopProposition

\startProposition[title={Prop. XXXVII. theor.},reference=prop:III.XXXVII]
\defineNewPicture{
pair A, B, C, D, E, F;
numeric r;
path cr[];
r := 2u;
F := (0, 0);
cr1 := (fullcircle scaled 2r) shifted F;
D := (4/3r, 4/3r) shifted F;
cr2 := ((fullcircle scaled abs(D-F)) rotated angle(D-F)) shifted 1/2[D, F];
B := cr1 intersectionpoint (subpath (0, 4) of cr2);
E := cr1 intersectionpoint (subpath (4, 8) of cr2);
A := (dir(angle(D-F) + 140))*r;
C := cr1 intersectionpoint (D--A);
draw byAngleWithName(D, B, F, byblue, 0)(B);
draw byAngleWithName(F, E, D, byred, 0)(E);
draw byLine(D, C, black, 1, 0);
draw byLine(C, A, black, 0, 0);
draw byLine(D, F, byyellow, 0, 0);
byLineDefine(B, F, byred, 1, 0);
byLineDefine(E, F, byblue, 1, 0);
byLineDefine(D, B, byred, 0, 0);
byLineDefine(D, E, byblue, 0, 0);
draw byNamedLineSeq(0)(BF,DB,DE,EF);
draw byCircleR(F, r, black, 0, 0, 0)(F);
}
\drawCurrentPictureInMargin
\problemNP{I}{f}{from a point outside of a circle two straight lines be drawn, the one \drawSizedLine{CA,DC} cutting the circle, the other \drawSizedLine{DB} meeting it, and if the rectangle contained by the whole cutting line \drawSizedLine{CA,DC} and its external segment \drawSizedLine{DC} be equal to the square of the line meeting the circle, the latter \drawSizedLine{DB} is a tangent to the circle.}

\startCenterAlign
Draw from the given point \drawSizedLine{DE}, a tangent to the circle,\\
and draw from the centre \drawSizedLine{DF}, \drawSizedLine{BF}, and \drawSizedLine{EF},\\
$\drawSizedLine{DE}^2 = \drawSizedLine{CA,DC} \times \drawSizedLine{DC}$ \inprop[prop:III.XXXVI]\\
but $\drawSizedLine{DB}^2 = \drawSizedLine{CA,DC} \times \drawSizedLine{DC}$ (hyp.),\\
$\therefore \drawSizedLine{DB}^2 = \drawSizedLine{DE}^2$,\\
and $\therefore \drawSizedLine{DB} = \drawSizedLine{DE}$;

Then in \drawLine{DB,DF,BF} and \drawLine{EF,DF,DE}\\
$\drawSizedLine{BF} \mbox{ and } \drawSizedLine{DB} = \drawSizedLine{EF} \mbox{ and } \drawSizedLine{DE}$,\\
and \drawSizedLine{DF} is common,\\
$\therefore \drawAngle{B} = \drawAngle{E}$ \inprop[prop:I.VIII];\\
but $\drawAngle{E} = \drawRightAngle$ a right angle \inprop[prop:III.XVIII],

$\therefore \drawAngle{B} = \drawRightAngle$ a right angle,\\
and $\therefore$ \drawSizedLine{DB} is a tangent to the circle \inprop[prop:III.XVI].
\stopCenterAlign

\qed
\stopProposition
\stopbook

\startbook[title={Book 4}]

\startsupersection[title={Definitions}]

\startDefinitionOnlyNumber[reference=def:IV.I]
\defineNewPicture{
pair A, B, C, D, E, F, G, H;
A := (0, 0);
B := (1/2u, u);
C := (2u, 3/2u);
D := (4/3u, -1/2u);
E := 1/2[A, B];
F := 1/2[B, C];
G := 1/2[C, D];
H := 1/2[D, A];
draw byPolygon(E,F,G,H)(byred);
byLineDefine(A, B, black, 0, 0);
byLineDefine(B, C, black, 0, 0);
byLineDefine(C, D, black, 0, 0);
byLineDefine(D, A, black, 0, 0);
draw byNamedLineSeq(-1)(AB,BC,CD,DA);
}\drawCurrentPictureInMargin[inside]
A rectilinear figure is said to be \emph{inscribed in} another, when all the angular points of the inscribed figure are on sides of the figure in which it is said to be inscribed.
\stopDefinitionOnlyNumber

\startDefinitionOnlyNumber[reference=def:IV.II]
A figure is said to be \emph{described about} another figure, when all the sides of the circumscribed figure pass through the angular points of the other figure.
\stopDefinitionOnlyNumber

\defineNewPicture{
angleScale := 3/4;
pair A, B, C, D, O;
numeric r;
r := 4/5u;
A := (r, 0);
B := (0, r);
C := (-r, 0);
D := (0, -r);
O := (0, 0);
draw byAngleWithName(A, B, C, byred, 0)(B);
draw byAngleWithName(B, C, D, byred, 0)(C);
draw byAngleWithName(C, D, A, byred, 0)(D);
draw byAngleWithName(D, A, B, byred, 0)(A);
byLineDefine(A, B, byred, 0, 0);
byLineDefine(B, C, byred, 0, 0);
byLineDefine(C, D, byred, 0, 0);
byLineDefine(D, A, byred, 0, 0);
draw byNamedLineSeq(1)(DA,CD,BC,AB);
draw byCircleR(O, r, black, 0, 0, 1)(O);
}\drawCurrentPictureInMargin[inside]
\startDefinitionOnlyNumber[reference=def:IV.III]
A rectilinear figure is said to be \emph{inscribed in} a circle, when the vertex of each angle of the figure is in the circumference of the circle.
\stopDefinitionOnlyNumber

\defineNewPicture{
pair A, B, C, D, O;
numeric r;
r := 2/3u;
A := (r, r);
B := (-r, r);
C := (-r, -r);
D := (r, -r);
O := (0, 0);
draw byPolygon(A,B,C,D)(byred);
draw byArbitraryFigure(A--B--C--D--cycle, byred, 0, 0)(ABCDo);
draw byFilledCircleSector(O, r, 0, 8, white)(O);
}\drawCurrentPictureInMargin[inside]
\startDefinitionOnlyNumber[reference=def:IV.IV]
A rectilinear figure is said to be \emph{circumscribed about} a circle, when each of its sides is tangent to the circle.
\stopDefinitionOnlyNumber

\vfill\pagebreak

\startDefinitionOnlyNumber[reference=def:IV.V]
\defineNewPicture{
pair A, B, C, D, E, F, O;
numeric r[];
r1 := 3/4u;
A := dir(0)*r1;
B := dir(60)*r1;
C := dir(120)*r1;
D := dir(180)*r1;
E := dir(240)*r1;
F := dir(300)*r1;
r2 := abs(1/2[A, B]);
O := (0, 0);
draw byPolygon(A,B,C,D,E,F)(byblue);
draw byArbitraryFigure(A--B--C--D--E--F--cycle, byblue, 0, 0)(ABCDEFo);
draw byFilledCircleSector(O, r2, 0, 8, white)(O);
}\drawCurrentPictureInMargin[inside]
A circle is said to be \emph{inscribed in} a rectilinear figure, when each side of the figure is a tangent to the circle.
\stopDefinitionOnlyNumber


\startDefinitionOnlyNumber[reference=def:IV.VI]
\defineNewPicture[1/2]{
pair A, B, C, D, E, F, G, H, K, I;
numeric r;
A := (0, 0);
B := (3u, 7/3u);
C := (7/2u, 0);
D = whatever[A, B] = whatever[C, C shifted dir(angle(C-A)) shifted dir(angle(C-B))];
E = whatever[B, C] = whatever[A, A shifted dir(angle(A-B)) shifted dir(angle(A-C))];
F = whatever[C, A] = whatever[B, B shifted dir(angle(B-C)) shifted dir(angle(B-A))];
I = whatever[C, D] = whatever[A, E];
G = whatever[A, B] = whatever[I, I shifted ((A-B) rotated 90)];
H = whatever[B, C] = whatever[I, I shifted ((B-C) rotated 90)];
K = whatever[C, A] = whatever[I, I shifted ((C-A) rotated 90)];
r := abs(D-I);
draw byPolygon(G,H,K)(byblue);
byLineDefine(A, B, black, 0, 0);
byLineDefine(B, C, black, 0, 0);
byLineDefine(C, A, black, 0, 0);
draw byNamedLineSeq(-1)(AB,BC,CA);
draw byCircleR(I, r, black, 0, 0, -1)(I);
}\drawCurrentPictureInMargin[inside]
A circle is said to be \emph{circumscribed about} a rectilinear figure, when the circumference passes through the vertex of each angle of the figure.

\drawPolygon{GHK} is circumscribed.
\stopDefinitionOnlyNumber

\startDefinitionOnlyNumber[reference=def:IV.VI]
\defineNewPicture{
pair A, B, O;
numeric r;
r := 3/4u;
O := (0, 0);
A := dir(0)*r;
B := dir(100)*r;
draw byLine(A, B, black, 0, 0);
draw byCircleR(O, r, byblue, 0, 0, 0)(O);
}\drawCurrentPictureInMargin[inside]
A straight line is said to be \emph{inscribed in} a circle, when its extremities are in the circumference.
\stopDefinitionOnlyNumber

\vskip 1cm

The Fourth Book of the Elements is devoted to the solution of problems, chiefly relating to the inscription and circumscription of regular polygons and circles.

A regular polygon is one whose angles and sides are equal.
\stopsupersection

\vfill\pagebreak

\startProposition[title={Prop. I. prob.},reference=prop:IV.I]
\defineNewPicture{
pair A, B, C, E, G, H, O;
numeric r[];
path cr[];
r1 := 4/3u;
r2 := 7/4u;
O := (0, 0);
B := (r1, 0) shifted O;
C := (-r1, 0) shifted O;
cr1 := (fullcircle scaled 2r1) shifted O;
cr2 := (fullcircle scaled 2r2) shifted C;
A := cr1 intersectionpoint (subpath (0, 2) of cr2);
E := (B--C) intersectionpoint cr2;
G := (xpart(lrcorner(cr1)), ypart(lrcorner(cr2)));
H := G shifted (-r2, 0);
draw byLine(B, E, byred, 0, 0);
draw byLine(E, C, byred, 1, 0);
draw byLine(A, C, byyellow, 0, 0);
draw byLine(G, H, byblue, 0, 0);
draw byCircleR(O, r1, byyellow, 0, 0, 0)(O);
draw byCircle(C, E, byblue, 0, 0, 0)(C);
}
\drawCurrentPictureInMargin
\problemNP{I}{n}{a given circle \drawCircle{O} to place a straight line, equal to a given straight line (\drawUnitLine{GH}), not greater than the diameter of the circle.}

\startCenterAlign
Draw \drawUnitLine{EC,BE}, the diameter of \circleO;\\
and if $\drawUnitLine{EC,BE} = \drawUnitLine{GH}$, then the problem is solved.

But if \drawUnitLine{EC,BE} be not equal to \drawUnitLine{GH},\\
$\drawUnitLine{EC,BE} > \drawUnitLine{GH}$ (hyp.);

make $\drawUnitLine{EC} = \drawUnitLine{GH}$ \inprop[prop:I.III]\\
with \drawUnitLine{EC} as radius, describe \drawCircle{C}, cutting \circleO, \\
and draw \drawUnitLine{AC}, which is the line required.

For $\drawUnitLine{AC} = \drawUnitLine{EC} = \drawUnitLine{GH}$ (\indefL[prop:I.XV], const.)
\stopCenterAlign

\qed
\stopProposition

\startProposition[title={Prop. II. prob.},reference=prop:IV.II]
\defineNewPicture{
pair A, B, C, D, E, F, G, H, O, d;
numeric r;
path cr;
r := 7/4u;
O := (0, 0);
cr := (fullcircle scaled 2r) shifted O;
A := (0, -r) shifted O;
G := (-r, -r) shifted O;
H := (r, -r) shifted O;
D := (0, 0);
E := (-1/2r, 4/3r);
F := (3/4r, r);
d := -1/2[ulcorner(D--E--F), lrcorner(D--E--F)] shifted (0, -2r);
D := D shifted d;
E := E shifted d;
F := F shifted d;
C := (subpath (-1, 5) of cr) intersectionpoint (A -- A shifted (dir(angleValue(F, E, D))*2r));
B := (subpath (-1, 5) of cr) intersectionpoint (A -- A shifted (dir(180-angleValue(D, F, E))*2r));
draw byAngleWithName(C, B, A, black, 0)(B);
draw byAngleWithName(A, C, B, black, 1)(C);
draw byAngle(G, A, B, byyellow, 0);
draw byAngle(B, A, C, byred, 0);
draw byAngle(C, A, H, byblue, 0);
draw byAngleWithName(F, E, D, byblue, 0)(E);
draw byAngleWithName(E, D, F, byred, 0)(D);
draw byAngleWithName(D, F, E, byyellow, 0)(F);
draw byArbitraryFigure(B--A--C, black, 0, 1)(BAC);
draw byArbitraryFigure(D--E--F--cycle, black, 0, 1)(BAC);
draw byLine(B, C, byyellow, 0, 0);
draw byLineFull(G, H, byred, 0, 0)(G, H, 0, 0, 2);
draw byCircleR(O, r, black, 0, 0, 0)(O);
}
\drawCurrentPictureInMargin
\problemNP{I}{n}{a given circle \drawCircle{O} to inscribe a triangle equiangular to a given triangle.}

\startCenterAlign
To any point of the given circle draw \drawUnitLine{GH}, a tangent \inprop[prop:III.XVII];

and at the point of contact make $\drawAngle{CAH} = \drawAngle{E}$ \inprop[prop:I.XXIII]

and in like manner $\drawAngle{GAB} = \drawAngle{F}$,

and draw \drawUnitLine{BC}.

Because $\drawAngle{CAH} = \drawAngle{E}$ (const.)\\
and $\drawAngle{CAH} = \drawAngle{B}$ \inprop[prop:III.XXXII]\\
$\therefore \drawAngle{B} = \drawAngle{E}$;\\
also $\drawAngle{C} = \drawAngle{F}$ for the same reason.

$\therefore \drawAngle{BAC} = \drawAngle{D}$ \inprop[prop:I.XXXII],
\stopCenterAlign
and therefore the triangle inscribed in the circle is equiangular to the given one.

\qed
\stopProposition

\startProposition[title={Prop. III. prob.},reference=prop:IV.III]
\defineNewPicture[3/5]{
pair A, B, C, D, E, F, G, H, K, L, M, N, d;
numeric r;
r := 5/4u;
d := (-4/3r, 9/4r);
D := (0, 0) shifted d;
E := (1/3r, -r) shifted d;
F := (-5/6r, -r) shifted d;
G := 4/3[F, E];
H := 4/3[E, F];
K := (0, 0);
C := (0, -r) shifted K;
A := (dir(-90-angleValue(D, F, H))*r) shifted K;
B := (dir(-90+angleValue(G, E, D))*r) shifted K;
L = whatever[C, C shifted (dir(angle(C-K) + 90))] = whatever[A, A shifted (dir(angle(A-K) + 90))];
M = whatever[B, B shifted (dir(angle(B-K) + 90))] = whatever[A, L];
N = whatever[B, M] = whatever[C, L];
draw byAngle(D, F, H, byyellow, 0);
draw byAngle(C, K, A, byyellow, 0);
draw byAngle(G, E, D, byblue, 0);
draw byAngle(B, K, C, byblue, 0);
draw byAngleWithName(M, L, N, byred, 0)(L);
draw byAngle(D, F, E, byred, 0);
draw byAngleWithName(L, N, M, black, 1)(N);
draw byAngle(F, E, D, black, 1);
draw byAngleWithName(N, M, L, byred, 1)(M);
draw byAngleWithName(E, D, F, byred, 1)(D);
draw byAngleWithName(K, A, L, black, 0)(A);
draw byAngleWithName(L, C, K, black, 0)(C);
draw byArbitraryFigure(F--D--E, black, 0, 1)(FDE);
byLineDefine(N, L, byblue, 0, 0);
byLineDefine(L, M, byyellow, 0, 0);
byLineDefine(M, N, byred, 1, 0);
draw byLine(G, H, black, 0, 0);
draw byLine(K, C, byred, 0, 0);
draw byArbitraryFigure(A--K--B, black, 0, 1)(AKB);
draw byCircleR(K, r, byred, 0, 0, -1)(K);
draw byNamedLineSeq(0)(NL,LM,MN);
byArbitraryFigureDefine(L--A--K--C--cycle, black, 0, 1)(LAKC);
}
\drawCurrentPictureInMargin
\problemNP{A}{bout}{a given circle \drawCircle[middle][1/3]{K} to circumscribe a triangle equiangular to a given triangle.}

\startCenterAlign
Produce any side \drawUnitLine{GH}, of a given triangle both ways;

from the centre of the given circle draw \drawUnitLine{KC}, any radius.

Make $\drawAngle{CKA} = \drawAngle{DFH}$ \inprop[prop:I.XXIII]\\
and $\drawAngle{BKC} = \drawAngle{GED}$.

At the extremities of the radii draw \drawUnitLine{NL}, \drawUnitLine{LM} and \drawUnitLine{MN}, tangents to the given circle. \inprop[prop:III.XVII]

The four angles of
\drawFromCurrentPicture[bottom]{
startTempAngleScale(2/3);
draw byNamedAngle(A, C, L, CKA);
draw byNamedArbitraryFigure(LAKC);
stopTempAngleScale;
},
taken together are equal to four right angles. \inprop[prop:I.XXXII]

but \drawAngle{C} and \drawAngle{A} are right angles (const.)

$\therefore \drawAngle{L} + \drawAngle{CKA} = \drawTwoRightAngles$, two right angles

but $\drawAngle{DFH,DFE} = \drawTwoRightAngles$ \inprop[prop:I.XIII]\\
and $\drawAngle{CKA} = \drawAngle{DFH}$ (const.) and $\therefore \drawAngle{L} = \drawAngle{DFE}$.

In the same manner it can be demonstrated that\\
$\drawAngle{N} = \drawAngle{FED}$;

$\therefore \drawAngle{M} = \drawAngle{D}$ \inprop[prop:I.XXXII]\\
and therefore the triangle circumscribed about the given circle is equiangular to the given triangle.
\stopCenterAlign

\qed
\stopProposition

\startProposition[title={Prop. IV. prob.},reference=prop:IV.IV]
\defineNewPicture{
pair A, B, C, D, E, F, G;
numeric r;
A := (0, 0);
B := (-3/2u, -3u);
C := (3u, -3u);
D = whatever[A, A shifted (unitvector(B-A) + unitvector(C-A))] = whatever[B, B shifted (unitvector(A-B) + unitvector(C-B))];
E = whatever[D, D shifted ((A-B) rotated 90)] = whatever[A, B];
F = whatever[D, D shifted ((B-C) rotated 90)] = whatever[B, C];
G = whatever[D, D shifted ((C-A) rotated 90)] = whatever[A, C];
r := abs(E-D);
draw byAngleWithName(B, E, D, byred, 0)(E);
draw byAngleWithName(D, F, B, byred, 0)(F);
draw byAngle(E, B, D, byblue, 0);
draw byAngle(D, B, F, byyellow, 0);
draw byAngle(G, C, D, black, 0);
draw byAngle(D, C, F, black, 1);
draw byAngle(E, D, F, black, 1);
draw byLine(E, D, byyellow, 1, 0);
draw byLine(G, D, byred, 1, 0);
draw byLine(F, D, black, 1, 0);
byLineDefine(C, D, byblue, 0, 0);
byLineDefine(B, D, byblue, 1, 0);
draw byNamedLineSeq(0)(CD,BD);
byLineDefine(A, B, byyellow, 0, 0);
byLineDefine(B, C, black, 0, 0);
byLineDefine(C, A, byred, 0, 0);
draw byNamedLineSeq(0)(AB,BC,CA);
draw byCircle(D, G, black, 0, 0, -1)(D);
byLineDefine(B, F, black, 0, 0);
byLineDefine(B, E, byyellow, 0, 0);
}
\drawCurrentPictureInMargin
\problemNP{I}{n}{a given triangle \drawFromCurrentPicture[bottom]{
startTempScale(1/4);
draw byNamedLineSeq(0)(CA,BC,AB);
stopTempScale;
} to inscribe a circle.}

Bisect \drawAngle{EBD,DBF} and \drawAngle{GCD,DCF} \inprop[prop:I.IX] by \drawUnitLine{BD} and \drawUnitLine{CD};

From the point where these lines meet draw \drawUnitLine{FD}, \drawUnitLine{ED} and \drawUnitLine{GD} respectively perpendicular to \drawUnitLine{BC}, \drawUnitLine{AB} and \drawUnitLine{CA}.

\startCenterAlign
In
\drawFromCurrentPicture{
startTempScale(3/4);
draw byNamedAngle(F, DBF);
draw byNamedLineSeq(0)(BD,FD,BF);
stopTempScale;
}
and
\drawFromCurrentPicture{
startTempScale(3/4);
draw byNamedAngle(E, EBD);
draw byNamedLineSeq(0)(BE,ED,BD);
stopTempScale;
}\\
$\drawAngle{DBF} = \drawAngle{EBD}$, $\drawAngle{F} = \drawAngle{E}$ and \drawUnitLine{BD} common,\\
$\therefore \drawUnitLine{ED} = \drawUnitLine{FD}$ (\inpropL[prop:I.IV], \inpropL[prop:I.XXVI]

In like manner, it may be shown also that $\drawUnitLine{GD} = \drawUnitLine{FD}$,

$\therefore \drawUnitLine{FD} = \drawUnitLine{ED} = \drawUnitLine{GD}$;
\stopCenterAlign

hence with any of these lines as radius, describe \drawCircle[middle][1/2]{D} and it will pass through the extremities of the other two; and the sides of the given triangle, being perpendicular to the three radii at their extremities, touch the circle \inprop[prop:III.XVI], which is therefore inscribed in the given triangle.

\qed
\stopProposition

\startProposition[title={Prop. V. prob.},reference=prop:IV.V]
\defineNewPicture[1]{
def proptmp (expr a, b, c, s) =
pair A, B, C, D, E, F;
numeric r;
r := 7/4u;
F := s;
A := (dir(a)*r) shifted F;
B := (dir(b)*r) shifted F;
C := (dir(c)*r) shifted F;
D := 1/2[A, B];
E := 1/2[A, C];
draw byAngle(A, D, F, byred, 0);
draw byAngle(F, D, B, black, 0);
draw byLine(D, F, byyellow, 0, 0);
draw byLine(E, F, byyellow, 1, 0);
draw byLine(A, F, black, 1, 0);
draw byLine(A, D, byblue, 0, 0);
draw byLine(D, B, byblue, 1, 0);
if (sind(b)<>-sind(c)):
byLineDefine(B, F, black, 0, 0);
byLineDefine(C, F, black, 0, 1);
draw byNamedLineSeq(0)(BF,CF);
draw byLine(B, C, byred, 0, 0);
else:
draw byLine(B, C, black, 0, 0);
fi;
draw byLine(C, E, byred, 0, 0);
draw byLine(E, A, byred, 1, 0);
draw byCircleR(F, r, black, 0, 0, 0)(F);
enddef;
proptmp(100, 170, 10, (0, -8u));
proptmp(80, 160, -20, (0, -4u));
proptmp(100, 200, -30, (0, 0));
}
\drawCurrentPictureInMargin
\problemNP{T}{o}{describe a circle about a given triangle.}

Make $\drawUnitLine{AD} = \drawUnitLine{DB}$ and $\drawUnitLine{CE} = \drawUnitLine{EA}$ \inprop[prop:I.X]

From the points of bisection draw \drawUnitLine{DF} and \drawUnitLine{EF} $\perp$ \drawUnitLine{AD} and \drawUnitLine{CE} respectively \inprop[prop:I.XI], and from their point of concourse draw \drawUnitLine{BF}, \drawUnitLine{AF} and \drawUnitLine{CF} and describe a circle with any one of them, and it will be the circle required.

\startCenterAlign
In
\drawFromCurrentPicture{
draw byNamedAngle(FDB);
draw byNamedLineSeq(0)(DF,BF,DB);
}
and
\drawFromCurrentPicture{
draw byNamedAngle(ADF);
draw byNamedLineSeq(0)(AF,DF,AD);
}\\
$\drawUnitLine{DB} = \drawUnitLine{AD}$ (const.),\\
\drawUnitLine{DF} common,\\
$\drawAngle{FDB} = \drawAngle{ADF}$ (const.),

$\therefore \drawUnitLine{AF} = \drawUnitLine{BF}$ \inprop[prop:I.IV].

In like manner it may be shown that $\drawUnitLine{CF} = \drawUnitLine{AF}$.
\stopCenterAlign

$\therefore \drawUnitLine{AF} = \drawUnitLine{BF} = \drawUnitLine{CF}$; and therefore a circle described from the concourse of these three lines with any one of them as radius will circumscribe the given triangle.

\qed
\stopProposition

\startProposition[title={Prop. VI. prob.},reference=prop:IV.VI]
\defineNewPicture{
pair A, B, C, D, E;
numeric r;
r := 7/4u;
E := (0, 0);
A := (0, r);
B := (-r, 0);
C := (0, -r);
D := (r, 0);
draw byAngleWithName(D, A, B, byyellow, 0)(A);
draw byAngleWithName(C, D, A, black, 0)(D);
draw byAngle(A, E, B, byblue, 0);
draw byAngle(D, E, A, byred, 0);
draw byLine(B, E, byred, 1, 0);
draw byLine(D, E, byblue, 1, 0);
draw byLine(A, E, black, 1, 0);
draw byLine(C, E, black, 0, 1);
byLineDefine(A, B, black, 0, 0);
byLineDefine(B, C, byyellow, 0, 0);
byLineDefine(C, D, byblue, 0, 0);
byLineDefine(D, A, byred, 0, 0);
draw byNamedLineSeq(0)(AB,BC,CD,DA);
draw byCircle(E, D, byred, 0, 0, 1)(E);
}
\drawCurrentPictureInMargin
\problemNP{I}{n}{a given circle \drawCircle{E} to inscribe a square.}

Draw the two diameters of the circle $\perp$ to each other, and draw \drawUnitLine{BC}, \drawUnitLine{AB}, \drawUnitLine{DA} and \drawUnitLine{CD}.

\startCenterAlign
\drawLine[middle][squareABCD]{DA,CD,BC,AB} is a square.

For, since \drawAngle{A} and \drawAngle{D} are, each of them, in a semicircle, they are right angles \inprop[prop:III.XXXI],

$\therefore \drawUnitLine{CD} \parallel \drawUnitLine{AB}$ \inprop[prop:I.XXVIII]:

and in like manner $\drawUnitLine{DA} \parallel \drawUnitLine{BC}$.

And because $\drawAngle{AEB} = \drawAngle{DEA}$ (const.),

and $\drawUnitLine{BE} = \drawUnitLine{AE} = \drawUnitLine{DE}$ \inprop[prop:I.XV].

$\therefore \drawUnitLine{AB} = \drawUnitLine{DA}$ \inprop[prop:I.IV];
\stopCenterAlign

and since the adjacent sides and angles of the parallelogram \squareABCD\ are equal, they are all equal \inprop[prop:I.XXXIV]; and $\therefore$ \squareABCD, inscribed in the given circle is a square.

\qed
\stopProposition

\startProposition[title={Prop. VII. prob.},reference=prop:IV.VII]
\defineNewPicture[1/4]{
pair A, B, C, D, E, F, G, H, K;
numeric r;
r := 7/4u;
E := (0, 0);
A := (0, r);
B := (-r, 0);
C := (0, -r);
D := (r, 0);
F := (r, r);
G := (-r, r);
H := (-r, -r);
K := (r, -r);
draw byAngleWithName(F, G, H, byred, 0)(G);
draw byAngleWithName(G, H, K, byred, 0)(H);
draw byAngleWithName(H, K, F, byred, 0)(K);
draw byAngleWithName(K, F, G, byred, 0)(F);
draw byAngleWithName(B, E, C, black, 0)(E);
draw byAngleWithName(E, C, H, byyellow, 0)(C);
draw byLine(A, C, byred, 1, 0);
draw byLine(B, D, byblue, 1, 0);
byLineDefine(F, G, black, 0, 0);
byLineDefine(G, H, byred, 0, 0);
byLineDefine(H, K, byblue, 0, 0);
byLineDefine(K, F, byyellow, 0, 0);
draw byNamedLineSeq(0)(KF,HK,GH,FG);
draw byCircle(E, D, byblue, 0, 0, -1)(E);
}
\drawCurrentPictureInMargin
\problemNP{A}{bout}{a given circle \drawCircle{E} to circumscribe a square.}

Draw two diameters of the given circle perpendicular to each other, and through their extremities draw \drawUnitLine{HK}, \drawUnitLine{GH}, \drawUnitLine{FG} and \drawUnitLine{KF} tangents to the circle;

\startCenterAlign
and \drawLine[bottom][squareFGHK]{KF,HK,GH,FG} is a square.

$\drawAngle{C} = \drawRightAngle$ a right angle, \inprop[prop:III.XVIII]

also $\drawAngle{E} = \drawRightAngle$ (const.),

$\therefore \drawUnitLine{HK} \parallel \drawUnitLine{BD}$; in the same manner it can be demonstrated that $\drawUnitLine{FG} \parallel \drawUnitLine{BD}$, and also that $\drawUnitLine{GH} \mbox{ and } \drawUnitLine{KF} \parallel \drawUnitLine{AC}$;

$\therefore$ \squareFGHK\ is a parallelogram,

and because $\drawAngle{C} = \drawAngle{G} = \drawAngle{F} = \drawAngle{K} = \drawAngle{H}$ they are all right angles \inprop[prop:I.XXXIV];

it is also evident that \drawUnitLine{HK}, \drawUnitLine{GH}, \drawUnitLine{FG} and \drawUnitLine{KF} are equal.

$\therefore$ \squareFGHK\ is a square.
\stopCenterAlign

\qed
\stopProposition

\startProposition[title={Prop. VIII. prob.},reference=prop:IV.VIII]
\defineNewPicture[1/2]{
pair A, B, C, D, E, F, G, H, K;
numeric r;
r := 7/4u;
G := (0, 0);
A := (-r, r);
B := (-r,-r);
C := (r, -r);
D := (r, r);
E := (0, r);
F := (-r, 0);
H := (0, -r);
K := (r, 0);
draw byPolygon(E,D,K,G)(black);
draw byPolygon(F,G,H,B)(byred);
draw byPolygon(G,K,C,H)(byblue);
draw byLine(F, G, byyellow, 0, 0);
draw byLine(G, K, byyellow, 1, 0);
draw byLine(G, H, black, 0, 0);
draw byLine(E, G, black, 1, 0);
byLineDefine(A, B, black, 0, 0);
byLineDefine(A, E, byblue, 1, 0);
byLineDefine(E, D, byblue, 0, 0);
byLineDefine(D, K, byred, 1, 0);
byLineDefine(K, C, byred, 0, 0);
draw byNamedLineSeq(0)(AB,AE,ED,DK,KC);
draw byCircleR(G, r, byyellow, 0, 0, -1)(G);
}
\drawCurrentPictureInMargin
\problemNP{T}{o}{inscribe a circle in a given square.}

\startCenterAlign
Make $\drawUnitLine{ED} = \drawUnitLine{AE}$,\\
and $\drawUnitLine{KC} = \drawUnitLine{DK}$,\\
draw $\drawUnitLine{FG,GK} = \drawUnitLine{AE,ED}$,\\
and $\drawUnitLine{EG,GH} = \drawUnitLine{DK,KC}$\\
\inprop[prop:I.XXXI]

$\therefore$ \drawPolygon[bottom]{EDKG} is a parallelogram;

and since $\drawUnitLine{AE,ED} = \drawUnitLine{DK,KC}$ (hyp.)\\
$\drawUnitLine{ED} = \drawUnitLine{DK}$

$\therefore$ \polygonEDKG\ is equilateral \inprop[prop:I.XXXIV]

In like manner it can be shown that $ \drawPolygon[bottom]{GKCH} = \drawPolygon[bottom]{FGHB}$ are equilateral parallelograms;

$\therefore \drawUnitLine{AE} = \drawUnitLine{GK} = \drawUnitLine{GH} = \drawUnitLine{FG}$.
\stopCenterAlign

and therefore if a circle be described from the concourse of these lines with any of them as radius, it will be inscribed in the given square. \inprop[prop:III.XVI]

\qed
\stopProposition

\startProposition[title={Prop. IX. prob.},reference=prop:IV.IX]
\defineNewPicture{
pair A, B, C, D, E;
numeric r;
r := 7/4u;
E := (0, 0);
A := (dir(45+90)*r) shifted E;
B := (dir(45+180)*r) shifted E;
C := (dir(45+270)*r) shifted E;
D := (dir(45+360)*r) shifted E;
draw byPolygon(A,C,D)(byred);
draw byPolygon(A,B,C)(byyellow);
draw byAngle(D, A, E, byyellow, 0);
draw byAngle(E, A, B, black, 0);
draw byAngle(A, B, E, byred, 0);
draw byAngle(E, B, C, byblue, 0);
draw byLine(A, E, byblue, 0, 0);
draw byLine(E, C, byblue, 1, 0);
draw byLine(B, E, black, 0, 0);
draw byLine(E, D, black, 1, 0);
draw byCircleR(E, r, black, 0, 0, 0)(E);
}
\drawCurrentPictureInMargin
\problemNP{T}{o}{describe a circle about a given square \drawPolygon{ACD,ABC}.}

Draw the diagonals \drawUnitLine{AE,EC} and \drawUnitLine{BE,ED} intersecting each other;

then, because \drawPolygon{ACD} and \drawPolygon{ABC} have their sides equal, and the base \drawUnitLine{AE,EC} common to both,

\startCenterAlign
$\drawAngle{DAE} = \drawAngle{EAB}$ \inprop[prop:I.VIII],\\
or \drawAngle{DAE,EAB} is bisected: \\
in like manner it may be shown that \drawAngle{ABE,EBC} is bisected;

but $\drawAngle{DAE,EAB} = \drawAngle{ABE,EBC}$,

hence $\drawAngle{EAB} = \drawAngle{ABE}$ their halves,

$\therefore \drawUnitLine{BE} = \drawUnitLine{AE}$ \inprop[prop:I.VI];

and in like manner it can be proved that \\
$\drawUnitLine{AE} = \drawUnitLine{BE} = \drawUnitLine{EC} = \drawUnitLine{ED}$.
\stopCenterAlign

If from the confluence of these lines with any one of them as radius, a circle be described, it will circumscribe the given square.

\qed
\stopProposition

\startProposition[title={Prop. X. prob.},reference=prop:IV.X]
\defineNewPicture[1/2]{
pair A, B, C, D, E, F;
numeric r[];
path cr[];
r1 := 5/2u;
A := (0, 0);
cr1 := (fullcircle scaled 2r1) shifted A;
B := (r1, 0);
r2 := r1*sqrt(5)/2 - 1/2r1;
cr2 := (fullcircle scaled 2r2) shifted B;
C := (r2, 0);
D := cr1 intersectionpoint (subpath (0, 4) of cr2);
F = whatever[1/2[A, C], 1/2[A, C] shifted ((A-C) rotated 90)]
 = whatever[1/2[A, D], 1/2[A, D] shifted ((A-D) rotated 90)];
r3 := abs(F-A);
draw byAngleWithName(D, A, B, black, 1)(A);
draw byAngle(C, D, A, black, 0);
draw byAngle(B, D, C, byyellow, 0);
draw byAngleWithName(D, C, B, byblue, 0)(C);
draw byAngleWithName(C, B, D, byred, 0)(B);
draw byLine(A, D, byyellow, 0, 0);
draw byLine(B, D, byblue, 0, 0);
draw byLine(C, D, byred, 0, 0);
draw byLine(A, C, black, 0, 0);
draw byLine(C, B, black, 1, 0);
draw byCircleABC(A, C, D, byblue, 0, 0, 0)(F);
draw byCircle(A, B, byred, 0, 0, 0)(A);
}
\drawCurrentPicture
\problemNP{T}{o}{construct an isosceles triangle, in which each of the angles at the base shall be double of the vertical angle.}

\startCenterAlign
Take any straight line \drawProportionalLine{AC,CB} \\
and divide it so that $\drawProportionalLine{AC,CB} \times \drawProportionalLine{CB} = \drawProportionalLine{AC}^2$ \inprop[prop:II.XI]

With \drawProportionalLine{AC,CB} as radius, describe \drawCircle[middle][1/5]{A} and place in it from the extremity of the radius, $\drawProportionalLine{BD} = \drawProportionalLine{AC}$ \inprop[prop:IV.I];\\
draw \drawProportionalLine{AD}.

Then \drawLine{AD,BD,CB,AC} is the required triangle.

For, draw \drawProportionalLine{CD} and describe \drawCircle[middle][1/3]{F} about \drawLine{AD,CD,AC} \inprop[prop:IV.V]

Since $\drawProportionalLine{AC,CB} \times \drawProportionalLine{CB} = \drawProportionalLine{AC}^2 = \drawProportionalLine{BD}^2$,\\
$\therefore$ \drawProportionalLine{BD} is tangent to \circleF\ \inprop[prop:III.XXXVII]\\
$\therefore \drawAngle{BDC} = \drawAngle{A}$ \inprop[prop:III.XXXII],\\
add \drawAngle{CDA} to each,\\
$\therefore \drawAngle{BDC} + \drawAngle{CDA} = \drawAngle{A} + \drawAngle{CDA}$;

but $\drawAngle{BDC} + \drawAngle{CDA} \mbox{ or } \drawAngle{BDC,CDA} = \drawAngle{B}$ \inprop[prop:I.V]:\\
since $\drawProportionalLine{AD} = \drawProportionalLine{AC,CB}$ \inprop[prop:I.VI]\\
consequently $\drawAngle{B} = \drawAngle{A} + \drawAngle{CDA} = \drawAngle{C}$ \inprop[prop:I.XXXII]\\
$\therefore \drawProportionalLine{CD} = \drawProportionalLine{BD}$ \inprop[prop:I.VI]\\
$\therefore \drawProportionalLine{BD} = \drawProportionalLine{AC} = \drawProportionalLine{CD}$ (const.)\\
$\therefore \drawAngle{A} = \drawAngle{CDA}$ \inprop[prop:I.V]

$\therefore \drawAngle{B} = \drawAngle{BDC,CDA} = \drawAngle{C} = \drawAngle{A} + \drawAngle{CDA} = \mbox{ twice } \drawAngle{A}$;

and consequently each angle at the base is double of the vertical angle.
\stopCenterAlign

\qed
\stopProposition

\startProposition[title={Prop. XI. prob.},reference=prop:IV.XI]
\defineNewPicture[1/4]{
pair A, B, C, D, E, O;
numeric r;
r := 9/4u;
O := (0, 0);
A := (dir(90+0/5(360))*r) shifted O;
B := (dir(90+1/5(360))*r) shifted O;
C := (dir(90+2/5(360))*r) shifted O;
D := (dir(90+3/5(360))*r) shifted O;
E := (dir(90+4/5(360))*r) shifted O;
draw byArbitraryFigure(C--E--B--D, black, 0, 1)(pg);
draw byPolygon(A,C,D)(byyellow);
draw byAngleWithName(E, A, B, black, -1)(A);
draw byAngleWithName(A, B, C, black, -1)(B);
draw byAngleWithName(B, C, D, black, -1)(C);
draw byAngleWithName(C, D, E, black, -1)(D);
draw byAngleWithName(D, E, A, black, -1)(E);
draw byAngle(C, A, D, black, 0);
draw byAngle(E, C, A, byyellow, 0);
draw byAngleWithName(E, C, A, black, -1)(ECAb);
draw byAngle(D, C, E, byblue, 0);
draw byAngle(A, D, B, black, 1);
draw byAngle(B, D, C, byred, 0);
draw byNamedAngleDummySides(ADB,BDC);
byLineDefine(A, B, byblue, 0, 0);
byLineDefine(B, C, byred, 0, 0);
byLineDefine(C, D, black, 0, 0);
byLineDefine(D, E, byred, 1, 0);
byLineDefine(E, A, byyellow, 0, 0);
draw byNamedLineSeq(0)(AB,BC,CD,DE,EA);
draw byCircleABC(A, C, D, byblue, 0, 0, 1)(O);
}
\drawCurrentPictureInMargin
\problemNP[2]{I}{n}{a given circle \drawCircle[middle][1/4]{O} to inscribe an equilateral and equiangular pentagon.}

Construct an isosceles triangle, in which each of the angles at the base shall be double of the angle at the vertex, and inscribe in the given circle a triangle \drawPolygon[bottom]{ACD} equiangular to it \inprop[prop:IV.II];

Bisect \drawAngle{ECA,DCE} and \drawAngle{ADB,BDC} \inprop[prop:I.IX]

draw \drawUnitLine{BC}, \drawUnitLine{AB}, \drawUnitLine{EA} and \drawUnitLine{DE}.

Because each of the angles \drawAngle{ECA}, \drawAngle{DCE}, \drawAngle{CAD}, \drawAngle{BDC} and \drawAngle{ADB} are equal, the arcs upon which they stand are equal \inprop[prop:III.XXVI]; and $\therefore$ \drawUnitLine{CD}, \drawUnitLine{BC}, \drawUnitLine{AB}, \drawUnitLine{EA} and \drawUnitLine{DE} which subtended these arcs are equal \inprop[prop:III.XIX] and $\therefore$ the pentagon is equilateral, it is also equiangular, as each of its angles stand upon equal arcs \inprop[prop:III.XXVII].

\qed
\stopProposition

\startProposition[title={Prop. XII. prob.},reference=prop:IV.XII]
\defineNewPicture[1/2]{
pair B, C, D, G, H, K, L, M, F;
numeric r[];
r1 := 5/2u;
F := (0, 0);
G := (dir(90+0/5(360))*r1) shifted F;
H := (dir(90+1/5(360))*r1) shifted F;
K := (dir(90+2/5(360))*r1) shifted F;
L := (dir(90+3/5(360))*r1) shifted F;
M := (dir(90+4/5(360))*r1) shifted F;
B := 1/2[H, K];
C := 1/2[K, L];
D := 1/2[L, M];
r2 := abs(F-B);
draw byAngle(B, F, K, byred, 0);
draw byAngle(K, F, C, byyellow, 0);
draw byAngle(C, F, L, byblue, 0);
draw byAngle(L, F, D, byred, 0);
draw byAngle(F, K, B, byyellow, 1);
draw byAngle(C, K, F, black, 0);
draw byAngle(F, C, K, byblue, 1);
draw byAngle(L, C, F, byblue, 1);
draw byAngle(F, L, C, black, 0);
draw byAngle(D, L, F, black, 1);
draw byLine(F, K, byblue, 0, 0);
draw byLine(F, C, black, 1, 0);
draw byLine(F, L, black, 0, 0);
byLineDefine(F, B, byred, 1, 0);
byLineDefine(F, D, byyellow, 1, 0);
draw byNamedLineSeq(0)(FB,FD);
byLineDefine(G, H, black, 0, 0);
byLineDefine(H, B, byblue, 1, 0);
byLineDefine(B, K, black, 0, 0);
byLineDefine(K, C, byred, 0, 0);
byLineDefine(C, L, byyellow, 0, 0);
byLineDefine(L, M, black, 0, 0);
byLineDefine(M, G, black, 0, 0);
draw byNamedLineSeq(0)(GH,HB,BK,KC,CL,LM,MG);
draw byCircleR(F, r2, byred, 0, 0, -1)(F);
}
\drawCurrentPictureInMargin
\problemNP{T}{o}{describe an equilateral pentagon about a given circle \drawCircle[middle][1/4]{F}.}

Draw five tangents through the vertices of the angles of any regular pentagon inscribed in the given circle \circleF\ \inprop[prop:III.XVII].

These five tangents will form the required pentagon.

\startCenterAlign
Draw
$\left\{\vcenter{
\nointerlineskip\hbox{\drawProportionalLine{FB}}
\nointerlineskip\hbox{\drawProportionalLine{FK}}
\nointerlineskip\hbox{\drawProportionalLine{FC}}
\nointerlineskip\hbox{\drawProportionalLine{FL}}
\nointerlineskip\hbox{\drawProportionalLine{FD}}
}\right.$.

In \drawLine{FB,FK,BK} and \drawLine{FK,FC,KC}\\
$\drawProportionalLine{BK} = \drawProportionalLine{KC}$\\
$\drawProportionalLine{FC} = \drawProportionalLine{FB}$, and \drawProportionalLine{FK} common;

$\therefore \drawAngle{FKB} = \drawAngle{CKF}$ and $\therefore \drawAngle{BFK} = \drawAngle{KFC}$ \inprop[prop:I.VIII]

$\therefore \drawAngle{FKB,CKF} = \mbox{ twice } \drawAngle{CKF}$, and $\drawAngle{BFK,KFC} = \mbox{ twice } \drawAngle{KFC}$;

In the same manner it can be demonstrated that\\
$\drawAngle{FLC,DLF} = \mbox{ twice } \drawAngle{FLC}$, and $\drawAngle{CFL,LFD} = \mbox{ twice } \drawAngle{CFL}$;

but $\drawAngle{BFK,KFC} = \drawAngle{CFL,LFD}$ \inprop[prop:III.XXVII],

$\therefore$ their halves $\drawAngle{KFC} = \drawAngle{CFL}$, also $\drawAngle{FCK} = \drawAngle{LCF}$,\\
and \drawProportionalLine{FC} common;

$\therefore \drawAngle{CKF} = \drawAngle{FLC}$ and $\drawProportionalLine{KC} = \drawProportionalLine{CL}$,

$\therefore \drawProportionalLine{KC,CL} = \mbox{ twice } \drawProportionalLine{KC}$;

In the same manner it can be demonstrated that $\drawProportionalLine{BK,HB} = \mbox{ twice } \drawProportionalLine{BK}$,\\
but $\drawProportionalLine{BK} = \drawProportionalLine{KC}$

$\therefore \drawProportionalLine{BK,HB} = \drawProportionalLine{KC,CL}$;
\stopCenterAlign

In the same manner it can be demonstrated that the other sides are equal, and therefore the pentagon is equilateral, it is also equiangular, for

\startCenterAlign
$\drawAngle{FLC,DLF} = \mbox{ twice } \drawAngle{FLC}$ and $\drawAngle{FKB,CKF} = \mbox{ twice } \drawAngle{CKF}$,

and therefore $ \drawAngle{CKF} = \drawAngle{FLC}$,

$\therefore \drawAngle{FLC,DLF} = \drawAngle{FKB,CKF}$;
\stopCenterAlign

in the same manner it can be demonstrated that the other angles of the descried pentagon are equal.

\qed
\stopProposition

\startProposition[title={Prop. XIII. prob.},reference=prop:IV.XIII]
\defineNewPicture[1/2]{
pair A, B, C, D, E, F, G, H, K, L, M;
numeric r[];
r1 := 5/2u;
F := (0, 0);
A := (dir(90+0/5(360))*r1) shifted F;
B := (dir(90+1/5(360))*r1) shifted F;
C := (dir(90+2/5(360))*r1) shifted F;
D := (dir(90+3/5(360))*r1) shifted F;
E := (dir(90+4/5(360))*r1) shifted F;
G := 1/2[A, B];
H := 1/2[B, C];
K := 1/2[C, D];
L := 1/2[D, E];
M := 1/2[E, A];
r2 := abs(F-G);
draw byAngle(G, B, F, byyellow, 0);
draw byAngle(F, B, H, byyellow, 0);
draw byAngleWithName(F, H, C, black, 0)(H);
draw byAngle(H, C, F, byblue, 0);
draw byAngle(F, C, K, byblue, 0);
draw byAngleWithName(C, K, F, black, 0)(K);
draw byAngle(K, D, F, byred, 0);
draw byAngle(F, D, E, byred, 0);
draw byAngle(D, E, F, black, 1);
draw byAngle(F, E, M, black, 1);
draw byLine(F, G, black, 0, 1);
draw byLine(F, M, black, 0, 1);
draw byLine(F, A, black, 0, 1);
draw byLine(F, H, byblue, 1, 0);
draw byLine(F, K, byblue, 0, 0);
draw byLine(F, C, black, 0, 0);
draw byLine(F, D, byyellow, 1, 0);
byLineDefine(F, B, byred, 0, 0);
byLineDefine(F, E, byred, 1, 0);
draw byNamedLineSeq(0)(FB,FE);
byLineDefine(A, B, black, 0, 1);
byLineDefine(B, H, black, 1, 0);
byLineDefine(H, C, byyellow, 0, 0);
byLineDefine(C, K, byyellow, 0, 0);
byLineDefine(K, D, black, 1, 0);
byLineDefine(D, E, black, 0, 1);
byLineDefine(E, A, black, 0, 1);
draw byNamedLineSeq(0)(AB,BH,HC,CK,KD,DE,EA);
draw byCircle(F, H, byyellow, 0, 0, -1)(F);
}
\drawCurrentPictureInMargin
\problemNP{T}{o}{inscribe a circle in a given equiangular and equilateral pentagon.}

Let
\drawFromCurrentPicture[bottom]{
startTempScale(1/3);
draw byNamedLineSeq(0)(EA,DE,KD,CK,HC,BH,AB);
stopTempScale;
}
be a given equi\-an\-gu\-lar and equilateral pentagon; it is required to inscribe a circle in it.

\startCenterAlign
Make $\drawAngle{HCF} = \drawAngle{FCK}$, and $\drawAngle{FDE} = \drawAngle{KDF}$ \inprop[prop:I.IX]

Draw \drawUnitLine{FD}, \drawUnitLine{FC}, \drawUnitLine{FB}, \drawUnitLine{FE}, \&c.

Because $\drawUnitLine{KD,CK} = \drawUnitLine{HC,BH}$, $\drawAngle{HCF} = \drawAngle{FCK}$, and \drawUnitLine{FC} common to the two triangles

\drawFromCurrentPicture{
draw byNamedAngle(FBH,HCF);
draw byNamedLineSeq(0)(FB,FC,HC,BH);
}
and
\drawFromCurrentPicture{
draw byNamedAngle(FCK,KDF);
draw byNamedLineSeq(0)(FC,FD,KD,CK);
};

$\therefore \drawUnitLine{FB} = \drawUnitLine{FD}$ and $\drawAngle{FBH} = \drawAngle{KDF}$ \inprop[prop:I.IV]

And because $\drawAngle{HCF,FCK} = \drawAngle{KDF,FDE} = \mbox{ twice } \drawAngle{KDF}$

$\therefore \mbox{ twice } \drawAngle{FBH}$, hence \drawAngle{HCF,FCK} is bisected by \drawUnitLine{FB}.

In like manner it may be demonstrated that \drawAngle{DEF,FEM} is bisected by \drawUnitLine{FE}, and that the remaining angle of polygon is bisected in a similar manner.

Draw \drawUnitLine{FK}, \drawUnitLine{FH}, \&c. perpendicular to the sides of the pentagon.

Then in the two triangles
\drawFromCurrentPicture{
startTempScale(2/3);
draw byNamedAngle(H,HCF);
draw byNamedLineSeq(0)(FH,FC,HC);
stopTempScale;
}
and
\drawFromCurrentPicture{
startTempScale(2/3);
draw byNamedAngle(FCK,K);
draw byNamedLineSeq(0)(CK,FC,FK);
stopTempScale;
}\\
we have $\drawAngle{HCF} = \drawAngle{FCK}$ (const,), \drawUnitLine{FC} common,\\
and $\drawAngle{H} = \drawAngle{K} = \mbox{ a right angle }$;

$\therefore \drawUnitLine{FK} = \drawUnitLine{FH}$ \inprop[prop:I.XXVI]
\stopCenterAlign

In the same way it may be shown that the five perpendiculars on the sides of the pentagon are equal to one another.

Describe \drawCircle[middle][1/5]{F} with any one of the perpendiculars as radius, and it will be the inscribed circle required. For if it does not touch the sides of the pentagon, but cut them, then a line drawn from the extremity at right angles to the diameter of a circle will fall within the circle, which has been shown to be absurd. \inprop[prop:III.XVI]

\qed
\stopProposition

\startProposition[title={Prop. XIV. prob.},reference=prop:IV.XIV]
\defineNewPicture[1/2]{
pair A, B, C, D, E, F;
numeric r;
r := 9/4u;
F := (0, 0);
A := (dir(90+0/5(360))*r) shifted F;
B := (dir(90+1/5(360))*r) shifted F;
C := (dir(90+2/5(360))*r) shifted F;
D := (dir(90+3/5(360))*r) shifted F;
E := (dir(90+4/5(360))*r) shifted F;
draw byAngle(C, F, B, byblue, 0);
draw byAngle(D, F, C, black, 1);
draw byAngle(B, C, F, black, 0);
draw byAngle(F, C, D, byyellow, 0);
draw byAngle(C, D, F, byyellow, 0);
draw byAngle(F, D, E, byred, 0);
draw byLine(F, A, byyellow, 1, 0);
draw byLine(F, C, byred, 1, 0);
draw byLine(F, D, byblue, 1, 0);
byLineDefine(F, B, black, 0, 0);
byLineDefine(F, E, byyellow, 0, 0);
draw byNamedLineSeq(0)(FB,FE);
byLineDefine(A, B, black, 0, 1);
byLineDefine(B, C, byblue, 0, 0);
byLineDefine(C, D, byred, 0, 0);
byLineDefine(D, E, black, 1, 0);
byLineDefine(E, A, black, 0, 1);
draw byNamedLineSeq(0)(AB,BC,CD,DE,EA);
draw byCircleR(F, r, byred, 0, 0, 1)(F);
}
\drawCurrentPictureInMargin
\problemNP{T}{o}{describe a circle about a given equilateral and equiangular pentagon.}

\startCenterAlign
Bisect \drawAngle{BCF,FCD} and \drawAngle{CDF,FDE} by \drawUnitLine{FC} and \drawUnitLine{FD}, and from the point of section, draw \drawUnitLine{FE}, \drawUnitLine{FA} and \drawUnitLine{FB}.

$\drawAngle{BCF,FCD} = \drawAngle{CDF,FDE}$,\\
$\drawAngle{FCD} = \drawAngle{CDF}$,

$\therefore \drawUnitLine{FD} = \drawUnitLine{FC}$ \inprop[prop:I.VI];

and since in
\drawFromCurrentPicture{
draw byNamedAngle(BCF);
draw byNamedLineSeq(0)(FB,FC,BC);
}
and
\drawFromCurrentPicture{
draw byNamedAngle(FCD);
draw byNamedLineSeq(0)(FC,FD,CD);
},\\
$\drawUnitLine{BC} = \drawUnitLine{CD}$, and \drawUnitLine{FC} common,\\
also $\drawAngle{BCF} = \drawAngle{FCD}$;

$\therefore \drawUnitLine{FB} = \drawUnitLine{FD}$ \inprop[prop:I.IV].

In like manner it may be proved that\\
$\drawUnitLine{FA} = \drawUnitLine{FE} = \drawUnitLine{FB}$.

and therefore $\drawUnitLine{FA} = \drawUnitLine{FB} = \drawUnitLine{FC} = \drawUnitLine{FD} = \drawUnitLine{FE}$.
\stopCenterAlign

Therefore if a circle be described from the point where these five lines meet, with any one of them as a radius, it will circumscribe the given pentagon.

\qed
\stopProposition

\startProposition[title={Prop. XV. prob.},reference=prop:IV.XV]
\defineNewPicture[1/2]{
pair A, B, C, D, E, F, G, H;
numeric r;
r := 9/4u;
G := (0, 0);
A := (dir(90)*r) shifted G;
B := (dir(150)*r) shifted G;
C := (dir(210)*r) shifted G;
D := (dir(270)*r) shifted G;
E := (dir(330)*r) shifted G;
F := (dir(30)*r) shifted G;
H := (dir(270)*r) shifted D;
draw byAngle(A, G, F, black, -1);
draw byAngle(B, G, A, black, -1);
draw byAngle(C, G, B, black, -1);
draw byAngle(D, G, C, byred, 0);
draw byAngle(E, G, D, byblue, 0);
draw byAngle(F, G, E, black, 0);
draw byLine(D, H, byred, 0, 0);
draw byLine(A, D, black, 0, 0);
draw byLine(B, E, byyellow, 0, 0);
draw byLine(C, F, byblue, 0, 0);
draw byLine(A, B, black, 0, 0);
draw byLine(B, C, black, 0, 0);
draw byLine(C, D, black, 1, 0);
draw byLine(D, E, byred, 1, 0);
draw byLine(E, F, byblue, 1, 0);
draw byLine(F, A, black, 0, 0);
draw byCircle(D, G, byred, 0, 0, 0)(D);
draw byCircleR(G, r, byyellow, 0, 0, 0)(G);
byLineDefine(G, C, lineColor.CF, 0, 0);
byLineDefine(G, D, lineColor.AD, 0, 0);
byLineDefine(G, E, lineColor.BE, 0, 0);
}
\drawCurrentPictureInMargin
\problemNP[2]{T}{o}{inscribe an equilateral and equiangular hexagon in a given circle \drawCircle[middle][1/4]{G}.}

From any point in the circumference of the given circle describe \drawCircle[middle][1/4]{D} passing through its centre, and draw the diameters \drawUnitLine{AD}, \drawUnitLine{CF} and \drawUnitLine{BE}; draw \drawUnitLine{CD}, \drawUnitLine{DE}, \drawUnitLine{EF}, \&c. and the required hexagon is inscribed in the given circle.

Since \drawUnitLine{GD} passes through the centres of the circles
\drawFromCurrentPicture{
draw byNamedAngle(DGC);
draw byNamedLineSeq(0)(GC,GD,CD);
}
and
\drawFromCurrentPicture{
draw byNamedAngle(EGD);
draw byNamedLineSeq(0)(GD,GE,DE);
}
are equilateral triangles, hence $\drawAngle{DGC} = \drawAngle{EGD} = \mbox{ one-third of } \drawTwoRightAngles$ \inprop[prop:I.XXXII] but $\drawAngle{DGC,EGD,FGE} = \drawTwoRightAngles$ \inprop[prop:I.XIII];

$\therefore \drawAngle{DGC} = \drawAngle{EGD} = \drawAngle{FGE} = \mbox{ one-third of } \drawTwoRightAngles$ \inprop[prop:I.XXXII], and the angles vertically opposite to these are all equal to one another \inprop[prop:I.XV], and stand on equal arches \inprop[prop:III.XXVI], which are subtended by equal chords \inprop[prop:III.XIX]; and since each of the angles of the hexagon is double of the angle of an equilateral triangle, it is also equiangular.

\qed
\stopProposition

\startProposition[title={Prop. XV. prob.},reference=prop:IV.XV]
\defineNewPicture{
pair A, B, C, D, E, F, G, H, O;
numeric r;
r := 9/4u;
O := (0, 0);
A := (dir(90 + 0*120)*r) shifted O;
C := (dir(90 + 1*120)*r) shifted O;
D := (dir(90 + 2*120)*r) shifted O;
B := (dir(90 + 1*72)*r) shifted O;
F := (dir(90 + 2*72)*r) shifted O;
G := (dir(90 + 3*72)*r) shifted O;
H := (dir(90 + 4*72)*r) shifted O;
draw byLine(C, F, black, 1, 0);
byLineDefine(A, B, byred, 0, 0);
byLineDefine(B, F, byblue, 0, 0);
byLineDefine(F, G, black, 0, 1);
byLineDefine(G, H, black, 0, 1);
byLineDefine(H, A, black, 0, 1);
byLineDefine(A, C, byyellow, 0, 0);
byLineDefine(C, D, black, 0, 1);
byLineDefine(D, A, black, 0, 1);
draw byNamedLineSeq(0)(AB,BF,FG,GH,HA);
draw byNamedLineSeq(0)(AC,CD,DA);
draw byCircleR(O, r, byblue, 0, 0, 1)(O);
}
\drawCurrentPictureInMargin
\problemNP{T}{o}{inscribe a circle in an equilateral and equiangular quindecagon in a given circle.}

Let \drawUnitLine{AB} and \drawUnitLine{BF} be the sides of an equilateral pentagon inscribed in the given circle, and \drawUnitLine{AC} the side of an inscribed equilateral triangle.

\startCenterAlign
\startformula
\left.\vcenter{\hbox{The arc subtended} \hbox{by \drawUnitLine{AB} and \drawUnitLine{BF}}}\right\} = \dfrac{2}{5} = \dfrac{6}{15} \left\{\vcenter{\hbox{of the whole} \hbox{circumference}}\right.
\stopformula

\startformula
\left.\vcenter{\hbox{The arc subtended} \hbox{by \drawUnitLine{AC}}}\right\} = \dfrac{1}{3} = \dfrac{5}{15} \left\{\vcenter{\hbox{of the whole} \hbox{circumference}}\right.
\stopformula

Their difference $ = \dfrac{1}{15}$
\stopCenterAlign

$\therefore$ the arc subtended by $\drawUnitLine{CF} = \dfrac{1}{15}$ difference of the whole circumference.

Hence if straight lines equal to \drawUnitLine{CF} be placed in the circle \inprop[prop:IV.I], an equilateral and equiangular quindecagon will be thus inscribed in the circle.

\qed
\stopProposition
\stopbook

\startbook[title={Book V}]\unskip

\startsupersection[title={Definitions}]\unskip

\startDefinitionOnlyNumber[reference=def:V.I]
A less magnitude is said to be an aliquot part or submultiple of a greater magnitude, when the less measures the greater; that is, when the less is contained a certain number of times exactly in the greater.
\stopDefinitionOnlyNumber

\startDefinitionOnlyNumber[reference=def:V.II]
A greater magnitude is said to be multiple of a less, when the greater is measured be the less; that is, when the greater contains the less a certain number of times exactly.
\stopDefinitionOnlyNumber

\startDefinitionOnlyNumber[reference=def:V.III]
Ratio is the relation which one quantity bears to another of the same kind, with respect to magnitude.
\stopDefinitionOnlyNumber

\startDefinitionOnlyNumber[reference=def:V.IV]
Magnitudes are said to have a ratio to one another, when they are of the same kind, and the one which is not the greater can be multiplied so as to exceed the other.
\stopDefinitionOnlyNumber

\startalignment[middle]
\emph{The other definitions will be given throughout the book where their aid is first required.}
\stopalignment
\stopsupersection

\startsupersection[title={Axioms}]\unskip

\startAxiomOnlyNumber[reference=ax:V.I]
Equimultiples of equisubmultiples of the same, or of equal magnitudes, are equal.

\startalignment[middle]
$\eqalign{
\mbox{If } A &= B \mbox{, then}\cr
\mbox{twice } A &= \mbox{twice } B \mbox{,}\cr
\mbox{that is, }2A &= 2B \mbox{;}\cr
3A &= 3B \mbox{;}\cr
4A &= 4B \mbox{;}\cr
\mbox{\&c.} & \mbox{ \&c.}\cr
\mbox{and } \frac{1}{2} \mbox{ of } A &= \frac{1}{2} \mbox{ of } B \mbox{;}\cr
\frac{1}{3} \mbox{ of } A &= \frac{1}{3} \mbox{ of } B \mbox{;}\cr
\frac{1}{4} \mbox{ of } A &= \frac{1}{4} \mbox{ of } B \mbox{;}\cr
\mbox{\&c. } & \mbox{ \&c.}\cr
}$
\stopalignment
\stopAxiomOnlyNumber

\startAxiomOnlyNumber[reference=ax:V.II]
A multiple of a greater magnitude is greater than the same multiple of a less.

\startalignment[middle]
$\eqalign{
\mbox{Let } A &> B \mbox{,} \cr
\mbox{ then } 2A &> 2B \mbox{;}\cr
3A&> 3B \mbox{;} \cr
4A &> 4B \mbox{;}\cr
\mbox{\&c. } &\mbox{\&c.} \cr
}$
\stopalignment
\stopAxiomOnlyNumber

\pagebreak

\startAxiomOnlyNumber[reference=ax:V.III]
That magnitude, of which a multiple is greater than the same multiple of another, is greater than the other.

\startalignment[middle]
$\eqalign{
\mbox{Let } 2A &> 2B \mbox{,}\cr
\mbox{then } A &> B \mbox{;}\cr
\mbox{or, let } 3A &> 3B \mbox{,}\cr
\mbox{then } A &> B \mbox{;}\cr
\mbox{or, let } mA &> mB  \mbox{,}\cr
\mbox{then } A &> B \mbox{.}\cr
}$
\stopalignment
\stopAxiomOnlyNumber
\stopsupersection

\vfill\pagebreak

\startProposition[title={Prop. I. Theor.}, reference=prop:V.I]
\defineNewPicture{
byMagnitudeSymbolDefine("semicircleUp", byred, 1)(ab);
byMagnitudeSymbolDefine("wedgeDown", byyellow, 0)(cd);
byMagnitudeSymbolDefine("sectorDown", byblue, 1)(ef);
byMagnitudeDefine(A, 0, false)(5)(ab);
byMagnitudeDefine(B, 0, false)(1)(ab);
byMagnitudeDefine(C, 0, false)(5)(cd);
byMagnitudeDefine(D, 0, false)(1)(cd);
byMagnitudeDefine(E, 0, false)(5)(ef);
byMagnitudeDefine(F, 0, false)(1)(ef);
}
\problemNP{I}{f}{any number of magnitudes be equimultiples of as many others, each of each: what multiple soever any one of the first of its part, the same multiple shall of the first magnitudes taken together be of all the others taken together}

\startCenterAlign
Let \drawMagnitude{A} be the same multiple of \drawMagnitude{B},
that \drawMagnitude{C} is of \drawMagnitude{D},\\
that \drawMagnitude{E} is of \drawMagnitude{F}.\\

Then is evident that\\
$\left.\vcenter{
\nointerlineskip\hbox{\drawMagnitude{A}}
\nointerlineskip\hbox{\drawMagnitude{C}}
\nointerlineskip\hbox{\drawMagnitude{E}}
}\right\}$
is the same multiple of
$\left\{\vcenter{
\nointerlineskip\hbox{\drawMagnitude{B}}
\nointerlineskip\hbox{\drawMagnitude{D}}
\nointerlineskip\hbox{\drawMagnitude{F}}
}\right.$\\
which that \drawMagnitude{A} is of \drawMagnitude{B};

because there are as many magnitudes\\
in
$\left{\vcenter{
\nointerlineskip\hbox{\drawMagnitude{A}}
\nointerlineskip\hbox{\drawMagnitude{C}}
\nointerlineskip\hbox{\drawMagnitude{E}}
}\right\}
=
\left\{\vcenter{
\nointerlineskip\hbox{\drawMagnitude{B}}
\nointerlineskip\hbox{\drawMagnitude{D}}
\nointerlineskip\hbox{\drawMagnitude{F}}
}\right.$\\
as there are in $\drawMagnitude{A} = \drawMagnitude{B}$.
\stopCenterAlign

The same demonstration holds in any number of magnitudes, which has here been applied to three.

$\therefore$ If any number of magnitudes, \&c.
\stopProposition

\vfill\pagebreak

\startProposition[title={Prop. II. Theor.}, reference=prop:V.II]
\defineNewPicture{
byMagnitudeSymbolDefine("circle", byyellow, 0)(ab);
byMagnitudeSymbolDefine("sectorUp", byyellow, 1)(cd);
byMagnitudeSymbolDefine("circle", byblue, 0)(e);
byMagnitudeSymbolDefine("sectorUp", black, 1)(f);
byMagnitudeDefine(A, 0, false)(3)(ab);
byMagnitudeDefine(B, 0, false)(1)(ab);
byMagnitudeDefine(C, 0, false)(3)(cd);
byMagnitudeDefine(D, 0, false)(1)(cd);
byMagnitudeDefine(E, 0, false)(4)(e);
byMagnitudeDefine(F, 0, false)(4)(f);
}
\problemNP{I}{f}{the first magnitude be the same multiple of the second that the third of the fourth, and the fifth the same multiple of the second that the sixth is of the fourth, then shall the first, together with the fifth, be the same multiple of the second that the third, together with the sixth, is of the fourth.}

Let \drawMagnitude{A}, the first, be the same multiple of \drawMagnitude{B}, the second, that \drawMagnitude{C}, the third, is of \drawMagnitude{D}, the fourth; and let \drawMagnitude{E}, the fifth, be the same multiple of \drawMagnitude{B}, the second, that \drawMagnitude{F}, the sixth, is of \drawMagnitude{D}, the fourth.

Then it is evident that
$\left{\vcenter{
\nointerlineskip\hbox{\drawMagnitude{A}}
\nointerlineskip\hbox{\drawMagnitude{E}}
}\right\}$,
the first and fifth together, is the same multiple of \drawMagnitude{B}, the second, that
$\left{\vcenter{
\nointerlineskip\hbox{\drawMagnitude{C}}
\nointerlineskip\hbox{\drawMagnitude{F}}
}\right\}$,
the third and sixth together, is of the same multiple of \drawMagnitude{D}, the fourth; because there are as many magnitudes in
$\left{\vcenter{
\nointerlineskip\hbox{\drawMagnitude{A}}
\nointerlineskip\hbox{\drawMagnitude{E}}
}\right\} = \drawMagnitude{B}$
as there are in
$\left{\vcenter{
\nointerlineskip\hbox{\drawMagnitude{C}}
\nointerlineskip\hbox{\drawMagnitude{F}}
}\right\} =\drawMagnitude{D}$.

$\therefore$ If the first magnitude, \&c.
\stopProposition

\vfill\pagebreak

\startProposition[title={Prop. III. Theor.}, reference=prop:V.III]
\defineNewPicture{
byMagnitudeSymbolDefine("square", byyellow, 0)(a);
byMagnitudeSymbolDefine("square", byred, 0)(be);
byMagnitudeSymbolDefine("rhombus", black, 0)(c);
byMagnitudeSymbolDefine("rhombus", byblue, 0)(df);
byMagnitudeDefine(A, 0, false)(1, 2, 1)(a);
byMagnitudeDefine(B, 0, false)(1)(be);
byMagnitudeDefine(C, 0, false)(2, 2)(c);
byMagnitudeDefine(D, 0, false)(1)(df);
byMagnitudeDefine(E, 0, false)(4, 4, 4, 4)(be);
byMagnitudeDefine(F, 0, false)(4, 4, 4, 4)(df);
}
\problemNP{I}{f}{the first of four magnitudes be the same multiple of the second that the third is to the fourth, and if any equimultiples whatever of the first and third be taken, those shall be equimultiples; one of the second, and the other of the fourth.}

\startCenterAlign
Let $\left\{\drawMagnitude{A}\right\}$ be the same multiple of \drawMagnitude{B}\\
which $\left\{\drawMagnitude{C}\right\}$ is of \drawMagnitude{D};\\
take $\left\{\drawMagnitude{E}\right\}$ the same multiple of $\left\{\drawMagnitude{A}\right.$,\\
which $\left\{\drawMagnitude{F}\right\}$ is of $\left\{\drawMagnitude{C}\right.$.

Then it is evident,\\
that $\left\{\drawMagnitude{E}\right\}$ is the same multiple of \drawMagnitude{B}\\
which $\left\{\drawMagnitude{F}\right\}$ if of \drawMagnitude{D};\\
$\because \left\{\drawMagnitude{E}\right\}$ contains $\left\{\drawMagnitude{A}\right\}$ contains \drawMagnitude{B} \\
as many times as \\
$\left.\drawMagnitude{F}\right\}$ contains $\left\{\drawMagnitude{C}\right\}$ contains \drawMagnitude{D}.\\

The same reasoning is applicable in all cases.

$\therefore$ If the first four, \&c.
\stopCenterAlign
\stopProposition

\vfill\pagebreak

\startDefinition[title={Definition V.}, reference=def:V.V]
\defineNewPicture{
byMagnitudeSymbolDefine("circle", byred, 0)(a);
byMagnitudeSymbolDefine("square", byyellow, 0)(b);
byMagnitudeSymbolDefine("rhombus", byblue, 0)(c);
byMagnitudeSymbolDefine("wedgeDown", black, 0)(d);
byMagnitudeDefine(A, 0, false)(1)(a);
byMagnitudeDefine(B, 0, false)(1)(b);
byMagnitudeDefine(C, 0, false)(1)(c);
byMagnitudeDefine(D, 0, false)(1)(d);
byMagnitudeDefine(Am, 1, false)(2, 3, 4, 5, 6)(a);
byMagnitudeDefine(Bm, 1, false)(2, 3, 4, 5, 6)(b);
byMagnitudeDefine(Cm, 1, false)(2, 3, 4, 5, 6)(c);
byMagnitudeDefine(Dm, 1, false)(2, 3, 4, 5, 6)(d);
}

Four magnitudes, \drawMagnitude{A}, \drawMagnitude{B}, \drawMagnitude{C}, \drawMagnitude{D}, are said to be proportionals when every equimultiple of the first and third be taken, and every equimultiple of the second and fourth, as

\sepSpace

\vbox{\hfill\hbox{
\vbox{\hbox{of the first}\hbox{\drawMagnitude{Am}}\hbox{\&c.}}
\vbox{\hbox{of the third}\hbox{\drawMagnitude{Cm}}\hbox{\&c.}}
}\hfill\ }

\sepSpace

\vbox{\hfill\hbox{
\vbox{\hbox{of the second}\hbox{\drawMagnitude{Bm}}\hbox{\&c.}}
\vbox{\hbox{of the fourth}\hbox{\drawMagnitude{Dm}}\hbox{\&c.}}
}\hfill\ }

\sepSpace

Then taking every pair of equimultiples of the first and third, and every pair of equimultiples of the second and fourth,

\startCenterAlign
If
$\left\{\vcenter{
\nointerlineskip\hbox{$\drawMagnitude[middle][1]{Am} >, = \mbox{or} < \drawMagnitude[middle][1]{Bm}$}
\nointerlineskip\hbox{$\drawMagnitude[middle][1]{Am} >, = \mbox{or} < \drawMagnitude[middle][2]{Bm}$}
\nointerlineskip\hbox{$\drawMagnitude[middle][1]{Am} >, = \mbox{or} < \drawMagnitude[middle][3]{Bm}$}
\nointerlineskip\hbox{$\drawMagnitude[middle][1]{Am} >, = \mbox{or} < \drawMagnitude[middle][4]{Bm}$}
\nointerlineskip\hbox{$\drawMagnitude[middle][1]{Am} >, = \mbox{or} < \drawMagnitude[middle][5]{Bm}$}
}\right.$

$\vcenter{\hbox{then}\hbox{will}}\left\{\vcenter{
\nointerlineskip\hbox{$\drawMagnitude[middle][1]{Cm} >, = \mbox{or} < \drawMagnitude[middle][1]{Dm}$}
\nointerlineskip\hbox{$\drawMagnitude[middle][1]{Cm} >, = \mbox{or} < \drawMagnitude[middle][2]{Dm}$}
\nointerlineskip\hbox{$\drawMagnitude[middle][1]{Cm} >, = \mbox{or} < \drawMagnitude[middle][3]{Dm}$}
\nointerlineskip\hbox{$\drawMagnitude[middle][1]{Cm} >, = \mbox{or} < \drawMagnitude[middle][4]{Dm}$}
\nointerlineskip\hbox{$\drawMagnitude[middle][1]{Cm} >, = \mbox{or} < \drawMagnitude[middle][5]{Dm}$}
}\right.$
\stopCenterAlign

That is, if twice the first be greater, equal or less than twice the third will be greater, equal, or less than twice the fourth; or, if twice the first be greater, equal, or less than three times the second, twice the third will be greater, equal, or less than three times the fourth, and so on, as above expressed.

\startCenterAlign
If
$\left\{\vcenter{
\nointerlineskip\hbox{$\drawMagnitude[middle][2]{Am} >, = \mbox{or} < \drawMagnitude[middle][1]{Bm}$}
\nointerlineskip\hbox{$\drawMagnitude[middle][2]{Am} >, = \mbox{or} < \drawMagnitude[middle][2]{Bm}$}
\nointerlineskip\hbox{$\drawMagnitude[middle][2]{Am} >, = \mbox{or} < \drawMagnitude[middle][3]{Bm}$}
\nointerlineskip\hbox{$\drawMagnitude[middle][2]{Am} >, = \mbox{or} < \drawMagnitude[middle][4]{Bm}$}
\nointerlineskip\hbox{$\drawMagnitude[middle][2]{Am} >, = \mbox{or} < \drawMagnitude[middle][5]{Bm}$}
}\right.$

$\vcenter{\hbox{then}\hbox{will}}\left\{\vcenter{
\nointerlineskip\hbox{$\drawMagnitude[middle][2]{Cm} >, = \mbox{or} < \drawMagnitude[middle][1]{Dm}$}
\nointerlineskip\hbox{$\drawMagnitude[middle][2]{Cm} >, = \mbox{or} < \drawMagnitude[middle][2]{Dm}$}
\nointerlineskip\hbox{$\drawMagnitude[middle][2]{Cm} >, = \mbox{or} < \drawMagnitude[middle][3]{Dm}$}
\nointerlineskip\hbox{$\drawMagnitude[middle][2]{Cm} >, = \mbox{or} < \drawMagnitude[middle][4]{Dm}$}
\nointerlineskip\hbox{$\drawMagnitude[middle][2]{Cm} >, = \mbox{or} < \drawMagnitude[middle][5]{Dm}$}
}\right.$
\stopCenterAlign

In other terms, if three times the first be greater, equal, or less than twice the second, three times the third will be greater, equal, or less than twice the fourth; or, if three times the first be greater, equal, or less than three times the second, then will three times the third be greater, equal, or less than three times the fourth; or if three times the first be greater, equal, or less than four times the second, then will three times the third be greater, equal, or less than four times the fourth, and so on. Again,

\startCenterAlign
If
$\left\{\vcenter{
\nointerlineskip\hbox{$\drawMagnitude[middle][3]{Am} >, = \mbox{or} < \drawMagnitude[middle][1]{Bm}$}
\nointerlineskip\hbox{$\drawMagnitude[middle][3]{Am} >, = \mbox{or} < \drawMagnitude[middle][2]{Bm}$}
\nointerlineskip\hbox{$\drawMagnitude[middle][3]{Am} >, = \mbox{or} < \drawMagnitude[middle][3]{Bm}$}
\nointerlineskip\hbox{$\drawMagnitude[middle][3]{Am} >, = \mbox{or} < \drawMagnitude[middle][4]{Bm}$}
\nointerlineskip\hbox{$\drawMagnitude[middle][3]{Am} >, = \mbox{or} < \drawMagnitude[middle][5]{Bm}$}
}\right.$

$\vcenter{\hbox{then}\hbox{will}}\left\{\vcenter{
\nointerlineskip\hbox{$\drawMagnitude[middle][3]{Cm} >, = \mbox{or} < \drawMagnitude[middle][1]{Dm}$}
\nointerlineskip\hbox{$\drawMagnitude[middle][3]{Cm} >, = \mbox{or} < \drawMagnitude[middle][2]{Dm}$}
\nointerlineskip\hbox{$\drawMagnitude[middle][3]{Cm} >, = \mbox{or} < \drawMagnitude[middle][3]{Dm}$}
\nointerlineskip\hbox{$\drawMagnitude[middle][3]{Cm} >, = \mbox{or} < \drawMagnitude[middle][4]{Dm}$}
\nointerlineskip\hbox{$\drawMagnitude[middle][3]{Cm} >, = \mbox{or} < \drawMagnitude[middle][5]{Dm}$}
}\right.$
\stopCenterAlign

And so on, with any other equimultiples of the four magnitudes, taken in the same manner.

Euclid expresses this definition as follows :—

The first of four magnitudes is said to have the same ratio to the second, which the third has to the fourth, when equimultiples whatsoever of the first and third being taken, and any equimultiples whatsoever of the second and fourth; if the multiple of the first be less than that of the second, and the multiple of the third is also less than that of the fourth; or, if the multiple of the first be equal to that of the second, the multiple of the third is also equal to that of the fourth; or, if the multiple of the first be greater than that of the second, the multiple of the third is also greater than that of the fourth.

In future we shall express this definition generally, thus:

\startCenterAlign
$\eqalign{
\mbox{If }M \drawMagnitude{A} &>, = \mbox{ or } < m \drawMagnitude{B} \mbox{,}\cr
\mbox{If }M \drawMagnitude{C} &>, = \mbox{ or } < m \drawMagnitude{D} \mbox{,}\cr
}$
\stopCenterAlign

Then we infer that \drawMagnitude{A}, the first, has the same ratio to \drawMagnitude{B}, the second, which \drawMagnitude{C}, the third, has to \drawMagnitude{D} the fourth; expressed in the succeeding demonstrations thus:

\startCenterAlign
$\eqalign{
\drawMagnitude{A} : \drawMagnitude{B} & :: \drawMagnitude{C} : \drawMagnitude{D} \mbox{;} \cr
\mbox{or thus, } \drawMagnitude{A} : \drawMagnitude{B} &= \drawMagnitude{C} : \drawMagnitude{D} \mbox{;} \cr
\mbox{or thus, } \dfrac{\drawMagnitude{A}}{\drawMagnitude{B}} &= \dfrac{\drawMagnitude{C}}{\drawMagnitude{D}} \mbox{:} \cr
}$

and is read, \\
\quotation{as \drawMagnitude{A} is to \drawMagnitude{B}, so is \drawMagnitude{C} to \drawMagnitude{D}.}

And if $\drawMagnitude{A} : \drawMagnitude{B} :: \drawMagnitude{C} : \drawMagnitude{D}$ we shall infer if \\
$M \drawMagnitude{A} >, = \mbox{ or } < m \drawMagnitude{B}$, then will \\
$M \drawMagnitude{C} >, = \mbox{ or } < m \drawMagnitude{D}$.
\stopCenterAlign

That is, if the first be to the second, as the third is to the fourth; then if $M$ times the first be greater than, equal to, or less than $m$ times the second, then shall $M$ times third be greater than, equal to, or less than $m$ times the fourth, in which $M$ and $m$ are not to be considered particular multiples, but every pair of multiples whatever; nor are such marks as \drawMagnitude{A}, \drawMagnitude{D}, \drawMagnitude{B}, \&c. to be considered any more than representatives of geometrical magnitudes.

The student should throughly understand this definition before proceeding further.
\stopDefinition

\vfill\pagebreak

\startProposition[title={Prop. IV. Theor.}, reference=prop:V.IV]
\defineNewPicture{
byMagnitudeSymbolDefine("circle", byyellow, 0)(I);
byMagnitudeSymbolDefine("square", black, 0)(II);
byMagnitudeSymbolDefine("rhombus", byred, 0)(III);
byMagnitudeSymbolDefine("wedgeDown", byblue, 0)(IV);
}
\problemNP{I}{f}{the first of four magnitudes have the same ratio to the second, which the third has to the fourth, then any equimultiples whatever of the first and third shall have the same ratio to any equimultiples of the second and fourth; viz., the equimultiple of the first shall have the same ratio to that of the second, which the equimultiple of the third has to that of the fourth.}

Let $\drawMagnitude{I} : \drawMagnitude{II} :: \drawMagnitude{III} : \drawMagnitude{IV}$, then $3\drawMagnitude{I} : 2\drawMagnitude{II} :: 3\drawMagnitude{III} : 2\drawMagnitude{IV}$, every equimultiple of $3\drawMagnitude{I}$ and $3\drawMagnitude{III}$ are equimultiples of \drawMagnitude{I} and \drawMagnitude{III}, and every equimultiple of $2\drawMagnitude{II}$ and $2\drawMagnitude{IV}$, are equimultiples of \drawMagnitude{II} and \drawMagnitude{IV} \inprop[prop:V.III]

That is, $M$ times $3\drawMagnitude{I}$ and $M$ times $3\drawMagnitude{III}$ are equimultiples of \drawMagnitude{I} and \drawMagnitude{III}, and $m$ times $2\drawMagnitude{II}$ and $m 2\drawMagnitude{IV}$ are equimultiples of $2\drawMagnitude{II}$ and $2\drawMagnitude{IV}$; but $\drawMagnitude{I} : \drawMagnitude{II} :: \drawMagnitude{III} : \drawMagnitude{IV}$ (hyp.); $\therefore$ if $M 3\drawMagnitude{I} <, =, \mbox{ or } > m 2 \drawMagnitude{II}$, then $M 3\drawMagnitude{III} <, =, \mbox{ or } > m 2 \drawMagnitude{IV}$ \indef[def:V.V] and therefore $3\drawMagnitude{I} : 2\drawMagnitude{II} :: 3\drawMagnitude{III} : 2\drawMagnitude{IV}$ \indef[def:V.V]

The same reasoning holds good if any other equimultiple of the first and third be taken, any other equimultiple of the second and fourth.

$\therefore$ If the first four magnitudes, \&c.
\stopProposition

\vfill\pagebreak

\startProposition[title={Prop. V. Theor.}, reference=prop:V.V]
\defineNewPicture{
byMagnitudeSymbolDefine("sectorDown", byblue, 1)(a);
byMagnitudeSymbolDefine("semicircleDown", byyellow, 1)(b);
byMagnitudeSymbolDefine("miniTriangleUp", black, 0)(c);
byMagnitudeSymbolDefine("miniSquare", byred, 0)(d);
byMagnitudeDefine(I, 0, false)(1, 2, 1)(a, a, b);
byMagnitudeDefine(II, 0, false)(1, 1)(c, d);
}
\problemNP{I}{f}{one magnitude be the same multiple of another, which a magnitude taken from the first is of a magnitude taken from the other, the remainder shall be the same multiple of the remainder, that the whole is of the whole.}

\startCenterAlign
Let $\drawMagnitude{I} = M' \drawMagnitude{II}$\\
and $\drawMagnitude[middle][3]{I} = M' \drawMagnitude[middle][2]{II}$,

$\therefore \drawMagnitude{I} \mbox{ minus } \drawMagnitude[middle][3]{I} = M' \drawMagnitude{II} \mbox{ minus } M' \drawMagnitude[middle][2]{II}$,

$\therefore \drawMagnitude[middle][-3]{I} = M' (\drawMagnitude{II} \mbox{ minus } \drawMagnitude[middle][2]{II})$,\\
and $\therefore \drawMagnitude[middle][-3]{I} = M' \drawMagnitude[middle][-2]{II}$.

$\therefore$ If one magnitude, \&c.
\stopCenterAlign
\stopProposition

\vfill\pagebreak

\startProposition[title={Prop. VI. Theor.}, reference=prop:V.VI]
\defineNewPicture{
byMagnitudeSymbolDefine("sectorDown", byyellow, 1)(a);
byMagnitudeSymbolDefine("miniSquare", byred, 0)(b);
byMagnitudeSymbolDefine("semicircleDown", black, 1)(c);
byMagnitudeSymbolDefine("miniTriangleUp", byblue, 0)(d);
byMagnitudeDefine(I, 0, false)(1, 2, 1)(a);
byMagnitudeDefine(II, 0, false)(1)(b);
byMagnitudeDefine(III, 0, false)(2)(c);
byMagnitudeDefine(IV, 0, false)(1)(d);
}
\problemNP{I}{f}{two magnitudes be equimultiples of two others, and if equimultiples of these be taken from the first two, the remainders are either equal to these others, or equimultiples of them.}

\startCenterAlign
Let $\drawMagnitude{I} = M'\drawMagnitude{II}$; and $\drawMagnitude{III} = M'\drawMagnitude{IV}$

then $\drawMagnitude{I} \mbox{ minus } m'\drawMagnitude{II} = M'\drawMagnitude{II} \mbox{ minus } m'\drawMagnitude{II} = (M' \mbox{ minus } m') \drawMagnitude{II}$\\
and $\drawMagnitude{III} \mbox{ minus } m'\drawMagnitude{IV} = M'\drawMagnitude{IV} \mbox{ minus } m'\drawMagnitude{IV} = (M' \mbox{ minus } m') \drawMagnitude{IV}$.

Hence, $(M' \mbox{ minus } m') \drawMagnitude{II}$ and $(M' \mbox{ minus } m') \drawMagnitude{IV}$ are equimultiples of \drawMagnitude{II} and \drawMagnitude{IV}, and equal to \drawMagnitude{II} and \drawMagnitude{IV}, when $M' \mbox{ minus } m' = 1$.

$\therefore$ If two magnitudes be equimultiples, \&c.
\stopCenterAlign
\stopProposition

\vfill\pagebreak

\startPropositionAZ[title={Prop. A. Theor.}, reference=prop:V.A]
\defineNewPicture{
byMagnitudeSymbolDefine("circle", byred, 0)(a);
byMagnitudeSymbolDefine("square", black, 0)(b);
byMagnitudeSymbolDefine("wedgeDown", byblue, 0)(c);
byMagnitudeSymbolDefine("rhombus", byyellow, 0)(d);
byMagnitudeDefine(I, 0, false)(1)(a);
byMagnitudeDefine(II, 0, false)(1)(b);
byMagnitudeDefine(III, 0, false)(1)(c);
byMagnitudeDefine(IV, 0, false)(1)(d);
byMagnitudeDefine(dI, 0, false)(2)(a);
byMagnitudeDefine(dII, 0, false)(2)(b);
byMagnitudeDefine(dIII, 0, false)(2)(c);
byMagnitudeDefine(dIV, 0, false)(2)(d);
}
\problemNP{I}{f}{the first of the four magnitudes has the same ratio to the second which the third has to the fourth, then if the first be greater than the second, the third is also greater than the fourth; and if equal, equal; if less, less.}

\startCenterAlign
Let $\drawMagnitude{I} : \drawMagnitude{II} :: \drawMagnitude{III} : \drawMagnitude{IV}$; therefore, by the fifth definition, if $\drawMagnitude{dI} > \drawMagnitude{dII}$, then will $\drawMagnitude{dIII} > \drawMagnitude{dIV}$;

but if $\drawMagnitude{I} > \drawMagnitude{II}$, then $\drawMagnitude{dI} > \drawMagnitude{dII}$ and $\drawMagnitude{dIII} > \drawMagnitude{dIV}$,\\
and $\therefore \drawMagnitude{III} > \drawMagnitude{IV}$.

Similarly, if $\drawMagnitude{I} =, \mbox{ or } < \drawMagnitude{II}$, then will $\drawMagnitude{III} =, \mbox{ or } < \drawMagnitude{IV}$.

$\therefore$ If the first of four, \&c.
\stopCenterAlign
\stopPropositionAZ

\startDefinition[title={Definition XIV},reference=def:V.XIV]
\def\varA{\color[byred]{A}}
\def\varB{\color[black]{B}}
\def\varC{\color[byblue]{C}}
\def\varD{\color[byyellow]{D}}
Geometricians make use of the technical term \quotation{Invertendo,} by inversion, when there are four proportionals, and it is inferred, that the second is to the first as the fourth to the third.

Let $\varA : \varB :: \varC : \varD$, then, by \quotation{invertendo} it is inferred $\varB : \varA :: \varD : \varC$
\stopDefinition

\vfill\pagebreak

\startPropositionAZ[title={Prop. B. Theor.}, reference=prop:V.B]
\defineNewPicture{
byMagnitudeSymbolDefine("wedgeDown", byblue, 0)(I);
byMagnitudeSymbolDefine("semicircleDown", black, 1)(II);
byMagnitudeSymbolDefine("square", byred, 0)(III);
byMagnitudeSymbolDefine("rhombus", byyellow, 0)(IV);
}
\problemNP{I}{f}{four magnitudes are proportionals, they are proportionals also when taken inversely.}

\startCenterAlign
Let $\drawMagnitude{I} : \drawMagnitude{II} :: \drawMagnitude{III} : \drawMagnitude{IV}$,\\
then, inversely, $\magnitudeII : \magnitudeI :: \magnitudeIV : \magnitudeIII$.

If $M \magnitudeI < m \magnitudeII$, then $M \magnitudeIII < m \magnitudeIV$\\
by the fifth definition.

Let $M \magnitudeI < m \magnitudeII$, that is, $m \magnitudeII > M \magnitudeI$,\\
$\therefore M \magnitudeIII < m \magnitudeIV$, or, $m \magnitudeIV > M \magnitudeIII$;

$\therefore$ if $m \magnitudeII > M \magnitudeI$, then will $m \magnitudeIV > M \magnitudeIII$.

In the same manner it may be shown,\\
that if $m \magnitudeII = \mbox{ or } < M \magnitudeI$,\\
then will $m \magnitudeIV =, \mbox{ or } < M \magnitudeIII$;

and therefore, by the fifth definition, we infer\\
that $\magnitudeII : \magnitudeI :: \magnitudeIV : \magnitudeIII$.

$\therefore$ If four magnitudes, \&c.
\stopCenterAlign
\stopPropositionAZ

\vfill\pagebreak

\startPropositionAZ[title={Prop. C. Theor.}, reference=prop:V.C]
\defineNewPicture{
byMagnitudeSymbolDefine("square", byblue, 0)(i);
byMagnitudeDefine(I, 0, false)(2, 2)(i);
byMagnitudeSymbolDefine("circle", black, 0)(II);
byMagnitudeSymbolDefine("rhombus", byyellow, 0)(iii);
byMagnitudeDefine(III, 0, false)(2, 2)(iii);
byMagnitudeSymbolDefine("wedgeUp", byred, 0)(IV);
}
\problemNP{I}{f}{four magnitudes are proportionals, they are proportionals also when taken inversely.}

\startCenterAlign
Let \drawMagnitude{I}, the first, be the same multiple of \drawMagnitude{II}, the second, \\
that \drawMagnitude{III}, the third, is of \drawMagnitude{IV}, the fourth.

Then $\magnitudeI : \magnitudeII :: \magnitudeIII : \magnitudeIV$

take $M \magnitudeI$, $m \magnitudeII$, $M \magnitudeIII$, $m \magnitudeIV$;

because \magnitudeI\ is the same multiple of \magnitudeII \\
that \magnitudeIII is of \magnitudeIV\ (according to the hypothesis);

and $M \magnitudeI$ if taken the same multiple of \magnitudeI \\
that $M \magnitudeIII$ is of \magnitudeIII,

$\therefore$ (according to the third proposition),\\
$M \magnitudeI$ is the same multiple of \magnitudeII \\
that $M \magnitudeIII$ is of \magnitudeIV.

Therefore, if $M \magnitudeI$ be of \magnitudeII\ a greater multiple than $m \magnitudeII$ is, \\
then $M \magnitudeIII$ is a greater multiple of \magnitudeIV\ than $m \magnitudeIV$ is;

that is, if $M \magnitudeI$ be greater than $m \magnitudeII$, then $M \magnitudeIII$ will be greater than $m \magnitudeIV$;

in the same manner it can be shown, if $M \magnitudeI$ be equal $m \magnitudeII$, then $M \magnitudeIII$ will be equal $m \magnitudeIV$.

And, generally, if $M \magnitudeI >, = \mbox{ or } < m \magnitudeII$\\
than $M \magnitudeIII \mbox{ will be } >, = \mbox{ or } < m \magnitudeIV$;

$\therefore$ by the fifth definition,\\
$\magnitudeI : \magnitudeII :: \magnitudeIII : \magnitudeIV$

Next, let \magnitudeII be the same part of \magnitudeI \\
that \magnitudeIV\ is of \magnitudeIII.

In this case also $\magnitudeII : \magnitudeI :: \magnitudeIV : \magnitudeIII$.

For, because \magnitudeII\ is the same part of \magnitudeI\ that \magnitudeIV\ is of \magnitudeIII,

therefore \magnitudeI\ is the same multiple of \magnitudeII \\
that \magnitudeIII\ is of \magnitudeIV.

Therefore, be the preceding case, \\
$\magnitudeI : \magnitudeII :: \magnitudeIII : \magnitudeIV$;

and $\therefore \magnitudeII : \magnitudeI :: \magnitudeIV : \magnitudeIII$, \\
by proposition B.

$\therefore$ If the first be the same multiple, \&c.
\stopCenterAlign
\stopPropositionAZ

\vfill\pagebreak

\startPropositionAZ[title={Prop. D. Theor.}, reference=prop:V.D]
\defineNewPicture{
byMagnitudeSymbolDefine("circle", byyellow, 0)(i);
byMagnitudeSymbolDefine("square", black, 0)(ii);
byMagnitudeSymbolDefine("rhombus", byred, 0)(iii);
byMagnitudeSymbolDefine("wedgeDown", byblue, 0)(iv);
byMagnitudeSymbolDefine("semicircleDown", byred, 1)(va);
byMagnitudeSymbolDefine("semicircleUp", byred, 1)(vb);
byMagnitudeSymbolDefine("sectorDown", black, 1)(via);
byMagnitudeSymbolDefine("sectorUp", black, 1)(vib);
byMagnitudeDefine(I, 0, false)(1, 2)(i);
byMagnitudeDefine(II, 0, false)(1)(ii);
byMagnitudeDefine(III, 0, false)(2, 2)(iii);
byMagnitudeDefine(IV, 0, false)(1)(iv);
byMagnitudeDefine(V, 0, false)(1, 2)(va, vb);
byMagnitudeDefine(VI, 0, false)(2, 2)(via, vib);
}
\problemNP{I}{f}{the first be to the second as the third to the fourth, and if the first be a multiple, or a part of the second; the third is the same multiple, or the same part of the fourth.}

\startCenterAlign
Let $\drawMagnitude{I} : \drawMagnitude{II} :: \drawMagnitude{III} : \drawMagnitude{IV}$;\\
and first, let \magnitudeI\ be a multiple \magnitudeII;\\
\magnitudeIII\ shall be the same multiple of \magnitudeIV.

\unprotect
\ \hfill\vbox{\halign{\hfil # \hfil & \hfil # \hfil & \hfil # \hfil & \hfil # \hfil\cr
{\tfxx First} & {\tfxx Second} & {\tfxx Third} & {\tfxx Fourth} \cr
\magnitudeI\ & \magnitudeII\ & \magnitudeIII\ & \magnitudeIV \cr
&\drawMagnitude{V}\ &\drawMagnitude{VI}\ & \cr
}}\hfill\
\protect

Take $\magnitudeV = \magnitudeI$.

Whatever miltiple \magnitudeI\ is of \magnitudeII\\
take \magnitudeVI\ the same multiple of \magnitudeIV, \\
then, because $\magnitudeI : \magnitudeII :: \magnitudeIII : \magnitudeIV$\\
and of the second and fourth, we have taken equimultiples,\\
\magnitudeI\ and \magnitudeVI\ therefore \inprop[prop:V.IV]\\
 $\magnitudeI : \magnitudeV :: \magnitudeIII : \magnitudeVI$, but (const.),\\
 $\magnitudeI = \magnitudeV \therefore$ \inprop[prop:V.A] $\magnitudeIII = \magnitudeVI$\\
 and \magnitudeVI\ is the same multiple of \magnitudeIV\\
 that \magnitudeI\ is of \magnitudeII.

 Next, let $\magnitudeII : \magnitudeI :: \magnitudeIV : \magnitudeIII$,\\
 and also \magnitudeII\ a part of \magnitudeI;\\
 then \magnitudeIV\ shall be the same part of \magnitudeIII.

Inversely (B. 5.), $\magnitudeI : \magnitudeII :: \magnitudeIII : \magnitudeIV$,\\
but \magnitudeII\ is a part of \magnitudeI;\\
that is, \magnitudeI\ is a multiple of \magnitudeII;\\
$\therefore$ by the preceding case, \magnitudeIII\ is the same multiple of \magnitudeIV\\
that is, \magnitudeIV\ is the same part of \magnitudeIII\\
that \magnitudeII\ is of \magnitudeI.

$\therefore$ if the first be to the second, \&c.
\stopCenterAlign
\stopPropositionAZ

\vfill\pagebreak

\startProposition[title={Prop. VII. Theor.}, reference=prop:V.VII]
\defineNewPicture{
byMagnitudeSymbolDefine("circle", byred, 0)(I);
byMagnitudeSymbolDefine("rhombus", byblue, 0)(II);
byMagnitudeSymbolDefine("square", byyellow, 0)(III);
}
\problemNP{E}{qual}{magnitudes have the same ratio to the same magnitude, and the same has the same ratio to equal magnitudes.}

\startCenterAlign
Let $\drawMagnitude{I} = \drawMagnitude{II}$ and \drawMagnitude{III} any other magnitude;\\
then $\magnitudeI : \magnitudeIII = \magnitudeII : \magnitudeIII$ and $\magnitudeIII : \magnitudeI = \magnitudeIII : \magnitudeII$.

Because $\magnitudeI = \magnitudeII$,\\
$\therefore M \magnitudeI = M \magnitudeII$;

$\therefore$ if $M \magnitudeI >, = \mbox{ or } < m \magnitudeIII$, then\\
$M \magnitudeII >, = \mbox{ or } < m \magnitudeIII$,\\
and $\therefore \magnitudeI : \magnitudeIII = \magnitudeII : \magnitudeIII$ \indef[def:V.V].

From the foregoing reasoning it is evident that,\\
if $m \magnitudeIII >, = \mbox{ or } < M \magnitudeI$, then\\
$m \magnitudeIII >, = \mbox{ or } < M \magnitudeII$\\
$\therefore \magnitudeIII : \magnitudeI = \magnitudeIII : \magnitudeII$ \indef[def:V.V].

$\therefore$ Equal magnitudes, \&c.
\stopCenterAlign
\stopProposition

\vfill\pagebreak

\startDefinition[title={Definition VII.}, reference=def:V.VII]
\defineNewPicture{
byMagnitudeSymbolDefine("circle", byred, 0)(i);
byMagnitudeSymbolDefine("square", byyellow, 0)(ii);
byMagnitudeSymbolDefine("rhombus", byblue, 0)(iii);
byMagnitudeSymbolDefine("wedgeDown", black, 0)(iv);
byMagnitudeDefine(I, 0, false)(5)(i);
byMagnitudeDefine(II, 0, false)(4)(ii);
byMagnitudeDefine(III, 0, false)(5)(iii);
byMagnitudeDefine(IV, 0, false)(4)(iv);
byMagnitudeSymbolDefine("wedgeDown", byred, 0)(Ia);
byMagnitudeSymbolDefine("semicircleDown", black, 1)(IIa);
byMagnitudeSymbolDefine("square", byblue, 0)(IIIa);
byMagnitudeSymbolDefine("rhombus", byyellow, 0)(IVa);
}

When of the equimultiples of four magnitudes (taken as in the fifth definition), the multiple of the first is greater than that of the second, but multiple of the third is not greater than the multiple of the fourth, then the first is said to have to the second a greater ratio than the third magnitude has to the fourth: and, on the contrary, the third is said to have to the fourth a less ratio than the first has to the second.

If, among the equimultiples of four magnitudes, compared in the fifth definition, we should find $\drawMagnitude{I} > \drawMagnitude{II}$, but $\drawMagnitude{III} = \mbox{ or } > \drawMagnitude{IV}$, or if we should find any particular multiple $M'$ of the first and third, and a particular multiple $m'$ of the second and fourth, such that $M'$ times the first is $> m'$ times the second, but $M'$ times the third is not $> m'$ times the fourth, i. e. $= \mbox{ or } < m'$ times the fourth; then the first is said to have to the second a greater ratio than the third has to the fourth; or the third has to the fourth, under such circumstances, a less ratio than the first has to the second: although several other equimultiples may tend to show that the four magnitudes are proportionals.

This definition will in future be expressed thus :—

\startCenterAlign
If $M' \drawMagnitude{Ia} > m' \drawMagnitude{IIa}$, but $M' \drawMagnitude{IIIa} = \mbox{ or } < m' \drawMagnitude{IVa}$,\\
then $\magnitudeIa : \magnitudeIIa > \magnitudeIIIa : \magnitudeIVa$.
\stopCenterAlign

In the above general expression, $M'$ and $m'$ are to be considered particular multiples, not like the multiples $M$ and $m$ introduced in the fifth definition, which are in that definition considered to be every pair of multiples that can be taken. It must also be here observed, that \magnitudeIa, \magnitudeIIa, \magnitudeIIIa, and like symbols are to be considered merely the representatives of geometrical magnitudes.

In a partial arithmetical way, this may be set forth as follows:

Let us take the four numbers $\color[byred]{8}$, $\color[black]{7}$, $\color[byblue]{10}$, and $\color[byyellow]{9}$.

\startCenterAlign

\unprotect
\ \hfill\vbox{
\offinterlineskip
\tabskip=0pt
\halign{\vrule height1.9ex depth0.9ex
	\hfil \color[byred] { # } \hfil &
	\hfil \color[black] { # } \hfil &
	\hfil \color[byblue] { # } \hfil &
	\hfil \color[byyellow] { # } \hfil \vrule \cr
	\noalign{\hrule}
	\color[black]{First} &
	\color[black]{Second} &
	\color[black]{Third} &
	\color[black]{Fourth} \cr
	8 	& 7 		& 10 	& 9 \cr
	\noalign{\hrule}
	16 	& 14 	& 20 	& 18 \cr
	24 	& 21 	& 30 	& 27 \cr
	32 	& 28 	& 40 	& 36 \cr
	40 	& 35 	& 50 	& 45 \cr
	48 	& 42 	& 60 	& 54 \cr
	56 	& 49 	& 70 	& 63 \cr
	64 	& 56 	& 80 	& 72 \cr
	72 	& 63 	& 90 	& 81 \cr
	80 	& 70 	& 100 	& 90 \cr
	88 	& 77 	& 110 	& 99 \cr
	96 	& 84 	& 120 	& 108 \cr
	104 	& 91 	& 130 	& 117 \cr
	112 	& 98 	& 140 	& 126 \cr
	\color[black]{ \&c.} &
	\color[black]{ \&c.} &
	\color[black]{ \&c.} &
	\color[black]{ \&c.} \cr
	\noalign{\hrule}
}}\hfill\
\protect
\stopCenterAlign

Among the above multiples we find $\color[byred]{16} > \color[black]{14}$ and $\color[byblue]{20} > \color[byyellow]{18}$; that is, twice the first is greater than twice the second, and twice the third is greater than twice the fourth; and $\color[byred]{16} < \color[black]{21}$ and $\color[byblue]{20} < \color[byyellow]{27}$; that is, twice the first is less than three times the second, and twice the third is less than three times the fourth; and among the same multiples we can find $\color[byred]{72} > \color[black]{56}$ and $\color[byblue]{90} > \color[byyellow]{72}$; that is, 9 times the first is greater than 8 times the second, and 9 times the third is greater than 8 times the fourth. Many other equimultiples might be selected, which would tend to show that the numbers $\color[byred]{8}$, $\color[black]{7}$, $\color[byblue]{10}$, and $\color[byyellow]{9}$ were proportionals, but they are not, for we can find multiple of the first $>$ a multiple of the second, but the same multiple of the third that has been taken of the first, not $>$ the same multiple of the fourth which has been taken to the second; for instance, 9 times the first is $>$ 10 times the second, but 9 times the third is not $>$ 10 times the fourth, that is $\color[byred]{72} > \color[black]{70}$, but $\color[byblue]{90} \mbox{ not } > \color[byyellow]{90}$, or 8 times the thirst we find $>$ 9 times the second, but 8 times the third is not greater than 9 times the fourth, that is, $\color[byred]{64} > \color[black]{63}$, but $\color[byblue]{80} \mbox{ not } > \color[byyellow]{81}$. When any such multiples as these can be found, the first $\color[byred]{(8)}$ is said to have the second $\color[black]{(7)}$ a greater ratio than the third $\color[byblue]{(10)}$ has to the fourth $\color[byyellow]{(9)}$, and on the contrary the third $\color[byblue]{(10)}$ is said to have to the fourth $\color[byyellow]{(9)}$ a less ratio than the first $\color[byred]{(8)}$ has to the second $\color[black]{(7)}$.
\stopDefinition

\vfill\pagebreak

\startProposition[title={Prop. VIII. Theor.}, reference=prop:V.VIII]
\defineNewPicture{
byMagnitudeSymbolDefine("miniTriangleUp", black, 0)(Ia);
byMagnitudeSymbolDefine("square", byred, 0)(Ib);
byMagnitudeDefine(I, 0, false)(1, 1)(Ia, Ib);
byMagnitudeSymbolDefine("square", byyellow, 0)(II);
byMagnitudeSymbolDefine("circle", byblue, 0)(III);
}
\problemNP{O}{f}{unequal magnitudes the greater has a greater ratio to the same than the less has: and the same magnitude has a greater ratio to the less than it has to the greater.}

\startCenterAlign
Let \drawMagnitude[bottom]{I} and \drawMagnitude{II} be two unequal magnitudes,\\
and \drawMagnitude{III} any other.
\stopCenterAlign

We shall first prove that \magnitudeI\ which is greater of the two unequal magnitudes, has a greater ratio to \magnitudeIII\ than \magnitudeII, the less, has to \magnitudeIII; that is, $\magnitudeI : \magnitudeIII > \magnitudeII : \magnitudeIII$;

\startCenterAlign
take $M' \magnitudeI$, $m' \magnitudeIII$, $M' \magnitudeII$, $m' \magnitudeIII$;\\
such that $M' \drawMagnitude{Ia}$ and $M' \drawMagnitude{Ib}$ shall be each $>$ \magnitudeIII;\\
also take $m' \magnitudeIII$ the least multiple of \magnitudeIII,\\
which will make $m' \magnitudeIII > M' \magnitudeII = M' \magnitudeIb$;

$\therefore M' \magnitudeII \mbox{ is not } > m' \magnitudeIII$,\\
but $M' \magnitudeI \mbox{ is } > m' \magnitudeIII$, for,\\
as $m' \magnitudeIII$ is the first multiple which first becomes $> M' \magnitudeIb$, than $(m' \mbox{ minus } 1)\magnitudeIII$ or $m' \magnitudeIII \mbox{ minus } \magnitudeIII$ is not $> M' \magnitudeIa$,

$\therefore m' \magnitudeIII \mbox{ minus } \magnitudeIII + \magnitudeIII$ must be $< M' \magnitudeIb + M' \magnitudeIa$;\\
that is, $m' \magnitudeIII$ must be $< M' \magnitudeI$;

$\therefore M' \magnitudeI \mbox{ is } > m' \magnitudeIII$; but it has been shown above that $M' \magnitudeII \mbox{ is not } > m' \magnitudeIII$, therefore, by the seventh definition, \magnitudeI\ has to \magnitudeIII\ a greater ratio than $\magnitudeII : \magnitudeIII$.

Next we shall prove that \magnitudeIII\ has a greater ratio to \magnitudeII, the less, than it has to \magnitudeI, the greater;\\
or, $\magnitudeIII : \magnitudeII > \magnitudeIII : \magnitudeI$.

Take $m' \magnitudeIII$, $M' \magnitudeII$, $m' \magnitudeIII$ and $M' \magnitudeI$,\\
the same as in the first case, such that\\
$M' \magnitudeIa$ and $M' \magnitudeIb$ will be each $> \magnitudeIII$, and $m' \magnitudeIII$ the least multiple of \magnitudeIII, which first becomes greater than $M' \magnitudeIb = M' \magnitudeII$.

$\therefore m' \magnitudeIII \mbox{ minus } \magnitudeIII \mbox{ is not } > M' \magnitudeIb$,\\
and \magnitudeIII is not $> M' \magnitudeIa$; consequently\\
$m' \magnitudeIII \mbox{ minus } \magnitudeIII + \magnitudeIII \mbox{ is } < M' \magnitudeIb + M' \magnitudeIa$;

$\therefore m' \magnitudeIII \mbox{ is } < M' \magnitudeI$, and $\therefore$ by the seventh definition, \magnitudeIII\ has to \magnitudeII\ a greater ratio than \magnitudeIII\ has to \magnitudeI.

$\therefore$ Of unequal magnitudes, \&c.
\stopCenterAlign

The contrivance employed in this proposition for finding among the multiples taken, as in the fifth definition, a multiple of the first greater than the multiple of the second, but the same multiple of the third which has been taken of the first, not greater than the same multiple of the fourth which has been taken of the second, may be illustrated numerically as follows :—

The number $\color[black]{9}$ has a greater ratio to $\color[byblue]{7}$ than $\color[byyellow]{8}$ has to $\color[byblue]{7}$: that is, $\color[black]{9} : \color[byblue]{7} > \color[byyellow]{8} : \color[byblue]{7}$; or $\color[byred]{8} + \color[black]{1} : \color[byblue]{7} > \color[byyellow]{8} : \color[byblue]{7}$.

The multiple of $\color[black]{1}$, which first becomes greater than $\color[byblue]{7}$, is $8$ times, therefore, we may multiply the first and third by $8$, $9$, $10$, or any other greater number; in this case, let us multiply the first and third by $8$, and we have $\color[byred]{64} + \color[black]{8}$ and $\color[byyellow]{64}$: again, the first multiple of $\color[byblue]{7}$ which becomes greater than $64$ is $10$ times; then, by multiplying the second and fourth by $10$, we shall have $\color[byblue]{70}$ and $\color[byblue]{70}$; then, arranging these multiples, we have—

\startCenterAlign
\unprotect
\ \hfill\vbox{
\offinterlineskip\lineskip3pt
\halign{\hfil # \hfil & \hfil # \hfil & \hfil # \hfil & \hfil # \hfil\cr
{\tfxx 8 times} & {\tfxx 10 times} & {\tfxx 8 times} & {\tfxx 10 times} \cr
{\tfxx the first} & {\tfxx the second} & {\tfxx the third} & {\tfxx the fourth} \cr
$\color[byred]{64} + \color[black]{8}$ & $\color[byblue]{70}$ & $\color[byyellow]{64}$ & $\color[byblue]{70}$ \cr
}}\hfill\
\protect
\stopCenterAlign

Consequently $\color[byred]{64} + \color[black]{8}$, or $72$, is greater than $\color[byblue]{70}$, but $\color[byyellow]{64}$ is not greater than $\color[byblue]{70}$, $\therefore$ by the seventh definition $\color[black]{9}$ has a greater ratio to $\color[byblue]{7}$ than $\color[byyellow]{8}$ has to $\color[byblue]{7}$.

The above is merely illustrative of the foregoing demonstration, for this property could be shown of these of other numbers very readily in the following manner; because, if an antecedent contains its consequent a greater number of times than another antecedent contains its consequent, or when a fraction is formed of an antecedent for the numerator, and its consequent for the denominator be greater than another fraction which is formed of another antecedent for the numerator and its consequent for the denominator, the ratio of the first antecedent to its consequent is greater than the ratio of the last antecedent to its consequent.

Thus, the number $9$ has a greater ratio to $7$, than $8$ has to $7$, for $\frac{9}{7}$ is greater than $\frac{8}{7}$.

Again, $17 : 19$ is a greater ratio than $13 : 15$, because $\frac{17}{19} = \frac{17 \times 15}{19 \times 15} = \frac{255}{185}$, and $\frac{13}{15} = \frac{13 \times 19}{15 \times 19} = \frac{247}{185}$, hence it is evident that $\frac{255}{185}$ is greater than $\frac{247}{185}$, $\therefore \frac{17}{19}$ is greater than $\frac{13}{15}$, and, according to what has been above shown, $17$ has to $19$ a greater ratio than $13$ has to $15$.

So that the general terms upon which a greater, equal, or less ratio exists are as follows :—

If $\frac{A}{B}$ be greater than $\frac{C}{D}$, $A$ is said to have to $B$ a greater ratio than $C$ has to $D$; if $\frac{A}{B}$ be equal to $\frac{C}{D}$, then $A$ has to $B$ the same ratio which $C$ has to $D$l and if $\frac{A}{B}$ be less to $\frac{C}{D}$, $A$ is said to have to $B$ a less ratio than $C$ has to $D$.

The student should understand all up to this proposition perfectly before proceeding further, in order fully to comprehend the following propositions of this book. We therefore strongly recommend the learner to commence again, and read up to this slowly, and carefully reason at each step, as he proceeds, particularly guarding against the mischievous system of depending wholly on the memory. By following these instructions, he will find that the parts which usually present considerable difficulties will present no difficulties whatever, in prosecuting the study of this important book.
\stopProposition

\vfill\pagebreak

\startProposition[title={Prop. IX. Theor.}, reference=prop:V.IX]
\defineNewPicture{
byMagnitudeSymbolDefine("rhombus", byblue, 0)(I);
byMagnitudeSymbolDefine("circle", byred, 0)(II);
byMagnitudeSymbolDefine("square", byyellow, 0)(III);
}
\problemNP{M}{agnitudes}{which have the same ratio to the same magnitude are equal to one another; and those to which the same magnitude has the same ratio are equal to one another}

\startCenterAlign
Let $\drawMagnitude{I} : \drawMagnitude{III} :: \drawMagnitude{II} : \drawMagnitude{III}$, then $\magnitudeI = \magnitudeII$.

For, if not, let $\magnitudeI > \magnitudeII$, then will\\
$\magnitudeI : \magnitudeIII > \magnitudeII : \magnitudeIII$ \inprop[prop:V.VIII],
which is absurd according to the hypothesis.

$\therefore \magnitudeI \mbox{ is not } > \magnitudeII$.

In the same manner it may be shown, that\\
$\magnitudeII \mbox{ is not } > \magnitudeI$,

$\therefore \magnitudeI = \magnitudeII$.

Again, let $\magnitudeIII : \magnitudeI :: \magnitudeIII : \magnitudeII$,\\
then will $\magnitudeI = \magnitudeII$.

For (invert.) $\magnitudeI : \magnitudeIII :: \magnitudeII : \magnitudeIII$,\\
therefore, by the first case, $\magnitudeI = \magnitudeII$.

$\therefore$ Magnitudes which have the same ratio, \&c.
\stopCenterAlign

This may be shown otherwise, as follows :—

Let $\color[byyellow]{A} : \color[byblue]{B} = \color[byyellow]{A} : \color[byred]{C}$, then $\color[byblue]{B} = \color[byred]{C}$, for, as the fraction $\frac{\color[byyellow]{A}}{\color[byblue]{B}} = \mbox{ the fraction } \frac{\color[byyellow]{A}}{\color[byred]{C}}$, and the numerator of one equal to the numerator of the other, therefore, denominators of these fractions are equal, that is $\color[byblue]{B} = \color[byred]{C}$.

Again, if $\color[byblue]{B} : \color[byyellow]{A} = \color[byred]{C} : \color[byyellow]{A}$, $\color[byblue]{B} = \color[byred]{C}$. For, as $\frac{\color[byblue]{B}}{\color[byyellow]{A}} = \frac{\color[byred]{C}}{\color[byyellow]{A}}$, $\color[byblue]{B} \mbox{ must } = \color[byred]{C}$.
\stopProposition

\vfill\pagebreak

\startProposition[title={Prop. X. Theor.}, reference=prop:V.X]
\defineNewPicture{
byMagnitudeSymbolDefine("wedgeDown", byblue, 0)(I);
byMagnitudeSymbolDefine("circle", byred, 0)(II);
byMagnitudeSymbolDefine("square", byyellow, 0)(III);
}
\problemNP{T}{hat}{magnitude which has a greater ratio than another has unto the same magnitude, is the greater of the two; and that magnitude to which the same has a greater ratio than it has unto another magnitude, is the less of the two.}

\startCenterAlign
Let $\drawMagnitude{I} : \drawMagnitude{III} > \drawMagnitude{II} : \drawMagnitude{III}$, then $\magnitudeI > \magnitudeII$.

For if not, let $\magnitudeI = \mbox{ or } < \magnitudeII$;\\
then, $\magnitudeI : \magnitudeIII = \magnitudeII : \magnitudeIII$ \inprop[prop:V.VII] or\\
$\magnitudeI : \magnitudeIII > \magnitudeII : \magnitudeIII$ \inprop[prop:V.VIII] and (invert.), which is absurd according to the hypothesis.

$\therefore \magnitudeI \mbox{ is not } = \mbox{ or } < \magnitudeII$, and\\
$\therefore \magnitudeI \mbox{ must be } >\magnitudeII$.

Again, let $\magnitudeIII : \magnitudeII > \magnitudeIII : \magnitudeI$,\\
then, $\magnitudeII < \magnitudeI$.

For if not, $\magnitudeII \mbox{ must be } > \mbox{ or } = \magnitudeI$,\\
Then $\magnitudeIII : \magnitudeII < \magnitudeIII : \magnitudeI$ \inprop[prop:V.VIII] and (invert.);\\
or $\magnitudeIII : \magnitudeII = \magnitudeIII : \magnitudeI$ \inprop[prop:V.VII], which is absurd (hyp.);

$\therefore \magnitudeII \mbox{ is not } > \mbox{ or } = \magnitudeI$,\\
and $\therefore \magnitudeII \mbox{ must be } < \magnitudeI$.

$\therefore$ That magnitude which has, \&c.
\stopCenterAlign
\stopProposition

\vfill\pagebreak

\startProposition[title={Prop. XI. Theor.}, reference=prop:V.XI]
\defineNewPicture{
byMagnitudeSymbolDefine("rhombus", byblue, 0)(I);
byMagnitudeSymbolDefine("square", byblue, 0)(II);
byMagnitudeSymbolDefine("miniTriangleUp", black, 0)(III);
byMagnitudeSymbolDefine("miniCircle", black, 0)(IV);
byMagnitudeSymbolDefine("circle", byred, 0)(V);
byMagnitudeSymbolDefine("wedgeDown", byyellow, 0)(VI);
}
\problemNP{R}{atios}{that are the same to the same ratio, are same to each other.}

\startCenterAlign
Let $\drawMagnitude{I} : \drawMagnitude{II} = \drawMagnitude{V} : \drawMagnitude{VI}$ and $\magnitudeV : \magnitudeVI = \drawMagnitude{III} : \drawMagnitude{IV}$,\\
then will $\magnitudeI : \magnitudeII = \magnitudeIII : \magnitudeIV$.

For if $M \magnitudeI >, = \mbox{ or } < m \magnitudeII$,\\
then $M \magnitudeV >, = \mbox{ or } < m \magnitudeVI$,\\
and if $M \magnitudeV >, = \mbox{ or } < m \magnitudeVI$,\\
then $M \magnitudeIII >, = \mbox{ or } < m \magnitudeIV$ \indef[def:V.V];

$\therefore$ if $M \magnitudeI >, = \mbox{ or } < m \magnitudeII$, $M \magnitudeIII >, = \mbox{ or } < m \magnitudeIV$\\
and $\therefore$ \indef[def:V.V] $\magnitudeI : \magnitudeII = \magnitudeIII : \magnitudeIV$.

$\therefore$ Ratios that are the same \&c.
\stopCenterAlign
\stopProposition

 \vfill\pagebreak

\startProposition[title={Prop. XII. Theor.}, reference=prop:V.XII]
\defineNewPicture{
byMagnitudeSymbolDefine("square", byred, 0)(I);
byMagnitudeSymbolDefine("circle", byred, 0)(II);
byMagnitudeSymbolDefine("semicircleDown", black, 1)(III);
byMagnitudeSymbolDefine("sectorUp", black, 1)(IV);
byMagnitudeSymbolDefine("rhombus", byyellow, 0)(V);
byMagnitudeSymbolDefine("wedgeDown", byyellow, 0)(VI);
byMagnitudeSymbolDefine("miniCircle", byblue, 0)(VII);
byMagnitudeSymbolDefine("miniTriangleDown", byblue, 0)(VIII);
byMagnitudeSymbolDefine("miniTriangleUp", black, 0)(IX);
byMagnitudeSymbolDefine("miniCircle", black, 0)(X);
}
\problemNP{I}{f}{any number of magnitudes be proportionals, as one of the antecedents is to its consequent, so shall all the antecedents taken together be to all the consequents.}

\startCenterAlign
Let $\drawMagnitude{I} : \drawMagnitude{II} =
\drawMagnitude{III} : \drawMagnitude{IV} =
\drawMagnitude{V} : \drawMagnitude{VI} =
\drawMagnitude{VII} : \drawMagnitude{VIII} =
\drawMagnitude{IX} : \drawMagnitude{X}$;\\
then will $\magnitudeI : \magnitudeII =
\magnitudeI + \magnitudeIII + \magnitudeV + \magnitudeVII + \magnitudeIX :
\magnitudeII + \magnitudeIV + \magnitudeVI + \magnitudeVIII + \magnitudeX$.

For if $M \magnitudeI > m \magnitudeII$, then $M \magnitudeIII > m \magnitudeIV$,\\
and $M \magnitudeV > m \magnitudeVI$, $M \magnitudeVII > m \magnitudeVIII$,\\
also $M \magnitudeIX > m \magnitudeX$ \indef[def:V.V]

Therefore, if $M \magnitudeI > m \magnitudeII$, then will\\
$M \magnitudeI + M \magnitudeIII + M \magnitudeV + M \magnitudeVII + M \magnitudeIX$, or $M (\magnitudeI + \magnitudeIII + \magnitudeV + \magnitudeVII + \magnitudeIX)$ be greater than $m \magnitudeII + m \magnitudeIV + m \magnitudeVI + m \magnitudeVIII + m \magnitudeX$, or $m (\magnitudeII + \magnitudeIV + \magnitudeVI + \magnitudeVIII + \magnitudeX)$.
\stopCenterAlign

In the same way it may be shown, if $M$ times one of the antecedents be equal to or less than $m$ times one of the consequents, $M$ times all the antecedents taken together, will be equal to or less than $m$ times all the consequents taken together. Therefore, by the fifth definition, as one of the antecedents is to its consequent, so are all the antecedents taken together to all the consequents taken together.

$\therefore$ If any number of magnitudes, \&c.
\stopProposition

\vfill\pagebreak


\startProposition[title={Prop. XIII. Theor.}, reference=prop:V.XIII]
\defineNewPicture{
byMagnitudeSymbolDefine("wedgeDown", byblue, 0)(I);
byMagnitudeSymbolDefine("semicircleDown", byblue, 1)(II);
byMagnitudeSymbolDefine("square", byred, 0)(III);
byMagnitudeSymbolDefine("rhombus", byyellow, 0)(IV);
byMagnitudeSymbolDefine("sectorUp", black, 1)(V);
byMagnitudeSymbolDefine("circle", black, 0)(VI);
}
\problemNP{I}{f}{the first has to the second the same ratio which the third has to the fourth, but the third to the fourth a greater ratio than the fifth has to the sixth; the first shall also have to the second a greater ratio than the fifth to the sixth.}

\startCenterAlign
Let $\drawMagnitude{I} : \drawMagnitude{II} = \drawMagnitude{III} : \drawMagnitude{IV}$, but $\magnitudeIII : \magnitudeIV > \drawMagnitude{V} : \drawMagnitude{VI}$,\\
then $\magnitudeI : \magnitudeII > \magnitudeV : \magnitudeVI$

For, because $\magnitudeIII : \magnitudeIV > \magnitudeV : \magnitudeVI$, there are some multiples ($M'$ and $m'$) of \magnitudeIII\ and \magnitudeV, and of \magnitudeIV\ and \magnitudeVI,\\
such that $M' \magnitudeIII > m' \magnitudeIV$,\\
but $M' \magnitudeV \mbox{ not } > m' \magnitudeVI$, by the seventh definition.

Let these multiples be taken, and take the same multiples of \magnitudeI\ and \magnitudeII.

$\therefore$ \indef[def:V.V] if $M' \magnitudeI >, =, \mbox{ or } < m' \magnitudeII$;\\
then will $M' \magnitudeIII >, =, \mbox{ or } < m' \magnitudeIV$,\\
but $M' \magnitudeIII > m' \magnitudeIV$ (construction);

$\therefore M' \magnitudeI > m' \magnitudeII$,\\
but $M' \magnitudeV \mbox{ is not } > m' \magnitudeVI$ (construction);\\
and therefore by the seventh definition,\\
$\magnitudeI : \magnitudeII > \magnitudeV : \magnitudeVI$

$\therefore$ If the first has to the second, \&c.
\stopCenterAlign
\stopProposition

\vfill\pagebreak

\startProposition[title={Prop. XIV. Theor.}, reference=prop:V.XIV]
\defineNewPicture{
byMagnitudeSymbolDefine("wedgeDown", byred, 0)(I);
byMagnitudeSymbolDefine("semicircleDown", black, 1)(II);
byMagnitudeSymbolDefine("square", byyellow, 0)(III);
byMagnitudeSymbolDefine("rhombus", byblue, 0)(IV);
}
\problemNP{I}{f}{the first has the same ratio to the second which the third has to the fourth; then, if the first be greater than the third, the second shall be greater than the fourth; and if equal, equal; and if less, less.}

\startCenterAlign
Let $\drawMagnitude{I} : \drawMagnitude{II} :: \drawMagnitude{III} : \drawMagnitude{IV}$, and first suppose\\
$\magnitudeI > \magnitudeIII$, then will $\magnitudeII > \magnitudeIV$.

For $\magnitudeI : \magnitudeII > \magnitudeIII : \magnitudeII$ \inprop[prop:V.VIII], and by the hypothesis $\magnitudeI : \magnitudeII = \magnitudeIII : \magnitudeIV$;\\
$\therefore \magnitudeIII : \magnitudeIV > \magnitudeIII : \magnitudeII$ \inprop[prop:V.XIII],\\
$\therefore \magnitudeIV < \magnitudeII$ \inprop[prop:V.X], or $\magnitudeII > \magnitudeIV$.

Secondly, let $\magnitudeI = \magnitudeIII$, then will $\magnitudeII = \magnitudeIV$.

For $\magnitudeI : \magnitudeII = \magnitudeIII : \magnitudeII$ \inprop[prop:V.VII],\\
and $\magnitudeI : \magnitudeII = \magnitudeIII : \magnitudeIV$ (hyp.);\\
$\therefore \magnitudeIII : \magnitudeII = \magnitudeIII : \magnitudeIV$ \inprop[prop:V.XI],\\
and $\therefore \magnitudeII = \magnitudeIV$ \inprop[prop:V.IX].

Thirdly, if $\magnitudeI < \magnitudeIII$, then will $\magnitudeII < \magnitudeIV$;\\
because $\magnitudeIII > \magnitudeI$ and $\magnitudeIII : \magnitudeIV = \magnitudeI : \magnitudeII$;\\
$\therefore \magnitudeIV > \magnitudeII$, by the first case,\\
that is, $\magnitudeII < \magnitudeIV$.

$\therefore$ If the first has the same ratio, \&c.
\stopCenterAlign
\stopProposition

\vfill\pagebreak

\startProposition[title={Prop. XV. Theor.}, reference=prop:V.XV]
\defineNewPicture{
byMagnitudeSymbolDefine("circle", byred, 0)(I);
byMagnitudeSymbolDefine("square", byyellow, 0)(II);
}
\problemNP{M}{agnitudes}{have the same ratio to one another which their equimultiples have.}

\startCenterAlign
Let \drawMagnitude{I} and \drawMagnitude{II} be two magnitudes;\\
then, $\magnitudeI : \magnitudeII :: M' \magnitudeI : M' \magnitudeII$.

$\eqalign{
\mbox{For } \magnitudeI : \magnitudeII &= \magnitudeI : \magnitudeII \cr
&= \magnitudeI : \magnitudeII \cr
&= \magnitudeI : \magnitudeII \cr
}$\\
$\therefore \magnitudeI : \magnitudeII :: 4 \magnitudeI : 4 \magnitudeII$. \inprop[prop:V.XII].

And as the same reasoning is generally applicable, we have:\\
$\magnitudeI : \magnitudeII :: M' \magnitudeI : M' \magnitudeII$.

$\therefore$ Magnitudes have the same ratio, \&c.
\stopCenterAlign
\stopProposition

\vfill\pagebreak

\startDefinition[title={Definition XIII},reference=def:V.XIII]
\defineNewPicture{
byMagnitudeSymbolDefine("circle", byyellow, 0)(I);
byMagnitudeSymbolDefine("rhombus", black, 0)(II);
byMagnitudeSymbolDefine("wedgeDown", byred, 0)(III);
byMagnitudeSymbolDefine("square", byblue, 0)(IV);
}
The technical term permutando, or alterando, by permutation or alternately, is used when there are four proportionals, and it is inferred that the first has the same ratio to the third which the second has to the fourth; or that the first is to the third as the second to the fourth: as is shown in the following proposition :—

\startCenterAlign
Let $\drawMagnitude{I} : \drawMagnitude{II} :: \drawMagnitude{III} : \drawMagnitude{IV}$,\\
by \quotation{permutando} or \quotation{alterando} it is inferred\\
$\magnitudeI : \magnitudeIII :: \magnitudeII : \magnitudeIV$.
\stopCenterAlign

It may be necessary here to remark that the magnitudes \magnitudeI, \magnitudeII, \magnitudeIII, \magnitudeIV, must be homogeneous, that is, of the same nature or similitude of kind; we must therefore, in such cases, compare lines with lines, surfaces with surfaces, solids with solids, \&c. Hence the student will readily perceive that a line and a surface, a surface and a solid, or other heterogenous magnitudes, can never stand in the relation of antecedent and consequent.
\stopDefinition

\vfill\pagebreak

\startProposition[title={Prop. XVI. Theor.}, reference=prop:V.XVI]
\defineNewPicture{
byMagnitudeSymbolDefine("wedgeDown", byred, 0)(I);
byMagnitudeSymbolDefine("semicircleDown", black, 1)(II);
byMagnitudeSymbolDefine("square", byyellow, 0)(III);
byMagnitudeSymbolDefine("rhombus", byblue, 0)(IV);
}
\problemNP{I}{f}{four magnitudes of the same kind be proportionals, they are also proportionals when taken alternately.}

\startCenterAlign
Let $\drawMagnitude{I} : \drawMagnitude{II} :: \drawMagnitude{III} : \drawMagnitude{IV}$, then $\magnitudeI : \magnitudeIII :: \magnitudeII : \magnitudeIV$.

For $M \magnitudeI : M \magnitudeII :: \magnitudeI : \magnitudeII$ \inprop[prop:V.XV],\\
and $M \magnitudeI : M \magnitudeII :: \magnitudeIII : \magnitudeIV$ (hyp.) and \inprop[prop:V.XI];\\
also $m \magnitudeIII : m \magnitudeIV :: \magnitudeIII : \magnitudeIV$ \inprop[prop:V.XV];

$\therefore M \magnitudeI : M \magnitudeII :: m \magnitudeIII : m \magnitudeIV$,\\
then will $M \magnitudeII >, =, \mbox{ or } < m \magnitudeIV$ \inprop[prop:V.XIV];

therefore, by the fifth definition,\\
 $\magnitudeI : \magnitudeIII :: \magnitudeII : \magnitudeIV$

 $\therefore$ If four magnitudes of the same kind, \&c.
\stopCenterAlign
\stopProposition

\vfill\pagebreak

\startDefinition[title={Definition XIV},reference=def:V.XIV]
\def\varA{\color[byred]{A}}
\def\varB{\color[black]{B}}
\def\varC{\color[byblue]{C}}
\def\varD{\color[byyellow]{D}}
Dividendo, by definition, when there are four proportionals, and it is inferred, that the excess of the first above the second is to the second as the excess of the third above the fourth is to the fourth.

\startCenterAlign
Let $\varA : \varB :: \varC : \varD$;\\
by \quotation{dividendo} it is inferred\\
$\varA \mbox{ minus } \varB : \varB :: \varC \mbox{ minus } \varD : \varD$.
\stopCenterAlign

According to the above, $\varA$ is supposed to be greater than $\varB$, and $\varC$ greater than $\varD$; if this be not the case, but to have $\varB$ greater than $\varA$, and $\varD$ greater than $\varC$, $\varB$ and $\varD$ can be made to stand as antecedents, and $\varA$ and $\varC$ as consequents, by \quotation{invertion}

\startCenterAlign
$\varB : \varA :: \varD : \varC$;\\
then, by \quotation{dividendo,} we infer\\
$\varB \mbox{ minus } \varA : \varA :: \varD \mbox{ minus } \varC : \varC$
\stopCenterAlign
\stopDefinition

\vfill\pagebreak

\startProposition[title={Prop. XVII. Theor.}, reference=prop:V.XVII]
\defineNewPicture{
byMagnitudeSymbolDefine("wedgeDown", byred, 0)(I);
byMagnitudeSymbolDefine("semicircleDown", black, 1)(II);
byMagnitudeSymbolDefine("square", byyellow, 0)(III);
byMagnitudeSymbolDefine("rhombus", byblue, 0)(IV);
}
\problemNP{I}{f}{magnitudes, taken jointly, be proportionals, they shall also be proportionals when taken separately: that is, if two magnitudes together have to one of them the same ratio which two others have to one of these, the remaining one of the first two shall have to the other the same ratio which the remaining one of the last two has to the other of these.}

\startCenterAlign
Let $\drawMagnitude{I} + \drawMagnitude{II} : \magnitudeII :: \drawMagnitude{III} + \drawMagnitude{IV} : \magnitudeIV$,\\
then will $\magnitudeI : \magnitudeII :: \magnitudeIII : \magnitudeIV$.

Take $\magnitudeI > m \magnitudeII$ to each add $M \magnitudeII$,\\
then we have $M \magnitudeI + M \magnitudeII > m \magnitudeII + M \magnitudeII$,\\
or $M (\magnitudeI + \magnitudeII) > (m + M) \magnitudeII$:\\
but because $\magnitudeI + \magnitudeII: \magnitudeII :: \magnitudeIII + \magnitudeIV: \magnitudeIV$ (hyp.),\\
and $M (\magnitudeI + \magnitudeII) > (m + M) \magnitudeII$;\\
$\therefore M(\magnitudeIII + \magnitudeIV) > (m + M) \magnitudeIV$ \indef[def:V.V];\\
$\therefore M \magnitudeIII + M \magnitudeIV > m \magnitudeIV + M \magnitudeIV$;\\
$\therefore M \magnitudeIII > m \magnitudeIV$, by taking $M \magnitudeIV$ from both sides:\\
that is, when $M \magnitudeI > m \magnitudeII$, then $M \magnitudeIII > m \magnitudeIV$.

In the same manner it may be proved, that if $M \magnitudeI = \mbox{ or } < m \magnitudeII$, then will $M \magnitudeIII = \mbox{ or } < m \magnitudeIV$;\\
and $\therefore \magnitudeI : \magnitudeII :: \magnitudeIII : \magnitudeIV$ \indef[def:V.V]

$\therefore$ If magnitudes taken jointly, \&c.
\stopCenterAlign
\stopProposition

\vfill\pagebreak

\startDefinition[title={Definition XV},reference=def:V.XV]
\def\varA{\color[byred]{A}}
\def\varB{\color[black]{B}}
\def\varC{\color[byyellow]{C}}
\def\varD{\color[byblue]{D}}
The term componendo, by composition, is used when there are four proportionals; and it is inferred that the first together with the second is to the second as the third together with the fourth is to fourth.

\startCenterAlign
Let $\varA : \varB :: \varC : \varD$;

then, by term \quotation{componendo,} it is inferred that\\
$\varA + \varB : \varB :: \varC + \varD: \varD$
\stopCenterAlign

Be \quotation{invertion} $\varB$ and $\varD$ may become the first and the third, and $\varA$ and $\varC$ the second and fourth, as

\startCenterAlign
$\varB : \varA :: \varD : \varC$,

then, by \quotation{componendo,} we infer that\\
$\varB + \varA : \varA :: \varD + \varC : \varC$.
\stopCenterAlign
\stopDefinition


\vfill\pagebreak

\startProposition[title={Prop. XVIII. Theor.}, reference=prop:V.XVIII]
\defineNewPicture{
byMagnitudeSymbolDefine("wedgeDown", byred, 0)(I);
byMagnitudeSymbolDefine("semicircleDown", black, 1)(II);
byMagnitudeSymbolDefine("square", byyellow, 0)(III);
byMagnitudeSymbolDefine("rhombus", byblue, 0)(IV);
byMagnitudeSymbolDefine("circle", black, 0)(V);
}
\problemNP{I}{f}{magnitudes, taken separately, be proportionals, they shall also be proportionals taken jointly: that is, if the first be to the second as the third is to the fourth, the first and second together shall be to the second as the third and fourth together is to the fourth.}

\startCenterAlign
Let $\drawMagnitude{I} : \drawMagnitude{II} :: \drawMagnitude{III} : \drawMagnitude{IV}$,\\
then $\magnitudeI + \magnitudeII : \magnitudeII :: \magnitudeIII + \magnitudeIV : \magnitudeIV$;

for if not, let $\magnitudeI + \magnitudeII : \magnitudeII :: \magnitudeIII + \drawMagnitude{V} : \magnitudeV$,\\
supposing $\magnitudeV \mbox{ not } = \magnitudeIV$;

$\therefore \magnitudeI : \magnitudeII :: \magnitudeIII : \magnitudeV$ \inprop[prop:V.XVII]\\
but $\magnitudeI : \magnitudeII :: \magnitudeIII : \magnitudeIV$ (hyp.);

$\therefore \magnitudeIII : \magnitudeV :: \magnitudeIII : \magnitudeIV$ \inprop[prop:V.XI];

$\therefore \magnitudeV = \magnitudeIV$ \inprop[prop:V.IX],\\
which is contrary to the supposition;

$\therefore \magnitudeV \mbox{ is not unequal to } \magnitudeIV$;\\
that is $\magnitudeV = \magnitudeIV$;

$\therefore \magnitudeI + \magnitudeII : \magnitudeII :: \magnitudeIII + \magnitudeIV : \magnitudeIV$.

$\therefore$ If magnitudes, taken separately, \&c.
\stopCenterAlign
\stopProposition

\vfill\pagebreak

\startProposition[title={Prop. XIX. Theor.}, reference=prop:V.XIX]
\defineNewPicture{
byMagnitudeSymbolDefine("wedgeDown", byred, 0)(I);
byMagnitudeSymbolDefine("semicircleDown", black, 1)(II);
byMagnitudeSymbolDefine("square", byblue, 0)(III);
byMagnitudeSymbolDefine("rhombus", byyellow, 0)(IV);
}
\problemNP{I}{f}{a whole magnitude be to a whole, as a magnitude taken from the first is to a magnitude taken from the other; the remainder shall be to the reminder, as the whole to the whole.}

\startCenterAlign
Let $\drawMagnitude{I} + \drawMagnitude{II} : \drawMagnitude{III} + \drawMagnitude{IV} :: \magnitudeI : \magnitudeIII$,\\
then will $\magnitudeII : \magnitudeIV :: \magnitudeI + \magnitudeII : \magnitudeIII + \magnitudeIV$,

For $\magnitudeI + \magnitudeII : \magnitudeI :: \magnitudeIII + \magnitudeIV : \magnitudeIII$ (alter.),

$\therefore \magnitudeII : \magnitudeI :: \magnitudeIV : \magnitudeIII$ (divid.),

again $\magnitudeII : \magnitudeIV :: \magnitudeI : \magnitudeIII$ (alter.),

but $\magnitudeI + \magnitudeII : \magnitudeIII + \magnitudeIV :: \magnitudeI : \magnitudeIII$ (hyp.);

therefore $\magnitudeII : \magnitudeIV :: \magnitudeI + \magnitudeII : \magnitudeIII + \magnitudeIV$ \inprop[prop:V.XI].

$\therefore$ If a whole magnitude be to a whole, \&c.
\stopCenterAlign
\stopProposition

\startDefinition[title={Definition XVII},reference=def:V.XVII]
The term \quotation{convertendo,} by conversion, is made use of by geometricians, when there are four proportionals, and it is inferred, that the first is to its excess above the second, as the third is to its excess above the fourth. See the following proposition :—
\stopDefinition

\vfill\pagebreak

\startPropositionAZ[title={Prop. E. Theor.}, reference=prop:V.E]
\defineNewPicture{
byMagnitudeSymbolDefine("circle", byblue, 0)(i);
byMagnitudeSymbolDefine("sectorUp", black, 1)(II);
byMagnitudeSymbolDefine("square", byred, 0)(iii);
byMagnitudeSymbolDefine("rhombus", byyellow, 0)(IV);
byMagnitudeDefine(I, 0, true)(1, 1)(i, II);
byMagnitudeDefine(III, 0, true)(1, 1)(iii, IV);
}
\problemNP{I}{f}{four magnitudes be proportionals, they are also proportionals by conversion: that is, the first is to its excess above the second, as the third is to its excess above the fourth.}

\startCenterAlign
Let $\drawMagnitude{I} : \drawMagnitude{II} :: \drawMagnitude{III} : \drawMagnitude{IV}$,\\
then shall $\magnitudeI : \drawMagnitude{i} :: \magnitudeIII : \drawMagnitude{iii}$,

Because $\magnitudeI : \magnitudeII :: \magnitudeIII : \magnitudeIV$;\\
therefore $\magnitudei : \magnitudeII :: \magnitudeiii : \magnitudeIV$ (divid.),

$\therefore \magnitudeII : \magnitudei :: \magnitudeIV : \magnitudeiii$ (inver.),

$\therefore \magnitudeI : \magnitudei :: \magnitudeIII : \magnitudeiii$ (compo.).

$\therefore$ If four magnitudes, \&c.
\stopCenterAlign
\stopPropositionAZ

\startDefinition[title={Definition XVIII},reference=def:V.XVIII]
\quotation{Ex \ae quali} (sc. distantia), or ex \ae quo, from equality of distance: when there is any number of magnitudes more than two, and as many others, such that they are proportionals when taken two and two of each rank, and it is inferred that the first is to the last of the first rank of magnitudes, as the first is to the last of the others: \quotation{of this there are two following kinds, which arise from different order in which the magnitudes are taken, two and two.}
\stopDefinition

\vfill\pagebreak

\startDefinition[title={Definition XIX},reference=def:V.XIX]
\def\varA{\color[byred]{A}}
\def\varB{\color[byred]{B}}
\def\varC{\color[byyellow]{C}}
\def\varD{\color[byyellow]{D}}
\def\varE{\color[byblue]{E}}
\def\varF{\color[byblue]{F}}
\def\varL{\color[byred]{L}}
\def\varM{\color[byred]{M}}
\def\varN{\color[byyellow]{N}}
\def\varO{\color[byyellow]{O}}
\def\varP{\color[byblue]{P}}
\def\varQ{\color[byblue]{Q}}
\quotation{Ex \ae quali,} from equality. This term is used simply by itself, when the first magnitude is to the second of the first rank, as the first to the second of the other rank; and as the second to the third of the first rank, so is the second to the third of the other; and so in order: and the inference is as mentioned in the preceding definition; whence this is called ordinate proposition. It is demonstrated in \inpropL[prop:V.XXII].

\startCenterAlign
Thus, if there be two ranks of magnitudes,\\
$\varA, \varB, \varC, \varD, \varE, \varF$, the first rank,\\
and $\varL, \varM, \varN, \varO, \varP, \varQ$, the second,\\
such that $\varA : \varB :: \varL : \varM$, $\varB : \varC :: \varM : \varB$, $\varC : \varD :: \varN : \varO$, $\varD : \varE :: \varO : \varP$, $\varE : \varF :: \varP : \varQ$;\\
we infer by the term \quotation{ex \ae quali} that $\varA : \varF :: \varL : \varQ$
\stopCenterAlign
\stopDefinition

\vfill\pagebreak

\startDefinition[title={Definition XX},reference=def:V.XX]
\def\varA{\color[byred]{A}}
\def\varB{\color[byred]{B}}
\def\varC{\color[byblue]{C}}
\def\varD{\color[byblue]{D}}
\def\varE{\color[byyellow]{E}}
\def\varF{\color[byyellow]{F}}
\def\varL{\color[byyellow]{L}}
\def\varM{\color[byyellow]{M}}
\def\varN{\color[byblue]{N}}
\def\varO{\color[byblue]{O}}
\def\varP{\color[byred]{P}}
\def\varQ{\color[byred]{Q}}
\quotation{Ex \ae quali in proportione perturbata seu inordinata,} from equality in perturbate, or disorderly proposition. This term is used when the first magnitude is to the second of the first rank as the last but on is to the last of the second rank; and as the second is to the third of the first rank, so is the last but two to the last but one of the second rank; and as the third is to the fourth of the first rank, so is the third from the last to the last but two of the second rank; and so on in order: and the inference is in the 18th definition. It is demonstrated in \inpropL[prop:V.XXIII].

\startCenterAlign
Thus, if there be two ranks of magnitudes,\\
$\varA, \varB, \varC, \varD, \varE, \varF$, the first rank,\\
and $\varL, \varM, \varN, \varO, \varP, \varQ$, the second,\\
such that $\varA : \varB :: \varP : \varQ$, $\varB : \varC :: \varO : \varP$, $\varC : \varD :: \varN : \varO$, $\varD : \varE :: \varM : \varN$, $\varE : \varF :: \varL : \varM$;\\
the term \quotation{ex \ae quali in proportione perturbata seu inordinata} infers that $\varA : \varF :: \varL : \varQ$
\stopCenterAlign
\stopDefinition

\vfill\pagebreak

\startProposition[title={Prop. XX. Theor.}, reference=prop:V.XX]
\defineNewPicture{
byMagnitudeSymbolDefine("wedgeDown", byblue, 0)(I);
byMagnitudeSymbolDefine("semicircleDown", byred, 1)(II);
byMagnitudeSymbolDefine("square", byyellow, 0)(III);
byMagnitudeSymbolDefine("rhombus", byblue, 0)(IV);
byMagnitudeSymbolDefine("sectorUp", byred, 1)(V);
byMagnitudeSymbolDefine("circle", byyellow, 0)(VI);
}
\problemNP{I}{f}{there be three magnitudes, and the other three, which taken two and two, have the same ratio; then, if the first be greater than the third, the fourth shall be greater than the sixth; and if equal, equal; and if less, less.}

\startCenterAlign
Let \drawMagnitude{I}, \drawMagnitude{II}, \drawMagnitude{III}, the first three magnitudes,\\
and \drawMagnitude{IV}, \drawMagnitude{V}, \drawMagnitude{VI}, be the other three,\\
such that $\magnitudeI : \magnitudeII :: \magnitudeIV : \magnitudeV$, and $\magnitudeII : \magnitudeIII :: \magnitudeV : \magnitudeVI$,

Then, if $\magnitudeI >, =, \mbox{ or } < \magnitudeIII$,\\
then will $\magnitudeIV >, =, \mbox{ or } < \magnitudeVI$.

From the hypothesis, by alternando, we have\\
$\magnitudeI : \magnitudeIV :: \magnitudeII : \magnitudeV$,\\
and $\magnitudeII : \magnitudeV :: \magnitudeIII : \magnitudeVI$

$\therefore \magnitudeI : \magnitudeIV :: \magnitudeIII : \magnitudeVI$ \inprop[prop:V.XI];

$\therefore$ if $\magnitudeI >, =, \mbox{ or } < \magnitudeIII$,\\
then will $\magnitudeIV >, =, \mbox{ or } < \magnitudeVI$ \inprop[prop:V.XIV].

$\therefore$ If there be three magnitudes, \&c.
\stopCenterAlign
\stopProposition

\vfill\pagebreak

\startProposition[title={Prop. XXI. Theor.}, reference=prop:V.XXI]
\defineNewPicture{
byMagnitudeSymbolDefine("wedgeDown", byyellow, 0)(I);
byMagnitudeSymbolDefine("wedgeUp", byred, 0)(II);
byMagnitudeSymbolDefine("square", byblue, 0)(III);
byMagnitudeSymbolDefine("rhombus", byblue, 0)(IV);
byMagnitudeSymbolDefine("sectorUp", byred, 1)(V);
byMagnitudeSymbolDefine("circle", byyellow, 0)(VI);
}
\problemNP{I}{f}{there be three magnitudes, and other three, which have the same ratio, taken two and two, but in a cross order; them if the first magnitude be greater than the third, the fourth shall be greater than the sixth; and if equal, equal, and if less, less.}

\startCenterAlign
Let \drawMagnitude{I}, \drawMagnitude{II}, \drawMagnitude{III}, be the first three magnitudes,\\
and \drawMagnitude{IV}, \drawMagnitude{V}, \drawMagnitude{VI}, the other three,\\
such that $\magnitudeI : \magnitudeII :: \magnitudeV : \magnitudeVI$,\\
and $\magnitudeII : \magnitudeIII :: \magnitudeIV : \magnitudeV$.

Then if $\magnitudeI >, =, \mbox{ or } < \magnitudeIII$,\\
then will $\magnitudeIV >, =, \mbox{ or } < \magnitudeVI$.

First, let $\magnitudeI \mbox{ be } > \magnitudeIII$:\\
then, because \magnitudeII\ is any other magnitude,\\
$\magnitudeI : \magnitudeII > \magnitudeIII : \magnitudeII$ \inprop[prop:V.VIII];\\
but $\magnitudeV : \magnitudeVI :: \magnitudeI : \magnitudeII$ (hyp.);\\
$\therefore \magnitudeV : \magnitudeVI > \magnitudeIII : \magnitudeII$ \inprop[prop:V.XIII];\\
and because $\magnitudeII : \magnitudeIII :: \magnitudeIV : \magnitudeV$ (hyp.);\\
$\therefore \magnitudeIII : \magnitudeII :: \magnitudeV : \magnitudeIV$ (inv.),\\
and it was shown that $\magnitudeV : \magnitudeVI > \magnitudeIII : \magnitudeII$,
$\therefore \magnitudeV : \magnitudeVI > \magnitudeV : \magnitudeIV$ \inprop[prop:V.XIII];\\
$\therefore \magnitudeVI < \magnitudeIV$,\\
that is $\magnitudeIV > \magnitudeVI$.

Secondly, let $\magnitudeI = \magnitudeIII$; then shall $\magnitudeIV = \magnitudeVI$.\\
For because $\magnitudeI = \magnitudeIII$,\\
$\magnitudeI : \magnitudeII = \magnitudeIII : \magnitudeII$ \inprop[prop:V.VII];\\
but $\magnitudeI : \magnitudeII = \magnitudeV : \magnitudeVI$ (hyp.),\\
and $\magnitudeIII : \magnitudeII = \magnitudeV : \magnitudeIV$ (hyp. and inv.),\\
$\therefore \magnitudeV : \magnitudeVI = \magnitudeV : \magnitudeIV$ \inprop[prop:V.XI],\\
$\therefore \magnitudeIV = \magnitudeVI$ \inprop[prop:V.IX].

Next, let $\magnitudeI \mbox{ be } < \magnitudeIII$, then $\magnitudeIV \mbox{ shall be } < \magnitudeVI$;\\
for $\magnitudeIII > \magnitudeI$,\\
and it has been shown that $\magnitudeIII : \magnitudeII = \magnitudeV : \magnitudeIV$,\\
and $\magnitudeII : \magnitudeI = \magnitudeVI : \magnitudeV$;\\
$\therefore$ by the first case $\magnitudeVI \mbox{ is } > \magnitudeIV$,\\
that is $\magnitudeIV < \magnitudeVI$.

$\therefore$ If there be three, \&c.
\stopCenterAlign
\stopProposition

\vfill\pagebreak

\startProposition[title={Prop. XXII. Theor.}, reference=prop:V.XXII]
\defineNewPicture{
byMagnitudeSymbolDefine("wedgeDown", byred, 0)(I);
byMagnitudeSymbolDefine("rhombus", byblue, 0)(II);
byMagnitudeSymbolDefine("square", byyellow, 0)(III);
byMagnitudeSymbolDefine("rhombus", byred, 0)(IV);
byMagnitudeSymbolDefine("sectorUp", byblue, 1)(V);
byMagnitudeSymbolDefine("circle", byyellow, 0)(VI);
byMagnitudeSymbolDefine("wedgeDown", byblue, 0)(Ia);
byMagnitudeSymbolDefine("rhombus", black, 0)(IIa);
byMagnitudeSymbolDefine("square", byyellow, 0)(IIIa);
byMagnitudeSymbolDefine("rhombus", byred, 0)(IVa);
byMagnitudeSymbolDefine("sectorUp", byblue, 1)(Va);
byMagnitudeSymbolDefine("circle", black, 0)(VIa);
byMagnitudeSymbolDefine("halfsquare", byyellow, 0)(VIIa);
byMagnitudeSymbolDefine("halfrhombusUp", byred, 0)(VIIIa);
}
\problemNP{I}{f}{there be any number of magnitudes, and as many others, which, taken two and two in order, have the same ratio; the first shall have to the last of the first magnitudes the same ratio which the first of the others has to the last of the same.\\
N.B.— This is usually cited by the words \quotation{ex \ae quali,} or \quotation{ex \ae quo.}}

\startCenterAlign
First, let there be magnitudes \drawMagnitude{I}, \drawMagnitude{II}, \drawMagnitude{III},\\
and as many others \drawMagnitude{IV}, \drawMagnitude{V}, \drawMagnitude{VI},\\
such that\\
$\magnitudeI : \magnitudeII :: \magnitudeIV : \magnitudeV$,\\
and $\magnitudeII : \magnitudeIII :: \magnitudeV : \magnitudeVI$,\\
then shall $\magnitudeI : \magnitudeIII :: \magnitudeIV : \magnitudeVI$,\\
\stopCenterAlign

Let these magnitudes, as well as any equimultiples whatever of the antecedents and consequents of the ratios, stand as follows :—

\startCenterAlign
$\magnitudeI, \magnitudeII, \magnitudeIII, \magnitudeIV, \magnitudeV, \magnitudeVI,$\\
and\\
$M \magnitudeI, m \magnitudeII, N \magnitudeIII, M \magnitudeIV, m \magnitudeV, n \magnitudeVI,$\\
because $\magnitudeI : \magnitudeII :: \magnitudeIV : \magnitudeV$,\\;
$\therefore M \magnitudeI : m \magnitudeII :: M \magnitudeIV : m \magnitudeV$ \inprop[prop:V.IV].

For the same reason\\
$m \magnitudeII : N \magnitudeIII :: m \magnitudeV : N \magnitudeVI$;\\
and because there are three magnitudes\\
$M \magnitudeI, m \magnitudeII, N \magnitudeIII$,\\
and other three, $M \magnitudeIV, m \magnitudeV, N \magnitudeVI$,\\
which taken two and two, have the same ratio;

$\therefore$ if $M \magnitudeI >, = \mbox{ or } < N \magnitudeIII$\\
then will $M \magnitudeIV >, = \mbox{ or } < N \magnitudeVI$, by \inprop[prop:V.XX],\\
and $\therefore \magnitudeI : \magnitudeIII :: \magnitudeIV : \magnitudeVI$ \indef[def:V.V].

Next, let there be four magnitudes, \drawMagnitude{Ia}, \drawMagnitude{IIa}, \drawMagnitude{IIIa}, \drawMagnitude{IVa},\\
and other four, \drawMagnitude{Va}, \drawMagnitude{VIa}, \drawMagnitude{VIIa}, \drawMagnitude{VIIIa},\\
which, taken two and two, have the same ratio,\\
thah is to say, $\magnitudeIa : \magnitudeIIa :: \magnitudeVa : \magnitudeVIa$,\\
$\magnitudeIIa : \magnitudeIIIa :: \magnitudeVIa : \magnitudeVIIa$,\\
and $\magnitudeIIIa : \magnitudeIVa :: \magnitudeVIIa : \magnitudeVIIIa$,\\
then shall $\magnitudeIa : \magnitudeIVa :: \magnitudeVa : \magnitudeVIIIa$;\\
for, because $\magnitudeIa, \magnitudeIIa, \magnitudeIIIa$, are three magnitudes,\\
and $\magnitudeVa, \magnitudeVIa, \magnitudeVIIa$, other three,\\
which, taken two and two, have the same ratio;\\
therefore, by foregoing case $\magnitudeIa : \magnitudeIIIa :: \magnitudeVa : \magnitudeVIIa$,\\
but $\magnitudeIIIa : \magnitudeIVa :: \magnitudeVIIa : \magnitudeVIIIa$;\\
therefore, again, by the first case, $\magnitudeIa : \magnitudeIVa :: \magnitudeVa : \magnitudeVIIIa$;\\
and so on, whatever number of magnitudes be.

$\therefore$ If there be any number, \&c.
\stopCenterAlign
\stopProposition

\vfill\pagebreak

\startProposition[title={Prop. XXIII. Theor.}, reference=prop:V.XXIII]
\defineNewPicture{
byMagnitudeSymbolDefine("wedgeDown", byyellow, 0)(I);
byMagnitudeSymbolDefine("semicircleDown", byblue, 1)(II);
byMagnitudeSymbolDefine("square", byred, 0)(III);
byMagnitudeSymbolDefine("rhombus", byyellow, 0)(IV);
byMagnitudeSymbolDefine("sectorUp", byblue, 1)(V);
byMagnitudeSymbolDefine("circle", byred, 0)(VI);
byMagnitudeSymbolDefine("halfsquare", black, 0)(VII);
byMagnitudeSymbolDefine("halfrhombusUp", black, 0)(VIII);
}
\problemNP{I}{f}{there be any number of magnitudes, and as many others, which, taken two and two in a cross order, have the same ratio; the first shall have to the last of the first magnitudes the same ratio which the first of the others has to the last of the same.\\
N.B.— This is usually cited by the words \quotation{ex \ae quali in proportione perturbata;} or \quotation{ex \ae quo perturbato.}}

\startCenterAlign
First, let there be magnitudes \drawMagnitude{I}, \drawMagnitude{II}, \drawMagnitude{III},\\
and other three, \drawMagnitude{IV}, \drawMagnitude{V}, \drawMagnitude{VI},\\
which, taken two and two in a cross order, have the same ratio;

that is, $\magnitudeI : \magnitudeII :: \magnitudeV : \magnitudeVI$,\\
and $\magnitudeII : \magnitudeIII :: \magnitudeIV : \magnitudeV$,\\
then shall $\magnitudeI : \magnitudeIII :: \magnitudeIV : \magnitudeVI$.
\stopCenterAlign

Let these magnitudes and their respective equimultiples be arranged as follows :—

\startCenterAlign
$\magnitudeI, \magnitudeII, \magnitudeIII, \magnitudeIV, \magnitudeV, \magnitudeVI,$\\
$M \magnitudeI, M \magnitudeII, m \magnitudeIII, M \magnitudeIV, m \magnitudeV, m \magnitudeVI,$\\
then $\magnitudeI : \magnitudeII :: M \magnitudeI : M \magnitudeII$ \inprop[prop:V.XV];\\
and for the same reason\\
$\magnitudeV : \magnitudeVI :: m \magnitudeV : m \magnitudeVI$;\\
but $\magnitudeI : \magnitudeII :: \magnitudeV : \magnitudeVI$ (hyp.),\\
$\therefore M \magnitudeI : M \magnitudeII :: \magnitudeV : \magnitudeVI$ \inprop[prop:V.XI];\\
and because $\magnitudeII : \magnitudeIII :: \magnitudeIV : \magnitudeV$ (hyp.),\\
$\therefore M \magnitudeII : m \magnitudeIII :: M \magnitudeIV : m \magnitudeV$ \inprop[prop:V.IV];\\
then, because there are three magnitudes,\\
$M \magnitudeI, M \magnitudeII, m \magnitudeII$,\\
and other three $M \magnitudeIV, m \magnitudeV, m \magnitudeVI$,\\
which, taken two and two in a cross order, have the same ratio;\\
therefore, if $M \magnitudeI >, =, \mbox{ or } < m \magnitudeIII$,\\
then will $M \magnitudeIV >, =, \mbox{ or } < m \magnitudeVI$ \inprop[prop:V.XXI],\\
and $\therefore \magnitudeI : \magnitudeIII :: \magnitudeIV : \magnitudeIV$ \indef[def:V.V].

Next, let there be four magnitudes,\\
\magnitudeI, \magnitudeII, \magnitudeIII, \magnitudeIV,\\
and other four,\\
\magnitudeV, \magnitudeVI, \drawMagnitude{VII}, \drawMagnitude{VIII},\\
which, when taken two and two in a cross order, have the same ratio;\\
namely, $\magnitudeI : \magnitudeII :: \magnitudeVII : \magnitudeVIII$,\\
$\magnitudeII : \magnitudeIII :: \magnitudeVI : \magnitudeVII$,\\
and $\magnitudeIII : \magnitudeIV :: \magnitudeV : \magnitudeVI$,\\
then shall $\magnitudeIII : \magnitudeIV :: \magnitudeV : \magnitudeVI$.\\
For, because $\magnitudeI, \magnitudeII, \magnitudeIII$ are three magnitudes,\\
and $\magnitudeVI, \magnitudeVII, \magnitudeVIII$, other three,\\
which, taken two and two in a cross order have the same ratio,\\
therefore, by the first case, $\magnitudeI : \magnitudeIII :: \magnitudeVI : \magnitudeVIII$,\\
but $\magnitudeIII : \magnitudeIV :: \magnitudeV : \magnitudeVI$,\\
therefore again, by the first case,\\
$\magnitudeIII : \magnitudeIV :: \magnitudeV : \magnitudeVI$;\\
and so on, whatever be the number of such magnitudes.

$\therefore$ If there be any number, \&c.
\stopCenterAlign
\stopProposition

\vfill\pagebreak

\startProposition[title={Prop. XXIV. Theor.}, reference=prop:V.XXIV]
\defineNewPicture{
byMagnitudeSymbolDefine("wedgeDown", byred, 0)(I);
byMagnitudeSymbolDefine("semicircleDown", black, 1)(II);
byMagnitudeSymbolDefine("square", byblue, 0)(III);
byMagnitudeSymbolDefine("rhombus", byyellow, 0)(IV);
byMagnitudeSymbolDefine("sectorUp", byred, 1)(V);
byMagnitudeSymbolDefine("circle", byblue, 0)(VI);
}
\problemNP{I}{f}{the first has to the second the same ratio which the third has to the fourth, and the fifth to the second the same which the sixth has to the fourth, the first and fifth together shall have to the second the same ratio which the third and sixth together have to the fourth.}

\startCenterAlign
\unprotect
\ \hfill\vbox{
\offinterlineskip\lineskip3pt
\halign{\hfil # \hfil & \hfil # \hfil & \hfil # \hfil & \hfil # \hfil\cr
{\tfxx First} & {\tfxx Second} & {\tfxx Third} & {\tfxx Fourth} \cr
\drawMagnitude{I} & \drawMagnitude{II} & \drawMagnitude{III} & \drawMagnitude{IV} \cr
{\tfxx Fifth} & & {\tfxx Sixth} & \cr
\drawMagnitude{V} & & \drawMagnitude{VI} & \cr
}}\hfill\
\protect

Let $\magnitudeI : \magnitudeII :: \magnitudeIII : \magnitudeIV$,\\
and $\magnitudeV : \magnitudeII :: \magnitudeVI : \magnitudeIV$,\\
then $\magnitudeI + \magnitudeV : \magnitudeII :: \magnitudeIII + \magnitudeVI : \magnitudeIV$.

For $\magnitudeV : \magnitudeII :: \magnitudeVI : \magnitudeIV$ (hyp.),\\
and $\magnitudeII : \magnitudeI :: \magnitudeIV : \magnitudeIII$ (hyp.) and (invert.),

$\therefore \magnitudeV : \magnitudeI :: \magnitudeVI : \magnitudeIII$ \inprop[prop:V.XXII];\\
and, because these magnitudes are proportionals, they are proportionals when taken jointly,

$\therefore \magnitudeI + \magnitudeV : \magnitudeV :: \magnitudeVI + \magnitudeIII : \magnitudeVI$ \inprop[prop:V.XVIII],\\
but $\magnitudeV : \magnitudeII :: \magnitudeVI : \magnitudeIV$ (hyp.),

$\therefore \magnitudeI + \magnitudeV : \magnitudeII :: \magnitudeVI + \magnitudeIII : \magnitudeIV$ \inprop[prop:V.XXII]

$\therefore$ If the first, \&c.
\stopCenterAlign
\stopProposition

\vfill\pagebreak

\startProposition[title={Prop. XXV. Theor.}, reference=prop:V.XXV]
\defineNewPicture{
byMagnitudeSymbolDefine("wedgeDown", byred, 0)(Ia);
byMagnitudeSymbolDefine("semicircleDown", black, 1)(III);
byMagnitudeSymbolDefine("square", byblue, 0)(IIa);
byMagnitudeSymbolDefine("rhombus", byyellow, 0)(IV);
}
\problemNP{I}{f}{four magnitudes of the same kind are proportionals, the greatest and least of them together are greater than other two together.}

\startCenterAlign
Let four magnitudes, $\drawMagnitude{Ia} + \drawMagnitude{III}, \drawMagnitude{IIa} + \drawMagnitude{IV}, \magnitudeIII, \mbox{ and } \magnitudeIV$, of the same kind be proportionals, that is to say,\\

$\magnitudeIa + \magnitudeIII : \magnitudeIIa + \magnitudeIV :: \magnitudeIII : \magnitudeIV$,\\
and let $\magnitudeIa + \magnitudeIII$ be the greatest of the four, and consequently by \inpropL[prop:V.A] and \inpropL[prop:V.XIV], \magnitudeIV is the least;\\
then will $\magnitudeIa + \magnitudeIII + \magnitudeIV \mbox{ be } > \magnitudeIIa + \magnitudeIV + \magnitudeIII$;\\
because $\magnitudeIa + \magnitudeIII : \magnitudeIIa + \magnitudeIV :: \magnitudeIII : \magnitudeIV$,

$\therefore \magnitudeIa : \magnitudeIIa :: \magnitudeIa + \magnitudeIII : \magnitudeIIa + \magnitudeIV$ \inprop[prop:V.XIX],\\
but $\magnitudeIa + \magnitudeIII > \magnitudeIIa + \magnitudeIV$ (hyp.),

$\therefore \magnitudeIa > \magnitudeIIa$ \inprop[prop:V.A];\\
to each of these add $\magnitudeIII + \magnitudeIV$,\\
$\therefore \magnitudeIa + \magnitudeIII + \magnitudeIV > \magnitudeIIa + \magnitudeIII + \magnitudeIV$.

$\therefore$ If four magnitudes, \&c.
\stopCenterAlign
\stopProposition

\vfill\pagebreak

\startDefinition[title={Definition X},reference=def:V.X]
\def\varA{\color[byred]{A}}
\def\varB{\color[byyellow]{B}}
\def\varC{\color[byblue]{C}}
\def\vararS{\color[byred]{ar^2}}
\def\varar{\color[byyellow]{ar}}
\def\vara{\color[byblue]{a}}
When three magnitudes are proportionals, the first is said to have to the third the duplicate ratio of that which it has to the second.

For example, if $\varA$, $\varB$, $\varC$, be continued proportionals, that is, $\varA : \varB :: \varB : \varC$, $\varA$ is said to have to $\varC$ the duplicate ratio of $\varA : \varB$;

\startCenterAlign
or $\dfrac{\varA}{\varC} = \mbox{ the square of } \dfrac{\varA}{\varB}$.

This will be more readily seen of the quantities $\vararS$, $\varar$, $\vara$, for $\vararS : \varar :: \varar : \vara$;

and $\dfrac{\vararS}{\vara} = r^2 = \mbox{ the square of } \dfrac{\vararS}{\varar} = \r$,\\
or of $\vara$, $\varar$, $\vararS$;\\
for $\dfrac{\vara}{\vararS} = \dfrac{1}{r^2} = \mbox{ the square of } \dfrac{\vara}{\varar} = \dfrac{1}{r}$.
\stopCenterAlign
\stopDefinition

\startDefinition[title={Definition XI},reference=def:V.XI]
\def\varA{\color[byred]{A}}
\def\varB{\color[byyellow]{B}}
\def\varC{\color[byblue]{C}}
\def\varD{\color[black]{D}}
\def\vararQ{\color[byred]{ar^3}}
\def\vararS{\color[byyellow]{ar^2}}
\def\varar{\color[byblue]{ar}}
\def\vara{\color[black]{a}}
When four magnitudes are continual proportionals, the first is said to have to the fourth the triplicate ratio of that which is has to the second; and so on, quadruplicate, \&c. increasing the denomination still by unity, in any number of proportionals.

For example, let $\varA$, $\varB$, $\varC$, $\varD$, be four continued proportionals, that is, $\varA : \varB :: \varB : \varC :: \varC : \varD$; $\varA$ is said to have to $\varD$, the triplicate ratio of $\varA$ to $\varB$;

\startCenterAlign
or $\dfrac{\varA}{\varD} = \mbox{ the cube of } \dfrac{\varA}{\varB}$.
\stopCenterAlign

This definition will be better understood, and applied to a great number of magnitudes than four that are continued proportionals, as follows :—

\startCenterAlign
Let $\vararQ$, $\vararS$, $\varar$, $\vara$, be four magnitudes in continued proportion, that is, $\vararQ : \vararS :: \vararS : \varar :: \varar : \vara$,\\
then $\dfrac{\vararQ}{\vara} = r^3 = \mbox{ the cube of } \dfrac{\vararQ}{\vararS} = r$.
\stopCenterAlign

Or, let $ar^5$, $ar^4$, $ar^3$, $ar^2$, $ar$, $a$, be six magnitudes in proportion, that is

\startCenterAlign
$ar^5 : ar^4 :: ar^4 : ar^3 :: ar^3 : ar^2 :: ar^2 : ar :: ar : a$,\\
then the ratio $\dfrac{ar^5}{a} = r^5 = \mbox{ the fifth power of } \dfrac{ar^5}{ar^4} = r$
\stopCenterAlign

Or, let $a$, $ar$, $ar^2$, $ar^3$, $ar^4$, be five magnitudes in continued proportion; then $\dfrac{a}{ar^4} = \dfrac{1}{r^4} = \mbox{ the fourth power of } \dfrac{a}{ar} = \dfrac{1}{r}$.
\stopDefinition

\startDefinitionAZ[title={Definition A},reference=def:V.A]
\def\varA{\color[byred]{A}}
\def\varB{\color[byred]{B}}
\def\varC{\color[byred]{C}}
\def\varD{\color[byred]{D}}
\def\varE{\color[byblue]{E}}
\def\varF{\color[byblue]{F}}
\def\varG{\color[byblue]{G}}
\def\varH{\color[byblue]{H}}
\def\varK{\color[byblue]{K}}
\def\varL{\color[byblue]{L}}
\def\varM{\color[byyellow]{M}}
\def\varN{\color[byyellow]{N}}
To know a compound ratio :—

\figureInMargin{
\framed[align=middle]{
$\varA\ \varB\ \varC\ \varD$\\
$\varE\ \varF\ \varG\ \varH\ \varK\ \varL$\\
$\varM\ \varN$
}}
When there are any number of magnitudes of the same kind, the first is said to have to the last of them the ratio compounded of the ratio which the first has to the second, and of the ratio which the second has to the third, and of the ratio which the third has to the fourth; and so on, unto the last magnitude. For example, if $\varA, \varB, \varC, \varD$, be four magnitudes of the same kind, the first $\varA$ is said to have to the last $\varD$ the ratio compounded of the ratio of $\varA$ to $\varB$, and of the ratio of $\varB$ to $\varC$, and of the ratio of $\varC$ to $\varD$; or, the ratio of $\varA$ to $\varD$ is said to be compounded of the ratios of $\varA$ to $\varB$, $\varB$ to $\varC$, and $\varC$ to $\varD$.

And if $\varA$ has to $\varB$ the same ratio which $\varE$ has to $\varF$, and $\varB$ to $\varC$ the same ratio that $\varG$ has to $\varH$, and $\varC$ to $\varD$ the same that $\varK$ has to $\varL$; then by this definition, $\varA$ is said to have to $\varD$ the ratio compounded of ratios which are the same with the ratios of $\varE$ to $\varF$, $\varG$ to $\varH$, and $\varK$ to $\varL$. And the same thing is to be saying, $\varA$ has to $\varD$ the ratio compounded of the ratios of $\varE$ to $\varF$, $\varG$ to $\varH$, and $\varK$ to $\varL$.

In like manner, the same things being supposed; if $\varM$ has to $\varN$ the same ratio which $\varA$ has to $\varD$, then for shortness sake, $\varM$ is said to have to $\varN$ the ratio compounded of the ratios $\varE$ to $\varF$, $\varG$ to $\varH$, and $\varK$ to $\varL$.

This definition may be better understood from an arithmetical or algebraical illustration; for, in fact, a ratio compounded of several other ratios, is nothing more than a ratio which has for its antecedent the continued product of all antecedents of the ratios compounded, and for its consequent the continued product of all the consequents of the ratios compounded.

\startCenterAlign
Thus, the ratio compounded of the ratios of\\
$\color[byred]{2} : \color[byred]{3}$, $\color[byyellow]{4} : \color[byyellow]{7}$, $\color[byblue]{6} : \color[byblue]{11}$, $\color[black]{2} : \color[black]{5}$,\\
is the ratio of $\color[byred]{2} \times \color[byyellow]{4} \times \color[byblue]{6} \times \color[black]{2} : \color[byred]{3} \times \color[byyellow]{7} \times \color[byblue]{11} \times \color[black]{5}$,\\
or the ratio of $96 : 1155$ or $32 : 285$.
\stopCenterAlign

\def\varA{\color[byred]{A}}
\def\varB{\color[byred]{B}}
\def\varC{\color[byyellow]{C}}
\def\varD{\color[byyellow]{D}}
\def\varE{\color[byblue]{E}}
\def\varF{\color[byblue]{F}}
And of the magnitudes $\varA, \varB, \varC, \varD, \varE, \varF$, of the same kind, $\varA : \varF$ is the ratio compounded of the ratios of

\startCenterAlign
$\varA : \varB$, $\varB : \varC$, $\varC : \varD$, $\varD : \varE$, $\varE : \varF$;\\
for $\varA \times \varB \times \varC \times \varD \times \varE : \varB \times \varC \times \varD \times \varE \times \varF$,\\
or $\dfrac{\varA \times \varB \times \varC \times \varD \times \varE}{\varB \times \varC \times \varD \times \varE \times \varF} = \dfrac{\varA}{\varF}$, or the ratio of $\varA : \varF$.
\stopCenterAlign
\stopDefinitionAZ

\vfill\pagebreak

\startPropositionAZ[title={Prop. F. Theor.}, reference=prop:V.F]
\def\varA{\color[byred]{A}}
\def\varB{\color[byyellow]{B}}
\def\varC{\color[byyellow]{C}}
\def\varD{\color[byyellow]{D}}
\def\varE{\color[byred]{E}}
\def\varF{\color[byblue]{F}}
\def\varG{\color[byyellow]{G}}
\def\varH{\color[byyellow]{H}}
\def\varK{\color[byyellow]{K}}
\def\varL{\color[byblue]{L}}
\problemNP{R}{atios}{which are compounded of the same ratios are same to one another.}

\startCenterAlign
Let $\varA : \varB :: \varF : \varG$,\\
 $\varB : \varC :: \varG : \varH$,\\
$\varC : \varD :: \varH : \varK$,\\
and $\varD : \varE :: \varK : \varL$,
\stopCenterAlign
\figureInMargin{
\framed[align=middle]{
$\varA\ \varB\ \varC\ \varD\ \varE$ \\
\ $\varF\ \varG\ \varH\ \varK\ \varL$
}}
Then the ratio which is compounded by the ratios of $\varA : \varB$, $\varB : \varC$, $\varC : \varD$, $\varD : \varE$, or the ratio of $\varA : \varE$, is the same as the ratio compounded of the ratios $\varF : \varG$, $\varG : \varH$, $\varH : \varK$, $\varK : \varL$, or the ratio of $\varF : \varL$.

\startCenterAlign
For $\dfrac{\varA}{\varB} = \dfrac{\varF}{\varG}$,\\
$\dfrac{\varB}{\varC} = \dfrac{\varG}{\varH}$,\\
$\dfrac{\varC}{\varD} = \dfrac{\varH}{\varK}$,\\
and $\dfrac{\varD}{\varE} = \dfrac{\varK}{\varL}$;\\
$\therefore \dfrac{\varA \times \varB \times \varC \times \varD}{\varB \times \varC \times \varD \times \varE} = \dfrac{\varF \times \varG \times \varH \times \varK}{\varG \times \varH \times \varK \times \varL}$,\\
and $\therefore \dfrac{\varA}{\varE} = \dfrac{\varF}{\varL}$,\\
or the ratio of $\varA : \varE$ is the same as the ratio $\varF : \varL$.
\stopCenterAlign

The same may be demonstrated of any number of ratios so circumstanced.

\startCenterAlign
Next, let $\varA : \varB :: \varK : \varL$,\\
 $\varB : \varC :: \varH : \varK$,\\
$\varC : \varD :: \varG : \varH$,\\
and $\varD : \varE :: \varF : \varG$,
\stopCenterAlign

Then the ratio which is compounded of the ratios of $\varA : \varB$, $\varB : \varC$, $\varC : \varD$, $\varD : \varE$, or the ratio of $\varA : \varE$, is the same as the ratio compounded of the ratios of $\varK : \varL$, $\varH : \varK$, $\varG : \varH$, $\varF : \varG$, or the ratio of $\varF : \varL$

\startCenterAlign
For $\dfrac{\varA}{\varB} = \dfrac{\varK}{\varL}$,\\
$\dfrac{\varB}{\varC} = \dfrac{\varH}{\varK}$,\\
$\dfrac{\varC}{\varD} = \dfrac{\varG}{\varH}$,\\
and $\dfrac{\varD}{\varE} = \dfrac{\varF}{\varG}$;\\
$\therefore \dfrac{\varA \times \varB \times \varC \times \varD}{\varB \times \varC \times \varD \times \varE} = \dfrac{\varK \times \varH \times \varG \times \varF}{\varL \times \varK \times \varH \times \varG}$,\\
and $\therefore \dfrac{\varA}{\varE} = \dfrac{\varF}{\varL}$,\\
or the ratio of $\varA : \varE$ is the same as the ratio $\varF : \varL$.

$\therefore$ Ratios which are compounded, \&c.
\stopCenterAlign
\stopPropositionAZ

\vfill\pagebreak

\startPropositionAZ[title={Prop. G. Theor.}, reference=prop:V.G]
\def\varA{\color[byblue]{A}}
\def\varB{\color[byblue]{B}}
\def\varC{\color[byblue]{C}}
\def\varD{\color[byblue]{D}}
\def\varE{\color[byblue]{E}}
\def\varF{\color[byblue]{F}}
\def\varG{\color[byblue]{G}}
\def\varH{\color[byblue]{H}}
\def\varP{\color[byred]{P}}
\def\varQ{\color[byred]{Q}}
\def\varR{\color[byred]{R}}
\def\varS{\color[byred]{S}}
\def\varT{\color[byred]{T}}
\def\varV{\color[byyellow]{V}}
\def\varW{\color[byyellow]{W}}
\def\varX{\color[byyellow]{X}}
\def\varY{\color[byyellow]{Y}}
\def\varZ{\color[byyellow]{Z}}
\problemNP{I}{f}{several ratios be the same to several ratios, each to each, the ratio which is compounded of ratios which are the same to the first ratios, each to each, shall be the same to the ratio compounded of ratios which are the same to the other ratios, each to each.}

\startCenterAlign
\unprotect
\vbox{\halign{\vrule height2.2ex depth1ex\ \hskip 5pt\ # & # & # & # & # & # & # & # \hskip 20pt & # & # & # & # & # \hskip 5pt \vrule \cr
\noalign{\hrule}
$\varA$ & $\varB$ & $\varC$ & $\varD$ & $\varE$ & $\varF$ & $\varG$ & $\varH$ &
$\varP$ & $\varQ$ & $\varR$ & $\varS$ & $\varT$ \cr
$a$ & $b$ & $c$ & $d$ & $e$ & $f$ & $g$ & $h$ &
$\varV$ & $\varW$ & $\varX$ & $\varY$ & $\varZ$ \cr
\noalign{\hrule}
}}
\protect

\unprotect
\setbox0\vbox{\halign{ # & # & # \cr
$\eqalign{
\mbox{ If \ } \varA : \varB & :: a : b\cr
\varC : \varD & :: c : d\cr
\varE : \varF & :: e : f\cr
\mbox{ and \ } \varG : \varH & :: g : h\cr
}$ &
$\eqalign{
\mbox{ and \ } \varA : \varB & :: \varP : \varQ\cr
\varC : \varD & :: \varQ : \varR\cr
\varE : \varF & :: \varR : \varS\cr
\varG : \varH & :: \varS : \varT\cr
}$ &
$\eqalign{
\mbox{ and \ } a : b & :: \varV : \varW\cr
c : d & :: \varW : \varX\cr
e : f & :: \varX : \varY\cr
f : h & :: \varY : \varZ\cr
}$ \cr
}}
\protect
\hskip-\wd0\hskip\textwidth\box0

then $\varP : \varT = \varV : \varZ$.

$\eqalign{
\mbox{ For } \frac{\varP}{\varQ} = \frac{\varA}{\varB}
	& = \frac{a}{b} = \frac{\varV}{\varW} \cr
\frac{\varQ}{\varR} = \frac{\varC}{\varD}
	& = \frac{c}{d} = \frac{\varW}{\varX} \cr
\frac{\varR}{\varS} = \frac{\varE}{\varF}
	& = \frac{e}{f} = \frac{\varX}{\varY} \cr
\frac{\varS}{\varT} = \frac{\varG}{\varH}
	& = \frac{g}{h} = \frac{\varY}{\varZ} \cr
}$

and $\therefore
\dfrac{\varP \times \varQ \times \varR \times \varS}
	{\varQ \times \varR \times \varS \times \varT} =
\dfrac{\varV \times \varW \times \varX \times \varY}
	{\varW \times \varX \times \varY \times \varZ}$,

and $\therefore \dfrac{\varP}{\varT} = \dfrac{\varV}{\varZ}$,

or $\varP : \varT = \varV : \varZ$.

$\therefore$ If several ratios, \&c.
\stopCenterAlign
\stopPropositionAZ

\vfill\pagebreak

\startPropositionAZ[title={Prop. H. Theor.}, reference=prop:V.H]
\def\varA{\color[byred]{A}}
\def\varB{\color[byred]{B}}
\def\varC{\color[byred]{C}}
\def\varD{\color[byred]{D}}
\def\varE{\color[byred]{E}}
\def\varF{\color[byred]{F}}
\def\varG{\color[byred]{G}}
\def\varH{\color[byred]{H}}
\def\varP{\color[byblue]{P}}
\def\varQ{\color[byblue]{Q}}
\def\varR{\color[byblue]{R}}
\def\varS{\color[byblue]{S}}
\def\varT{\color[byblue]{T}}
\def\varX{\color[byblue]{X}}
\problemNP{I}{f}{a ratio which is compounded of several ratios be the same to a ratio which is compounded of several other ratios; and if one of the first ratios, or the ratio which is compounded of several of them; then the remaining ratio of the first, or, if there be more than one, the ratio compounded of the remaining ratios, shall be the same to the remaining ratios, shall be the same to the remaining ratio of the last, or, if there be more than one, to the ratio, compounded of these remaining ratios.}
\figureInMargin{
\framed[align=middle]{
$\varA\ \varB\ \varC\ \varD\ \varE\ \varF\ \varG\ \varH$\\
$\varP\ \varQ\ \varR\ \varS\ \varT\ \varX$\\
}}
Let $\varA : \varB$, $\varB : \varC$, $\varC : \varD$, $\varD : \varE$, $\varE : \varF$, $\varF : \varG$, $\varG : \varH$, be the first ratios, and $\varP : \varQ$, $\varQ : \varR$, $\varR : \varS$, $\varS : \varT$, $\varT : \varX$, the other ratios; also, let $\varA : \varH$, which is compounded of the first ratios, be the same as the ratio of $\varP : \varX$, which is the ratio compounded of the other ratios; and, let the ratio $\varA : \varE$, which is compounded of the ratios of $\varA : \varB$, $\varB : \varC$, $\varC : \varD$, $\varD : \varE$, be the same as the ratio of $\varP : \varR$, which is compounded of the ratios $\varP : \varQ$, $\varQ : \varR$.

Then the ratio which is compounded of the remaining first ratios, that is, the ratio compounded of the ratios $\varE : \varF$, $\varF : \varG$, $\varG : \varH$, that is,the ratio $\varE : \varH$, shall be the same as the ratio $\varR : \varX$, which is compounded of the ratios $\varR : \varS$, $\varS : \varT$, $\varT : \varX$, the remaining ratios.

\startCenterAlign
Because $\dfrac
{\varA \times \varB \times \varC \times \varD \times \varE \times \varF \times \varG}
{\varB \times \varC \times \varD \times \varE \times \varF \times \varG \times \varH} =
\dfrac
{\varP \times \varQ \times \varR \times \varS \times \varT}
{\varQ \times \varR \times \varS \times \varT \times \varX}$,

or $\dfrac
{\varA \times \varB \times \varC \times \varD}
{\varB \times \varC \times \varD \times \varE}
\times
\dfrac
{\varE \times \varF \times \varG}
{\varF \times \varG \times \varH} =
\dfrac
{\varP \times \varQ}
{\varQ \times \varR}
\times
\dfrac
{\varR \times \varS \times \varT}
{\varS \times \varT \times \varX}$,

and $\dfrac
{\varA \times \varB \times \varC \times \varD}
{\varB \times \varC \times \varD \times \varE} =
\dfrac
{\varP \times \varQ}
{\varQ \times \varR}$

$\therefore \dfrac
{\varE \times \varF \times \varG}
{\varF \times \varG \times \varH} =
\dfrac
{\varR \times \varS \times \varT}
{\varS \times \varT \times \varX}$,

$\therefore \dfrac{\varE}{\varH} = \dfrac{\varR}{\varX}$,

$\therefore \varE : =\varH = \varR : \varX$.

$\therefore$ If a ratio which, \&c.
\stopCenterAlign
\stopPropositionAZ

\vfill\pagebreak

\startPropositionAZ[title={Prop. K. Theor.}, reference=prop:V.K]
\def\varA{\color[byred]{A}}
\def\varB{\color[byred]{B}}
\def\varC{\color[byred]{C}}
\def\varD{\color[byred]{D}}
\def\varE{\color[byred]{E}}
\def\varF{\color[byred]{F}}
\def\varG{\color[byred]{G}}
\def\varH{\color[byred]{H}}
\def\varK{\color[byred]{K}}
\def\varL{\color[byred]{L}}
\def\varM{\color[byred]{M}}
\def\varN{\color[byred]{N}}
\def\varO{\color[byyellow]{O}}
\def\varP{\color[byyellow]{P}}
\def\varQ{\color[byyellow]{Q}}
\def\varR{\color[byyellow]{R}}
\def\varS{\color[byyellow]{S}}
\def\varT{\color[byyellow]{T}}
\def\varV{\color[byyellow]{V}}
\def\varW{\color[byyellow]{W}}
\def\varX{\color[byyellow]{X}}
\def\varY{\color[byyellow]{Y}}
\def\vara{\color[byblue]{a}}
\def\varb{\color[byblue]{b}}
\def\varc{\color[byblue]{c}}
\def\vard{\color[byblue]{d}}
\def\vare{\color[byblue]{e}}
\def\varf{\color[byblue]{f}}
\def\varg{\color[byblue]{g}}
\problemNP{I}{f}{there be any number of ratios, and any number of other ratios, such that the ratio which is compounded of ratios, which are the same to the first ratios, each to each, is the same to the ratio which is compounded of ratios, which are the same, each to each, to the last ratios—and if one of the first ratios, or the ratio which is compounded of ratios, which are the same to several of the first ratios, each to each, be the same to one of the last ratios, or to the ratio which is compounded of ratios, which are the same, each to each, to several of the last ratios—then the remaining ratio of the first; or, if there be more than one, the ratio which is compounded of ratios, which are the same, each to each, to the remaining ratios of the first, shall be the same to the remaining ratio of the last; or, if there be more than one, to the ratio which is compounded of ratios, which are the same, each to each, to these remaining ratios.}


\unprotect
\vbox{\halign{\vrule height2.2ex depth1ex\ \hskip 2pt\
#&#& #&#& #&#& #&#& #&#& #&# & \hfil#\hfil
\hskip 2pt \vrule \cr
\noalign{\hrule}
& & & & & & & $h$ & $k$ & $m$ & $n$ & $s$ & \cr
$\varA$ & $\varB$, & $\varC$ & $\varD$, & $\varE$ & $\varF$, & $\varG$ & $\varH$, & $\varK$ & $\varL$, & $\varM$ & $\varN$ & $\vara\ \varb\ \varc\ \vard\ \vare\ \varf\ \varg$ \cr
$\varO$ & $\varP$, & $\varQ$ & $\varR$, & $\varS$ & $\varT$, & $\varV$ & $\varW$, & $\varX$ & $\varY$ & & & $h\ k\ l\ m\ n\ p$ \cr
& & $a$ & $b$ & $k$ & $m$ & & $e$ & $f$ & $g$ & & & \cr
\noalign{\hrule}
}}
\protect

Let $\varA : \varB$, $\varC : \varD$, $\varE : \varF$, $\varG : \varH$, $\varK : \varL$, $\varM : \varN$, be the first ratios, and $\varO : \varP$, $\varQ : \varR$, $\varS : \varT$, $\varV : \varW$, $\varX : \varY$, the other ratios;

\startCenterAlign
$\eqalign{
\mbox{and let }
\varA : \varB &= \vara : \varb ,\cr
\varC : \varD &= \varb : \varc ,\cr
\varE : \varF &= \varc : \varf ,\cr
\varG : \varH &= \vard : \vare ,\cr
\varK : \varL &= \vare : \varf ,\cr
\varM : \varN &= \varf : \varg .
}$
\stopCenterAlign

Then, by the definition of a compound ratio, the ratio of $\vara : \varg$ is compounded of the ratios $\vara : \varb$, $\varb : \varc$, $\varc : \vard$, $\vard : \vare$, $\vare : \varf$, $\varf : \varg$, which are the same as the ratio of $\varA : \varB$, $\varC : \varD$, $\varE : \varF$, $\varG : \varH$, $\varK : \varL$, $\varM : \varN$, each to each.

\startCenterAlign
$\eqalign{
\mbox{Also, }
\varO : \varP &= h : k ,\cr
\varQ : \varR &= k : l ,\cr
\varS : \varT &= l : m ,\cr
\varV : \varW &= m : n ,\cr
\varX : \varY &= n : p ,\cr
}$
\stopCenterAlign

Then will the ratio of $h : p$ be the ratio compounded of the ratios $h : k$, $k : l$, $l : m$, $m : n$, $n : p$, which are thesame as the ratios of $\varO : \varP$, $\varQ : \varR$, $\varS : \varT$, $\varV : \varW$, $\varX : \varY$, each to each.

\startCenterAlign
$\therefore$ by the hypothesis $\vara : \varg = h : p$.
\stopCenterAlign

Also, let the ratio which is compounded if the ratios of $\varA : \varB$, $\varC : \varD$, two of the first ratios (or the ratios of $\vara : \varc$ for $\varA : \varB = \vara : \varb$, and $\varC : \varD = \varb : \varc$), be the same as the ratio of $a : d$, which is compounded of the ratios of $a : b$, $b : c$, $c : d$, which are the same as the ratios of $\varO : \varP$, $\varQ : \varR$, $\varS : \varT$, three of the other ratios.

And let thre ratios of $h : s$, which is compounded of the ratios $h : k$, $k : m$, $m : n$, $n : s$, which are the same as the remaining first ratios, namely, $\varE : \varF$, $\varG : \varH$, $\varK : \varL$, $\varM : \varN$; also, let the ratio of $e : g$, be that which is compounded of the ratios $e : f$ , $f : g$, which are thre same, each to each, to the remaining other ratios, namely $\varV : \varW$, $\varX : \varY$. Then the ratios of $h : s$ shall be the same as the ratio of $e : g$; or $h : s = e : g$.

\startCenterAlign
For $\dfrac
{\varA \times \varC \times \varE \times \varG \times \varK \times \varM}
{\varB \times \varD \times \varF \times \varH \times \varL \times \varN}
=
\dfrac
{\vara \times \varb \times \varc \times \vard \times \vare \times \varf}
{\varb \times \varc \times \vard \times \vare \times \varf \times \varg}$,

and $\dfrac
{\varO \times \varQ \times \varS \times \varV \times \varX}
{\varP \times \varR \times \varT \times \varW \times \varY}
=
\dfrac
{h \times k \times l \times m \times n}
{k \times l \times m \times n \times p}$,\\
by the composition of the ratios;

$\therefore \dfrac
{\vara \times \varb \times \varc \times \vard \times \vare \times \varf}
{\varb \times \varc \times \vard \times \vare \times \varf \times \varg}
=
\dfrac
{h \times k \times l \times m \times n}
{k \times l \times m \times n \times p}$ (hyp.),

or $\dfrac{\vara \times \varb}{\varb \times \varc}
\times
\dfrac
{\varc \times \vard \times \vare \times \varf}
{\vard \times \vare \times \varf \times \varg}
=
\dfrac
{h \times k \times l}
{k \times l \times m}
\times
\dfrac
{m \times n}
{n \times p}$,

but $\dfrac{\vara \times \varb}{\varb \times \varc} = \dfrac{\varA \times \varC}{\varB \times \varD}
=
\dfrac
{\varO \times \varQ \times \varS}
{\varP \times \varR \times \varT}
=
\dfrac{a \times b \times c}{b \times c \times d}
=
\dfrac{h \times k \times l}{k \times l \times m}$;

$\therefore
\dfrac
{\varc \times \vard \times \vare \times \varf}
{\vard \times \vare \times \varf \times \varg}
=
\dfrac
{m \times n}
{n \times p}$.

And $\dfrac
{\varc \times \vard \times \vare \times \varf}
{\vard \times \vare \times \varf \times \varg}
= \dfrac
{h \times k \times l \times m \times n}
{k \times l \times m \times n \times p}$ (hyp.),

and $\dfrac
{m \times n}
{n \times p}
=
\dfrac
{e \times f}
{f \times g}$ (hyp.),

$\therefore \dfrac
{h \times k \times l \times m \times n}
{k \times l \times m \times n \times p}
= \dfrac{e f}{f g}$,

$\therefore \dfrac{h}{s} = \dfrac{e}{g}$,\\
$\therefore h : s = e : g$.

$\therefore$ If there be any number, \&c.
\stopCenterAlign
\stopPropositionAZ
\stopbook

\startbook[title={Book 6}]\unskip

\vskip -\baselineskip

\startsupersection[title={Definitions}]\unskip

\startDefinitionOnlyNumber[reference=def:VI.I]
\defineNewPicture{
pair a, b, c, A, B, C, d, e, f, g, D, E ,F ,G;
a := (0, 0);
b := (u, 0);
c := (1/2u, u);
A := (a scaled 2) shifted (u, -u);
B := (b scaled 2) shifted (u, -u);
C := (c scaled 2) shifted (u, -u);
d := (0, 0);
e := (2/3u, 0);
f := (u, 2/3u);
g := (1/3u, 2/3u);
D := d scaled 2;
E := e scaled 2;
F := f scaled 2;
G := g scaled 2;
d := d shifted (0, -5/2u);
e := e shifted (0, -5/2u);
f := f shifted (0, -5/2u);
g := g shifted (0, -5/2u);
D := D shifted (u, -5/2u);
E := E shifted (u, -5/2u);
F := F shifted (u, -5/2u);
G := G shifted (u, -5/2u);
draw byPolygon(a,b,c)(byblue);
draw byPolygon(A,B,C)(byblue);
draw byPolygon(d,e,f,g)(byred);
draw byPolygon(D,E,F,G)(byred);
}\drawCurrentPictureInMargin
Rectilinear figures are said to be similar, when they have their several angles equal, each to each, and the sides about the equal angles proportional.
\stopDefinitionOnlyNumber

\startDefinitionOnlyNumber[reference=def:VI.II]
Two sides of one figure are said to be reciprocally proportional to two sides of another figure when one of the sides of the first is to the second, as the remaining side of the second is to the remaining side of the first.
\stopDefinitionOnlyNumber

\vskip -\baselineskip

\startDefinitionOnlyNumber[reference=def:VI.III]
A straight line is said to be cut in extreme and mean ratio, when the whole is to the greater segment, as the greater segment is to the less.
\stopDefinitionOnlyNumber

\startDefinitionOnlyNumber[reference=def:VI.IV]
\defineNewPicture{
pair d[];
d1 := (0, 0);
d2 := (5/2u, 0);
d3 := (5u, 0);
d4 := (15/2u, 0);
numeric h;
h := 6/5u;
pair A, B, C, H;
A := (0, 0) shifted d1;
B := (3/2u, 0) shifted d1;
C := (3/4u, h) shifted d1;
H := (xpart(C), 0);
draw byPolygon (A,B,C)(byyellow);
draw byLine(C, H)(byyellow, 1, 0);
pair A, B, C, D, H, G;
A := (0, 0) shifted d2;
B := (1/2u, 0) shifted d2;
C := (3/2u, h) shifted d2;
D := (u, h) shifted d2;
H := (xpart(C), 0);
G := (xpart(D), 0);
draw byPolygon (A,B,C,D)(byblue);
byLineDefine(C, H, byblue, 1, 0);
draw byLine(D, G, byblue, 1, 0);
byLineDefine(A, H, byblue, 1, 0);
draw byNamedLineSeq(0)(CH,AH);
pair A, B, C, D, H;
A := (0, 0) shifted d3;
B := (2/3u, 0) shifted d3;
C := (u, h) shifted d3;
H := (xpart(C), 0);
draw byPolygon (A,B,C)(black);
byLineDefine(C, H, black, 1, 0);
byLineDefine(A, H, black, 1, 0);
draw byNamedLineSeq(0)(CH,AH);
pair A, B, C, D, E, H;
A := (0, 0) shifted d4;
B := (1/2u, 0) shifted d4;
C := (u, 1/2h) shifted d4;
D := (1/4u, h) shifted d4;
E := (-1/2u, 1/2h) shifted d4;
H := (xpart(D), 0);
draw byPolygon (A,B,C,D,E)(byred);
draw byLine(D, H)(byred, 1, 0);
}
The altitude of any figure is the straight line drawn from its vertex perpendicular to its base, or the base produced.

\noindent\ \hfill\reuseMPgraphic{\currentInstance::currentPicture}\hfill\
\stopDefinitionOnlyNumber
\stopsupersection

\vfill\pagebreak

\startProposition[title={Prop. I. theor.},reference=prop:VI.I]
\defineNewPicture[2/5]{
pair A, B, C, D, H, G, K, L, M, N, O, P, Q;
numeric w, h;
w := 9/2u;
h := 3u;
A := (6/11w, h);
B := (5/11w, 0);
C := (6/11w, 0);
D := (7/11w, 0);
H := (3/11w, 0);
G := (4/11w, 0);
K := (8/11w, 0);
L := (9/11w, 0);
M := (0/11w, 0);
N := (1/11w, 0);
O := (2/11w, 0);
P := (10/11w, 0);
Q := (11/11w, 0);
draw byPolygon(A,M,N)(black);
draw byPolygon(A,O,H)(black);
draw byPolygon(A,G,B)(black);
draw byPolygon(A,B,C)(byred);
draw byPolygon(A,C,D)(byblue);
draw byPolygon(A,K,L)(byyellow);
draw byPolygon(A,P,Q)(byyellow);
draw byLine(M, B)(black, 0, 0);
draw byLine(B, C)(byblue, 0, 0);
draw byLine(C, D)(byred, 0, 0);
draw byLine(D, Q)(black, 0, 0);
}
\drawCurrentPictureInMargin
\problemNP{T}{riangles}{and parallelograms having the same altitude are to one another as their bases.}

Let the triangles \drawPolygon[bottom]{ABC} and \drawPolygon[bottom]{ACD} have a common vertex and their bases \drawUnitLine{BC} and \drawUnitLine{CD} in the same straight line.

Produce \drawUnitLine{BC,CD} both ways, take successively on \drawUnitLine{CD} produced lines equal to it; and on \drawUnitLine{BC} produced lines successively equal to it; and draw lines from the common vertex to their extremities.

The triangles
\drawFromCurrentPicture[bottom][trianglesAMC]{
draw byNamedPolygon(AMN,AOH,AGB,ABC);
draw byNamedLine(MB,BC);
}
thus formed are all equal to one another, since their bases are equal. \inprop[prop:I.XXXVIII]

$\therefore$ \trianglesAMC\ and its base are respectively equimultiples of \polygonABC\ and the base \drawUnitLine{BC}.

In like manner
\drawFromCurrentPicture[bottom][trianglesACQ]{
draw byNamedPolygon(ACD,AKL,APQ);
draw byNamedLine(CD,DQ);
}
and its base are respectively equimultiples of \polygonACD\ and the base \drawUnitLine{CD}.

$\therefore$ if $m$ or $6$ times \polygonABC\ $>, = \mbox{ or } < n$ or $5$ times \polygonACD\ then $m$ or $6$ times \drawUnitLine{BC} $>, = \mbox{ or } < n$ or $5$ times \drawUnitLine{CD}, $m$ and $n$ stand for every multiple taken as in the fifth definition of the Fifth Book. Although we have only shown that this property exists when $m$ equal $6$, and $n$ equal $5$, yet it is evident that the property holds good for every multiple value that may be given to $m$, and to $n$.

\startCenterAlign
$\therefore \polygonABC\ : \polygonACD\ :: \drawUnitLine{BC} : \drawUnitLine{CD}$ \indef[def:V.V]
\stopCenterAlign

Parallelograms having the same altitude are the doubles of the triangles, on their bases, and are proportional to them (Part I), and hence their doubles, the parallelograms, are as their bases. \inprop[prop:V.XV]

\qed
\stopProposition

\startProposition[title={Prop. II. theor.},reference=prop:VI.II]
\defineNewPicture[1/2]{
pair A, B, C, D, E, Ab, Bb, Cb, Db, Eb, Ac, Bc, Cc, Dc, Ec, Fc, d[];
d1 := (0, 0);
d2 := (u, -4u);
d3 := (5/2u, -9/2u);
A := (0, 0) shifted d1;
B := (-1/3u, -7/2u) shifted d1;
C := B shifted (5/2u, 0);
D := 1/2[A, B];
E := 1/2[A, C];
Ab := (0, 0) shifted d2;
Bb := (-u, -2u) shifted d2;
Cb := Bb shifted (2u, 0);
Db := 4/3[Ab, Bb];
Eb := 4/3[Ab, Cb];
Ac := (0, 0) shifted d3;
Bc := (-3u, -5u) shifted d3;
Cc := Bc shifted (2u, 0);
Dc := 3/5[Ac, Bc];
Ec := 3/5[Ac, Cc];
Fc = whatever[Bc, Ec] = whatever[Cc, Dc];
draw byLine(D, E, black, 0, 0);
draw byLine(B, E, byred, 0, 0);
draw byLine(C, D, byblue, 0, 0);
byLineDefine(A, D, byred, 1, 0);
byLineDefine(D, B, byyellow, 0, 0);
byLineDefine(B, C, black, 1, 0);
byLineDefine(A, E, byblue, 1, 0);
byLineDefine(E, C, byyellow, 1, 0);
draw byNamedLineSeq(0)(AD,DB,BC,EC,AE);
draw byLine(Bb, Eb, byred, 0, 0);
draw byLine(Cb, Db, byblue, 0, 0);
draw byLine(Bb, Cb, black, 1, 0);
byLineDefine(Ab, Bb, byred, 1, 0);
byLineDefine(Bb, Db, byyellow, 0, 0);
byLineDefine(Db, Eb, black, 0, 0);
byLineDefine(Ab, Cb, byblue, 1, 0);
byLineDefine(Cb, Eb, byyellow, 1, 0);
draw byNamedLineSeq(0)(AbBb,BbDb,DbEb,CbEb,AbCb);
draw byLine(Bc, Fc, byyellow, 0, 0);
draw byLine(Fc, Ec, byred, 1, 0);
draw byLine(Cc, Fc, byyellow, 1, 0);
draw byLine(Fc, Dc, byblue, 1, 0);
byLineDefine(Bc, Cc, black, 1, 0);
byLineDefine(Bc, Dc, byred, 0, 0);
byLineDefine(Dc, Ec, black, 0, 0);
byLineDefine(Cc, Ec, byblue, 0, 0);
draw byNamedLineSeq(0)(BcCc,BcDc,DcEc,CcEc);
}
\drawCurrentPictureInMargin
\problemNP{I}{f}{a straight line \drawUnitLine{DE} be drawn parallel to any side \drawUnitLine{BC} of a triangle, it shall cut the other sides, or those sides produced, into proportional segments\\
And if any straight line \drawUnitLine{DE} divide the sides of a triangle, or those sides produced, into proportional segments, it is parallel to the remaining side \drawUnitLine{BC}.}

\startsubproposition[title={Part I.}]
\startCenterAlign
Let $\drawUnitLine{DE} \parallel \drawUnitLine{BC}$, then shall\\
$\drawUnitLine{DB} : \drawUnitLine{AD} :: \drawUnitLine{EC} : \drawUnitLine{AE}$.

Draw \drawUnitLine{BE} and \drawUnitLine{CD},\\
and
$\drawLine[middle][triangleBDE]{DE,BE,DB}
=
\drawLine[middle][triangleCDE]{EC,CD,DE}$ \inprop[prop:I.XXXVII];

$\therefore \triangleBDE\ :
\drawLine[middle][triangleADE]{AD,AE,DE} :: \triangleCDE\ : \triangleADE$ \inprop[prop:V.VII];\\
but $\triangleBDE\ : \triangleADE\ :: \drawUnitLine{DB} : \drawUnitLine{AD}$ \inprop[prop:VI.I],\\
$\therefore \drawUnitLine{DB} : \drawUnitLine{AD} :: \drawUnitLine{EC} : \drawUnitLine{AE}$ \inprop[prop:V.XI].
\stopCenterAlign
\stopsubproposition

\vfill\pagebreak

\startsubproposition[title={Part II.}]
\startCenterAlign
Let $\drawUnitLine{DB} : \drawUnitLine{AD} :: \drawUnitLine{EC} : \drawUnitLine{AE}$.\\
then $\drawUnitLine{DE} \parallel \drawUnitLine{BC}$.

Let the same construction remain,\\
$\left.\eqalign{
\mbox{ because } \drawUnitLine{DB} : \drawUnitLine{AD} &:: \triangleBDE\ : \triangleADE \cr
\mbox{ and } \drawUnitLine{EC} : \drawUnitLine{AE} &:: \triangleCDE\ : \triangleADE \cr
}\right\}\mbox{ \inprop[prop:VI.I] }$;\\
but $\drawUnitLine{DB} : \drawUnitLine{AD} :: \drawUnitLine{EC} : \drawUnitLine{AE}$ (hyp.)\\
$\therefore \triangleBDE\ : \triangleADE :: \triangleCDE\ : \triangleADE$ \inprop[prop:V.XI]
$\therefore \triangleBDE\ = \triangleCDE$ \inprop[prop:V.IX];

but they are on the same base \drawUnitLine{BC}, and at the same side of it, and\\
$\therefore \drawUnitLine{DE} \parallel \drawUnitLine{BC}$ \inprop[prop:I.XXXIX].
\stopCenterAlign
\stopsubproposition

\qed
\stopProposition

\startProposition[title={Prop. III. theor.},reference=prop:VI.III]
\defineNewPicture{
pair A, B, C, D, E;
numeric a;
a := 80;
A := (0, 0);
B := A shifted (dir(220)*3u);
C = whatever[A, A + dir(220+a)] = whatever[B, B shifted dir(0)];
D = whatever[A, A + dir(220+1/2a)] = whatever[B, C];
E = whatever[A, B] = whatever[C, C shifted (A-D)];
draw byAngle(B, A, D, byyellow, 0);
draw byAngle(D, A, C, black, 0);
draw byAngle(C, A, E, black, 1);
draw byAngleWithName(A, E, C, byblue, 0)(E);
draw byAngleWithName(E, C, A, byred, 0)(C);
draw byLine(C, A, byyellow, 0, 0);
draw byLine(A, D, byblue, 0, 0);
byLineDefine(A, E, byred, 1, 0);
byLineDefine(C, E, byblue, 1, 0);
byLineDefine(A, B, byred, 0, 0);
byLineDefine(B, D, black, 0, 0);
byLineDefine(D, C, black, 1, 0);
draw byNamedLineSeq(0)(AE,CE,DC,BD,AB);
}
\drawCurrentPictureInMargin
\problemNP{A}{right}{line (\drawUnitLine{AD}) bisecting the angle of a triangle, divides the opposite side into segments (\drawUnitLine{BD}, \drawUnitLine{DC}) proportional to the conterminous sides (\drawUnitLine{AB}, \drawUnitLine{CA}).\\
And if a straight line (\drawUnitLine{AD}) drawn from any angle of a triangle divide the opposite side (\drawUnitLine{BD,DC}) into segments (\drawUnitLine{BD}, \drawUnitLine{DC}) proportional to the conterminous sides (\drawUnitLine{AB}, \drawUnitLine{CA}), it bisects the angle.}

\startsubproposition[title={Part I.}]
\startCenterAlign
Draw $\drawUnitLine{CE} \parallel \drawUnitLine{AD}$, to meet \drawUnitLine{AE};\\
then, $\drawAngle{BAD} = \drawAngle{E}$ \inprop[prop:I.XXIX].

$\therefore \drawAngle{DAC} = \drawAngle{E}$; but $\drawAngle{DAC} = \drawAngle{C}$, $\therefore \drawAngle{C} = \drawAngle{E}$,\\
$\therefore \drawUnitLine{AE} = \drawUnitLine{CA}$ \inprop[prop:I.VI];\\
and because $\drawUnitLine{AD} \parallel \drawUnitLine{CE}$,\\
$\drawUnitLine{AE} : \drawUnitLine{AB} :: \drawUnitLine{DC} : \drawUnitLine{BD}$ \inprop[prop:VI.II];

but $\drawUnitLine{AE} = \drawUnitLine{CA}$;\\
$\therefore \drawUnitLine{CA} : \drawUnitLine{AB} :: \drawUnitLine{DC} : \drawUnitLine{BD}$ \inprop[prop:V.VII].
\stopCenterAlign
\stopsubproposition

\vfill\pagebreak

\startsubproposition[title={Part II.}]
\startCenterAlign
Let the same construction remain,\\
and $\drawUnitLine{AB} : \drawUnitLine{AE} :: \drawUnitLine{BD} : \drawUnitLine{DC}$ \inprop[prop:VI.II];

but $\drawUnitLine{BD} : \drawUnitLine{DC} :: \drawUnitLine{AB} : \drawUnitLine{CA}$ (hyp.)\\
$\therefore \drawUnitLine{AB} : \drawUnitLine{AE} :: \drawUnitLine{AB} : \drawUnitLine{CA}$ \inprop[prop:V.XI].

and $\therefore \drawUnitLine{AE} = \drawUnitLine{CA}$ \inprop[prop:V.IX],\\
and $\therefore \drawAngle{E} = \drawAngle{C}$ \inprop[prop:I.V];\\
but since $\drawUnitLine{AD} \parallel \drawUnitLine{CE}$; $\drawAngle{DAC} = \drawAngle{C}$,\\
and $\drawAngle{BAD} = \drawAngle{E}$ \inprop[prop:I.XXIX];

$\therefore \drawAngle{C} = \drawAngle{E}$, and $\drawAngle{BAD} = \drawAngle{DAC}$,\\
and $\therefore$ \drawUnitLine{AD} bisects \drawAngle{BAD,DAC}.
\stopCenterAlign
\stopsubproposition

\qed
\stopProposition

\startProposition[title={Prop. IV. theor.},reference=prop:VI.IV]
\defineNewPicture{
pair A, B, C, D, E, F;
B := (0, 0);
A := (5/4u, u);
C := (2u, 0);
D := (A scaled 4/3) shifted (C-B);
E := (C scaled 4/3) shifted (C-B);
F = whatever[A, B] = whatever[D, E];
draw byAngleWithName(A, B, C, byyellow, 0)(B);
draw byAngle(B, C, A, byblue, 0);
draw byAngleWithName(C, A, B, black, 1)(A);
draw byAngle(D, C, E, byred, 0);
draw byAngleWithName(C, E, D, black, 0)(E);
draw byAngleWithName(E, D, C, byred, 1)(D);
draw byLine(C, A, byred, 0, 0);
draw byLine(C, D, byblue, 1, 0);
byLineDefine(A, F, byyellow, 1, 0);
byLineDefine(D, F, byyellow, 0, 0);
byLineDefine(A, B, byblue, 0, 0);
byLineDefine(D, E, byred, 1, 0);
byLineDefine(E, C, black, 1, 0);
byLineDefine(B, C, black, 0, 0);
draw byNamedLineSeq(0)(AF,DF,DE,EC,BC,AB);
}
\drawCurrentPictureInMargin
\problemNP{I}{n}{equiangular triangles
(\drawLine[bottom][triangleABC]{CA,BC,AB}
and
\drawLine[bottom][triangleCDE]{CD,DE,EC})
the sides about the equal angles are proportional, and the sides which are opposite to the equal angles are homologous.
}

Let the equiangular triangles be so placed that two sides \drawUnitLine{BC}, \drawUnitLine{EC} opposite to equal angles \drawAngle{D} and \drawAngle{A} may be conterminous, and in the same straight line; and that the triangles lying at the same side of that straight line; and that the triangles lying at the same side of that straight line, may have the equal angles not conterminous, i. e. \drawAngle{DCE} opposite to \drawAngle{B}, and \drawAngle{BCA} to \drawAngle{E}.

\startCenterAlign
Draw \drawUnitLine{AF} and \drawUnitLine{DF}.\\
Then, because $\drawAngle{BCA} = \drawAngle{E}$, $\drawUnitLine{CA} \parallel \drawUnitLine{DF,DE}$ \inprop[prop:I.XXVIII];\\
and for a like reason $\drawUnitLine{EC} \parallel \drawUnitLine{AB,AF}$,\\
$\therefore$
\drawFromCurrentPicture[bottom][polygonACDF]{
startGlobalRotation(-lineAngle.CD);
draw byNamedLineSeq(0)(CA,AF,DF,CD);
stopGlobalRotation;
}
is a parallelogram.

But $\drawUnitLine{BC} : \drawUnitLine{EC} :: \drawUnitLine{DF} : \drawUnitLine{DE}$ \inprop[prop:VI.II];\\
and since $\drawUnitLine{DF} = \drawUnitLine{CA}$ \inprop[prop:I.XXXIV],\\
$\drawUnitLine{BC} : \drawUnitLine{EC} :: \drawUnitLine{CA} : \drawUnitLine{DE}$;\\
and by alternation $\drawUnitLine{BC} : \drawUnitLine{CA} :: \drawUnitLine{EC} : \drawUnitLine{DE}$ \inprop[prop:V.XVI].

In like manner it may be shown, that\\
$\drawUnitLine{AB} : \drawUnitLine{CD} :: \drawUnitLine{BC} : \drawUnitLine{EC}$;\\
and by alternation, that\\
$\drawUnitLine{AB} : \drawUnitLine{BC} :: \drawUnitLine{CD} : \drawUnitLine{EC}$;\\
but it has been already proved that\\
$\drawUnitLine{BC} : \drawUnitLine{CA} :: \drawUnitLine{EC} : \drawUnitLine{DE}$\\
and therefore, ex \ae quali,\\
$\drawUnitLine{AB} : \drawUnitLine{CA} :: \drawUnitLine{CD} : \drawUnitLine{DE}$ \inprop[prop:V.XXII],\\
therefore the sides about the equal angles are proportional, and those which are opposite to the equal angles are homologous.
\stopCenterAlign

\qed
\stopProposition

\startProposition[title={Prop. V. theor.},reference=prop:VI.V]
\defineNewPicture{
pair A, B, C, D, E, F, G, d;
B := (0, 0);
A := (2u, 5/2u);
C := (3u, 0);
draw byAngleWithName(B, A, C, byyellow, 0)(A);
draw byAngleWithName(A, B, C, byblue, 0)(B);
draw byAngleWithName(B, C, A, byred, 0)(C);
byLineDefine(A, B, byred, 1, 0);
byLineDefine(B, C, black, 1, 0);
byLineDefine(C, A, byblue, 1, 0);
draw byNamedLineSeq(0)(AB,BC,CA);
d := (0, -3u);
D := A shifted d;
E := B shifted d;
F := C shifted d;
G := (A yscaled -1) shifted d;
draw byAngleWithName(F, D, E, black, 0)(D);
draw byAngle(D, E, F, byyellow, 0);
draw byAngle(E, F, D, byred, 1);
draw byAngleWithName(E, G, F, black, 0)(G);
draw byAngle(G, E, F, byblue, 0);
draw byAngle(E, F, G, byred, 0);
draw byLine(E, F, black, 0, 0);
byLineDefine(F, G, byyellow, 1, 0);
byLineDefine(G, E, byyellow, 0, 0);
byLineDefine(D, E, byred, 0, 0);
byLineDefine(F, D, byblue, 0, 0);
draw byNamedLineSeq(0)(FG,GE,DE,FD);
}
\drawCurrentPictureInMargin
\problemNP{I}{f}{two triangles have their sides proportional
($\drawUnitLine{CA} : \drawUnitLine{BC} :: \drawUnitLine{FD} : \drawUnitLine{EF}$)
and
($\drawUnitLine{BC} : \drawUnitLine{AB} :: \drawUnitLine{EF} : \drawUnitLine{DE}$)
they are equiangular, and the equal angles are subtended by the homologous sides.
}
\startCenterAlign
From the extremities of \drawUnitLine{EF}, draw \drawUnitLine{FG} and \drawUnitLine{GE}, making\\
$\drawAngle{GEF} = \drawAngle{B}$, $\drawAngle{EFG} = \drawAngle{C}$ \inprop[prop:I.XXIII];\\
and consequently $\drawAngle{G} = \drawAngle{A}$ \inprop[prop:I.XXXII],\\
and since the triangles are equiangular,\\
$\drawUnitLine{AB} : \drawUnitLine{BC} :: \drawUnitLine{GE} : \drawUnitLine{EF}$ \inprop[prop:VI.IV];\\
but $\drawUnitLine{AB} : \drawUnitLine{BC} :: \drawUnitLine{DE} : \drawUnitLine{EF}$ (hyp.);\\
$\therefore \drawUnitLine{DE} : \drawUnitLine{EF} :: \drawUnitLine{GE} : \drawUnitLine{EF}$ \inprop[prop:V.IX].\\
In like manner it may be shown that\\
$\drawUnitLine{FD} = \drawUnitLine{FG}$.
\stopCenterAlign

\startCenterAlign
Therefore, the two triangles having a common base \drawUnitLine{EF}, and their sides equal, have also equal angles opposite to equal sides, i. e.\\
$\drawAngle{DEF} = \drawAngle{GEF}$ and $\drawAngle{EFD} = \drawAngle{EFG}$ \inprop[prop:I.VIII].\\
but $\drawAngle{GEF} = \drawAngle{B}$ (const.) and $\therefore \drawAngle{DEF} = \drawAngle{B}$;\\
for the same reason $\drawAngle{EFD} = \drawAngle{C}$,\\
and consequently $\drawAngle{D} = \drawAngle{A}$ \inprop[prop:I.XXXII];\\
and therefore the triangles are equiangular, and it is evident that the homologous sides subtended by the equal angles.
\stopCenterAlign

\qed
\stopProposition

\startProposition[title={Prop. VI. theor.},reference=prop:VI.VI]
\defineNewPicture[1/4]{
pair A, B, C, D, E, F, G, d;
A := (0, 0);
B := (2u, 5/2u);
C := (3u, 0);
draw byAngleWithName(A, B, C, byyellow, 0)(B);
draw byAngleWithName(B, A, C, byblue, 0)(A);
draw byAngleWithName(A, C, B, byred, 0)(C);
byLineDefine(A, B, byred, 1, 0);
byLineDefine(C, A, black, 1, 0);
byLineDefine(B, C, byblue, 1, 0);
draw byNamedLineSeq(0)(AB,BC,CA);
d := (0, -3u);
D := A shifted d;
E := B shifted d;
F := C shifted d;
G := (B yscaled -1) shifted d;
draw byAngle(F, D, E, byblue, 1);
draw byAngleWithName(D, E, F, byyellow, 1)(E);
draw byAngle(E, F, D, byred, 1);
draw byAngle(F, D, G, byblue, 0);
draw byAngle(G, F, D, byred, 0);
draw byAngleWithName(D, G, F, black, 0)(G);
draw byLine(F, D, byblue, 0, 0);
byLineDefine(F, G, byyellow, 1, 0);
byLineDefine(G, D, byyellow, 0, 0);
byLineDefine(D, E, byred, 0, 0);
byLineDefine(E, F, black, 0, 0);
draw byNamedLineSeq(0)(FG,GD,DE,EF);
}
\drawCurrentPictureInMargin
\problemNP{I}{f}{two triangles (\drawLine[bottom][triangleABC]{AB,BC,CA} and \drawLine[bottom][triangleDEF]{DE,EF,FD}) have one angle (\drawAngle{C}) of the one, equal to one angle (\drawAngle{FDE}) of the other, and the sides about the equal angles proportional, the triangles shall be equiangular, and have those angles equal which the homologous sides subtend.}

\startCenterAlign
From the extremities of \drawUnitLine{FD}, on of the sides of \triangleDEF, about \drawAngle{EFD},\\
draw \drawUnitLine{GD} and \drawUnitLine{FG},\\
making $\drawAngle{GFD} = \drawAngle{C}$,\\
and $\drawAngle{FDG} = \drawAngle{A}$;\\
then $\drawAngle{G} = \drawAngle{B}$ \inprop[prop:I.XXXII],\\
and two triangles being equiangular,\\
$\drawUnitLine{BC} : \drawUnitLine{CA} :: \drawUnitLine{FG} : \drawUnitLine{FD}$ \inprop[prop:VI.IV];\\
but $\drawUnitLine{BC} : \drawUnitLine{CA} :: \drawUnitLine{EF} : \drawUnitLine{FD}$ (hyp.)\\
$\therefore \drawUnitLine{FG} : \drawUnitLine{FD} :: \drawUnitLine{EF} : \drawUnitLine{FD}$ \inprop[prop:V.XI],\\
and consequently $\drawUnitLine{FG} = \drawUnitLine{EF}$ \inprop[prop:V.IX];\\
$\therefore \triangleDEF\ =
\drawLine[bottom][triangleDGF]{FD,FG,GD}$ in every respect \inprop[prop:I.IV].\\
but $\drawAngle{GFD} = \drawAngle{A}$ (const.),\\
and $\therefore \drawAngle{EFD} = \drawAngle{A}$;\\
and since also $\drawAngle{EFD} = \drawAngle{C}$,\\
$\drawAngle{E} = \drawAngle{B}$ \inprop[prop:I.XXXII];\\
and $\therefore$ \triangleABC\ and \triangleDEF\ are equiangular, with their equal angles opposite to homologous sides.
\stopCenterAlign

\qed
\stopProposition

\startProposition[title={Prop. VII. theor.},reference=prop:VI.VII]
\defineNewPicture[1/4]{
pair A, B, C, D, E, F, G, d;
A := (0, 0);
B := (3u, -1/2u);
C := (5/2u, 3u);
G := 1/3[A, C];
d := (0, -4u);
D := (A scaled 4/5) shifted d;
E := (B scaled 4/5) shifted d;
F := (C scaled 4/5) shifted d;
draw byAngleWithName(B, C, A, byblue, 0)(C);
draw byAngleWithName(C, A, B, byred, 0)(A);
draw byAngle(A, B, G, black, 0);
draw byAngle(G, B, C, black, 1);
draw byAngle(C, G, B, byyellow, 0);
draw byAngle(B, G, A, byred, 0);
draw byLine(B, G, byblue, 0, 0);
byLineDefine(A, B, byyellow, 0, 0);
byLineDefine(B, C, byred, 0, 0);
byLineDefine(C, A, black, 0, 0);
draw byNamedLineSeq(0)(CA,BC,AB);
draw byAngleWithName(D, E, F, byblue, 1)(E);
draw byAngleWithName(E, F, D, byyellow, 1)(F);
draw byAngleWithName(F, D, E, byred, 1)(D);
draw byAngleWithName(D, E, F, byblue, 1)(E);
byLineDefine(D, E, byyellow, 1, 0);
byLineDefine(E, F, byred, 1, 0);
byLineDefine(F, D, byblue, 1, 0);
draw byNamedLineSeq(0)(FD,EF,DE);
}
\drawCurrentPictureInMargin
\problemNP{I}{f}{two triangles (\drawLine[bottom][triangleABC]{CA,BC,AB} and \drawLine[bottom][triangleDEF]{FD,EF,DE}) have one angle in each equal (\drawAngle{F} equal to \drawAngle{C}), the sides about two other angles ($\drawUnitLine{BC} : \drawUnitLine{AB} :: \drawUnitLine{EF} : \drawUnitLine{DE}$), and each of the remaining angles (\drawAngle{A} and \drawAngle{D}) either less or not less than a right angle, the triangles are equiangular, and those angles are equal about which the sides are proportional.}

\startCenterAlign
First let it be assumed that the angles \drawAngle{A} and \drawAngle{D} are each less than a right angle: then if it be supposed that \drawAngle{ABG,GBC} and \drawAngle{E} contained by the proportional sides are not equal, let \drawAngle{ABG,GBC} be greater, and make $\drawAngle{GBC} = \drawAngle{E}$.

Because $\drawAngle{C} = \drawAngle{F}$ (hyp.). and $\drawAngle{GBC} = \drawAngle{E}$ (const.)\\
$\therefore \drawAngle{CGB} = \drawAngle{D}$ \inprop[prop:I.XXXII];

$\therefore \drawUnitLine{BC} : \drawUnitLine{BG} :: \drawUnitLine{EF} : \drawUnitLine{DE}$ \inprop[prop:VI.IV],\\
but $\drawUnitLine{BC} : \drawUnitLine{AB} :: \drawUnitLine{EF} : \drawUnitLine{DE}$ (hyp.)

$\therefore \drawUnitLine{BC} : \drawUnitLine{BG} :: \drawUnitLine{BC} : \drawUnitLine{AB}$;\\
$\therefore \drawUnitLine{BG} = \drawUnitLine{AB}$ \inprop[prop:V.IX],\\
and $\therefore \drawAngle{A} = \drawAngle{BGA}$ \inprop[prop:I.V].

But \drawAngle{A} is less than a right angle (hyp.)\\
$\therefore$ \drawAngle{BGA} is less than a right angle;\\ 
and $\therefore$ \drawAngle{CGB} must be greater than a right angle \inprop[prop:I.XIII], but it has been proven $= \drawAngle{D}$ and therefore less than a right angle, which is absurd. $\therefore$ \drawAngle{ABG,GBC} and \drawAngle{E} are not unequal;

$\therefore$ they are equal, and since $\drawAngle{C} = \drawAngle{F}$ (hyp.)\\
$\therefore \drawAngle{A} = \drawAngle{D}$ \inprop[prop:I.XXXII], and therefore the triangles are equiangular.
\stopCenterAlign

But if \drawAngle{A} and \drawAngle{D} be assumed to be each not less than a right angle, it may be proved as before, that the triangles are equiangular, and have the sides about equal angles proportional. \inprop[prop:VI.IV].

\qed
\stopProposition

\startProposition[title={Prop. VIII. theor.},reference=prop:VI.VIII]
\defineNewPicture{
pair A, B, C, D;
A := (0, 0);
B := A shifted (dir(-145)*4u);
C = whatever[B, B shifted (1, 0)] = whatever[A, A shifted dir(-145 - 90)];
D := (xpart(A), ypart(B));
draw byPolygon(A,B,D)(byyellow);
draw byPolygon(A,D,C)(byred);
draw byAngleWithName(A, B, C, black, 0)(B);
draw byAngleWithName(B, C, A, byblue, 1)(C);
draw byNamedAngleDummySides(C);
draw byAngleWithName(B, D, A, byblue, 0)(D);
draw byAngle(D, A, B, byred, 0);
draw byAngle(C, A, D, byyellow, 0);
draw byLine(A, D, black, 0, 0);
}
\drawCurrentPictureInMargin
\problemNP{I}{n}{a right angled tiangle (\drawPolygon[bottom][triangleABC]{ABD,ADC}), if a perpendicular (\drawUnitLine{AD}) be drawn from the right angle to the opposite side, the triangles (\drawPolygon[bottom][triangleABD]{ABD}, \drawPolygon[bottom][triangleADC]{ADC}) on each side of it are similar to the whole triangle and to each other.}

\startCenterAlign
Because $\drawAngle{DAB,CAD} = \drawAngle{D}$ \inax[ax:XI], and \drawAngle{B} common to \triangleABC\ and \triangleABD;\\
$\drawAngle{C} = \drawAngle{DAB}$ \inprop[prop:I.XXXII];

$\therefore$ \triangleABC\ and \triangleABD\ are equiangular and consequently have their sides about equal angles proportional \inprop[prop:VI.IV], and are therefore similar \indef[def:VI.I].

In like manner it may be proved that \triangleADC\ is similar to \triangleABC; but \triangleABD\ has been shown to be similar to \triangleABC; $\therefore$ \triangleABD\ and \triangleADC\ are similar to the whole and to each other.
\stopCenterAlign

\qed
\stopProposition

\startProposition[title={Prop. IX. prob.},reference=prop:VI.IX]
\defineNewPicture{
pair A, B, C, D, E, F;
A := (0, 0);
B := (3u, 5u);
C := B xscaled 0;
F := 2/5[A, B];
D := F xscaled 0;
E := 4/3[A, C];
draw byLine(D, F, byred, 1, 0);
byLineDefine(B, C, byred, 0, 0);
byLineDefine(A, F, byyellow, 0, 0);
byLineDefine(A, D, byblue, 0, 0);
byLineDefine(D, C, byblue, 1, 0);
byLineDefine(C, E, black, 1, 0);
byLineDefine(F, B, byyellow, 1, 0);
draw byNamedLineSeq(0)(noLine,CE,DC,AD,AF,FB,BC);
}
\drawCurrentPictureInMargin
\problemNP{F}{rom}{a given straight line (\drawSizedLine{AF,FB}) to cut off any required part.}

\startCenterAlign
From either extremity of the given line draw \drawSizedLine{AD,DC} making any angle with \drawSizedLine{AF,FB};\\
and produce \drawSizedLine{AD,DC} till the whole produced line \drawSizedLine{AD,DC,CE} contains \drawSizedLine{AD} as often as \drawSizedLine{AF,FB} contains the required part.

Draw \drawSizedLine{BC},\\
and draw $\drawSizedLine{DF} \parallel \drawSizedLine{BC}$.\\

\drawSizedLine{AF} is the required part of \drawSizedLine{AF,FB}.

For since $\drawSizedLine{DF} \parallel \drawSizedLine{BC}$\\
$\drawSizedLine{AF} : \drawSizedLine{FB} :: \drawSizedLine{AD} : \drawSizedLine{DC}$ \inprop[prop:VI.II], and by composition \inprop[prop:V.XVIII];\\
$\drawSizedLine{AF,FB} : \drawSizedLine{AF} :: \drawSizedLine{AD,DC} : \drawSizedLine{AD}$;

but \drawSizedLine{AD,DC} contains \drawSizedLine{AD} as often as \drawSizedLine{AF,FB} contains the required part (const.);

$\therefore$ \drawSizedLine{AF} is the required part.
\stopCenterAlign

\qed
\stopProposition

\startProposition[title={Prop. X. prob.},reference=prop:VI.X]
\defineNewPicture{
pair Ab, Db, Eb, Cb, A, B, C, D, E, F, G, L, d;
numeric a[];
Ab := (0, 0);
Db := (3/2u, 0);
Eb := (5/2u, 0);
Cb := (7/2u, 0);
draw byLine(Ab, Db, byblue, 0, 1);
draw byLine(Db, Eb, byred, 0, 1);
draw byLine(Eb, Cb, byyellow, 0, 1);
d := (0, -7/2u);
a1 := 40;
a2 := -70;
A := (Ab rotated a1) shifted d;
D := (Db rotated a1) shifted d;
E := (Eb rotated a1) shifted d;
C := (Cb rotated a1) shifted d;
L := 6/5[A, C];
F = whatever[A, A shifted dir(0)] = whatever[D, D shifted dir(a2)];
G = whatever[A, A shifted dir(0)] = whatever[E, E shifted dir(a2)];
B = whatever[A, A shifted dir(0)] = whatever[C, C shifted dir(a2)];
draw byLine(D, F, black, 0, 0);
draw byLine(E, G, black, 1, 0);
byLineDefine(C, B, black, 0, 1);
byLineDefine(A, D, byblue, 1, 0);
byLineDefine(D, E, byred, 1, 0);
byLineDefine(E, C, byyellow, 1, 0);
byLineDefine(C, L, black, 0, 0);
byLineDefine(A, F, byblue, 0, 0);
byLineDefine(F, G, byred, 0, 0);
byLineDefine(G, B, byyellow, 0, 0);
draw byNamedLineSeq(0)(noLine,CL,EC,DE,AD,AF,FG,GB,CB);
}
\drawCurrentPictureInMargin
\problemNP{T}{o}{divide a given straight line (\drawSizedLine{AF,FG,GB}) similarly to a given divided line (\drawSizedLine{AbDb,DbEb,EbCb}).}

\startCenterAlign
From either extremity of the given line \drawSizedLine{AF,FG,GB} draw \drawSizedLine{AD,DE,EC} making any angle;

take \drawSizedLine{AD}, \drawSizedLine{DE} and \drawSizedLine{EC} equal to \drawSizedLine{AbDb}, \drawSizedLine{DbEb} and \drawSizedLine{EbCb} respectively \inprop[prop:I.II];

draw \drawSizedLine{CB}, and draw \drawSizedLine{EG} and \drawSizedLine{DF} $\parallel$ to it.

Since
$\left\{\vcenter{
\nointerlineskip\hbox{\drawSizedLine{CB}}
\nointerlineskip\hbox{\drawSizedLine{EG}}
\nointerlineskip\hbox{\drawSizedLine{DF}}}\right\}$
are $\parallel$,\\
$\drawSizedLine{GB} : \drawSizedLine{FG} :: \drawSizedLine{EC} : \drawSizedLine{DE}$ \inprop[prop:VI.II],\\
or $\drawSizedLine{GB} : \drawSizedLine{FG} :: \drawSizedLine{EbCb} : \drawSizedLine{DbEb}$ (const.),\\
and $\drawSizedLine{FG} : \drawSizedLine{AF} :: \drawSizedLine{DE} : \drawSizedLine{AD}$ \inprop[prop:VI.II],\\
$\drawSizedLine{FG} : \drawSizedLine{AF} :: \drawSizedLine{DbEb} : \drawSizedLine{AbDb}$ (const.),

and $\therefore$ the given line \drawSizedLine{AF,FG,GB} is divided similarly to \drawSizedLine{AbDb,DbEb,EbCb}.
\stopCenterAlign

\qed
\stopProposition

\startProposition[title={Prop. XI. prob.},reference=prop:VI.XI]
\defineNewPicture{
pair Ab, Cb, A, B, C, D, E, d;
numeric a[], l[];
a1 := 85;
a2 := 60;
l1 := 2u;
l2 := 5/2u;
A := (0, 0);
B := A shifted (dir(a1)*l1);
C := A shifted (dir(a2)*l2);
D := B shifted (dir(a1)*l2);
E = whatever[A, C] = whatever[D, D shifted (B-C)];
d := (2u, 0);
Ab := A shifted d;
Cb := Ab shifted (dir(a1)*l2);
draw byLine(B, C, byyellow, 0, 0);
byLineDefine(D, E, byyellow, 1, 0);
byLineDefine(A, B, black, 0, 0);
byLineDefine(B, D, byblue, 1, 0);
byLineDefine(A, C, byred, 1, 0);
byLineDefine(C, E, byred, 0, 0);
draw byNamedLineSeq(0)(DE,CE,AC,AB,BD);
draw byLine(Ab, Cb, byblue, 0, 0);
}
\drawCurrentPictureInMargin
\problemNP{T}{o}{find a third proportional to two given straight lines (\drawSizedLine{AbCb} and \drawSizedLine{AB}).}

\startCenterAlign
At either extremity of the given line \drawSizedLine{AB} draw \drawSizedLine{AC,CE} making an angle;\\
take $\drawSizedLine{AC} = \drawSizedLine{AbCb}$, and draw \drawSizedLine{BC};\\
make $\drawSizedLine{BD} = \drawSizedLine{AbCb}$,\\
and draw $\drawSizedLine{DE} \parallel \drawSizedLine{BC}$; \inprop[prop:I.XXXI]\\
\drawSizedLine{CE} is the third proportional to \drawSizedLine{AB} and \drawSizedLine{AbCb}.

For since $\drawSizedLine{BC} \parallel \drawSizedLine{DE}$,\\
$\therefore \drawSizedLine{AB} : \drawSizedLine{BD} :: \drawSizedLine{AC} : \drawSizedLine{CE}$ \inprop[prop:VI.II];\\
but $\drawSizedLine{BD} = \drawSizedLine{AC} = \drawSizedLine{AbCb}$ (const.);\\
$\therefore \drawSizedLine{AB} : \drawSizedLine{AbCb} :: \drawSizedLine{AbCb} : \drawSizedLine{CE}$.
\stopCenterAlign

\qed
\stopProposition

\startProposition[title={Prop. XII. prob.},reference=prop:VI.XII]
\defineNewPicture[1/2]{
pair A, Ae, B, Be, C, Ce, D, E, F, G, H;
numeric l[], a;
l1 := 4/3u;
l2 := 11/6u;
l3 := 3/2u;
A := (0, 0); Ae := (l1, 0);
B := (0, 0); Be := (l2, 0);
C := (0, 0); Ce := (l3, 0);
a := 45;
byLineDefine(A, Ae) (byblue, 1, 0);
byLineDefine(B, Be) (byred, 1, 0);
byLineDefine(C, Ce) (byyellow, 1, 0);
D := (0, 0);
G := D shifted (dir(a)*l1);
E := G shifted (dir(a)*l2);
H := D shifted (l3, 0);
F = whatever[D, H] = whatever[E, E shifted (G-H)];
draw byLine(G, H, black, 0, 1);
byLineDefine(E, F, black, 1, 0);
byLineDefine(D, H, byyellow, 0, 0);
byLineDefine(H, F, black, 0, 0);
byLineDefine(D, G, byblue, 0, 0);
byLineDefine(G, E, byred, 0, 0);
draw byNamedLineSeq(0)(EF,HF,DH,DG,GE);
}
\drawCurrentPictureInMargin
\problemNP{T}{o}{find a fourth proportional to three given lines
$\left\{\vcenter{
\nointerlineskip\hbox{\drawSizedLine{AAe}}
\nointerlineskip\hbox{\drawSizedLine{BBe}}
\nointerlineskip\hbox{\drawSizedLine{CCe}}}\right\}$.}

\startCenterAlign
Draw \drawSizedLine{DG,GE} and \drawSizedLine{DH,HF} making any angle;\\
take $\drawUnitLine{DG} = \drawUnitLine{AAe}$,\\
take $\drawUnitLine{GE} = \drawUnitLine{BBe}$,\\
take $\drawUnitLine{DH} = \drawUnitLine{CCe}$,\\
draw \drawUnitLine{GH},\\
and $\drawUnitLine{EF} \parallel \drawUnitLine{GH}$ \inprop[prop:I.XXXI];\\
\drawSizedLine{HF} is the fourth proportional.

On account of the parallels,\\
$\drawUnitLine{DG} : \drawUnitLine{GE} :: \drawUnitLine{DH} : \drawUnitLine{HF}$ \inprop[prop:VI.II];\\
but
$\left\{\vcenter{
\nointerlineskip\hbox{\drawSizedLine{AAe}}
\nointerlineskip\hbox{\drawSizedLine{BBe}}
\nointerlineskip\hbox{\drawSizedLine{CCe}}}\right\}
=
\left\{\vcenter{
\nointerlineskip\hbox{\drawSizedLine{DG}}
\nointerlineskip\hbox{\drawSizedLine{GE}}
\nointerlineskip\hbox{\drawSizedLine{DH}}}\right\}$
(const.);

$\therefore \drawUnitLine{AAe} : \drawUnitLine{BBe} :: \drawUnitLine{CCe} : \drawUnitLine{HF}$ \inprop[prop:V.VII].
\stopCenterAlign

\qed
\stopProposition

\startProposition[title={Prop. XIII. prob.},reference=prop:VI.XIII]
\defineNewPicture{
pair Ab, Bb, Cb, A, B, C, D, O;
numeric l[], r;
path q;
l1 := 3u;
l2 := 2u;
r := 1/2*(l1 + l2);
Ab := (0, 0);
Bb := (l1, 0);
Cb := (l1 + l2, 0);
byLineDefine(Ab, Bb, byblue, 1, 0);
byLineDefine(Bb, Cb, byred, 1, 0);
A := (0, 0);
B := (l1, 0);
C := (l1 + l2, 0);
O := 1/2[A, C];
q := (fullcircle scaled 2r) shifted O;
D := q intersectionpoint (B--(B shifted (0, r)));
draw byAngleWithName(A, D, C, byblue, 0)(D);
byLineDefine(A, C, black, 0, 1);
byLineStylize(A, C, 0, 0, -1)(AC);
draw byMarkLine(1/2, byblue)(AC);
draw byLine(B, D, black, 0, 0);
byLineDefine(A, B, byblue, 0, 0);
byLineDefine(B, C, byred, 0, 0);
byLineDefine(A, D, byyellow, 0, 0);
byLineDefine(C, D, byyellow, 1, 0);
draw byNamedLineSeq(-1)(AB,BC,CD,AD);
draw byArc(O, C, A, r, byred, 0, 0, 0, 0)(O);
}
\drawCurrentPictureInMargin
\problemNP{T}{o}{find a mean proportional between two given straight lines $\left\{\vcenter{
\nointerlineskip\hbox{\drawSizedLine{AbBb}}
\nointerlineskip\hbox{\drawSizedLine{BbCb}}}\right\}$.}

\startCenterAlign
Draw any straight line \drawSizedLine{AB,BC},\\
make $\drawSizedLine{AB} = \drawSizedLine{AbBb}$ and $\drawSizedLine{BC} = \drawSizedLine{BbCb}$;\\
bisect \drawSizedLine{AB,BC}; and from the point of bisection as a centre, and half line as a radius, describe a semicircle
\drawFromCurrentPicture[bottom]{
draw byNamedLineFull(A, C, 0, 0, -1)(AC);
draw byNamedArc(O);
},\\
draw $\drawSizedLine{BD} \perp \drawSizedLine{AB}$:\\
\drawSizedLine{BD} is the mean proportional required.

Draw \drawUnitLine{AD} and \drawUnitLine{CD}.

Since \drawAngle{D} is a right angle \inprop[prop:III.XXXI],\\
and \drawUnitLine{BD} is $\perp$ from it upon the opposite side,\\
$\therefore$ \drawUnitLine{BD} is a mean proportional between \drawUnitLine{AB} and \drawUnitLine{BC} \inprop[prop:VI.VIII],\\
and $\therefore$ between \drawUnitLine{AbBb} and \drawUnitLine{BbCb} (const.).
\stopCenterAlign

\qed
\stopProposition

\startProposition[title={Prop. XIV. theor.},reference=prop:VI.XIV]
\defineNewPicture[1/4]{
pair A, B, C, D, E, F, G, H, d[];
numeric a;
d1 := (3/2u, 0);
d2 := (1/2u, -2u);
d3 := 2/3d1;
d4 := 3/2d2;
C := (0, 0);
E := C shifted d1;
G := C shifted d2;
B := C shifted (d1 + d2);
F := B shifted d3;
D := B shifted d4;
A := B shifted (d3 + d4);
H = whatever[A, F] = whatever[C, E];
draw byPolygon(C,E,B,G)(byyellow);
draw byPolygon(B,F,A,D)(byblue);
draw byPolygon(E,H,F,B)(byred);
draw byLine(B, G, byred, 0, 0);
draw byLine(B, F, black, 0, 0);
draw byLine(B, E, byblue, 0, 0);
draw byLine(B, D, byyellow, 0, 0);
}
\drawCurrentPictureInMargin
\problemNP{E}{qual}{parallelograms \drawPolygon{BFAD} and \drawPolygon{CEBG}, which have one angle in each equal, have the sides about equal angles reciprocally proportional\\
($\drawUnitLine{BG} : \drawUnitLine{BF} :: \drawUnitLine{BD} : \drawUnitLine{BE}$)\\
And parallelograms which have one angle in each equal, and the sides about the reciprocally proportional, are equal.}

\startCenterAlign
Let \drawUnitLine{BG} and \drawUnitLine{BF}; and \drawUnitLine{BD} and \drawUnitLine{BE}, be so placed that \drawUnitLine{BG,BF} and \drawUnitLine{BD,BE} may be continued right lines. It is evident that they may assume this position. (\inpropL[prop:I.XIII], \inpropL[prop:I.XIV], \inpropL[prop:I.XV])

Complete \drawPolygon{EHFB}.

Since $\drawPolygon{CEBG} = \drawPolygon{BFAD}$;

$\therefore \drawPolygon{CEBG} : \drawPolygon{EHFB} :: \drawPolygon{BFAD} : \drawPolygon{EHFB}$ \inprop[prop:V.VII]\\
$\therefore \drawUnitLine{BG} : \drawUnitLine{BF} :: \drawUnitLine{BD} : \drawUnitLine{BE}$ \inprop[prop:VI.I]

The same construction remaining:\\
$\drawUnitLine{BG} : \drawUnitLine{BF} ::
\left\{\eqalign{
\drawPolygon{CEBG} &: \drawPolygon{EHFB} \mbox{\inprop[prop:VI.I]}\cr
\drawUnitLine{BD} &: \drawUnitLine{BE} \mbox{(hyp.)}\cr
\drawPolygon{BFAD} &: \drawPolygon{EHFB} \mbox{\inprop[prop:VI.I]}\cr
}\right.$\\
$\therefore \drawPolygon{CEBG} : \drawPolygon{EHFB} :: \drawPolygon{BFAD} : \drawPolygon{EHFB}$ \inprop[prop:V.XI]\\
and $\therefore \drawPolygon{CEBG} = \drawPolygon{BFAD}$ \inprop[prop:V.IX].
\stopCenterAlign

\qed
\stopProposition

\startProposition[title={Prop. XV. theor.},reference=prop:VI.XV]
\defineNewPicture{
pair A, B, C, D, E;
D := (0, 0);
B := (3u, 0);
A := (5/4u, -3/2u);
C := A shifted 3/2(A-D);
E := A shifted 3/2(A-B);
draw byPolygon(D,A,B)(byblue);
draw byPolygon(D,A,E)(byred);
draw byPolygon(A,B,C)(byyellow);
draw byAngle(D, A, E, byblue, 0);
draw byAngle(B, A, C, byred, 0);
byLineDefine(B, D, black, 1, 0);
byLineDefine(D, A, byyellow, 0, 0);
byLineDefine(B, A, black, 0, 0);
byLineDefine(A, C, byred, 0, 0);
byLineDefine(A, E, byblue, 0, 0);
draw byNamedLineSeq(-1)(noLine,AE,BA,BD,DA,AC);
}
\drawCurrentPictureInMargin
\problemNP{E}{qual}{triangles, which have one angle in each equal ($\drawAngle{DAE} = \drawAngle{BAC}$), have the sides about equal angles reciprocally proportional\\
($\drawUnitLine{AE} : \drawUnitLine{BA} :: \drawUnitLine{AC} : \drawUnitLine{DA}$)\\
And two triangles which have an angle of the one equal to an angle of the other, and the sides about the equal angles reciprocally proportional, are equal.}

\startsubproposition[title={I.}]
Let the triangles be so placed that the equal angles \drawAngle{DAE} and \drawAngle{BAC} may be vertically opposite, that is to say, so that \drawUnitLine{AE} and \drawUnitLine{BA} may be in the same straight line. Whence also \drawUnitLine{AC} and \drawUnitLine{DA} must be in the same straight line \inprop[prop:I.XIV].

\startCenterAlign
Draw \drawUnitLine{BD}, then\\
$\eqalign{
\drawUnitLine{DA} : \drawUnitLine{BA}
&:: \drawPolygon{DAE} : \drawPolygon{DAB} \mbox{\inprop[prop:VI.I]}\cr
&:: \drawPolygon{ABC} : \drawPolygon{DAB} \mbox{\inprop[prop:V.VII]}\cr
&:: \drawUnitLine{AC} : \drawUnitLine{DA} \mbox{\inprop[prop:VI.I]}\cr
}$

$\therefore \drawUnitLine{AE} : \drawUnitLine{BA} :: \drawUnitLine{AC} : \drawUnitLine{DA}$ \inprop[prop:V.XI].
\stopCenterAlign
\stopsubproposition

\vfill\pagebreak

\startsubproposition[title={II.}]
\startCenterAlign
Let the same construction remain, and\\
$\drawPolygon{DAE} : \drawPolygon{DAB} :: \drawUnitLine{DA} : \drawUnitLine{BA}$ \inprop[prop:VI.I]\\
and $\drawUnitLine{AC} : \drawUnitLine{DA} :: \drawPolygon{ABC} : \drawPolygon{DAB}$ \inprop[prop:VI.I]

but $ \drawUnitLine{AE} : \drawUnitLine{BA} :: \drawUnitLine{AC} : \drawUnitLine{DA}$, (hyp.)

$\therefore \drawPolygon{DAE} : \drawPolygon{DAB} :: \drawPolygon{ABC} : \drawPolygon{DAB}$ \inprop[prop:V.XI];

$\therefore \drawPolygon{DAE} = \drawPolygon{ABC}$ \inprop[prop:V.IX].
\stopCenterAlign
\stopsubproposition

\qed
\stopProposition

\startProposition[title={Prop. XVI. theor.},reference=prop:VI.XVI]
\defineNewPicture{
pair A, B, C, D, Eb, Fb, Ee, Fe, E, F, G, H, d[];
numeric l[], h[];
l1 := -5/2u;
l2 := -2u;
h1 := -3/2l2;
h2 := -3/2l1;
A := (0, 0);
B := (l1, 0);
G := (0, h1);
F := (l1, h1);
d1 := (0, -h2 - 1/4u);
C := (0, 0) shifted d1;
D := (l2, 0) shifted d1;
H := (0, h2) shifted d1;
E := (l2, h2) shifted d1;
d2 := (0, h1 + 1/2u);
d3 := (0, h1 + 1/4u);
Eb := (0, 0) shifted d2; Ee := (-h2, 0) shifted d2;
Fb := (0, 0) shifted d3; Fe := (-h1, 0) shifted d3;
draw byPolygon(A,B,F,G)(byred);
draw byPolygon(C,D,E,H)(byyellow);
byLineDefine(A, B, byyellow, 0, 0);
byLineDefine(A, G, black, 0, 0);
draw byNamedLineSeq(0)(AB,AG);
byLineDefine(C, D, byblue, 0, 0);
byLineDefine(C, H, byred, 0, 0);
draw byNamedLineSeq(0)(CD,CH);
draw byLine(Eb, Ee, byred, 1, 0);
draw byLine(Fb, Fe, black, 1, 0);
}
\drawCurrentPictureInMargin
\problemNP{I}{f}{four straight lines be proportional ($\drawUnitLine{AB} : \drawUnitLine{CD} :: \drawUnitLine{EbEe} : \drawUnitLine{FbFe}$) the rectangle ($\drawUnitLine{AB} \times \drawUnitLine{FbFe}$) contained by the extremes, is equal to the rectangle ($\drawUnitLine{CD} \times \drawUnitLine{EbEe}$) contained by the means\\
And if the rectangle contained by the extremes be equal to the rectangle contained by the means, the four straight lines are proportional.}

\startsubproposition[title={Part I.}]
\startCenterAlign
From the extremities of \drawUnitLine{AB} and \drawUnitLine{CD} draw \drawUnitLine{AG} and \drawUnitLine{CH} $\perp$ to them and $= \drawUnitLine{FbFe}$ and \drawUnitLine{EbEe} respectively:\\
complete the parallelograms \drawPolygon{ABFG} and \drawPolygon{CDEH}.

and since,\\
$\drawUnitLine{AB} : \drawUnitLine{CD} :: \drawUnitLine{EbEe} : \drawUnitLine{FbFe}$ (hyp.)\\
$\therefore \drawUnitLine{AB} : \drawUnitLine{CD} :: \drawUnitLine{CH} : \drawUnitLine{AG}$ (const.)\\
$\therefore \drawPolygon{ABFG} = \drawPolygon{CDEH}$ \inprop[prop:VI.XIV],\\
that is, the rectangle contained be the extremes, equal to the rectangle contained be the means.
\stopCenterAlign
\stopsubproposition

\vfill\pagebreak

\startsubproposition[title={Part II.}]
\startCenterAlign
Let the same construction remain;\\
because $\drawUnitLine{FbFe} = \drawUnitLine{AG}$, $\drawPolygon{ABFG} = \drawPolygon{CDEH}$\\
and $\drawUnitLine{CH} = \drawUnitLine{EbEe}$.\\
$\therefore \drawUnitLine{AB} : \drawUnitLine{CD} :: \drawUnitLine{CH} : \drawUnitLine{AG}$ \inprop[prop:VI.XIV]

But $\drawUnitLine{CH} = \drawUnitLine{EbEe}$,\\
and $\drawUnitLine{AG} = \drawUnitLine{FbFe}$ (const.)\\
$\therefore \drawUnitLine{AB} : \drawUnitLine{CD} :: \drawUnitLine{EbEe} : \drawUnitLine{FbFe}$ \inprop[prop:V.VII].
\stopCenterAlign
\stopsubproposition

\qed
\stopProposition

\startProposition[title={Prop. XVII. theor.},reference=prop:VI.XVII]
\defineNewPicture{
pair Ab, Ae, Bb, Be, Cb, Ce, Db, De;
pair A, B, C, D, E, F, G, H;
pair d[];
numeric l[], r;
r := 3/4;
l1 := 3u;
l2 := r*l1;
l3 := r*l2;
d1 := (0, 4/4u); d2 := (0, 3/4u); d3 := (0, 2/4u); d4 := (0, 1/4u);
Ab := (0, 0) shifted d1; Ae := (-l1, 0) shifted d1;
Bb := (0, 0) shifted d2; Be := (-l2, 0) shifted d2;
Cb := (0, 0) shifted d3; Ce := (-l3, 0) shifted d3;
Db := (0, 0) shifted d4; De := (-l2, 0) shifted d4;
d5 := (0, -l2);
A := (0, 0) shifted d5; B := (-l2, 0) shifted d5; C := (-l2, l2) shifted d5; D := (0, l2) shifted d5;
d6 := (0, -l2-l1-1/4u);
E := (0, 0) shifted d6; F := (-l3, 0) shifted d6; G := (-l3, l1) shifted d6; H := (0, l1) shifted d6;
draw byPolygon(A,B,C,D)(byred);
byLineDefine(A, B, byyellow, 0, 0);
byLineDefine(A, D, byblue, 0, 0);
draw byNamedLineSeq(0)(AB,AD);
draw byPolygon(E,F,G,H)(byyellow);
byLineDefine(E, F, black, 0, 0);
byLineDefine(E, H, byred, 0, 0);
draw byNamedLineSeq(0)(EF,EH);
draw byLineWithName(Ab, Ae, byred, 0, 0)(A);
draw byLineWithName(Bb, Be, byblue, 0, 0)(B);
draw byLineWithName(Cb, Ce, black, 0, 0)(C);
draw byLineWithName(Db, De, byyellow, 0, 0)(D);
}
\drawCurrentPictureInMargin
\problemNP{I}{f}{three striaght lines be proportional ($\drawUnitLine{A} : \drawUnitLine{B} :: \drawUnitLine{B} : \drawUnitLine{C}$) the rectangle under the extremes is equal to the square of the mean.\\
And if the rectangle under the extremes be equal to the square of the mean, the three straight lines are proportional.}

\startsubproposition[title={Part I.}]
\startCenterAlign
$\eqalign{
\mbox{ Assume } \drawUnitLine{D} &= \drawUnitLine{B}, \cr
\mbox{ and since } \drawUnitLine{A} : \drawUnitLine{B} &:: \drawUnitLine{B} : \drawUnitLine{C}, \cr
\mbox{ then } \drawUnitLine{A} : \drawUnitLine{B} &:: \drawUnitLine{D} : \drawUnitLine{C}, \cr
\therefore \drawUnitLine{A} \times \drawUnitLine{C} &= \drawUnitLine{B} \times \drawUnitLine{D} \cr
}$\\
\inprop[prop:VI.XVI].

But $\drawUnitLine{D} = \drawUnitLine{B}$,\\
$\therefore \drawUnitLine{B} \times \drawUnitLine{D} = \drawUnitLine{B} \times \drawUnitLine{B} \mbox{ or } = \drawUnitLine{B}^2$;\\
therefore, if the three straight lines are proportional, the rectangle contained be the extremes is equal to the square of the mean.
\stopCenterAlign
\stopsubproposition

\startsubproposition[title={Part II.}]
\startCenterAlign
Assume $\drawUnitLine{D} = \drawUnitLine{B}$,\\
then $\drawUnitLine{A} \times \drawUnitLine{C} = \drawUnitLine{D} \times \drawUnitLine{B}$,\\
$\therefore \drawUnitLine{A} : \drawUnitLine{B} :: \drawUnitLine{D} : \drawUnitLine{C}$ \inprop[prop:VI.XVI],\\
and $\drawUnitLine{A} : \drawUnitLine{B} :: \drawUnitLine{B} : \drawUnitLine{C}$.
\stopCenterAlign
\stopsubproposition

\qed
\stopProposition

\startProposition[title={Prop. XVIII. prob.},reference=prop:VI.XVIII]
\defineNewPicture{
pair C, D, E, F, L, d;
C := (0, 0);
D := (3/2u, 0);
E := (u, 3u);
F := (-u, 2u);
L := (5/2u, u);
draw byPolygon(D,C,F)(byyellow);
draw byPolygon(D,F,E)(byblue);
draw byPolygon(D,E,L)(byred);
draw byAngleExtendedWithName(D, F, E, byred, 1)(F)(black);
draw byAngleExtendedWithName(D, E, L, byyellow, 1)(E)(byred);
draw byAngleWithName(D, C, F, byred, 1)(C);
draw byAngle(F, D, C, byblue, 1);
draw byAngle(E, D, F, black, 1);
draw byAngle(L, D, E, byyellow, 1);
draw byLine(D, F, byred, 1, 0);
byLineDefine(D, E, byyellow, 1, 0);
byLineDefine(C, D, black, 1, 0);
byLineDefine(C, F, byblue, 0, 1);
byLineDefine(F, E, byyellow, 0, 0);
draw byNamedLineSeq(0)(DE,FE,CF,CD);
pair A, B, H, G, K;
numeric s;
s := 5/6;
d := (0, -3u);
A := (C scaled s) shifted d;
B := (D scaled s) shifted d;
H := (E scaled s) shifted d;
G := (F scaled s) shifted d;
K := (L scaled s) shifted d;
draw byPolygon(B,A,G)(byyellow);
draw byPolygon(B,G,H)(byblue);
draw byPolygon(B,H,K)(byred);
draw byAngleExtendedWithName(B, G, H, black, 1)(G)(byred);
draw byAngleExtendedWithName(B, H, K, byyellow, 1)(H)(byred);
draw byAngleWithName(B, A, G, byred, 0)(A);
draw byAngle(G, B, A, byblue, 0);
draw byAngle(H, B, G, black, 0);
draw byAngle(K, B, H, byyellow, 0);
byLineDefine(B, G, byred, 0, 0);
byLineDefine(A, B, black, 0, 0);
byLineDefine(A, G, byblue, 0, 0);
byLineDefine(G, H, byblue, 1, 0);
draw byNamedLineSeq(0)(noLine,BG,AB,AG,GH);
}
\drawCurrentPictureInMargin
\problemNP{O}{n}{a given straight line (\drawUnitLine{AB}) to construct a rectilinear figure similar to a given one (\drawPolygon[middle][polygonCD]{DCF,DFE,DEL}) and similarly placed.}

\startCenterAlign
Resolve the given figure into triangles by drawing lines \drawUnitLine{DF} and \drawUnitLine{DE}.

At the extremities of \drawUnitLine{AB} make\\
$\drawAngle{GBA} = \drawAngle{FDC}$ and $\drawAngle{A} = \drawAngle{C}$;\\
again at the extremities of \drawUnitLine{BG} make\\
$\drawAngle{G} = \drawAngle{F}$ and $\drawAngle{HBG} = \drawAngle{EDF}$;\\
in like manner make\\
$\drawAngle{KBH} = \drawAngle{LDE}$ and $\drawAngle{H} = \drawAngle{E}$.

Then $\polygonCD = \drawPolygon[middle][polygonAB]{BAG,BGH,BHK}$.

It is evident from the construction and \inprop[prop:I.XXXII] that the figures are equiangular; and since the triangles \drawPolygon{DCF} and \drawPolygon{BAG} are equiangular;\\
then by \inprop[prop:VI.IV],
$\drawUnitLine{AB}:\drawUnitLine{AG} :: \drawUnitLine{CD} : \drawUnitLine{CF}$\\
and $\drawUnitLine{AG}:\drawUnitLine{BG} :: \drawUnitLine{CF} : \drawUnitLine{DF}$

Again, because \drawPolygon{DFE} and \drawPolygon{BGH} are equiangular,\\
$\drawUnitLine{BG}:\drawUnitLine{GH} :: \drawUnitLine{DF} : \drawUnitLine{FE}$\\
$\therefore$ ex \ae quali,\\
$\drawUnitLine{AG}:\drawUnitLine{GH} :: \drawUnitLine{CF} : \drawUnitLine{FE}$ \inprop[prop:V.XXII]

In like manner it may be shown that the remaining sides of the two figures are proportional.

$\therefore$ by \inprop[prop:VI.I]\\
\polygonAB\ is similar to \polygonCD\ and similarly situated; and on the given line \drawUnitLine{AB}.
\stopCenterAlign

\qed
\stopProposition

\startProposition[title={Prop. XIX. theor.},reference=prop:VI.XIX]
\defineNewPicture[1/4]{
pair A, B, C, D, E, F, G, d;
numeric s;
A := (5/2u, -3u);
B := (0, 0);
C := (-u, ypart(A));
d := (0, -4u);
s := 3/4;
D := (A scaled s) shifted d;
E := (B scaled s) shifted d;
F := (C scaled s) shifted d;
G := B shifted (unitvector(C-B) scaled ((abs(E-F)/abs(B-C))*abs(E-F)));
draw byPolygon(A,B,G)(byblue);
draw byPolygon(A,G,C)(byred);
draw byAngleWithName(A, B, C, byred, 0)(B);
draw byLine(A, G, byyellow, 1, 0);
byLineDefine(A, B, byyellow, 0, 0);
byLineDefine(B, G, black, 1, 0);
byLineDefine(G, C, black, 0, 0);
draw byNamedLineSeq(0)(AB,BG,GC);
draw byPolygon(D,E,F)(byyellow);
draw byAngleWithName(D, E, F, black, 0)(E);
byLineDefine(D, E, byred, 0, 0);
byLineDefine(E, F, byblue, 0, 0);
draw byNamedLineSeq(0)(DE,EF);
}
\drawCurrentPictureInMargin
\problemNP{S}{imilar}{triangles (\drawPolygon[bottom][triangleDEF]{DEF} and \drawPolygon[bottom][triangleABC]{ABG,AGC}) are to one another in the duplicate ratio of their homologous sides.}

Let \drawAngle{E} and \drawAngle{B} be equal angles, and \drawUnitLine{BG,GC} and \drawUnitLine{EF} homologous sides of the similar triangles \triangleDEF\ and \triangleABC\ and on \drawUnitLine{BG,GC} the greater of these lines take \drawUnitLine{BG} a third proportional, so that
\startCenterAlign
$\drawUnitLine{BG,GC} : \drawUnitLine{EF} :: \drawUnitLine{EF} : \drawUnitLine{BG}$;\\
draw \drawUnitLine{AG}.\\
$\drawUnitLine{BG,GC} : \drawUnitLine{AB} :: \drawUnitLine{EF} : \drawUnitLine{DE}$ \inprop[prop:VI.IV];\\
$\therefore \drawUnitLine{BG,GC} : \drawUnitLine{EF} :: \drawUnitLine{AG} : \drawUnitLine{DE}$ \inprop[prop:V.XVI],\\
but $\drawUnitLine{BG,GC} : \drawUnitLine{EF} :: \drawUnitLine{EF} : \drawUnitLine{BG}$ (const.),\\
$\therefore \drawUnitLine{EF} : \drawUnitLine{BG} :: \drawUnitLine{AB} : \drawUnitLine{DE}$\\
consequently $\triangleDEF = \drawPolygon[bottom][triangleABG]{ABG}$ for they have the sides about equal angles \drawAngle{E} and \drawAngle{B} reciprocally proportional \inprop[prop:V.XV];

$\therefore \triangleABC : \triangleDEF :: \triangleABC : \triangleABG$ \inprop[prop:V.VII];\\
but $\triangleABC : \triangleABG :: \drawUnitLine{BG,GC} : \drawUnitLine{BG}$ \inprop[prop:VI.I],

$\therefore \triangleABC : \triangleDEF :: \drawUnitLine{BG,GC} : \drawUnitLine{BG}$,\\
that is to say, the triangles are to one another in the duplicate ratio of their homologous sides \drawUnitLine{EF} and \drawUnitLine{BG,GC} \indef[def:V.XI].
\stopCenterAlign

\qed
\stopProposition

\startProposition[title={Prop. XX. theor.},reference=prop:VI.XX]
\defineNewPicture{
pair A, B, C, D, E;
A := (-2u, 5/2u);
B := (-1/2u, 0);
C := (3/2u, ypart(B));
D := (2u, 3/2u);
E := (u, ypart(A));
draw byPolygon(B,E,A)(byblue);
draw byPolygon(B,D,E)(byred);
draw byPolygon(B,C,D)(byyellow);
draw byAngle(B, D, E, byblue, 0);
draw byAngleWithName(B, C, D, black, 0)(C);
draw byAngle(B, D, C, byred, 0);
draw byLine(B, D, black, 1, 0);
byLineDefine(B, E, black, 0, 0);
byLineDefine(B, C, byblue, 0, 0);
byLineDefine(C, D, byblue, 1, 0);
byLineDefine(D, E, byyellow, 0, 0);
draw byNamedLineSeq(0)(BE,BC,CD,DE);
pair F, G, H, K, L, d;
numeric s;
s := 3/4;
d := (0, -3u);
F := (A scaled s) shifted d;
G := (B scaled s) shifted d;
H := (C scaled s) shifted d;
K := (D scaled s) shifted d;
L := (E scaled s) shifted d;
draw byPolygon(G,L,F)(byblue);
draw byPolygon(G,K,L)(byred);
draw byPolygon(G,H,K)(byyellow);
draw byAngle(G, K, L, byblue, 1);
draw byAngleWithName(G, H, K, black, 1)(H);
draw byAngle(G, K, H, byred, 1);
draw byLine(G, K, black, 1, 1);
byLineDefine(G, L, black, 0, 1);
byLineDefine(G, H, byred, 0, 1);
byLineDefine(H, K, byred, 1, 0);
byLineDefine(K, L, byyellow, 0, 0);
draw byNamedLineSeq(0)(GL,GH,HK,KL);
}
\drawCurrentPictureInMargin
\problemNP{S}{imilar}{polygons may be divided into the same number of similar triangles, each similar pair of which are proportional to the polygons; and the polygons are to each other in the duplicate ratio of their homologous sides.}

Draw \drawUnitLine{BE} and \drawUnitLine{BD}, and \drawUnitLine{GL} and \drawUnitLine{GK}, resolving the polygons into triangles. Then because the polygons are similar $\drawAngle{C} = \drawAngle{H}$, and $\drawUnitLine{BC} : \drawUnitLine{CD} :: \drawUnitLine{GH} : \drawUnitLine{HK}$

\startCenterAlign
$\therefore$ \drawPolygon{BCD} and \drawPolygon{GHK} are similar,\\
and $\drawAngle{BDE} = \drawAngle{GKL}$ \inprop[prop:VI.VI];

but $\drawAngle{BDE,BDC} = \drawAngle{GKL,GKH}$ because they are angles of similar polygons; therefore the remainders \drawAngle{BDE} and \drawAngle{GKL} are equal;\\
hence $\drawUnitLine{BD} : \drawUnitLine{CD} :: \drawUnitLine{GK} : \drawUnitLine{HK}$,\\
on account if the similar triangles,\\
and $\drawUnitLine{CD} : \drawUnitLine{DE} :: \drawUnitLine{HK} : \drawUnitLine{KL}$,\\
on account of the similar polygons,\\
$\therefore \drawUnitLine{BD} : \drawUnitLine{DE} :: \drawUnitLine{GK} : \drawUnitLine{KL}$,\\
ex \ae quali \inprop[prop:V.XXII], and as these proportional sides contain equal angles, the triangles \drawPolygon{BDE} and \drawPolygon{GKL} are similar \inprop[prop:VI.VI].

In like manner it may be shown that the triangles \drawPolygon{BEA} and \drawPolygon{GLF} are similar.

But \drawPolygon{BCD} is to \drawPolygon{GHK} in the duplicate ratio of \drawUnitLine{BD} to \drawUnitLine{GK} \inprop[prop:VI.XIX],\\
and \drawPolygon{BDE} is to \drawPolygon{GKL} in like manner, in the duplicate ratio of \drawUnitLine{BD} to \drawUnitLine{GK};\\
$\therefore \drawPolygon{BCD} : \drawPolygon{GHK} :: \drawPolygon{BDE} : \drawPolygon{GKL}$ \inprop[prop:V.XI];

Again \drawPolygon{BDE} is to \drawPolygon{GKL} in the duplicate ratio of \drawUnitLine{BE} to \drawUnitLine{GL}, and \drawPolygon{BEA} is to \drawPolygon{GLF} in the duplicate ratio of \drawUnitLine{BE} to \drawUnitLine{GL}.\\
$\eqalign{
\drawPolygon{BCD} : \drawPolygon{GHK} &:: \drawPolygon{BDE} : \drawPolygon{GKL} \cr
&:: \drawPolygon{BEA} : \drawPolygon{GLF} \cr
}$;\\
and as one of the antecedents is to one of the consequents, so is the sum of all the antecedents to the sum of all the consequents; that is to say, the similar triangles have to one another the same ratio as the polygons \inprop[prop:V.XII].


But \drawPolygon{BCD} is to \drawPolygon{GHK} in the duplicate ratio of \drawUnitLine{BC} to \drawUnitLine{GH};\\
$\therefore$ \drawPolygon{BEA,BDE,BCD} is to \drawPolygon{GLF,GKL,GHK} in the duplicate ratio of \drawUnitLine{BC} to \drawUnitLine{GH}.
\stopCenterAlign

\qed
\stopProposition

\startProposition[title={Prop. XXI. theor.},reference=prop:VI.XXI]
\defineNewPicture[1/2]{
pair Aa, Ab, Ac, Ba, Bb, Bc, Ca, Cb, Cc, d[];
numeric s[];
Aa := (0, 0);
Ab := (0, 3/2u);
Ac := (-3u, 0);
s1 := 5/6;
d1 := (0, -2u);
Ba := (Aa scaled s1) shifted d1;
Bb := (Ab scaled s1) shifted d1;
Bc := (Ac scaled s1) shifted d1;
s2 := 4/6;
d2 := d1 shifted (0, -2u * s1) ;
Ca := (Aa scaled s2) shifted d2;
Cb := (Ab scaled s2) shifted d2;
Cc := (Ac scaled s2) shifted d2;
draw byPolygonWithName(Aa,Ab,Ac)(byred)(A);
draw byPolygonWithName(Ba,Bb,Bc)(byblue)(B);
draw byPolygonWithName(Ca,Cb,Cc)(byyellow)(C);
}
\drawCurrentPictureInMargin
\problemNP{R}{ectilinear}{figures (\drawPolygon[bottom]{A} and \drawPolygon[bottom]{B}) which are similar to the same figure (\drawPolygon[bottom]{C}) are similar also to each other.}

Since \polygonA\ and \polygonC\ are similar, they are equiangular, and have the sides about the equal angles proportional \indef[def:VI.I]; and since figures \polygonB\ and \polygonC\ are also similar, they are equiangular, and have the sides about the equal angles proportional; therefore \polygonA\ and \polygonC\ are also equiangular, and have the sides about the equal angles proportional \inprop[prop:V.XI], and therefore similar.

\qed
\stopProposition

\startProposition[title={Prop. XXII. theor.},reference=prop:VI.XXII]
\defineNewPicture[1/2]{
pair A, B, C, D, E, F, G, H, Ob, Oe, Pb, Pe;
pair K, L, Ma, Mb, Mc, Md, Na, Nb, Nc, Nd;
pair d[];
numeric s[];
s1 := 6/5;
A := (0, 0); B := (3/2u, 0); K := (u, u);
d1 := (2u, 0);
C := (A scaled s1) shifted d1;
D := (B scaled s1) shifted d1;
L := (K scaled s1) shifted d1;
d2 := (4u, 0);
Ob := (A scaled (s1*s1)) shifted d2;
Oe := (B scaled (s1*s1)) shifted d2;
E := (0, 0); F := (u, 0); Ma := (3/2u, 2/3u); Mb := (u, 4/3u); Mc := (1/5u, 3/2u); Md := (-1/4u, 5/6u);
G := (E scaled s1) shifted d1;
H := (F scaled s1) shifted d1;
Na := (Ma scaled s1) shifted d1;
Nb := (Mb scaled s1) shifted d1;
Nc := (Mc scaled s1) shifted d1;
Nd := (Md scaled s1) shifted d1;
Pb := (E scaled (s1*s1)) shifted d2;
Pe := (F scaled (s1*s1)) shifted d2;
d3 := (1/4u, -5/2u);
forsuffixes i=E,F,Ma,Mb,Mc,Md,G,H,Na,Nb,Nc,Nd,Pb,Pe:
i := i shifted d3;
endfor;
draw byPolygonWithName(A,B,K)(byyellow)(AB);
draw byPolygonWithName(C,D,L)(byred)(CD);
draw byPolygonWithName(E,F,Ma,Mb,Mc,Md)(byblue)(EF);
draw byPolygonWithName(G,H,Na,Nb,Nc,Nd)(white)(GH);
draw byLine(A, B, black, 0, 0);
draw byLine(C, D, byblue, 0, 0);
draw byLineWithName(Ob, Oe, black, 1, 0)(O);
draw byLine(E, F, byred, 0, 0);
draw byLine(G, H, byyellow, 0, 0);
draw byLineWithName(Pb, Pe, byred, 1, 0)(P);
}
\drawCurrentPicture
\initialIndentation{8}
\problem{I}{f}{four straight lines ($\drawUnitLine{AB} : \drawUnitLine{CD} :: \drawUnitLine{EF} : \drawUnitLine{GH}$), the similar rectilinear figures similarly described on them are also proportional.\\
And if four similar rectilinear figures, similarly described on four straight lines, be proportional, the straight lines are also proportional.}

\startsubproposition[title={Part I.}]
\startCenterAlign
Take \drawUnitLine{O} a third proportional to \drawUnitLine{AB} and \drawUnitLine{CD}, and \drawUnitLine{P} a third proportional to \drawUnitLine{EF} and \drawUnitLine{GH} \inprop[prop:VI.XI];

since $\drawUnitLine{AB} : \drawUnitLine{CD} :: \drawUnitLine{EF} : \drawUnitLine{GH}$ (hyp.)\\
$\drawUnitLine{CD} : \drawUnitLine{O} :: \drawUnitLine{GH} : \drawUnitLine{P}$ (const.)

$\therefore$ ex \ae quali,\\
$\drawUnitLine{AB} : \drawUnitLine{O} :: \drawUnitLine{EF} : \drawUnitLine{P}$;

but $\drawPolygon{AB} : \drawPolygon{CD} :: \drawUnitLine{AB} : \drawUnitLine{O}$ \inprop[prop:VI.XX],\\
and $\drawPolygon{EF} : \drawPolygon{GH} :: \drawUnitLine{EF} : \drawUnitLine{P}$;

$\therefore \drawPolygon{AB} : \drawPolygon{CD} :: \drawPolygon{EF} : \drawPolygon{GH}$ \inprop[prop:V.XI].
\stopCenterAlign
\stopsubproposition

\vfill\pagebreak

\startsubproposition[title={Part II.}]
\startCenterAlign
Let the same construction remain;

$\drawPolygon{AB} : \drawPolygon{CD} :: \drawPolygon{EF} : \drawPolygon{GH}$ (hyp.),

$\therefore \drawUnitLine{AB} : \drawUnitLine{O} :: \drawUnitLine{EF} : \drawUnitLine{P}$ (hyp.),

and $\therefore \drawUnitLine{AB} : \drawUnitLine{CD} :: \drawUnitLine{EF} : \drawUnitLine{GH}$ \inprop[prop:V.XI].
\stopCenterAlign
\stopsubproposition

\qed
\stopProposition

\startProposition[title={Prop. XXIII. theor.},reference=prop:VI.XXIII]
\defineNewPicture[1/2]{
pair A, B, C, D, E, F, G ,H, d[];
d1 := (u, 0);
d2 := d1 scaled -2;
d3 := (1/2u, 2u);
d4 := d3 scaled -2/3;
C := (0, 0);
G := C shifted d1;
B := C shifted d2;
E := C shifted d3;
D := C shifted d4;
H := C shifted d1 shifted d4;
A := C shifted d2 shifted d4;
F := C shifted d1 shifted d3;
draw byPolygon(A,B,C,D)(byyellow);
draw byPolygon(E,F,G,C)(byblue);
draw byPolygon(C,D,H,G)(byred);
draw byAngle(B, C, D, byred, 0);
draw byAngle(D, C, G, byyellow, 0);
draw byAngle(G, C, E, black, 0);
draw byLine(D, C, black, 0, 0);
draw byLine(C, E, byred, 0, 0);
draw byLine(B, C, byblue, 0, 0);
draw byLine(C, G, byyellow, 0, 0);
}
\drawCurrentPictureInMargin
\problemNP[2]{E}{quiangular}{parallelograms (\drawPolygon[bottom]{ABCD} and \drawPolygon[bottom]{EFGC}) are to one another in a ratio compounded of the ratios of their sides.}

Let two of the sides \drawUnitLine{BC} and \drawUnitLine{CG} about the equal angles be placed so that they may form on straight line.

\startCenterAlign
Since $\drawAngle{BCD} + \drawAngle{DCG} = \drawTwoRightAngles$,\\
and $\drawAngle{GCE} = \drawAngle{BCD}$ (hyp.),\\
$\drawAngle{GCE} + \drawAngle{DCG} = \drawTwoRightAngles$,\\
and $\therefore$ \drawUnitLine{CE} and \drawUnitLine{DC} form one straight line \inprop[prop:I.XIV];

complete \drawPolygon[bottom]{CDHG}.

Since $\polygonABCD\ : \polygonCDHG\ :: \drawUnitLine{BC} : \drawUnitLine{CG}$ \inprop[prop:VI.I],\\
and $\polygonCDHG\ : \polygonEFGC\ :: \drawUnitLine{DC} : \drawUnitLine{CE}$ \inprop[prop:VI.I],\\
\polygonABCD\ has to \polygonEFGC\ a ratio compounded of the ratios of \drawUnitLine{BC} to \drawUnitLine{CG}, and of \drawUnitLine{DC} to \drawUnitLine{CE}.
\stopCenterAlign

\qed
\stopProposition

\startProposition[title={Prop. XXIV. theor.},reference=prop:VI.XXIV]
\defineNewPicture{
pair A, B, C, D, E, F, G, H, K, d[];
numeric s;
d1 := (3u, 0);
d2 := (u, 3u);
s := 3/5;
A := (0, 0);
B := A shifted d1;
C := A shifted d1 shifted d2;
D := A shifted d2;
G := s[A, D];
E := s[A, B];
H := s[B, C];
K := s[D, C];
F = whatever[G, H] = whatever[E, K];
draw byLine(F, H, black, 0, 1);
draw byLine(E, F, black, 0, 1);
draw byPolygon(G,F,K,D)(byyellow);
draw byPolygon(A,G,F)(byblue);
draw byPolygon(F,C,K)(byred);
draw byLine(G, F, byred, 0, 0);
byLineDefine(A, E, black, 0, 1);
byLineDefine(E, B, black, 0, 1);
byLineDefine(B, H, black, 0, 1);
byLineDefine(H, C, black, 0, 1);
byLineDefine(D, K, byblue, 0, 0);
byLineDefine(K, C, byblue, 1, 0);
byLineDefine(D, G, byred, 1, 0);
byLineDefine(G, A, byyellow, 0, 0);
draw byNamedLineSeq(0)(GA,DG,DK,KC,HC,BH,EB,AE);
}
\drawCurrentPictureInMargin
\problemNP[4]{I}{n}{any parallelogram
(\drawFromCurrentPicture[bottom][parallelogramABCD]{
draw byNamedLine(AE,EB,BH,HC,FH,EF);
draw byNamedPolygon(GFKD,AGF,FCK);
})
the parallelograms
(\drawFromCurrentPicture[bottom][parallelogramFHCK]{
draw byNamedLine(FH,HC);
draw byNamedPolygon(FCK);
}
and
\drawFromCurrentPicture[bottom][parallelogramAEFG]{
draw byNamedLine(AE,EF);
draw byNamedPolygon(AGF);
})
which are about the diagonal are similar to the whole, and to each other.}

\startCenterAlign
As \parallelogramABCD\ and \parallelogramAEFG\ have a common angle they are equiangular;\\
but because $\drawUnitLine{GF} \parallel \drawUnitLine{DK,KC}$\\
\drawFromCurrentPicture[bottom]{
startGlobalRotation(180-angle(A-C));
draw byNamedPolygon(AGF);
stopGlobalRotation;
}
and
\drawFromCurrentPicture[bottom]{
startGlobalRotation(180-angle(A-C));
draw byNamedPolygon(GFKD,AGF,FCK);
stopGlobalRotation;
}
are similar \inprop[prop:VI.IV],\\
$\therefore \drawUnitLine{GA} : \drawUnitLine{GF} :: \drawUnitLine{GA,DG} : \drawUnitLine{DK,KC}$;

and the remaining opposite sides are equal to those,\\
$\therefore$ \parallelogramAEFG\ and \parallelogramABCD\ have sides about the equal angles proportional, and are therefore similar.

In the same manner it can be demonstrated that the parallelograms \parallelogramABCD\ and \parallelogramFHCK\ are similar.\\
Since, therefore, each of the parallelograms \parallelogramAEFG\ and \parallelogramFHCK\ is similar to \parallelogramABCD, they are similar to each other.
\stopCenterAlign

\qed
\stopProposition

\startProposition[title={Prop. XXV. prob.},reference=prop:VI.XXV]
\defineNewPicture[1/2]{
pair d[];
pair A, B, C;
A := (5/3u, 2u);
B := (0, 0);
C := (9/4u, 0);
draw byPolygon(A,B,C)(byred);
pair L, E;
L := (xpart(B), -1/2ypart(A));
E := (xpart(C), ypart(L));
draw byPolygon(B,C,E,L)(byblue);
draw byAngleWithName(C, B, L, byred, 0)(B);
pair Da, Db, Dc, Dd, De;
Da := dir(0) scaled 1/2u;
Db := dir(72) scaled 2/5u;
Dc := dir(144) scaled 1/2u;
Dd := dir(-144) scaled 3/5u;
De := dir(-72) scaled 1/2u;
d1 := (3u, 3/2u);
forsuffixes i=Da,Db,Dc,Dd,De:
	i := ((i rotated 36) scaled 3/2) shifted d1;
endfor;
draw byPolygonWithName(Da,Db,Dc,Dd,De)(byblue)(D);
pair F, M;
numeric a, l;
a := 	((abs(Da-Db)*distanceToLine(Dc, Da--Db))+
	(abs(Da-Dc)*distanceToLine(Dd, Da--Dc))+
	(abs(Da-Dd)*distanceToLine(De, Da--Dd)))/2;
l := a/abs(C-E);
F := C shifted (l, 0);
M := E shifted (l, 0);
draw byPolygon(C,F,M,E)(white);
draw byAngleWithName(F, C, E, byyellow, 0)(C);
draw byLine(C, E, byred, 0, 0);
draw byLine(B, C, byyellow, 0, 0);
draw byLine(C, F, black, 1, 0);
pair K, G, H;
numeric s;
s := (a/(abs(B-C)*abs(B-L)))**(1/2);
d2 := (4/5u, -3u);
K := (A scaled s) shifted d2;
G := (B scaled s) shifted d2;
H := (C scaled s) shifted d2;
draw byPolygon(K,G,H)(byyellow);
draw byLine(G, H, byblue, 0, 0);
}
\drawCurrentPictureInMargin
\problemNP[2]{T}{o}{describe a rectilinear figure, which shall be similar to a given rectilinear figure (\drawPolygon[bottom][triangleABC]{ABC}), and equal to another (\drawPolygon[middle][polygonD]{D}).}

\startCenterAlign
Upon \drawUnitLine{BC} describe $\drawPolygon[bottom][polygonBCEL]{BCEL} = \triangleABC$,\\
and upon \drawUnitLine{CE} describe
$\drawFromCurrentPicture[bottom][polygonCFME]{
startTempAngleScale(angleScale*1/2);
draw byNamedAngle(C);
draw byNamedPolygon(CFME);
stopTempAngleScale;
} = \polygonD$,\\
and having $\drawAngle{B} = \drawAngle{C}$ \inprop[prop:I.XLV],\\
and then \drawUnitLine{BC} and \drawUnitLine{CF} will lie in the same straight line (\inpropL[prop:I.XIX], \inpropL[prop:I.XIV]).

Between \drawUnitLine{BC} and \drawUnitLine{CF} find a mean proportional \drawUnitLine{GH} \inprop[prop:VI.XIII],\\
and upon \drawUnitLine{GH} describe \drawPolygon[bottom][triangleKGH]{KGH}, similar to \triangleABC, and similarly situated.

Then $\triangleKGH\ = \polygonD$.

For since \triangleABC\ and \triangleKGH\ are similar, and\\
$\drawUnitLine{BC} : \drawUnitLine{GH} :: \drawUnitLine{GH} : \drawUnitLine{CF}$ (const.),\\
$\triangleABC\ : \triangleKGH\ :: \drawUnitLine{BC} : \drawUnitLine{CF}$ \inprop[prop:VI.XX];

but $\polygonBCEL\ : \polygonCFME\ :: \drawUnitLine{BC} : \drawUnitLine{CF}$ \inprop[prop:VI.I];\\
$\therefore \triangleABC\ : \triangleKGH\ :: \polygonBCEL\ : \polygonCFME$ \inprop[prop:V.XI];\\
but $\triangleABC\ = \polygonBCEL$ (const.),\\
and $\therefore \triangleKGH\ = \polygonCFME$ \inprop[prop:V.XIV];\\
and $\polygonCFME\ = \polygonD$ (const.);\\
consequently, \triangleKGH\ which is similar to \triangleABC\ is also $= \polygonD$.
\stopCenterAlign

\qed
\stopProposition

\startProposition[title={Prop. XXVI. theor.},reference=prop:VI.XXVI]
\defineNewPicture{
pair A, B, C, D, E, F, G, H, K, d[];
d1 := (3u, 0);
d2 := (u, 3u);
A := (0, 0);
B := A shifted d1;
C := A shifted d1 shifted d2;
D := A shifted d2;
F := 3/4[A, C];
E = whatever[A, B] = whatever[F, F shifted d2];
G = whatever[A, D] = whatever[F, F shifted d1];
H := 2/3[G, F];
K = whatever[A, B] = whatever[H, H shifted d2];
draw byPolygon(F,H,K,E)(byred);
draw byPolygon(G,D,C,B,E,F)(byblue);
draw byAngleWithName(B, A, D, byyellow, 0)(A);
draw byLine(K, H, byyellow, 0, 0);
draw byLine(G, H, byyellow, 1, 0);
draw byLine(A, C, black, 0, 1);
byLineDefine(A, G, byred, 0, 0);
byLineDefine(G, D, byred, 1, 0);
byLineDefine(A, K, byblue, 0, 0);
byLineDefine(K, E, byblue, 1, 0);
byLineDefine(E, B, black, 1, 0);
draw byNamedLineSeq(0)(GD,AG,AK,KE,EB);
draw byArbitraryFigure(A..H..C, black, 0, 0)(AHC);
byLineDefineWithName(K, A, black, 0, 1)(KAt);
byLineStylize(H, G, 1, 0, 1)(KAt);
byLineDefineWithName(A, G, black, 0, 1)(AGt);
byLineStylize(K, H, 0, 0, 1)(AGt);
byLineDefineWithName(G, H, black, 0, 1)(GHt);
byLineStylize(A, K, 0, 1, 1)(GHt);
byLineDefineWithName(E, A, black, 0, 1)(EAt);
byLineStylize(F, G, 1, 0, 1)(EAt);
}
\drawCurrentPictureInMargin
\problemNP[2]{I}{f}{similar and similarly posited parallelograms
(\drawFromCurrentPicture[bottom][parallelogramAEFG]{
draw byNamedPolygon(FHKE);
draw byNamedLine(KAt,AGt,GHt);
}
and
\drawFromCurrentPicture[bottom][parallelogramABCD]{
draw byNamedPolygon(GDCBEF);
draw byNamedLine(EAt,AC);
draw byNamedLineFull(E, F, 0, 1, 1)(AGt);
})
have a common angle, they are about the same diagonal.}

\startCenterAlign
For, if possible, let
\drawFromCurrentPicture[bottom][lineAHC]{
startGlobalRotation(180-angle(A-C));
draw byNamedArbitraryFigure(AHC);
stopGlobalRotation;
}
be the diagonal of \parallelogramABCD\ and draw $\drawUnitLine{KH} \parallel \drawUnitLine{AG}$ \inprop[prop:I.XXXI].

Since \drawLine[bottom][parallelogramAKHG]{AG,GH,KH,AK} and \parallelogramABCD\ are about the same diagonal \lineAHC, and have \drawAngle{A} common, they are similar \inprop[prop:VI.XXIV];

$\therefore \drawSizedLine{AG} : \drawSizedLine{AK} :: \drawSizedLine{AG,GD} : \drawSizedLine{AK,KE,EB}$;\\
but $\drawSizedLine{AG} : \drawSizedLine{AK,KE} :: \drawSizedLine{AG,GD} : \drawSizedLine{AK,KE,EB}$ (hyp.),\\
$\therefore \drawSizedLine{AG} : \drawSizedLine{AK} :: \drawSizedLine{AG} : \drawSizedLine{AK,KE}$,\\
and $\therefore \drawSizedLine{AK} = \drawSizedLine{AK,KE}$ \inprop[prop:V.IX], which is absurd.

$\therefore$ \lineAHC\ is not the diagonal of \parallelogramABCD\ in the same manner it can be demonstrated that no other line is except \drawUnitLine{AC}.
\stopCenterAlign

\qed
\stopProposition

\startProposition[title={Prop. XXVII. theor.},reference=prop:VI.XXVII]
\defineNewPicture[1/2]{
pair A, B, C, D, E, F, G, H, K;
numeric l, s;
l := 5u;
s := 1/5;
A := (0, 0);
B := (xpart(A), -l);
C := 1/2[A, B];
D := s[A, B];
E := (-1/2l, ypart(C));
F := (xpart(E), ypart(A));
G := (-s*l, ypart(D));
H := (xpart(G), ypart(B));
K = whatever[C, E] = whatever[G, H];
draw byPolygon(A,D,G,K,E,F)(byred);
draw byPolygon(C,D,G,K)(byblue);
draw byPolygon(B,C,K,H)(byyellow);
draw byLine(A, D, byyellow, 0, 0);
draw byLine(D, C, byred, 0, 0);
draw byLine(C, B, byblue, 0, 0);
}
\drawCurrentPictureInMargin
\problemNP{O}{f}{all the rectangles contained by the segments of a given straight line, the greatest is the square which is described on half the line.}

\startCenterAlign
Let \drawSizedLine{AD,DC,CB} be the given line,\\
\drawSizedLine{AD} and \drawSizedLine{DC,CB} unequal segments,\\
and \drawSizedLine{AD,DC} and \drawSizedLine{CB} equal segments;

then $\drawPolygon[bottom]{ADGKEF,CDGK} > \drawPolygon[bottom]{BCKH,CDGK}$.
\stopCenterAlign

For it has been demonstrated already \inprop[prop:II.V], that the square of half the line is equal to the rectangle contained by any unequal segments together with the square of the part intermediate between the middle point and the point of unequal section. The square described on half the line exceeds therefore the rectangle contained by any unequal segments of the line.

\qed
\stopProposition

\startProposition[title={Prop. XXVIII. prob.},reference=prop:VI.XXVIII]
\defineNewPicture{
pair A, B, C, D, E, F, G, H, d;
path a;
numeric r;
A := (0, 0);
B := (4u, 0);
C := 1/2[A, B];
D := C shifted (0, 3/2u);
r := 2u;
a := (fullcircle scaled (2r)) shifted D;
E := a intersectionpoint (A--C);
F := a intersectionpoint (D--2[D, C]);
byLineDefine(D, E, byyellow, 0, 0);
byLineDefine(D, C, byred, 0, 0);
byLineDefine(C, F, black, 1, 0);
draw byNamedLineSeq(0)(CF,DC,DE);
draw byLine(A, E, byred, 1, 0);
draw byLine(E, C, byblue, 0, 0);
draw byLine(C, B, byblue, 1, 0);
draw byArcBE(D, -1/2, -4 + 1/2, r, byred, 0, 0, 0, 0)(a);
d := (0, 2u);
G := A shifted d;
H := C shifted d;
draw byLine(G, H, byyellow, 1, 0);
}
\drawCurrentPictureInMargin
\problemNP{T}{o}{divide a given straight line (\drawSizedLine{AE,EC,CB}) so that the rectangle contained by its segments may be equal to a given area, not exceeding the square of half the line.}

\startCenterAlign
Let the given area be $=\drawSizedLine{GH}^2$.

Bisect \drawSizedLine{AE,EC,CB}, or make $\drawSizedLine{AE,EC} = \drawSizedLine{CB}$;\\
and if $\drawSizedLine{AE,EC}^2 = \drawSizedLine{GH}^2$,\\
problem is solved.

But if $\drawSizedLine{AE,EC}^2 \neq \drawSizedLine{GH}^2$,\\
then must $\drawSizedLine{AE,EC} > \drawSizedLine{GH}$ (hyp.).

Draw $\drawSizedLine{DC} \perp \drawSizedLine{AE,EC} = \drawSizedLine{GH}$;\\
make $\drawSizedLine{DC,CF} = \drawSizedLine{AE,EC} \mbox{ or } \drawSizedLine{CB}$;\\
with \drawSizedLine{DC,CF} as radius describe a circle cutting the given line; draw \drawSizedLine{DE}.

Then $\drawSizedLine{AE} \times \drawSizedLine{EC,CB} + \drawSizedLine{EC}^2 = \drawSizedLine{AE,EC}^2$ \inprop[prop:II.V] $= \drawSizedLine{DE}^2$.

But $\drawSizedLine{DE}^2 = \drawSizedLine{DC}^2 + \drawSizedLine{EC}^2$ \inprop[prop:I.XLVII];

$\therefore \drawSizedLine{AE} \times \drawSizedLine{EC,CB} + \drawSizedLine{EC}^2 = \drawSizedLine{DC}^2 + \drawSizedLine{EC}^2$,\\
from both take $\drawSizedLine{EC}^2$,\\
and $\drawSizedLine{AE} \times \drawSizedLine{EC,CB} = \drawSizedLine{DC}^2$.

But $\drawSizedLine{DC} = \drawSizedLine{GH}$ (const.),\\
and $\therefore$ \drawSizedLine{AE,EC,CB} is so divided that $\drawSizedLine{AE} \times \drawSizedLine{EC,CB} = \drawSizedLine{GH}^2$.
\stopCenterAlign

\qed
\stopProposition

\startProposition[title={Prop. XXIX. prob.},reference=prop:VI.XXIX]
\defineNewPicture[1/2]{
pair A, B, C, D, E, F, G, H, d;
d := (0, -1/2u);
G := (0, 0) shifted d;
H := (3/2u, 0) shifted d;
A := (0, 0);
B := (2u, ypart(A));
C := 1/2[A, B];
D := (xpart(B), abs(G-H));
E := C shifted (-abs(C-D), 0);
F := C shifted (abs(C-D), 0);
draw byLine(G, H, black, 0, 0);
draw byLineFull(C, D, byred, 0, 0)(B, D, 1, 0, 0);
draw byLine(B, D, byred, 1, 0);
draw byLineFull(E, A, byyellow, 1, 0)(E, A, 0, 0, -1);
draw byLineFull(A, C, byblue, 0, 0)(A, C, 0, 0, -1);
draw byLineFull(C, B, byblue, 1, 0)(C, B, 0, 0, -1);
draw byLineFull(B, F, byyellow, 0, 0)(B, F, 0, 0, -1);
draw byArcBE(C, 0, 4, abs(C-D), byred, 0, 0, 0, 0)(a);
}
\drawCurrentPictureInMargin
\problemNP{T}{o}{produce a given straight line (\drawSizedLine{AC,CB}), so that the rectangle contained by the segments between the extremities of the given line and the point to which it is produced, may be equal to a given area, i. e. equal to the square on \drawSizedLine{GH}.}

\startCenterAlign
Make $\drawSizedLine{AC} = \drawSizedLine{CB}$,\\
and draw $\drawSizedLine{BD} \perp \drawSizedLine{CB} = \drawSizedLine{GH}$;\\
draw \drawSizedLine{CD};\\
and with the radius \drawSizedLine{CD}, describe a circle meeting \drawSizedLine{AC,CB} produced.

Then $\drawSizedLine{AC,CB,BF} \times \drawSizedLine{BF} + \drawSizedLine{CB}^2 = \drawSizedLine{AC,CB}^2$ \inprop[prop:II.VI] $= \drawSizedLine{CD}^2$.

But $\drawSizedLine{CD}^2 = \drawSizedLine{BD}^2 + \drawSizedLine{CB}^2$ \inprop[prop:I.XLVII]

$\therefore \drawSizedLine{AC,CB,BF} \times \drawSizedLine{BF} + \drawSizedLine{CB}^2 = \drawSizedLine{BD}^2 + \drawSizedLine{CB}^2$,\\
from both take $\drawSizedLine{CB}^2$,\\
and $\therefore \drawSizedLine{AC,CB,BF} \times \drawSizedLine{BF}= \drawSizedLine{BD}^2$\\
but $\drawSizedLine{BD} = \drawSizedLine{GH}$,\\
$\therefore \drawSizedLine{BD}^2 = \mbox{ the given area.}$
\stopCenterAlign

\qed
\stopProposition

\startProposition[title={Prop. XXX. prob.},reference=prop:VI.XXX]
\defineNewPicture{
pair A, B, C, D, E, F, G, H;
numeric w;
w := 3u;
A := (0, 0);
B := (w, 0);
C := (0, w);
H := (w, w);
G := 1/2[A, C] shifted (0, -abs((1/2[A, C]) - B));
D := G shifted (abs(G-A), 0);
E = whatever[D, D shifted (0, 1)] = whatever[A, B];
F = whatever[C, H] = whatever[D, E];
draw byPolygon(A,C,F,E)(byyellow);
draw byPolygon(E,F,H,B)(byblue);
draw byLine(A, E, byred, 0, 0);
draw byLine(E, B, byred, 1, 0);
byLineDefine(C, A, byblue, 0, 0);
byLineDefine(A, G, byblue, 1, 0);
byLineDefine(G, D, byyellow, 0, 0);
byLineDefine(D, E, black, 0, 0);
draw byNamedLineSeq(0)(CA,DE,GD,AG);
byLineDefineWithName(A, G, black, 0, 1)(AGt);
byLineStylize(A, D, 1, 0, -1)(AGt);
byLineDefineWithName(G, D, black, 0, 1)(GDt);
byLineStylize(A, E, 0, 0, -1)(GDt);
byLineDefineWithName(D, E, black, 0, 1)(DEt);
byLineStylize(G, E, 0, 1, -1)(DEt);
}
\drawCurrentPictureInMargin
\problemNP{T}{o}{cut a given straight line (\drawProportionalLine{AE,EB}) in extreme and mean ratio.}

\startCenterAlign
On \drawProportionalLine{AE,EB} describe the square \drawPolygon[middle][squareABHC]{ACFE,EFHB} \inprop[prop:I.XLIV];\\
and produce \drawProportionalLine{CA}, so that $\drawProportionalLine{CA,AG} \times \drawProportionalLine{AG} = \drawProportionalLine{AE,EB}^2$ \inprop[prop:VI.XXIX];\\
take $\drawProportionalLine{AE} = \drawProportionalLine{AG}$,\\
and draw $\drawProportionalLine{DE} \parallel \drawProportionalLine{CA,AG}$,\\
meeting $\drawProportionalLine{GD} \parallel \drawProportionalLine{AE,EB}$ \inprop[prop:I.XXXI].

Then $\drawFromCurrentPicture[middle][rectangleCFDG]{
draw byNamedPolygon(ACFE);
draw byNamedLine(AGt,GDt,DEt);
}
= \drawProportionalLine{CA,AG} \times \drawProportionalLine{AG}$, and is $\therefore\ = \squareABHC$;\\
and if from both these equals be taken the common part \drawPolygon{ACFE},\\
\drawLine{DE,GD,AG,AE}, which is the square of \drawProportionalLine{AE},\\
will be $= \drawPolygon{EFHB}$, which is $= \drawProportionalLine{AE,EB} \times \drawProportionalLine{EB}$;\\
that is $\drawProportionalLine{AE}^2 = \drawProportionalLine{AE,EB} \times \drawProportionalLine{EB}$;\\
$\therefore \drawProportionalLine{AE,EB} : \drawProportionalLine{AE} :: \drawProportionalLine{AE} : \drawProportionalLine{EB}$,\\
and \drawProportionalLine{AE,EB} is divided in extreme and mean ratio \indef[def:VI.III].
\stopCenterAlign

\qed
\stopProposition

\startProposition[title={Prop. XXXI. theor.},reference=prop:VI.XXXI]
\defineNewPicture[1/2]{
pair A, B, C, D, E, F, G, H, K, L;
numeric a, r, l[];
a := -125;
A := (0, 0);
B := A shifted (dir(a)*2u);
C = whatever[A, A shifted dir(a+90)] = whatever[B, B shifted dir(0)];
D = whatever[A, A shifted dir(-90)] = whatever[B, C];
l1 := abs(A-B);
l2 := abs(B-C);
l3 := abs(C-A);
r := 1/4;
E := A shifted (dir(a-90)*l1*r);
F := B shifted (dir(a-90)*l1*r);
G := B shifted (dir(-90)*l2*r);
H := C shifted (dir(-90)*l2*r);
K := C shifted (dir(a + 180)*l3*r);
L := A shifted (dir(a + 180)*l3*r);
draw byPolygonWithName(A,B,F,E)(byblue)(AB);
draw byPolygonWithName(B,C,H,G)(byred)(BC);
draw byPolygonWithName(C,A,L,K)(byyellow)(CA);
draw byLine(A, D, black, 0, 0);
byLineDefine(A, B, byyellow, 0, 0);
byLineDefine(B, D, byblue, 1, 0);
byLineDefine(D, C, byblue, 0, 0);
byLineDefine(C, A, byred, 0, 0);
draw byNamedLineSeq(-1)(AB,BD,DC,CA);
}
\drawCurrentPictureInMargin
\problemNP{I}{f}{any similar rectilinear figures be similarly described on the sides of a right angled triangle (\drawLine[bottom]{CA,DC,BD,AB}), the figure described on the side (\drawProportionalLine{BD,DC}) subtending the right angle is equal to the sum of the figures on the other sides.}

\startCenterAlign
From the right angle draw \drawProportionalLine{AD} perpendicular to \drawProportionalLine{BD,DC};\\
then $\drawProportionalLine{BD,DC} : \drawProportionalLine{CA} :: \drawProportionalLine{CA} : \drawProportionalLine{DC}$ \inprop[prop:VI.VIII].

$\therefore
\drawFromCurrentPicture[bottom][figBC]{
startGlobalRotation(-angle(B-C));
draw byNamedPolygon(BC);
stopGlobalRotation;
} :
\drawFromCurrentPicture[bottom][figCA]{
startGlobalRotation(-angle(C-A));
draw byNamedPolygon(CA);
stopGlobalRotation;
} :: \drawProportionalLine{BD,DC} : \drawProportionalLine{DC}$ \inprop[prop:VI.XX].\\
but $\figBC\ :
\drawFromCurrentPicture[bottom][figAB]{
startGlobalRotation(-angle(A-B));
draw byNamedPolygon(AB);
stopGlobalRotation;
} :: \drawProportionalLine{BD,DC} : \drawProportionalLine{BD}$ \inprop[prop:VI.XX].

Hence $\drawProportionalLine{DC} + \drawProportionalLine{BD} : \drawProportionalLine{BD,DC} :: \figAB\ + \figCA\ : \figBC$;\\
but $\drawProportionalLine{DC} + \drawProportionalLine{BD} = \drawProportionalLine{BD,DC}$;\\
and $\therefore \figAB\ + \figCA\ = \figBC$.
\stopCenterAlign

\qed
\stopProposition

\startProposition[title={Prop. XXXII. theor.},reference=prop:VI.XXXII]
\defineNewPicture[1/2]{
pair A, B, C, D, E;
numeric s;
B := (0, 0);
C := (5/2u, 0);
A := (u, 2u);
s := 3/5;
D := (A scaled s) shifted C;
E := (C scaled s) shifted C;
draw byAngleWithName(C, A, B, byyellow, 0)(A);
draw byAngleWithName(A, B, C, byred, 0)(B);
draw byAngle(B, C, A, byblue, 0);
draw byAngleWithName(E, D, C, black, 0)(D);
draw byAngle(D, C, E, black, 1);
draw byAngle(A, C, D, byyellow, 0);
byLineDefine(A, B, byblue, 0, 0);
byLineDefine(B, C, byyellow, 0, 0);
byLineDefine(C, E, byyellow, 1, 0);
byLineDefine(E, D, byred, 1, 0);
byLineDefine(D, C, byblue, 1, 0);
byLineDefine(C, A, byred, 0, 0);
draw byNamedLineSeq(0)(ED,DC,noLine,CA,AB,BC,CE);
byLineDefineWithName(A, B, black, 0, 1)(ABt);
byLineDefineWithName(B, C, black, 0, 1)(BCt);
byLineDefineWithName(C, A, black, 0, 1)(CAt);
byLineDefineWithName(D, C, black, 0, 1)(DCt);
byLineDefineWithName(C, E, black, 0, 1)(CEt);
byLineDefineWithName(E, D, black, 0, 1)(EDt);
}
\drawCurrentPictureInMargin
\problemNP{I}{f}{two triangles
(\drawFromCurrentPicture[bottom]{
startTempAngleScale(angleScale*3/5);
draw byNamedAngle(A, B, BCA);
draw byNamedLineSeq(0)(CAt,BCt,ABt);
stopTempAngleScale;
}
and
\drawFromCurrentPicture[bottom]{
startTempAngleScale(angleScale*3/5);
draw byNamedAngle(D, DCE);
draw byNamedLineSeq(0)(EDt,CEt,DCt);
stopTempAngleScale;
}),
have two sides proportional ($\drawUnitLine{AB} : \drawUnitLine{CA} :: \drawUnitLine{DC} : \drawUnitLine{ED}$), and be so placed at an angle that the homologous sides are parallel, the remaining sides (\drawUnitLine{BC} and \drawUnitLine{CE}) form one right line.}

\startCenterAlign
Since $\drawUnitLine{AB} \parallel \drawUnitLine{DC}$,\\
$\drawAngle{A} = \drawAngle{ACD}$ \inprop[prop:I.XXIX];\\
and also since $\drawUnitLine{CA} \parallel \drawUnitLine{ED}$,\\
$\drawAngle{ACD} = \drawAngle{D}$ \inprop[prop:I.XXIX];\\
$\therefore \drawAngle{A} = \drawAngle{D}$;\\
and since $\drawUnitLine{AB} : \drawUnitLine{CA} :: \drawUnitLine{DC} : \drawUnitLine{ED}$ (hyp.),\\
the triangles are equiangular \inprop[prop:VI.VI];

$\therefore \drawAngle{B} = \drawAngle{DCE}$;\\
but $\drawAngle{A} = \drawAngle{ACD}$;

$\therefore \drawAngle{BCA} + \drawAngle{ACD} + \drawAngle{DCE} = \drawAngle{BCA} + \drawAngle{A} + \drawAngle{B} = \drawTwoRightAngles$ \inprop[prop:I.XXXII],\\
and $\therefore$ \drawUnitLine{BC} and \drawUnitLine{CE} lie in the same straight line \inprop[prop:I.XIV].
\stopCenterAlign

\qed
\stopProposition

\startProposition[title={Prop. XXXIII. theor.},reference=prop:VI.XXXIII]
\defineNewPicture[1/2]{
pair A, B, C, D, E, F, G, H, K, L, M, N;
numeric r, a[], ba, aa, q;
path c[];
q := 8/360;
r := 9/4u;
aa := 125;
ba := 205;
a1 := 30;
a2 := 35;
G := (0, 0);
A := (dir(aa)*r) shifted G;
B := (dir(ba)*r) shifted G;
C := (dir(ba + a1)*r) shifted G;
K := (dir(ba + 2a1)*r) shifted G;
L := (dir(ba + 3a1)*r) shifted G;
draw byAngleWithName(B, A, C, byyellow, 0)(A);
draw byAngleWithName(B, G, C, black, 0)(BC);
draw byAngleWithName(C, G, K, byred, 0)(CK);
draw byAngleWithName(K, G, L, byblue, 0)(KL);
draw byLine(A, B, black, 0, 1);
draw byLine(A, C, black, 0, 1);
draw byLine(G, C, black, 0, 1);
draw byLine(G, K, black, 0, 1);
byLineDefine(G, B, black, 0, 1);
byLineDefine(G, L, black, 0, 1);
draw byNamedLineSeq(0)(GB,GL);
draw byArc(G, L, B, r, byred, 0, 0, 0, 0)(LB);
draw byArc(G, B, C, r, black, 0, 0, 0, 0)(BC);
draw byArc(G, C, K, r, byred, 0, 0, 0, 0)(CK);
draw byArc(G, K, L, r, byblue, 0, 0, 0, 0)(KL);
byCircleDefineR(G, r, byred, 0, 0, 0)(G);
H := (0, -1/2u - 2r);
D := (dir(aa)*r) shifted H;
E := (dir(ba)*r) shifted H;
F := (dir(ba + a2)*r) shifted H;
M := (dir(ba + 2a2)*r) shifted H;
N := (dir(ba + 3a2)*r) shifted H;
draw byAngleWithName(E, D, F, byyellow, 1)(D);
draw byAngleWithName(E, H, F, black, 1)(EF);
draw byAngleWithName(F, H, M, byred, 1)(FM);
draw byAngleWithName(M, H, N, byblue, 1)(MN);
draw byLine(D, E, black, 0, 1);
draw byLine(D, F, black, 0, 1);
draw byLine(H, F, black, 0, 1);
draw byLine(H, M, black, 0, 1);
byLineDefine(H, E, black, 0, 1);
byLineDefine(H, N, black, 0, 1);
draw byNamedLineSeq(0)(HE,HN);
draw byArc(H, N, E, r, byblue, 0, 0, 0, 0)(NE);
draw byArc(H, E, F, r, byyellow, 1, 0, 0, 0)(EF);
draw byArc(H, F, M, r, byred, 1, 0, 0, 0)(FM);
draw byArc(H, M, N, r, byblue, 1, 0, 0, 0)(MN);
byCircleDefineR(H, r, byblue, 0, 0, 0)(H);
}
\drawCurrentPictureInMargin
\problemNP{I}{n}{equal circles (\drawCircle[middle][1/4]{G}, \drawCircle[middle][1/4]{H}), angles, whether at the centre or circumference, are in the same ratio to one another as the arcs on which they stand ($\drawAngle{BC} : \drawAngle{EF} ::
\drawFromCurrentPicture[middle][arcBC]{
startGlobalRotation(180-angle(B-C));
draw byNamedArc(BC);
stopGlobalRotation;
} : \drawFromCurrentPicture[middle][arcEF]{
startGlobalRotation(180-angle(E-F));
draw byNamedArc(EF);
stopGlobalRotation;
}$); so also are sectors.}

Take in te circumference of \circleG\ any number of arcs
\drawFromCurrentPicture[middle][arcCK]{
startGlobalRotation(180-angle(C-K));
draw byNamedArc(CK);
stopGlobalRotation;
},
\drawFromCurrentPicture[middle][arcKL]{
startGlobalRotation(180-angle(K-L));
draw byNamedArc(KL);
stopGlobalRotation;
}, \&c. each $= \arcBC$, and also in the circumference of \circleH\ take any number of arcs
\drawFromCurrentPicture[middle][arcFM]{
startGlobalRotation(180-angle(F-M));
draw byNamedArc(FM);
stopGlobalRotation;
},
\drawFromCurrentPicture[middle][arcMN]{
startGlobalRotation(180-angle(M-N));
draw byNamedArc(MN);
stopGlobalRotation;
}, \&c. each $= \arcEF$, draw the radii to the extremities of the equal arcs.

The since the arcs \arcBC, \arcCK, \arcKL, \&c. are all equal, the angles \drawAngle{BC}, \drawAngle{CK}, \drawAngle{KL}, \&c. are also equal \inprop[prop:III.XXXVII]; $\therefore$ \drawAngle{BC,CK,KL} is the same multiple of \drawAngle{BC} which are \drawFromCurrentPicture[middle][arcBL]{
startGlobalRotation(180-angle(B-L));
draw byNamedArc(BC,CK,KL);
stopGlobalRotation;
} is of \arcBC; and in the same manner \drawAngle{EF,FM,MN} is the same multiple of \drawAngle{EF}, which the arc\drawAngle{BC} which are \drawFromCurrentPicture[middle][arcEN]{
startGlobalRotation(180-angle(E-N));
draw byNamedArc(EF,FM,MN);
stopGlobalRotation;
} is of arc \arcEF.


\startCenterAlign
Then it is evident \inprop[prop:III.XXVII],\\
if \drawAngle{BC,CK,KL} (or if $m$ times \drawAngle{BC}) $>, =, < \drawAngle{EF,FM,MN}$ (or $n$ times \drawAngle{EF})\\
then \arcBL (or $m$ times \arcBC) $>, =, < \arcEN$ (or $n$ times \arcEF);
\stopCenterAlign

$\therefore \drawAngle{BC} : \drawAngle{EF} :: \arcBC\ : \arcEF$ \indef[def:V.V], or the angles at the centre are as the arcs on which they stand; but the angles at the circumference being halves of the angles at the centre \inprop[prop:III.XX] are in the same ratio \inprop[prop:V.XV], and therefore are as the arcs on which they stand.

It is evident, that sectors in equal circles, and on equal arcs are equal (\inpropL[prop:I.IV], \inpropN[prop:III.XXIV], \inpropN[prop:III.XXVII], and \indefL[def:III.X]). Hence, if the sectors be substituted for the angles in the above demonstration, the second part of the proportion will be established, that is, in equal circles the sectors have the same ratio to one another as the arcs on which they stand.

\qed
\stopProposition

\startPropositionAZ[title={Prop. A. theor.},reference=prop:VI.A]
\defineNewPicture{
pair A, B, C, D, E, F;
A := (0, 0);
B := (2u, 0);
C := (3u, 7/4u);
F := 4/3[A, C];
D = whatever[A, B] = whatever[C, C shifted 1/2[unitvector(F-C), unitvector(B-C)]];
E = whatever[A, C] = whatever[B, B shifted (C-D)];
draw byAngleWithName(C, E, B, byyellow, 0)(E);
draw byAngle(E, B, C, byblue, 1);
draw byAngle(C, B, D, byred, 0);
draw byAngleWithName(B, D, C, byyellow, 1)(D);
draw byAngle(B, C, E, black, 1);
draw byAngle(D, C, B, byblue, 0);
draw byAngle(F, C, D, black, 0);
draw byLine(B, E, byred, 0, 0);
draw byLine(B, C, byyellow, 0, 0);
byLineDefine(A, B, byblue, 0, 0);
byLineDefine(B, D, byblue, 1, 0);
byLineDefine(D, C, byred, 1, 0);
byLineDefine(A, E, black, 0, 0);
byLineDefine(E, C, black, 1, 0);
byLineDefine(C, F, byyellow, 1, 0);
draw byNamedLineSeq(0)(noLine,DC,BD,AB,AE,EC,CF);
}
\drawCurrentPictureInMargin
\problemNP[2]{I}{f}{the right line (\drawSizedLine{DC}) bisecting an external angle \drawAngle{DCB,FCD} of the triangle \drawFromCurrentPicture[bottom]{
startTempScale(1/2);
draw byNamedLineSeq(0)(AE,EC,BC,AB);
stopTempScale;
} meet the opposite side (\drawSizedLine{AB}) produced, that whole produced side (\drawSizedLine{AB,BD}), and its external segment (\drawSizedLine{BD}) will be proportional to the sides (\drawSizedLine{AE,EC} and \drawSizedLine{BC}), which contain the angle adjacent to the external bisected angle.}

\startCenterAlign
For if \drawSizedLine{BE} be drawn $\parallel \drawSizedLine{DC}$,\\
$\eqalign{
\mbox{then } \drawAngle{DCB} &= \drawAngle{EBC} \mbox{, \inprop[prop:I.XXIX];}\cr
& = \drawAngle{FCD} \mbox{, (hyp.),}\cr
& = \drawAngle{E} \mbox{, \inprop[prop:I.XXIX];}
}$\\
and $\therefore \drawSizedLine{EC} = \drawSizedLine{BC}$, \inprop[prop:I.VI],\\
and $\drawSizedLine{AE,EC} : \drawSizedLine{BC} :: \drawSizedLine{AE,EC} : \drawSizedLine{EC}$ \inprop[prop:V.VII];\\
But also, $\drawSizedLine{AB,BD} : \drawSizedLine{BD} :: \drawSizedLine{AE,EC} : \drawSizedLine{EC}$ \inprop[prop:VI.II];\\
and therefore $\drawSizedLine{AB,BD} : \drawSizedLine{BD} :: \drawSizedLine{AE,EC} : \drawSizedLine{BC}$ \inprop[prop:V.XI].
\stopCenterAlign

\qed
\stopPropositionAZ

\startPropositionAZ[title={Prop. B. theor.},reference=prop:VI.B]
\defineNewPicture{
pair A, B, C, D, E, O;
numeric r;
path c;
r := 7/4u;
O := (0, 0);
c := (fullcircle scaled 2r) shifted O;
A := (dir(75)*r) shifted O;
B := (dir(180 + 10)*r) shifted O;
C := (dir(-10)*r) shifted O;
D = whatever[B, C] = whatever[A, A shifted 1/2[unitvector(B-A), unitvector(C-A)]];
E := (subpath (0, -4) of c) intersectionpoint (A--4[A, D]);
draw byAngleWithName(A, B, C, byyellow, 0)(B);
draw byAngleWithName(A, E, C, black, 0)(E);
draw byAngle(C, A, E, byblue, 0);
draw byAngle(E, A, B, byred, 0);
draw byLine(A, B, byblue, 0, 0);
draw byLine(B, D, byred, 1, 0);
draw byLine(D, C, byred, 0, 0);
draw byLine(C, A, black, 0, 0);
draw byLine(A, D, byyellow, 0, 0);
draw byLine(D, E, byyellow, 1, 0);
draw byLine(C, E, byblue, 1, 0);
draw byCircleABC(A, B, C, byyellow, 0, 0, 1/2)(O);
}
\drawCurrentPictureInMargin
\problemNP{I}{f}{an angle of a triangle be bisected by a straight line, which likewise cuts the base; the rectangle contained by the sides of the triangle is equal to the rectangle contained by the segments of the base, together with the square of the straight line which bisects the angle.}

\startCenterAlign
Let \drawSizedLine{AD} be drawn, making $\drawAngle{CAE} = \drawAngle{EAB}$;\\
then shall $\drawSizedLine{AB} \times \drawSizedLine{CA} = \drawSizedLine{BD} \times \drawSizedLine{DC} + \drawSizedLine{AD}^2$.

About \drawLine[bottom]{CA,DC,BD,AB} describe \drawCircle[middle][1/4]{O} \inprop[prop:IV.V],\\
produce \drawSizedLine{AD} to meet the circle, and draw \drawSizedLine{CE}.

Since $\drawAngle{CAE} = \drawAngle{EAB}$ (hyp.),\\
and $\drawAngle{B} = \drawAngle{E}$ \inprop[prop:III.XXI],\\
$\therefore$ \drawLine[bottom]{AD,BD,AB} and \drawLine[middle]{CA,CE,DE,AD} are equiangular \inprop[prop:I.XXXII];\\
$\therefore \drawSizedLine{AB} : \drawSizedLine{AD} :: \drawSizedLine{AD,DE} : \drawSizedLine{CA}$ \inprop[prop:VI.IV];\\
$\therefore \drawSizedLine{AB} \times \drawSizedLine{CA} = \drawSizedLine{AD} \times \drawSizedLine{AD,DE}$ \inprop[prop:VI.XVI]\\
$= \drawSizedLine{DE} \times \drawSizedLine{AD} + \drawSizedLine{AD}^2$ \inprop[prop:II.III];\\
but $\drawSizedLine{DE} \times \drawSizedLine{AD} = \drawSizedLine{BD} \times \drawSizedLine{DC}$ \inprop[prop:III.XXXV];\\
$\therefore \drawSizedLine{AB} \times \drawSizedLine{CA} = \drawSizedLine{BD} \times \drawSizedLine{DC} + \drawSizedLine{AD}^2$
\stopCenterAlign

\qed
\stopPropositionAZ

\startPropositionAZ[title={Prop. C. theor.},reference=prop:VI.C]
\defineNewPicture{
pair A, B, C, D, E, O;
numeric r;
r := 9/4u;
O := (0, 0);
A := (dir(60)*r) shifted O;
B := (dir(180-5)*r) shifted O;
C := (dir(5)*r) shifted O;
D = whatever[B, C] = whatever[A, A shifted ((B-C) rotated 90)];
E := (dir(60 + 180)*r) shifted O;
draw byAngleWithName(A, B, C, byyellow, 1)(B);
draw byAngleWithName(A, E, C, byred, 1)(E);
draw byAngleWithName(B, D, A, byblue, 0)(D);
draw byAngle(B, C, A, byblue, 1);
draw byAngle(E, C, B, byyellow, 0);
draw byAngleWithName(C, A, B, byred, 0)(A);
draw byLine(A, E, byblue, 0, 0);
draw byLine(C, E, black, 0, 0);
draw byLine(A, D, byyellow, 1, 0);
draw byLine(A, B, byblue, 1, 0);
draw byLine(B, D, byred, 1, 0);
draw byLine(D, C, byred, 0, 0);
draw byLine(C, A, byyellow, 0, 0);
draw byCircleABC(A, B, C, byred, 0, 0, 1/2)(O);
}
\drawCurrentPictureInMargin
\problemNP{I}{f}{from any angle of a triangle a straight line be drawn perpendicular to the base; the rectangle contained by the sides of the triangle is equal to the rectangle contained by the perpendicular and the diameter of the circle described about the triangle.}

\startCenterAlign
From \drawAngle{A} of \drawLine[bottom]{CA,DC,BD,AB}\\
draw $\drawUnitLine{AD} \perp \drawUnitLine{BD,DC}$;\\
then shall $\drawUnitLine{AB} \times \drawUnitLine{CA} = \drawUnitLine{AD} \times $ the diameter of the described circle.

Describe \drawCircle[middle][1/4]{O} \inprop[prop:IV.V], draw its diameter \drawUnitLine{AE}, and draw \drawUnitLine{CE};\\
then because $\drawAngle{D} = \drawAngle{BCA,ECB}$ (const. and \inpropL[prop:III.XXXI]);\\
and $\drawAngle{B} = \drawAngle{E}$ \inprop[prop:III.XXI];\\
$\therefore$ \drawLine[bottom]{AD,BD,AB} is equiangular to \drawLine[middle]{CA,CE,AE} \inprop[prop:VI.IV];

$\therefore \drawUnitLine{AB} : \drawUnitLine{AD} :: \drawUnitLine{AE} : \drawUnitLine{CA}$;\\
and $\therefore \drawUnitLine{AB} \times \drawUnitLine{CA} = \drawUnitLine{AD} \times \drawUnitLine{AE}$ \inprop[prop:VI.XVI].
\stopCenterAlign

\qed
\stopPropositionAZ

\startPropositionAZ[title={Prop. D. theor.},reference=prop:VI.D]
\defineNewPicture[1/4]{
pair A, B, C, D, E, O;
numeric r, a[];
a1 := 110;
a2 := 190;
a3 := -50;
a4 := 15;
r := 9/4u;
O := (0, 0);
A := (dir(a1)*r) shifted O;
B := (dir(a2)*r) shifted O;
C := (dir(a3)*r) shifted O;
D := (dir(a4)*r) shifted O;
E = whatever[B, D] = whatever[A, A shifted dir((a1-180)-(a3-(a1-180)))];
draw byAngle(B, C, A, byyellow, 0);
draw byAngle(A, C, D, byred, 1);
draw byAngleWithName(B, D, A, byblue, 1)(D);
draw byAngleWithName(A, B, D, byyellow, 1)(B);
draw byAngle(E, A, B, byblue, 0);
draw byAngle(C, A, E, black, 0);
draw byAngle(D, A, C, byred, 0);
draw byLine(A, E, byyellow, 1, 0);
draw byLine(A, C, byblue, 0, 0);
draw byLine(B, E, byred, 1, 0);
draw byLine(E, D, byred, 0, 0);
draw byLine(A, B, black, 1, 0);
draw byLine(B, C, byblue, 1, 0);
draw byLine(C, D, black, 0, 0);
draw byLine(D, A, byyellow, 0, 0);
draw byCircleABC(A, B, C, byred, 0, 0, 1/2)(O);
}
\drawCurrentPictureInMargin
\problemNP{T}{he}{rectangle contained by the diagonals of a quadrilateral figure inscribed in a circle, is equal to both the rectangles contained by its opposite sides.}

\startCenterAlign
Let \drawLine{DA,CD,BC,AB} be any quadrilateral figure inscribed in \drawCircle[middle][1/5]{O};\\
and draw \drawUnitLine{BE,ED} and \drawUnitLine{AC};\\
then $\drawUnitLine{BE,ED} \times \drawUnitLine{AC} = \drawUnitLine{AB} \times \drawUnitLine{CD} + \drawUnitLine{DA} \times \drawUnitLine{BC}$.

Make $\drawAngle{EAB} = \drawAngle{DAC}$ \inprop[prop:I.XXIII],\\
$\therefore \drawAngle{EAB,CAE} = \drawAngle{CAE,DAC}$; and $\drawAngle{BCA} = \drawAngle{D}$ \inprop[prop:III.XXI];

$\therefore \drawUnitLine{DA} : \drawUnitLine{ED} :: \drawUnitLine{AC} : \drawUnitLine{BC}$ \inprop[prop:VI.IV];\\
and $\therefore \drawUnitLine{ED} \times \drawUnitLine{AC} = \drawUnitLine{DA} \times \drawUnitLine{BC}$ \inprop[prop:VI.XVI];\\
again, because $\drawAngle{EAB} = \drawAngle{DAC}$ (const.),\\
and $\drawAngle{B} = \drawAngle{ACD}$ \inprop[prop:III.XXI];\\
$\therefore \drawUnitLine{AB} : \drawUnitLine{BE} : \drawUnitLine{AC} : \drawUnitLine{CD}$ \inprop[prop:VI.IV];\\
and $\therefore \drawUnitLine{BE} \times \drawUnitLine{AC} = \drawUnitLine{AB} \times \drawUnitLine{CD}$ \inprop[prop:VI.XVI];\\
but, from above, $\drawUnitLine{ED} \times \drawUnitLine{AC} = \drawUnitLine{DA} \times \drawUnitLine{BC}$;\\
$\therefore \drawUnitLine{BE,ED} \times \drawUnitLine{AC} = \drawUnitLine{AB} \times \drawUnitLine{CD} + \drawUnitLine{DA} \times \drawUnitLine{BC}$ \inprop[prop:II.I].
\stopCenterAlign

\qed
\stopPropositionAZ
\stopbook

\stoptext
%\closeout \lettrineslist

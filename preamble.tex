\newdimen\topmargin
\topmargin=\topspace
\advance \topmargin by \headerdistance

\newcount\mpInst
\mpInst = 0
% Every time new proposition is introduced, we define new picture
% with all its colors and styles, preceded with inclusion of byrne.mp library. 
% It is done with \startMPinclusions in ConTeXt.
% Note that every time new MPinstance is defined.
% Before doing so we clear everything after previous definitions (explained below)

\def\mpPre{}

\def\defineNewPicture{\dodoubleempty\definenewpicture}
\def\definenewpicture[#1][#2]#3{\ignorespaces
	\undefineList\ignorespaces
	\def\undefineList{}\ignorespaces
	\iffirstargument\ignorespaces
		\global\def\sfA{#1}\ignorespaces
	\else\ignorespaces
		\global\def\sfA{1/3}\ignorespaces
	\fi\ignorespaces
	\ifsecondargument\ignorespaces
		\global\def\sfB{#2}\ignorespaces
	\else\ignorespaces
		\global\def\sfB{1}\ignorespaces
	\fi\ignorespaces
	\advance \mpInst by 1\ignorespaces
	\global\def\currentInstance{\the\mpInst}\ignorespaces
	\defineMPinstance[\currentInstance]\ignorespaces
	\startreusableMPgraphic{\currentInstance::currentPicture}
		input byrne;
		\mpPre
		startMainPictureMode;
		scaleFactor := \sfB;
			#3
		scaleFactor := \sfA;
		stopMainPictureMode;
	\stopreusableMPgraphic\ignorespaces\setbox0\vbox{\reuseMPgraphic{\currentInstance::currentPicture}}\global\addToUndefineList{#2}\ignorespaces}

% Every time some simple picture is drawn, new macro is defined
% with \markPictAsReady. If macro is defined, predefined picture is used.
% Every time new picture is added, instructions to undefine it are added with
% \addToUndefineList. \undefineList is thus generated command, that, once executed,
% undefines everything that was appended to it with \addToUndefineList.

\def\unmarkPictAsReady#1{\expandafter\let\csname #1\endcsname\relax}

\def\addToUndefineList#1{\expandafter\def\expandafter\undefineList\expandafter{\undefineList \unmarkPictAsReady{#1}}}

\def\undefineList{}
\addToUndefineList{lastPicture}

% When larger picture is defined, we can visualize its parts with \drawFromCurrentPicture
% First (optional) argument is for picture's vertical alignment relative to the string
% Currently only `middle' and anything other (for `top') is working
% Second (optional) argument is for picture's name for future reuse
% If picture with the same name exists, it's not defined again.
% Third argument is MP code for picture itself.
% Whenever everything is defined, \drawDefinedPicture macro is called for
\def\drawFromCurrentPicture{\dodoubleempty\drawfromcurrentpicture}
\def\drawfromcurrentpicture[#1][#2]#3{{\ignorespaces
	\iffirstargument\ignorespaces
		\global\def\dfcpl{#1}\ignorespaces
	\else\ignorespaces
		\global\def\dfcpl{middle}\ignorespaces
	\fi\ignorespaces
	\ifsecondargument\ignorespaces
		\global\def\lastPict{#2}\ignorespaces
		\expandafter\ifx\csname #2\endcsname\relax\ignorespaces
			\startreusableMPgraphic{\currentInstance::#2}
				linecap := butt;
				#3
			\stopreusableMPgraphic\ignorespaces
			\global\addToUndefineList{#2}\ignorespaces
		\fi\ignorespaces
	\else\ignorespaces
		\global\def\lastPict{lastPicture}\ignorespaces
		\startreusableMPgraphic{\currentInstance::\lastPict}
			linecap := butt;
			#3
		\stopreusableMPgraphic
	\fi\ignorespaces
\global\expandafter\edef\csname\lastPict\endcsname{{\noexpand\drawDefinedPicture{\expandafter\lastPict}{\expandafter\dfcpl}}}\csname\lastPict\endcsname}}


% \drawDefinedPicture draws defined and named picture, aligned as needed and,
% possibly, with margins modified with \offsetPicture (i. e., if we want picture to act as if it's less tall that it actually is,
% we put some value as its first argument, if we want picture to be less deep, we use second argument, and place picture
% itself as third).

\newdimen\pictOffsetTop
\newdimen\pictOffsetBottom

\pictOffsetTop=0pt
\pictOffsetBottom=0pt

\def\offsetPicture#1#2#3{{\pictOffsetTop=#1\pictOffsetBottom=#2#3\pictOffsetTop=0pt\pictOffsetBottom=0pt}}

\newdimen\midht
\newdimen\middp
\def\drawDefinedPicture#1#2{{\ignorespaces
	\setbox0\vbox{\vskip2pt\vskip-\pictOffsetTop\reuseMPgraphic{\currentInstance::#1}\vskip2pt\vskip-\pictOffsetBottom}\ignorespaces
	\def\tmpmiddle{middle}\ignorespaces
	\def\tmpalignment{#2}\ignorespaces
	\ifx\tmpalignment\tmpmiddle\ignorespaces
		\middp=0.5\ht0
		\midht=0.5\ht0
		\advance\middp by -4pt
		\advance\midht by 4pt
	\else\ignorespaces
		\middp=0pt
		\midht=\ht0
		\advance\middp by 3pt
		\advance\midht by -3pt
	\fi\ignorespaces
	\advance\middp by 0.5\pictOffsetTop
	\advance\midht by -0.5\pictOffsetTop
	\advance\middp by -0.5\pictOffsetBottom
	\advance\midht by 0.5\pictOffsetBottom
	\dp0=\middp
	\ht0=\midht
	\nobreak\hskip0pt\nobreak\,\nobreak\box0\,\ignorespaces
}}


% Some convenient shorthand macros for commonly used derivations of definition picture are defined below
\def\currentPicture{\reuseMPgraphic{\currentInstance::currentPicture}}

\def\drawUnitLine{\dosingleempty\drawunitline}
\def\drawunitline[#1]#2{{\ignorespaces
	\iffirstargument\ignorespaces
		\drawFromCurrentPicture[middle][uline#2#1]{startAutoLabeling;draw byNamedCompoundLine(#1, 0)(#2);stopAutoLabeling;}\ignorespaces
	\else\ignorespaces
		\drawFromCurrentPicture[middle][uline#2]{startAutoLabeling;draw byNamedCompoundLine(1cm, 0)(#2);stopAutoLabeling;}\ignorespaces
	\fi\ignorespaces}}

\def\drawProportionalLine#1{\drawFromCurrentPicture[middle][lineprop#1]{startAutoLabeling;draw byNamedCompoundLine(1cm, 1)(#1);stopAutoLabeling;}}

\def\drawSizedLine{\dosingleempty\drawsizedline}
\def\drawsizedline[#1]#2{{\ignorespaces
	\iffirstargument\ignorespaces
		\drawFromCurrentPicture[middle][linesized#2#1]{startAutoLabeling;draw byNamedCompoundLine(#1, 2)(#2);stopAutoLabeling;}\ignorespaces
	\else\ignorespaces
		\drawFromCurrentPicture[middle][linesized#2]{startAutoLabeling;draw byNamedCompoundLine(2cm, 2)(#2);stopAutoLabeling;}\ignorespaces
	\fi\ignorespaces}}
	
\def\drawTwoRightAngles{\ignorespaces
\drawFromCurrentPicture[middle][tworightangles]{draw twoRightAngles;}}
	
\def\drawRightAngle{\ignorespaces
\drawFromCurrentPicture[middle][onerightangle]{draw rightAngle;}}

\def\drawAngle#1{\drawFromCurrentPicture[middle][angle#1]{startAutoLabeling;draw byNamedAngle(#1);draw byNamedAngleDummySides(#1);stopAutoLabeling;}}

\def\drawPolygon{\dodoubleempty\drawpolygon}
\def\drawpolygon[#1][#2]#3{{\ignorespaces
	\iffirstargument\ignorespaces
		\global\def\plal{#1}\ignorespaces
	\else\ignorespaces
		\global\def\plal{middle}\ignorespaces
	\fi\ignorespaces
	\ifsecondargument\ignorespaces
		\drawFromCurrentPicture[\plal][#2]{startAutoLabeling;draw byNamedPolygon(#3);stopAutoLabeling;}\ignorespaces
	\else\ignorespaces
		\drawFromCurrentPicture[\plal][polygon#3]{startAutoLabeling;draw byNamedPolygon(#3);stopAutoLabeling;}\ignorespaces
	\fi\ignorespaces}}

\def\drawCircle{\dodoubleempty\drawcircle}
\def\drawcircle[#1][#2]#3{{\ignorespaces
	\iffirstargument\ignorespaces
		\global\def\plal{#1}\ignorespaces
	\else\ignorespaces
		\global\def\plal{middle}\ignorespaces
	\fi\ignorespaces
	\ifsecondargument\ignorespaces
		\drawFromCurrentPicture[\plal][circle#3]{
		startAutoLabeling;
		startTempScale(#2);
		draw byNamedCircle(#3);
		stopTempScale;
		stopAutoLabeling;
		}\ignorespaces
	\else\ignorespaces
		\drawFromCurrentPicture[\plal][circle#3]{startAutoLabeling;draw byNamedCircle(#3);stopAutoLabeling;}\ignorespaces
	\fi\ignorespaces}}

\def\drawArc{\dodoubleempty\drawarc}
\def\drawarc[#1][#2]#3{{\ignorespaces
	\iffirstargument\ignorespaces
		\global\def\plal{#1}\ignorespaces
	\else\ignorespaces
		\global\def\plal{middle}\ignorespaces
	\fi\ignorespaces
	\ifsecondargument\ignorespaces
		\drawFromCurrentPicture[\plal][arc#3]{
		startAutoLabeling;
		startTempScale(#2);
		draw byNamedArc(#3);
		stopTempScale;
		stopAutoLabeling;
		}\ignorespaces
	\else\ignorespaces
		\drawFromCurrentPicture[\plal][arc#3]{startAutoLabeling;draw byNamedArc(#3);stopAutoLabeling;}\ignorespaces
	\fi\ignorespaces}}

\def\drawLine{\dodoubleempty\drawline}
\def\drawline[#1][#2]#3{{\ignorespaces
	\iffirstargument\ignorespaces
		\global\def\plal{#1}\ignorespaces
	\else\ignorespaces
		\global\def\plal{middle}\ignorespaces
	\fi\ignorespaces
	\ifsecondargument\ignorespaces
		\drawFromCurrentPicture[\plal][#2]{startAutoLabeling;draw byNamedLineSeq(0)(#3);stopAutoLabeling;}\ignorespaces
	\else\ignorespaces
		\drawFromCurrentPicture[\plal][line#3]{startAutoLabeling;draw byNamedLineSeq(0)(#3);stopAutoLabeling;}\ignorespaces
	\fi\ignorespaces}}
	
\def\drawPointM#1{\drawFromCurrentPicture[middle][pointm#1]{startAutoLabeling; draw byNamedPointMark(#1); stopAutoLabeling;}}

\def\drawPointL#1{\drawFromCurrentPicture[middle][pointm#1]{startAutoLabeling; draw byNamedPointLines(#1,""); stopAutoLabeling;}}

\def\drawMagnitude{\dodoubleempty\drawmagnitude}
\def\drawmagnitude[#1][#2]#3{{\ignorespaces
	\iffirstargument\ignorespaces
		\global\def\plal{#1}\ignorespaces
	\else\ignorespaces
		\global\def\plal{middle}\ignorespaces
	\fi\ignorespaces
	\ifsecondargument\ignorespaces
		\drawFromCurrentPicture[\plal][magnitude#2#3]{startAutoLabeling;draw byNamedMagnitude(#2)(#3);stopAutoLabeling;}\ignorespaces
	\else\ignorespaces
		\drawFromCurrentPicture[\plal][magnitude#3]{startAutoLabeling;draw byNamedMagnitude(0)(#3);stopAutoLabeling;}\ignorespaces
	\fi\ignorespaces}}
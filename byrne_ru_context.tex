\definepapersize[custom]
 [width=145mm,height=200mm]
\setuppapersize[custom][custom]
\setuppagenumbering[alternative=doublesided]
\setuplayout[backspace=12mm,
	width=76mm,	height=172mm,
	header=5mm,
	headerdistance=8mm,
	topspace=10mm,
	footer=0mm,
	margin=52mm]
\definelayout[title][backspace=15mm,
	width=115mm,	height=172mm,
	header=5mm,
	headerdistance=8mm,
	topspace=10mm,
	footer=0mm]	
	
\switchtobodyfont[10pt]
\setupbodyfont[ebgaramond-be]

\input preamble.tex
\input preamble_be.tex

\mainlanguage[ru]

\def\inpropstr{пр.}
\def\inpoststr{пост.}
\def\indefstr{опр.}
\def\inaxstr{акс.}
\def\qedstr{ч. т. д.}
\def\hypstr{гип.}
\def\conststr{постр.}

\def\mpPre{textLabels := true;}

\starttext

\setuplayout[title]

\setupheader [state=stop]

\startalignment[middle]

{\tfb \WORD{Первые шесть книг}}

\vskip 0.5\baselineskip

{\tfc \WORD{Начал Евклида}}

\vskip 0.5\baselineskip

{\WORD{в которых используются цветные схемы и~знаки вместо букв для большего удобства обучающихся}}

\vskip 0.75\baselineskip

{\tfb \WORD{Оливера Бирна}}

%{\WORD{SURVEYOR OF HER MAJESTY'S SETTLEMENTS IN THE FALKLAND ISLANDS AND AUTHOR OF NUMEROUS MATHEMATICAL WORKS}}

\defineNewPicture{
scaleFactor := 7/6;
angleScale := 4/3;
pair A, B, C, D, E, F, G, H, I, J, K, L, M, d[];
A := (0, 0);
B := A shifted (-7/10u, -8/7u);
C = whatever[A, A shifted ((A-B) rotated 90)] = whatever[B, B shifted dir(0)];
d1 := (B-A) rotated -90;
D := A shifted d1;
E := B shifted d1;
d2 := (A-C) rotated -90;
F := C shifted d2;
G := A shifted d2;
d3 := (C-B) rotated -90;
H := B shifted d3;
I := C shifted d3;
J = whatever[A, A shifted dir(90)];
J = whatever[B, C];
K = whatever[A, A shifted dir(90)];
K = whatever[H, I];
L = whatever[B, F];
L = whatever[A, C];
M = whatever[A, I];
M = whatever[B, C];
draw byPolygon(A,B,E,D)(byblack);
draw byPolygon(L,A,G,F)(byred);
draw byPolygon(C,L,F)(byred);
draw byPolygon(J,M,I,K)(byyellow);
draw byPolygon (M,C,I)(byyellow);
draw byPolygon(B,J,K,H)(byblue);
byAngleDefine(F, C, A, byyellow, 0);
byAngleDefine(B, C, I, byblue, 0);
byAngleDefine(A, C, B, byblack, 0);
draw byNamedAngleResized();
draw byLine(A, K, byred, 1, 0);
draw byLineFull(B, F, byblack, 0, 0)(G, G, 1, 1, -1);
draw byLineFull(A, I, byblack, 0, 0)(K, K, 1, 1, 1);
byLineDefine(C, F, byblue, 1, 0);
byLineDefine(C, I, byblack, 1, 0);
draw byNamedLineSeq(0)(CF,CI);
byLineDefine(A, B, byyellow, 0, 0);
byLineDefine(B, C, byred, 0, 0);
byLineDefine(C, A, byblue, 0, 0);
draw byNamedLineSeq(-1)(AB,BC,CA);
byLineDefine.CAb(C, A, byblack, 0, 0);
byLineStylize (M, M, 1, 0, -1) (CAb);
byLineDefine.AMb(A, M, byblack, 0, 0);
byLineStylize (C, C, 0, 1, -1) (AMb);
byLineDefine.BCb(B, C, byblack, 0, 0);
byLineStylize (L, L, 0, 1, -1) (BCb);
byLineDefine.BLb(L, B, byblack, 0, 0);
byLineStylize (C, C, 1, 0, -1) (BLb);
}

\vfill\vfill

~\hfill\currentPicture\hfill~

\vfill\vfill\vfill

{\tfa github.com/jemmybutton}
\vskip 0.25\baselineskip

{\tfb 2019 изд.\,0.4}
\vskip \baselineskip

\symbol[cc][cc] \symbol[cc][by] \symbol[cc][sa]
\vskip 0.25\baselineskip

\startnarrower
\setuplocalinterlinespace[line=2ex]
{\tfx Эта редакция и~перевод книги Оливера Бирна \emph{The first six books of the Elements of Euclid} подготовлены Сергеем Слюсаревым и~распространяются под лицензией CC-BY-SA 4.0}
\stopnarrower

\vskip -\baselineskip

\stopalignment

\pagebreak\ \pagebreak

\setupheader[state=start]
\setuplayout[reset]

\startintro[title={Введение}]

\regularLettrine{И}{\sc скусства} и~науки получили такое развитие, что облегчить их усвоение не менее важно, чем расширить их границы. Иллюстрации если и~не уменьшают времени обучения, то по крайней мере делают его более приятным. Цель этой работы, однако, больше, чем просто иллюстрация; мы не вводим цвета лишь ради развлечения или услаждения взгляда \emph{определенными сочетаниями окраски и~формы}, но чтобы помочь разуму в~его поисках истины, облегчить знакомство с~материалом и~распространить постоянные знания. Пожелай мы услышать мнение великих о~важности и~полезности геометрии, мы могли бы вспомнить любого философа со времен Платона. У~греков, как и~в школе Песталоцци и~прочих в~наши дни, геометрия преподавалась как лучшее упражнение для ума. В~действительности, Начала Евклида стали, по общему признанию, основой математической науки во всем цивилизованном  мире. Но это не удивительно, учитывая тот факт, что эта величественная наука не только  больше прочих пригодна к~тому, чтобы пробудить дух исследования, возывсить разум и~усилить способности к~рассуждению, но и~составляет лучшее введение в~наиболее полезные и~важные для человека профессии. Арифметика, топография, гидростатика, пневматика, оптика, физическая астрономия и~т.~д. все полагаются на предложения геометрии.

\kerncharacters[0.005]{Многое, однако, зависит от того, как наука преподнесена учащемуся, хотя лучшие и~самые простые методы применяются редко. О~предложениях, которые предоставляются ученику, пусть и~обладающему достаточным пониманием, говорится столь же мало, сколь неблагоприятное  предубеждение он получает относительно будущего изучения этого  занимательного предмета. Другими словами, \quotation{формальности и~технические подробности настолько выставляются напоказ, что почти скрывают за собой действительность. Бесконечные  озадачивающие повторения, не добавляющие ясности суждениям, делают демонстрации запутанными и~туманными и~скрывают от ученика последовательность доказательства}. В~результате у~учащегося возникает отвращение, и~предмет, предназначенный улучшить мыслительные способности и~привить привычку к~вдумчивости, через сухое и~черствое изложение опускается до уровня скучного упражнения для памяти. Целью учителя должно быть возбуждение любопытства и~пробуждение бесчисленных спящих сил молодых умов, но там, где больше всего требуются образцы совершенства, попытки достичь его немногочисленны, при том, что выдающиеся примеры привлекают внимание и~порождают подражания. Целью данной работы является представление метода обучения геометрии, получившего одобрение многих ученых этой страны, а~также Франции и~Америки. Предложенный здесь план ярко обращается к~глазам, самому чувствительному и~всестороннему из наших органов, превосходство которого в~запечатлевании своего предмета в~разуме подтверждается неопровержимой максимой, выраженной известными словами Горация:}

\vskip 0.5\baselineskip

\startalignment[middle]
\emph{Segnius irritant animos demissa per aurem\\
Quam quae sunt oculis subjecta fidelibus}\\
\marginNote{Перевод М.~Дмитриева}
Что к~нам доходит чрез слух, то слабее в~нас трогает сердце,\\
Нежели то, что само представляется верному глазу. % Пер. М. Дмитриева http://lib.ru/POEEAST/GORACIJ/hor1_6.txt
%Трогает душу слабее, что приемлется слухом,\\
%Чем все то, что, видя глазами верными, зритель\\
%Сам себе сообщает... % Пер. А. А. Фета http://lib.ru/POEEAST/GORACIJ/hor3_1.txt_with-big-pictures.html
\stopalignment

\vskip 0.5\baselineskip

\kerncharacters[0.005]{Язык целиком состоит из знаков, и~те знаки лучшие, какие выполняют свое назначение с~наибольшей точностью и~быстротой. Так, для всех обычных целей применяются слышимые знаки, называемые словами, которые считаются слышимыми, доносятся ли они непосредственно через уши или опосредованно через глаза в~виде букв. Геометрические построения~— это не знаки, но принадлежности геометрической науки, задача которых~— показать относительные размеры своих частей, с~помощью процесса рассуждения, называемого демонстрацией. Такое рассуждение чаще всего доносилось в~виде слов, букв и~черных, нераскрашенных диаграмм, но поскольку использование цветных символов, знаков и~диаграмм в~искусствах и~науках делает ход рассуждения более точным, а~усвоение более скорым, в~данном случае они были соответствующим образом внедрены.}

Этот весьма заманчивый способ донесения знаний столь действенен, что «Начала» Евклида могут быть усвоены менее чем за треть от времени, какое требуется обычно, а~в~памяти они сохранятся, напротив, намного дольше. Эти факты были установлены во множестве экспериментов, проведенных как автором, так и~другими, внедрившими его задумку. Рецепт же короток и~очевиден: буквы, присвоенные точкам, линиям или другим частям построения, в~действительности всего лишь произвольные имена и~представляют те в~демонстрации, вместо этого части, получающие различные цвета, называют сами себя, так как их формы и~соответствующие цвета представляют их в~демонстрации.

\defineNewPicture{
	pair A, B, C;
	B := (0, 0);
	A := B shifted (dir(-145)*3u);
	C = whatever[A, A shifted (1,0)] = whatever[B, B shifted dir(-145+90)];
		byAngleDefine(A, B, C, byyellow, 0);
		byAngleDefine(B, C, A, byblue, 0);
		byAngleDefine(C, A, B, byred, 0);
		draw byNamedAngleResized();
		byLineDefine(A, B, byblue, 0, 0);
		byLineDefine(B, C, byred, 0, 0);
		byLineDefine(C, A, byyellow, 0, 0);
		draw byNamedLineSeq(0)(AB,BC,CA);
	label.top(\sometxt{B}, B);
		label.rt(\sometxt{C}, C);
		label.lft(\sometxt{A}, A);
	textLabels := false;
	angleScale := 3/4;
}
\drawCurrentPictureInMargin
Чтобы дать лучшее представление об этой системе и~преимуществах, получаемых от ее внедрения, возьмем прямоугольный треугольник и~выразим некоторые его свойства как цветами, так и~общепринятым способом.

\vskip \baselineskip

\startalignment[middle]
\emph{Некоторые свойства прямоугольного треугольника ABC, выраженные общепринятым способом:}
\stopalignment

\vskip 0.5\baselineskip

\startitemize[m,joinedup,nowhite]
\item Угол BAC вместе с~углами BCA и~ABC равны двум прямым углам или дважды углу ABC.
\item Угол CAB вместе с~углом ACB равны углу ABC.
\item Угол ABC больше как BAC, так и~BCA.
\item Угол BCA, как и~угол CAB, меньше угла ABC.
\item Если из угла ABC вычесть угол BAC, остаток будет равен углу ACB.
\item Квадрат стороны AC равен сумме квадратов сторон AB и~BC.
\stopitemize

\vskip \baselineskip

\startalignment[middle]
\emph{Те же свойства, выраженные раскрашиванием различных частей:}
\stopalignment

\vskip 0.5\baselineskip

\startitemize[m,joinedup,nowhite]
\item $\drawAngle{A} + \drawAngle{B} + \drawAngle{C} = 2 \drawAngle{B} = \drawTwoRightAngles$. \\ То есть, красный угол, вместе с~желтым углом, вместе с~синим углом равны дважды желтому углу, равны двум прямым углам.
\item $\drawAngle{A} + \drawAngle{C} = \drawAngle{B}$. \\ Или, словами, красный угол, добавленный к~синему углу, равен желтому углу.
\item $\drawAngle{B} > \drawAngle{A} \mbox{ или } > \drawAngle{C}$. \\ Желтый угол больше красного или синего.
\item $\drawAngle{A} \mbox{ или } \drawAngle{C} < \drawAngle{B}$. \\ И~красный, и~синий углы меньше желтого.
\item $\drawAngle{B} - \drawAngle{C} = \drawAngle{A}$. \\ Иначе говоря, желтый угол за вычетом синего угла равен красному углу.

\vfill\pagebreak

\item $\drawUnitLine{CA}^2 = \drawUnitLine{AB}^2 + \drawUnitLine{BC}^2$. \\ То есть квадрат желтой линии равен сумме квадратов синей и~красной.
\stopitemize

\vskip \baselineskip

Цвета в~устных демонстрациях дают нам то важное преимущество, что позволяет обратиться к~глазам и~ушам одновременно, так что для обучения геометрии и~другим наукам в~аудитории эта система является лучшей из когда-либо предложенных, что очевидно из приведенных примеров.

Отсюда очевидно, что отсылки в~тексте к~построению считываются быстрее и~надежнее, если приводить формы и~цвета их частей или называть части по их цветам, чем если называть части и~буквы на построении. Помимо превосходящей легкости, система также выделяется тем, что обеспечивает сосредоточение и~полностью исключает вредоносную, хотя и~распространенную практику передоверять всю демонстрацию памяти так, что рассуждение, факты и~доказательства создают лишь впечатление понимания.

\kerncharacters[-0.01]{Опять же если мы, говоря о~свойствах фигур перед аудиторией, упомянем цвет части или частей, о~которых идет речь, как то: красный угол, синяя линия или линии и~т. п., так обозначенные часть или часть будут сразу же видны всем; совсем не то что сказать про угол ABC, треугольник PFQ, фигуру EGKt и~так далее, поскольку буквы приходится разыскивать по одной, прежде чем ученики составят в~уме ту конкретную величину, о~которой идет речь, что часто может привести к~путанице и~ошибкам, а~также к~потере времени. Кроме того, если части, принятые равными, окрашены в~один цвет на построении, разуму не придется отклоняться от рассматриваемого объекта, то есть такое расположение наглядно предоставляет части, равенство которых предстоит показать, и~учащийся не упускает из виду эти сведения в~течение всего рассуждения. Но какими бы ни были преимущества представленного плана, если даже не прибегать к~нему всегда, он может быть мощным дополнительным инструментом наряду с~другими методами, используемым для введения в~материал, более быстрого запоминания или более надежного закрепления в~памяти.}

Опыт тех, кому приходилось создавать системы для закрепления фактов пониманием, доказывает, что цветные представления, такие как изображения, вырезанные фигуры, диаграммы и~т. п., укладываются в~сознании много легче, чем простые, ничем не выделяющиеся высказывания. Любопытно, что поэтам эта истина знакома лучше, чем математикам, многие современные поэты указывают на подобную систему донесения знаний, один из них выразился так:

\vskip 0.5\baselineskip

\marginNote{В~оригинале автор, по-видимому по ошибке, приводит другой перевод тех же строк Горация. Здесь на~его месте перевод фрагмента из~поэмы Марка Эйкенсайда (1721–1770) The~Pleasures Of~Imagination.}
\startalignment[middle] % Вместо другого перевода тех же строк Горация, вольный перевод близких по смыслу стихов Марка Эйкенсайда The Pleasures Of Imagination by Mark Akenside (1721–1770) https://archive.org/stream/pleasuresofimagi00aken#page/50/mode/2up
Ведь жаждет знаний человек, и~истины лучи\\
Охотней тронут пониманья глаз,\\
Чем уши звуков лесть,\\
Чем всякий вкус язык...
% For man loves knowledge, and the beams of Truth
% More welcome touch his understanding's eye,
% Than all the blandishments of sound his ear,
% Than all of taste his tongue...
\stopalignment

\vskip 0.5\baselineskip

Это, возможно, единственное улучшение, которое получила геометрия на плоскости со времен Евклида, а~если до того и~были значимые геометры, то успех Евклида затмил их в~памяти до такой степени, что многие достижения в~этой области приписываются ему, подобно Эзопу среди баснописцев. Также нелишним будет отметить, что как осязаемые построения представляют единственный способ донесения геометрии до слепых, так и~зримая система не менее приспособлена к~нуждам глухих и~немых.

Следует уделить внимание, объяснению того, что цвета не имеют иного отношения к~линиям, углам, величинам, кроме как обозначить их. Математическая линия, представляющая собой длину без ширины, не может иметь цвета, но место соприкосновения двух цветов на одной плоскости дает хорошее представление о~том, что такое линия в~математике. Запомним, что, строго говоря, мы имеем в~виду такое соприкосновение, а~не цвет, когда говорим о~черной линии, красной линии или линиях и~т. п.

Цвета и~раскрашенные диаграммы могут, на первый взгляд, показаться неуклюжим способом донесения правильных представлений о~свойствах и~частях математических фигур и~величин, однако, как мы увидим, они являются средством более изящным и~емким, чем любые предложенные ранее.

Теперь мы дадим определение точке, линии и~поверхности и~продемонстрируем предложение, чтобы показать верность этого суждения.

Точкой называют то, у~чего есть положение, но нет величины, или точка — это одно лишь положение, лишенное длины, ширины и~толщины. Следующее описание, возможно, более пригодно для объяснения сущности математической точки тем, кто еще не освоил идею, имеющую приведенное выше обманчивое определение.

\defineNewPicture{
	angleScale := 2;
	pair O, A, B, C;
	O := (0, 0);
	A := dir(0) scaled 3u;
	B := dir(120) scaled 3u;
	C := dir(240) scaled 3u;
		draw byAngle(A, O, B, byred, 0);
		draw byAngle(B, O, C, byblue, 0);
		draw byAngle(C, O, A, byyellow, 0);
}
\drawCurrentPictureInMargin Пусть три цвета встречаются и~покрывают участок бумаги. В~месте, где они встречаются, цвет не будет ни синим, ни желтым, ни красным, так как оно не занимает пространства на бумаге, а~если бы занимало, то относилось бы к~синей, красной или желтой части. Но все же оно существует и~имеет положение без величины, так что без особенных сложностей можно заметить, что место встречи трех цветов на плоскости дает хорошее представление о~математической точке.

Линия~— это длина без ширины. С~помощью цветов почти так же, как и~выше, понятие линии можно донести следующим образом:

\defineNewPicture{
	pair A, B, C, D, E, F;
	A := (0, 0);
	B := (5/2u, ypart(A));
	C := (xpart(A), -2u);
	D := (xpart(B), ypart(C));
	E := 1/2[A, C];
	F := 1/2[B, D];
		draw byPolygon(A,B,F,E)(byred);
		draw byPolygon(C,D,F,E)(byblue);
}
\drawCurrentPictureInMargin
Пусть два цвета встречаются и~покрывают участок бумаги. Где они встречаются, цвет не будет ни красным, ни синим. Следовательно, соединение не занимает никакой части плоскости, и~значит, не имеет ширины, а~только длину. Отсюда легко получить представление о~том, что понимается под математической линией. Для иллюстрации было бы достаточно одного цвета, отличного от цвета бумаги или другой поверхности, на которую нанесен рисунок. Итак, в~дальнейшем, когда мы говорим о~красной линии, синей линии или линиях и~т. п., в~виду будет иметься место соприкосновения с~плоскостью, на которой они нарисованы.

Поверхность~— это то, у~чего есть длина и~ширина, но нет толщины.

\defineNewPicture{
	pair A', A'', A''', B', B'', B''', C', C'', C''', D', D'', D''', d[];
	d1 := (3/2u, 0);
	d2 := (-3/4u, -2/3u);
	d3 := (0, -3/2u);
	A' := (0, 0);
	B' := A' shifted d1;
	C' := A' shifted d2;
	D' := C' shifted d1;
	A'' := A' shifted d3;
	B'' := B' shifted d3;
	C'' := C' shifted d3;
	D'' := D' shifted d3;
	A''' := A'' shifted d3;
	B''' := B'' shifted d3;
	C''' := C'' shifted d3;
	D''' := D'' shifted d3;
		draw byPolygon(A',B',B'',A'',C'',C')(byred);
		draw byPolygon(A'',B'',D'',C'')(byblue);
		draw byPolygon(C'',D'',B'',B''',D''',C''')(byyellow);
		draw byLine(A''', B''', white, 0, 2);
		draw byLine(A''', C''', white, 0, 2);
		draw byLine(A''', A', white, 0, 2);
		draw byLine(D', C', white, 0, 1);
		draw byLine(D', B', white, 0, 1);
		draw byLine(D', D''', white, 0, 1);
		label.lft(\sometxt{P}, C');
		label.lft(\sometxt{R}, C'');
		label.rt(\sometxt{S}, B'');
		label.rt(\sometxt{Q}, B''');
}
\drawCurrentPictureInMargin
Рассматривая объемное тело (PQ), мы сразу понимаем, что у~него есть три измерения, а~именно: длина, ширина и~толщина. Теперь предположим, что одна часть тела (PS) красная, а~другая (QR) желтая, и~эти цвета раздельны и~не смешиваются. Синяя поверхность (RS), разделяющая эти части, или, иначе говоря, разделяющая тело без потери материала, должна быть без толщины и~иметь только длину и~ширину. Это, очевидно из рассуждений, подобных примененным выше для определения, а~точнее для описания точки или линии.

\defineNewPicture[1/4]{
pair A, B, C, D, E;
A := (0, 0);
B := A shifted (u, -2u);
C := B xscaled -1;
D := 9/5[A,B];
E := 9/5[A,C];
byAngleDefine(B, A, C, byblack, 0);
byAngleDefine(A, B, C, byblue, 0);
byAngleDefine(B, C, A, byblue, 0);
byAngleDefine(C, B, E, byyellow, 0);
byAngleDefine(D, C, B, byyellow, 0);
byAngleDefine(B, D, C, byred, 0);
byAngleDefine(C, E, B, byred, 0);
byAngleDefine(E, B, D, byyellow, 1);
byAngleDefine(D, C, E, byyellow, 1);
draw byNamedAngleResized(BAC,ABC,BCA,CBE,DCB,BDC,CEB,EBD,DCE);
byLineDefine(B, D, byyellow, 0, 0);
byLineDefine(C, E, byyellow, 0, 0);
byLineDefine(B, E, byblue, 0, 0);
byLineDefine(C, D, byblue, 0, 0);
byLineDefine(A, B, byred, 0, 0);
byLineDefine(A, C, byred, 0, 0);
byLineDefine(B, C, byblack, 0, 0);
draw byNamedLineSeq(0)(CD,noLine,BC,noLine,BE,CE,AC,AB,BD);
label.top(\sometxt{A}, A);
label.lft(\sometxt{C}, C);
label.rt(\sometxt{B}, B);
label.lft(\sometxt{E}, E);
label.rt(\sometxt{D}, D);
}
\drawCurrentPictureInMargin
Для того чтобы прояснить, как именно применяются эти принципы, мы выбрали предложение из первой книги.

В равнобедренном треугольнике ABC внутренние углы при основании ABC и~ACB равны, а~если продлить стороны AB и~AC, внешние углы при основании BCE, CBD, также будут равны.

\startCenterAlign
Продлим \drawUnitLine{AB} и~\drawUnitLine{AC},\\
сделаем $\drawUnitLine{BD} = \drawUnitLine{CE}$, проведем \drawUnitLine{BE} и~\drawUnitLine{CD}.

В
\drawFromCurrentPicture{
startAutoLabeling;
startTempAngleScale(angleScale*4/5);
draw byNamedAngle(BAC);
draw byNamedLineSeq(0)(BE,CE,AC,AB);
stopTempAngleScale;
stopAutoLabeling;
}
и
\drawFromCurrentPicture{
startAutoLabeling;
startTempAngleScale(angleScale*4/5);
draw byNamedAngle(BAC);
draw byNamedLineSeq(0)(BD,CD,AC,AB);
stopTempAngleScale;
stopAutoLabeling;
}\\
получим $\drawUnitLine{AB,BD} = \drawUnitLine{AC,CE}$,\\
\drawAngle{BAC} общий и~$\drawUnitLine{AB} = \drawUnitLine{AC}$.

$\therefore \drawAngle{BCA,DCB} = \drawAngle{ABC,CBE}$, $\drawUnitLine{BE} = \drawUnitLine{CD}$\\
и $\drawAngle{CEB} = \drawAngle{BDC}$ \inprop[prop:I.IV].

Теперь в~\drawFromCurrentPicture{
startAutoLabeling;
startTempAngleScale(angleScale*4/5);
draw byNamedAngle(E);
draw byNamedLineSeq(0)(BE,CE,BC);
stopTempAngleScale;
stopAutoLabeling;
} и~\drawFromCurrentPicture{
startAutoLabeling;
startTempAngleScale(angleScale*4/5);
draw byNamedAngle(D);
draw byNamedLineSeq(0)(BD,CD,BC);
stopTempAngleScale;
stopAutoLabeling;
}\\
$\drawUnitLine{BD} = \drawUnitLine{CE}$, $\drawAngle{CEB} = \drawAngle{BDC}$\\
и $\drawUnitLine{BE} = \drawUnitLine{CD}$.

$\therefore \drawAngle{DCE,DCB} = \drawAngle{EBD,CBE}$
и $\drawAngle{DCB} = \drawAngle{CBE}$ \inprop[prop:I.IV].

Но $\drawAngle{BCA,DCB} = \drawAngle{ABC,CBE}$, $\therefore \drawAngle{BCA} = \drawAngle{ABC}$.
\stopCenterAlign

\qedNB

\startalignment[middle]
\emph{Если заменить буквами диаграммы}
\stopalignment

Продлим равные стороны AB и~AC через концы третьей стороны BC. На любой из продленных частей BD возьмем любую точку D и~от другой отсечем AE, равную AD \inprop[prop:I.III]. Теперь взятые таким образом на продленных сторонах точки E и~D соединим прямыми линиями DC и~BE с~противолежащими концами третьей стороны треугольника.

\kerncharacters[0.005]{В треугольниках DAC и~EAB стороны DA и~AC соответственно равны EA и~AB, а~прилежащий угол A общий обоим. Следовательно \inprop[prop:I.IV], линия DC равна BE, угол ADC равен AEB, а~угол ACD — углу ABE. Если из равных линий AD и~AE вычесть равные AB и~AC, остатки BD и~CE также будут равны. А~значит, в~треугольниках BDC и~CEB стороны BD и~DC соответственно равны CE и~EB и~углы D и~E, заключенные между этими сторонами, также равны. А~значит \inprop[prop:I.IV], углы DBC и~ECB, заключенные между третьей стороной BC и~продолжениями сторон AB и~AC, также равны. Кроме того, углы DCB и~EBC равны, если равные углы вычитаются из углов DCA и~EBA, равенство которых было показано выше, а~значит остатки, то есть углы ABC и~ACB, противолежащие равным сторонам, будут равны.}

\emph{Следовательно, в~равностороннем треугольнике…} и~т. д.

\qedNB

\vskip \baselineskip

\kerncharacters[0.005]{Поскольку здесь нашей целью было представить систему, а~не объяснить какой-либо набор предложений, мы взяли то, что выше, из самого курса. В~школах и~других общественных учебных заведениях с~подобными построениями помогут цветные мелки, для индивидуального использования очень удобны цветные карандаши.}

Мы очень рады, что начала математики теперь составляют часть любого достойного образования для женщин, поэтому мы призываем обратить внимание на этот привлекательный метод донесения знания и~дальнейшую работу по его развитию тех, кто заинтересован или вовлечен в~образование для дам.

В заключение отметим, что поскольку можно так же надежно и~быстро обратиться к~чувствам зрения и~слуха тысячи, как и~одного, \emph{миллион} можно обучить геометрии и~другим отраслям математики с~большой легкостью; это продвинуло бы успехи образования больше, чем что-либо, что можно назвать, поскольку это научило бы людей как думать, а~не, что характерно для образования, чт\oacute\ думать, в~чем и~заключается его большая ошибка.

\vfill\pagebreak

\noindent~\hfill {\tfb \WORD{Начала Eвклида}} \hfill~ % Положение этого заголовка в оригинальной структуре неочевидно

\vskip \baselineskip

\noindent~\hfill {\tfa \WORD{Книга I}} \hfill~

\startsupersection[title={Определения}]

\startDefinitionOnlyNumber[reference=def:I.I]
\startalignment[last]
\emph{Точка} есть то, что не имеет частей.
\stopalignment
\stopDefinitionOnlyNumber

\startDefinitionOnlyNumber[reference=def:I.II]
\startalignment[last]
\emph{Линия} — это длина без ширины.
\stopalignment
\stopDefinitionOnlyNumber

\startDefinitionOnlyNumber[reference=def:I.III]
\startalignment[last]
Концы линии ­— точки.
\stopalignment
\stopDefinitionOnlyNumber

\startDefinitionOnlyNumber[reference=def:I.IV]
\startalignment[last]
Прямая линия есть та, которая равно расположена по отношению к~точкам на ней.
\stopalignment
\stopDefinitionOnlyNumber

\startDefinitionOnlyNumber[reference=def:I.V]
\startalignment[last]
Поверхность есть то, что имеет только длину и~ширину.
\stopalignment
\stopDefinitionOnlyNumber

\startDefinitionOnlyNumber[reference=def:I.VI]
\startalignment[last]
Концы поверхности — линии.
\stopalignment
\stopDefinitionOnlyNumber

\startDefinitionOnlyNumber[reference=def:I.VII]
\startalignment[last]
Плоская поверхность есть та, которая равно расположена по отношению к~прямым на ней.
\stopalignment
\stopDefinitionOnlyNumber

\startDefinitionOnlyNumber[reference=def:I.VIII]
\startalignment[last]
Плоский угол есть наклонение друг к~другу двух линий, в~плоскости встречающихся друг с~другом, но не направленных одинаково.
\stopalignment
\stopDefinitionOnlyNumber

\defineNewPicture{
	pair A, B, C;
	A := (0, 0);
	B := (2/3u, 2/3u);
	C := (u, ypart(A));
		byAngleDefine(B, A, C, byyellow, 0);
		draw byNamedAngleResized();
		byLineDefine(A, B, byblue, 0, 0);
		byLineDefine(A, C, byred, 0, 0);
		draw byNamedLineSeq(0)(AC,AB);
}
\startDefinitionOnlyNumber[reference=def:I.IX]
\startalignment[last]
\drawCurrentPictureInMargin[inside] Прямолинейный угол есть наклонение друг к~другу двух пересекающихся и~не совпадающих прямых линий.
\stopalignment
\stopDefinitionOnlyNumber

\defineNewPicture{
	pair A, B, C, D;
	A := (0, 0);
	B := (u, 0);
	C := (0, 2/3u);
	D := (-u, 0);
		byAngleDefine(B, A, C, byblack, 1);
		byAngleDefine(D, A, C, byblack, 1);
		draw byNamedAngleResized();
		draw byLine(D, B, byblack, 0, 0);
		draw byLine(A, C, byblack, 0, 0);
}
\startDefinitionOnlyNumber[reference=def:I.X]
\startalignment[last]
\drawCurrentPictureInMargin[inside] Когда прямая восстановленная на другой прямой образует равные смежные углы, каждый из этих углов называется \emph{прямым углом}, а~каждая из этих прямых называется \emph{перпендикуляром} к~другой.
\stopalignment
\stopDefinitionOnlyNumber

\defineNewPicture{
	pair A, B, C;
	A := (0, 0);
	B := (-2/3u, 2/3u);
	C := (u, ypart(A));
		byAngleDefine(B, A, C, byred, 0);
		draw byNamedAngleResized();
		byLineDefine(A, B, byyellow, 0, 0);
		byLineDefine(A, C, byblue, 0, 0);
		draw byNamedLineSeq(0)(AC,AB);
}
\drawCurrentPictureInMargin[inside]
\startDefinitionOnlyNumber[reference=def:I.XI]
\startalignment[last]
\emph{Тупым} называется угол больше прямого.
\stopalignment
\stopDefinitionOnlyNumber

\defineNewPicture{
	pair A, B, C;
	A := (0, 0);
	B := (2/3u, 2/3u);
	C := (u, ypart(A));
		byAngleDefine(B, A, C, byblue, 0);
		draw byNamedAngleResized();
		byLineDefine(A, B, byyellow, 0, 0);
		byLineDefine(A, C, byred, 0, 0);
		draw byNamedLineSeq(0)(AC,AB);
}
\drawCurrentPictureInMargin[inside]
\startDefinitionOnlyNumber[reference=def:I.XII]
\startalignment[last]
\emph{Острым} называется угол меньше прямого.
\stopalignment
\stopDefinitionOnlyNumber

\startDefinitionOnlyNumber[reference=def:I.XII]
\startalignment[last]
\emph{Граница} есть то, что является оконечностью чего-либо.
\stopalignment
\stopDefinitionOnlyNumber

\startDefinitionOnlyNumber[reference=def:I.XII]
\startalignment[last]
\emph{Фигура} есть то, что содержится внутри какой-нибудь или каких-нибудь границ.
\stopalignment
\stopDefinitionOnlyNumber

\defineNewPicture{
	pair O, A, B, C, D, E;
	numeric r;
	r := 1/2u;
	O := (0, 0);
	A := dir(0) scaled r;
	B := dir(60) scaled r;
	C := dir(130) scaled r;
	D := dir(180) scaled r;
	E := dir(-60) scaled r;
		draw byLine(O, B)(byblack, 0, 0);
		draw byLine(O, C)(byred, 0, 0);
		draw byLine(O, E)(byyellow, 0, 0);
		draw byLine(D, A)(byblue, 0, 0);
		draw byCircleR(O, r, byred, 0, 0, 0);
}
\startDefinitionOnlyNumber[reference=def:I.XV]
\startalignment[last]
\drawCurrentPictureInMargin[inside] \emph{Круг} есть плоская фигура, содержащаяся внутри одной линии, называемой окружностью, в~которой есть точка, все прямые падающие на окружность из которой равны.
\stopalignment
\stopDefinitionOnlyNumber

\startDefinitionOnlyNumber[reference=def:I.XVI]
\startalignment[last]
Точка в~круге, все прямые к~окружности из которой равны называется \emph{центром.}
\stopalignment
\stopDefinitionOnlyNumber

\defineNewPicture{
	pair O, A, B;
	numeric r;
	r := 1/2u;
	O := (0, 0);
	A := dir(0) scaled r;
	B := dir(180) scaled r;
		draw byLine(A, B)(byyellow, 0, 0);
		draw byCircleR(O, r, byred, 0, 0, 0);
}
\startDefinitionOnlyNumber[reference=def:I.XVII]
\startalignment[last]
\drawCurrentPictureInMargin[inside] \emph{Диаметр} круга есть прямая линия, проходящая через центр и~ограниченная с~обеих сторон окружностью.
\stopalignment
\stopDefinitionOnlyNumber

\defineNewPicture{
	pair O, A, B;
	numeric r;
	r := 1/2u;
	O := (0, 0);
	A := dir(0) scaled r;
	B := dir(180) scaled r;
		draw byLine(A, B)(byblue, 0, 0);
		draw byArc(O, A, B)(r, byyellow, 0, 0, 0, 0);
		draw byArc(O, B, A)(r, byyellow, 1, 0, 0, 0);
}
\startDefinitionOnlyNumber[reference=def:I.XVIII]
\startalignment[last]
\drawCurrentPictureInMargin[inside] \emph{Полукруг} есть фигура, содержащаяся между диаметром и~отсекаемой им частью окружность.
\stopalignment
\stopDefinitionOnlyNumber

\vfill\pagebreak

\defineNewPicture{
	pair O, A, B;
	path P;
	numeric r;
	r := 1/2u;
	P := fullcircle scaled 2r;
	O := (0, 0);
	A := point 1 of P;
	B := point 3 of P;
		draw byLine(A, B)(byred, 0, 0);
		draw byArc(O, A, B)(r, byblue, 0, 0, 0, 0);
		draw byArc(O, B, A)(r, byblue, 1, 0, 0, 0);
}
\startDefinitionOnlyNumber[reference=def:I.XIX]
\startalignment[last]
\drawCurrentPictureInMargin[inside] \emph{Сегментом круга} называется фигура, содержащаяся между прямой линией и~частью окружности, ей отсекаемой.
\stopalignment
\stopDefinitionOnlyNumber

\startDefinitionOnlyNumber[reference=def:I.XX]
\startalignment[last]
\emph{Прямолинейной фигурой} называется фигура, содержащаяся между прямыми линиями.
\stopalignment
\stopDefinitionOnlyNumber

\startDefinitionOnlyNumber[reference=def:I.XXI]
\startalignment[last]
\emph{Треугольником} называется прямолинейная фигура с~тремя сторонами.
\stopalignment
\stopDefinitionOnlyNumber

\defineNewPicture{
	pair A, B, C, D;
	A := (0, 0);
	B := (u, 1/2u);
	C := (-1/2u, -4/3u);
	D := (4/3u, -u);
		draw byLine(C, B)(byred, 0, 0);
		draw byLine(A, D)(byblue, 0, 0);
		byLineDefine(A, B, byyellow, 0, 0);
		byLineDefine(A, C, byyellow, 0, 0);
		byLineDefine(B, D, byyellow, 0, 0);
		byLineDefine(C, D, byblack, 0, 0);
		draw byNamedLineSeq(0)(BD,CD,AC,AB);
		draw byLabelsOnPolygon(A, B, D, C)(0, 0);
}
\startDefinitionOnlyNumber[reference=def:I.XXII]
\startalignment[last]
\drawCurrentPictureInMargin[inside] \emph{Четырехсторонняя фигура} та, у~которой четыре стороны. Прямые линии  \drawUnitLine{AD} и~\drawUnitLine{CB}, соединяющие вершины противоположных углов четырехсторонней фигуры называются \emph{диагоналями.}
\stopalignment
\stopDefinitionOnlyNumber

\startDefinitionOnlyNumber[reference=def:I.XXIII]
\startalignment[last]
\emph{Многоугольник} есть прямоугольная фигура с~более чем четырьмя сторонами.
\stopalignment
\stopDefinitionOnlyNumber

\defineNewPicture{
	pair A, B, C;
	A := dir(-30) scaled 1/2u;
	B := dir(-150) scaled 1/2u;
	C := dir(90) scaled 1/2u;
		byLineDefine(A, B, byblue, 0, 0);
		byLineDefine(B, C, byred, 0, 0);
		byLineDefine(C, A, byyellow, 0, 0);
		draw byNamedLineSeq(0)(AB,BC,CA);
}
\startDefinitionOnlyNumber[reference=def:I.XXIV]
\startalignment[last]
\drawCurrentPictureInMargin[inside] \emph{Равносторонним} называется треугольник, все стороны которого равны.
\stopalignment
\stopDefinitionOnlyNumber

\defineNewPicture{
	pair A, B, C;
	A := dir(-60) scaled 1/2u;
	B := dir(-120) scaled 1/2u;
	C := dir(90) scaled 1/2u;
		byLineDefine(A, B, byblue, 0, 0);
		byLineDefine(B, C, byred, 0, 0);
		byLineDefine(C, A, byred, 0, 0);
		draw byNamedLineSeq(0)(AB,BC,CA);
}
\startDefinitionOnlyNumber[reference=def:I.XXV]
\startalignment[last]
\drawCurrentPictureInMargin[inside] Треугольник у~которого равны две стороны называется \emph{равнобедренным.}
\stopalignment
\stopDefinitionOnlyNumber

\startDefinitionOnlyNumber[reference=def:I.XXVI]
\startalignment[last]
\emph{Разносторонним} называется треугольник со всеми неравными сторонами.
\stopalignment
\stopDefinitionOnlyNumber

\defineNewPicture{
	pair A, B, C;
	A := (0, 0);
	B := (-u, 0);
	C := (0, 3/4u);
		byLineDefine(A, B, byred, 0, 0);
		byLineDefine(B, C, byyellow, 0, 0);
		byLineDefine(C, A, byblue, 0, 0);
		draw byNamedLineSeq(0)(AB,BC,CA);
}
\startDefinitionOnlyNumber[reference=def:I.XXVII]
\startalignment[last]
\drawCurrentPictureInMargin[inside] Треугольник у~которого есть прямой угол называется \emph{прямоугольным}.
\stopalignment
\stopDefinitionOnlyNumber

\defineNewPicture{
	pair A, B, C;
	A := (-1/4u, 0);
	B := (-u, 0);
	C := (0, 3/4u);
		byLineDefine(A, B, byred, 0, 0);
		byLineDefine(B, C, byblue, 0, 0);
		byLineDefine(C, A, byyellow, 0, 0);
		draw byNamedLineSeq(0)(AB,BC,CA);
}
\startDefinitionOnlyNumber[reference=def:I.XXVIII]
\startalignment[last]
\drawCurrentPictureInMargin[inside] \emph{Тупоугольным} называется треугольник, у~которого есть тупой угол.
\stopalignment
\stopDefinitionOnlyNumber

\defineNewPicture{
	pair A, B, C;
	A := (0, 0);
	B := (-u, 0);
	C := (-1/4u, 3/4u);
		byLineDefine(A, B, byblue, 0, 0);
		byLineDefine(B, C, byyellow, 0, 0);
		byLineDefine(C, A, byred, 0, 0);
		draw byNamedLineSeq(0)(AB,BC,CA);
}
\startDefinitionOnlyNumber[reference=def:I.XXIX]
\startalignment[last]
\drawCurrentPictureInMargin[inside] \emph{Остроугольным} называется треугольник, все углы которого острые.
\stopalignment
\stopDefinitionOnlyNumber

\defineNewPicture{
	pair A, B, C, D;
	numeric s;
	s := u;
	A := (0, 0);
	B := (s, 0);
	C := (0, s);
	D := (s, s);
		byLineDefine(A, B, byred, 0, 0);
		byLineDefine(A, C, byblue, 0, 0);
		byLineDefine(B, D, byyellow, 0, 0);
		byLineDefine(C, D, byblack, 0, 0);
		draw byNamedLineSeq(0)(AB,AC,CD,BD);
}
\startDefinitionOnlyNumber[reference=def:I.XXX]
\startalignment[last]
\drawCurrentPictureInMargin[inside] Из четырехсторонних фигур, \emph{квадратом} называется та, все стороны которой равны между собой, и~все углы прямые.
\stopalignment
\stopDefinitionOnlyNumber

\defineNewPicture{
	pair A, B, C, D;
	numeric s;
	s := u;
	A := (0, 0);
	B := (s, 0);
	C := A shifted (dir(80) scaled s);
	D := B shifted (dir(80) scaled s);
		byLineDefine(A, B, byred, 0, 0);
		byLineDefine(A, C, byblue, 0, 0);
		byLineDefine(B, D, byyellow, 0, 0);
		byLineDefine(C, D, byblack, 0, 0);
		draw byNamedLineSeq(0)(AB,AC,CD,BD);
}
\startDefinitionOnlyNumber[reference=def:I.XXXI]
\startalignment[last]
\drawCurrentPictureInMargin[inside] \emph{Ромбом} называется четырехсторонняя фигура, все стороны которой равны, но углы не прямые.
\stopalignment
\stopDefinitionOnlyNumber

\defineNewPicture{
	pair A,B,C,D;
	numeric s;
	s := u;
	A := (0, 0);
	B := (4/3s, 0);
	C := (0, 3/4s);
	D := (4/3s, 3/4s);
		byLineDefine(A, B, byblue, 0, 0);
		byLineDefine(A, C, byred, 0, 0);
		byLineDefine(B, D, byred, 0, 0);
		byLineDefine(C, D, byblue, 0, 0);
		draw byNamedLineSeq(0)(AB,AC,CD,BD);
}
\startDefinitionOnlyNumber[reference=def:I.XXXII]
\startalignment[last]
\drawCurrentPictureInMargin[inside] \emph{Прямоугольник} есть четырехугольник, все углы которого прямые, но не все стороны равны.
\stopalignment
\stopDefinitionOnlyNumber

\defineNewPicture{
	pair A, B, C, D;
	numeric s;
	s := u;
	A := (0, 0);
	B := (s, 0);
	C := (1/4s, 3/4s);
	D := (s + 1/4s, 3/4s);
		byLineDefine(A, B, byblue, 0, 0);
		byLineDefine(A, C, byred, 0, 0);
		byLineDefine(B, D, byred, 0, 0);
		byLineDefine(C, D, byblue, 0, 0);
		draw byNamedLineSeq(0)(AB,AC,CD,BD);
}
\startDefinitionOnlyNumber[reference=def:I.XXXIII]
\startalignment[last]
\drawCurrentPictureInMargin[inside] \emph{Ромбоид} есть фигура, противоположные стороны которой равны, но ни все стороны равны, ни все углы прямые.
\stopalignment
\stopDefinitionOnlyNumber

\startDefinitionOnlyNumber[reference=def:I.XXXIV]
\startalignment[last]
Все прочие четырехугольники называются \emph{неправильными.}
\stopalignment
\stopDefinitionOnlyNumber

\defineNewPicture{
	pair A, B, C, D;
	numeric s;
	s := u;
	A := (0, 0);
	B := (4/3s, 0);
	C := (0, 1/2s);
	D := (4/3s, 1/2s);
		draw byLine(A, B, byred, 0, 0);
		draw byLine(C, D, byyellow, 0, 0);
}
\startDefinitionOnlyNumber[reference=def:I.XXXV]
\startalignment[last]
\drawCurrentPictureInMargin[inside] \emph{Параллельными} называются такие прямые линии, которые находясь в~одной плоскости и~будучи продолженными в~обе стороны неограниченно не встречаются.
\stopalignment
\stopDefinitionOnlyNumber
\stopsupersection

\vfill\pagebreak

\startsupersection[title={Постулаты}]

\marginNote{Этими тремя положениями описываются разрешенные к~использованию инструменты: \emph{линейка} и~\emph{циркуль}. При этом линейка не имеет делений. А~циркуль пригоден только для построения окружностей, то есть измерять и откладывать расстояния с его помощью также нельзя.}
\startPostulateOnlyNumber[reference=post:I.I]
Будем считать, что прямую линию можно провести от любой точки до любой другой точки.
\stopPostulateOnlyNumber

\startPostulateOnlyNumber[reference=post:I.II]
Будем считать, что ограниченную прямую линию можно непрерывно продолжить по прямой.
\stopPostulateOnlyNumber

\startPostulateOnlyNumber[reference=post:I.III]
Будем считать, что из всякого центра и~на любом расстоянии от центра можно построить круг.
\stopPostulateOnlyNumber
\stopsupersection

\startsupersection[title={Аксиомы}]

\startAxiomOnlyNumber[reference=ax:I.I]
Равные одному и~тому же величины равны между собой.
\stopAxiomOnlyNumber

\startAxiomOnlyNumber[reference=ax:I.II]
Если к~равным величинам прибавляются равные, то и~суммы будут равны.
\stopAxiomOnlyNumber

\startAxiomOnlyNumber[reference=ax:I.III]
Если из равных величин вычитаются равные, то и~остатки будут равны.
\stopAxiomOnlyNumber

\startAxiomOnlyNumber[reference=ax:I.IV]
Если к~неравным величинам прибавляются равные, то суммы будут неравными.
\stopAxiomOnlyNumber

\startAxiomOnlyNumber[reference=ax:I.V]
Если из неравных величин вычитаются  равные, остатки будут неравными.
\stopAxiomOnlyNumber

\startAxiomOnlyNumber[reference=ax:I.VI]
Удвоенные одной и~той же или равных величин равны.
\stopAxiomOnlyNumber

\startAxiomOnlyNumber[reference=ax:I.VII]
Половины одной и~той же или равных величин равны.
\stopAxiomOnlyNumber

\startAxiomOnlyNumber[reference=ax:I.VIII]
Совпадающие, или занимающие в~точности равное пространство величины равны.
\stopAxiomOnlyNumber

\startAxiomOnlyNumber[reference=ax:I.IX]
Целое больше его части.
\stopAxiomOnlyNumber

\startAxiomOnlyNumber[reference=ax:I.X]
Две прямых линии не содержат пространства.
\stopAxiomOnlyNumber

\marginNote{Аксиомы XI и~XII обычно включаются в~качестве IV и~V в~состав постулатов.}
\startAxiomOnlyNumber[reference=ax:I.XI]
Все прямые углы равны между собой.
\stopAxiomOnlyNumber

\startAxiomOnlyNumber[reference=ax:I.XII]
\defineNewPicture{
	pair A, B, C, D, E, F, G, H;
	numeric s;
	s := 3/2u;
	A := (0, 0);
	B := (4/3s, 0);
	C := (0, s);
	D := (4/3s, s);
	E := (1/3s, 8/6s);
	F := (xpart(E), -2/6s);
	G = whatever[A, B] = whatever[E, F];
	H = whatever[C, D] = whatever[E, F];
		byAngleDefine(B, G, E, byred, 0);
		byAngleDefine(D, H, F, byyellow, 0);
		draw byNamedAngleResized();
		draw byLine(A, B, byblue, 0, 0);
		draw byLine(C, D, byred, 0, 0);
		draw byLine(E, F, byblack, 0, 0);
		draw byLabelLine(0)(AB, CD, EF);
		draw byLabelsOnPolygon(E, H, D)(2, 0);
		draw byLabelsOnPolygon(B, G, F)(2, 0);
}
\drawCurrentPictureInMargin
Если две прямые $\left(\vcenter{\nointerlineskip\hbox{\drawUnitLine{AB}}\nointerlineskip\hbox{\drawUnitLine{CD}}}\right)$ встречаются с~третьей \drawUnitLine{EF} так, что два внутренних угла по одну сторону \drawAngle{H} и~\drawAngle{G} меньше двух прямых углов, эти две линии пересекутся, если их продлить по ту сторону, с~которой углы меньше двух прямых углов.
\stopAxiomOnlyNumber
\stopsupersection

\vfill\pagebreak

\startsupersection[title={Пояснения}]

Двенадцатую аксиому можно сформулировать любым из следующих способов:
\startitemize[m,joinedup,nowhite]
\item Две расходящиеся прямых линии не могут быть параллельны одной и~той же прямой линии.
\item Если прямая линия пересекает одну из двух параллельных прямых линий, то она пересекает и~вторую.
\item Через данную точку может быть проведена лишь одна прямая, параллельная данной.
\stopitemize
Основными предметами геометрии являются изложение и~объяснение свойств \emph{фигур}, а~фигура определяется как отношение, существующее между  границами пространства. Пространство же или величина бывает трех видов: \emph{линейного}, \emph{поверхностного}, и~\emph{телесного}.

\defineNewPicture{
	pair A, B, C;
	numeric s;
	s := 3/2u;
	A := (0, s);
	B := (1/2s, 0);
	C := B xscaled -1;
		byAngleDefine(B, A, C, byyellow, 0);
		draw byNamedAngleResized();
		byLineDefine(B, A, byblue, 0, 0);
		byLineDefine(C, A, byred, 0, 0);
		draw byNamedLineSeq(0)(CA,BA);
		label.urt(\sometxt{A}, A);
}\drawCurrentPictureInMargin
Углы справедливо называть четвертым видом величин. Угловые величины очевидно состоят из частей, и, следовательно, их нужно признать разновидностью количеств. Ученик не должен считать, что величина угла зависит от длин прямых линий, заключающих его, и~мерой взаимного расхождения которых он является. \emph{Вершиной} угла называется точка, в~которой встречаются \emph{стороны} угла, как, например, A.

\defineNewPicture{
	pair B, C, D, E, F, G, H;
	numeric s;
	s := 5/4u;
	C := (0, 0);
	B := dir(0)*s;
	D := dir(50)*s;
	E := dir(-30)*s;
	F := E scaled -1;
	G := D scaled -1;
	H := B scaled -1;
	angleScale := 4/3;
		draw byAngle(E, C, B, byyellow, 0);
		draw byAngle(B, C, D, byblack, 0);
		draw byAngle(D, C, F, byblue, 0);
		draw byAngle(F, C, H, byred, 0);
		draw byAngle(H, C, G, byyellow, 1);
		draw byAngle(G, C, E, byblue, 1);
		draw byLine(B, H, byblue, 0, 0);
		draw byLine(D, G, byred, 0, 0);
		draw byLine(E, F, byblack, 0, 0);
		label.bot(\sometxt{C}, C shifted (0, -3pt));
		label.bot(\sometxt{B}, B);
		label.lrt(\sometxt{D}, D);
		label.llft(\sometxt{F}, F);
		label.bot(\sometxt{H}, H);
		label.lrt(\sometxt{G}, G);
		label.llft(\sometxt{E}, E);
}
\drawCurrentPictureInMargin
Угол часто обозначается одной буквой, когда в~вершине сходятся только его стороны. Так, красная и~синяя линия образуют желтый угол, который иначе назывался бы углом A. Но если более двух линий встречаются в~одной точке, с~использованием старых методов во избежание путаницы было бы необходимо использовать три буквы, чтобы обозначить угол при этой точке, так что буква, обозначающая вершину, всегда помещалась бы посередине. Так, черная и~красная линия, встречающиеся в~точке C, образуют синий угол, и~обычным способом он бы обозначался как угол FCD или DCF. Линии FC и~CD, стороны угла, а~точка C~— его вершина. Таким же образом черный  угол обозначался бы DCB или BCD. Красный и~синий углы вместе, или угол HCD вместе с~FCD, составляли бы угол HCD, и~так далее для других углов.

\defineNewPicture{
	textLabels := false;
	pair B, C, D, E, F, G, H;
	numeric s;
	s := u;
	C := (0, 0);
	B := dir(15)*s;
	E := dir(-15)*s;
	F := E scaled -1;
	H := B scaled -1;
		draw byAngle(E, C, B, byyellow, 0);
		draw byAngle(F, C, H, byred, 0);
		draw byLine(B, H, byblack, 0, 1);
		draw byLine(E, F, byblack, 0, 1);
} % в оригинале отдельной иллюстрации не было
Когда стороны угла продлеваются далее вершины, углы, образованные ими по обе стороны вершины называются \emph{вертикально противоположными} друг другу: \drawCurrentPictureInMargin так \drawAngle{FCH} и \drawAngle{ECB} вертикально противоположны.

\emph{Совмещением} называют действие, когда одну величину можно представить помещенной поверх другой так, что она точно покрывает ее, или так, что все их части в~точности совпадают.

Линию называют \emph{продленной}, когда она вытягивается, продолжается или увеличивается в~длине, а~увеличение в~длине, которое получает линия, называется \emph{продленной частью} или ее \emph{продолжением}.

Полная длина линии или линий, заключающих фигуру, называется ее \emph{периметром}. %Первые шесть книг Евклида касаются только плоских фигур. это было в оригинале, но, вероятно, не нужно.
Линия, проведенная из центра круга к~окружности, называется \emph{радиусом}. Сторона прямоугольного треугольника, противолежащая прямому углу, называется \emph{гипотенузой}. % oblong=rectangle, по русски это не нужно
Все линии, рассматриваемые в~первых шести книгах, считаются лежащими в~одной плоскости.

В евклидовой геометрии разрешается пользоваться только \emph{линейкой} и~\emph{циркулем}. Постулаты призваны донести это ограничение.

\emph{Аксиомы} геометрии~— это некоторые общие предложения, истинность которых принимается как самоочевидная, и~которую нельзя показать с~помощью демонстрации.

\emph{Предложения}~— это результаты в~геометрии, полученные через процесс рассуждения. В~геометрии есть два типа предложений, \emph{задачи} и~\emph{теоремы}.

\emph{Задача}~— это предложение, в~котором предлагается что-либо сделать, например провести линию, соответствующую данным условиям, описать круг, построить фигуру и~т. п.

\emph{Решение} задачи состоит в~том, чтобы показать, как можно сделать требуемое с~помощью линейки и~циркуля.

\emph{Демонстрация} состоит в~доказательстве того, что действия, представленные в~решении, позволяют достичь требуемого.

\emph{Теорема}~— это предложение, в~котором требуется доказать истинность некоего утверждения. Это утверждение должно быть выведено из аксиом и~определений или ранее и~независимо установленных истин. Это и~является целью демонстрации.

\emph{Задача} подобна постулату.

\emph{Теорема} напоминает аксиому.

\emph{Постулат}~— это задача, решение которой предопределено.

\emph{Аксиома}~— это теорема, истинность которой принимается как данность, без демонстрации.

\emph{Следствие}~— это умозаключение, выводимое непосредственно из предложения.

\emph{Примечание}~— это заметка или наблюдение, касающееся предложения, не содержащее умозаключений достаточной важности, чтобы называться \emph{следствием}.

\emph{Лемма}~— это предложение, введенное исключительно с~целью доказательства более важного предложения.

\stopsupersection

\vfill\pagebreak

\startsupersection[title={Обозначения}]

\symb{$\therefore$}
 обозначает слово \emph{следовательно}.

\symb{$\because$}
 обозначает слово \emph{поскольку}.

\symb{$=$}
 обозначает слово \emph{равно}. Знак равенства можно читать как \emph{равно} или \emph{равны}, род и~число на геометрическую строгость не влияют.

\symb{$\neq$}
 обозначает то же, как если бы было написано \emph{не равно}.

\symb{$>$}
 обозначает \emph{больше чем}.

\symb{$<$}
 обозначает \emph{меньше чем}.

\symb{$\ngtr$}
 обозначает \emph{не больше чем}.

\symb{$\nless$}
 обозначает \emph{не меньше чем}.

\symb{$+$}
 читается как \emph{плюс} и~обозначает сложение; будучи помещенным между двумя и~более величинами, обозначает их сумму.

\symb{$-$}
 читается как \emph{минус} и~обозначает вычитание; будучи расположенным между двумя количествами, указывает на то, что последнее вычитается из первого.

\symb{$\times$}
 этот символ обозначает произведение двух или более чисел, будучи помещенным между ними в~арифметике или алгебре. Но в~геометрии он обычно используется для обозначения \emph{прямоугольника}, когда помещен между \quotation{двумя прямыми линиями, содержащими один из его прямых углов}. \emph{Прямоугольник} может также обозначаться точкой между двумя смежными сторонами.

\symb{$:\ ::\ :$}
 обозначает \emph{аналогию} или \emph{пропорцию}. Так, если A, B, C и~D представляют четыре величины и~A имеет к~B такое же отношение, как C к~D, то пропорция кратко записывается следующим образом:

$A : B :: C : D$, $A : B = C : D$, или $\dfrac{A}{B} = \dfrac{C}{D}$.

Это равенство или одинаковость отношений читается: 

\emph{как A к~B, так и~C к~D;} или \emph{A к~B, как C к~D.}

\symb{$\parallel$}
 обозначает \emph{параллельно к}.

\symb{$\perp$}
 обозначает \emph{перпендикулярно к}.

\defineNewPicture{
	pair A, B, C, D;
	numeric s;
	s := 3/2u;
	A := (0, 0);
	B := dir(0)*s;
	C := dir(50)*s;
	D := dir(90)*s;
	byAngleDefine(B, A, C, byblack, 1);
	byAngleDefine(B, A, D, byblack, 1);
	byPointLabelRemove(A, B, C, D);
}

\symb{\drawAngle{BAC}}
 обозначает \emph{угол}.

\symb{\drawAngle{BAD}}
 обозначает \emph{прямой угол}.

\symb{\drawTwoRightAngles}
 обозначает \emph{два прямых угла}.


\defineNewPicture{
	pair A, B, C, D;
	A := (0, -1/4u);
	B := (u, 0);
	C := (-u, 0);
	D := (0, u);
		byLineDefine (A, D, byblack, 0, 0);
		byLineDefine (B, D, byblack, 0, 0);
		byLineDefine (C, D, byblack, 0, 0);
	byPointLabelRemove(A, D);
}

\symb{\drawFromCurrentPicture{
draw byNamedLine(AD);
draw byNamedLineSeq(0)(BD,CD);
}
или
\drawFromCurrentPicture{
draw byNamedLineSeq(0)(AD,BD);
}}
кратко обозначает \emph{точку}.

Квадрат, построенный на линии, кратко записывается так: $\drawUnitLine{AD}^2$.

Таким же образом дважды квадрат обозначается так: $2 \cdot \drawUnitLine{AD}^2$.

\symb{\indefstr}
 обозначает \emph{определение}.

\symb{\inpoststr}
 обозначает \emph{постулат}.

\symb{\inaxstr}
 обозначает \emph{аксиому}.

\symb{\hypstr}
 обозначает \emph{гипотезу}. Здесь важно отметить, что \emph{гипотеза} — это условие, которое принимается как данное. Так, гипотеза предложения, данного во введении, в~том, что треугольник равнобедренный, или что две его стороны равны.

\symb{\conststr}
 обозначает \emph{построение}.  \emph{Построения} — это изменения, сделанные в~исходном изображении добавлением линий, углов, кругов и~т. п. с~целью приспособления его к~демонстрации или решению задачи. Условия, при которых сделаны эти изменения так же бесспорны, как и~содержащиеся в~гипотезе. Например, если мы делаем угол равным данному углу, то эти два угла равны по построению.

\symb{\qedstr}
 обозначает \emph{что и~требовалось доказать}.
\stopsupersection

\stopintro

\startBook[title={Книга I}]
\startVerboseProposition[title={Предл. I. Задача}, reference=prop:I.I]

\defineNewPicture[1/2]{
	pair A, B, C;
	path P[];
	numeric r;
	r := 3/2u;
	A := (0, 0);
	B := (r, 0);
	P1 := fullcircle scaled 2r;
	P2 := fullcircle scaled 2r shifted B;
	C := P1 intersectionpoint P2;
		byLineDefine(A, B, byblack, 0, 0);
		byLineDefine(B, C, byred, 0, 0);
		byLineDefine(C, A, byyellow, 0, 0);
		draw byNamedLineSeq(1)(AB,CA,BC);
		draw byCircle.A(A, B, byblue, 0, 0, 1/2);
		draw byCircle.B(B, A, byred, 0, 0, 1/2);
		draw byLabelsOnPolygon(A, C, B)(0, -1);
}
\drawCurrentPictureInMargin
\problemNP{Н}{а}{данной ограниченной прямой \drawUnitLine{AB} построить равносторонний треугольник.}

\startCenterAlign
Опишем \offsetPicture{15pt}{0pt}{\drawFromCurrentPicture{
draw byNamedLine(AB);
draw byNamedCircle(A);
draw byLabelLineEnd(A, B, 0);
draw byLabelLineEnd(B, A, 1);
}} и~\offsetPicture{15pt}{0pt}{\drawFromCurrentPicture{
draw byNamedLine(AB);
draw byNamedCircle(B);
draw byLabelLineEnd(A, B, 1);
draw byLabelLineEnd(B, A, 0);
}}
\inpost[post:I.III].

Проведем \drawUnitLine{CA} и~\drawUnitLine{BC} \inpost[post:I.I].\\
Тогда \drawLine[bottom][triangleABC]{AB,CA,BC} равносторонний.

Поскольку $\drawUnitLine{AB} = \drawUnitLine{CA}$ \indef[def:I.XV]\\
и $\drawUnitLine{AB} = \drawUnitLine{BC}$ \indef[def:I.XV],
$\therefore \drawUnitLine{CA} = \drawUnitLine{BC}$ \inax[ax:I.I],\\
и значит, \triangleABC\ и~есть искомый треугольник.
\stopCenterAlign

\qed
\stopVerboseProposition

\startProposition[title={Предл. II. Задача}, reference=prop:I.II]
\defineNewPicture{
pair A, B, C, D, E, F;
path P[];
numeric r[], a;
A := (0, 0);
B := (-3/5u, -3/5u);
C := (-2u, -1/3u);
a := angle(B-C);
forsuffixes i := A, B, C:
	i := i rotated -a;
endfor; 
r1 := abs(A-B);
D := (fullcircle scaled 2r1 shifted A) intersectionpoint (fullcircle scaled 2r1 shifted B);
r2 := abs(B-C);
r3 := r1 + r2;
P1 := fullcircle scaled 2r2 shifted B;
P2 := fullcircle scaled 2r3 shifted D;
E := (D -- 10[D, B]) intersectionpoint P1;
F := (D -- 10[D, A]) intersectionpoint P2;
byLineDefine(A, B, byblack, 1, 0);
byLineDefine(B, C, byblack, 0, 0);
byLineDefine(B, D, byred, 1, 0);
byLineDefine(D, A, byred, 0, 0);
byLineDefine(B, E, byyellow, 0, 0);
byLineDefine(A, F, byblue, 0, 0);
draw byNamedLineSeq(0)(BE,BD,DA,AF);
draw byNamedLineSeq(0)(AB,BC);
draw byCircle.A(D, E, byred, 0, 0, 1/2);
draw byCircle.B(B, C, byblue, 0, 0, -1/2);
draw byLabelsOnPolygon(E, D, A, F)(2, -1);
draw byLabelsOnPolygon(E, B, C)(2, -1);
draw byLabelsOnCircle(C)(B);
draw byLabelsOnCircle(E, F)(A);
}
\drawCurrentPictureInMargin
\problemNP{О}{т}{данной точки \drawFromCurrentPicture[middle][pointA]{
startGlobalRotation(-lineAngle.DA);
draw byNamedPointLines(A,"AB");
stopGlobalRotation;
} отложить прямую, равную данной прямой \drawUnitLine{BC}.}

\startCenterAlign
Проведем \drawUnitLine{AB} \inpost[post:I.I], построим \drawFromCurrentPicture[middle]{
startAutoLabeling;
startTempScale(scaleFactor*3);
draw byNamedLineSeq(0)(AB,BD,DA);
stopTempScale;
stopAutoLabeling;
} \inprop[prop:I.I],\\
продлим \drawUnitLine{BD} \inpost[post:I.II],\\
опишем
\drawFromCurrentPicture{
draw byNamedLine (BC);
draw byNamedCircle(B);
draw byLabelLineEnd(B, C, 0);
draw byLabelLineEnd(C, B, 0);
}
\inpost[post:I.III] и
\drawFromCurrentPicture{
draw byNamedLine (BD, BE);
draw byNamedCircle(A);
draw byLabelLineEnd(D, E, 0);
draw byLabelLineEnd(E, D, 1);
}
\inpost[post:I.III].

Продлим \drawUnitLine{DA} \inpost[post:I.II],\\
тогда искомая прямая~— это \drawUnitLine{AF}.

Поскольку $\drawUnitLine{BE,BD} = \drawUnitLine{DA,AF}$ \indef[def:I.XV]\\
и $\drawUnitLine{BD} = \drawUnitLine{DA}$ (\conststr),\\
$\therefore \drawUnitLine{BE} = \drawUnitLine{AF}$ \inax[ax:I.III],\\
но \indef[def:I.XV] $\drawUnitLine{BC} = \drawUnitLine{BE} = \drawUnitLine{AF}$.

$\therefore \drawUnitLine{AF}$, проведенная из данной точки \pointA, равна данной прямой \drawUnitLine{BC} \inax[ax:I.I].
\stopCenterAlign

\qed
\stopProposition

\startProposition[title={Предл. III. Задача}, reference=prop:I.III]
\defineNewPicture{
pair A, B, C, D, E, F;
path P;
numeric r;
A := (0, 0);
r := 7/4u;
B := A shifted (r, 0);
C := A shifted (4/3r, 0);
D := A shifted dir(30)*r;
E := A shifted (7/6r, -1/6r);
F := A shifted (7/6r, -7/6r);
byLineDefine(A, B, byblack, 0, 0);
byLineDefine(B, C, byblack, 1, 0);
byLineDefine(A, D, byred, 0, 0);
draw byNamedLineSeq(0)(BC,AB,AD);
draw byLine(E, F, byblue, 0, 0);
draw byCircle.A(A, D, byblue, 0, 0, 0);
draw byLabelsOnPolygon(B, A, D)(2, -1);
draw byLabelLineEnd(D, A, 0);
draw byLabelLineEnd(C, A, 0);
draw byLabelPoint(B, angle(B-A) + 45, 2);
draw byLabelsOnPolygon(E, F)(0, 0);
}
\drawCurrentPictureInMargin
\problemNP{О}{т}{большей \drawUnitLine{AB,BC}  из двух данных прямых отнять прямую, равную меньшей \drawUnitLine{EF}.}

\startCenterAlign
Проведем $\drawUnitLine{AD} = \drawUnitLine{EF}$ \inprop[prop:I.II].

Опишем
\drawFromCurrentPicture{
draw byNamedLine (AD);
draw byNamedCircle(A);
draw byLabelLineEnd(D, A, 0);
draw byLabelLineEnd(A, D, 0);
} \inpost[post:I.III].

Тогда $\drawUnitLine{EF} = \drawUnitLine{AB}$.

Поскольку $\drawUnitLine{AD} = \drawUnitLine{AB}$ \indef[def:I.XV]\\
и $\drawUnitLine{EF} = \drawUnitLine{AD}$ (\conststr).

$\therefore \drawUnitLine{EF} = \drawUnitLine{AB}$ \inax[ax:I.I].
\stopCenterAlign

\qed
\stopProposition

\startProposition[title={Предл. IV. Теорема}, reference=prop:I.IV]
\defineNewPicture[1/4]{
pair A, B, C, D, E, F, d;
A := (0, 0);
B := A shifted (-5/2u, -7/2u);
C := A shifted (1/2u, -3u);
d := (0, -4u);
D := A shifted d;
E := B shifted d;
F := C shifted d;
byAngleDefine(B, A, C, byyellow, 0);
byAngleDefine(A, B, C, byblue, 0);
byAngleDefine(B, C, A, byred, 0);
draw byNamedAngleResized(BAC, ABC, BCA);
byLineDefine(A, B, byred, 0, 0);
byLineDefine(B, C, byblack, 0, 0);
byLineDefine(C, A, byblue, 0, 0);
draw byNamedLineSeq(0)(CA,BC,AB);
byAngleDefine(E, D, F, byyellow, 0);
byAngleDefine(D, E, F, byblue, 0);
byAngleDefine(E, F, D, byred, 0);
draw byNamedAngleResized(EDF, DEF, EFD);
byLineDefine(D, E, byred, 0, 1);
byLineDefine(E, F, byblack, 0, 1);
byLineDefine(F, D, byblue, 0, 1);
draw byNamedLineSeq(0)(FD,EF,DE);
draw byLabelsOnPolygon(F, E, D)(0, 0);
draw byLabelsOnPolygon(B, A, C)(0, -1);
}
\drawCurrentPictureInMargin
\problemNP[3]{Е}{сли}{два треугольника имеют по две стороны, равные каждая каждой ($\drawUnitLine{AB} = \drawUnitLine{DE}$ и~$\drawUnitLine{CA} = \drawUnitLine{FD}$), и~по равному углу ($\drawAngle{A} = \drawAngle{D}$), содержащемуся между равными прямыми, то они будут иметь и~основание, равное основанию ($\drawUnitLine{BC} = \drawUnitLine{EF}$), и~один треугольник будет равен другому, и~остальные углы, стягиваемые равными сторонами, будут равны каждый каждому ($\drawAngle{B} = \drawAngle{E}$ и~$\drawAngle{C} = \drawAngle{F}$).}

\startCenterAlign
Представим, что два треугольника расположены таким образом, что вершина одного из двух равных углов, \drawAngle{A} или \drawAngle{D}, совпадает с~вершиной другого\\ 
и~\drawUnitLine{AB} совпадает с~\drawUnitLine{DE}.

Тогда \drawUnitLine{CA} при наложении совпадет с~\drawUnitLine{FD}.

$\therefore$ \drawUnitLine{BC} совпадает с~\drawUnitLine{EF},\\ 
или же две прямые будут содержать пространство, что невозможно \inax[ax:I.X].

$\therefore \drawUnitLine{BC} = \drawUnitLine{EF}$,\\ 
$\drawAngle{B} = \drawAngle{E}$ и~$\drawAngle{C} = \drawAngle{F}$.

$\therefore$ треугольники \drawLine{CA,BC,AB} и~\drawLine{FD,EF,DE}\\ совпадают при наложении.

$\therefore$ они равны во всех отношениях.
\stopCenterAlign

\qed
\stopProposition

\startProposition[title={Предл. V. Теорема}, reference=prop:I.V]
\defineNewPicture[1/4]{
pair A, B, C, D, E;
A := (0, 0);
B := A shifted (u, -2u);
C := B xscaled -1;
D := 9/5[A,B];
E := 9/5[A,C];
byAngleDefine(B, A, C, byblack, 0);
byAngleDefine(A, B, C, byblue, 0);
byAngleDefine(B, C, A, byblue, 0);
byAngleDefine(C, B, E, byyellow, 0);
byAngleDefine(D, C, B, byyellow, 0);
byAngleDefine(B, D, C, byred, 0);
byAngleDefine(C, E, B, byred, 0);
byAngleDefine(E, B, D, byyellow, 1);
byAngleDefine(D, C, E, byyellow, 1);
draw byNamedAngleResized(BAC,ABC,BCA,CBE,DCB,BDC,CEB,EBD,DCE);
byLineDefine(B, D, byyellow, 0, 0);
byLineDefine(C, E, byyellow, 0, 0);
byLineDefine(B, E, byblue, 0, 0);
byLineDefine(C, D, byblue, 0, 0);
byLineDefine(A, B, byred, 0, 0);
byLineDefine(A, C, byred, 0, 0);
byLineDefine(B, C, byblack, 0, 0);
draw byNamedLineSeq(0)(CD,noLine,BC,noLine,BE,CE,AC,AB,BD);
draw byLabelsOnPolygon(E, C, A, B, D, C, B)(0, 0);
}
\drawCurrentPictureInMargin
\problemNP[3]{У}{глы}{при основании любого равнобедренного треугольника \drawLine[bottom]{BC,AC,AB} равны между собой, и~по продолжении равных сторон углы под основанием будут равны между собой.}

\startCenterAlign
Продлим \drawUnitLine{AB} и~\drawUnitLine{AC} \inpost[post:I.II],\\
возьмем $\drawUnitLine{BD} = \drawUnitLine{CE}$ \inprop[prop:I.III],\\
проведем \drawUnitLine{BE} и~\drawUnitLine{CD}.

Тогда в
\drawFromCurrentPicture{
startAutoLabeling;
startTempAngleScale(angleScale*4/5);
draw byNamedAngle(BAC);
draw byNamedLineSeq(0)(BE,CE,AC,AB);
stopTempAngleScale;
stopAutoLabeling;
}
и
\drawFromCurrentPicture{
startAutoLabeling;
startTempAngleScale(angleScale*4/5);
draw byNamedAngle(BAC);
draw byNamedLineSeq(0)(BD,CD,AC,AB);
stopTempAngleScale;
stopAutoLabeling;
}\\
получим $\drawUnitLine{AB,BD} = \drawUnitLine{AC,CE}$ (конст.),\\
\drawAngle{BAC} общий обоим,\\
и $\drawUnitLine{AB} = \drawUnitLine{AC}$ (\hypstr).

$\therefore \drawAngle{BCA,DCB} = \drawAngle{ABC,CBE}$, $\drawUnitLine{BE} = \drawUnitLine{CD}$ и~$\drawAngle{CEB} = \drawAngle{BDC}$ \inprop[prop:I.IV].

Так же у~\drawFromCurrentPicture{
startAutoLabeling;
startTempAngleScale(angleScale*4/5);
draw byNamedAngle(E);
draw byNamedLineSeq(0)(BE,CE,BC);
stopTempAngleScale;
stopAutoLabeling;
} и~\drawFromCurrentPicture{
startAutoLabeling;
startTempAngleScale(angleScale*4/5);
draw byNamedAngle(D);
draw byNamedLineSeq(0)(BD,CD,BC);
stopTempAngleScale;
stopAutoLabeling;
}\\
получим $\drawUnitLine{BD} = \drawUnitLine{CE}$, $\drawAngle{CEB} = \drawAngle{BDC}$ и~$\drawUnitLine{BE} = \drawUnitLine{CD}$,\\
$\therefore \drawAngle{DCE,DCB} = \drawAngle{EBD,CBE}$ и~$\drawAngle{DCB} = \drawAngle{CBE}$ \inprop[prop:I.IV],\\
но $\drawAngle{BCA,DCB} = \drawAngle{ABC,CBE}$, $\therefore \drawAngle{BCA} = \drawAngle{ABC}$ \inax[ax:I.III].
\stopCenterAlign

\qed
\stopProposition

\startProposition[title={Предл. VI. Теорема}, reference=prop:I.VI]
\defineNewPicture[1/4]{
pair A, B, C, D;
A := (0, 0);
B := A shifted (7/2u, 0);
D := A shifted (7/4u, 3u);
C := 2/3[A, D];
byAngleDefine(B, A, D, byyellow, 0);
byAngleDefine(A, B, D, byblack, 0);
draw byNamedAngleResized();
byLineDefine(B, C, byyellow, 0, 0);
byLineDefine(A, B, byred, 0, 0);
byLineDefine(B, D, byblue, 0, 0);
byLineDefine(C, A, byblack, 0, 0);
byLineDefine(C, D, byblack, 1, 0);
draw byNamedLine(BC);
draw byNamedLineSeq(0)(CA,CD,BD,AB);
draw byLabelsOnPolygon(A, C, D, B)(0, 0);
}
\drawCurrentPictureInMargin
\problemNP[3]{Е}{сли}{у любого треугольника \drawLine[bottom][triangleABD]{CA,CD,BD,AB} два угла \drawAngle{A} и~\drawAngle{B} равны между собой, то и~стороны \drawUnitLine{CA,CD} и~\drawUnitLine{BD}, стягивающие равные углы, будут равны.}

Предположим, что стороны не равны и~одна из них \drawUnitLine{CA,CD} больше, чем другая \drawUnitLine{BD}, тогда отрежем от нее $\drawUnitLine{CA} = \drawUnitLine{BD}$ \inprop[prop:I.III] и~проведем \drawUnitLine{BC}.

\startCenterAlign
Тогда в~\drawLine[bottom]{BC,AB,CA} и~\triangleABD\\
$\drawUnitLine{CA} = \drawUnitLine{BD}$ (\conststr),\\
$\drawAngle{A} = \drawAngle{B}$ (\hypstr)\\
и \drawUnitLine{AB} общая обоим.

$\therefore$ эти треугольники равны \inprop[prop:I.IV],\\
часть равна целому, что невозможно.

$\therefore$ ни одна из сторон \drawUnitLine{CA,CD} или \drawUnitLine{BD} не больше другой,\\
$\therefore$ они равны.
\stopCenterAlign

\qed
\stopProposition

\startProposition[title={Предл. VII. Теорема}, reference=prop:I.VII]
\defineNewPicture{
pair A, B, C, D, E, F, G, H;
A := (0, 0);
B := A shifted (4u, 0);
C := A shifted (u, 3u);
D := C shifted (7/4u, 0);
E := 1/2[C, D] yscaled -0.7;
F := E shifted (0, -2u);
G := 5/4[A, E];
H := 5/4[A, F];
byAngleDefine.C(B, C, A, byblack, 0);
byAngleDefine(D, C, B, byred, 0);
byAngleDefine.D(A, D, B, byyellow, 0);
byAngleDefine(C, D, A, byblue, 0);
byAngleDefine(B, F, H, byblack, 0);
byAngleDefine(B, F, E, byred, 0);
byAngleDefine(B, E, G, byyellow, 0);
byAngleDefine(G, E, F, byblue, 0);
draw byNamedAngleResized();
draw byLine(C, D, byblack, 1, 0);
draw byLine(E, F, byblack, 1, 0);
draw byLine(A, B, byblack, 0, 0);
byLineDefine(B, C, byblue, 0, 0);
byLineDefine(C, A, byred, 0, 0);
byLineDefine(B, D, byblue, 0, 0);
byLineDefine(D, A, byred, 0, 0);
byLineDefine(B, E, byblue, 0, 0);
byLineDefine(E, A, byred, 0, 0);
byLineDefine(B, F, byblue, 0, 0);
byLineDefine(F, A, byred, 0, 0);
byLineDefine(E, G, byred, 1, 0);
byLineDefine(F, H, byred, 1, 0);
draw byNamedLine(EG,FH);
draw byNamedLineSeq(0)(BC,CA,EA,BE);
draw byNamedLineSeq(0)(BD,DA,FA,BF);
byPointLabelDefine(F, "C");
byPointLabelDefine(E, "D");
draw byLabelsOnPolygon(F, A, C, D, B, F, noPoint)(2, 0);
draw byLabelsOnPolygon(A, E, B)(2, 0);
draw byLabelsOnPolygon(H, F, A)(2, 0);
}
\drawCurrentPictureInMargin
\problemNP{П}{о}{одну сторону одной и~той же прямой \drawUnitLine{AB} нельзя построить два разных треугольника с~равными друг другу смежными сторонами $\drawUnitLine{CA} = \drawUnitLine{DA}$ и~$\drawUnitLine{BC} = \drawUnitLine{BD}$.}

Если два треугольника построены на одном основании и~по одну сторону от него, то вершина одного может находиться вовне другого, внутри или на одной из его сторон.

\startCenterAlign
Если такое возможно, то построим два треугольника таких, что $\left\{\eqalign{\drawUnitLine{CA}&=\drawUnitLine{DA}\cr \drawUnitLine{BC}&=\drawUnitLine{BD}\cr}\right\}$,\\
затем проведем \drawUnitLine{CD}, тогда\\
$\drawAngle{C,DCB} = \drawAngle{CDA}$ \inprop[prop:I.V].

$\therefore\drawAngle{DCB} < \drawAngle{CDA}$.

$\left.
\eqalign{
\mbox{И } \therefore\drawAngle{DCB} &< \drawAngle{CDA,D}\mbox{,}\cr
\mbox{но \inprop[prop:I.V]} \drawAngle{DCB} &= \drawAngle{CDA,D}
}\right\}\mbox{, что невозможно.}$
\stopCenterAlign

\noindent Следовательно, смежные стороны таких двух треугольников не могут быть равны.

\qed
\stopProposition

\startProposition[title={Предл. VIII. Теорема}, reference=prop:I.VIII]
\defineNewPicture{
pair A, B, C, D, E, F, d;
A := (0, 0);
B := A shifted (-u, -4u);
C := A shifted (3/2u, -3u);
d := (0, -9/2u);
D := A shifted d;
E := B shifted d;
F := C shifted d;
byAngleDefine(F, D, E, byblack, 0);
byAngleDefine(C, A, B, byblack, 0);
draw byNamedAngleResized();
byLineDefine(A, B, byred, 0, 0);
byLineDefine(B, C, byblack, 0, 0);
byLineDefine(C, A, byblue, 0, 0);
byLineDefine(D, E, byred, 0, 1);
byLineDefine(E, F, byblack, 0, 1);
byLineDefine(F, D, byblue, 0, 1);
draw byNamedLineSeq(0)(CA,BC,AB);
draw byNamedLineSeq(0)(FD,EF,DE);
draw byLabelsOnPolygon(C, B, A)(0, 0);
draw byLabelsOnPolygon(F, E, D)(0, 0);
}
\drawCurrentPictureInMargin
\problemNP{Е}{сли}{у двух треугольников по две попарно равных стороны ($\drawUnitLine{CA} = \drawUnitLine{FD}$ и~$\drawUnitLine{AB} = \drawUnitLine{DE}$), а~также равные основания ($\drawUnitLine{BC} = \drawUnitLine{EF}$), то углы \drawAngle{A} и~\drawAngle{D}, заключенные между равными сторонами, равны.}

\startCenterAlign
Если совместить равные основания \drawUnitLine{BC} и~\drawUnitLine{EF} так, чтобы треугольники находились по одну сторону,\\ 
а~их равные стороны \drawUnitLine{AB} и~\drawUnitLine{DE}, \drawUnitLine{CA} и~\drawUnitLine{FD} были смежными,\\
вершина одного будет совпадать с~вершиной другого \inprop[prop:I.VII].

$\therefore$ стороны \drawUnitLine{AB} и~\drawUnitLine{CA}\\
будут совпадать с~\drawUnitLine{DE} и~\drawUnitLine{FD}\\
и~$\therefore \drawAngle{A} = \drawAngle{D}$.
\stopCenterAlign

\qed
\stopProposition

\startProposition[title={Предл. IX. Задача}, reference=prop:I.IX]
\defineNewPicture{
pair A, B, C, D, E, F;
A := (0, 5/3u);
B := (-4/3u, 0);
C := B xscaled -1;
D = whatever[B, B shifted ((C-B) rotated -60)] = whatever[C, C shifted ((B-C) rotated 60)];
E := 5/4[A, B];
F := 5/4[A, C];
byAngleDefine(B, A, D, byblue, 0);
byAngleDefine(C, A, D, byyellow, 0);
draw byNamedAngleResized();
byLineDefine(B, C, byyellow, 0, 0);
byLineDefine(A, D, byblack, 0, 0);
byLineDefine(D, B, byblue, 0, 0);
byLineDefine(C, D, byblue, 0, 0);
byLineDefine(A, B, byred, 0, 0);
byLineDefine(C, A, byred, 0, 0);
byLineDefine(B, E, byred, 1, 0);
byLineDefine(C, F, byred, 1, 0);
draw byNamedLine(BC,AD);
draw byNamedLineSeq(0)(DB,CD);
draw byNamedLineSeq(0)(BE,AB,CA,CF);
draw byLabelsOnPolygon(D, B, A, C)(0, 0);
}
\drawCurrentPictureInMargin
\problemNP{Р}{ассечь}{данный прямолинейный угол \drawAngle{BAD,CAD} пополам.}

\startCenterAlign
Возьмем $\drawUnitLine{AB} = \drawUnitLine{CA}$ \inprop[prop:I.III].

Проведем \drawUnitLine{BC}, на которой построим \drawLine{CD,DB,BC} \inprop[prop:I.I],\\
проведем \drawUnitLine{AD}.

Поскольку в \drawLine[middle]{DB,BA,AD} и \drawLine[middle]{CD,DA,AC}\\
$\drawUnitLine{AB} = \drawUnitLine{CA}$ (\conststr),\\
$\drawUnitLine{CD} = \drawUnitLine{DB}$ (\conststr)\\
и \drawUnitLine{AD} общая обоим,\\
$\therefore \drawAngle{BAD} = \drawAngle{CAD}$ \inprop[prop:I.VIII].
\stopCenterAlign

\qed
\stopProposition

\startProposition[title={Предл. X. Задача}, reference=prop:I.X]
\defineNewPicture{
pair A, B, C, D;
A := (0, 3u);
B := (-7/4u, 0);
C := B xscaled -1;
D := 1/2[B, C];
byAngleDefine(B, A, D, byblue, 0);
byAngleDefine(C, A, D, byyellow, 0);
draw byNamedAngleResized();
draw byLine(A, D, byred, 0, 0);
byLineDefine(D, B, byblack, 0, 0);
byLineDefine(C, D, byblack, 1, 0);
byLineDefine(A, B, byyellow, 0, 0);
byLineDefine(C, A, byblue, 0, 0);
draw byNamedLineSeq(0)(AB,CA,CD,DB);
draw byLabelsOnPolygon(B, A, C, D)(0, 0);
}
\drawCurrentPictureInMargin
\problemNP{Р}{ассечь}{данную ограниченную прямую линию \drawUnitLine{DB,CD} пополам.}

\startCenterAlign
Построим \drawLine[bottom]{AB,CA,CD,DB} \inprop[prop:I.I],\\
проведем \drawUnitLine{AD}, делая $\drawAngle{BAD} = \drawAngle{CAD}$ \inprop[prop:I.IX].

Тогда $\drawUnitLine{BD} = \drawUnitLine{DC}$ \inprop[prop:I.IV],\\
ведь в \drawLine[bottom]{DB,BA,AD} и \drawLine[bottom]{CD,DA,AC}\\
$\drawUnitLine{AB} = \drawUnitLine{AC}$, $\drawAngle{BAD} = \drawAngle{CAD}$ (\conststr)\\
и \drawUnitLine{AD} общая обоим.

Следовательно, данная линия рассечена пополам.
\stopCenterAlign

\qed
\stopProposition

\startProposition[title={Предл. XI. Задача}, reference=prop:I.XI]
\defineNewPicture{
pair A, B, C, D, E, F;
A := (0, 5/2u);
B := (-3/2u, 0);
C := B xscaled -1;
D := 1/2[B, C];
E := 3/2[D, B];
F := 3/2[D, C];
byAngleDefine(A, D, B, byred, 0);
byAngleDefine(C, D, A, byblue, 0);
draw byNamedAngleResized();
draw byLine(A, D, byyellow, 0, 0);
byLineDefine(A, B, byblue, 0, 0);
byLineDefine(C, A, byblue, 0, 0);
draw byNamedLineSeq(0)(AB,CA);
draw byLine(D, B, byblack, 0, 0);
draw byLine(B, E, byblack, 1, 0);
draw byLine(C, D, byred, 0, 0);
draw byLine(F, C, byred, 1, 0);
draw byLabelsOnPolygon(F, C, D, B, E)(2, 0);
draw byLabelsOnPolygon(B, A, C)(2, 0);
}
\drawCurrentPictureInMargin
\problemNP{И}{з}{данной точки \drawPointL[middle][AD]{D} на данной прямой \drawUnitLine{DB,CD} построить перпендикуляр.}

\startCenterAlign
Возьмем любую точку \drawPointL[middle][CA]{C} на данной прямой,\\
отсечем $\drawUnitLine{DB} = \drawUnitLine{CD}$ \inprop[prop:I.III],\\
построим \drawLine[bottom]{AB,CA,CD,DB} \inprop[prop:I.I],\\
проведем \drawUnitLine{AD}, и~она будет перпендикуляром к~данной прямой.

Поскольку в \drawLine[bottom]{DB,BA,AD} и \drawLine[bottom]{CD,DA,AC}
$\drawUnitLine{AB} = \drawUnitLine{CA}$ (\conststr),\\
$\drawUnitLine{CD} = \drawUnitLine{DB}$ (\conststr)\\
и \drawUnitLine{AD} общая обоим,\\
$\therefore \drawAngle{ADB} = \drawAngle{CDA}$ \inprop[prop:I.VIII].

$\therefore \drawUnitLine{AD} \perp \drawUnitLine{DB,CD}$ \indef[def:I.X].
\stopCenterAlign

\qed
\stopProposition


\startProposition[title={Предл. XII. Задача}, reference=prop:I.XII]
\defineNewPicture{
pair A, B, C, D, E, F;
path c;
numeric r, a[];
A := (0, 3u);
B := (-7/4u, 0);
C := B xscaled -1;
D := 1/2[B, C];
E := 4/3[D, B];
F := 4/3[D, C];
r := abs(A-B);
c := fullcircle scaled 2r shifted A;
a1 := xpart(c intersectiontimes (F--1/2[B, C]));
a2 := xpart(c intersectiontimes (E--1/2[B, C]));
byAngleDefine(A, D, B, byyellow, 0);
byAngleDefine(C, D, A, byblue, 0);
draw byNamedAngleResized();
draw byLine(A, D, byred, 0, 0);
byLineDefine(A, B, byblue, 0, 0);
byLineDefine(C, A, byblue, 0, 0);
draw byNamedLineSeq(0)(AB,CA);
draw byArc.O(A, B, C)(r, byred, 0, 0, 0, 0);
draw byArcBE.Ol(A, a2-1/4, a2, r, byred, 1, 0, 0, 0);
draw byArcBE.Or(A, a1, a1+1/4, r, byred, 1, 0, 0, 0);
draw byLine(D, B, byblack, 0, 0);
draw byLine(B, E, byblack, 1, 0);
draw byLine(C, D, byyellow, 0, 0);
draw byLine(F, C, byyellow, 1, 0);
draw byLabelsOnPolygon(B, A, C)(2, 0);
draw byLabelLineEnd(B, A, 0);
draw byLabelLineEnd(D, A, 0);
draw byLabelLineEnd(C, A, 0);
}
\drawCurrentPictureInMargin
\problemNP{П}{ровести}{перпендикуляр к~данной неограниченной прямой \drawUnitLine{DB,CD} из данной не находящейся на ней точки \drawPointL{A}.}

\startCenterAlign
Взяв данную точку \drawPointL{A} в~качестве центра по одну сторону прямой и~любое расстояние, позволяющее достигнуть другой стороны, построим \drawArc{O}.

Возьмем $\drawUnitLine{DB} = \drawUnitLine{CD}$ \inprop[prop:I.X],\\
проведем \drawUnitLine{AB}, \drawUnitLine{CA} и~\drawUnitLine{AD}.

Тогда $\drawUnitLine{AD} \perp \drawUnitLine{DB,CD}$.

Поскольку в \drawLine[bottom]{DB,BA,AD} и \drawLine[bottom]{CD,DA,AC}
$\drawUnitLine{DB} = \drawUnitLine{CD}$ (\conststr),\\
\drawUnitLine{AD} общая обоим\\
и $\drawUnitLine{AB} = \drawUnitLine{CA}$ \indef[def:I.XV],\\
$\therefore \drawAngle{ADB} = \drawAngle{CDA}$\inprop[prop:I.VIII],\\
и $\therefore \drawUnitLine{AD} \perp \drawUnitLine{DB,CD}$ \indef[def:I.X].
\stopCenterAlign

\qed
\stopProposition

\startProposition[title={Предл. XIII. Теорема}, reference=prop:I.XIII]
\defineNewPicture{
pair A, B, C, D, E;
A := (0, 5/2u);
B := (-7/4u, 0);
C := B xscaled -1;
D := (xpart(A), ypart(B));
E := (2/3xpart(C), 2/3ypart(A));
byAngleDefine(A, D, B, byyellow, 0);
byAngleDefine(E, D, A, byred, 0);
byAngleDefine(C, D, E, byblue, 0);
draw byNamedAngleResized();
draw byLine(A, D, byblack, 0, 0);
draw byLine(E, D, byyellow, 0, 0);
draw byLine(B, C, byred, 0, 0);
draw byLabelsOnPolygon(C, D, B, noPoint)(0, 0);
draw byLabelLineEnd(E, D, 0);
draw byLabelLineEnd(A, D, 0);
}
\drawCurrentPictureInMargin
\problemNP[5]{Е}{сли}{прямая линия \drawUnitLine{ED}, восставленная на другой прямой линии \drawUnitLine{BC}, образует с~ней углы, то это будут либо два прямых угла, либо их сумма будет равна двум прямым углам.}

\startCenterAlign
Если $\drawUnitLine{ED} \perp \drawUnitLine{BC}$, тогда\\
$\drawAngle{ADB,EDA} + \drawAngle{CDE} = \drawTwoRightAngles$ \indef[def:I.X].

Но если \drawUnitLine{ED} будет не $\perp$ к~\drawUnitLine{BC},\\
проведем $\drawUnitLine{AD} \perp \drawUnitLine{BC}$ \inprop[prop:I.XI].

$\drawAngle{ADB} +\drawAngle{CDE,EDA} = \drawTwoRightAngles$ (\conststr),\\
$\drawAngle{ADB} = \drawAngle{CDE,EDA} = \drawAngle{EDA} + \drawAngle{CDE}$.

$\therefore \drawAngle{ADB} + \drawAngle{CDE,EDA} = \drawAngle{ADB} + \drawAngle{EDA} + \drawAngle{CDE}$ \inax[ax:I.II]\\
$= \drawAngle{ADB,EDA} + \drawAngle{CDE} = \drawTwoRightAngles$.
\stopCenterAlign

\qed
\stopProposition

\startProposition[title={Предл. XIV. Теорема}, reference=prop:I.XIV]
\defineNewPicture[1/4]{
pair A, B, C, D, E;
A := (u, 5/2u);
B := (-7/4u, 0);
C := B xscaled -1;
D := (0, 0);
E := (xpart(C), -1/2ypart(A));
byAngleDefine(B, D, A, byyellow, 0);
byAngleDefine(C, D, A, byblue, 0);
byAngleDefine(E, D, C, byred, 0);
draw byNamedAngleResized();
draw byLine(A, D, byred, 0, 0);
draw byLine(E, D, byyellow, 0, 0);
draw byLine(B, D, byblue, 0, 0);
draw byLine(C, D, byblack, 0, 0);
draw byLabelsOnPolygon(E, D, B, noPoint)(0, 0);
draw byLabelsOnPolygon(C, B, noPoint)(4, 0);
draw byLabelLineEnd(A, D, 0);
}
\drawCurrentPictureInMargin
\problemNP{Е}{сли}{две прямые \drawUnitLine{BD} и~\drawUnitLine{DC} образуют с~третьей \drawUnitLine{AD} смежные углы, находясь по разные стороны от нее, и~эти углы \drawAngle{BDA} и~\drawAngle{CDA} равны двум прямым углам, то эти прямые будут лежать на одной прямой.}

\startCenterAlign
Действительно, пусть \drawUnitLine{ED}, а~не~\drawUnitLine{DC} будет продолжением \drawUnitLine{BD},\\
тогда $\drawAngle{BDA} + \drawAngle{CDA,EDC} = \drawTwoRightAngles$.

Но, согласно гипотезе, $\drawAngle{BDA} + \drawAngle{CDA} = \drawTwoRightAngles$

$\therefore\drawAngle{CDA,EDC} = \drawAngle{CDA}$ \inax[ax:I.III],\\
что невозможно \inax[ax:I.IX].

$\therefore \drawUnitLine{ED}$ не является продолжением \drawUnitLine{BD}, и~то же можно показать для любой другой прямой линии, за исключением \drawUnitLine{DC}.

$\therefore \drawUnitLine{DC}$ является продолжением \drawUnitLine{BD}.
\stopCenterAlign

\qed
\stopProposition

\startProposition[title={Предл. XV. Теорема}, reference=prop:I.XV]
\defineNewPicture{
pair A, B, C, D, E;
A := (7/4u, 3/2u);
B := A scaled -1;
C := A xscaled -1;
D := C scaled -1;
E := (A--B) intersectionpoint (C--D);
byAngleDefine(B, E, C, byyellow, 0);
byAngleDefine(C, E, A, byred, 0);
byAngleDefine(A, E, D, byblack, 0);
byAngleDefine(D, E, B, byblue, 0);
draw byNamedAngleResized();
draw byLine(A, B, byred, 0, 0);
draw byLine(C, D, byblack, 0, 0);
draw byLabelsOnPolygon(C, E, A, noPoint)(0, 0);
draw byLabelPoint(B, lineAngle.AB + 90, 1);
draw byLabelPoint(D, lineAngle.CD - 90, 1);
}
\drawCurrentPictureInMargin
\problemNP{Е}{сли}{две прямых линии \drawUnitLine{AB} и~\drawUnitLine{CD} пересекаются, вертикальные углы \drawAngle{BEC} и~\drawAngle{AED}, \drawAngle{CEA} и~\drawAngle{DEB} будут равны между собой.}

\startCenterAlign
$\drawAngle{BEC} + \drawAngle{CEA} = \drawTwoRightAngles$ \inprop[prop:I.XIII].

$\drawAngle{AED} + \drawAngle{CEA} = \drawTwoRightAngles$\inprop[prop:I.XIII].

$\therefore \drawAngle{BEC} = \drawAngle{AED}$ \inax[ax:I.III].

Таким же образом можно показать,\\
что $\drawAngle{CEA} = \drawAngle{DEB}$.
\stopCenterAlign

\qed
\stopProposition

\startProposition[title={Предл. XVI. Теорема}, reference=prop:I.XVI]
\defineNewPicture[1/4]{
pair A, B, C, D, E, F, G;
A := (0, 0);
B := A shifted (3/2u, 7/2u);
C := A shifted (3u, 0);
D := B shifted (3u, 0);
E = whatever[A, D] = whatever[B, C];
F := (xpart(D), ypart(A));
G := 4/3[B, C];
byAngleDefine(B, A, C, byblue, 0);
byAngleDefine(C, B, A, byblack, 0);
byAngleDefine(A, E, B, byyellow, 0);
byAngleDefine(D, E, C, byyellow, 0);
byAngleDefine(E, C, D, byblack, 0);
byAngleDefine(G, C, A, byred, 0);
byAngleDefine(D, C, F, byblack, 1);
draw byNamedAngleResized();
byLineDefine(C, F, byblack, 1, 0);
byLineDefine(C, G, byblack, 0, 0);
byLineDefine(B, E, byblue, 0, 0);
byLineDefine(E, C, byblue, 1, 0);
byLineDefine(A, E, byred, 0, 0);
byLineDefine(E, D, byred, 1, 0);
byLineDefine(A, B, byyellow, 1, 0);
byLineDefine(A, C, byblack, 0, 0);
byLineDefine(C, D, byyellow, 0, 0);
draw byNamedLineSeq(0)(AE,ED,CD);
draw byNamedLineSeq(0)(EC,CG,noLine,CF,AC,AB,BE);
draw byLabelsOnPolygon(F, A, B, E, D, C)(2, 0);
draw byLabelsOnPolygon(F, C, G, noPoint)(0, 0);
}
\drawCurrentPictureInMargin
\problemNP[2]{П}{ри}{продолжении стороны треугольника \drawLine[middle]{BE,EC,AC,AB} внешний угол \drawAngleWithSides{FCE} будет больше любого из противолежащих ему внутренних углов \drawAngle{B} или \drawAngle{A}.
}

\startCenterAlign
Сделаем $\drawUnitLine{BE} = \drawUnitLine{EC}$ \inprop[prop:I.X],\\
проведем \drawUnitLine{AE} и~продлим до $\drawUnitLine{ED} = \drawUnitLine{AE}$,\\
проведем \drawUnitLine{CD}. 

В~\drawLine{BE,AE,AB} и~\drawLine{EC,ED,CD}\\
$\drawUnitLine{BE} = \drawUnitLine{EC}$, $\drawAngle{AEB} = \drawAngle{DEC}$ \inprop[prop:I.XV]\\
и~$\drawUnitLine{AE} = \drawUnitLine{ED}$ (\conststr).

$\therefore \drawAngle{B} = \drawAngle{ECD}$ \inprop[prop:I.IV],\\
$\therefore \drawAngleWithSides{FCE} > \drawAngle{B}$.

Так же можно показать, что при продлении \drawUnitLine{BC} $\drawAngle{GCA} > \drawAngle{A}$, \\
и, следовательно, \drawAngleWithSides{FCE}, который $= \drawAngle{GCA}$, будет $> \drawAngle{A}$.
\stopCenterAlign

\qed
\stopProposition

\startProposition[title={Предл. XVII. Теорема}, reference=prop:I.XVII]
\defineNewPicture[1/4]{
pair A, B, C, D;
A := (0, 0);
B := A shifted (3/2u, 5/2u);
C := A shifted (9/4u, 0);
D := C shifted (u, 0);
byAngleDefine(B, A, C, byblue, 0);
byAngleDefine(A, B, C, byblack, 0);
byAngleDefine(A, C, B, byred, 0);
byAngleDefine(B, C, D, byyellow, 0);
draw byNamedAngleResized();
byLineDefine(A, B, byred, 0, 0);
byLineDefine(B, C, byblue, 0, 0);
byLineDefine(A, C, byblack, 0, 0);
byLineDefine(C, D, byblack, 0, 0);
draw byNamedLineSeq(0)(noLine,BC,AB,AC,CD);
draw byLabelsOnPolygon(D, C, A, B)(0, 0);
}
\drawCurrentPictureInMargin
\problemNP{В}{о}{всяком треугольнике \drawLine[bottom]{AB,BC,AC} любые два угла, взятые вместе, меньше двух прямых углов.}

\startCenterAlign
Продлим \drawUnitLine{AC}, тогда\\
$\drawAngle{ACB} + \drawAngle{BCD} = \drawTwoRightAngles$.

Но $\drawAngle{BCD} > \drawAngle{A}$ \inprop[prop:I.XVI].

$\therefore \drawAngle{ACB} + \drawAngle{A} < \drawTwoRightAngles$.
\stopCenterAlign

\noindent И~таким же образом можно показать, что любые два других угла вместе будут меньше двух прямых углов.

\qed
\stopProposition

\startProposition[title={Предл. XVIII. Теорема}, reference=prop:I.XVIII]
\defineNewPicture[1/4]{
pair A, B, C, D;
numeric a;
A := (0, 0);
B := A shifted (4u, 2u);
C := B shifted (-3/2u, 2u);
D := C shifted (unitvector(A-C) scaled abs(B-C));
a := angle(B-D);
forsuffixes i=A, B, C, D:
i := i rotated -a;
endfor;
byAngleDefine(C, D, B, byblue, 0);
byAngleDefine(D, B, C, byblack, 0);
byAngleDefine(A, B, D, byred, 0);
byAngleDefine(B, A, D, byyellow, 0);
draw byNamedAngleResized();
draw byLine(D, B, byyellow, 0, 0);
byLineDefine(D, C, byred, 0, 0);
byLineDefine(B, C, byblue, 0, 0);
byLineDefine(B, A, byblack, 0, 0);
byLineDefine(A, D, byred, 1, 0);
draw byNamedLineSeq(0)(DC,BC,BA,AD);
draw byLabelsOnPolygon(D, C, B, A)(0, 0);
}
\drawCurrentPictureInMargin
\problemNP[3]{В}{о}{всяком треугольнике \drawLine{DC,BC,BA,AD} если одна сторона \drawUnitLine{AD,DC} больше другой \drawUnitLine[0.5cm]{BC}, то противолежащий большей стороне угол будет больше противолежащего меньшей стороне угла, т.~е. $\drawAngle{DBC,ABD} > \drawAngle{A}$.}

\startCenterAlign
Сделаем $\drawUnitLine{DC} = \drawUnitLine{BC}$ \inprop[prop:I.III],\\ проведем \drawUnitLine{DB}.

Тогда $\drawAngle{D} = \drawAngle{DBC}$ \inprop[prop:I.V].

Но $\drawAngle{D} > \drawAngle{A}$ \inprop[prop:I.XVI].

$\therefore \drawAngle{DBC} > \drawAngle{A}$,\\
и тем более $\drawAngle{DBC,ABD} > \drawAngle{A}$.
\stopCenterAlign

\qed
\stopProposition

\startProposition[title={Предл. XIX. Теорема}, reference=prop:I.XIX]
\defineNewPicture[1/4]{
pair A, B, C;
A := (0, 0);
B := A shifted (7/2u, 0);
C := A shifted (u, 3u);
byAngleDefine(C, A, B, byblue, 0);
byAngleDefine(A, B, C, byred, 0);
draw byNamedAngleResized();
byLineDefine(A, B, byblack, 0, 0);
byLineDefine(B, C, byblue, 0, 0);
byLineDefine(C, A, byred, 0, 0);
draw byNamedLineSeq(0)(CA,BC,AB);
draw byLabelsOnPolygon(B, A, C)(0, 0);
}
\drawCurrentPictureInMargin
\problemNP[3]{В}{о}{всяком треугольнике \drawLine[bottom]{CA,BC,AB} если один угол \drawAngle{A} больше другого \drawAngle{B}, то сторона \drawUnitLine{BC}, противолежащая большему углу, больше стороны \drawUnitLine{CA}, противолежащей меньшему.}

\startCenterAlign
Если \drawUnitLine{BC} не больше \drawUnitLine{CA}, тогда\\
$\drawUnitLine{BC} =$ или $< \drawUnitLine{CA}$.

Если $\drawUnitLine{BC} = \drawUnitLine{CA}$, тогда\\
$\drawAngle{A} = \drawAngle{B}$ \inprop[prop:I.V],\\
что противоречит гипотезе.

\drawUnitLine{BC} также не меньше \drawUnitLine{CA},\\
поскольку если $\drawUnitLine{BC} < \drawUnitLine{CA}$,\\
то $\drawAngle{A} < \drawAngle{B}$ \inprop[prop:I.XVIII]\\
что противоречит гипотезе.

$\therefore \drawUnitLine{BC} > \drawUnitLine{CA}$.
\stopCenterAlign

\qed
\stopProposition

\startProposition[title={Предл. XX. Теорема}, reference=prop:I.XX]
\defineNewPicture{
pair A, B, C, D;
A := (0, 0);
B := A shifted (7/2u, 0);
D := A shifted (4/3u, 3/2u);
C := ((fullcircle scaled 2arclength(D--B)) shifted D) intersectionpoint (D--10[A, D]);
byAngleDefine(B, C, A, byred, 0);
byAngleDefine(C, B, D, byblue, 0);
byAngleDefine(D, B, A, byyellow, 0);
draw byNamedAngleResized();
byLineDefine(B, D, byred, 0, 0);
byLineDefine(A, B, byblack, 0, 0);
byLineDefine(B, C, byyellow, 0, 0);
byLineDefine(C, D, byblue, 1, 0);
byLineDefine(D, A, byblue, 0, 0);
draw byNamedLineSeq(0)(BD);
draw byNamedLineSeq(0)(DA,CD,BC,AB);
draw byLabelsOnPolygon(D, C, B, A)(0, 0);
}
\drawCurrentPictureInMargin
\problemNP[3]{Л}{юбые}{две стороны \drawUnitLine{DA} и~\drawUnitLine{BD} всякого треугольника \drawLine[bottom]{DA,BD,AB}, взятые вместе, больше третьей стороны  \drawUnitLine{AB}.}

\startCenterAlign
Продлим \drawUnitLine{DA}\\
и сделаем $\drawUnitLine{CD} = \drawUnitLine{BD}$ \inprop[prop:I.III].

Проведем \drawUnitLine{BC}.

Тогда, поскольку $\drawUnitLine{CD} = \drawUnitLine{BD}$ (\conststr),\\
$\drawAngle{CBD} = \drawAngle{C}$ \inprop[prop:I.V].

$\therefore \drawAngle{CBD,DBA} > \drawAngle{C}$ \inax[ax:I.IX].

$\therefore \drawUnitLine{DA} + \drawUnitLine{CD} > \drawUnitLine{AB}$ \inprop[prop:I.XIX].

И $\therefore \drawUnitLine{DA} + \drawUnitLine{BD} > \drawUnitLine{AB}$.
\stopCenterAlign

\qed
\stopProposition

\startProposition[title={Предл. XXI. Теорема}, reference=prop:I.XXI]
\defineNewPicture[1/4]{
pair A, B, C, D, E;
A := (0, 0);
B := A shifted (7/2u, 0);
C := A shifted (5/2u, 3u);
D := 1/2[1/2[A, B], C];
E = whatever[A, D] = whatever[B, C];
byAngleDefine(B, D, A, byred, 0);
byAngleDefine(B, E, D, byblue, 0);
byAngleDefine(B, C, A, byyellow, 0);
draw byNamedAngleResized();
byLineDefine(B, D, byyellow, 0, 0);
byLineDefine(A, D, byblack, 0, 0);
byLineDefine(D, E, byblack, 1, 0);
byLineDefine(A, B, byblue, 1, 0);
byLineDefine(B, E, byred, 1, 0);
byLineDefine(E, C, byred, 0, 0);
byLineDefine(C, A, byblue, 0, 0);
draw byNamedLine(BD);
draw byNamedLineSeq(0)(AD,DE);
draw byNamedLineSeq(0)(CA,EC,BE,AB);
draw byLabelsOnPolygon(A, C, E, B)(0, 0);
draw byLabelsOnPolygon(A, D, E)(2, 0);
}
\drawCurrentPictureInMargin
\problemNP[2]{Е}{сли}{из любой точки \drawPointL[middle][DE]{D} внутри треугольника \drawLine[bottom]{CA,EC,BE,AB} провести прямые линии к~концам стороны \drawSizedLine{AB}, эти прямые  вместе меньше двух других сторон треугольника и~будут заключать больший угол.}

\startCenterAlign
Продлим \drawSizedLine{AD},\\
$\drawSizedLine{CA} + \drawSizedLine{EC} > \drawSizedLine{AD,DE}$ \inprop[prop:I.XX],\\
добавим к~каждой \drawSizedLine{BE},\\
$\drawSizedLine{CA} + \drawSizedLine{EC,BE} > \drawSizedLine{AD,DE} + \drawSizedLine{BE}$ \inax[ax:I.IV].

Таким же образом можно показать, что\\
$\drawSizedLine{AD,DE} + \drawSizedLine{BE} > \drawSizedLine{AD} + \drawSizedLine{BD}$,\\
$\therefore \drawSizedLine{CA} + \drawSizedLine{EC,BE} > \drawSizedLine{AD} + \drawSizedLine{BD}$,\\
что и~требовалось доказать.

Далее $\drawAngle{E} > \drawAngle{C}$ \inprop[prop:I.XVI]\\
и так же $\drawAngle{D} > \drawAngle{E}$ \inprop[prop:I.XVI].

$\therefore \drawAngle{D} > \drawAngle{C}$.
\stopCenterAlign

\qed
\stopProposition

\startProposition[title={Предл. XXII. Задача}, reference=prop:I.XXII]
\defineNewPicture[1/2]{
numeric r[], d;
pair A, B, C, D, E, LI, LII, LIII, LIV, LV, LVI;
path q[];
r1 := 3/2u;
r2 := 4/3u;
r3 := (2/3)*(r1+r2);
d := 1/3u;
A := (0, 0);
B := A shifted (r3, 0);
q1 := (fullcircle scaled 2r1) shifted A;
q2 := (fullcircle scaled 2r2) shifted B;
C := q1 intersectionpoint q2;
D := point 11/2 of q1;
E := point 3/4 of q2;
LI := (xpart(point 0 of q2), ypart(point 6 of q1) - 1/2d);
LII := LI shifted (-r3, 0);
LIII := LI shifted (0, -d);
LIV := LIII shifted (-r2, 0);
LV := LIII shifted (0, -d);
LVI := LV shifted (-r1, 0);
draw byCircle.A(A, D, byblue, 0, 0, 0);
byLineDefine(A, D, byblue, 0, 0);
byLineDefine(B, E, byred, 0, 0);
byLineDefine(A, B, byblack, 0, 0);
byLineDefine(B, C, byyellow, 0, 0);
byLineDefine(C, A, byyellow, 1, 0);
draw byNamedLineSeq(0)(BC,CA);
draw byNamedLineSeq(0)(AD,AB,BE);
draw byLineWithName (LII, LI, byblack, 1, 0)(L');
draw byLineWithName (LIV, LIII, byred, 1, 0)(L'');
draw byLineWithName (LVI, LV, byblue, 1, 0)(L''');
draw byCircle.B(B, E, byred, 0, 0, 0);
draw byLabelsOnPolygon(D, A, C)(2, 0);
draw byLabelsOnPolygon(E, B, A)(2, 0);
draw byLabelsOnPolygon(A, C, B)(2, 0);
draw byLabelsOnCircle(D)(A);
draw byLabelsOnCircle(E)(B);
draw byLabelLine(0)(L', L'', L''');
}
\drawCurrentPictureInMargin
\problemNP{И}{з}{трех прямых линий $\left\{\vcenter{
\nointerlineskip\hbox{\drawSizedLine{L'}}
\nointerlineskip\hbox{\drawSizedLine{L''}}
\nointerlineskip\hbox{\drawSizedLine{L'''}}}\right.$
таких, что любые две вместе длиннее третьей, составить треугольник.}

\startCenterAlign
Предположим, $\drawSizedLine{AB} = \drawSizedLine{L'}$ \inprop[prop:I.III].

$\left.\eqalign{
\mbox{Проведем } \drawSizedLine{BE} &= \drawSizedLine{L''}\cr
\mbox{и } \drawSizedLine{AD} &= \drawSizedLine{L'''}
}\right\}\mbox{\inprop[prop:I.II].}$

Взяв \drawSizedLine{AD} и~\drawSizedLine{BE} как радиусы, опишем
\drawFromCurrentPicture{
draw byNamedLine(AD); draw byNamedCircle(A);
draw byLabelLineEnd(A, D, 0);
draw byLabelLineEnd(D, A, 0);
} и
\offsetPicture{12pt}{0pt}{\drawFromCurrentPicture{
draw byNamedLine(BE); draw byNamedCircle(B);
draw byLabelLineEnd(B, E, 0);
draw byLabelLineEnd(E, B, 0);
}} \inpost[post:I.III].

Проведем \drawSizedLine{CA} и~\drawSizedLine{BC}.

Тогда \drawLine[bottom]{CA,BC,AB} будет искомым треугольником.

$\left.\eqalign{
\mbox{Поскольку } \drawSizedLine{AB} &= \drawSizedLine{L'} \mbox{,} \cr
\drawSizedLine{BC} &= \drawSizedLine{BE} = \drawSizedLine{L''} \cr
\mbox{и } \drawSizedLine{CA} &= \drawSizedLine{AD} = \drawSizedLine{L'''} \cr
}\right\}\mbox{(\conststr).}$
\stopCenterAlign

\qed
\stopProposition

\startProposition[title={Предл. XXIII. Задача}, reference=prop:I.XXIII]
\defineNewPicture{
pair A, B, C, D, E, F, G, H, J, d;
A := (0, 0);
B := A shifted (7/2u, 0);
C := A shifted (3u, 11/5u);
D := 5/4[A, B];
E := 7/6[A, C];
d := (0, -3u);
F := A shifted d;
G := B shifted d;
H := C shifted d;
J := D shifted d;
byAngleDefine(B, A, C, byred, 0);
byAngleDefine(G, F, H, byblue, 0);
draw byNamedAngleResized();
byLineDefine(B, D, byblack, 1, 1);
byLineDefine(C, E, byblue, 1, 1);
byLineDefine(A, B, byblack, 0, 1);
byLineDefine(C, A, byblue, 0, 1);
draw byLine(B, C, byred, 0, 1);
draw byNamedLineSeq(0)(CE,CA,AB,BD);
byLineDefine(G, J, byblack, 1, 0);
byLineDefine(F, G, byblack, 0, 0);
byLineDefine(G, H, byred, 0, 0);
byLineDefine(H, F, byyellow, 0, 0);
draw byNamedLineSeq(0)(noLine,GH,HF,FG,GJ);
draw byLabelsOnPolygon(D, B, A, C, E)(2, 0);
draw byLabelsOnPolygon(noPoint, J, G, F, H, G)(2, 0);
}
\drawCurrentPictureInMargin
\problemNP{П}{ри}{данной точке \drawPointL{F} на данной прямой \drawUnitLine{FG,GJ} построить угол, равный данному прямолинейному углу \drawAngle{A}.}

Проведем \drawUnitLine{BC} между любыми двумя точками на сторонах данного угла.

\startCenterAlign
Построим \drawLine[bottom]{HF,GH,FG} \inprop[prop:I.XXII] такой,\\
что $\drawUnitLine{FG} = \drawUnitLine{AB}$,\\
$\drawUnitLine{HF} = \drawUnitLine{CA}$\\
и $\drawUnitLine{GH} = \drawUnitLine{BC}$.

Тогда $\drawAngle{A} = \drawAngle{F}$ \inprop[prop:I.VIII].
\stopCenterAlign

\qed
\stopProposition

\startProposition[title={Предл. XXIV. Теорема}, reference=prop:I.XXIV]
\defineNewPicture{
pair A, B, C, D, E, F, G, d;
A := (0, 0);
B := A shifted (u, -5/2u);
C := A shifted (-u, -7/2u);
D := (xpart(C) - 3/2u, ypart(B));
d := (0, -4u);
E := A shifted d;
F := B shifted d;
G := C shifted d;
byAngleDefine(B, A, C, byblack, 2);
byAngleDefine(C, A, D, byred, 1);
byAngleDefine(F, E, G, byblack, 2);
byAngleDefine(B, D, A, byblue, 0);
byAngleDefine(C, D, B, byred, 0);
byAngleDefine(D, C, A, byyellow, 0);
byAngleDefine(A, C, B, byblack, 0);
draw byNamedAngleResized();
byLineDefine(A, B, byblue, 0, 0);
byLineDefine(B, C, byblack, 1, 0);
draw byLine(C, A, byred, 0, 0);
draw byLine(B, D, byblack, 0, 0);
byLineDefine(A, D, byred, 1, 0);
byLineDefine(C, D, byblue, 1, 0);
draw byNamedLineSeq(0)(AB,AD,CD,BC);
byLineDefine(E, F, byblue, 0, 1);
byLineDefine(F, G, byyellow, 0, 1);
byLineDefine(G, E, byred, 0, 1);
draw byNamedLineSeq(0)(EF,FG,GE);
draw byLabelsOnPolygon(D, A, B, C)(0, 0);
draw byLabelsOnPolygon(G, E, F)(0, 0);
}
\drawCurrentPictureInMargin
\problemNP{Е}{сли}{у двух треугольников по две стороны соответственно равны друг другу ($\drawUnitLine{AB} = \drawUnitLine{EF}$ и~$\drawUnitLine{AD} = \drawUnitLine{GE}$) и~угол, заключенный между ними в~одном \drawAngleWithSides{DAB}, больше, чем в~другом \drawAngleWithSides{FEG}, то сторона \drawUnitLine{DB}, противолежащая большему углу, больше стороны, противолежащей меньшему \drawUnitLine{FG}.}

\startCenterAlign
Сделаем $\drawAngleWithSides{BAC} = \drawAngleWithSides{FEG}$ \inprop[prop:I.XXIII]\\
и $\drawUnitLine{CA} = \drawUnitLine{GE}$ \inprop[prop:I.III],\\
проведем \drawUnitLine{CD} и~\drawUnitLine{BC}.

Поскольку $\drawUnitLine{CA} = \drawUnitLine{AD}$ (\inaxL[ax:I.I], \hypstr,  \conststr)\\
$\therefore \drawAngle{BDA,CDB} = \drawAngle{DCA}$ \inprop[prop:I.V],
но $\drawAngle{CDB} < \drawAngle{DCA}$,\\
и $\therefore \drawAngle{CDB} < \drawAngle{DCA,ACB}$.

$\therefore \drawUnitLine{DB} > \drawUnitLine{BC}$ \inprop[prop:I.XIX].

Но $\drawUnitLine{BC} = \drawUnitLine{FG}$ \inprop[prop:I.IV].

$\therefore \drawUnitLine{DB} > \drawUnitLine{FG}$.
\stopCenterAlign

\qed
\stopProposition

\startProposition[title={Предл. XXV. Теорема}, reference=prop:I.XXV]
\defineNewPicture{
pair A, B, C, D, E, F, d;
A := (0, 0);
B := A shifted (u, -3u);
C := A shifted (-7/4u, -4u);
d := (0, -9/2u);
D := A shifted d;
E := ((B shifted -A) rotated -10) shifted d;
F := C shifted d;
byAngleDefine(B, A, C, byyellow, 0);
byAngleDefine(E, D, F, byblack, 0);
draw byNamedAngleResized();
byLineDefine(A, B, byblue, 0, 0);
byLineDefine(B, C, byblack, 0, 0);
byLineDefine(C, A, byred, 0, 0);
draw byNamedLineSeq(0)(AB,BC,CA);
byLineDefine(D, E, byblue, 0, 1);
byLineDefine(E, F, byyellow, 0, 1);
byLineDefine(F, D, byred, 0, 1);
draw byNamedLineSeq(0)(DE,EF,FD);
draw byLabelsOnPolygon(A, B, C)(0, 0);
draw byLabelsOnPolygon(D, E, F)(0, 0);
}
\drawCurrentPictureInMargin
\problemNP{Е}{сли}{у двух треугольников две стороны \drawUnitLine{AB} и~\drawUnitLine{CA} соответственно равны двум сторонам \drawUnitLine{DE} и~\drawUnitLine{FD} другого, но основания неравны, то угол над большим основанием \drawUnitLine{BC} одного треугольника меньше угла под меньшим \drawUnitLine{EF} другого.}

\startCenterAlign
$\drawAngle{A} =\mbox{, } > \mbox{ или } < \drawAngle{D}$.

\drawAngle{A} не равен \drawAngle{D},\\
поскольку если $\drawAngle{A} = \drawAngle{D}$,\\ 
то $\drawUnitLine{CB} = \drawUnitLine{FE}$ \inprop[prop:I.IV],\\
что противоречит гипотезе.

\drawAngle{A} не меньше \drawAngle{D},\\
поскольку если $\drawAngle{A} < \drawAngle{D}$,\\
то $\drawUnitLine{CB} < \drawUnitLine{FE}$ \inprop[prop:I.XXIV],\\
что противоречит гипотезе.

$\therefore \drawAngle{A} > \drawAngle{D}$.
\stopCenterAlign

\qed
\stopProposition

\startProposition[title={Предл. XXVI. Теорема}, reference=prop:I.XXVI]
\defineNewPicture{
pair A, B, C, D, E, F, G, d;
A := (0, 0);
B := A shifted (3u, 0);
C := A shifted (2u, 3u);
d := (0, -4u);
D := A shifted d;
E := B shifted d;
F := C shifted d;
G := 3/4[D, F];
byAngleDefine(B, A, C, byyellow, 0);
byAngleDefine(C, B, A, byred, 0);
byAngleDefine(A, C, B, byblack, 1);
draw byNamedAngleResized(BAC, CBA, ACB);
byLineDefine(A, B, byblue, 0, 0);
byLineDefine(B, C, byblack, 0, 0);
byLineDefine(C, A, byred, 0, 0);
draw byNamedLineSeq(0)(CA,BC,AB);
byAngleDefine(E, D, F, byyellow, 0);
byAngleDefine(G, E, D, byblack, 0);
byAngleDefine(F, E, G, byblue, 0);
byAngleDefine(D, F, E, byblack, 1);
draw byNamedAngleResized(EDF, GED, FEG, DFE);
draw byLine(E, G, byyellow, 0, 1);
byLineDefine(D, E, byblue, 0, 1);
byLineDefine(E, F, byblack, 0, 1);
byLineDefine(F, G, byred, 1, 1);
byLineDefine(G, D, byred, 0, 1);
draw byNamedLineSeq(0)(GD,FG,EF,DE);
draw byLabelsOnPolygon(C, B, A)(0, 0);
draw byLabelsOnPolygon(D, G, F, E)(0, 0);
}
\problemNP[3]{Е}{сли}{два угла одного треугольника соответственно равны двум углам другого треугольника ($\drawAngle{A} = \drawAngle{D}$ и~$\drawAngle{B} = \drawAngle{GED,FEG}$) и~одна сторона равна одного равна так же расположенной стороне другого, то и~остальные стороны и~углы соответственно равны друг другу.}

\drawCurrentPictureInMargin
\startsubproposition[title={Случай I.}]
\startCenterAlign
Пусть  \drawUnitLine{AB} и~\drawUnitLine{DE}, лежащие между равными углами, равны,\\
тогда $\drawUnitLine{CA} = \drawUnitLine{GD,FG}$.

Поскольку, если \drawUnitLine{GD,FG} больше,\\
сделаем $\drawUnitLine{CA} = \drawUnitLine{GD}$, проведем \drawUnitLine{EG}.

В \drawLine[bottom]{CA,BC,AB} и
\drawLine[bottom]{GD,EG,DE} получим \\
$\drawUnitLine{CA} = \drawUnitLine{GD}$, $\drawAngle{A} = \drawAngle{D}$, $\drawUnitLine{AB} = \drawUnitLine{DE}$;\\
$\therefore \drawAngle{B} = \drawAngle{GED}$ (pr. 4.)\\
но $\drawAngle{B} = \drawAngle{GED,FEG}$ (\hypstr).

И следовательно $\drawAngle{GED} = \drawAngle{GED,FEG}$, что не~имеет смысла, а~значит ни \drawUnitLine{CA}, ни \drawUnitLine{GD,FG} не~больше другой, и~$\therefore$ они равны.

$\therefore \drawUnitLine{BC} = \drawUnitLine{EF}$, и~$\drawAngle{C} = \drawAngle{F}$ \inprop[prop:I.IV].
\stopCenterAlign
\stopsubproposition

\vfill\pagebreak

\defineNewPicture{
pair A, B, C, D, E, F, G, d;
d := (0, -4u);
A := (0, 0);
B := A shifted (3u, 0);
C := A shifted (1u, 3u);
D := A shifted d;
E := B shifted d;
F := C shifted d;
G := 3/4[D, E];
byAngleDefine(B, A, C, byyellow, 0);
byAngleDefine(C, B, A, byred, 0);
draw byNamedAngleResized(BAC, CBA);
byLineDefine(A, B, byblue, 0, 0);
byLineDefine(B, C, byblack, 0, 0);
byLineDefine(C, A, byred, 0, 0);
draw byNamedLineSeq(0)(CA,AB,BC);
byAngleDefine(F, D, E, byyellow, 0);
byAngleDefine(F, G, D, byblack, 0);
byAngleDefine(F, E, D, byred, 0);
draw byNamedAngleResized(FDE, FGD, FED);
draw byLine(F, G, byyellow, 0, 1);
byLineDefine(D, G, byblue, 0, 1);
byLineDefine(G, E, byblue, 1, 1);
byLineDefine(E, F, byblack, 0, 1);
byLineDefine(F, D, byred, 0, 1);
draw byNamedLineSeq(0)(FD,EF,GE,DG);
draw byLabelsOnPolygon(C, B, A)(0, 0);
draw byLabelsOnPolygon(D, F, E, G)(0, 0);
}
\drawCurrentPictureInMargin
\startsubproposition[title={Случай II.}]
\startCenterAlign
Теперь пусть $\drawUnitLine{CA} = \drawUnitLine{FD}$, лежат против равных углов \drawAngle{B} и~\drawAngle{E}.\\
Если такое возможно, пусть $\drawUnitLine{DG,GE} > \drawUnitLine{AB}$, тогда возьмем $\drawUnitLine{DG} = \drawUnitLine{AB}$, проведем \drawUnitLine{FG}.

Тогда в~\drawLine[bottom]{CA,BC,AB} и~\drawLine[bottom]{FD,FG,DG} получим $\drawUnitLine{CA} = \drawUnitLine{FD}$, $\drawUnitLine{AB} = \drawUnitLine{DG}$ и~$\drawAngle{A} = \drawAngle{D}$.

$\therefore \drawAngle{B} = \drawAngle{G}$ \inprop[prop:I.IV],\\
но $\drawAngle{B} = \drawAngle{E}$ (\hypstr).

$\therefore \drawAngle{G} = \drawAngle{E}$, что не~имеет смысла \inprop[prop:I.XVI]. % question: при чем тут I.XVI ?

Следовательно, ни \drawUnitLine{AB}, ни \drawUnitLine{DG,GE} не~больше другой, а~значит, они равны. Следовательно (согласно \inpropL[prop:I.IV]), треугольники равны во всех отношениях.
\stopCenterAlign
\stopsubproposition

\qed
\stopProposition

\startProposition[title={Предл. XXVII. Теорема}, reference=prop:I.XXVII]
\defineNewPicture{
pair A, B, C, D, E, F, G, H, I, d;
A := (0, 0);
B := A shifted (8/3u, 0);
d := (0, -7/4u);
C := A shifted d;
D := B shifted d;
E := 1/3[A, B];
F := 2/3[C, D];
G := 3/2[F, E];
H := 3/2[E, F];
I := 1/2[A, C] shifted (-2u, 0);
byAngleDefine(A, E, H, byyellow, 0);
byAngleDefine(H, E, B, byred, 0);
byAngleDefine(C, F, G, byblue, 0);
byAngleDefine(G, F, D, byyellow, 0);
draw byNamedAngleResized();
byLineDefine(I, A, byblue, 0, 0);
byLineDefine(A, B, byblue, 0, 0);
byLineDefine(I, C, byred, 0, 0);
byLineDefine(C, D, byred, 0, 0);
draw byNamedLineSeq(0)(CD,IC,IA,AB);
draw byLine(G, H, byblack, 0, 0);
draw byLabelLine(0)(AB, CD, GH);
draw byLabelsOnPolygon(G, E, B)(2, 0);
draw byLabelsOnPolygon(H, F, C)(2, 0);
}
\drawCurrentPictureInMargin
\problemNP[3]{Е}{сли}{прямая \drawUnitLine{GH}, пересекая две другие прямые \drawUnitLine{CD} и~\drawUnitLine{AB}, образует накрест лежащие углы \drawAngle{CFG}, \drawAngle{HEB} и~\drawAngle{GFD} и~\drawAngle{AEH}, равные между собой, то две пересекаемые прямые параллельны.}

Если \drawUnitLine{CD} не параллельна \drawUnitLine{AB}, то они сойдутся по продолжении.

Если это возможно, пусть они не будут параллельны, но сойдутся, если их продолжить; тогда внешний угол \drawAngle{HEB} будет больше \drawAngle{CFG} \inprop[prop:I.XVI], но они равны (\hypstr), что невозможно. Таким же образом можно показать, что они не сойдутся с~другой стороны, $\therefore$ они параллельны.

\qed
\stopProposition

\startProposition[title={Предл. XXVIII. Теорема}, reference=prop:I.XXVIII]
\defineNewPicture{
pair A, B, C, D, E, F, G, H, d;
A := (0, 0);
B := A shifted (7/2u, 0);
d := (0, -3/2u);
C := A shifted d;
D := B shifted d;
E := 9/20[A, B];
F := 11/20[C, D];
G := 7/4[F, E];
H := 4/3[E, F];
byAngleDefine(G, E, A, byblack, 0);
byAngleDefine(B, E, G, byyellow, 0);
byAngleDefine(A, E, H, byred, 0);
byAngleDefine(H, E, B, byblue, 0);
byAngleDefine(C, F, G, byblue, 0);
byAngleDefine(G, F, D, byred, 0);
draw byNamedAngleResized();
draw byLine(A, B, byred, 0, 0);
draw byLine(C, D, byyellow, 0, 0);
draw byLine(G, H, byblack, 0, 0);
draw byLabelLine(0)(AB, CD, GH);
draw byLabelsOnPolygon(G, E, B)(2, 0);
draw byLabelsOnPolygon(H, F, C)(2, 0);
}
\drawCurrentPictureInMargin
\problemNP{Е}{сли}{прямая \drawUnitLine{GH}, падающая на две прямые \drawUnitLine{AB} и~\drawUnitLine{CD}, образует внешний угол, равный внутреннему противолежащему с~той же стороны $\drawAngle{GEA} = \drawAngle{CFG}$ или $\drawAngle{BEG} = \drawAngle{GFD}$, или внутренние углы с~одной стороны \drawAngle{GFD} и~\drawAngle{HEB} или \drawAngle{CFG} и~\drawAngle{AEH} вместе равны двум прямым углам, то прямые параллельны.}

\startCenterAlign
Во-первых, если $\drawAngle{GEA} = \drawAngle{CFG}$,\\
то $\drawAngle{GEA} = \drawAngle{HEB}$ \inprop[prop:I.XV],\\
$\therefore \drawAngle{CFG} = \drawAngle{HEB} \therefore \drawUnitLine{AB} \parallel \drawUnitLine{CD}$ \inprop[prop:I.XXVII].

Во-вторых, если $\drawAngle{CFG} + \drawAngle{AEH} = \drawTwoRightAngles$,\\
то $\drawAngle{AEH} + \drawAngle{HEB} = \drawTwoRightAngles$ \inprop[prop:I.XIII],\\
$\therefore \drawAngle{CFG} + \drawAngle{AEH} = \drawAngle{AEH} + \drawAngle{HEB}$ \inax[ax:I.I].

$\therefore \drawAngle{CFG} = \drawAngle{HEB}$ \inax[ax:I.III].

$\therefore \drawUnitLine{AB} \parallel \drawUnitLine{CD}$ \inprop[prop:I.XXVII].
\stopCenterAlign

\qed
\stopProposition

\startProposition[title={Предл. XXIX. Теорема}, reference=prop:I.XXIX]
\defineNewPicture{
pair A, B, C, D, E, F, G, H, I, J, d[];
A := (0, 0);
B := A shifted (7/2u, 0);
d1 := (0, -2u);
C := A shifted d1;
D := B shifted d1;
E := 11/20[A, B];
F := 9/20[C, D];
G := 7/4[F, E];
H := 4/3[E, F];
d2 := (3/2u, 1/2u);
I := E shifted -d2;
J := E shifted d2;
byAngleDefine(I, E, A, byblue, 0);
byAngleDefine(H, E, I, byyellow, 0);
byAngleDefine(B, E, H, byblack, 0);
byAngleDefine(G, E, B, byred, 0);
byAngleDefine(G, F, D, byblack, 0);
draw byNamedAngleResized();
draw byLine(I, E, byblack, 0, 0);
draw byLine(E, J, byblack, 1, 0);
draw byLine(A, B, byyellow, 0, 0);
draw byLine(C, D, byred, 0, 0);
draw byLine(G, H, byblue, 0, 0);
draw byLabelLine(0)(AB, CD, GH);
draw byLabelsOnPolygon(A, E, G)(2, 0);
draw byLabelsOnPolygon(H, F, C)(2, 0);
draw byLabelPoint(I, lineAngle.IE - 90, 1);
draw byLabelPoint(J, lineAngle.EJ + 90, 1);
}
\drawCurrentPictureInMargin
\problemNP{П}{рямая}{\drawUnitLine{GH}, падающая на две параллельные прямые \drawUnitLine{AB} и~\drawUnitLine{CD}, образует накрест лежащие углы, равные между собой, внешний и~противолежащий с~той же стороны внутренний углы, равные между собой, а~также внутренние односторонние углы, равные двум прямым углам.}

\startCenterAlign
Если накрест лежащие углы \drawAngle{IEA,HEI} и~\drawAngle{GFD} не равны,\\ проведем \drawUnitLine{IE} так, чтобы $\drawAngle{HEI} = \drawAngle{GFD}$ \inprop[prop:I.XXIII].

$\therefore \drawUnitLine{IE,EJ} \parallel \drawUnitLine{CD}$ \inprop[prop:I.XXVII]\\
и $\therefore$ две пересекающихся прямых параллельны одной и~той же прямой, что невозможно \inax[ax:I.XII].

А значит, \drawAngle{IEA,HEI} и~\drawAngle{GFD} не являются неравными,\\ 
то есть они равны $\drawAngle{IEA,HEI} = \drawAngle{GEB}$ \inprop[prop:I.XV].

$\therefore \drawAngle{GEB} = \drawAngle{GFD}$, внешний угол равен внутреннему, противолежащему с~той же стороны:\\
если к~обоим добавить \drawAngle{BEH}, \\
то $\drawAngle{GFD} + \drawAngle{BEH} = \drawAngle{BEH,GEB} = \drawTwoRightAngles$ \inprop[prop:I.XIII].
\stopCenterAlign

Другими словами, два внутренних угла по одну сторону пересекающей прямой равны двум прямым углам.

\qed
\stopProposition

\startProposition[title={Предл. XXX. Теорема}, reference=prop:I.XXX]
\defineNewPicture{
pair A, B, C, D, E, F, G, H, I, J, K, d;
A := (0, 0);
B := A shifted (7/2u, 0);
d := (0, -u);
C := A shifted d;
D := B shifted d;
E := C shifted d;
F := D shifted d;
G := 13/20[A, B];
H := 7/20[E, F];
I := (G--H) intersectionpoint (C--D);
J := 3/2[H, G];
K := 5/4[G, H];
byAngleDefine(B, G, J, byyellow, 0);
byAngleDefine(D, I, J, byblue, 0);
byAngleDefine(F, H, J, byred, 0);
draw byNamedAngleResized();
draw byLine(A, B, byred, 0, 0);
draw byLine(C, D, byyellow, 0, 0);
draw byLine(E, F, byblue, 0, 0);
draw byLine(J, K, byblack, 0, 0);
draw byLabelLine(0)(AB, CD, EF, JK);
draw byLabelsOnPolygon(J, G, A)(2, 0);
draw byLabelsOnPolygon(J, I, C)(2, 0);
draw byLabelsOnPolygon(J, H, E)(2, 0);
}
\drawCurrentPictureInMargin
\problemNP{П}{рямые}{\drawUnitLine{AB} и~\drawUnitLine{EF}, параллельные одной и~той же прямой \drawUnitLine{CD}, параллельны между собой.}

\startCenterAlign
Пусть \drawUnitLine{JK} пересекает $\left\{\vcenter{\nointerlineskip\hbox{\drawUnitLine{AB}}\nointerlineskip\hbox{\drawUnitLine{CD}}\nointerlineskip\hbox{\drawUnitLine{EF}}}\right\}$.

Тогда $\drawAngle{G} = \drawAngle{I} = \drawAngle{H}$ \inprop[prop:I.XXIX].

$\therefore \drawAngle{G} = \drawAngle{H}$.

$\therefore \drawUnitLine{AB} \parallel \drawUnitLine{EF}$ \inprop[prop:I.XXVII].
\stopCenterAlign

\qed
\stopProposition

\startProposition[title={Предл. XXXI. Задача}, reference=prop:I.XXXI]
\defineNewPicture[1/6]{
pair A, B, C, D, E, F, d;
A := (0, 0);
B := A shifted (7/2u, 0);
d := (0, -3u);
C := A shifted d;
D := B shifted d;
E := 4/5[A, B];
F := 1/5[C, D];
byAngleDefine(F, E, A, byyellow, 0);
byAngleDefine(E, F, D, byred, 0);
draw byNamedAngleResized();
draw byLine(E, F, byblack, 0, 0);
draw byLine(A, E, byred, 0, 0);
draw byLine(E, B, byred, 1, 0);
draw byLine(C, F, byblue, 0, 0);
draw byLine(F, D, byblue, 0, 0);
draw byLabelsOnPolygon(A, E, B, noPoint, D, F, C, noPoint)(0, 0);
}
\drawCurrentPictureInMargin
\problemNP{П}{ровести}{через данную точку \drawPointL[middle][EB]{E} прямую, параллельную данной прямой \drawUnitLine{CF,FD}.}

\startCenterAlign
Проведем \drawUnitLine{EF} из точки \drawPointL[middle][EB]{E}  к~любой точке \drawPointL[middle][CF]{F} на \drawUnitLine{CF,FD}.

Сделаем $\drawAngle{E} = \drawAngle{F}$ \inprop[prop:I.XXIII].

Тогда $\drawUnitLine{AE,EB} \parallel \drawUnitLine{CF,FD}$ \inprop[prop:I.XXVII].
\stopCenterAlign

\qed
\stopProposition

\startProposition[title={Предл. XXXII. Теорема}, reference=prop:I.XXXII]
\defineNewPicture[1/6]{
pair A, B, C, D, E;
A := (0, 0);
B := A shifted (-7/4u, -3u);
C := A shifted (7/4u, -3u);
D := 4/3[B, A];
E := A shifted (unitvector(C-B) scaled 3/2u);
byAngleDefine(D, A, E, byred, 0);
byAngleDefine(E, A, C, byblack, 0);
byAngleDefine(C, A, B, byblue, 0);
byAngleDefine(A, B, C, byyellow, 0);
byAngleDefine(B, C, A, byblack, 0);
draw byNamedAngleResized();
draw byLine(A, E, byblue, 0, 0);
byLineDefine(A, D, byblack, 1, 0);
byLineDefine(A, B, byblack, 0, 0);
byLineDefine(B, C, byred, 0, 0);
byLineDefine(C, A, byyellow, 0, 0);
draw byNamedLineSeq(0)(CA,noLine,AD,AB,BC);
draw byLabelsOnPolygon(C, B, A, D, noPoint, E, A)(0, 0);
}
\drawCurrentPictureInMargin
\problemNP[2]{П}{о}{продлении любой стороны треугольника \drawUnitLine{AB} внешний угол \drawAngle{DAE,EAC} равен сумме двух внутренних и~противолежащих \drawAngle{B} и~\drawAngle{C}, а~три внутренних угла треугольника вместе равны двум прямым углам.}

\startCenterAlign
Через точку \drawPointL[middle][AD,AE]{A} проведем \\
$\drawUnitLine{AE} \parallel \drawUnitLine{BC}$ \inprop[prop:I.XXXI].

Тогда $\left\{\eqalign{\drawAngle{DAE} &= \drawAngle{B}\cr \drawAngle{EAC} &= \drawAngle{C}\cr}\right\}$ \inprop[prop:I.XXIX].

$\therefore \drawAngle{B} + \drawAngle{C} = \drawAngle{DAE,EAC}$ \inax[ax:I.II].

$\therefore \drawAngle{B} + \drawAngle{CAB} + \drawAngle{C} = \drawAngle{DAE,EAC,CAB} = \drawTwoRightAngles$ \inprop[prop:I.XIII].
\stopCenterAlign

\qed
\stopProposition

\startProposition[title={Предл. XXXIII. Теорема}, reference=prop:I.XXXIII]
\defineNewPicture[1/4]{
pair A, B, C, D, d[];
d1 := (5/2u, 0);
d2 := (-7/8u, -3u);
A := (0, 0);
B := A shifted d1;
C := A shifted d2;
D := C shifted d1;
byAngleDefine(B, A, D, byyellow, 0);
byAngleDefine(D, A, C, byred, 0);
byAngleDefine(C, D, A, byblack, 0);
byAngleDefine(A, D, B, byblue, 0);
draw byNamedAngleResized();
draw byLine(A, D, byblack, 0, 0);
byLineDefine(A, B, byred, 0, 0);
byLineDefine(C, D, byred, 1, 0);
byLineDefine(A, C, byblue, 0, 0);
byLineDefine(B, D, byyellow, 0, 0);
draw byNamedLineSeq(0)(AB,BD,CD,AC);
draw byLabelsOnPolygon(A, B, D, C)(0, 0);
}
\drawCurrentPictureInMargin
\problemNP{П}{рямые}{\drawUnitLine{AC} и~\drawUnitLine{BD}, соединяющие с~одной и~той же стороны равные и~параллельные прямые\drawUnitLine{AB} и~\drawUnitLine{CD}, сами равны и~параллельны.}

\startCenterAlign
Проведем диагональ \drawUnitLine{AD}.

Поскольку в \drawLine{DC,CA,AD} и \drawLine{DA,AB,BD}\\
$\drawUnitLine{CD} = \drawUnitLine{AB}$ (\hypstr),\\
$\drawAngle{CDA} = \drawAngle{BAD}$ \inprop[prop:I.XXIX]\\
и \drawUnitLine{AD} общая обоим,\\
$\therefore \drawUnitLine{AC} = \drawUnitLine{BD}$\\
и~$\drawAngle{DAC} = \drawAngle{ADB}$ \inprop[prop:I.IV].

И $\therefore \drawUnitLine{AC} \parallel \drawUnitLine{BD}$ \inprop[prop:I.XXVII].
\stopCenterAlign

\qed
\stopProposition

\startProposition[title={Предл. XXXIV. Теорема}, reference=prop:I.XXXIV]
\defineNewPicture[1/4]{
pair A, B, C, D, d[];
d1 := (5/2u, 0);
d2 := (-7/8u, -3u);
A := (0, 0);
B := A shifted d1;
C := A shifted d2;
D := C shifted d1;
byAngleDefine(B, A, D, byblue, 0);
byAngleDefine(D, A, C, byred, 0);
byAngleDefine(C, D, A, byyellow, 0);
byAngleDefine(A, D, B, byred, 0);
byAngleDefine(A, C, D, byblack, 0);
byAngleDefine(D, B, A, byblack, 0);
draw byNamedAngleResized();
draw byLine(A, D, byblack, 0, 0);
byLineDefine(A, B, byred, 0, 0);
byLineDefine(C, D, byred, 1, 0);
byLineDefine(A, C, byyellow, 0, 0);
byLineDefine(B, D, byblue, 0, 0);
draw byNamedLineSeq(0)(AB,BD,CD,AC);
draw byLabelsOnPolygon(A, B, D, C)(0, 0);
}
\drawCurrentPictureInMargin
\problemNP{П}{ротивоположные}{стороны и~углы параллелограмма равны, а~диагональ \drawUnitLine{AD} делит его на две равные части.}

\startCenterAlign
Поскольку $\left\{\eqalign{\drawAngle{BAD} &= \drawAngle{CDA}\cr\drawAngle{DAC} &= \drawAngle{ADB}\cr}\right\}$ \inprop[prop:I.XXIX]\\
и \drawUnitLine{AD} общая для \drawLine{AD,CD,AC} и~\drawLine{AB,BD,AD}.

$\therefore \left\{\eqalign{\drawUnitLine{AB} &= \drawUnitLine{CD}\cr \drawUnitLine{AC} &= \drawUnitLine{BD}\cr \drawAngle{B} &= \drawAngle{C}\cr}\right\}$ \inprop[prop:I.XXVI]\\
и $\drawAngle{BAD,DAC} = \drawAngle{CDA,ADB}$ \inax[ax:I.II].
\stopCenterAlign

Следовательно, противоположные стороны и~углы параллелограмма равны. И, поскольку треугольники \drawLine{AD,CD,AC} и~\drawLine{AB,BD,AD} равны во всех отношениях \inprop[prop:I.IV], диагональ делит параллелограмм на две равные части.

\qed
\stopProposition

\startProposition[title={Предл. XXXV. Теорема}, reference=prop:I.XXXV]
\defineNewPicture{
pair A, B, C, D, E, F, G, d[];
d1 := (7/4u, 0);
d2 := (u, -3u);
d3 := (-2u, -3u);
A := (0, 0);
B := A shifted d1;
C := A shifted d2;
D := C shifted d1;
E := C shifted -d3;
F := D shifted -d3;
G := (B--D) intersectionpoint (C--E);
draw byPolygon(A,B,G,C)(byyellow);
draw byPolygon(E,F,D,G)(byyellow);
draw byPolygon(B,E,G)(byred);
draw byPolygon(C,D,G)(byblue);
byAngleDefine(E, A, C, byred, 0);
byAngleDefine(F, B, D, byblue, 0);
byAngleDefine(A, E, C, byblack, 0);
byAngleDefine(B, F, D, white, 0);
draw byNamedAngleResized();
draw byNamedAngleDummySides(BFD);
draw byLine(A, C, byblue, 0, 0);
draw byLine(B, D, byred, 0, 0);
draw byLabelsOnPolygon(C, A, B, E, F, D)(0, 0);
}
\drawCurrentPictureInMargin
\problemNP{П}{араллелограммы,}{ находящиеся на одном и~том же основании и~между одними и~теми же параллельными прямыми, равны по площади между собой.}

\startCenterAlign
Из параллельности прямых следует, что\\
$\left.\eqalign{
\drawAngle{A} &= \drawAngle{B},\cr
\drawAngle{E} &= \drawAngle{F}\cr
\mbox{и } \drawUnitLine{AC} &= \drawUnitLine{BD}.\cr
}\right\}
\eqalign{
&\mbox{\inprop[prop:I.XXIX]}\cr
&\mbox{\inprop[prop:I.XXIX]}\cr
&\mbox{\inprop[prop:I.XXXIV]}\cr
}$

$\therefore \drawFromCurrentPicture[middle][polygonABC]{
startAutoLabeling;
draw byNamedPolygon (ABGC, BEG);
stopAutoLabeling;
draw byNamedLine (AC);
}
=
\drawFromCurrentPicture[middle][polygonEFD]{
startAutoLabeling;
draw byNamedPolygon (EFDG, BEG);
stopAutoLabeling;
draw byNamedLine (BD);
}$ \inprop[prop:I.XXVI].

Но $\drawFromCurrentPicture[middle][polygonAFDC]{
startAutoLabeling;
draw byNamedPolygon (ABGC, EFDG, BEG, CDG);
stopAutoLabeling;
draw byNamedLine (AC);
} - \polygonEFD =
\drawPolygon{ABGC, CDG}$\\
и $\polygonAFDC - \polygonABC =
\drawPolygon{EFDG, CDG}$.

$\therefore \drawPolygon{ABGC, CDG} = \drawPolygon{EFDG, CDG}$.
\stopCenterAlign

\qed
\stopProposition

\startProposition[title={Предл. XXXVI. Теорема}, reference=prop:I.XXXVI]
\defineNewPicture[1/4]{
pair A, B, C, D, E, F, G, H, J, I, d[];
numeric h;
h := 3u;
d1 := (3/2u, 0);
d2 := (2/3u, -h);
d3 := (-8/3u, -h);
d4 := (-1/2u, -h);
A := (0, 0);
B := A shifted d1;
C := A shifted d2;
D := C shifted d1;
E := C shifted -d3;
F := D shifted -d3;
G := E shifted d4;
H := F shifted d4;
I := (B--D) intersectionpoint (C--E);
J := (E--G) intersectionpoint (D--F);
draw byPolygon(A,B,I,C)(byred);
draw byPolygon(C,D,I)(byred);
draw byPolygon(I,D,J,E)(byblue);
draw byPolygon(E,F,J)(byyellow);
draw byPolygon(J,F,H,G)(byyellow);
byLineDefine(C, E, byyellow, 0, 0);
byLineDefine(D, F, byblack, 1, 0);
byLineDefine(C, D, byblack, 0, 0);
byLineDefine(E, F, byred, 0, 0);
draw byNamedLineSeq(0)(CE,EF,DF,CD);
draw byLineFull(G, H, byblue, 0, 0)(E, F, 1, 1, 0);
draw byLabelsOnPolygon(A, B, D, C)(0, 0);
draw byLabelsOnPolygon(E, F, H, G)(0, 0);
}
\drawCurrentPictureInMargin
\problemNP{П}{араллелограммы}{\drawPolygon[bottom][polygonABDC]{ABIC, CDI}~и~\drawPolygon[bottom][polygonEFHG]{EFJ, JFHG} на равных основаниях и~между одними параллельными прямыми равны по площади.}

\startCenterAlign
Проведем \drawUnitLine{CE} и~\drawUnitLine{DF},\\
$\drawUnitLine{CD} = \drawUnitLine{GH} = \drawUnitLine{EF}$ (\hypstr и~\inpropL[prop:I.XXXIV]).

$\therefore \drawUnitLine{CD} = \mbox{ и~} \parallel \drawUnitLine{EF}$.

$\therefore \drawUnitLine{CE} = \mbox{ и~} \parallel \drawUnitLine{DF}$ \inprop[prop:I.XXXIII].

Следовательно,
\drawPolygon[bottom][polygonCDFE]{CDI, IDJE, EFJ}
параллелограмм.

Но $\polygonABDC = \polygonCDFE = \polygonEFHG$ \inprop[prop:I.XXXV].

$\therefore \polygonABDC = \polygonEFHG$ \inax[ax:I.I].
\stopCenterAlign

\qed
\stopProposition

\startProposition[title={Предл. XXXVII. Теорема}, reference=prop:I.XXXVII]
\defineNewPicture{
pair A, B, C, D, E, F, G, H, I, d[];
d1 := (3/2u, 0);
d2 := (3/4u, -5/2u);
d3 := (-7/4u, -5/2u);
A := (0, 0);
B := A shifted d1;
C := A shifted d2;
D := C shifted d1;
E := C shifted -d3;
F := D shifted -d3;
G := (B--D) intersectionpoint (C--E);
H := 11/10[F, A];
I := 11/10[A, F];
draw byPolygon(A,B,C)(byblue);
draw byPolygon(B,C,G)(byyellow);
draw byPolygon(C,D,G)(byyellow);
draw byPolygon(D,G,E)(byblack);
draw byPolygon(E,F,D)(byred);
draw byLine(B, D, byred, 0, 0);
draw byLine(E, C, byblue, 0, 0);
byLineDefine(A, C, byred, 1, 0);
byLineDefine(F, D, byblue, 1, 0);
byLineDefine(C, D, byblack, 0, 0);
draw byNamedLineSeq(0)(AC,CD,FD);
draw byLine(H, I, byblack, 1, 0);
draw byLabelsOnPolygon(H, A, B, E, F, I, noPoint)(0, 0);
draw byLabelsOnPolygon(F, D, C, A)(2, 0);
}
\drawCurrentPictureInMargin
\problemNP{Т}{реугольники}{
\drawPolygon[bottom][polygonBCD]{BCG, CDG} и~\drawPolygon[bottom][polygonCDE]{DGE, CDG}
на одном основании \drawUnitLine{CD} и~между одними и теми же параллельными прямыми равны.}

\startCenterAlign
$\left.\eqalign{\mbox{Проведем} \drawUnitLine{AC} &\parallel \drawUnitLine{BD}\cr
\drawUnitLine{FD} &\parallel \drawUnitLine{EC}}\right\}\mbox{\inprop[prop:I.XXXI].}$

Проведем \drawUnitLine{HI}.

\drawPolygon[bottom][polygonABDC]{ABC, BCG, CDG}
и
\drawPolygon[bottom][polygonEFDC]{DGE, CDG, EFD}
являются параллелограммами на одном основании и~между одними параллельными прямыми и,~следовательно, равны между собой. \inprop[prop:I.XXXV]

$\therefore \left\{\eqalign{\polygonABDC &= \mbox{ дважды } \polygonBCD\cr\polygonEFDC &= \mbox{ дважды } \polygonCDE\cr}\right\}$ \inprop[prop:I.XXXIV].

$\therefore \polygonBCD = \polygonCDE$.
\stopCenterAlign

\qed
\stopProposition

\startProposition[title={Предл. XXXVIII. Теорема}, reference=prop:I.XXXVIII]
\defineNewPicture{
pair A, B, C, D, E, F, G, H, J, I, d[];
numeric h;
h := 5/2u;
d1 := (3/2u, 0);
d2 := (2/3u, -h);
d3 := (-8/3u, -h);
d4 := (-1/2u, -h);
A := (0, 0);
B := A shifted d1;
C := A shifted d2;
D := C shifted d1;
E := C shifted -d3;
F := D shifted -d3;
G := E shifted d4;
H := F shifted d4;
I := (xpart(A), ypart(C));
J := (xpart(F), ypart(C));
draw byPolygon(A,B,C)(byyellow);
draw byPolygon(B,C,D)(byred);
draw byPolygon(E,F,H)(byblack);
draw byPolygon(E,G,H)(byblue);
draw byLine(B, D, byblue, 0, 0);
draw byLine(E, G, byred, 0, 0);
byLineDefine(A, C, byblue, 1, 0);
byLineDefine(F, H, byred, 1, 0);
byLineDefine(A, F, byblack, 1, 0);
draw byNamedLineSeq(0)(AC,AF,FH);
draw byLine(I, J, byblack, 1, 0);
draw byLabelsOnPolygon(A, B, E, F, noPoint)(0, 0);
draw byLabelsOnPolygon(H, G, D, C, noPoint)(0, 0);
}
\drawCurrentPictureInMargin
\problemNP{Т}{реугольники}{\drawPolygon[bottom][polygonBCD]{BCD} и~\drawPolygon[bottom][polygonEGH]{EGH} на равных основаниях и~между теми же параллельными прямыми равны.}

\startCenterAlign
$\left.\eqalign{\mbox{Проведем } \drawUnitLine{AC} &\parallel \drawUnitLine{BD}\cr
\mbox{и } \drawUnitLine{FH} &\parallel \drawUnitLine{EG}}\right\}\mbox{\inprop[prop:I.XXXI].}$

$\drawPolygon[bottom][polygonABDC]{ABC, BCD} = \drawPolygon[bottom][polygonEFHG]{EFH, EGH}$ \inprop[prop:I.XXXVI].

Но $\polygonABDC = \mbox{ дважды } \polygonBCD$ \inprop[prop:I.XXXIV]\\
и $\polygonEFHG = \mbox{ дважды } \polygonEGH$ \inprop[prop:I.XXXIV].

$\therefore \polygonBCD = \polygonEGH$ \inax[ax:I.VII].
\stopCenterAlign

\qed
\stopProposition

\startProposition[title={Предл. XXXIX. Теорема}, reference=prop:I.XXXIX]
\defineNewPicture{
pair A, B, C, D, E, F, G;
A := (0, 0);
B := A shifted (5/2u, 0);
C := A shifted (u, -5/2u);
D := C shifted (3u, 0);
E = whatever[A, D] = whatever[B, C];
F := 11/8[C, B];
G := 13/8[C, B];
draw byPolygon(A,B,E)(byred);
draw byPolygon(A,E,C)(byyellow);
draw byPolygon(B,E,D)(byblack);
draw byPolygon(E,C,D)(byyellow);
draw byPolygon(F,B,D)(byblue);
byLineDefine(A, F, byred, 0, 0);
byLineDefine(D, F, byyellow, 0, 0);
byLineDefine(A, B, byblue, 0, 0);
byLineDefine(C, D, byblack, 0, 0);
byLineDefine(C, G, byblack, 1, 0);
draw byNamedLineSeq(4/5)(AB,AF,DF,CD,CG);
draw byLabelsOnPolygon(F, D, C, A)(2, 0);
draw byLabelsOnPolygon(A, B, F)(2, 0);
draw byLabelsOnPolygon(A, F, G, noPoint)(0, 0);
}
\drawCurrentPictureInMargin
\problemNP{Р}{авные}{треугольники
\drawPolygon[bottom][polygonADC]{AEC, ECD} и~\drawPolygon[bottom][polygonBDC]{BED, ECD}, находящиеся на одном основании \drawUnitLine{CD} и~с одной и~той же стороны, находятся между теми же параллельными прямыми.}

\startCenterAlign
Если \drawUnitLine{AB}, соединяющая вершины треугольников, не $\parallel \drawUnitLine{CD}$, проведем $\drawUnitLine{AF} \parallel \drawUnitLine{CD}$ \inprop[prop:I.XXXI], касающуюся \drawUnitLine{CG}.

Проведем \drawUnitLine{DF}.

Поскольку $\drawUnitLine{AF} \parallel \drawUnitLine{CD}$ (\conststr)\\
$\polygonADC =
\drawPolygon[bottom][polygonFDC]{BED, ECD, FBD}$ \inprop[prop:I.XXXVII].

Но $\polygonADC = \polygonBDC$ (\hypstr).

$\therefore \polygonBDC = \polygonFDC$, часть равна целому, что невозможно.

$\therefore \drawUnitLine{AF} \nparallel \drawUnitLine{CD}$. 

И~таким же образом можно показать, что никакая другая линия, кроме \drawUnitLine{AB},\\
не $\parallel \drawUnitLine{CD}$; $\therefore \drawUnitLine{AB} \parallel \drawUnitLine{CD}$.
\stopCenterAlign

\qed
\stopProposition

\startProposition[title={Предл. XL. Теорема}, reference=prop:I.XL]
\defineNewPicture{
pair A, B, C, D, E, F, G, H, d;
A := (0, 0);
B := A shifted (5/2u, 0);
C := A shifted (-u, -9/4u);
d := (7/4u, 0);
D := C shifted d;
E := B shifted (-1/2u, -9/4u);
F := E shifted d;
G := 5/4[E, B];
H := 2[B, G];
draw byPolygon(A,C,D)(byyellow);
draw byPolygon(B,E,F)(byred);
draw byPolygon(G,B,F)(byblue);
draw byLine(E, H, byblack, 1, 0);
byLineDefine(A, G, byred, 0, 0);
byLineDefine(F, G, byyellow, 0, 0);
byLineDefine(A, B, byblue, 0, 0);
byLineDefine(C, D, byblack, 0, 0);
byLineDefine(E, F, byblack, 0, 0);
byLineDefine(D, E, byblue, 1, 0);
draw byNamedLineSeq(4/5)(AB,AG,FG,EF,DE,CD);
draw byLabelsOnPolygon(G, F, E, D, C, A)(2, 0);
draw byLabelsOnPolygon(A, B, G)(2, 0);
draw byLabelsOnPolygon(A, G, H, noPoint)(0, 0);
}
\drawCurrentPictureInMargin
\problemNP{Р}{авные}{треугольники
\drawFromCurrentPicture[bottom][polygonACD]{
startAutoLabeling;
draw byNamedPolygon (ACD);
stopAutoLabeling;
draw byNamedLineFull(A, A, 1, 1, 0)(CD);
}
и
\drawFromCurrentPicture[bottom][polygonBEF]{
startAutoLabeling;
draw byNamedPolygon (BEF);
stopAutoLabeling;
draw byNamedLineFull(B, B, 1, 1, 0)(EF);
}, находящиеся на равных основаниях и~с одной и~той же стороны, находятся между одними и теми же параллельными прямыми.}

\startCenterAlign
Если \drawSizedLine{AB}, соединяющая вершины треугольников, не $\parallel \drawSizedLine{CD,DE,EF}$,\\
проведем \drawSizedLine{AG} $\parallel \drawSizedLine{CD,DE,EF}$ \inprop[prop:I.XXXI],\\
касающуюся \drawSizedLine{EH}.

Проведем \drawSizedLine{FG}.

Поскольку $\drawSizedLine{AG} \parallel \drawSizedLine{CD,DE,EF}$ (\conststr),\\
$\polygonACD =
\drawFromCurrentPicture[bottom][polygonGEF]{
startAutoLabeling;
draw byNamedPolygon (BEF, GBF);
stopAutoLabeling;
draw byNamedLineFull(G, G, 1, 1, 0)(EF);
}$.

Но $\polygonACD = \polygonBEF$.

$\therefore \polygonBEF = \polygonGEF$, часть равна целому, что невозможно.

$\therefore \drawSizedLine{AG} \nparallel \drawSizedLine{CD,DE,EF}$: и~таким же образом можно показать, что никакая другая линия, кроме \drawSizedLine{AB}, не $\parallel \drawSizedLine{CD,DE,EF}$.

$\therefore \drawSizedLine{AB} \parallel \drawSizedLine{CD,DE,EF}$.
\stopCenterAlign

\qed
\stopProposition

\startProposition[title={Предл. XLI. Теорема}, reference=prop:I.XLI]
\defineNewPicture[1/4]{
pair A, B, C, D, E, F, G, d;
A := (0, 0);
d := (2u, 0);
B := A shifted d;
C := B shifted (2u, 0);
D := A shifted (4/3u, -3u);
E := D shifted d;
F = whatever[B, E] = whatever[D, C];
G = whatever[A, E] = whatever[D, C];
draw byPolygon(A,B,F,G)(byyellow);
draw byPolygon(G,F,E)(byyellow);
draw byPolygon(A,G,D)(byblue);
draw byPolygon(D,E,G)(byblue);
draw byPolygon(C,F,E)(byred);
draw byLine(A, E, byred, 0, 0);
draw byLineFull(A, C, byblack, 1, 0)(D, E, 1, 1, 0);
draw byLineFull(D, E, byblack, 0, 0)(A, C, 1, 1, 0);
draw byLabelsOnPolygon(E, D, A, B, C)(0, 0);
}
\drawCurrentPictureInMargin
\problemNP[3]{Е}{сли}{параллелограмм \drawPolygon[bottom][polygonABED]{ABFG,GFE,AGD,DEG} и~треугольник \drawPolygon[bottom][polygonCED]{GFE,DEG,CFE} стоят на одном основании \drawUnitLine{DE} и~между теми же параллельными прямыми \drawUnitLine{AC} и~\drawUnitLine{DE}, то параллелограмм вдвое больше треугольника.}

\startCenterAlign
Проведем диагональ \drawUnitLine{AE}.

Тогда $\drawPolygon[bottom][polygonAED]{AGD,DEG} = \polygonCED$ \inprop[prop:I.XXXVII],\\
$\polygonABED = \mbox{ дважды } \polygonAED$ \inprop[prop:I.XXXIV].

$\therefore \polygonABED = \mbox{ дважды } \polygonCED$ \inax[ax:I.VI].
\stopCenterAlign

\qed
\stopProposition

\startProposition[title={Предл. XLII. Задача}, reference=prop:I.XLII]
\defineNewPicture{
pair A, B, C, D, E, F, G, H, I, J, d[];
A := (0, 0);
d1 := (8/5u, 0);
B := A shifted d1;
C := B shifted (2u, 0);
D := A shifted (u, -13/5u);
E := D shifted d1;
F := (B--E) intersectionpoint (D--C);
G := 2[D, E];
d2 := (-xpart(E)+xpart(D)-1/2u, 0);
H := B shifted d2;
I := E shifted d2;
J := D shifted d2;
draw byPolygon(A,B,F,D)(byyellow);
draw byPolygon(D,E,F)(byyellow);
draw byPolygon(C,F,E)(byblue);
draw byPolygon(E,C,G)(byblack);
byAngleDefine(B, E, D, byblue, 0);
byAngleDefine(H, I, J, byyellow, 0);
setAttribute("angle", "Standalone", "HIJ", 1);
draw byNamedAngleResized();
draw byLine(A, D, byred, 1, 0);
draw byLine(B, E, byred, 0, 0);
draw byLine(C, E, byyellow, 0, 0);
draw byLine(A, C, byblue, 0, 0);
draw byLine(D, E, byblack, 0, 0);
draw byLine(E, G, byblack, 1, 0);
draw byLabelsOnPolygon(G, E, D, A, B, C)(0, 0);
startAutoLabeling;
draw byNamedAngleSidesFull(HIJ)();
stopAutoLabeling;
}
\drawCurrentPictureInMargin
\problemNP{П}{остроить}{параллелограмм, равный данному треугольнику
\drawPolygon[bottom][polygonCDG]{DEF,CFE,ECG} и~с данным углом \drawAngle{I}.}

\startCenterAlign
Сделаем $\drawUnitLine{DE} = \drawUnitLine{EG}$ \inprop[prop:I.X].

Проведем \drawUnitLine{CE}.

Сделаем $\drawAngle{E} = \drawAngle{I}$ \inprop[prop:I.XXIII].

Проведем $\left\{\eqalign{\drawUnitLine{AD} &\parallel \drawUnitLine{BE}\cr\drawUnitLine{AC} &\parallel \drawUnitLine{DE}\cr}\right\}$ \inprop[prop:I.XXXI].

$\drawPolygon[bottom][polygonABED]{ABFD,DEF}
= \mbox{ дважды }
\drawPolygon[bottom][polygonCED]{DEF,CFE}$ \inprop[prop:I.XLI],\\
но $\polygonCED =
\drawPolygon[bottom][polygonDCG]{ECG}$ \inprop[prop:I.XXXVIII].

$\therefore \polygonABED = \polygonCDG$ \inax[ax:I.VI].
\stopCenterAlign

\qed
\stopProposition

\startProposition[title={Предл. XLIII. Теорема}, reference=prop:I.XLIII]
\defineNewPicture[2/5]{
pair A, B, C, D, E, F, G, H, I, d[];
path q[];
d1 := (5/2u, 0);
d2 := (-u, -3u);
A := (0, 0);
B := A shifted d1;
C := A shifted d2;
D := C shifted d1;
E := 2/5[A, D];
q1 := (E shifted d1) -- (E shifted -d1);
q2 := (E shifted d2) -- (E shifted -d2);
F := q1 intersectionpoint (A--C);
G := q1 intersectionpoint (B--D);
H := q2 intersectionpoint (A--B);
I := q2 intersectionpoint (C--D);
draw byPolygon(A,E,H)(byyellow);
draw byPolygon(A,E,F)(byyellow);
draw byPolygon(H,B,G,E)(byblue);
draw byPolygon(F,C,I,E)(byblack);
draw byPolygon(I,D,E)(byred);
draw byPolygon(G,D,E)(byred);
draw byLabelsOnPolygon(C, F, A, H, B, G, D, I)(0, 0);
draw byLabelsOnPolygon(F, E, H)(2, 0);
}
\drawCurrentPictureInMargin
\problemNP{Д}{ополнения}{\drawPolygon[bottom][polygonHBGE]{HBGE} и~\drawPolygon[bottom][polygonFCIE]{FCIE} параллелограммов на одной диагонали параллелограмма равны между собой.}

\startCenterAlign
$\drawPolygon[bottom][polygonADC]{AEF,FCIE,IDE} = \drawPolygon[bottom][polygonABD]{AEH,HBGE,GDE}$ \inprop[prop:I.XXXIV]\\
и $\drawPolygon[bottom][polygonAEFpIDE]{AEF,IDE} = \drawPolygon[bottom][polygonAEHpGDE]{AEH,GDE}$ \inprop[prop:I.XXXIV].

$\therefore \polygonFCIE = \polygonHBGE$ \inax[ax:I.III].
\stopCenterAlign

\qed
\stopProposition

\startProposition[title={Предл. XLIV. Задача}, reference=prop:I.XLIV]
\defineNewPicture{
pair A, B, C, D, E, F, G, H, I, J, K, L, M, N, O, d[];
path q[];
d1 := (3u, 0);
d2 := (-3/2u, -3u);
d3 := (3/2u, 5/2u);
d4 := -d2 +1/2d1;
A := (0, 0);
B := A shifted d1;
C := A shifted d2;
D := C shifted d1;
E := 2/5[C, B];
q1 := (E shifted d1) -- (E shifted -d1);
q2 := (E shifted d2) -- (E shifted -d2);
F := q1 intersectionpoint (A--C);
G := q1 intersectionpoint (B--D);
H := q2 intersectionpoint (A--B);
I := q2 intersectionpoint (C--D);
J := A shifted d3;
K := J shifted (2(xpart(A)-xpart(H)), 0);
L := (xpart(1/3[J, K]), ypart(F)-ypart(A)+ypart(J));
M := A shifted d4;
N := F shifted d4;
O := E shifted d4;
draw byPolygon(J,K,L)(byred);
draw byPolygon(A,H,E,F)(byyellow);
draw byPolygon(E,G,D,I)(byblue);
byAngleDefine(A, F, E, byblue, 0);
byAngleDefine(F, E, I, byred, 0);
byAngleDefine(E, I, D, byblack, 0);
byAngleDefine(M, N, O, byyellow, 0);
setAttribute("angle", "Standalone", "MNO", 1);
draw byNamedAngleResized();
draw byLine(B, E, byred, 0, 0);
draw byLine(E, C, byblack, 0, 1);
byLineDefine(A, F, byred, 1, 0);
byLineDefine(F, C, byblack, 0, 1);
draw byLine(H, E, byblue, 1, 0);
byLineDefine(B, G, byyellow, 0, 0);
byLineDefine(A, H, byblue, 0, 0);
byLineDefine(H, B, byblack, 0, 0);
byLineDefine(F, E, byblack, 1, 0);
byLineDefine(E, G, byblack, 0, 0);
byLineDefine(C, D, byyellow, 1, 0);
draw byNamedLineSeq(0)(FE,EG,BG,HB,AH,AF,FC,CD);
draw byLabelsOnPolygon(K, J, L)(0, 0);
draw byLabelsOnPolygon(F, A, H, B, G, D, I, C)(0, 0);
draw byLabelsOnPolygon(F, E, H)(2, 0);
startAutoLabeling;
draw byNamedAngleSidesFull(MNO)();
stopAutoLabeling;
}
\drawCurrentPictureInMargin
\problemNP[3]{К}{ данной}{прямой линии \drawUnitLine{EG} приложить параллелограмм, равный данному треугольнику \drawPolygon[middle][polygonJKL]{JKL}, и~с углом, равным данному прямолинейному углу \drawAngle{N}.}

\startCenterAlign
Сделаем $\drawPolygon[middle][polygonAHEF]{AHEF} = \polygonJKL$ с~$\drawAngle{F} = \drawAngle{N}$ \inprop[prop:I.XLII]\\
и со стороной \drawUnitLine{FE}, смежной и~являющейся продолжением \drawUnitLine{EG}.

Продлим \drawUnitLine{AH} до $\drawUnitLine{BG} \parallel \drawUnitLine{HE}$\\
проведем \drawUnitLine{BE}, продлим ее до продолжения \drawUnitLine{AF};\\
проведем $\drawUnitLine{CD} \parallel \drawUnitLine{FE,EG}$ до продолжения  \drawUnitLine{BG} и~продлим \drawUnitLine{HE}.

$\polygonAHEF = \drawPolygon[middle][polygonEGDI]{EGDI}$ \inprop[prop:I.XLIII],\\
но $\polygonAHEF = \polygonJKL$ (\conststr).

$\therefore \polygonEGDI = \polygonJKL$.

И $\drawAngle{F} = \drawAngle{E} =\drawAngle{I} = \drawAngle{N}$ (\inpropL[prop:I.XXIX] и~\conststr).
\stopCenterAlign

\qed
\stopProposition

\startProposition[title={Предл. XLV. Задача}, reference=prop:I.XLV]
\defineNewPicture{
pair A, B, C, D, E, F, G, H, I, J, K, L, M, N, O, P, d[];
numeric a, h[], b[], s[];
a := 15;
A := (0, 0);
B := A shifted (0, 2u);
C := A shifted (4/3u, u);
D := A shifted (2u, -3/2u);
E := A shifted (-6/5u, -u);
b1 := arclength(B--C);
h1 := distanceToLine(A, B--C);
s1 := (b1 * h1)/2;
b2 := arclength(C--D);
h2 := distanceToLine(A, C--D);
s2 := (b2 * h2)/2;
b3 := arclength(D--E);
h3 := distanceToLine(A, D--E);
s3 := (b3 * h3)/2;
d1 := (0, ypart(D)-u);
d2 := (0, -b3/2) rotated -a;
d3 := (h3*(1/cosd(a)), 0);
d6 := (u, 0);
F := (-u, 0) shifted d1;
G := F shifted d3;
H := F shifted d2;
I := G shifted d2;
d4 := (2*(s2/b3)*(1/cosd(a)), 0);
J := G shifted d4;
K := J shifted d2;
d5 := (2*(s1/b3)*(1/cosd(a)), 0);
L := J shifted d5;
M := L shifted d2;
N := L shifted d6;
O := M shifted d6;
P := K shifted d6;
draw byPolygon(A,B,C)(byred);
draw byPolygon(A,C,D)(byyellow);
draw byPolygon(A,D,E)(byblue);
byLineDefine(A, C, byblue, 0, 0);
byLineDefine(A, D, byred, 0, 0);
draw byNamedLineSeq(0)(AC,AD);
draw byPolygon(F,G,I,H)(byblue);
draw byPolygon(G,J,K,I)(byyellow);
draw byPolygon(J,L,M,K)(byred);
byAngleDefine(G, I, H, byyellow, 0);
byAngleDefine(J, K, I, byblack, 0);
byAngleDefine(L, M, K, byblue, 0);
byAngleDefine(N, O, P, byred, 0);
setAttribute("angle", "Standalone", "NOP", 1);
draw byNamedAngleResized();
draw byLine(G, I, byred, 0, 0);
draw byLine(J, K, byblue, 0, 0);
draw byLabelsOnPolygon(A, B, C, D, E)(0, 0);
draw byLabelsOnPolygon(F, G, J, L, M, K, I, H)(0, 0);
startAutoLabeling;
draw byNamedAngleSidesFull(NOP)();
stopAutoLabeling;
}
\drawCurrentPictureInMargin
\problemNP[2]{П}{остроить}{параллелограмм, равный данной прямолинейной фигуре \drawPolygon[middle][polygonABCDE]{ABC,ACD,ADE}, в~угле, равном данному прямолинейному углу \drawAngle{O}.}

\startCenterAlign
Проведем \drawUnitLine{AD} и~\drawUnitLine{AC}, делящие прямолинейную фигуру на треугольники.

Построим $\drawPolygon{FGIH} = \drawPolygon{ADE}$\\
с $\drawAngle{I} = \drawAngle{O}$ \inprop[prop:I.XLII].\\
К \drawUnitLine{GI} приложим $\drawPolygon{GJKI} = \drawPolygon{ACD}$\\
с $\drawAngle{K} = \drawAngle{O}$ \inprop[prop:I.XLIV].\\
К \drawUnitLine{JK} приложим $\drawPolygon{JLMK} = \drawPolygon{ABC}$\\
с $\drawAngle{M} = \drawAngle{O}$ \inprop[prop:I.XLIV].\\
$\therefore \drawPolygon[middle][polygonFLMH]{FGIH,GJKI,JLMK} = \polygonABCDE$\\
и \polygonFLMH является параллелограммом (\inpropL[prop:I.XXIX], \inpropN[prop:I.XIV], \inpropN[prop:I.XXX])\\
с $\drawAngle{M} = \drawAngle{O}$.
\stopCenterAlign

\qed
\stopProposition

\startProposition[title={Предл. XLVI. Задача}, reference=prop:I.XLVI]
\defineNewPicture{
pair A, B, C, D;
numeric d;
d := 7/2u;
A := (0, 0);
B := A shifted (d, 0);
C := A shifted (0, -d);
D := A shifted (d, -d);
byAngleDefine(B, A, C, byblack, 0);
byAngleDefine(D, B, A, byblue, 0);
byAngleDefine(C, D, B, byred, 0);
byAngleDefine(A, C, D, byyellow, 0);
draw byNamedAngleResized();
byLineDefine(A, B, byred, 0, 0);
byLineDefine(B, D, byyellow, 0, 0);
byLineDefine(D, C, byblack, 0, 0);
byLineDefine(C, A, byblue, 0, 0);
draw byNamedLineSeq(0)(AB,BD,DC,CA);
draw byLabelsOnPolygon(A, B, D, C)(0, 0);
}
\drawCurrentPictureInMargin
\problemNP{Н}{а}{данной прямой \drawUnitLine{DC} построить квадрат.}

\startCenterAlign
Проведем $\drawUnitLine{CA} \perp \mbox{ и~} = \drawUnitLine{DC}$ (\inpropL[prop:I.XI], \inpropN[prop:I.III]).

Проведем $\drawUnitLine{AB} \parallel \drawUnitLine{DC}$ и~касающуюся \drawUnitLine{BD}, проведенную $\parallel \drawUnitLine{CA}$.

В
\drawFromCurrentPicture[bottom][polygonABDC]{
startTempAngleScale(angleScale*3/4);
draw byNamedAngle(A,B,C,D);
draw byNamedLineSeq(0)(AB,BD,DC,CA);
draw byLabelsOnPolygon(A, B, D, C)(0, 0);
stopTempAngleScale;
}
$\drawUnitLine{CA} = \drawUnitLine{DC}$ (\conststr),\\
$\drawAngle{C} = \drawRightAngle$ (\conststr).

$\therefore \drawAngle{D} = \drawAngle{C} = \mbox{прямому углу}$ \inprop[prop:I.XXIX], и~оставшиеся стороны и~углы должны быть равны \inprop[prop:I.XXXIV].

И $\therefore \polygonABDC$ является квадратом \indef[def:I.XXX].
\stopCenterAlign

\qed
\stopProposition

\startProposition[title={Предл. XLVII. Теорема}, reference=prop:I.XLVII]
\defineNewPicture[1/2]{
pair A, B, C, D, E, F, G, H, I, J, K, L, M, d[];
%byPointLabelDefine(A, "α");
%byPointLabelDefine(B, "β");
%byPointLabelDefine(C, "γ");
%byPointLabelDefine(D, "δ");
%byPointLabelDefine(E, "ε");
%byPointLabelDefine(F, "ζ");
%byPointLabelDefine(G, "η");
%byPointLabelDefine(H, "θ");
%byPointLabelDefine(I, "ι");
%byPointLabelDefine(J, "κ");
%byPointLabelDefine(K, "λ");
%byPointLabelDefine(L, "μ");
%byPointLabelDefine(M, "ν");
A := (0, 0);
B := A shifted (-7/10u, -8/7u);
C = whatever[A, A shifted ((A-B) rotated 90)] = whatever[B, B shifted dir(0)];
d1 := (B-A) rotated -90;
D := A shifted d1;
E := B shifted d1;
d2 := (A-C) rotated -90;
F := C shifted d2;
G := A shifted d2;
d3 := (C-B) rotated -90;
H := B shifted d3;
I := C shifted d3;
J = whatever[A, A shifted dir(90)];
J = whatever[B, C];
K = whatever[A, A shifted dir(90)];
K = whatever[H, I];
L = whatever[B, F];
L = whatever[A, C];
M = whatever[A, I];
M = whatever[B, C];
draw byPolygon(A,B,E,D)(byblack);
draw byPolygon(L,A,G,F)(byred);
draw byPolygon(C,L,F)(byred);
draw byPolygon(J,M,I,K)(byblue);
draw byPolygon (M,C,I)(byblue);
draw byPolygon(B,J,K,H)(byyellow);
byAngleDefine(F, C, A, byyellow, 0);
byAngleDefine(B, C, I, byyellow, 0);
byAngleDefine(A, C, B, byblack, 0);
draw byNamedAngleResized();
draw byLine(A, K, byblack, 1, 0);
draw byLineFull(B, F, byblack, 0, 0)(G, G, 1, 1, -1);
draw byLineFull(A, I, byblack, 0, 0)(K, K, 1, 1, 1);
byLineDefine(C, F, byblue, 1, 0);
byLineDefine(C, I, byred, 1, 0);
draw byNamedLineSeq(0)(CF,CI);
byLineDefine(A, B, byyellow, 0, 0);
byLineDefine(B, C, byred, 0, 0);
byLineDefine(C, A, byblue, 0, 0);
draw byNamedLineSeq(-1)(AB,BC,CA);
byLineDefine.CAb(C, A, byblack, 0, 0);
byLineStylize (M, M, 1, 0, -1) (CAb);
byLineDefine.AMb(A, M, byblack, 0, 0);
byLineStylize (C, C, 0, 1, -1) (AMb);
byLineDefine.BCb(B, C, byblack, 0, 0);
byLineStylize (L, L, 0, 1, -1) (BCb);
byLineDefine.BLb(L, B, byblack, 0, 0);
byLineStylize (C, C, 1, 0, -1) (BLb);
draw byLabelsOnPolygon(B, E, D, A, G, F, C, I, K, H)(0, 1);
draw byLabelsOnPolygon(A, J, C)(2, -1);
}
\drawCurrentPictureInMargin
\problemNP{В}{прямоугольном}{треугольнике \drawLine[bottom][triangleABC]{CA,BC,AB} квадрат гипотенузы \drawUnitLine{BC} равен сумме квадратов катетов \drawUnitLine{CA} и~\drawUnitLine{AB}.}

\startCenterAlign
На \drawUnitLine{BC}, \drawUnitLine{CA}, \drawUnitLine{AB} построим квадраты \inprop[prop:I.XLVI].

Проведем $\drawUnitLine{AK} \parallel \drawUnitLine{CI}$ \inprop[prop:I.XXXI],\\
также проведем \drawUnitLine{BF} и~\drawUnitLine{AI}.

$\drawAngle{BCI} = \drawAngle{FCA}$.

К каждому добавим \drawAngle{ACB},\\
$\therefore \drawAngle{BCI,ACB} = \drawAngle{FCA,ACB}$.

Вместе с~тем $\drawUnitLine{BC} = \drawUnitLine{CI}$ и~$\drawUnitLine{CA} = \drawUnitLine{CF}$.

$\therefore
\drawFromCurrentPicture[middle][polygonAFC]{
startTempAngleScale(2/3);
draw byNamedPolygon(MCI);
draw byNamedAngle(ACB,BCI);
draw byNamedLine(CAb,AMb);
draw byLabelsOnPolygon(I,A,C)(1, 1);
stopTempAngleScale;
}
=
\drawFromCurrentPicture[middle][polygonBLC]{
startTempAngleScale(2/3);
draw byNamedPolygon(CLF);
draw byNamedAngle(ACB,FCA);
draw byNamedLine(BCb,BLb);
draw byLabelsOnPolygon(B,F,C)(1, 1);
stopTempAngleScale;
}
$ \inprop[prop:I.IV].

Теперь, поскольку $\drawUnitLine{AB} \parallel \drawUnitLine{CF}$,\\
$\drawPolygon[middle][polygonACFG]{LAGF,CLF} = \mbox{ дважды } \polygonBLC$\\
и $\drawPolygon[middle][polygonJMCK]{JMIK,MCI} = \mbox{ дважды } \polygonAFC$ \inprop[prop:I.XLI].

$\therefore \polygonACFG = \polygonJMCK$ \inax[ax:I.VI].

Так же можно показать, что $\drawPolygon[middle][polygonABED]{ABED} = \drawPolygon[middle][polygonBJKH]{BJKH}$.

А значит, $\drawPolygon[middle][polygonABEDpACFG]{ABED,LAGF,CLF} = \drawPolygon[middle][polygonBCIH]{JMIK,MCI,BJKH}$ \inax[ax:I.II].

\stopCenterAlign

\qed
\stopProposition

\startProposition[title={Предл. XLVIII. Теорема}, reference=prop:I.XLVIII]
\defineNewPicture{
pair A, B, C, D;
numeric d;
d := 7/4u;
A := (0, 0);
B := A shifted (0, 3u);
C := A shifted (d, 0);
D := A shifted (-d, 0);
byAngleDefine(B, A, C, byred, 0);
byAngleDefine(D, A, B, byyellow, 0);
draw byNamedAngleResized();
draw byLine(A, B, byblue, 0, 0);
byLineDefine(A, C, byblack, 0, 0);
byLineDefine(A, D, byblack, 1, 0);
byLineDefine(B, C, byred, 0, 0);
byLineDefine(B, D, byred, 1, 0);
draw byNamedLineSeq(0)(AC,AD,BD,BC);
draw byLabelsOnPolygon(D, B, C, A)(0, 0);
}
\drawCurrentPictureInMargin
\problemNP{Е}{сли}{в треугольнике квадрат одной стороны \drawUnitLine{BC} равен сумме квадратов двух других сторон \drawUnitLine{AB} и~\drawUnitLine{AC}, то угол \drawAngle{BAC}, заключенный между этими двумя сторонами, прямой.}

\startCenterAlign
Проведем $\drawUnitLine{AD} \perp \drawUnitLine{AB}$ и~$= \drawUnitLine{AC}$ (\inpropL[prop:I.XI], \inpropN[prop:I.III]),\\
также проведем \drawUnitLine{BD}.

Поскольку $\drawUnitLine{AD} = \drawUnitLine{AC}$ (\conststr),\\
$\drawUnitLine{AD}^2 = \drawUnitLine{AC}^2$.

$\therefore \drawUnitLine{AD}^2 + \drawUnitLine{AB}^2 = \drawUnitLine{AC}^2 + \drawUnitLine{AB}^2$.

Но $\drawUnitLine{AD}^2 + \drawUnitLine{AB}^2 = \drawUnitLine{BD}^2$ \inprop[prop:I.XLVII],\\
и $\drawUnitLine{AC}^2 + \drawUnitLine{AB}^2 = \drawUnitLine{BC}^2$ (\hypstr).

$\therefore \drawUnitLine{BD}^2 = \drawUnitLine{BC}^2$.

$\therefore \drawUnitLine{BD} = \drawUnitLine{BC}$.

И $\therefore \drawAngle{DAB} = \drawAngle{BAC}$ \inprop[prop:I.VIII].

Следовательно, \drawAngle{BAC} — прямой угол.
\stopCenterAlign

\qed

\stopProposition
\stopBook

\startBook[title={Книга II}]
\startDefinition[title={Определение I},reference=def:II.I]

\defineNewPicture{
pair A, B, C, D;
numeric w, h;
w := 7/2u;
h := 3u;
A := (0, 0);
B := (w, 0);
C := (0, h);
D := (w, h);
draw byPolygon(A,B,D,C)(byblue);
byLineDefine(A, B, byblack, 0, 0);
byLineDefine(A, C, byred, 0, 0);
draw byNamedLineSeq(0)(AB,AC);
draw byLabelsOnPolygon(A, C, D, B)(0, 0);
}
\drawCurrentPictureInMargin
\problemNP{О}{ всяком}{прямоугольнике или прямоугольном параллелограмме говорят, что он заключается между любыми двумя своими смежными сторонами.}

Таким образом, про прямоугольный параллелограмм \drawPolygon{ABDC} можно сказать, что он заключен между сторонами \drawUnitLine{AB} и~\drawUnitLine{AC}, что можно записать короче в~виде $\drawUnitLine{AB} \cdot \drawUnitLine{AC}$.

Если смежные стороны равны, т. е. $\drawUnitLine{AB} = \drawUnitLine{AC}$, то $\drawUnitLine{AB} \cdot \drawUnitLine{AC}$, обозначает прямоугольник, заключенный между \drawUnitLine{AB} и~\drawUnitLine{AC}, являющийся квадратом\\
и равный $\left\{\eqalign{
\drawUnitLine{AB} \cdot \drawUnitLine{AC}&\mbox{ или } \drawUnitLine{AC}^2\cr
\drawUnitLine{AB} \cdot \drawUnitLine{AC}&\mbox{ или } \drawUnitLine{AC}^2}\right.$

\stopDefinition

\vfill\pagebreak

\startDefinition[title={Определение II},reference=def:II.II]

\defineNewPicture{
pair A, B, C, D, E, F, G, H, I, d[];
d1 := (3u, 0);
d2 := (1/2u, 3u);
A := (0, 0);
B := A shifted d1;
C := A shifted d2;
D := A shifted d1 shifted d2;
E := 2/3[A, B];
F := 2/3[C, D];
G := 2/3[A, C];
H := 2/3[B, D];
I = whatever[E, F] = whatever[G, H];
draw byPolygon(A,E,I,G)(byblue);
draw byPolygon(E,B,H,I)(byyellow);
draw byPolygon(G,I,F,C)(byyellow);
draw byPolygon(I,H,D,F)(byred);
draw byLabelsOnPolygon(C, F, D, H, B, E, A, G)(0,0);
draw byLabelsOnPolygon(G, I, F)(2,0);
}
\drawCurrentPictureInMargin
\problemNP{В}{о}{ всяком параллелограмме фигура, образованная  одним из параллелограммов на его диаметре вместе с~двумя дополнениями, называется гномоном.}

\noindent Так, \drawPolygon{AEIG,EBHI,GIFC} и~\drawPolygon{EBHI,GIFC,IHDF} являются гномонами.
\stopDefinition

\vfill\pagebreak

\startProposition[title={Предл. I. Теорема},reference=prop:II.I]
\defineNewPicture[1/4]{
pair B, C, D, E, G, H, K, L, M, N;
numeric w, h;
w := 7/2u;
h := 3u;
G := (0, 0);
H := (w, 0);
B := (0, h);
C := (w, h);
K := 2/5[G, H];
D := 2/5[B, C];
L := 3/4[G, H];
E := 3/4[B, C];
M := G shifted (0, -2/3u);
N := M shifted (h, 0);
draw byPolygon(G,K,D,B)(byyellow);
draw byPolygon(K,L,E,D)(byblue);
draw byPolygon(L,H,C,E)(byred);
draw byLine(K, D, byblack, 1, 0);
draw byLine(L, E, byblack, 1, 0);
byLineDefine(G, B, byblack, 0, 0);
draw byLineWithName(M, N, byblack, 0, 0)(A);
byLineDefine(H, C, byblack, 1, 0);
byLineDefine(G, K, byblue, 0, 0);
byLineDefine(K, L, byred, 0, 0);
byLineDefine(L, H, byyellow, 0, 0);
byLineDefine(B, D, byblue, 1, 0);
byLineDefine(D, E, byred, 1, 0);
byLineDefine(E, C, byyellow, 1, 0);
draw byNamedLineSeq(0)(GK,KL,LH,HC,EC,DE,BD,GB);
draw byLabelsOnPolygon(B, D, E, C, H, L, K, G)(0, 0);
draw byLabelLine(0)(A);
}
\drawCurrentPictureInMargin
\problemNP{П}{рямоугольник,}{заключенный между двумя прямыми линиями, одна из которых рассечена на сколько угодно отрезков, равен сумме прямоугольников, заключенных между нерассеченной прямой и~каждым из этих отрезков.\\
$\drawProportionalLine{GK,KL,LH} \cdot \drawProportionalLine{A} = \left\{\eqalign{
&\drawProportionalLine{A} \cdot \drawProportionalLine{GK}\cr
+&\drawProportionalLine{A} \cdot \drawProportionalLine{KL}\cr
+&\drawProportionalLine{A} \cdot \drawProportionalLine{LH}}\right.$}

\startCenterAlign
Проведем $\drawProportionalLine{GB} \perp \drawProportionalLine{GK,KL,LH} \mbox{ и~} = \drawProportionalLine{A}$ (\inpropL[prop:I.II], \inpropL[prop:I.III]).

Достроим параллелограммы, то есть\\
проведем $\left\{\eqalign{
\drawProportionalLine{BD,DE,EC} &\parallel \drawProportionalLine{GK,KL,LH} \cr
\vcenter{
\nointerlineskip\hbox{\drawProportionalLine{KD}}
\nointerlineskip\hbox{\drawProportionalLine{LE}}
\nointerlineskip\hbox{\drawProportionalLine{HC}}} &\parallel \drawProportionalLine{GB}
}\right\}$ \inprop[prop:I.XXXI].

$\drawPolygon{GKDB,KLED,LHCE} =
\drawPolygon{GKDB} +
\drawPolygon{KLED} +
\drawPolygon{LHCE}$\\
$\drawPolygon{GKDB,KLED,LHCE} = \drawProportionalLine{GK,KL,LH} \cdot \drawProportionalLine{GB}$\\
$\polygonGKDB = \drawProportionalLine{GK} \cdot \drawProportionalLine{GB}$,
$\polygonKLED = \drawProportionalLine{KL} \cdot \drawProportionalLine{GB}$,\\
$\polygonLHCE = \drawProportionalLine{LH} \cdot \drawProportionalLine{GB}$.

$\therefore \drawProportionalLine{GK,KL,LH} \cdot \drawProportionalLine{A} = \drawProportionalLine{GK} \cdot \drawProportionalLine{A} + \drawProportionalLine{KL} \cdot \drawProportionalLine{A} + \drawProportionalLine{LH} \cdot \drawProportionalLine{A}$.
\stopCenterAlign

\qed
\stopProposition

\startProposition[title={Предл. II. Теорема},reference=prop:II.II]
\defineNewPicture[1/4]{
pair A, B, C, D, E, F;
numeric w;
w := 7/2u;
A := (0, w);
B := (w, w);
C := 2/3[A, B];
D := (0, 0);
E := (w, 0);
F := 2/3[D, E];
draw byPolygon(A,C,F,D)(byred);
draw byPolygon(C,B,E,F)(byyellow);
draw byLine(C, F, byblack, 0, 0);
byLineDefine(A, D, byblack, 1, 0);
byLineDefine(B, E, byblack, 1, 0);
byLineDefine(A, C, byblue, 0, 0);
byLineDefine(C, B, byred, 0, 0);
draw byNamedLineSeq(0)(AD,AC,CB,BE);
draw byLabelsOnPolygon(A, C, B, E, F, D)(0, 0);
}
\drawCurrentPictureInMargin
\problemNP{Е}{сли}{прямая линия \drawProportionalLine{AC,CB} как-либо рассечена, квадрат всей линии равен сумме прямоугольников, заключенных между целой линией и~каждой из ее частей.\\
$\drawProportionalLine{AC,CB}^2 = \left\{\eqalign{
& \drawProportionalLine{AC,CB} \cdot \drawProportionalLine{AC} \cr
+ & \drawProportionalLine{AC,CB} \cdot \drawProportionalLine{CB}
}\right.$
}

\startCenterAlign
Опишем \drawPolygon[middle][polygonABED]{ACFD,CBEF} \inprop[prop:I.XLVI].

Проведем \drawProportionalLine{CF} параллельную \drawProportionalLine{AD} \inprop[prop:I.XXXI].

$\polygonABED = \drawProportionalLine{AC,CB}^2$.

$\drawPolygon{ACFD} = \drawProportionalLine{CF} \cdot \drawProportionalLine{AC} = \drawProportionalLine{AC,CB} \cdot \drawProportionalLine{AC}$.

$\drawPolygon{CBEF} = \drawProportionalLine{CF} \cdot \drawProportionalLine{CB} = \drawProportionalLine{AC,CB} \cdot \drawProportionalLine{CB}$.

$\polygonABED = \drawPolygon{ACFD} + \drawPolygon{CBEF}$.

$\therefore \drawProportionalLine{AC,CB}^2 = \drawProportionalLine{AC,CB} \cdot \drawProportionalLine{AC} + \drawProportionalLine{AC,CB} \cdot \drawProportionalLine{CB}$.
\stopCenterAlign

\qed
\stopProposition

\startProposition[title={Предл. III. Теорема},reference=prop:II.III]
\defineNewPicture{
pair A, B, C, D, E, F;
numeric w, h;
w := -4u;
h := 11/4u;
A := (0, h);
B := (w, h);
C := (w+h, h);
D := (w+h, 0);
E := (w, 0);
F := (0, 0);
draw byPolygon(A,C,D,F)(byyellow);
draw byPolygon(C,B,E,D)(byred);
byLineDefine(D, F, byred, 0, 0);
byLineDefine(B, C, byblue, 0, 0);
byLineDefine(C, D, byblue, 0, 0);
byLineDefine(D, E, byblue, 0, 0);
byLineDefine(E, B, byblue, 0, 0);
draw byNamedLineSeq(0)(CD,noLine,DF,DE,EB,BC);
draw byLabelsOnPolygon(B, C, A, F, D, E)(0, 0);
}
\drawCurrentPictureInMargin
\problemNP{Е}{сли}{прямая линия \drawProportionalLine{DE,DF} как-либо рассечена, то прямоугольник, заключенный между всей прямой и~ее частью, равен квадрату этой части, вместе с~прямоугольником, заключенным между частями.\\
$\drawProportionalLine{DE,DF} \cdot \drawProportionalLine{DE} = \drawProportionalLine{DE}^2 + \drawProportionalLine{DE} \cdot \drawProportionalLine{DF}$, или \\
$\drawProportionalLine{DE,DF} \cdot \drawProportionalLine{DF} = \drawProportionalLine{DF}^2 + \drawProportionalLine{DE} \cdot \drawProportionalLine{DF}$.}

\startCenterAlign
Опишем \drawPolygon{CBED} \inprop[prop:I.XLVI].

Опишем \drawPolygon{ACDF} \inprop[prop:I.XXXI].

Тогда $\drawPolygon[middle][polygonABEF]{ACDF,CBED} = \polygonCBED + \polygonACDF$, но\\
$\polygonABEF = \drawProportionalLine{DE,DF} \cdot \drawProportionalLine{DE}$\\
и $\polygonCBED = \drawProportionalLine{DE}^2$, $\polygonACDF = \drawProportionalLine{DE} \cdot \drawProportionalLine{DF}$.

$\therefore \drawProportionalLine{DE,DF} \cdot \drawProportionalLine{DE} = \drawProportionalLine{DE}^2 + \drawProportionalLine{DE} \cdot \drawProportionalLine{DF}$.

Таким же образом можно показать, что $\drawProportionalLine{DE,DF} \cdot \drawProportionalLine{DF} = \drawProportionalLine{DF}^2 + \drawProportionalLine{DE} \cdot \drawProportionalLine{DF}$.
\stopCenterAlign

\qed
\stopProposition

\startProposition[title={Предл. IV. Теорема},reference=prop:II.IV]
\defineNewPicture[1/4]{
pair A, B, C, D, E, F, G, H, K;
numeric w;
w := 4u;
A := (0, w);
B := (w, w);
C :=2/3[A, B];
D := (0, 0);
E := (w, 0);
F := 2/3[D, E];
H := 2/3[D, A];
K := 2/3[E, B];
G = whatever[H, K] = whatever[F, C];
draw byPolygon(A,C,G,H)(byyellow);
draw byPolygon(G,K,E,F)(byyellow);
draw byPolygon(D,H,G)(byblue);
draw byPolygon(G,B,C)(byred);
byAngleDefine(F, D, G, byyellow, 0);
byAngleDefine(F, G, D, byred, 0);
byAngleDefine(K, G, B, byblack, 0);
byAngleDefine(G, B, K, byblue, 0);
draw byNamedAngleResized();
draw byLine(H, G, byred, 1, 0);
draw byLine(G, K, byred, 0, 0);
draw byLine(C, G, byblue, 1, 0);
draw byLine(F, G, byblue, 0, 0);
byLineDefine(E, K, byblue, 0, 0);
byLineDefine(F, E, byred, 0, 0);
byLineDefine(D, F, byblue, 0, 0);
byLineDefine(K, B, byred, 0, 0);
byLineDefine(G, D, byblack, 0, 0);
byLineDefine(B, G, byblack, 1, 0);
draw byNamedLineSeq(-1)(BG,GD,DF,FE,EK,KB);
byLineDefine(A, D, byblack, 0, 0);
byLineStylize(B, E, 0, 0, -1)(AD);
byLineDefine(B, A, byblack, 0, 0);
byLineStylize(E, D, 0, 0, -1)(BA);
draw byLabelsOnPolygon(A, C, B, K, E, F, D, H)(0, 0);
draw byLabelsOnPolygon(H, G, C)(2, 0);
}
\drawCurrentPictureInMargin
\problemNP{Е}{сли}{прямая как-либо рассечена \drawProportionalLine{DF,FE}, квадрат всей прямой равен квадратам частей, с~дважды взятым прямоугольником, заключенным между частями.\\
$\drawProportionalLine{DF,FE}^2 = \drawProportionalLine{DF}^2 + \drawProportionalLine{FE}^2 + \mbox{дважды} \drawProportionalLine{DF} \cdot \drawProportionalLine{FE}$
}

\startCenterAlign
Опишем \drawLine[middle][squareABED]{AD,BA,KB,EK,FE,DF} \inprop[prop:I.XLVI],
проведем \drawProportionalLine{BG,GD} \inpost[post:I.I]\\
и $\left\{\eqalign{
\drawProportionalLine{FG,CG} &\parallel \drawProportionalLine{EK,KB} \cr
\drawProportionalLine{HG,GK} &\parallel \drawProportionalLine{DF,FE}
}\right\}$ \inprop[prop:I.XXXI].

$\drawAngle{GBK} = \drawAngle{FDG}$ \inprop[prop:I.V], $\drawAngle{GBK} = \drawAngle{FGD}$ \inprop[prop:I.XXIX],\\
$\therefore \drawAngle{FDG} = \drawAngle{FGD}$.

$\therefore$ согласно \inpropL[prop:I.VI], \inpropL[prop:I.XXIX] и \inpropL[prop:I.XXXIV],\\
$\drawFromCurrentPicture[middle][squareFGHD]{
draw byNamedPolygon(DHG);
draw byNamedLineFull(G, G, 1, 0, -1)(DF);
draw byNamedLineFull(D, D, 0, 1, -1)(FG);
draw byLabelsOnPolygon(D, H, G, F)(0, 0);
} \mbox{ является квадратом } = \drawProportionalLine{DF}^2$.\\
Аналогично \drawFromCurrentPicture[middle][squareKBCG]{
draw byNamedPolygon(GBC);
draw byNamedLineFull(B, B, 1, 0, -1)(GK);
draw byNamedLineFull(G, G, 0, 1, -1)(KB);
draw byLabelsOnPolygon(G, C, B, K)(0, 0);
} является квадратом $= \drawProportionalLine{GK}^2$,\\
$\drawPolygon[middle]{ACGH} = \drawPolygon[middle]{GKEF} = \drawProportionalLine{DF} \cdot \drawProportionalLine{GK}$ \inprop[prop:I.XLIII].

Но $\drawFromCurrentPicture[middle][squareABEDf]{
draw byNamedPolygon(GBC,DHG,ACGH,GKEF);
draw byNamedLineFull(G, G, 1, 0, -1)(DF);
draw byNamedLineFull(D, D, 0, 1, -1)(FG);
draw byNamedLineFull(B, B, 1, 0, -1)(GK);
draw byNamedLineFull(G, G, 0, 1, -1)(KB);
draw byLabelsOnPolygon(A, B, E, D)(0, 0);
} = \squareFGHD + \drawPolygon{ACGH} + \drawPolygon{GKEF} + \squareKBCG$.

$\therefore \drawProportionalLine{DF,FE}^2 = \drawProportionalLine{DF}^2 + \drawProportionalLine{FE}^2 + \mbox{ дважды } \drawProportionalLine{DF} \cdot \drawProportionalLine{FE}$
\stopCenterAlign

\qed
\stopProposition

\startProposition[title={Предл. V. Теорема},reference=prop:II.V]
\defineNewPicture[1/5]{
pair A, B, C, D, E, F, G, H, K, L, M;
numeric h;
h := 6u;
A := (0, -h);
B := (0, 0);
C := 1/2[A, B];
D := 2/5[B, C];
E := (1/2h, -1/2h);
F := (1/2h, 0);
G := (xpart(F), ypart(D));
H = whatever[D, G] = whatever[B, E];
K := (xpart(H), ypart(A));
L := (xpart(H), ypart(C));
M := (xpart(H), ypart(B));
draw byPolygon(B,D,H,M)(byblue);
draw byPolygon(D,C,L,H)(byyellow);
draw byPolygon(C,L,K,A)(byblack);
draw byPolygon(M,H,G,F)(byyellow);
draw byPolygon(H,L,E,G)(byred);
draw byLine(H, G, byblack, 1, 0);
draw byLine(D, H, byred, 0, 0);
byLineDefine(L, M, byblack, 1, 0);
byLineDefine(C, L, byred, 0, 0);
byLineDefine(B, D, byred, 0, 0);
byLineDefine(D, C, byblue, 0, 0);
byLineDefine(C, A, byyellow, 0, 0);
byLineDefine(E, B, byblack, 0, 0);
byLineDefine(A, K, byred, 1, 0);
byLineDefine(K, L, byyellow, 0, 0);
byLineDefine(L, E, byblue, 1, 0);
draw byNamedLineSeq(-1)(BD,DC,CA,AK,KL,LM,CL,LE,EB);
draw byLabelsOnPolygon(A, C, D, B, M, F, G, E, L, K)(0, 0);
draw byLabelsOnPolygon(M, H, G)(2, 0);
}
\drawCurrentPictureInMargin
\problemNP{Е}{сли}{прямая \drawProportionalLine{BD,DC,CA} рассечена на равные \drawProportionalLine{BD,DC} \drawProportionalLine{CA} и~неравные \drawProportionalLine{BD} \drawProportionalLine{DC,CA} отрезки, прямоугольник, заключенный между неравными частями, вместе с~квадратом отрезка между сечениями, равен квадрату половины всей прямой. \\
$\drawProportionalLine{BD} \cdot \drawProportionalLine{DC,CA} + \drawProportionalLine{DC}^2 = \drawProportionalLine{CA}^2 = \drawProportionalLine{BD,DC}^2$.
}

\startCenterAlign
Опишем \drawPolygon[middle][squareCBFE]{BDHM,DCLH,MHGF,HLEG} \inprop[prop:I.XLVI], проведем \drawProportionalLine{EB}\\
и $\left\{\eqalign{
\drawProportionalLine{DH,HG} & \parallel \drawProportionalLine{CL,LE} \cr
\drawProportionalLine{LM,KL} & \parallel \drawProportionalLine{BD,DC,CA} \cr
\drawProportionalLine{AK} & \parallel \drawProportionalLine{CL,LE}
}\right\}$ \inprop[prop:I.XXXI].

$\drawPolygon{CLKA} = \drawPolygon{DCLH,BDHM}$ \inprop[prop:I.XXXVI],$\drawPolygon{MHGF} = \drawPolygon{DCLH}$ \inprop[prop:I.XLIII].

$\therefore \mbox{\inax[ax:I.II] } \drawPolygon[middle]{DCLH,BDHM,MHGF} = \drawPolygon{DCLH,CLKA} = \drawProportionalLine{BD} \cdot \drawProportionalLine{DC,CA}$.

Но $\drawPolygon{HLEG} = \drawProportionalLine{DC}^2$ \inprop[prop:II.IV]\\
и $\squareCBFE = \drawProportionalLine{BD,DC}^2$ (\conststr).

$\therefore \mbox{\inax[ax:I.II] } \squareCBFE = \drawPolygon{DCLH,CLKA,HLEG}$.

$\therefore \drawProportionalLine{BD} \cdot \drawProportionalLine{DC,CA} + \drawProportionalLine{DC}^2 = \drawProportionalLine{CA}^2 = \drawProportionalLine{BD,DC}^2$
\stopCenterAlign

\qed
\stopProposition

\startProposition[title={Предл. VI. Теорема},reference=prop:II.VI]
\defineNewPicture[1/4]{
pair A, B, C, D, E, F, G, H, K, L, M;
numeric h, s;
h := 5u;
s := 2/5h;
A := (0, h);
B := (0, 0);
C := A shifted (0, -s);
D := A shifted (0, -2s);
E := (-h+s, h-s);
F := (-h+s, 0);
G := (xpart(F), ypart(D));
H = whatever[D, G] = whatever[B, E];
K := (xpart(H), ypart(A));
L := (xpart(H), ypart(C));
M := (xpart(H), ypart(B));
draw byPolygon(B,D,H,M)(byblue);
draw byPolygon(D,C,L,H)(byyellow);
draw byPolygon(C,L,K,A)(byblack);
draw byPolygon(M,H,G,F)(byyellow);
draw byPolygon(H,L,E,G)(byred);
draw byLine(H, G, byblue, 1, 0);
draw byLine(D, H, byred, 0, 0);
byLineDefine(L, M, byblack, 1, 0);
byLineDefine(C, L, byred, 0, 0);
byLineDefine(B, D, byred, 0, 0);
byLineDefine(D, C, byblue, 0, 0);
byLineDefine(C, A, byyellow, 0, 0);
byLineDefine(E, B, byblack, 0, 0);
byLineDefine(A, K, byred, 1, 0);
byLineDefine(K, L, byyellow, 0, 0);
byLineDefine(L, E, byblack, 1, 0);
draw byNamedLineSeq(-1)(BD,DC,CA,AK,KL,LM,CL,LE,EB);
draw byLabelsOnPolygon(F, G, E, L, K, A, C, D, B, M)(0, 0);
draw byLabelsOnPolygon(L, H, D)(2, -1);
}
\drawCurrentPictureInMargin
\problemNP{Е}{сли}{прямая рассечена пополам \drawProportionalLine{DC,CA} и~продлена до любой точки \drawProportionalLine{BD,DC,CA}, прямоугольник, заключенный между всей прямой и~продленной частью, вместе с~квадратом половины исходной линии, равен квадрату линии, составленной из половины исходной линии и~продленной части.\\
$\drawProportionalLine{BD,DC,CA} \cdot \drawProportionalLine{BD} + \drawProportionalLine{DC}^2 = \drawProportionalLine{BD,DC}^2$
}

\startCenterAlign
Опишем \drawPolygon[middle][squareCBFE]{BDHM,DCLH,MHGF,HLEG} \inprop[prop:I.XLVI], проведем \drawProportionalLine{EB}

и $\left\{\eqalign{
\drawProportionalLine{HG,DH} & \parallel \drawProportionalLine{LE,CL} \cr
\drawProportionalLine{LM,KL} & \parallel \drawProportionalLine{BD,DC,CA} \cr
\drawProportionalLine{AK} & \parallel \drawProportionalLine{CL,LE}
}\right\}$ \inprop[prop:I.XXXI].

$\drawPolygon{MHGF} = \drawPolygon{DCLH} = \drawPolygon{CLKA}$ (\inpropL[prop:I.XXXVI], \inpropL[prop:I.XLIII]).

$\therefore \drawPolygon{BDHM,DCLH,MHGF} = \drawPolygon{BDHM,DCLH,CLKA} = \drawProportionalLine{BD} \cdot \drawProportionalLine{BD,DC,CA}$.

Но $\drawPolygon{HLEG} = \drawProportionalLine{DC}^2$ \inprop[prop:II.IV],\\
$\therefore \squareCBFE = \drawProportionalLine{HG,DH}^2 = \drawPolygon{BDHM,DCLH,CLKA,HLEG}$ (\conststr, \inaxL[ax:I.II]).

$\therefore \drawProportionalLine{BD,DC,CA} \cdot \drawProportionalLine{BD} + \drawProportionalLine{DC}^2 = \drawProportionalLine{BD,DC}^2$.
\stopCenterAlign

\qed
\stopProposition

\startProposition[title={Предл. VII. Теорема},reference=prop:II.VII]
\defineNewPicture[1/4]{
pair A, B, C, D, E, F, G, H, N;
numeric w;
w := 7/2u;
A := (0, w);
B := (w, w);
C := 3/5[A, B];
D := (0, 0);
E := (w, 0);
N := 3/5[D, E];
F := 3/5[E, B];
G = whatever[D, B] = whatever[N, C];
H := whatever[A, D] = whatever[F, G];
draw byPolygon(D,N,G,H)(byred);
draw byPolygon(N,E,F,G)(byblack);
draw byPolygon(H,G,C,A)(byyellow);
draw byPolygon(G,F,B,C)(byblue);
draw byLine(G, N, byblue, 0, 0);
draw byLine(G, F, byred, 0, 0);
draw byLine(G, H, byblack, 1, 0);
draw byLine(G, C, byblack, 1, 0);
byLineDefine(B, D, byblack, 0, 0);
byLineDefine(D, N, byblue, 0, 0);
byLineDefine(N, E, byred, 0, 0);
byLineDefine(E, B, byyellow, 0, 0);
draw byNamedLineSeq(-1)(BD,DN,NE,EB);
draw byLabelsOnPolygon(D, H, A, C, B, F, E, N)(0, 0);
draw byLabelsOnPolygon(H, G, C)(2, 0);
}
\drawCurrentPictureInMargin
\problemNP{Е}{сли}{прямая как-либо рассечена \drawProportionalLine{DN,NE}, то вместе квадрат всей прямой и~одной из ее частей равны дважды взятому прямоугольнику, заключенному между всей прямой и~этой ее частью, вместе с~квадратом другой части.\\
$\drawProportionalLine{DN,NE}^2 + \drawProportionalLine{NE}^2 = 2\drawProportionalLine{DN,NE} \cdot \drawProportionalLine{NE} + \drawProportionalLine{DN}^2$
}

\startCenterAlign
Опишем \drawPolygon[middle][squareABED]{DNGH,NEFG,HGCA,GFBC}. \inprop[prop:I.XLVI], проведем \drawProportionalLine{BD} \inpost[post:I.I],\\
и $\left\{\eqalign{
\drawProportionalLine{GN,GC} &\parallel \drawProportionalLine{EB} \cr
\drawProportionalLine{GH,GF} &\parallel \drawProportionalLine{DN,NE}
}\right\}$.

$\drawPolygon{HGCA} = \drawPolygon{NEFG}$ \inprop[prop:I.XLIII].

Добавим $\drawPolygon{GFBC} = \drawProportionalLine{NE}^2$ к~обоим \inprop[prop:II.IV].

$\drawPolygon{HGCA,GFBC} = \drawPolygon{NEFG,GFBC} = \drawProportionalLine{DN,NE} \cdot \drawProportionalLine{NE}$,\\
$\drawPolygon{DNGH} = \drawProportionalLine{DN}^2$ \inprop[prop:II.IV],\\
$\drawPolygon{HGCA,GFBC} + \drawPolygon{NEFG,GFBC} + \drawPolygon{DNGH} = 2\drawProportionalLine{DN,NE} \cdot \drawProportionalLine{NE} + \drawProportionalLine{DN}^2 = \squareABED + \drawPolygon{GFBC}$.

$\drawProportionalLine{DN,NE}^2 + \drawProportionalLine{NE}^2 = 2\drawProportionalLine{DN,NE} \cdot \drawProportionalLine{NE} + \drawProportionalLine{DN}^2$.
\stopCenterAlign

\qed
\stopProposition

\startProposition[title={Предл. VIII. Теорема},reference=prop:II.VIII]
\defineNewPicture{
pair A, B, C, D, E, F, G, H, K, L, M, N, O, P, Q, R;
numeric w, d;
w := 7/2u;
d := u;
A := (0, w + d);
B := (w, w + d);
C := (w - d, w + d);
D := (w + d, w + d);
E := (0, 0);
F := (w + d, 0);
G := (w - d, w);
H := (w - d, 0);
K := (w, w);
L := (w, 0);
M := (0, w);
N := (w + d, w);
O := (0, w - d);
P := (w + d, w - d);
Q := (w - d, w - d);
R := (w, w - d);
draw byLine(C, Q, byblue, 1, 0);
draw byLine(B, R, byblack, 1, 0);
draw byLine(Q, H, byblue, 0, 0);
draw byLine(R, L, byblue, 0, 0);
draw byLine(M, G, byblack, 1, 0);
draw byLine(O, Q, byred, 1, 0);
draw byLine(G, N, byred, 0, 0);
draw byLine(Q, P, byred, 0, 0);
draw byLine(D, E, byblack, 0, 0);
byLineDefine(E, H, byblue, 0, 0);
byLineDefine(H, L, byred, 0, 0);
byLineDefine(L, F, byyellow, 0, 0);
byLineDefine(F, P, byblue, 0, 0);
byLineDefine(P, N, byyellow, 0, 0);
byLineDefine(N, D, byred, 0, 0);
byLineDefine(A, D, byblack, 0, 0);
byLineDefine(E, A, byblack, 0, 0);
draw byNamedLineSeq(0)(EH,HL,LF,FP,PN,ND,AD,EA);
byLineDefine(D, F, byblack, 0, 0);
byLineDefine(B, L, byblack, 0, 1);
byLineDefine(C, H, byblack, 0, 1);
byLineDefine(M, N, byblack, 0, 1);
byLineDefine(O, P, byblack, 0, 1);
draw byLabelsOnPolygon(A, C, B, D, N, P, F, L, H, E, O, M)(0, 0);
}
\drawCurrentPictureInMargin
\problemNP{Е}{сли}{прямая как-либо рассечена \drawProportionalLine{EH,HL}, квадрат всей прямой вместе с~любой ее частью равен четырежды прямоугольнику, заключенному между всей прямой и~этой ее частью, вместе с~квадратом другой части. \\
$\drawProportionalLine{EH,HL,LF}^2 = 4 \cdot \drawProportionalLine{EH,HL} \cdot \drawProportionalLine{HL} + \drawProportionalLine{EH}^2$
}

\startCenterAlign
Продлим \drawProportionalLine{EH,HL} и~сделаем $\drawProportionalLine{LF} = \drawProportionalLine{HL}$.

Построим \drawFromCurrentPicture{
draw byNamedLine(BL,CH,MN,OP);
startAutoLabeling;
draw byNamedLineSeq(0)(LF,HL,EH,EA,AD,DF);
stopAutoLabeling;
} \inprop[prop:I.XLVI].

Проведем \drawProportionalLine{DE}.

$\left.\eqalign{
\left.\eqalign{
\vcenter{
\nointerlineskip\hbox{\drawProportionalLine{CQ,QH}}
\nointerlineskip\hbox{\drawProportionalLine{BR,RL}}
}
}\right\} & \parallel \drawProportionalLine{FP,PN,ND}\cr
\left.\eqalign{
\vcenter{
\nointerlineskip\hbox{\drawProportionalLine{OQ,QP}}
\nointerlineskip\hbox{\drawProportionalLine{MG,GN}}
}
}\right\} & \parallel \drawProportionalLine{EH,HL,LF}\cr
}\right\}$ \inprop[prop:I.XXXI].

$\drawProportionalLine{EH,HL,LF}^2 = \drawProportionalLine{LF}^2 + \drawProportionalLine{EH,HL}^2 + 2 \cdot \drawProportionalLine{EH,HL} \cdot \drawProportionalLine{LF}$ \inprop[prop:II.IV].

Но $\drawProportionalLine{HL}^2 + \drawProportionalLine{EH,HL}^2 = 2 \cdot \drawProportionalLine{EH,HL} \cdot \drawProportionalLine{HL} + \drawProportionalLine{EH}^2$ \inprop[prop:II.VII].

$\therefore \drawProportionalLine{EH,HL,LF}^2 = 4 \cdot \drawProportionalLine{EH,HL} \cdot \drawProportionalLine{HL} + \drawProportionalLine{EH}^2$.
\stopCenterAlign

\qed
\stopProposition

\startProposition[title={Предл. IX. Теорема},reference=prop:II.IX]
\defineNewPicture{
pair A, B, C, D, E, F, G, H;
numeric w;
w := 9/2u;
A := (-1/2w, 0);
B := (1/2w, 0);
C := (0, 0);
D := (1/4w, 0);
E := (0, 1/2w);
F = whatever[E, B] = (xpart(D), whatever);
G = whatever[E, C] = (whatever, ypart(F));
H := 1/2[E, F];
byAngleDefine(E, A, B, byyellow, 0);
byAngleDefine(A, B, E, byblue, 0);
byAngleDefine(A, E, C, byyellow, 0);
byAngleDefine(C, E, B, byred, 0);
byAngleDefine(E, F, G, byred, 0);
byAngleDefine(D, F, B, byblack, 0);
draw byNamedAngleResized();
draw byLine(A, F, byblack, 0, 0);
draw byLine(F, D, byred, 1, 0);
draw byLine(G, F, byyellow, 0, -1);
draw byLine(C, G, byblue, 0, 0);
draw byLine(G, E, byblue, 1, 0);
byLineDefine(A, C, byblue, 0, 0);
byLineDefine(C, D, byyellow, 0, 0);
byLineDefine(D, B, byred, 0, 0);
byLineDefine(F, B, byyellow, 1, 0);
byLineDefine(H, F, byblack, 0, 0);
byLineDefine(E, H, byblack, 1, 0);
byLineDefine(A, E, byblack, 1, 0);
draw byNamedLineSeq(0)(AC,CD,DB,FB,HF,EH,AE);
draw byLabelsOnPolygon(E, F, B, D, C, A)(0, 0);
draw byLabelsOnPolygon(C, G, E)(2, 0);
}
\drawCurrentPictureInMargin
\problemNP{Е}{сли}{прямая рассечена на равные \drawProportionalLine{AC} \drawProportionalLine{CD,DB} и~неравные \drawProportionalLine{AC,CD} \drawProportionalLine{DB} части, квадраты неравных частей вместе вдвое больше квадрата на половине, вместе с~квадратом отрезка между сечениями.\\
$\drawProportionalLine{AC,CD}^2 + \drawProportionalLine{DB}^2 = 2 \cdot \drawProportionalLine{AC}^2 + 2 \cdot \drawProportionalLine{CD}^2$
}

\startCenterAlign
Сделаем $\drawProportionalLine{CG,GE} \perp \mbox{ и~} = \drawProportionalLine{AC} \mbox{ или } \drawProportionalLine{CD,DB}$.

Проведем \drawProportionalLine{AE} и~\drawProportionalLine{EH,HF,FB},\\
$\drawProportionalLine{FD} \parallel \drawProportionalLine{CG,GE}$, $ \drawProportionalLine{GF} \parallel \drawProportionalLine{CD,DB}$ и~проведем \drawProportionalLine{AF}.

$\drawAngle{A} = \drawAngle{AEC}$ \inprop[prop:I.V] $= \frac{1}{2}\drawRightAngle$ \inprop[prop:I.XXXII],
$\drawAngle{B} = \drawAngle{DFB}$ \inprop[prop:I.V] $= \frac{1}{2}\drawRightAngle$ \inprop[prop:I.XXXII],
$\therefore \drawAngle{AEC,CEB} = \drawRightAngle$.

$\drawAngle{B} = \drawAngle{CEB} = \drawAngle{EFG} = \drawAngle{DFB}$ (\inpropL[prop:I.V], \inpropL[prop:I.XXIX]).

Значит, $\drawProportionalLine{FD} = \drawProportionalLine{DB}$, $\drawProportionalLine{GE} = \drawProportionalLine{GF} = \drawProportionalLine{CD}$ (\inpropL[prop:I.VI], \inpropL[prop:I.XXXIV]).

$\drawProportionalLine{AF}^2 = \left\{\eqalign{
&\drawProportionalLine{AC,CD}^2 + \drawProportionalLine{FD}^2 \mbox{, или } + \drawProportionalLine{DB}^2 \cr
&= \left\{\eqalign{
&= \drawProportionalLine{AE}^2 + \drawProportionalLine{EH,HF}^2 \cr
&= 2 \cdot \drawProportionalLine{AC}^2 + 2 \cdot \drawProportionalLine{CD}^2
}\right. \mbox{ \inprop[prop:I.XLVII].}
}\right.$

$\therefore \drawProportionalLine{AC,CD}^2 + \drawProportionalLine{DB}^2 = 2 \cdot \drawProportionalLine{AC}^2 + 2 \cdot \drawProportionalLine{CD}^2$.
\stopCenterAlign

\qed
\stopProposition

\startProposition[title={Предл. X. Теорема},reference=prop:II.X]
\defineNewPicture{
pair A, B, C, D, E, F, G;
numeric w;
w := 7/2u;
A := (0, 0);
B := (w, 0);
C := 1/2[A, B];
D := 4/3[A, B];
E := (w/2, w/2);
F := (xpart(D), ypart(E));
G = whatever[E, B] = whatever[F, D];
byAngleDefine(E, A, C, byblack, 0);
byAngleDefine(C, E, A, byyellow, 0);
byAngleDefine(B, E, C, byyellow, 0);
byAngleDefine(F, E, B, byblue, 0);
byAngleDefine(C, B, E, byred, 0);
byAngleDefine(D, B, G, byred, 0);
byAngleDefine(B, G, D, byblue, 0);
draw byNamedAngleResized();
draw byLine(C, E, byred, 0, -1);
draw byLine(A, C, byred, 0, 0);
draw byLine(C, B, byyellow, 0, 0);
draw byLine(B, D, byblue, 0, 0);
draw byLine(E, B, byblack, 0, 0);
draw byLine(B, G, byblack, 1, 0);
byLineDefine(E, F, byyellow, 1, 0);
byLineDefine(F, D, byred, 0, 0);
byLineDefine(D, G, byred, 1, 0);
byLineDefine(A, G, byblack, 0, 0);
byLineDefine(A, E, byblue, 1, 0);
draw byNamedLineSeq(0)(EF,FD,DG,AG,AE);
draw byLabelsOnPolygon(A, E, F, D, G)(0, 0);
draw byLabelsOnPolygon(G, B, C, A)(2, 0);
}
\drawCurrentPictureInMargin
\problemNP{Е}{сли}{прямая линия \drawProportionalLine{AC,CB} рассечена пополам и~продлена до любой точки \drawProportionalLine{AC,CB,BD}, квадрат всей линии вместе с~квадратом продленной части равны дважды квадрату половины исходной линии вместе с~квадратом половины вместе с~продленной частью.\\
$\drawProportionalLine{AC,CB,BD}^2 + \drawProportionalLine{BD}^2 = 2 \cdot \drawProportionalLine{CB}^2 + 2 \cdot \drawProportionalLine{CB,BD}^2$
}

\startCenterAlign
Сделаем $\drawProportionalLine{CE} \perp \mbox{ и~} = \drawProportionalLine{AC} \mbox{ или } \drawProportionalLine{CB}$.

Проведем \drawProportionalLine{AE} и~\drawProportionalLine{EB,BG},\\
и $\left\{\eqalign{
\drawProportionalLine{FD,DG} & \parallel \drawProportionalLine{CE} \cr
\drawProportionalLine{EF} & \parallel \drawProportionalLine{CB,BD}
}\right\}$ \inprop[prop:I.XXXI]\\
также проведем \drawProportionalLine{AG}.

$\drawAngle{A} = \drawAngle{CEA}$ \inprop[prop:I.V] $= \frac{1}{2}\drawRightAngle$ \inprop[prop:I.XXXII],\\
$\drawAngle{CBE} = \drawAngle{BEC}$ \inprop[prop:I.V] $= \frac{1}{2}\drawRightAngle$ \inprop[prop:I.XXXII].

$\therefore \drawAngle{CEA,BEC} = \drawRightAngle$.

$\drawAngle{BEC} = \drawAngle{G}$, $\drawAngle{FEB} = \drawAngle{CBE}$\inprop[prop:I.XXIX].

$\drawAngle{DBG} = \drawAngle{CBE}$\inprop[prop:I.XV].

$\therefore \drawAngle{DBG} = \drawAngle{CBE} = \drawAngle{BEC} = \drawAngle{FEB} = \drawAngle{G} = \frac{1}{2}\drawRightAngle$.


И $\drawProportionalLine{BD} = \drawProportionalLine{DG}$, $\drawProportionalLine{CB,BD} = \drawProportionalLine{EF} = \drawProportionalLine{FD,DG}$, (\inpropL[prop:I.VI], \inpropN[prop:I.XXXIV]).

Тогда, согласно \inprop[prop:I.XLVII]\\
$\drawUnitLine{AG}^2 = \left\{\eqalign{
& \drawProportionalLine{AC,CB,BD}^2 + \drawProportionalLine{DG}^2 \mbox{ или } \drawProportionalLine{BD}^2 \cr
& \left\{\eqalign{
& + \drawProportionalLine{AE}^2 = 2 \cdot \drawProportionalLine{AC}^2 \cr
& + \drawProportionalLine{EB,BG}^2 = 2\cdot \drawProportionalLine{EF}^2
}\right.
}\right.$

$\therefore \drawProportionalLine{AC,CB,BD}^2 + \drawProportionalLine{BD}^2 = 2 \cdot \drawProportionalLine{CB}^2 + 2 \cdot \drawProportionalLine{CB,BD}^2$.
\stopCenterAlign

\qed
\stopProposition

\startProposition[title={Предл. XI. Задача},reference=prop:II.XI]
\defineNewPicture{
pair A, B, C, D, E, F, G, H, K;
numeric w;
w := 7/2u;
A := (0, 0);
B := (w, 0);
C := (0, w);
D := (w, w);
E := 1/2[A, C];
F := E shifted (0, -abs(E-B));
G := F shifted (abs(F-A), 0);
H = whatever[A, B] = (xpart(G), whatever);
K = whatever[G, H] = whatever[C, D];
draw byPolygon(A,H,K,C)(byyellow);
draw byPolygon(H,B,D,K)(byblue);
draw byPolygon(A,H,G,F)(byblue);
byLineDefine(K, G, byblack, 1, 0);
byLineDefine(A, H, byred, 0, 0);
byLineDefine(H, B, byred, 1, 0);
byLineDefine(B, E, byblack, 0, 0);
byLineDefine(C, E, byblue, 1, 0);
byLineDefine(E, A, byblue, 0, 0);
byLineDefine(A, F, byyellow, 0, 0);
draw byNamedLineSeq(1)(BE,HB,AH);
draw byNamedLine(CE,EA,AF,KG);
draw byLabelsOnPolygon(F, A, E, C, K, D, B, H, G)(0, 0);
}
\drawCurrentPictureInMargin
\problemNP{Д}{анную}{прямую \drawProportionalLine{AH,HB} рассечь так, чтобы прямоугольник, заключенный между всей прямой и~одной из ее частей был равен квадрату другой части.\\
$\drawProportionalLine{AH,HB} \cdot \drawProportionalLine{HB} = \drawProportionalLine{AH}^2$
}

\startCenterAlign
Опишем \drawPolygon{AHKC,HBDK} \inprop[prop:I.XLVI].

Cделаем $\drawProportionalLine{EA} = \drawProportionalLine{CE}$ \inprop[prop:I.X],\\
проведем \drawProportionalLine{BE},\\
возьмем $\drawProportionalLine{EA,AF} = \drawProportionalLine{BE}$ \inprop[prop:I.III],\\
на \drawProportionalLine{AF} опишем \drawPolygon{AHGF} \inprop[prop:I.XLVI].

Продлим \drawProportionalLine{KG} \inpost[post:I.II].

Тогда \inprop[prop:II.VI] $\drawProportionalLine{CE,EA,AF} \cdot \drawProportionalLine{AF} + \drawProportionalLine{EA}^2 = \drawProportionalLine{EA,AF}^2 = \drawProportionalLine{BE}^2 = \drawProportionalLine{AH,HB}^2 + \drawProportionalLine{EA}^2 \therefore \drawProportionalLine{CE,EA,AF} \cdot \drawProportionalLine{AF} = \drawProportionalLine{AH,HB}^2$,\\
или $\drawPolygon{AHKC,AHGF} = \drawPolygon{AHKC,HBDK} \therefore \drawPolygon{AHGF} = \drawPolygon{HBDK}$

$\therefore \drawProportionalLine{AH,HB} \cdot \drawProportionalLine{HB} = \drawProportionalLine{AH}^2$.

\stopCenterAlign

\qed
\stopProposition

\startProposition[title={Предл. XII. Теорема},reference=prop:II.XII]
\defineNewPicture[1/4]{
pair A, B, C, D;
numeric w, h;
w := 7/2u;
h := 3u;
A := (3/5w, 0);
B := (w, h);
C := (0, 0);
D := (w, 0);
draw byLine(B, A, byred, 0, 0);
byLineDefine(C, A, byblack, 0, 0);
byLineDefine(A, D, byblack, 1, 0);
byLineDefine(D, B, byyellow, 0, 0);
byLineDefine(B, C, byblue, 0, 0);
draw byNamedLineSeq(0)(DB,AD,CA,BC);
draw byLabelsOnPolygon(C, B, D, A)(0, 0);
}
\drawCurrentPictureInMargin
\problemNP{В}{о}{всяком тупоугольном треугольнике квадрат на стороне, стягивающей тупой угол, больше суммы квадратов на сторонах, содержащих тупой угол, на дважды прямоугольник, заключенный между какой-либо из этих сторон и~продолжением этой стороны от тупого угла до перпендикуляра, падающего от противоположного острого угла.\\
$\drawProportionalLine{BC}^2 > \drawProportionalLine{CA}^2 + \drawProportionalLine{BA}^2$ на $2 \cdot \drawProportionalLine{CA} \cdot \drawProportionalLine{AD}$
}

\startCenterAlign
Согласно \inpropL[prop:II.IV]\\
$\drawProportionalLine{CA,AD}^2 = \drawProportionalLine{CA}^2 + \drawProportionalLine{AD}^2 + 2 \cdot \drawProportionalLine{CA} \cdot \drawProportionalLine{AD}^2$.

Добавим $\drawProportionalLine{DB}^2$ к~обоим.

$\drawProportionalLine{CA,AD}^2 + \drawProportionalLine{DB}^2 = \drawProportionalLine{CA}^2 + \drawProportionalLine{AD}^2 + 2 \cdot \drawProportionalLine{CA} \cdot \drawProportionalLine{AD}^2 + \drawProportionalLine{DB}^2$.

Но $\drawProportionalLine{CA,AD}^2 + \drawProportionalLine{DB}^2 = \drawProportionalLine{BC}^2$ \inprop[prop:I.XLVII],\\
и $\drawProportionalLine{AD}^2 + \drawProportionalLine{DB}^2 = \drawProportionalLine{BA}^2$ \inprop[prop:I.XLVII].

$\therefore \drawProportionalLine{BC}^2 = 2 \cdot \drawProportionalLine{CA} \cdot \drawProportionalLine{AD} + \drawProportionalLine{CA}^2 + \drawProportionalLine{BA}^2$. 

Значит, $ \drawProportionalLine{BC}^2 > \drawProportionalLine{CA}^2 + \drawProportionalLine{BA}^2$ на $2 \cdot \drawProportionalLine{CA} \cdot \drawProportionalLine{AD}^2$.
\stopCenterAlign

\qed
\stopProposition

\startProposition[title={Предл. XIII. Теорема},reference=prop:II.XIII]
\defineNewPicture{
pair A,B,C,D,E,F,G,H, d;
numeric w, h;
w := 3u;
h := 3u;
A := (2/5w, h);
B:= (0, 0);
C := (w, 0);
D = whatever[B, C] = (xpart(A), whatever);
d := (0, -h -4/3u);
E := (w, h) shifted d;
F := (0, 0) shifted d;
G := (2/5w, 0) shifted d;
H = whatever[F, G] = (xpart(E), whatever);
draw byLine(A, D, byyellow, 0, 0);
byLineDefine(A, B, byred, 0, 0);
byLineDefine(B, D, byblack, 0, 0);
byLineDefine(D, C, byblack, 1, 0);
byLineDefine(C, A, byblue, 0, 0);
draw byNamedLineSeq(0)(AB,BD,DC,CA);
draw byLine(E, G, byblue, 0, 0);
byLineDefine(E, F, byred, 0, 0);
byLineDefine(F, G, byblack, 0, 0);
byLineDefine(G, H, byblack, 1, 0);
byLineDefine(H, E, byyellow, 0, 0);
draw byNamedLineSeq(0)(EF,FG,GH,HE);
label.top(\sometxt{Первый случай}, (xpart(1/2[B, C]), ypart(A) + 1/4u));
label.top(\sometxt{Второй случай}, (xpart(1/2[F, H]), ypart(E)));
draw byLabelsOnPolygon(B, A, C, D)(0, 0);
draw byLabelsOnPolygon(F, E, H, G)(0, 0);
}
\drawCurrentPictureInMargin
\problemNP{В}{о}{всяком треугольнике квадрат стороны, стягивающей острый угол, меньше суммы квадратов сторон, содержащих этот угол, на дважды прямоугольник, заключенный между любой из этих сторон и~отрезком, отсекаемым перпендикуляром из противоположного угла от этого отрезка или от продленного отрезка.\\
Первый случай.\\
$\drawSizedLine{CA}^2 < \drawSizedLine{BD,DC}^2 + \drawSizedLine{AB}^2$ на $2 \cdot \drawSizedLine{BD,DC} \cdot \drawSizedLine{BD}$.\\
Второй случай.\\
$\drawSizedLine{EG}^2 < \drawSizedLine{EF}^2 + \drawSizedLine{FG}^2$ на $2 \cdot \drawSizedLine{FG} \cdot \drawSizedLine{FG,GH}$.
}

\startCenterAlign
Предположим, перпендикуляр падает внутри треугольника,\\
тогда  \inprop[prop:II.VII]\\
$\drawSizedLine{BD,DC}^2 + \drawSizedLine{BD}^2 = 2 \cdot \drawSizedLine{BD,DC} \cdot \drawSizedLine{BD} + \drawSizedLine{DC}^2$.

К каждой добавим $\drawSizedLine{AD}^2$,\\ 
тогда $\drawSizedLine{BD,DC}^2 + \drawSizedLine{BD}^2 + \drawSizedLine{AD}^2 = 2 \cdot \drawSizedLine{BD,DC} \cdot \drawSizedLine{BD} + \drawSizedLine{DC}^2 + \drawSizedLine{AD}^2$.

$\therefore$ \inprop[prop:I.XLVII]
$\drawSizedLine{BD,DC}^2 + \drawSizedLine{AB}^2 = 2 \cdot \drawSizedLine{BD,DC} \cdot \drawSizedLine{BD} + \drawSizedLine{CA}^2$,\\
и $\therefore \drawSizedLine{CA}^2 < \drawSizedLine{BD,DC}^2 + \drawSizedLine{AB}^2$ на $2 \cdot \drawSizedLine{BD,DC} \cdot \drawSizedLine{DC}$.

Теперь предположим, что перпендикуляр падает вовне треугольника, тогда \inprop[prop:II.VII]\\
$\drawSizedLine{FG,GH}^2 + \drawSizedLine{FG}^2 = 2 \cdot \drawSizedLine{FG,GH} \cdot \drawSizedLine{FG} + \drawSizedLine{GH}^2$.

К каждой добавим $\drawSizedLine{HE}^2$, тогда\\
$\drawSizedLine{FG,GH}^2 + \drawSizedLine{FG}^2 + \drawSizedLine{HE}^2= 2 \cdot \drawSizedLine{FG,GH} \cdot \drawSizedLine{FG} + \drawSizedLine{GH}^2 + \drawSizedLine{HE}^2$.

$\therefore$ \inprop[prop:I.XLVII]
$\drawSizedLine{EF} + \drawSizedLine{FG}^2 = 2 \cdot \drawSizedLine{FG,GH} \cdot \drawSizedLine{FG}^2 + \drawSizedLine{EG}^2$.

$\therefore \drawSizedLine{EG}^2 < \drawSizedLine{EF}^2 + \drawSizedLine{FG}^2$ на $2 \cdot \drawSizedLine{FG,GH} \cdot \drawSizedLine{FG}$.
\stopCenterAlign

\qed
\stopProposition

\startProposition[title={Предл. XIV. Задача},reference=prop:II.XIV]
\defineNewPicture[1/3]{
path A;
pair a, b, c, d, e, f;
pair B, C, D, E, F, G, H;
numeric w, h, s, r;
a := (0, 0);
b := (u, 1/4u);
c := (2u, 1/5u);
d := (11/5u, -u);
e := (7/8u, -2u);
f := (1/8u, -3/4u);
A := a--b--c--d--e--f--cycle;
s := 0;
for i := 1 step 1 until length(A) - 1:
	s := s + 1/2(
		abs((point 0 of A) - (point i of A))
		*distanceToLine((point i + 1 of A), (point 0 of A)--(point i of A))
		);
endfor;
w := 5/2u;
h := s/w;
B := (w, 0);
C := (w, -h);
D := (0, -h);
E := (0, 0);
F := (-h, 0);
G := 1/2[B, F];
r := abs(B - G);
H := (D--(E shifted ((E-D)*10))) intersectionpoint ((fullcircle scaled 2r) shifted G);
byPointLabelRemove(a,b,c,d,e,f);
forsuffixes i=a,b,c,d,e,f:
	i := i shifted (xpart(G)-xpart(1/2[urcorner(A),ulcorner(A)]), r - ypart(lrcorner(A)) + 1/2u);
endfor;
draw byPolygon(a,b,c,d,e,f)(byyellow);
draw byPolygon(B,C,D,E)(byred);
draw byLine(H, G, byred, 0, 0);
draw byLine(H, E, byblue, 0, 0);
draw byLine(E, D, byyellow, 0, 0);
byLineDefine(H, F, byblack, 0, 1);
byLineDefine(H, B, byblack, 0, 1);
byLineDefine(F, E, byblack, 1, 0);
byLineDefine(E, G, byblue, 1, 0);
byLineDefine(G, B, byblack, 0, 0);
draw byNamedLineSeq(-1)(FE,EG,GB,HB,HF);
draw byArc.G(G, B, F, r, byred, 0, 0, 0, 0);
byLineDefine(B, F, byblack, 0, 0);
draw byLabelsOnPolygon(B, C, D, E, F, H)(0, 0);
draw byLabelsOnPolygon(H, G, B)(2, 0);
}
\drawCurrentPictureInMargin
\problemNP[3]{П}{остроить}{квадрат, равный данной прямолинейной фигуре.\\~\\~\\
Провести \drawSizedLine{HE} такую, что $\drawSizedLine{HE}^2 = \drawPolygon{abcdef}$
}

\startCenterAlign
Сделаем $\drawPolygon{BCDE} = \drawPolygon{abcdef}$ \inprop[prop:I.XLV].

Продлим \drawSizedLine{EG,GB} до $\drawSizedLine{FE} = \drawSizedLine{ED}$.

Возьмем $\drawSizedLine{FE,EG} = \drawSizedLine{GB}$ \inprop[prop:I.X].

Опишем
\drawFromCurrentPicture{
startTempScale(1/3);
draw byNamedLineFull(B, F, 0, 0, 1)(BF);
startAutoLabeling;
draw byNamedArc(G);
stopAutoLabeling;
stopTempScale;
}
\inpost[post:I.III]\\
и продлим до нее \drawSizedLine{ED}: проведем \drawSizedLine{HG}.

$\drawSizedLine{HF}^2 \mbox{ или } \drawSizedLine{HG}^2 = \drawSizedLine{FE} \cdot \drawSizedLine{EG,GB} + \drawSizedLine{EG}^2$ \inprop[prop:II.V],\\
но $\drawSizedLine{HG}^2 = \drawSizedLine{HE} ^2 + \drawSizedLine{EG}^2$ \inprop[prop:I.XLVII].

$\therefore \drawSizedLine{HE}^2 + \drawSizedLine{EG}^2 = \drawSizedLine{FE} \cdot \drawSizedLine{EG,GB} + \drawSizedLine{EG}^2$.

$\therefore \drawSizedLine{HE}^2 = \drawSizedLine{FE} \cdot \drawSizedLine{EG,GB}$.

И $\therefore \drawSizedLine{HE}^2 = \drawPolygon{BCDE} = \drawPolygon{abcdef}$.
\stopCenterAlign

\qed
\stopProposition
\stopBook

\startBook[title={Книга III}]

\startsupersection[title={Определения}]

\startDefinitionOnlyNumber[reference=def:III.I]
Равными кругами называют те, у~которых равны диаметры.
\stopDefinitionOnlyNumber

\startDefinitionOnlyNumber[reference=def:III.II]
\defineNewPicture{
	pair O, A, B;
	numeric r;
	r := 2/3u;
	O := (0, 0);
	A := (-r, -r);
	B := (r, -r);
	draw byCircleR(O, r, byblack, 0, 0, -1);
	draw byLine(A, B, byblack, 0, 0);
}\drawCurrentPictureInMargin
Про прямую говорят, что она касается круга, если она встречает круг, но при продолжении не пересекает его.
\stopDefinitionOnlyNumber

\startDefinitionOnlyNumber[reference=def:III.III]
\defineNewPicture{
	pair A, B, C;
	numeric r[];
	r1 := 3/5u;
	r2 := 1/2r1;
	r3 := 2/5r1;
	A := (0, 0);
	B := A shifted (dir(110) scaled (r1-r2));
	C := A shifted (dir(-130) scaled (r1+r3));
	fill (fullcircle scaled 2r1) shifted A withcolor byyellow;
	draw byCircleR(A, r1, byyellow, 0, 0, 0);
	fill (fullcircle scaled 2r2) shifted B withcolor byred;
	draw byCircleR(B, r2, byblack, 0, 0, -1/2);
	fill (fullcircle scaled 2r3) shifted C withcolor byblue;
	draw byCircleR(C, r3, byblue, 0, 0, -1/2);
}\drawCurrentPictureInMargin
Про круги говорят, что они касаются друг друга, если они встречаясь, не пересекают друг друга.
\stopDefinitionOnlyNumber

\startDefinitionOnlyNumber[reference=def:III.IV]
\defineNewPicture{
	pair O, A, B, C, D, E, F;
	numeric r;
	r := 2/3u;
	O := (0, 0);
	A := dir(170) scaled r;
	B := dir(-110) scaled r;
	C := A xscaled -1;
	D := B xscaled -1;
	E := 1/2[A, B];
	F := 1/2[C, D];
	draw byLine(A, B, byblack, 0, 0);
	draw byLine(C, D, byblack, 0, 0);
	byLineDefine(O, E, byred, 0, 0);
	byLineDefine(O, F, byblue, 0, 0);
	draw byNamedLineSeq(0)(OE,OF);
	draw byCircleR(O, r, byblack, 0, 0, 0);
}\drawCurrentPictureInMargin
Про прямые говорят, что они равноотстоят от центра круга, если перпендикуляры проведенные к~ним из центра круга равны.
\stopDefinitionOnlyNumber

\startDefinitionOnlyNumber[reference=def:III.V]
Про прямую, перпендикуляр к~которой длиннее, говорят, что она отстоит дальше от центра круга.
\stopDefinitionOnlyNumber

\vfill\pagebreak

\startDefinitionOnlyNumber[reference=def:III.VI]
\defineNewPicture{
	pair A, B;
	numeric r;
	r := 3/4u;
	A := (0, 0);
	B := (0, -1/8u);
	draw byFilledCircleSegment(A, r, 1/2, 4 - 1/2, byred);
	draw byFilledCircleSegment(B, r, 4 - 1/2, 8 + 1/2, byblue);
}\drawCurrentPictureInMargin
Сегментом круга называется фигура, заключающаяся между прямой и~частью круга, отсекаемой этой прямой.
\stopDefinitionOnlyNumber

\startDefinitionOnlyNumber[reference=def:III.VII]
\defineNewPicture{
	pair O, A, B, C;
	numeric r, b, e;
	r := u;
	b := 1/2;
	e := 4-1/2;
	O := (0, 0);
	A := O shifted (point b of fullcircle scaled 2r);
	B := O shifted (point e of fullcircle scaled 2r);
	C := B shifted ((unitvector(B-O) rotated -90) scaled r);
	byAngleDefine(C, B, A, byred, 0);
	draw byNamedAngleResized();
	draw byArcBE(O, b, e, r, byblack, 0, 0, 0, 0);
	draw byLine(A, B, byblack, 0, 0);
	draw byLine(B, C, byblack, 0, 0);
	draw byLine(O, B, byblack, 1, 0);
}\drawCurrentPictureInMargin
Углом сегмента называется угол между основанием сегмента и~перпендикуляром к~отрезку, соединяющему центр круга с~одним из концов основания сегмента.
\stopDefinitionOnlyNumber

\startDefinitionOnlyNumber[reference=def:III.VIII]
\defineNewPicture{
	pair O, A, B, C, D;
	numeric r, b, e;
	r := u;
	b := -1;
	e := 5;
	O := (0, 0);
	A := dir(100) scaled r;
	B := dir(30) scaled r;
	C := point b of fullcircle scaled 2r;
	D := point e of fullcircle scaled 2r;
	byAngleDefine(C, A, D, byblue, 0);
	byAngleDefine(C, B, D, byyellow, 0);
	draw byNamedAngleResized();
	draw byArcBE(O, b, e, r, byblack, 0, 0, 0, 0);
	draw byLine(C, A, byblack, 0, 0);
	draw byLine(C, B, byblack, 0, 0);
	draw byLine(D, A, byblack, 0, 0);
	draw byLine(D, B, byblack, 0, 0);
	draw byLine(C, D, byblack, 0, 0);
}\drawCurrentPictureInMargin
Углом в~сегменте называют угол, заключающийся между прямыми, проведенными из какой-либо точки на окружности сегмента к~концам основания сегмента.
\stopDefinitionOnlyNumber

\startDefinitionOnlyNumber[reference=def:III.IX]
\defineNewPicture{
	pair O, A, B, C;
	numeric r, b, e;
	r := 2/3u;
	b := -3/2;
	e := 9/2;
	O := (0, 0);
	A := dir(80) scaled r;
	B := point b of fullcircle scaled 2r;
	C := point e of fullcircle scaled 2r;
	byAngleDefine(C, A, B, byblue, 0);
	draw byNamedAngleResized();
	draw byArcBE.Op(O, b, e, r, byblack, 1, 0, 0, 0);
	draw byArcBE.Om(O, e, b + 8, r, byblack, 0, 0, 0, 0);
	draw byLine(C, A, byblack, 0, 0);
	draw byLine(B, A, byblack, 0, 0);
}\drawCurrentPictureInMargin
Про угол говорят, что он опирается на дугу, если заключающие его прямые отсекают эту дугу.
\stopDefinitionOnlyNumber

\startDefinitionOnlyNumber[reference=def:III.X]
\defineNewPicture{
	pair O;
	numeric r, b, e;
	r := 2/3u;
	b := 1;
	e := 3;
	O := (0, 0);
	draw byFilledCircleSector(O, r, b, e, byyellow);
	draw byArcBE(O, e, b + 8, r, byblack, 0, 0, -1, 0);
}\drawCurrentPictureInMargin
Сектором круга называют фигуру, заключающуюся между двумя радиусами и~дугой между ними.
\stopDefinitionOnlyNumber

\startDefinitionOnlyNumber[reference=def:III.XI]
\defineNewPicture{
	pair M, N, A, B, C, D, E, F;
	numeric r[], b, e;
	r1 := 3/2u;
	r2 := u;
	b := 1;
	e := 3;
	M := (0, 0);
	N := (0, -1/3u);
	A := (point b of fullcircle scaled 2r1) shifted M;
	B := (point 1/3[b,e] of fullcircle scaled 2r1) shifted M;
	C := (point e of fullcircle scaled 2r1) shifted M;
	D := (point b of fullcircle scaled 2r2) shifted N;
	E := (point 1/3[b,e] of fullcircle scaled 2r2) shifted N;
	F := (point e of fullcircle scaled 2r2) shifted N;
	draw byPolygon(A,B,C)(byred);
	draw byPolygon(D,E,F)(byred);
	draw byArcBE(M, b, e, r1, byblack, 0, 0, 0, 1);
	draw byArcBE(N, b, e, r2, byblack, 0, 0, 0, 1);
}\drawCurrentPictureInMargin
Подобными сегментами называют сегменты, углы в~которых равны.
\stopDefinitionOnlyNumber

\startDefinitionOnlyNumber[reference=def:III.XII]
\defineNewPicture{
	pair O;
	numeric r[];
	r1 := 1/3u;
	r2 := 1/2u;
	r3 := 3/4u;
	O := (0, 0);
	draw byFilledCircleSegment(O, r3, 0, 8, byred);
	draw byFilledCircleSegment(O, r2, 0, 8, white);
	draw byCircleR(O, r2, byblack, 0, 0, 0);
	draw byFilledCircleSegment(O, r1, 0, 8, byblue);
}\drawCurrentPictureInMargin
Круги, имеющие один центр называют концентрическими.
\stopDefinitionOnlyNumber
\stopsupersection

\vfill\pagebreak

\startProposition[title={Предл. I. Задача},reference=prop:III.I]
\defineNewPicture{
pair A, B, C, D, E, F, G;
numeric r, a;
r := 9/4u;
F := (0, 0);
A := F shifted (dir(170)*r);
B := F shifted (dir(-95)*r);
D := 1/2[A, B];
C := F shifted (dir(angle(A-B) - 90)*r);
E := F shifted (dir(angle(A-B) +90)*r);
G := F shifted (dir(-45)*1/2r);
a := -angle(G-D);
forsuffixes i=A, B, D, C, E, F, G:
	i := i rotated a;
endfor;
byAngleDefine(A, D, F, byblue, 0);
byAngleDefine(F, D, G, byyellow, 0);
byAngleDefine(G, D, B, byblack, 0);
draw byNamedAngleResized();
draw byLine(D, G)(byblue, 1, 0);
draw byLine(A, D)(byred, 0, 0);
draw byLine(D, B)(byred, 1, 0);
draw byLine(E, C)(byblack, 0, 0);
draw byMarkLine(1/2, byblack)(EC);
byLineDefine(A, G, byblue, 0, 0);
byLineDefine(B, G, byblack, 1, 0);
draw byNamedLineSeq(0)(AG,BG);
draw byCircleR(F, r, byblue, 0, 0, 0);
draw byLabelsOnPolygon(A, G, B)(2, 0);
draw byLabelsOnPolygon(E, D, A)(2, 0);
draw byLabelsOnPolygon(E, F, C)(2, -4);
draw byLabelsOnCircle(A, B, C, E)(F);
}
\drawCurrentPictureInMargin
\problemNP{Н}{айти}{центр данного круга \drawCircle[middle][1/4]{F}.}

\startCenterAlign
Проведем внутри круга любую прямую \drawUnitLine{AD,DB},\\
сделаем $\drawUnitLine{AD} = \drawUnitLine{DB}$, проведем $\drawUnitLine{EC} \perp \drawUnitLine{AD,DB}$.

Рассечем пополам \drawUnitLine{EC}, точка рассечения и~есть центр.

Действительно, пусть это не так, тогда пусть другая точка будет центром, и~проведем от нее \drawUnitLine{GA}, \drawUnitLine{GD} и~\drawUnitLine{GB}.

Поскольку в~\drawLine{AG,DG,AD} и~\drawLine{DB,DG,BG} $\drawUnitLine{AG} = \drawUnitLine{BG}$ (\hypstr и~\indefL[def:I.XV]),\\
$\drawUnitLine{AD} = \drawUnitLine{DB}$ (\conststr) и~\drawUnitLine{DG} общая обоим,\\
$\drawAngle{ADF,FDG} = \drawAngle{GDB}$ \inprop[prop:I.VIII], и, сдедовательно, являются прямыми углами.

Но $\drawAngle{FDG,GDB} = \drawRightAngle$ (\conststr), $\drawAngle{GDB} = \drawAngle{FDG,GDB}$ \inax[ax:I.XI]\\
что невозможно.
\stopCenterAlign

\noindent Значит, выбранная точка не центр круга, и~так же можно показать, что никакая другая точка, не на \drawUnitLine{EC} не является центром, значит, центр находится на \drawUnitLine{EC}, и, следовательно, точка, где \drawUnitLine{EC} рассекается пополам и~есть центр.

\qed
\stopProposition

\startProposition[title={Предл. II. Теорема},reference=prop:III.II]
\defineNewPicture{
pair A, B, D, E, F;
numeric r;
r := 9/4u;
D := (0, 0);
A := (dir(185) scaled r) shifted D;
B := (dir(-70) scaled r) shifted D;
E := 3/5[A, B];
F := (dir(angle(E-D)) scaled r) shifted D;
byAngleDefine(D, A, B, byblue, 0);
byAngleDefine(A, E, D, byyellow, 0);
byAngleDefine(A, B, D, byblack, 0);
draw byNamedAngleResized();
draw byLine(D, E, byblack, 0, 0);
draw byLine(E, F, byblack, 1, 0);
draw byLine(A, B, byred, 0, 0);
byLineDefine(A, D, byyellow, 0, 0);
byLineDefine(B, D, byblue, 0, 0);
draw byNamedLineSeq(0)(AD,BD);
draw byCircleR(D, r, byred, 0, 0, 0);
draw byLabelsOnPolygon(B, F, A, D)(0, 0);
draw byLabelsOnPolygon(A, E, F)(2, 0);
}
\drawCurrentPictureInMargin
\problemNP[2]{П}{рямая}{\drawSizedLine{AB}, соединяющая две точки на окружности круга \drawCircle[middle][1/4]{D}, целиком находится внутри круга.}

\startCenterAlign
Найдем центр \circleD\ \inprop[prop:III.I].

Из центра проведем \drawSizedLine{DE} к~любой точке \drawSizedLine{AB}, пересекающую окружность.

Проведем \drawSizedLine{AD} и~\drawSizedLine{BD}.

Тогда $\drawAngle{A} = \drawAngle{B}$ \inprop[prop:I.V],\\
но $\drawAngle{E} > \drawAngle{A} \mbox{ или } > \drawAngle{B}$ \inprop[prop:I.XVI].

$\therefore \drawSizedLine{AD} > \drawSizedLine{DE}$ \inprop[prop:I.XIX],\\
но $\drawSizedLine{AD} = \drawSizedLine{DE,EF}$.

$\therefore \drawSizedLine{DE,EF} > \drawSizedLine{DE}$.

$\therefore \drawSizedLine{DE} < \drawSizedLine{DE,EF}$.

$\therefore$ любая точка на \drawSizedLine{AB} находится внутри круга.
\stopCenterAlign

\qed
\stopProposition

\startProposition[title={Предл. III. Теорема},reference=prop:III.III]
\defineNewPicture[1/2]{
pair A, B, C, D, E, F;
numeric r;
r := 9/4u;
E := (0, 0);
A := (dir(-90 - 60) scaled r) shifted E;
B := (dir(-90 + 60) scaled r) shifted E;
C := (dir(90) scaled r) shifted E;
D := (dir(-90) scaled r) shifted E;
F = whatever[A, B] = whatever[C, D];
byAngleDefine(E, A, F, byblue, 0);
byAngleDefine(A, F, E, byblack, 0);
byAngleDefine(E, F, B, byyellow, 0);
byAngleDefine(F, B, E, byred, 0);
draw byNamedAngleResized();
draw byLine(C, E, byblack, 1, 0);
draw byLine(E, F, byblack, 0, 0);
draw byLine(F, D, byblack, 1, 0);
draw byLine(A, F, byred, 0, 0);
draw byLine(F, B, byred, 1, 0);
byLineDefine(A, E, byyellow, 0, 0);
byLineDefine(E, B, byblue, 0, 0);
draw byNamedLineSeq(0)(AE,EB);
draw byCircleR(E, r, byblue, 0, 0, 0);
draw byLabelsOnCircle(A, B)(E);
draw byLabelsOnPolygon(D, F, A)(2, -2);
draw byLabelsOnPolygon(A, E, C)(2, 0);
}
\drawCurrentPictureInMargin
\problemNP[3]{Е}{сли}{некоторая прямая \drawUnitLine{EF}, проходящая через центр круга \drawCircle[middle][1/6]{E}, рассекает пополам хорду \drawUnitLine{AF,FB}, не проходящую через центр, то эта прямая ей перпендикулярна; а~если перпендикулярна, то рассекает хорду пополам.}

\startCenterAlign
Проведем \drawUnitLine{AE} и~\drawUnitLine{EB} к~центру круга.

В \drawLine[bottom][triangleAEF]{AE,EF,AF} и~\drawLine[bottom][triangleFEB]{EF,EB,FB}\\
$\drawUnitLine{AE} = \drawUnitLine{EB}$, \drawUnitLine{EF} общая,\\
и $\drawUnitLine{AF} = \drawUnitLine{FB}$.

$\therefore \drawAngle{AFE} = \drawAngle{EFB}$ \inprop[prop:I.VIII]\\
и $\therefore \drawUnitLine{EF} \perp \drawUnitLine{AF,FB}$ \indef[def:I.X].

Теперь, пусть $\drawUnitLine{EF} \perp \drawUnitLine{AF,FB}$.

Тогда в~ \triangleAEF\ и~\triangleFEB\\
$\drawAngle{A} = \drawAngle{B}$ \inprop[prop:I.V]\\
$\drawAngle{AFE} = \drawAngle{EFB}$ (\hypstr)\\
и $\drawUnitLine{AE} = \drawUnitLine{EB}$.

$\therefore \drawUnitLine{AF} = \drawUnitLine{FB}$ \inprop[prop:I.XXVI].

И $\therefore \drawUnitLine{EF}$ рассекает \drawUnitLine{AF,FB} пополам.
\stopCenterAlign

\qed
\stopProposition

\startProposition[title={Предл. IV. Теорема},reference=prop:III.IV]
\defineNewPicture{
pair A, B, C, D, E, F;
numeric r;
r := 9/4u;
F := (0, 0);
A := (dir(-175)*r) shifted F;
B := (dir(-140)*r) shifted F;
C := (dir(-50)*r) shifted F;
D := (dir(-10)*r) shifted F;
E = whatever[A, C] = whatever[B, D];
byAngleDefine(F, E, D, byblue, 0);
byAngleDefine(D, E, C, byyellow, 0);
draw byNamedAngleResized();
draw byLine(E, F, byblack, 1, 0);
draw byLine(B, D, byred, 0, 0);
draw byLine(A, C, byblack, 0, 0);
draw byCircleR(F, r, byblue, 0, 0, 0);
draw byLabelsOnCircle(A, B, C, D)(F);
draw byLabelLineEnd(F, E, 0);
draw byLabelsOnPolygon(C, E, B)(2, 0);
}
\drawCurrentPictureInMargin
\problemNP{Е}{сли}{в круге две прямые, не проходящие через центр, пересекаются, они не делят друг друга пополам.}

Если одна из прямых проходит через центр, очевидно, она ее не может рассекать пополам другая прямая, не проходящая через центр.

Но если ни одна из прямых \drawUnitLine{AC} или \drawUnitLine{BD} не проходит через центр, проведем \drawUnitLine{EF} из центра к~точке их пересечения.

\startCenterAlign
Если \drawUnitLine{AC} делится пополам, \drawUnitLine{EF} $\perp$ ей \inprop[prop:III.III].

$\therefore \drawAngle{FED,DEC} = \drawRightAngle$.

И если \drawUnitLine{BD} делится пополам, $\drawUnitLine{EF} \perp \drawUnitLine{BD}$ \inprop[prop:III.III].

$\therefore \drawAngle{FED} = \drawRightAngle$.

И $\therefore \drawAngle{FED} = \drawAngle{FED,DEC}$,\\
часть равна целому, что невозможно.

$\therefore$ \drawUnitLine{AC} и~\drawUnitLine{BD} не~делят друг друга пополам.
\stopCenterAlign

\qed
\stopProposition

\startProposition[title={Предл. V. Теорема},reference=prop:III.V]
\defineNewPicture{
pair M, N, E, F, G, C;
numeric r[], s;
path c[];
r1 := 2u;
r2 := 2u;
s := u;
M := (1/2s, 0);
N := (-1/2s, 0);
c1 := (fullcircle scaled 2r1) shifted M;
c2 := (fullcircle scaled 2r2) shifted N;
E := 1/2[M, N];
C := (subpath(0, 4) of c1) intersectionpoint (subpath(0, 4) of c2);
G := point 7/2 of c2;
F := c1 intersectionpoint (E--G);
byLineDefine(C, E, byyellow, 0, 0);
byLineDefine(E, F, byblack, 0, 0);
byLineDefine(F, G, byblack, 1, 0);
draw byNamedLineSeq(0)(CE,EF,FG);
draw byArcBE.Ma(M, 4, 0, r1, byred, 0, 0, 0, 0);
draw byArcBE.Na(N, 4, 0, r2, byblue, 0, 0, 0, 0);
draw byArcBE.Nb(N, 4, 8, r2, byblue, 0, 0, 0, 0);
draw byArcBE.Mb(M, 4, 8, r1, byred, 0, 0, 0, 0);
draw byLabelLineEnd(C, E, 0);
draw byLabelLineEnd(G, E, 0);
draw byLabelsOnPolygon(C, E, F)(2, 0);
draw byLabelsOnPolygon(C, F, E)(2, -1);
}
\drawCurrentPictureInMargin
\problemNP{Е}{сли}{два круга секут друг друга \drawArc{Ma,Na,Nb,Mb} их центры не совпадают.}

Допустим это возможно, и~два пересекающихся круга имеют общий центр. Из предполагаемого центра проведем \drawUnitLine{CE} к~точке пересечения и~\drawUnitLine{EF,FG} пересекающую окружности.

\startCenterAlign
Тогда $\drawUnitLine{CE} = \drawUnitLine{EF}$ \indef[def:I.XV]\\
и $\drawUnitLine{CE} = \drawUnitLine{EF,FG}$ \indef[def:I.XV].

$\therefore \drawUnitLine{EF} = \drawUnitLine{EF,FG}$\\
часть равна целому, что невозможно.

$\therefore$ круги, пересекающиеся в~любой точке не могут иметь общего центра.
\stopCenterAlign

\qed
\stopProposition

\startProposition[title={Предл. VI. Теорема},reference=prop:III.VI]
\defineNewPicture[1/4]{
pair M, N, B, C, E, F;
numeric r[], a;
path c[];
a := 80;
r1 := 9/4u;
r2 := 7/4u;
M := (0, 0);
N := M shifted (dir(a)*(r1-r2));
c1 := (fullcircle scaled 2r1) shifted M;
c2 := (fullcircle scaled 2r2) shifted N;
F := 1/2[M, N];
C :=c1 intersectionpoint (M--(M shifted (dir(a)*2r1)));
B := point -3/2 of c1;
E := c2 intersectionpoint (F--B);
byLineDefine(C, F, byyellow, 0, 0);
byLineDefine(F, E, byblue, 1, 0);
byLineDefine(E, B, byblue, 0, 0);
draw byNamedLineSeq(0)(CF,FE,EB);
draw byCircle.M(M, C, byred, 0, 0, 0);
draw byCircle.N(N, C, byblack, 0, 0, -1);
draw byLabelsOnCircle(B, C)(M);
draw byLabelsOnPolygon(E, F, C)(2, 0);
draw byLabelPoint(E, angle(B-F)+45, 2);
byPointLabelRemove(M, N);
}
\drawCurrentPictureInMargin
\problemNP{Е}{сли}{два круга \drawCircle{N,M} касаются друг друга, то у~них не один и~тот же центр.}

Действительно, пусть это возможно, и~у кругов будет один центр. Из предполагаемого центра проведем \drawUnitLine{FE,EB} и~\drawUnitLine{CF} к~точке касания.

\startCenterAlign
Тогда $\drawUnitLine{CF} = \drawUnitLine{FE}$ \indef[def:I.XV]\\
и $\drawUnitLine{CF} = \drawUnitLine{FE,EB}$ \indef[def:I.XV].

$\therefore \drawUnitLine{FE} = \drawUnitLine{FE,EB}$.
\stopCenterAlign
\noindent Часть равна целому, что невозможно. Следовательно, выбранная точка не является центром обоих кругов и~таким же образом можно показать, что так же и~никакая другая.

\qed
\stopProposition

\startProposition[title={Предл. VII. Теорема},reference=prop:III.VII]
\defineNewPicture[1/2]{
pair A, B, C, D, E, F, G, H;
numeric r;
r := 2u;
E := (0, 0);
A := E shifted (dir(90)*r);
D := E shifted (dir(-90)*r);
F := 2/3[E, D];
B := E shifted (dir(90-70)*r);
C := E shifted (dir(-5)*r);
H := E shifted (dir(-170)*r);
G := E shifted (dir(90+70)*r);
draw byLine(F, B, byred, 0, 0);
draw byLine(F, C, byblue, 0, 0);
draw byLine(F, A, byblack, 0, 0);
draw byLine(F, D, byyellow, 0, 0);
draw byLine(F, G, byblue, 1, 0);
draw byCircleR(E, r, byblue, 0, 0, 0);
draw byLabelsOnCircle(A, B, C, D, G)(E);
draw byLabelsOnPolygon(D, F, G)(2, 0);
draw byLabelsOnPolygon(F, E, A)(2, 0);
}
\drawCurrentPictureInMargin
\problemNP[2]{Е}{сли}{из точки в~круге \drawFromCurrentPicture{
startTempScale(4/9);
draw byNamedCircle(E);
draw byLabelPoint(F, 0, 0);
stopTempScale;
}, не являющейся центром к~окружности проведены прямые линии
$\left\{\vcenter{
\nointerlineskip\hbox{\drawUnitLine{FA}, \drawUnitLine{FB}}
\nointerlineskip\hbox{\drawUnitLine{FC}, \drawUnitLine{FD}}
}\right.$
наибольшая из них та \drawUnitLine{FA}, что проходит через центр, а~меньшая — та, что является оставшейся частью диаметра \drawUnitLine{FD}. \\
Из остальных, та \drawUnitLine{FB}, что ближе к~проходящей через центр больше той \drawUnitLine{FC}, что проходит дальше.\\
Линии \drawUnitLine{FG} и~\drawUnitLine{FB} под равными углами к~линии, проходящей через центр и~по разные стороны от нее равны между собой и~из той же точки нельзя провести третью линию той же длины к~окружности.}

\defineNewPicture[1/2]{
pair A, B, C, D, E, F, G, H;
numeric r;
r := 2u;
E := (0, 0);
A := E shifted (dir(90)*r);
D := E shifted (dir(-90)*r);
F := 2/3[E, D];
B := E shifted (dir(20)*r);
C := E shifted (dir(-5)*r);
G := E shifted (dir(15)*r);
H := E shifted (dir(-170)*r);
byAngleDefine(B, E, C, byblack, 0);
byAngleDefine(C, E, F, byyellow, 0);
draw byNamedAngleResized();
draw byLine(F, B, byred, 0, 0);
draw byLine(E, C, byblue, 1, 0);
draw byLine(F, C, byblue, 0, 0);
draw byLine(E, B, byred, 1, 0);
draw byLine(A, E, byblack, 1, 0);
draw byLine(E, F, byblack, 0, 0);
draw byLine(F, D, byyellow, 0, 0);
draw byCircleR(E, r, byblue, 0, 0, 0);
draw byLabelsOnCircle(A, B, C, D)(E);
draw byLabelsOnPolygon(D, F, E, A)(2, 0);
}
\drawCurrentPictureInMargin
\startsubproposition[title={Часть I}]
Из центра круга проведем \drawUnitLine{EB} и~\drawUnitLine{EC}, тогда $\drawUnitLine{AE} = \drawUnitLine{EB}$ \indef[def:I.XV] $\drawUnitLine{EF,AE} = \drawUnitLine{EF} + \drawUnitLine{EB} > \drawUnitLine{FB}$ \inprop[prop:I.XX]. Таким же образом можно показать, что \drawUnitLine{EF,AE} больше \drawUnitLine{FC}, или любой другой линии, проведенной из той же точки к~окружности.

Теперь, согласно \inprop[prop:I.XX] $\drawUnitLine{EF} + \drawUnitLine{FC} > \drawUnitLine{EC} = \drawUnitLine{FD} + \drawUnitLine{EF}$, вычтем \drawUnitLine{EF} из обеих. $\therefore \drawUnitLine{FC} > \drawUnitLine{FD}$ \inax[ax:I.III], и~так же можно показать, что \drawUnitLine{FD} меньше любой другой линии, проведенной из той же точки к~окружности.

Теперь, в~\drawLine[middle][triangleEFB]{FB,EF,EB} и~\drawLine[middle][triangleEFC]{FC,EF,EC}, \drawUnitLine{EF} общая, $\drawAngle{BEC,CEF} > \drawAngle{CEF}$, и~$\drawUnitLine{EB} > \drawUnitLine{EC} \therefore \drawUnitLine{FB} > \drawUnitLine{FC}$ \inprop[prop:I.XXIV] и~так же можно показать, что \drawUnitLine{FB} больше любой другой линии, проведенной из той же точки к~окружности и~проходящей дальше \drawUnitLine{EF,AE}.
\stopsubproposition

\defineNewPicture[1/2]{
pair A, B, C, D, E, F, G, H, M;
numeric r;
r := 2u;
E := (0, 0);
A := E shifted (dir(90)*r);
D := E shifted (dir(-90)*r);
F := 2/3[E, D];
B := E shifted (dir(90-70)*r);
H := E shifted (dir(-170)*r);
G := E shifted (dir(90+70)*r);
M = whatever[E, H] = whatever[F, G];
byAngleDefine(B, F, E, byyellow, 0);
byAngleDefine(G, F, E, byred, 0);
draw byNamedAngleResized();
draw byLine(F, B, byred, 0, 0);
draw byLine(E, B, byred, 1, 0);
draw byLine(E, M, byyellow, 0, 0);
draw byLine(M, H, byyellow, 1, 0);
draw byLine(F, M, byblue, 0, 0);
draw byLine(M, G, byblue, 1, 0);
draw byLine(A, E, byblack, 1, 0);
draw byLine(E, F, byblack, 0, 0);
draw byLine(F, D, byblack, 0, 0);
draw byCircleR(E, r, byblue, 0, 0, 0);
draw byLabelsOnCircle(B, G, H)(E);
draw byLabelsOnPolygon(D, F, M, H)(2, 0);
draw byLabelsOnPolygon(M, E, A)(2, 0);
}
\drawCurrentPictureInMargin
\startsubproposition[title={Часть II}]
\startCenterAlign
Если $\drawAngle{GFE} = \drawAngle{BFE}$, то $\drawUnitLine{FM,MG} = \drawUnitLine{FB}$.

Если нет, возьмем $\drawUnitLine{FM} = \drawUnitLine{FB}$, проведем \drawUnitLine{EM,MH}.

Тогда в~\drawLine[middle][triangleEFM]{EM,EF,FM} и~\drawLine[middle][triangleEFB]{FB,EF,EB}, \drawUnitLine{EF} общая,\\
$\drawAngle{GFE} = \drawAngle{BFE}$ и~$\drawUnitLine{FB} = \drawUnitLine{FM}$.

$\therefore \drawUnitLine{EB} = \drawUnitLine{EM}$ \inprop[prop:I.IV].

$\therefore \drawUnitLine{EB} = \drawUnitLine{EM,MH} = \drawUnitLine{EM}$\\
часть равна целому, что невозможно.
\stopCenterAlign

\noindent $\therefore \drawUnitLine{FB} = \drawUnitLine{FM,MG}$; и~никакая другая линия равная \drawUnitLine{FB}, проведенная из той же точки к~окружности, поскольку, будь она ближе к~проходящей через центр, она была бы больше, а~дальше — меньше.
\stopsubproposition

\qed
\stopProposition

\startProposition[title={Предл. VIII. Теорема},reference=prop:III.VIII]
\problemNP{П}{редложение}{разделено на три части.}

\startsubproposition[title={I.}]
\defineNewPicture[1/4]{
pair M, D, A, E, F;
numeric r;
r := 7/4u;
M := (0, 0);
D := M shifted (dir(90)*3/2r);
A := (dir(-90)*r) shifted M;
E := (dir(-140)*r) shifted M;
F := (dir(-170)*r) shifted M;
byAngleDefine(D, M, F, byyellow, 0);
byAngleDefine(F, M, E, byblack, 0);
draw byNamedAngleResized();
draw byLine(D, E, byred, 0, 0);
draw byLine(M, A, byblack, 1, 0);
draw byLine(M, E, byred, 1, 0);
draw byLine(M, F, byblue, 1, 0);
byLineDefine(D, M, byblack, 0, 0);
byLineDefine(D, F, byblue, 0, 0);
draw byNamedLineSeq(0)(DM,DF);
draw byCircle.M(M, E, byblack, 0, 0, 0);
draw byLabelsOnCircle(F, E, A)(M);
draw byLabelsOnPolygon(F, D, M, A)(2, 0);
}
\drawCurrentPictureInMargin
Если из точки вне круга провести прямые линии \drawProportionalLine{DM,MA}, \drawProportionalLine{DE} и~\drawProportionalLine{DF} к~окружности, из тех, что падают на вогнутую часть окружности, наибольшей будет та \drawUnitLine{DM,MA}, что проходит через центр, а~та, что ближе к~ней \drawUnitLine{DE} будет длиннее той, что дальше \drawUnitLine{DF}.

\startCenterAlign
Проведем \drawUnitLine{MF} и~\drawUnitLine{ME} к~центру.

Тогда \drawUnitLine{DM,MA} проходящая через центр будет наибольшей,\\
поскольку, раз $\drawUnitLine{MA} = \drawUnitLine{ME}$, если к~обеим добавить \drawUnitLine{DM}, $\drawUnitLine{DM,MA} = \drawUnitLine{DM} + \drawUnitLine{ME}$,
но $> \drawUnitLine{DE}$ \inprop[prop:I.XX],\\
$\therefore$ \drawUnitLine{DM,MA} больше любой другой линии, проведенной из той же точки к~вогнутой части окружности.

Теперь в~\drawLine{DM,MF,DF} и~\drawLine{DM,ME,DE}, $\drawUnitLine{MF} = \drawUnitLine{ME}$, и~\drawUnitLine{DM} общая,\\
но $\drawAngle{DMF,FME} > \drawAngle{DMF}$, $\therefore \drawUnitLine{DE} > \drawUnitLine{DF}$ \inprop[prop:I.XXIV].

Так же можно показать, что $\drawUnitLine{DE} >$ любой другой линии, более далекой от \drawUnitLine{DM,MA}.
\stopCenterAlign
\stopsubproposition

\vfill\pagebreak

\startsubproposition[title={II.}]
\defineNewPicture{
pair M, D, G, H, K;
numeric r;
r := 7/4u;
M := (0, 0);
D := M shifted (dir(90)*2r);
G := (dir(90)*r) shifted M;
H := (dir(130)*r) shifted M;
K := (dir(110)*r) shifted M;
draw byLine(G, M, byblack, 0, 0);
draw byLine(H, M, byblue, 0, 0);
draw byLine(D, K, byred, 1, 0);
draw byLine(K, M, byred, 0, 0);
byLineDefine(D, G, byblack, 1, 0);
byLineDefine(D, H, byblue, 1, 0);
draw byNamedLineSeq(0)(DG,DH);
draw byCircle.M(M, G, byblack, 0, 0, 0);
draw byLabelsOnPolygon(K, H, M)(2, -2);
draw byLabelsOnPolygon(G, K, M)(2, -2);
draw byLabelsOnPolygon(D, G, K)(2, -2);
draw byLabelsOnPolygon(H, D, G)(2, 0);
draw byLabelsOnPolygon(G, M, H)(2, 0);
}
\drawCurrentPictureInMargin
Из линий, падающий на выпуклую часть окружности наименьшей \drawUnitLine{DG} является та, которая при продолжении проходит через центр, а~линия, ближняя к~наименьшей будет меньше дальней.

\startCenterAlign
Действительно, поскольку $\drawUnitLine{KM} + \drawUnitLine{DK} > \drawUnitLine{DG,GM}$ \inprop[prop:I.XX]\\
и $\drawUnitLine{KM} = \drawUnitLine{GM}$,\\
$\therefore \drawUnitLine{DK} > \drawUnitLine{DG}$ \inax[ax:I.V].

Значит, поскольку $\drawUnitLine{HM} + \drawUnitLine{DH} > \drawUnitLine{KM} + \drawUnitLine{DK}$ \inprop[prop:I.XXI],\\
и $\drawUnitLine{HM} = \drawUnitLine{KM}$,\\
$\therefore \drawUnitLine{DK} < \drawUnitLine{DH}$. Так же и~для прочих.
\stopCenterAlign
\stopsubproposition

\startsubproposition[title={III.}]
\defineNewPicture{
pair M, D, B, G, H, N, O;
numeric r;
r := 7/4u;
M := (0, 0);
D := M shifted (dir(90)*2r);
G := (dir(90)*r) shifted M;
B := (dir(90 - 20)*r) shifted M;
H := (dir(90 + 45)*r) shifted M;
N := (dir(90 - 45)*r) shifted M;
O = whatever[D, N] = whatever[M, B];
byAngleDefine(H, D, M, byyellow, 0);
byAngleDefine(M, D, N, byblue, 0);
draw byNamedAngleResized();
draw byLine(B, O, byred, 0, 0);
draw byLine(O, N, byyellow, 0, 0);
draw byLine(D, M, byblack, 0, 0);
draw byLine(H, M, byblue, 0, 0);
draw byLine(B, M, byblack, 1, 0);
byLineDefine(N, M, byyellow, 1, 0);
byLineDefine(D, H, byblue, 1, 0);
byLineDefine(D, O, byred, 1, 0);
draw byNamedLineSeq(0)(DH,DO,NM);
draw byCircle.M(M, H, byblack, 0, 0, 0);
draw byLabelsOnPolygon(H, D, O, N)(2, 0);
draw byLabelsOnPolygon(N, M, H)(2, 0);
draw byLabelsOnPolygon(G, H, M)(2, -2);
draw byLabelsOnPolygon(M, B, G)(2, -2);
draw byLabelsOnPolygon(M, N, B)(2, -2);
}
\drawCurrentPictureInMargin
Кроме того, линии, образующие равные углы с~проходящей через центр равны между собой, падают ли они на вогнутую или выпуклую часть окружности, и~нельзя провести третьей линии из той же точки той же длины к~окружности.

\startCenterAlign
Действительно, пусть $\drawUnitLine{DO,ON} > \drawUnitLine{DH}$.

Взяв $\drawAngle{HDM} = \drawAngle{MDN}$,\\
сделаем $\drawUnitLine{DO} = \drawUnitLine{DH}$ и~проведем \drawUnitLine{BM,BO}.

Тогда в~\drawLine{DM,DO,BO,BM} и~\drawLine{HM,DH,DM} получим $\drawUnitLine{DO} = \drawUnitLine{DH}$\\
и общую \drawUnitLine{DM}, а~также $\drawAngle{MDN} = \drawAngle{HDM}$.

$\therefore \drawUnitLine{BM,BO} = \drawUnitLine{HM}$ \inprop[prop:I.IV].

Но $\drawUnitLine{HM} = \drawUnitLine{BM}$.

$\therefore \drawUnitLine{BM} = \drawUnitLine{BM,BO}$, что невозможно.

$\therefore \drawUnitLine{DH} \neq \drawUnitLine{DO}$, ни какой либо часть \drawUnitLine{DO,ON}, $\therefore \drawUnitLine{DO,ON} \ngtr \drawUnitLine{DH}$.

Так же и~$\drawUnitLine{DH} \ngtr \drawUnitLine{DO,ON}$, они $\therefore  =$ друг другу.
\stopCenterAlign
Так же и~любая другая линия, проведенная из той же точки к~окружности, должна будет лежать по одну сторону с~одной из этих линий и~быть ближе или дальше, чем они от линии, проходящей через центр и~не может, следовательно, быть равной им.

\qed
\stopsubproposition
\stopProposition

\startProposition[title={Предл. IX. Теорема},reference=prop:III.IX]
\defineNewPicture[1/4]{
pair D, A, B, C, F, L, H;
numeric r;
r := 7/4u;
D := (0, 0);
A := (dir(170)*r) shifted D;
B := (dir(-90)*r) shifted D;
C := (dir(-45)*r) shifted D;
L := (dir(45)*r) shifted D;
H := (dir(45 + 180)*r) shifted D;
F := 2/4[D, L];
draw byLine(D, A, byyellow, 1, 0);
draw byLine(D, B, byyellow, 0, 0);
draw byLine(D, C, byblue, 0, 0);
draw byLine(D, F, byblack, 0, 0);
draw byLine(F, L, byred, 1, 0);
draw byLine(D, H, byred, 0, 0);
draw byCircleR(D, r, byblue, 0, 0, 0);
draw byLabelsOnCircle(A, B, C, H, L)(D);
draw byLabelsOnPolygon(A, D, F, L)(2, 0);
}
\drawCurrentPictureInMargin
\problemNP{Е}{сли}{внутри круга \drawCircle[middle]{D} взята точка, из которой к~окружности может быть проведено больше двух равных прямых линии \drawUnitLine{DA}, \drawUnitLine{DB}, \drawUnitLine{DC}, эта точка является центром круга.}

Действительно, если рассматриваемая точка \drawPointL[middle][DH,DA,DF]{D}, в~которой встречается больше двух равных прямых линий, не центр, то какая-то другая \drawFromCurrentPicture[middle][pointF]{
startGlobalRotation(-lineAngle.DF);
draw byNamedPointLines(F)("");
stopGlobalRotation;
} должна быть центром, проведем между этими двумя точками \drawUnitLine{DF} и~продлим в~обе стороны до окружности.

Теперь, поскольку более чем две равных прямых линии проведено к~окружности из точки, не являющейся центром, две из них должны лежать по одну сторону диаметра \drawUnitLine{DH,DF,FL}, и, поскольку из точки \drawPointL[middle][DA,DF]{D}, не являющейся центром прямые линии проведены к~окружности, наибольшая из них \drawUnitLine{DF,FL}, та, что проходит через центр, а~\drawUnitLine{DC}, которая ближе к~\drawUnitLine{DF,FL}, $> \drawUnitLine{DB}$ расположенной дальше \inprop[prop:III.VIII], но $\drawUnitLine{DC} = \drawUnitLine{DB}$ (\hypstr), что невозможно.

То же можно показать для любой другой точки, кроме \drawPointL[middle][DA,DF]{D}, которая, таким образом, должна быть центром круга.

\qed
\stopProposition

\startProposition[title={Предл. X. Теорема},reference=prop:III.X]
\defineNewPicture[1/5]{
pair P, G, H, B, d, dd;
pair Pd, Gd, Hd, Bd, Pdd;
numeric r, t[];
path cr[], crd[];
r := 7/4u;
d := (0, -9/4r);
P := (0, 0);
cr1 := ((fullcircle scaled 7/3r xscaled 4/5) rotated 45) shifted P;
cr2 := ((fullcircle scaled 7/3r xscaled 4/5) rotated -45) shifted P;
H := (subpath (0, 2) of cr1) intersectionpoint cr2;
B := (subpath (0, -2) of cr1) intersectionpoint cr2;
G := (subpath (-2, -4) of cr1) intersectionpoint cr2;
Pd := P shifted d;
crd1 := (fullcircle scaled 2r) shifted Pd;
dd := (0, -3/2r);
crd2 := crd1 shifted dd;
Pdd := Pd shifted dd;
t1 := xpart(crd2 intersectiontimes (subpath (-2, -4) of crd1));
t2 := xpart(crd2 intersectiontimes (subpath (0, -2) of crd1));
Bd := point t1 of crd2;
Hd := point t2 of crd2;
Gd := point -2 of crd1;
crd2 := subpath (t1, t2 + 8) of crd2;
crd2 := crd2 .. (point -2 of crd1) .. cycle;
draw byLine(P, B, byyellow, 0, 0);
draw byLine(P, G, byblack, 0, 0);
draw byLine(P, H, byblue, 0, 0);
draw byArbitraryFigure.fI(cr1, byred, 0, 0);
draw byArbitraryFigure.fII(cr2, byblue, 0, 0);
draw byLine(Pdd, Gd, byblack, 0, 0);
byLineDefine(Pdd, Bd, byyellow, 0, 0);
byLineDefine(Pdd, Hd, byblue, 0, 0);
draw byNamedLineSeq(0)(PddBd,PddHd);
draw byArbitraryFigure.fdI(crd1, byred, 0, 0);
draw byArbitraryFigure.fdII(crd2, byblue, 0, 0);
byCircleDefineR.PI(P, r, byred, 0, 0, 0);
byCircleDefineR.PII(P, r, byblue, 0, 0, 0);
draw byLabelsOnCircle(G, B, H)(PI);
draw byLabelsOnPolygon(G, P, H)(2, 0);
byPointLabelDefine(Gd, "G");
byPointLabelDefine(Hd, "H");
byPointLabelDefine(Bd, "B");
byPointLabelDefine(Dd, "D");
byPointLabelDefine(Pdd, "P");
draw byLabelLineEnd(Gd, Pdd, 0);
draw byLabelPoint(Bd, 180, 2);
draw byLabelPoint(Hd, 0, 2);
draw byLabelsOnPolygon(Hd, Pdd, Bd)(2, 0);
}
\drawCurrentPictureInMargin
\problemNP{К}{руг}{\drawFromCurrentPicture{draw byNamedCircle(PII);} не сечет круга \drawFromCurrentPicture[middle][circlePI]{draw byNamedCircle(PI);} более чем в~двух точках.}

\startCenterAlign
Действительно, будь такое возможно, пусть они пересекаются в~трех точках.

Из центра \drawCircle[middle][1/5]{PII} проведем \drawUnitLine{PG}, \drawUnitLine{PB} и~\drawUnitLine{PH} к~точкам пересечения.

$\therefore \drawUnitLine{PG} = \drawUnitLine{PB} = \drawUnitLine{PH}$ \indef[def:I.XV],\\
но, поскольку круги пересекаются, у~них не один центр \inprop[prop:III.V].

$\therefore$ рассматриваемая точка не центр \circlePI,\\
и~$\therefore$, поскольку \drawUnitLine{PG}, \drawUnitLine{PB} и~\drawUnitLine{PH} проведены не из центра,\\
они не равны (\inpropL[prop:III.VII], \inpropN[prop:III.VIII]).

Но выше было показано, что они равны, что невозможно, круги, стало быть, не пересекаются в~трех точках.
\stopCenterAlign

\qed
\stopProposition

\startProposition[title={Предл. XI. Теорема},reference=prop:III.XI]
\defineNewPicture[1/2]{
pair M, N, A, D, F, G, H, K;
numeric r[];
path cr[];
r1 := 9/4u;
r2 := 2/3r1;
M := (0, 0);
N := M shifted (0, +r1-r2);
cr1 := (fullcircle scaled 2r1) shifted M;
cr2 := (fullcircle scaled 2r2) shifted N;
A := (0, r1) shifted M;
F := 1/2[M,N] shifted (dir(-20)*1/3r2);
G := 1/2[M,N] shifted (dir(-20 + 180)*1/3r2);
D := cr2 intersectionpoint (F--10[F, G]);
H := cr1 intersectionpoint (F--10[F, G]);
K := cr1 intersectionpoint (F--10[G, F]);
draw byPolygon(A,F,G)(byyellow);
draw byLine(A, G, byred, 0, 0);
draw byLine(A, F, byblue, 1, 0);
draw byLine(H, D, byyellow, 0, 0);
draw byLine(D, G, byyellow, 1, 0);
draw byLine(G, F, byblack, 0, 0);
draw byLine(F, K, byblue, 0, 0);
byPointLabelDefine(M, "F");
byPointLabelDefine(N, "G");
draw byCircle.M(M, A, byblack, 0, 0, 0);
draw byCircle.N(N, A, byblue, 0, 0, -1);
draw byLabelsOnPolygon(K, F, G, D)(2, 0);
draw byLabelsOnPolygon(A, D, N)(2, -1);
draw byLabelsOnCircle(A, H)(M);
}
\drawCurrentPictureInMargin
\problemNP{Е}{сли}{два круга \drawCircle[middle][1/4]{N} и~\drawCircle[middle][1/5]{M} касаются между собой изнутри, прямая, соединяющая их центры, при продлении проходит через точку касания.}

\startCenterAlign
Действительно, если такое возможно, соединим центры с~помощью \drawSizedLine{GF} и~продлим в~обе стороны.

Из точки касания проведем \drawSizedLine{AG} к~сентру \circleN,\\
из той же точки касания проведем \drawSizedLine{AF} к~центру \circleM.

В
\drawFromCurrentPicture{
draw byNamedPolygon(AFG);
draw byNamedLineFull(A, A, 1, 1, 0)(GF);
}
$\drawSizedLine{GF} + \drawSizedLine{AG} > \drawSizedLine{AF}$ \inprop[prop:I.XX],\\
а $\drawSizedLine{AF} = \drawSizedLine{HD,DG,GF}$, как радиусы\circleM.

Но $\drawSizedLine{GF} + \drawSizedLine{AG} > \drawSizedLine{HD,DG,GF}$.

Вычтем \drawSizedLine{GF} общую обеим,\\
получим $\drawSizedLine{AG} > \drawSizedLine{HD,DG}$.

Но $\drawSizedLine{AG} = \drawSizedLine{DG}$, как радиусы \circleN,\\
и $\therefore \drawSizedLine{DG} > \drawSizedLine{HD,DG}$ часть больше целого, что невозможно.

\stopCenterAlign

Центры, следовательно, расположены так, что линия, соединяющая их не может проходить через какую-либо точку, кроме точки касания.

\qed
\stopProposition

\startProposition[title={Предл. XII. Теорема},reference=prop:III.XII]
\defineNewPicture[1/4]{
pair M, N, A, C, D, F, G;
numeric r[];
path cr[];
r1 := 3/2u;
r2 := 2u;
M := (0, 0);
N := (0, -r1-r2);
cr1 := (fullcircle scaled 2r1) shifted M;
cr2 := (fullcircle scaled 2r2) shifted N;
A := M shifted (0, -r1);
F := M shifted (dir(185)*1/2r1);
G := N shifted (dir(175)*1/2r2);
C := cr1 intersectionpoint (F--G);
D := cr2 intersectionpoint (F--G);
byLineDefine(F, C, byred, 0, 0);
byLineDefine(C, D, byblack, 0, 0);
byLineDefine(D, G, byblue, 0, 0);
byLineDefine(A, F, byyellow, 1, 0);
byLineDefine(A, G, byyellow, 0, 0);
draw byNamedLineSeq(0)(FC,CD,DG,AG,AF);
draw byCircle.M(M, A, byblue, 0, 0, -1/2);
draw byCircle.N(N, A, byred, 0, 0, -1/2);
byPointLabelDefine(M, "F");
byPointLabelDefine(N, "G");
draw byLabelsOnPolygon(C, F, A)(2, 0);
draw byLabelsOnPolygon(A, G, D)(2, 0);
draw byLabelsOnPolygon(A, C, F)(2, 0);
draw byLabelsOnPolygon(G, D, D)(2, 0);
draw byLabelPoint(A, angle(F-A)-45, 2);
}
\drawCurrentPictureInMargin
\problemNP[4]{Е}{сли}{два круга \drawCircle{M} и~\drawCircle{N} касаются друг друг извне, прямая \drawUnitLine{FC,CD,DG} соединяющая их центры проходит через точку касания.}

Действительно, если такое возможно, соединим центры с~помощью \drawUnitLine{FC,CD,DG}, не проходящей через точку касания, из точки касания проведем \drawUnitLine{AF} и~\drawUnitLine{AG} к~центрам.

\startCenterAlign
Поскольку $\drawUnitLine{AF} + \drawUnitLine{AG} > \drawUnitLine{FC,CD,DG}$ \inprop[prop:I.XX],\\
$\drawUnitLine{FC} = \drawUnitLine{AF}$ \indef[def:I.XV]\\
и $\drawUnitLine{DG} = \drawUnitLine{AG}$ \indef[def:I.XV],\\
$\therefore \drawUnitLine{FC} + \drawUnitLine{DG} > \drawUnitLine{FC,CD,DG}$, часть больше целого, что невозможно.
\stopCenterAlign

Центры, следовательно, расположены так, что линия, соединяющая их не может проходить через какую-либо точку, кроме точки касания.

\qed
\stopProposition

\startProposition[title={Предл. XIII. Теорема},reference=prop:III.XIII]
\problemNP{К}{руг}{не касается круга более чем в~одной точке, как внутри, так и~снаружи.}

\defineNewPicture{
pair M, N, F, G, H, B, D;
numeric r[];
path cr[];
r1 := 7/4u;
r2 := 3/4r1;
M := (0, 0);
N := (dir(120)*(r1-r2)) shifted M;
cr1 := (fullcircle scaled 2r1) shifted M;
cr2 := (fullcircle scaled 2r2) shifted N;
t1 := xpart(cr1 intersectiontimes (M--10[M, N]));
t2 := xpart(cr2 intersectiontimes (M--10[M, N]));
cr2 := (subpath (t2 + 2/3, t2 - 2/3 + 8) of cr2) .. tension 3/2 .. cycle;
B := point (t1 - 1/2) of cr1;
D := point (t1 + 1/2) of cr1;
G := 3/4[B, 1/2[M, N]];
H := 5/4[B, 1/2[M, N]];
draw byLine(D, G, byblack, 0, 0);
byLineDefine(D, H, byred, 0, 0);
byLineDefine(B, G, byblue, 1, 0);
byLineDefine(G, H, byblue, 0, 0);
draw byNamedLineSeq(0)(BG,GH,DH);
draw byCircleR(M, r1, byyellow, 0, 0, 1);
draw byArbitraryFigure.fI(cr2, byblue, 0, 0);
byCircleDefineR(M, r1, byyellow, 0, 0, 0);
byCircleDefineR(N, r2, byblue, 0, 0, 0);
byPointLabelDefine(M, "H");
byPointLabelDefine(N, "G");
draw byLabelsOnCircle(D, B)(M);
draw byLabelsOnPolygon(B, G, H, D)(2, 0);
}

\startsubproposition[title={Случай I.}]
\drawCurrentPictureInMargin
Действительно, если это возможно, пусть \drawCircle{M} и~\drawCircle{N} касаются друг друга внутри в~двух точках, проведем \drawUnitLine{GH} соединяющую их центры, и~продлим до одной из точек касания \inprop[prop:III.XI].

\startCenterAlign
Проведем \drawUnitLine{DH} и~\drawUnitLine{DG}.

Но $\drawUnitLine{BG} = \drawUnitLine{DG}$ \indef[def:I.XV].

$\therefore$ если добавить к~каждой \drawUnitLine{GH},\\
$\drawUnitLine{GH,BG} = \drawUnitLine{GH} + \drawUnitLine{DG}$.

Но $\drawUnitLine{GH,BG} = \drawUnitLine{DH}$ \indef[def:I.XV]\\
и $\therefore \drawUnitLine{GH} + \drawUnitLine{DG} = \drawUnitLine{DH}$.

Но $\therefore \drawUnitLine{GH} + \drawUnitLine{DG} > \drawUnitLine{DH}$ \inprop[prop:I.XX],\\
что невозможно.
\stopCenterAlign
\stopsubproposition

\vfill\pagebreak

\defineNewPicture{
pair P, Q, A, C;
numeric r[], t[];
path cr[];
r1 := 4/3u;
P := (0, 0);
Q := P shifted (1/4r1, 0);
cr3 := (fullcircle scaled 2r1) shifted P;
cr4 := (fullcircle scaled 2r1) shifted Q;
t3 := xpart(cr3 intersectiontimes (subpath (0, 4) of cr4));
t4 := xpart(cr3 intersectiontimes (subpath (4, 8) of cr4));
t5 := xpart(cr4 intersectiontimes (subpath (0, 4) of cr3));
t6 := xpart(cr4 intersectiontimes (subpath (4, 8) of cr3));
A := point t3 of cr3;
C := point t4 of cr3;
draw byLine(A, C, byred, 0, 0);
draw byMarkLine(3/7, byblack)(AC);
draw byMarkLine(4/7, byblack)(AC);
draw byArcBE.PI(P, t3, t4, r1, byblue, 0, 0, 0, 0);
draw byArcBE.PII(P, t4, t3 + 8, r1, byblack, 0, 0, 0, 0);
draw byArcBE.QI(Q, t5, t6 + 8, r1, byblue, 0, 0, 0, 0);
draw byArcBE.QII(Q, t5, t6, r1, byblack, 0, 0, 0, 0);
}

\startsubproposition[title={Случай II.}]
\drawCurrentPictureInMargin
Если точки касания на концах прямой линии, соединяющей центры, такая прямая должна быть рассечена пополам в~двух разных точках, соответствующих центрам, поскольку она является диаметром обоих кругов, что невозможно.
\stopsubproposition

\defineNewPicture{
pair G, H, A, C;
numeric r[];
path cr[];
r1 := 7/4u;
r2 := 3/4r1;
H := (0, 0);
G := H shifted (0, r1 + r2);
cr5 := (fullcircle scaled 2r1) shifted H;
cr6 := (fullcircle scaled 2r2) shifted G;
A := 1/2[point 2 of cr5, point 6 of cr6];
C := A shifted (1/3r2, 0);
cr5 := (subpath (2 + 2/4, 2 - 2/4 + 8) of cr5) .. tension 2 .. cycle;
cr6 := (subpath (6 + 2/4, 6 - 2/4 + 8) of cr6) .. tension 2 .. cycle;
byLineDefine(H, A, byred, 0, 0);
byLineDefine(A, G, byred, 1, 0);
byLineDefine(C, H, byblue, 1, 0);
byLineDefine(C, G, byblack, 0, 0);
draw byNamedLineSeq(0)(HA,AG,CG,CH);
draw byArbitraryFigure.fII(cr5, byyellow, 0, 0);
draw byArbitraryFigure.fIII(cr6, byblue, 0, 0);
byCircleDefineR(H, r1, byyellow, 0, 0, 0);
byCircleDefineR(G, r2, byblue, 0, 0, 0);
draw byLabelsOnPolygon(H, G, C)(0, 0);
draw byLabelPoint(A, angle(G-A)+45, 3);
}
\startsubproposition[title={Случай III.}]\drawCurrentPictureInMargin
Теперь, если такое возможно, пусть круги \drawCircle{H} и~\drawCircle{G} касаются друг друга снаружи в~двух точках. Проведем \drawUnitLine{HA,AG}, соединяющую центры, и~проходящую через одну из точек касания, проведем также \drawUnitLine{CH} и~\drawUnitLine{CG}.

\startCenterAlign
$\drawUnitLine{CH} = \drawUnitLine{HA}$ \indef[def:I.XV]\\
и $\drawUnitLine{AG} = \drawUnitLine{CG}$ \indef[def:I.XV].

$\therefore \drawUnitLine{CG} + \drawUnitLine{CH} = \drawUnitLine{HA,AG}$.

Но $\drawUnitLine{CG} + \drawUnitLine{CH} > \drawUnitLine{HA,AG}$ \inprop[prop:I.XX], что невозможно.
\stopCenterAlign
\stopsubproposition

Следовательно, ни в~каком случае круги не касаются друг друга в~двух точках.

\qed
\stopProposition

\startProposition[title={Предл. XIV. Теорема},reference=prop:III.XIV]
\defineNewPicture{
pair A, B, C, D, E, F, G;
numeric r;
r := 9/4u;
E := (0, 0);
A := (dir(90-20)*r) shifted E;
B := (dir(90-130)*r) shifted E;
C := (dir(90+20)*r) shifted E;
D := (dir(90+130)*r) shifted E;
F = whatever[A, B] = whatever[E, E shifted ((A-B) rotated 90)];
G = whatever[C, D] = whatever[E, E shifted ((C-D) rotated 90)];
byAngleDefine(E, F, A, byyellow, 0);
byAngleDefine(E, G, C, byblack, 1);
draw byNamedAngleResized();
draw byLine(E, A, byblack, 0, 0);
draw byLine(E, C, byblue, 0, 0);
byLineDefine(E, F, byblack, 1, 0);
byLineDefine(E, G, byblue, 1, 0);
draw byNamedLineSeq(0)(EF,EG);
draw byLine(A, F, byred, 0, 0);
draw byLine(F, B, byred, 1, 0);
draw byLine(C, G, byyellow, 0, 0);
draw byLine(G, D, byyellow, 1, 0);
draw byCircleR(E, r, byblue, 0, 0, 0);
draw byLabelsOnCircle(A, B, C, D)(E);
draw byLabelsOnPolygon(F, E, G)(2, 0);
draw byLabelLineEnd(G, E, 0);
draw byLabelLineEnd(F, E, 0);
}
\drawCurrentPictureInMargin
\problemNP{Р}{авные}{прямые
$\left(\vcenter{\nointerlineskip\hbox{\drawProportionalLine{AF,FB}}\nointerlineskip\hbox{\drawProportionalLine{CG,GD}}}\right)$
в круге равноотстоят от центра и~равноотстоящие от центра прямые равны.}

\startCenterAlign
Из центра \drawCircle[middle][1/4]{E}\\
проведем $\drawProportionalLine{EF} \perp \drawProportionalLine{AF,FB}$ и~$\drawProportionalLine{EG} \perp \drawProportionalLine{CG,GD}$,\\
проведем \drawProportionalLine{EA} и~\drawProportionalLine{EC}.

Тогда $\drawProportionalLine{CG} = \frac{1}{2} \drawProportionalLine{CG,GD}$ \inprop[prop:III.III]\\
и $\drawProportionalLine{AF} = \frac{1}{2} \drawProportionalLine{AF,FB}$ \inprop[prop:III.III].

Поскольку $\drawProportionalLine{CG,GD} = \drawProportionalLine{AF,FB}$ (\hypstr),\\
$\therefore \drawProportionalLine{CG} = \drawProportionalLine{AF}$ \inax[ax:I.VII]\\
и $\drawProportionalLine{EA} = \drawProportionalLine{EC}$ \indef[def:I.XV],\\
$\therefore \drawProportionalLine{EA}^2 = \drawProportionalLine{EC}^2$.

Но поскольку \drawAngle{F} прямой (\conststr),\\
$\drawProportionalLine{EA}^2 = \drawProportionalLine{EF}^2 + \drawProportionalLine{AF}^2$ \inprop[prop:I.XLVII]\\
и $\drawProportionalLine{EC}^2 = \drawProportionalLine{EG}^2 + \drawProportionalLine{CG}^2$ по той же причине.

$\therefore \drawProportionalLine{EF}^2 + \drawProportionalLine{AF}^2 = \drawProportionalLine{EG}^2 + \drawProportionalLine{CG}^2$.

$\therefore \drawProportionalLine{EF}^2 = \drawProportionalLine{EG}^2$.

$\therefore \drawProportionalLine{EF} = \drawProportionalLine{EG}$.
\stopCenterAlign

Также, если прямые \drawProportionalLine{AF,FB} и~\drawProportionalLine{CG,GD} равноудалены от центра, то есть перпендикуляры \drawProportionalLine{EF} и~\drawProportionalLine{EG} равны, $\drawProportionalLine{AF,FB} = \drawProportionalLine{CG,GD}$.

\startCenterAlign
Поскольку, как и~в предыдущем случае,\\
$\drawProportionalLine{EG}^2 + \drawProportionalLine{CG}^2 = \drawProportionalLine{AF}^2 + \drawProportionalLine{EF}^2$,\\
но $\drawProportionalLine{EG}^2 = \drawProportionalLine{EF}^2$.

$\therefore \drawProportionalLine{CG} = \drawProportionalLine{AF}$,\\ 
и~удвоенные $\drawProportionalLine{AF,FB} = \drawProportionalLine{CG,GD}$ тоже равны.
\stopCenterAlign

\qed
\stopProposition

\startProposition[title={Предл. XV. Теорема},reference=prop:III.XV]
\problemNP{В}{круге}{наибольшая прямая — диаметр, а~из других более близкая к~центру больше более удаленной.}

\startsubproposition[title={Случай I.}]
\defineNewPicture[1/2]{
pair A, D, E, F, G, M, N;
numeric r;
r := 9/4u;
E := (0, 0);
A := (r, 0) shifted E;
D := (-r, 0) shifted E;
F := (dir(90-30)*r) shifted E;
G := (dir(90+30)*r) shifted E;
M := (dir(90-60)*r) shifted E;
N := (dir(90+60)*r) shifted E;
byAngleDefine(N, E, G, byred, 0);
byAngleDefine(G, E, F, byyellow, 0);
byAngleDefine(F, E, M, byred, 0);
draw byNamedAngleResized();
draw byLine(E, M, byyellow, 1, 0);
draw byLine(E, N, byyellow, 0, 0);
draw byLine(E, F, byblue, 1, 0);
draw byLine(E, G, byblack, 1, 0);
draw byLine(D, E, byred, 0, 0);
draw byLine(E, A, byblack, 0, 0);
draw byLine(F, G, byred, 1, 0);
draw byLine(M, N, byblue, 0, 0);
draw byCircleR(E, r, byblack, 0, 0, 0);
draw byLabelsOnCircle(A, D, M, N, F, G)(E);
draw byLabelsOnPolygon(A, E, D)(2, 0);
}
\drawCurrentPictureInMargin
\startCenterAlign
Диаметр \drawUnitLine{DE,EA} $>$ любой прямой \drawUnitLine{MN}.

Действительно, проведем \drawUnitLine{EN} и~\drawUnitLine{EM}.

Тогда $\drawUnitLine{EM} = \drawUnitLine{EA}$\\
и $\drawUnitLine{EN} = \drawUnitLine{DE}$,\\
$\therefore \drawUnitLine{EN} + \drawUnitLine{EM} = \drawUnitLine{DE,EA}$,\\
но $\drawUnitLine{EN} + \drawUnitLine{EM} > \drawUnitLine{MN}$ \inprop[prop:I.XX].

$\therefore \drawUnitLine{DE,EA} > \drawUnitLine{MN}$.
\stopCenterAlign

Теперь покажем, что та прямая что ближе больше той, что дальше.

Пусть это будут \drawUnitLine{MN} и~\drawUnitLine{FG}, расположенные по одну сторону от центра и~не пересекающиеся.

\startCenterAlign
Проведем \drawUnitLine{EN}, \drawUnitLine{EM}, \drawUnitLine{EG} и~\drawUnitLine{EF}.

В
\drawFromCurrentPicture{
draw byNamedAngle(NEG,GEF,FEM);
startAutoLabeling;
draw byNamedLineSeq(0)(MN,EM,EN);
stopAutoLabeling;
}
и
\drawFromCurrentPicture{
draw byNamedAngle(GEF);
startAutoLabeling;
draw byNamedLineSeq(0)(FG,EF,EG);
stopAutoLabeling;
}\\
$\drawUnitLine{EN} \mbox{ и~} \drawUnitLine{EM} = \drawUnitLine{EG} \mbox{ и~} \drawUnitLine{EF}$,\\
но $\drawAngle{NEG,GEF,FEM} > \drawAngle{GEF}$.

$\therefore \drawUnitLine{MN} > \drawUnitLine{EF}$ \inprop[prop:I.XXIV]
\stopCenterAlign
\stopsubproposition

\vfill\pagebreak

\startsubproposition[title={Случай II.}]
\defineNewPicture{
pair A, B, C, D, E, F, G, M, N, K, L, H;
numeric r;
r := 9/4u;
E := (0, 0);
A := (r, 0) shifted E;
D := (-r, 0) shifted E;
F := (dir(90-30)*r) shifted E;
G := (dir(90+30)*r) shifted E;
B := (dir(-90-60)*r) shifted E;
C := (dir(-90+60)*r) shifted E;
M := (dir(90-60)*r) shifted E;
N := (dir(90+60)*r) shifted E;
H := 1/2[B, C];
K := 1/2[F, G];
L := 1/2[M, N];
draw byLine(E, L, byyellow, 1, 0);
draw byLine(L, K, byred, 1, 0);
draw byLine(E, H, byblue, 1, 0);
draw byLine(F, G, byyellow, 0, 0);
draw byLine(M, N, blue, 0, 0);
draw byLine(B, C, byred, 0, 0);
draw byLine(D, A, byblack, 0, 0);
draw byCircleR(E, r, byblack, 0, 0, 0);
draw byLabelsOnCircle(A, D, B, C, F, G, M, N)(E);
draw byLabelLineEnd(K, H, 0);
draw byLabelLineEnd(H, K, 0);
draw byLabelPoint(E, angle(E-H)-45, 2);
draw byLabelPoint(L, angle(E-H)-45, 2);
}
\drawCurrentPictureInMargin
\startCenterAlign
Теперь возьмем \drawUnitLine{BC} и~\drawUnitLine{GF} расположенные по разные стороны от центра или пересекающиеся. 

Проведем из центра\\
\drawUnitLine{EL,LK} и~$\drawUnitLine{EH} \perp \drawUnitLine{GF} \mbox{ и~} \drawUnitLine{BC}$.

Сделаем $\drawUnitLine{EH} = \drawUnitLine{EL}$,\\
и проведем $\drawUnitLine{NM} \perp \drawUnitLine{EL,LK}$.

Поскольку \drawUnitLine{BC} и~\drawUnitLine{NM} равноудалены от~центра, $\drawUnitLine{BC} = \drawUnitLine{NM}$ \inprop[prop:III.XIV],\\
но $\drawUnitLine{NM} > \drawUnitLine{GF}$ (Случай I).

$\therefore \drawUnitLine{BC} > \drawUnitLine{GF}$.
\stopCenterAlign
\stopsubproposition

\qed
\stopProposition

\startProposition[title={Предл. XVI. Теорема},reference=prop:III.XVI]
\defineNewPicture[1/5]{
pair A, B, C, D, E, F, G, H, K;
numeric r;
r :=2u;
D := (0, 0);
A := (0, -r);
B := (0, r);
C := (dir(190)*r) shifted D;
E := (4/3r, -r);
F := (4/3r, -1/3r);
G := 11/12[A, F];
H := (dir(angle(G-D))*r) shifted D;
K := (-r, -r);
byAngleDefine(C, A, D, byyellow, 0);
byAngleDefine(D, A, G, byblue, 0);
byAngleDefine(G, A, E, byred, 0);
byAngleDefine(D, C, A, byblack, 0);
byAngleDefine(A, G, D, byblack, 1);
draw byNamedAngleResized();
draw byLine(A, C, byred, 0, 0);
draw byLine(D, C, byblue, 0, 0);
draw byLine(D, H, byblue, 1, 0);
draw byLine(H, G, byblack, 1, 0);
draw byLine(A, G, byred, 1, 0);
draw byLine(G, F, byblack, 1, 0);
draw byLine(B, D, byyellow, 1, 0);
draw byLine(D, A, byblack, 0, 0);
draw byLineFull(E, K, byyellow, 0, 0)(E, K, 0, 0, -1);
draw byCircle.D(D, A, byblue, 0, 0, 0);
draw byLabelsOnCircle(B, C)(D);
draw byLabelsOnPolygon(E, A, K, noPoint)(0, -1);
draw byLabelsOnPolygon(F, G, A)(2, 0);
draw byLabelsOnPolygon(C, D, B)(2, 0);
draw byLabelPoint(H, angle(G-H)+45, 2);
}
\drawCurrentPictureInMargin
\problemNP{П}{рямая}{\drawUnitLine{EK}, проведенная под прямым углом к~диаметру круга \drawUnitLine{BD,DA} проходит вне круга.\\
И если любая прямая линия \drawUnitLine{AG} проведена из точки по ту же сторону от перпендикуляра к~точке касания, она сечет круг.}

\startsubproposition[title={Часть I.}]
Если это возможно, пусть \drawUnitLine{AC}, пересекающая окружность в~еще одном месте будет $\perp \drawUnitLine{DA}$, и~проведем \drawUnitLine{DC}.

\startCenterAlign
Тогда, поскольку $\drawUnitLine{DA} = \drawUnitLine{DC}$,\\
$\drawAngle{CAD} = \drawAngle{C}$ \inprop[prop:I.V],\\
и $\therefore$ оба угла острые \inprop[prop:I.XVII],\\
но $\drawAngle{CAD} = \drawRightAngle$ (\hypstr), что невозможно.

Следовательно \drawUnitLine{AC} проведенная $\perp \drawUnitLine{DA}$ не пересекает окружность в~другом месте.
\stopCenterAlign
\stopsubproposition

\startsubproposition[title={Часть II.}]
Пусть $\drawUnitLine{EK} \perp \drawUnitLine{DA}$ и~пусть \drawUnitLine{AG} проведена из точки \drawFromCurrentPicture{draw byNamedPointLines(G,"GF");} между \drawUnitLine{EK} и~кругом, и, если возможно, не сечет круг.

\startCenterAlign
$\drawAngle{DAG,GAE} = \drawRightAngle$, $\therefore \drawAngle{DAG}$ острый угол.

Допустим $\drawUnitLine{DH,HG} \perp \drawUnitLine{AG}$, проведенной из центра круга, должна падать на сторону острого угла \drawAngle{DAG}.

$\therefore$ \drawAngle{G}, который должен быть прямым $> \drawAngle{DAG}$.

$\therefore \drawUnitLine{DA} > \drawUnitLine{DH,HG}$.

Но $\drawUnitLine{DH} = \drawUnitLine{DA}$.
\stopCenterAlign

И $\therefore \drawUnitLine{DH} > \drawUnitLine{DH,HG}$, часть больше целого, что невозможно. Следовательно, точка не находится вовне круга и, следовательно, прямая \drawUnitLine{AG} сечет круг.
\stopsubproposition

\qed
\stopProposition

\startProposition[title={Предл. XVII. Задача},reference=prop:III.XVII]
\defineNewPicture[1/4]{
pair A, B, D, E, F;
numeric r[], a;
path cr[];
r1 := 6/4u;
r2 := 9/4u;
E := (0, 0);
cr1 := (fullcircle scaled 2r1) shifted E;
cr2 := (fullcircle scaled 2r2) shifted E;
A := (dir(50)*r2) shifted E;
D := (dir(50)*r1) shifted E;
F := cr2 intersectionpoint (D--D shifted (dir(angle(A-E) - 90)*r2));
B := cr1 intersectionpoint (E--F);
a := angle(B-E);
forsuffixes i=A, B, D, F:
i := ((i shifted -E) rotated -a) shifted E;
endfor;
byAngleDefine(A, B, E, byyellow, 0);
byAngleDefine(F, D, E, byyellow, 0);
byAngleDefine(F, E, A, byblue, 0);
draw byNamedAngleResized();
draw byLine(A, B, byblue, 0, 0);
draw byLine(F, D, byblue, 1, 0);
byLineDefine(A, D, byred, 0, 0);
byLineDefine(D, E, byred, 1, 0);
byLineDefine(F, B, byblack, 0, 0);
byLineDefine(B, E, byblack, 1, 0);
draw byNamedLineSeq(0)(AD,DE,BE,FB);
draw byCircleR.EI(E, r1, byred, 0, 0, -1);
draw byCircleR.EII(E, r2, byyellow, 0, 0, 0);
draw byLabelsOnCircle(A, F)(EII);
draw byLabelsOnPolygon(F, E, A)(2, 0);
draw byLabelPoint(D, angle(A-E)+45, 2);
draw byLabelPoint(B, angle(F-E)-45, 2);
}
\drawCurrentPictureInMargin
\problemNP{П}{ровести}{касательную к~данному кругу \drawCircle[middle][1/5]{EI} из данной точки.}

Если данная точка \drawPointL[middle][FB]{B} расположена на окружности, ясно, что прямая $\drawUnitLine{AB} \perp \drawUnitLine{BE}$ радиусу и~будет искомой касательной \inprop[prop:III.XVI].

\startCenterAlign
Но если точка \drawPointL[middle][FB]{A} расположена вовне, проведем из нее \drawUnitLine{AD,DE} к~центру, секущую \circleEI,\\
и проведем $\drawUnitLine{FD} \perp \drawUnitLine{DE}$,\\
опишем \drawCircle[middle][1/6]{EII} концентрический с~\circleEI\\ 
с~радиусом $= \drawUnitLine{DE,AD}$,\\
проведем \drawUnitLine{BE,FB} к~центру из точки, где \drawUnitLine{FD} падает на окружность \circleEII,\\
проведем $\drawUnitLine{AB} \perp \drawUnitLine{BE,FB}$ из точки, где та сечет \circleEI.

Тогда \drawUnitLine{AB} и~будет искомой касательной.\\
Поскольку в~\drawLine[bottom]{FD,FB,BE,DE} и~\drawLine[bottom]{BE,DE,AD,AB} $\drawUnitLine{AD,DE} = \drawUnitLine{FB,BE}$, \drawAngle{E} общий,\\
и $\drawUnitLine{DE} = \drawUnitLine{BE}$,\\
$\therefore \mbox{ \inprop[prop:I.IV] } \drawAngle{B} = \drawAngle{D} = \drawRightAngle$,\\
$\therefore \drawUnitLine{FD}$ касательная к~\circleEI.
\stopCenterAlign

\qed
\stopProposition

\startProposition[title={Предл. XVIII. Теорема},reference=prop:III.XVIII]
\defineNewPicture{
pair B, C, D, F, G;
numeric r;
r := 7/4u;
F := (0, 0);
C := (r, 0);
G := (r, 6/5r);
D := 6/5[C, G];
B := (dir(angle(G-F))*r) shifted F;
draw byCircle.F(F, C, byyellow, 0, 0, 0);
byAngleDefine(F, C, G, byred, 0);
byAngleDefine(C, G, F, byyellow, 0);
draw byNamedAngleResized();
byLineDefine(F, B, byred, 0, 0);
byLineDefine(B, G, byred, 1, 0);
byLineDefine(C, D, byblue, 1, 0);
byLineDefine(F, C, byblue, 0, 0);
draw byNamedLineSeq(0)(BG,FB,FC,CD);
draw byLabelsOnCircle(C)(F);
draw byLabelsOnPolygon(C, F, G)(2, 0);
draw byLabelPoint(B, angle(G-F)+45, 2);
draw byLabelPoint(G, angle(D-C)-90, 1);
draw byLabelPoint(D, angle(D-C)-90, 1);
}
\drawCurrentPictureInMargin
\problemNP{Е}{сли}{прямая \drawUnitLine{CD} касается круга, прямая \drawUnitLine{FC}, проведенная из центра к~точке касания перпендикулярна ей.}

\startCenterAlign
Действительно, если возможно, пусть \drawUnitLine{FB,BG} будет $\perp \drawUnitLine{CD}$,\\
тогда, поскольку $\drawAngle{G} = \drawRightAngle$, \drawAngle{C} острый \inprop[prop:I.XVII].

$\therefore \drawUnitLine{FC} > \drawUnitLine{FB,BG}$ \inprop[prop:I.XIX].

Но $\drawUnitLine{FC} = \drawUnitLine{FB}$,\\
и $\therefore \drawUnitLine{FB} > \drawUnitLine{FB,BG}$, часть больше целого, что невозможно.

$\therefore \drawUnitLine{FB,BG}$ не $\perp \drawUnitLine{CD}$.
\stopCenterAlign

И так же можно показать, что никакая другая прямая, кроме \drawUnitLine{FC} не перпендикулярна \drawUnitLine{CD}.

\qed
\stopProposition

\startProposition[title={Предл. XIX. Теорема},reference=prop:III.XIX]
\defineNewPicture{
pair A, C, E, F, G;
numeric r;
r := 7/4u;
G := (0, 0);
A := (-r, 0) shifted G;
C := (r, 0) shifted G;
E := (r, 6/5r) shifted G;
F := (-1/7r, 1/2r) shifted G;
byAngleDefine(A, C, F, byblue, 0);
byAngleDefine(F, C, E, byyellow, 0);
draw byNamedAngleResized();
draw byLine(A, C, byyellow, 0, 0);
draw byLine(C, E, byblue, 0, 0);
draw byLine(C, F, byred, 1, 0);
draw byCircleR(G, r, byred, 0, 0, 0);
draw byLabelsOnCircle(A, C)(G);
draw byLabelLineEnd(F, C, 0);
draw byLabelLineEnd(E, C, 0);
}
\drawCurrentPictureInMargin
\problemNP{Е}{сли}{прямая \drawUnitLine{CE} касается круга, прямая  \drawUnitLine{AC}, перпендикулярная ей, проведенная из точки касания, проходит через центр круга.}

Действительно, если центр не находится на \drawUnitLine{AC}, проведем \drawUnitLine{CF} к~предполагаемому центру из точки касания.

\startCenterAlign
Поскольку $\drawUnitLine{CF} \perp \drawUnitLine{CE}$ \inprop[prop:III.XVIII]\\
$\therefore \drawAngle{FCE} = \drawRightAngle$, прямому углу.

Но $\drawAngle{ACF,FCE} = \drawRightAngle$ (\hypstr).

И $\therefore \drawAngle{FCE} = \drawAngle{ACF,FCE}$,\\
часть равна целому, что невозможно.
\stopCenterAlign

Следовательно, предполагаемая точка не центр, и~то же можно показать для любой другой точки, не лежащей на \drawUnitLine{AC}.

\qed
\stopProposition

\startProposition[title={Предл. XX. Теорема},reference=prop:III.XX]
\problemNP{У}{гол,}{при центре круга вдвое больше угла при окружности, когда их основание лежит на одной дуге.}\unskip

\defineNewPicture{
pair A, E, F, C;
r := 7/4u;
E := (0, 0);
A := (dir(80)*r) shifted E;
F := (dir(80 + 180)*r) shifted E;
C := (dir(-30)*r) shifted E;
byAngleDefine(C, A, F, byyellow, 0);
byAngleDefine(C, E, F, byblue, 0);
byAngleDefine(E, C, A, byred, 0);
draw byNamedAngleResized();
draw byLine(A, C, byblack, 0, 1);
draw byLine(E, C, byblack, 0, 0);
draw byLine(E, F, byred, 1, 0);
draw byLine(E, A, byred, 0, 0);
draw byCircleR(E, r, byblue, 0, 0, 0);
draw byLabelsOnCircle(F, A, C)(E);
draw byLabelsOnPolygon(F, E, A)(2, 0);
}
\drawCurrentPictureInMargin
\startsubproposition[title={Случай I.}]
\startCenterAlign
Пусть центр круга будет на \drawUnitLine{EF,EA},\\
стороне \drawAngle{CAF}.

Поскольку $\drawUnitLine{EC} = \drawUnitLine{EA}$,\\
$\drawAngle{CAF} = \drawAngle{C}$ \inprop[prop:I.V].

Но $\drawAngle{CEF} = \drawAngle{CAF} + \drawAngle{C}$,\\
или $\drawAngle{CEF} = \mbox{ дважды } \drawAngle{CAF}$ \inprop[prop:I.XXXII].
\stopCenterAlign
\stopsubproposition

\defineNewPicture{
pair A, E, F, C, B;
r := 7/4u;
E := (0, 0);
A := (dir(80)*r) shifted E;
F := (dir(80 + 180)*r) shifted E;
B := (dir(185)*r) shifted E;
C := (dir(-30)*r) shifted E;
byAngleDefine(C, A, F, byyellow, 0);
byAngleDefine(C, E, F, byblue, 0);
byAngleDefine(E, C, A, byyellow, 0);
byAngleDefine(B, A, F, byred, 0);
byAngleDefine(B, E, F, byblack, 0);
byAngleDefine(E, B, A, byred, 0);
draw byNamedAngleResized();
draw byLine(A, C, byblack, 0, 1);
draw byLine(E, C, byblack, 0, 1);
draw byLine(A, B, byblack, 0, 1);
draw byLine(E, B, byblack, 0, 1);
draw byLine(A, F, byblack, 0, 0);
draw byCircleR(E, r, byblue, 0, 0, 0);
draw byLabelsOnCircle(F, A, B, C)(E);
draw byLabelsOnPolygon(B, E, A)(2, 0);
}
\drawCurrentPictureInMargin
\startsubproposition[title={Случай II.}]
\startCenterAlign
Пусть центр будет в~\drawAngle{BAF,CAF}, углу на окружности.

Проведем \drawUnitLine{AF} из угла через центр круга.

Тогда $\drawAngle{B} = \drawAngle{BAF}$, и~$\drawAngle{C} = \drawAngle{CAF}$, вследствие равенства сторон \inprop[prop:I.V].

Значит $\drawAngle{BAF} + \drawAngle{B} + \drawAngle{CAF} + \drawAngle{C} = \mbox{ дважды } \drawAngle{BAF,CAF}$.

Но $\drawAngle{BEF} = \drawAngle{BAF} + \drawAngle{B}$\\
и $\drawAngle{CEF} = \drawAngle{CAF} + \drawAngle{C}$.

$\therefore \drawAngle{BEF,CEF} = \mbox{ дважды } \drawAngle{BAF,CAF}$.
\stopCenterAlign
\stopsubproposition

\defineNewPicture{
pair E, C, F, G, D;
r := 7/4u;
E := (0, 0);
F := (dir(80 + 180)*r) shifted E;
C := (dir(-30)*r) shifted E;
D := (dir(30)*r) shifted E;
G := (dir(30 + 180)*r) shifted E;
byAngleDefine(G, E, F, byblue, 0);
byAngleDefine(F, E, C, byyellow, 0);
byAngleDefine(G, D, F, byblack, 0);
byAngleDefine(F, D, C, byred, 0);
draw byNamedAngleResized();
draw byLine(D, C, byblack, 0, 1);
draw byLine(D, F, byblack, 0, 1);
draw byLine(E, C, byblack, 0, 1);
draw byLine(E, F, byblack, 0, 1);
draw byLine(D, G, byred, 0, 0);
draw byCircleR(E, r, byblue, 0, 0, 0);
draw byLabelsOnCircle(D, G, C, F)(E);
draw byLabelsOnPolygon(G, E, D)(2, 0);
}
\drawCurrentPictureInMargin
\startsubproposition[title={Случай III.}]
\startCenterAlign
Пусть центр будет вовне \drawAngle{FDC}.

Проведем диаметр \drawUnitLine{DG}.

Поскольку $\drawAngle{GEF,FEC} = \mbox{ дважды } \drawAngle{GDF,FDC}$
и $\drawAngle{GEF} = \mbox{ дважды } \drawAngle{GDF}$ (случай I.),\\
$\therefore \drawAngle{FEC} = \mbox{ дважды } \drawAngle{FDC}$.
\stopCenterAlign
\stopsubproposition

\qed
\stopProposition

\startProposition[title={Предл. XXI. Теорема},reference=prop:III.XXI]
\problemNP{В}{круге}{углы в~одном сегменте равны между собой.}

\defineNewPicture{
pair A, B, D, E, F;
numeric r;
r := 2u;
F := (0, 0);
B := (dir(-90 - 50)*r) shifted F;
D := (dir(-90 + 50)*r) shifted F;
A := (dir(90 + 25)*r) shifted F;
E := (dir(90 - 35)*r) shifted F;
byAngleDefine(B, A, D, byred, 0);
byAngleDefine(B, E, D, byblue, 0);
byAngleDefine(B, F, D, byyellow, 0);
draw byNamedAngleResized();
byLineDefine(B, F, byblue, 0, 0);
byLineDefine(D, F, byred, 0, 0);
draw byNamedLineSeq(0)(BF,DF);
draw byLine(B, D, byblack, 1, 0);
draw byLine(B, A, byblack, 0, 1);
draw byLine(B, E, byblack, 0, 1);
draw byLine(D, A, byblack, 0, 1);
draw byLine(D, E, byblack, 0, 1);
draw byCircleR(F, r, byblue, 0, 0, 0);
draw byLabelsOnCircle(B, D, E, A)(F);
draw byLabelsOnPolygon(B, F, D)(2, 0);
}\drawCurrentPictureInMargin
\startsubproposition[title={Случай I.}]
Пусть сегмент будет больше половины круга, проведем \drawUnitLine{DF} и~\drawUnitLine{BF} к~центру.

\startCenterAlign
$\drawAngle{F} = \mbox{ дважды } \drawAngle{A} \mbox{ или дважды } = \drawAngle{E}$ \inprop[prop:III.XX].

$\therefore \drawAngle{A} = \drawAngle{E}$.
\stopCenterAlign
\stopsubproposition

\defineNewPicture{
pair A, B, D, E, F, G;
numeric r;
path cr;
r := 2u;
F := (0, 0);
cr := (fullcircle scaled 2r) shifted F;
B := (dir(90 + 85)*r) shifted F;
D := (dir(90 - 85)*r) shifted F;
A := (dir(90 + 25)*r) shifted F;
E := (dir(90 - 35)*r) shifted F;
G := (dir(-90 + 20)*r) shifted F;
byAngleDefine(B, A, G, byyellow, 0);
byAngleDefine(G, A, D, byred, 0);
byAngleDefine(B, E, G, byblue, 0);
byAngleDefine(G, E, D, byblack, 0);
draw byNamedAngleResized();
draw byFilledCircleSegment.BG(F, r, xpart(cr intersectiontimes (F -- 2[F, B])), xpart(cr intersectiontimes (F -- 2[F, G])), byblue);
draw byFilledCircleSegment.GD(F, r, xpart(cr intersectiontimes (F -- 2[F, G])), 8 + xpart(cr intersectiontimes (F -- 2[F, D])), byyellow);
draw byLine(G, A, byblue, 0, 0);
draw byLine(G, E, byred, 0, 0);
draw byLine(B, D, byblack, 1, 0);
draw byLine(B, A, byblack, 0, 1);
draw byLine(B, E, byblack, 0, 1);
draw byLine(D, A, byblack, 0, 1);
draw byLine(D, E, byblack, 0, 1);
draw byCircleR(F, r, byblue, 0, 0, 0);
draw byLabelsOnCircle(B, D, E, A, G)(F);
}
\drawCurrentPictureInMargin
\startsubproposition[title={Случай II.}]
Пусть сегмент будет меньше или равен половине круга, проведем диаметр \drawUnitLine{GA}, также проведем \drawUnitLine{GE}.

\startCenterAlign
$\drawAngle{BAG} = \drawAngle{BEG}$ и~$\drawAngle{GAD} = \drawAngle{GED}$ (случай I.).

$\therefore \drawAngle{BAG,GAD} = \drawAngle{BEG,GED}$.
\stopCenterAlign
\stopsubproposition

\qed
\stopProposition

\startProposition[title={Предл. XXII. Теорема},reference=prop:III.XXII]
\defineNewPicture{
pair A, B, C, D, E;
numeric r;
r := 7/4u;
E := (0, 0);
A := (dir(80)*r) shifted E;
B := (dir(10)*r) shifted E;
C := (dir(-100)*r) shifted E;
D := (dir(150)*r) shifted E;
byAngleDefine(D, A, C, byred, 0);
byAngleDefine(C, A, B, byblue, 0);
byAngleDefine(A, B, D, byyellow, 0);
byAngleDefine(D, B, C, byred, 0);
byAngleDefine(B, C, A, byblack, 0);
byAngleDefine(A, C, D, byyellow, 0);
byAngleDefine(C, D, B, byblue, 0);
byAngleDefine(B, D, A, byblack, 0);
draw byNamedAngleResized();
draw byLine(A, B, byblack, 0, 1);
draw byLine(B, C, byblack, 0, 1);
draw byLine(C, D, byblack, 0, 1);
draw byLine(D, A, byblack, 0, 1);
draw byLine(A, C, byred, 0, 0);
draw byLine(B, D, byblack, 0, 0);
draw byCircleR(E, r, byred, 0, 0, 1/2);
draw byLabelsOnCircle(A, B, C, D)(E);
}
\drawCurrentPictureInMargin
\problemNP[2]{П}{ротивоположные}{углы \drawAngle{DAC,CAB} и~\drawAngle{BCA,ACD} или \drawAngle{CDB,BDA} и~\drawAngle{ABD,DBC} четырехугольника вписанного в~круг, вместе равны двум прямым углам.}

\startCenterAlign
Проведем диагонали \drawUnitLine{AC} и~\drawUnitLine{BD}.

Поскольку углы в~одной дуге равны,\\
$\drawAngle{CDB} = \drawAngle{CAB}$\\
и $\drawAngle{DAC} = \drawAngle{DBC}$.

Добавим к~каждому \drawAngle{BCA,ACD},\\
$\drawAngle{DAC,CAB} + \drawAngle{BCA,ACD} = \drawAngle{BCA,ACD} + \drawAngle{CDB} + \drawAngle{DBC} = \drawTwoRightAngles$ \inprop[prop:I.XXXII].

Так же можно показать, что\\
$\drawAngle{CDB,BDA} + \drawAngle{ABD,DBC} = \drawTwoRightAngles$.
\stopCenterAlign

\qed
\stopProposition

\startProposition[title={Предл. XXIII. Теорема},reference=prop:III.XXIII]
\defineNewPicture{
pair A, B, C, D, M, N;
numeric r[], t[];
path cr[];
r1 := 14/6u;
r2 := 15/6u;
M := (0, 0);
N := (0, 1/3r1);
cr1 := (fullcircle scaled 2r1) shifted M;
cr2 := (fullcircle scaled 2r2) shifted N;
t1 := xpart(cr1 intersectiontimes (subpath (-2, 2) of cr2));
t2 := xpart(cr1 intersectiontimes (subpath (2, 6) of cr2));
t3 := xpart(cr2 intersectiontimes (subpath (-2, 2) of cr1));
t4 := xpart(cr2 intersectiontimes (subpath (2, 6) of cr1));
A := point t1 of cr1;
B := point t2 of cr1;
C := point 3/2 of cr1;
D := cr2 intersectionpoint (1/2[A,C]--2[A,C]);
byAngleDefine(A, C, B, byyellow, 0);
byAngleDefine(A, D, B, byblue, 0);
draw byNamedAngleResized();
draw byLineFull(A, D, byred, 0, 0)(B, D, 1, 0, 0);
draw byLineFull(B, C, byblue, 0, 0)(A, C, 1, 0, 0);
draw byLineFull(B, D, byyellow, 0, 0)(A, D, 1, 0, 0);
draw byLineFull(A, B, byblack, 0, 0)(A, B, 0, 0, 1);
draw byArc.M(M, A, B, r1, byred, 0, 0, 0, 1);
draw byArc.N(N, A, B, r2, byblue, 0, 0, 0, 1);
draw byLabelsOnPolygon(A, B, noPoint)(0, 0);
draw byLabelsOnPolygon(B, D, A)(2, 0);
draw byLabelPoint(C, (angle(C-M)-90+angle(D-A))/2, 1);
}
\drawCurrentPictureInMargin
\problemNP{Н}{а}{одной прямой и~по одну ее сторону нельзя построить два подобных и~неравных сегмента круга.}

\startCenterAlign
Действительно, если это возможно, построим два подобных сегмента\\
\drawFromCurrentPicture[bottom]{
draw byNamedLine(AB);
draw byNamedArcExact(N);
draw byLabelsOnPolygon(A, B, noPoint)(0, 0);
}
и
\drawFromCurrentPicture[bottom]{
draw byNamedLine(AB);
draw byNamedArcExact(M);
draw byLabelsOnPolygon(A, B, noPoint)(0, 0);
}.

Проведем прямую \drawUnitLine{AD}, секущую оба.

Проведем \drawUnitLine{BC} и~\drawUnitLine{BD}.

Поскольку сегменты подобны,\\
$\drawAngle{C} = \drawAngle{D}$ \indef[def:III.XI].

Но $\drawAngle{C} > \drawAngle{D}$ \inprop[prop:I.XVI], что невозможно.

Следовательно, никакая точка одного сегмента не расположена вовне другого сегмента и~значит сегменты совпадают.
\stopCenterAlign

\qed
\stopProposition

\startProposition[title={Предл. XXIV. Теорема},reference=prop:III.XXIV]
\defineNewPicture{
pair A, B, C, D, M, N, d;
numeric r, t[];
path cr[];
r := 5/2u;
t1 := 2-1;
t2 := 2+1;
d := (0, -3/2u);
M := (0, 0);
N := M shifted d;
cr1 := ((subpath (t1, t2) of fullcircle) scaled 2r) shifted M;
cr2 := ((subpath (t1, t2) of fullcircle) scaled 2r) shifted N;
A := point length(cr1) of cr1;
B := point 0 of cr1;
C := point length(cr2) of cr2;
D := point 0 of cr2;
draw byFilledCircleSegment(M, r, t1, t2, byred);
draw byLineFull(A, B, byblack, 0, 0)(A, B, 0, 0, -1);
draw byArc.M(M, B, A, r, byblue, 0, 0, 1/2, 1);
draw byFilledCircleSegment(N, r, t1, t2, byyellow);
draw byLineFull(C, D, byblue, 0, 0)(C, D, 0, 0, -1);
draw byArc.N(N, D, C, r, byred, 0, 0, 1/2, 1);
draw byLabelsOnPolygon(B, A, noPoint)(0, 0);
draw byLabelsOnPolygon(D, C, noPoint)(0, 0);
}
\drawCurrentPictureInMargin
\problemNP{П}{одобные}{сегменты
\drawFromCurrentPicture[bottom]{
draw byNamedFilledCircleSegment(M);
draw byNamedLine(AB);
draw byNamedArcExact(M);
draw byLabelsOnPolygon(B, A, noPoint)(0, 0);
}
и
\drawFromCurrentPicture[bottom]{
draw byNamedFilledCircleSegment(N);
draw byNamedLine(CD);
draw byNamedArcExact(N);
draw byLabelsOnPolygon(D, C, noPoint)(0, 0);
}
кругов на равных прямых \drawUnitLine{AB} и~\drawUnitLine{CD} равны между собой.}

\startCenterAlign
Поскольку, если
\drawFromCurrentPicture[bottom]{
draw byNamedFilledCircleSegment(N);
draw byLabelsOnPolygon(D, C, noPoint)(0, 0);
}
наложить на
\drawFromCurrentPicture[bottom]{
draw byNamedFilledCircleSegment(M);
draw byLabelsOnPolygon(B, A, noPoint)(0, 0);
},\\
так, что \drawUnitLine{CD} совпадет с~\drawUnitLine{AB},\\
концы \drawUnitLine{CD} будут на концах \drawUnitLine{AB}\\
и \drawArc{M} по одну сторону с~\drawArc{N}.

Поскольку $\drawUnitLine{CD} = \drawUnitLine{AB}$,\\
\drawUnitLine{CD} будет полностью совпадать с~\drawUnitLine{AB}.
\stopCenterAlign

\noindent Равные сегменты на одной прямой и~по одну ее сторону также совпадают \inprop[prop:III.XXIII], и, следовательно, равны.

\qed
\stopProposition

\startProposition[title={Предл. XXV. Задача},reference=prop:III.XXV]
\defineNewPicture{
pair A, B, C, D, E, F, O;
numeric r;
r := 7/4u;
O := (0, 0);
A := (dir(-20)*r) shifted O;
B := (dir(85)*r) shifted O;
C := (dir(180)*r) shifted O;
D := 1/2[A, B];
E := 1/2[B, C];
F = whatever[D, D shifted ((A-B) rotated 90)] = whatever[E, E shifted ((B-C) rotated 90)];
byLineDefine(D, F, byred, 0, 0);
byLineDefine(E, F, byyellow, 0, 0);
draw byNamedLineSeq(0)(DF,EF);
draw byLine(A, B, byblack, 0, 0);
draw byLine(B, C, byblue, 0, 0);
draw byArcBE(O, -6/5, 5, r, byblue, 0, 0, 1/2, 0);
draw byLabelsOnCircle(A, B, C)(O);
draw byLabelLineEnd(D, F, 0);
draw byLabelLineEnd(E, F, 0);
draw byLabelsOnPolygon(D, F, E)(2, 0);
}
\drawCurrentPictureInMargin
\problemNP{К}{данному}{сегменту круга пристроить круг, сегментом которого он является.}

\startCenterAlign
Из любой точки сегмента проведем \drawUnitLine{BC} и~\drawUnitLine{AB}.

Рассечем обе пополам и~из точек рассечения.

Проведем $\drawUnitLine{EF} \perp \drawUnitLine{BC}$\\
и $\drawUnitLine{DF} \perp \drawUnitLine{AB}$.

Где они пересекаются, там и~находится центр круга.
\stopCenterAlign

Поскольку \drawUnitLine{BC} кончающаяся на окружности рассекается перпендикуляром \drawUnitLine{EF}, который проходит через центр \inprop[prop:III.I], так же и~\drawUnitLine{DF} проходит через центр, следовательно, их пересечение и~есть центр.

\qed
\stopProposition

\startProposition[title={Предл. XXVI. Теорема},reference=prop:III.XXVI]
\defineNewPicture[1/4]{
pair A, B, C, D, E, F, G, H, d;
numeric r, t[];
path cr[];
r := 7/4u;
t1 := 5;
t2 := 7;
G := (0, 0);
cr1 := (fullcircle scaled 2r) shifted G;
A := point 3/2 of cr1;
B := point t1 of cr1;
C := point t2 of cr1;
d := (0, -5/2r);
H := G shifted d;
cr2 := (fullcircle scaled 2r) shifted H;
D := point 5/2 of cr2;
E := point t1 of cr2;
F := point t2 of cr2;
byAngleDefine(B, A, C, byred, 0);
byAngleDefine(B, G, C, byyellow, 0);
draw byNamedAngleResized(BAC, BGC);
draw byFilledCircleSegment(G, r, t1, t2, byyellow);
draw byLine(B, C, byblack, 0, 0);
draw byLine(A, B, byblack, 0, 1);
draw byLine(A, C, byblack, 0, 1);
byLineDefine(B, G, byblue, 0, 0);
byLineDefine(C, G, byred, 0, 0);
draw byNamedLineSeq(0)(BG,CG);
draw byArc.G(G, C, B, r, byblue, 0, 0, 0, 0);
draw byArc.Gb(G, B, C, r, byblack, 0, 0, 0, 0);
byCircleDefineR(G, r, byblue, 0, 0, 0);
byAngleDefine(E, D, F, byblue, 0);
byAngleDefine(E, H, F, byblack, 0);
draw byNamedAngleResized(EDF, EHF);
draw byFilledCircleSegment(H, r, t1, t2, byyellow);
draw byLine(E, F, byblack, 1, 0);
draw byLine(D, E, byblack, 0, 1);
draw byLine(D, F, byblack, 0, 1);
byLineDefine(E, H, byblue, 1, 0);
byLineDefine(F, H, byred, 1, 0);
draw byNamedLineSeq(0)(EH,FH);
draw byArc.H(H, F, E, r, byred, 0, 0, 0, 0);
draw byArc.Hb(H, E, F, r, byblack, 0, 0, 0, 0);
byCircleDefineR(H, r, byred, 0, 0, 0);
draw byLabelsOnCircle(A, B, C)(G);
draw byLabelsOnCircle(D, E, F)(H);
draw byLabelsOnPolygon(B, G, C)(2, 0);
draw byLabelsOnPolygon(E, H, F)(2, 0);
}
\drawCurrentPictureInMargin
\problemNP[4]{В}{равных}{кругах \drawCircle[middle][1/5]{G} и~\drawCircle[middle][1/5]{H} дуги \drawArc{Gb} и~\drawArc{Hb}, на которые опираются равные, что при центре, что при окружности, углы, равны.}

\startCenterAlign
Поскольку, пусть $\drawAngle{G} = \drawAngle{H}$ при центре,\\
проведем \drawUnitLine{BC} и~\drawUnitLine{EF}.

Тогда, поскольку $\circleG = \circleH$,\\
у \drawLine[bottom]{BG,CG,BC} и~\drawLine[bottom]{EH,FH,EF}\\
$\drawUnitLine{BG} = \drawUnitLine{CG} = \drawUnitLine{EH} = \drawUnitLine{FH}$\\
и $\drawAngle{G} = \drawAngle{H}$.

$\therefore \drawUnitLine{BC} = \drawUnitLine{EF}$ \inprop[prop:I.IV].

Но $\drawAngle{A} = \drawAngle{D}$ \inprop[prop:III.XX].

$\therefore$
\drawFromCurrentPicture{
startTempScale(1/5);
draw byNamedArc(G);
draw byNamedLine(BC);
draw byLabelsOnCircle(B, C)(G);
stopTempScale;
}
и
\drawFromCurrentPicture{
startTempScale(1/5);
draw byNamedArc(H);
draw byNamedLine(EF);
draw byLabelsOnCircle(E, F)(H);
stopTempScale;
}
подобны \indef[def:III.XI]\\
и равны \inprop[prop:III.XXIV].

А если равные сегменты вычесть из равных кругов, оставшиеся сегменты будут равны,\\
то есть $
\drawFromCurrentPicture{
draw byNamedFilledCircleSegment(G);
draw byNamedArcLabel(Gb);
}
=
\drawFromCurrentPicture{
draw byNamedFilledCircleSegment(H);
draw byNamedArcLabel(Hb);
}
$ \inax[ax:I.III].

И $\therefore \drawArc{Gb} = \drawArc{Hb}$.
\stopCenterAlign

И если равные углы будут при окружности, очевидно, что углы в~центре, будучи вдвое больше углов при окружности, также равны, и, следовательно, дуги, на которые они опираются, тоже равны.

\qed
\stopProposition

\startProposition[title={Предл. XXVII. Теорема},reference=prop:III.XXVII]
\defineNewPicture[1/2]{
pair A, B, C, D, E, F, G, H, K, d;
numeric r, t[];
path cr[];
r := 7/4u;
t1 := 5;
t2 := 7;
t3 := 13/2;
G := (0, 0);
cr1 := (fullcircle scaled 2r) shifted G;
A := point 3/2 of cr1;
B := point t1 of cr1;
C := point t2 of cr1;
K := point t3 of cr1;
d := (0, -5/2r);
H := G shifted d;
cr2 := (fullcircle scaled 2r) shifted H;
D := point 3/2 of cr2;
E := point t1 of cr2;
F := point t3 of cr2;
byAngleDefine(B, A, K, byyellow, 0);
byAngleDefine(B, G, K, byyellow, 0);
byAngleDefine(K, A, C, byblue, 0);
byAngleDefine(K, G, C, byblue, 0);
draw byNamedAngleResized(BAK, BGK, KAC, KGC);
draw byLine(A, K, byblack, 0, 1);
draw byLine(A, B, byblack, 0, 1);
draw byLine(A, C, byblack, 0, 1);
byLineDefine(G, B, byblack, 0, 1);
byLineDefine(G, C, byblack, 0, 1);
draw byNamedLineSeq(0)(GB,GC);
draw byLine(G, K, byblack, 0, 1);
draw byArc.G(G, C, B, r, byred, 0, 0, 0, 0);
draw byArc.GbI(G, B, K, r, byblack, 0, 0, 0, 0);
draw byArc.GbII(G, K, C, r, byred, 1, 0, 0, 0);
byCircleDefineR(G, r, byred, 0, 0, 0);
byAngleDefine(E, D, F, byred, 0);
byAngleDefine(E, H, F, byred, 0);
draw byNamedAngleResized(EDF, EHF);
draw byLine(D, F, byblack, 0, 1);
draw byLine(D, E, byblack, 0, 1);
byLineDefine(H, E, byblack, 0, 1);
byLineDefine(H, F, byblack, 0, 1);
draw byNamedLineSeq(0)(HE,HF);
draw byArc.H(H, F, E, r, byblue, 0, 0, 0, 0);
draw byArc.Hb(H, E, F, r, byblack, 1, 0, 0, 0);
byCircleDefineR(H, r, byblue, 0, 0, 0);
draw byLabelsOnCircle(A, B, C, K)(G);
draw byLabelsOnCircle(D, E, F)(H);
draw byLabelsOnPolygon(B, G, C)(2, 0);
draw byLabelsOnPolygon(E, H, F)(2, 0);
}
\drawCurrentPictureInMargin
\problemNP[2]{В}{равных}{кругах \drawCircle[middle][1/3]{G} и~\drawCircle[middle][1/3]{H}, углы \drawAngle{BAK} и~\drawAngle{D}, опирающиеся на равные дуги, равны между собой, будь они при центрах или при окружностях.}

\startCenterAlign
Поскольку, если такое возможно, пусть один из них\\
\drawAngle{D} будет больше другого \drawAngle{BAK}.

Cделаем $\drawAngle{BAK,KAC} = \drawAngle{D}$.

$\therefore \drawArc{GbI,GbII} = \drawArc{Hb}$ \inprop[prop:III.XXVI].

Но $\drawArc{GbI} = \drawArc{Hb}$ (\hypstr)

$\therefore \drawArc{GbI} = \drawArc{GbI,GbII}$ часть равны целому, что невозможно.

$\therefore$ ни один из углов не больше другого,\\
и $\therefore$ они равны.
\stopCenterAlign

\qed
\stopProposition

\startProposition[title={Предл. XXVIII. Теорема},reference=prop:III.XXVIII]
\defineNewPicture[1/2]{
pair A, B, D, E, K, L, d;
numeric r, t[];
path cr[];
r := 7/4u;
t1 := 5;
t2 := 7;
K := (0, 0);
cr1 := (fullcircle scaled 2r) shifted K;
A := point t1 of cr1;
B := point t2 of cr1;
d := (0, -5/2r);
L := K shifted d;
cr2 := (fullcircle scaled 2r) shifted L;
D := point t1 of cr2;
E := point t2 of cr2;
byAngleDefine(A, K, B, byred, 0);
draw byNamedAngleResized(AKB);
draw byLine(A, B, byred, 0, 0);
byLineDefine(A, K, byblack, 0, 0);
byLineDefine(B, K, byblue, 0, 0);
draw byNamedLineSeq(0)(AK,BK);
draw byArc.K(K, B, A, r, byyellow, 0, 0, 0, 0);
draw byArc.Kb(K, A, B, r, byblue, 0, 0, 0, 0);
byCircleDefineR(K, r, byyellow, 0, 0, 0);
byAngleDefine(D, L, E, byyellow, 0);
draw byNamedAngleResized(DLE);
draw byLine(D, E, byred, 1, 0);
byLineDefine(D, L, byblack, 1, 0);
byLineDefine(E, L, byblue, 1, 0);
draw byNamedLineSeq(0)(DL,EL);
draw byArc.L(L, E, D, r, byblack, 0, 0, 0, 0);
draw byArc.Lb(L, D, E, r, byred, 0, 0, 0, 0);
byCircleDefineR(L, r, byblack, 0, 0, 0);
draw byLabelsOnCircle(A, B)(K);
draw byLabelsOnCircle(D, E)(L);
draw byLabelsOnPolygon(A, K, B)(2, 0);
draw byLabelsOnPolygon(D, L, E)(2, 0);
}
\drawCurrentPictureInMargin
\problemNP{В}{равных}{кругах \drawCircle[middle][1/3]{K} и~\drawCircle[middle][1/3]{L} равные хорды \drawUnitLine{AB} и~\drawUnitLine{DE} отсекают равные дуги.}

\startCenterAlign
Из центров равных кругов,\\
проведем \drawUnitLine{AK}, \drawUnitLine{BK} и~\drawUnitLine{DL}, \drawUnitLine{EL}.

Поскольку $\circleK = \circleL$\\
$\drawUnitLine{AK}, \drawUnitLine{BK} = \drawUnitLine{DL}, \drawUnitLine{EL}$,\\
а также $\drawUnitLine{AB} = \drawUnitLine{DE}$ (\hypstr).

$\therefore \drawAngle{K} = \drawAngle{L}$.

$\therefore \drawArc{Kb} = \drawArc{Lb}$ \inprop[prop:III.XXVI].

И $\therefore \drawArc[bottom][1/3]{K} = \drawArc[bottom][1/3]{L}$ \inax[ax:I.III].
\stopCenterAlign

\qed
\stopProposition

\startProposition[title={Предл. XXIX. Теорема},reference=prop:III.XXIX]
\defineNewPicture[1/2]{
pair A, B, D, E, K, L, d;
numeric r, t[];
path cr[];
r := 7/4u;
t1 := 5;
t2 := 7;
K := (0, 0);
cr1 := (fullcircle scaled 2r) shifted K;
A := point t1 of cr1;
B := point t2 of cr1;
d := (0, -5/2r);
L := K shifted d;
cr2 := (fullcircle scaled 2r) shifted L;
D := point t1 of cr2;
E := point t2 of cr2;
byAngleDefine(A, K, B, byred, 0);
draw byNamedAngleResized(AKB);
draw byLine(A, B, byred, 0, 0);
byLineDefine(A, K, byblack, 0, 0);
byLineDefine(B, K, byblue, 0, 0);
draw byNamedLineSeq(0)(AK,BK);
draw byArc.K(K, B, A, r, byyellow, 0, 0, 0, 0);
draw byArc.Kb(K, A, B, r, byblue, 0, 0, 0, 0);
byCircleDefineR(K, r, byyellow, 0, 0, 0);
byAngleDefine(D, L, E, byyellow, 0);
draw byNamedAngleResized(DLE);
draw byLine(D, E, byred, 1, 0);
byLineDefine(D, L, byblack, 1, 0);
byLineDefine(E, L, byblue, 1, 0);
draw byNamedLineSeq(0)(DL,EL);
draw byArc.L(L, E, D, r, byblack, 0, 0, 0, 0);
draw byArc.Lb(L, D, E, r, byred, 0, 0, 0, 0);
byCircleDefineR(L, r, byblack, 0, 0, 0);
draw byLabelsOnCircle(A, B)(K);
draw byLabelsOnCircle(D, E)(L);
draw byLabelsOnPolygon(A, K, B)(2, 0);
draw byLabelsOnPolygon(D, L, E)(2, 0);
}
\drawCurrentPictureInMargin
\problemNP{В}{равных}{кругах \drawCircle[middle][1/3]{K} и~\drawCircle[middle][1/3]{L}, хорды \drawUnitLine{AB} и~\drawUnitLine{DE}, стягивающие равные дуги, равны.}

\startCenterAlign
Если дуги равны половинам окружностей, то предложение очевидно, а~если нет,\\
пусть \drawUnitLine{AK}, \drawUnitLine{BK} и~\drawUnitLine{DL}, \drawUnitLine{EL}\\
проведены из центров.

Поскольку $\drawArc{Kb} = \drawArc{Lb}$ (\hypstr),\\
$\drawAngle{K} = \drawAngle{L}$ \inprop[prop:III.XXVII].

Но $\drawUnitLine{AK} \mbox{ и~} \drawUnitLine{BK} = \drawUnitLine{DL} \mbox{ и~} \drawUnitLine{EL}$.

$\therefore \drawUnitLine{AB} = \drawUnitLine{DE}$ \inprop[prop:I.IV],\\
а это и~есть хорды, стягивающие равные дуги.
\stopCenterAlign

\qed
\stopProposition

\startProposition[title={Предл. XXX. Задача},reference=prop:III.XXX]
\defineNewPicture{
pair A, B, C, D, E;
numeric r, t[];
path cr;
r := 9/4u;
t1 := -1/2;
t2 := 4 + 1/2;
t3 := 2;
E := (0, 0);
cr := (fullcircle scaled 2r) shifted E;
A := point t2 of cr;
B := point t1 of cr;
D := point t3 of cr;
C := 1/2[A, B];
byAngleDefine(A, C, D, byblue, 0);
byAngleDefine(D, C, B, byred, 0);
draw byNamedAngleResized();
draw byLine(D, A, byblue, 0, 0);
draw byLine(D, C, byyellow, 0, 0);
draw byLine(D, B, byblue, 1, 0);
draw byLine(A, C, byblack, 0, 0);
draw byLine(C, B, byblack, 1, 0);
draw byArc.Er(E, B, D, r, byred, 1, 0, 0, 0);
draw byArc.El(E, D, A, r, byred, 0, 0, 0, 0);
draw byLabelsOnCircle(A, B, D)(Er);
draw byLabelLineEnd(C, D, 0);
}
\drawCurrentPictureInMargin
\problemNP{Р}{ассечь}{данную дугу \drawArc[middle][1/5]{El,Er} пополам.}

\startCenterAlign
Проведем \drawUnitLine{AC,CB}.

Сделаем $\drawUnitLine{AC} = \drawUnitLine{CB}$ \inprop[prop:I.X].

Проведем $\drawUnitLine{DC} \perp \drawUnitLine{AC,CB}$ \inprop[prop:I.XI], она и~будет рассекать дугу.

Проведем \drawUnitLine{DA} и~\drawUnitLine{DB}.

Тогда у \drawLine[bottom]{AD,DC,CA} и \drawLine[bottom]{DB,BC,CD}\\
$\drawUnitLine{AC} = \drawUnitLine{CB}$ (\conststr)\\
\drawUnitLine{DC} общая,\\
и $\drawAngle{ACD} = \drawAngle{DCB}$ (\conststr).

$\therefore \drawUnitLine{DA} = \drawUnitLine{DB}$ \inprop[prop:I.IV]\\
и $
\drawFromCurrentPicture{
startGlobalRotation(-arcAngle.El);
startAutoLabeling;
draw byNamedArc(El);
stopAutoLabeling;
stopGlobalRotation;
}
=
\drawFromCurrentPicture{
startGlobalRotation(-arcAngle.Er);
startAutoLabeling;
draw byNamedArc(Er);
stopAutoLabeling;
stopGlobalRotation;
}
$ \inprop[prop:III.XXVIII],\\
и значит данная дуга рассечена.
\stopCenterAlign

\qed
\stopProposition

\startProposition[title={Предл. XXXI. Теорема},reference=prop:III.XXXI]
\problemNP{В}{круге}{угол заключенный в~полукруге прямой, в~сегменте больше полукруга острый, в~сегменте меньше полукруга тупой.}

\defineNewPicture{
pair A, B, C, E;
numeric r;
r := 7/4u;
E := (0, 0);
A := (dir(110)*r) shifted E;
B := (dir(180)*r) shifted E;
C := (dir(0)*r) shifted E;
byAngleDefine(A, B, C, byred, 0);
byAngleDefine(B, C, A, byblue, 0);
byAngleDefine(E, A, B, byyellow, 0);
byAngleDefine(C, A, E, byblack, 0);
draw byNamedAngleResized();
draw byLine(A, E, byred, 0, 0);
draw byLine(A, B, byblack, 0, 1);
draw byLine(A, C, byblack, 0, 1);
draw byLine(B, E, byblue, 0, 0);
draw byLine(E, C, byblack, 0, 0);
draw byCircleR(E, r, byblack, 0, 0, 0);
draw byLabelsOnCircle(A, B, C)(E);
draw byLabelsOnPolygon(C, E, B)(2, 0);
}
\drawCurrentPictureInMargin
\startsubproposition[title={Случай I.}]
\startCenterAlign
Угол \drawAngle{EAB,CAE} в~полукруге прямой.

Проведем \drawUnitLine{AE} и~\drawUnitLine{BE,EC}.

$\drawAngle{B}=\drawAngle{EAB}$ и~$\drawAngle{C} = \drawAngle{CAE}$ \inprop[prop:I.V]\\
$\drawAngle{C} + \drawAngle{B} = \drawAngle{EAB,CAE} = \mbox{ половина } \drawTwoRightAngles= \drawRightAngle$ \inprop[prop:I.XXXII].
\stopCenterAlign
\stopsubproposition

\defineNewPicture{
pair A, B, C, E, D;
numeric r;
r := 7/4u;
E := (0, 0);
A := (dir(110)*r) shifted E;
B := (dir(180 + 30)*r) shifted E;
C := (dir(0 + 30)*r) shifted E;
D := (dir(0 - 30)*r) shifted E;
byAngleDefine(B, A, D, byblue, 0);
byAngleDefine(D, A, C, byred, 0);
draw byNamedAngleResized();
draw byLine(A, D, byblack, 0, 1);
draw byLine(A, B, byblack, 0, 1);
draw byLine(B, D, byblack, 0, 1);
draw byLine(A, C, byblue, 0, 0);
draw byLine(B, C, byred, 0, 0);
draw byCircleR(E, r, byblue, 0, 0, 0);
draw byLabelsOnCircle(A, B, C, D)(E);
}
\drawCurrentPictureInMargin
\startsubproposition[title={Случай II.}]
\startCenterAlign
Угол \drawAngle{BAD} в~сегменте больше полукруга острый.

Проведем диаметр \drawUnitLine{BC} и~\drawUnitLine{AC}

$\therefore \drawAngle{BAD,DAC} = \mbox{ прямому углу}$.

$\therefore$ \drawAngle{BAD} острый.
\stopCenterAlign
\stopsubproposition

\vfill\pagebreak

\defineNewPicture{
pair A, B, C, E, D;
numeric r;
r := 7/4u;
E := (0, 0);
A := (dir(140)*r) shifted E;
B := (dir(-110)*r) shifted E;
C := (dir(5)*r) shifted E;
D := (dir(80)*r) shifted E;
byAngleDefine(A, D, C, byred, 0);
byAngleDefine(C, B, A, byyellow, 0);
draw byNamedAngleResized();
draw byLine(A, B, byred, 0, 0);
draw byLine(B, C, byblue, 0, 0);
draw byLine(C, D, byblack, 0, 1);
draw byLine(D, A, byblack, 0, 1);
draw byCircleR(E, r, byblue, 0, 0, 0);
draw byLabelsOnCircle(A, B, C, D)(E);
}
\drawCurrentPictureInMargin
\startsubproposition[title={Случай III.}]
\startCenterAlign
Угол \drawAngle{D} в~сегменте меньше полукруга тупой.

Возьмем на противоположной стороне окружности любую точку, к~которой проведем \drawUnitLine{BC} и~\drawUnitLine{AB}.

Поскольку $\drawAngle{B} + \drawAngle{D} = \drawTwoRightAngles$ \inprop[prop:III.XXII].

Но $\drawAngle{B} < \drawRightAngle$ (случай II.),\\
$\therefore$ \drawAngle{D} тупой.
\stopCenterAlign
\stopsubproposition

\qed
\stopProposition

\startProposition[title={Предл. XXXII. Теорема},reference=prop:III.XXXII]
\defineNewPicture{
pair A, B, C, D, E, F, O;
numeric r;
r := 9/4u;
O := (0, 0);
A := (0, r) shifted O;
B := (0, -r) shifted O;
C := (dir(-20)*r) shifted O;
D := (dir(30)*r) shifted O;
E := (-r, -r) shifted O;
F := (r, -r) shifted O;
byAngleDefine(A, B, E, byblack, 1);
byAngleDefine(D, B, A, byblue, 0);
byAngleDefine(F, B, D, byyellow, 0);
byAngleDefine(B, A, D, byyellow, 0);
byAngleDefine(A, D, B, byblack, 0);
byAngleDefine(C, D, B, byblack, 1);
byAngleDefine(D, C, B, byred, 0);
draw byNamedAngleResized();
draw byLine(A, D, byblack, 0, 1);
draw byLine(D, C, byblack, 0, 1);
draw byLine(C, B, byblack, 0, 1);
draw byLine(D, B, byred, 0, 0);
draw byLineFull(E, F, byblue, 0, 0)(E, F, 0, 0, 1);
draw byLine(A, B, byblack, 0, 0);
draw byCircleR(O, r, byred, 0, 0, 0);
draw byLabelsOnCircle(A, B, C, D)(O);
draw byLabelsOnPolygon(E, F, noPoint)(0, 0);
}
\drawCurrentPictureInMargin
\problemNP[3]{Е}{сли}{прямая \drawUnitLine{EF} касается круга, и~из точки касания проведена прямая \drawUnitLine{DB}, секущая круг, угол \drawAngle{FBD} между этой прямой и~касательной равен углу \drawAngle{A} в~накрестлежащем сегменте круга.}

Если хорда проходит через центр, то очевидно, что углы равны, поскольку и~тот и~тот прямые. (\inpropL[prop:III.XVI], \inpropL[prop:III.XXXI])

Если же нет, проведем $\drawUnitLine{AB} \perp \drawUnitLine{EF}$ из точки касания, которая будет проходить через центр \inprop[prop:III.XIX]

\startCenterAlign
$\therefore \drawAngle{ADB} = \drawAngle{ABE} $ \inprop[prop:III.XXXI],
$\drawAngle{A} + \drawAngle{DBA} = \drawAngle{ABE} = \drawAngle{FBA}$ \inprop[prop:I.XXXII]\\

$\therefore \drawAngle{A} = \drawAngle{FBD}$ \inax[ax:I.III].

Теперь $\drawAngle{B} = \drawTwoRightAngles = \drawAngle{A} + \drawAngle{C}$ \inprop[prop:III.XXII].

$\therefore \drawAngle{DBE} = \drawAngle{C}$ \inax[ax:I.III], который и~будет углом в~накрестлежащем сегменте.
\stopCenterAlign

\qed
\stopProposition

\startProposition[title={Предл. XXXIII. Задача},reference=prop:III.XXXIII]
\defineNewPicture{
pair A, B, D, E, G, K;
numeric r;
r := 7/4u;
G := (0, 0);
A := (0, -r) shifted G;
B := (dir(10)*r) shifted G;
E := (0, r) shifted G;
D := (r, -r) shifted G;
K := (-r, -r) shifted G;
byAngleDefine(E, A, K, byblack, 1);
byAngleDefine(B, A, E, byred, 0);
byAngleDefine(D, A, B, byyellow, 0);
byAngleDefine(G, B, A, byred, 0);
draw byNamedAngleResized(EAK,BAE,DAB,GBA);
draw byLine(A, B, byblack, 0, 0);
draw byLine(G, B, byyellow, 0, 0);
draw byLine(A, G, byblue, 0, 0);
draw byLine(G, E, byblue, 1, 0);
draw byLineFull(K, D, byred, 0, 0)(K, D, 0, 0, 1);
draw byCircle.G(G, B, byblue, 0, 0, 0);
byAngleDefine.givenRight(E, A, K, byblack, 1);
byAngleDefine.givenObtuse(B, A, K, byblack, 1);
byAngleDefine.givenAcute(D, A, B, byblack, 1);
setAttribute("angle", "Standalone", "givenRight", 2);
setAttribute("angle", "Standalone", "givenObtuse", 2);
setAttribute("angle", "Standalone", "givenAcute", 2);
draw byLabelsOnCircle(A, B)(G);
draw byLabelsOnPolygon(A, G, E)(2, 0);
draw byLabelsOnPolygon(K, D, noPoint)(0, 0);
}
\drawCurrentPictureInMargin
\problemNP{Н}{а}{данной прямой \drawUnitLine{AB} описать сегмент круга, вмещающий угол, равный данному \drawAngle{givenRight}, \drawAngle{givenObtuse}, \drawAngle{givenAcute}.}

Если данный угол прямой, то рассечем отрезок и~опишем на нем полукруг, который, очевидно, будет содержать прямой угол \inprop[prop:III.XXXI].

\startCenterAlign
Если же данный угол острый или тупой, сделаем с~данной прямой на одном из ее концов $\drawAngle{DAB} = \drawAngle{givenAcute}$.

Проведем $\drawUnitLine{AG} \perp \drawUnitLine{KD}$\\
и сделаем $\drawAngle{B} = \drawAngle{BAE}$.

Опишем \drawCircle[middle][1/5]{G} с~\drawUnitLine{AG} или \drawUnitLine{GB} в~качестве радиуса, ведь они равны.

\drawUnitLine{KD} касается \circleG\ \inprop[prop:III.XVI].

$\therefore$ \drawUnitLine{AB} делит круг на два сегмента, вмещающие углы равные \drawAngle{EAK,BAE} и~\drawAngle{DAB}, соответственно равные \drawAngle{givenObtuse} и~\drawAngle{givenAcute} \inprop[prop:III.XXXII].
\stopCenterAlign

\qed
\stopProposition

\startProposition[title={Предл. XXXIV. Задача},reference=prop:III.XXXIV]
\defineNewPicture[1/4]{
pair A, B, F, E, O;
numeric r;
path cr;
r := 7/4u;
O := (0, 0);
cr := (fullcircle scaled 2r) shifted O;
A := point 1 of cr;
B := point -2 of cr;
E := (r, -r) shifted O;
F := (-r, -r) shifted O;
draw byFilledCircleSegment(O, r, -2, 1, byyellow);
byAngleDefine.B(F, B, A, byblue, 0);
byAngleDefine.givenAngle(F, B, A, byred, 0);
setAttribute("angle", "Standalone", "givenAngle", 2);
draw byNamedAngleResized(B);
draw byLine(A, B, byblack, 0, 0);
draw byLineFull(E,F, byred, 0, 0)(E, F, 0, 0, -1);
draw byCircleR(O, r, byblue, 0, 0, 0);
draw byLabelsOnCircle(A)(O);
draw byLabelsOnPolygon(E, B, F, noPoint)(0, -2);
}
\drawCurrentPictureInMargin
\problemNP{О}{т}{данного круга \drawCircle{O} отсечь сегмент, вмещающий данный прямолинейный угол \drawAngle{givenAngle}.}

\startCenterAlign
Проведем \drawUnitLine{EF} \inprop[prop:III.XVII], касательную к~кругу в~любой точке.

В точке касания сделаем $\drawAngle{B} = \drawAngle{givenAngle}$, данному углу.

Тогда \drawFromCurrentPicture[middle][segmentO]{
draw byNamedFilledCircleSegment(O);
draw byLabelsOnCircle(A, B)(O);
} содержит угол $=$ данному углу.

Поскольку \drawUnitLine{EF} касательная,\\
и \drawUnitLine{AB} сечет ее,\\
угол в~$\segmentO\ = \drawAngle{B}$ \inprop[prop:III.XXXII].

Но $\drawAngle{B} = \drawAngle{givenAngle}$ (\conststr)
\stopCenterAlign

\qed
\stopProposition

\startProposition[title={Предл. XXXV. Теорема},reference=prop:III.XXXV]
\defineNewPicture{
pair A, B, C, D, E;
numeric r;
r :=u;
E := (0, 0);
A := (dir(50)*r) shifted E;
B := (dir(100)*r) shifted E;
C := (dir(50 + 180)*r) shifted E;
D := (dir(100 + 180)*r) shifted E;
draw byLine(A, E, byblack, 0, 0);
draw byLine(E, C, byblack, 1, 0);
draw byLine(D, E, byblue, 0, 0);
draw byLine(E, B, byblue, 1, 0);
draw byCircleR(E, r, byyellow, 0, 0, 0);
draw byLabelsOnCircle(A, B, C, D)(E);
draw byLabelsOnPolygon(C, E, B)(2, 0);
}
\problemNP{Е}{сли}{в круге две хорды
$\left\{\vcenter{
\nointerlineskip\hbox{\drawSizedLine{AE,EC}}
\nointerlineskip\hbox{\drawSizedLine{DE,EB}}}\right\}$
секут друг друга, прямоугольник, заключенный между частями одной равен прямоугольнику, заключенному между частями другой.}
\drawCurrentPictureInMargin
\startsubproposition[title={Случай I.}]
Если данные прямые проходят через центр, они секут друг друга посередине, следовательно прямоугольники заключенные между их частями являются квадратами, и~значит, равны.
\stopsubproposition

\defineNewPicture{
pair A, B, C, D, E, H;
numeric r;
r := u;
E := (0, 0);
A := (dir(30)*r) shifted E;
B := (dir(170)*r) shifted E;
C := (dir(30 + 180)*r) shifted E;
D := (dir(-50)*r) shifted E;
H = whatever[A, C] = whatever[B, D];
draw byLine(D, E, byred, 0, 0);
draw byLine(E, B, byyellow, 0, 0);
draw byLine(B, H, byblue, 0, 0);
draw byLine(H, D, byblue, 1, 0);
draw byLine(A, E, byblack, 0, 0);
draw byLine(E, H, byred, 1, 0);
draw byLine(H, C, byblack, 1, 0);
draw byCircleR(E, r, byred, 0, 0, 0);
draw byLabelsOnCircle(B, D, C, A)(E);
draw byLabelsOnPolygon(D, H, C)(2, 0);
draw byLabelsOnPolygon(B, E, A)(2, 0);
}
\drawCurrentPictureInMargin
\startsubproposition[title={Случай II.}]
\startCenterAlign
Пусть \drawSizedLine{HC,EH,AE} проходит через центр, а~\drawSizedLine{BH,HD} нет, проведем \drawSizedLine{EB} и~\drawSizedLine{DE}.

Тогда $\drawSizedLine{BH} \times \drawSizedLine{HD} = \drawSizedLine{EB}^2 - \drawSizedLine{EH}^2$ \inprop[prop:II.VI],\\
или $\drawSizedLine{BH} \times \drawSizedLine{HD} = \drawSizedLine{HC,EH}^2 - \drawSizedLine{EH}^2$.

$\therefore \drawSizedLine{BH} \times \drawSizedLine{HD} = \drawSizedLine{EH,AE} \times \drawSizedLine{AE}$ \inprop[prop:II.V].
\stopCenterAlign
\stopsubproposition

\defineNewPicture{
pair A, B, C, D, E, F, K, L;
numeric r;
path cr;
r := u;
F := (0, 0);
cr := (fullcircle scaled 2r) shifted F;
A := (dir(00)*r) shifted F;
B := (dir(170)*r) shifted F;
C := (dir(-150)*r) shifted F;
D := (dir(-50)*r) shifted F;
E = whatever[A, C] = whatever[B, D];
K := cr intersectionpoint (F -- 10[E, F]);
L := cr intersectionpoint (F -- 10[F, E]);
draw byLine(K, E, byred, 1, 0);
draw byLine(E, L, byred, 0, 0);
draw byLine(B, E, byblue, 0, 0);
draw byLine(E, D, byblue, 1, 0);
draw byLine(A, E, byblack, 0, 0);
draw byLine(E, C, byblack, 1, 0);
draw byCircleR(F, r, byblue, 0, 0, 0);
draw byLabelsOnCircle(B, D, K, L, A, C)(F);
draw byLabelsOnPolygon(B, E, K)(2, 0);
}
\drawCurrentPictureInMargin
\startsubproposition[title={Случай III.}]
\startCenterAlign
Пусть ни одна из прямых не проходит через центр, проведем через точку их пересечения диаметр \drawSizedLine{KE,EL}\\
и $\drawSizedLine{KE} \times \drawSizedLine{EL} = \drawSizedLine{BE} \times \drawSizedLine{ED}$ (случай II.),\\
а также $\drawSizedLine{KE} \times \drawSizedLine{EL} = \drawSizedLine{AE} \times \drawSizedLine{EC}$ (случай II.).

$\therefore \drawSizedLine{BE} \times \drawSizedLine{ED} = \drawSizedLine{AE} \times \drawSizedLine{EC}$.
\stopCenterAlign
\stopsubproposition

\qed
\stopProposition

\startProposition[title={Предл. XXXVI. Теорема},reference=prop:III.XXXVI]
\defineNewPicture{
pair A, B, C, D, F;
numeric r;
path cr[];
r := 7/4u;
F := (0, 0);
cr1 := (fullcircle scaled 2r) shifted F;
D := (r, 3/2r) shifted F;
C := (dir(angle(D-F)))*r;
A := (dir(angle(D-F) + 180))*r;
cr2 := ((fullcircle scaled abs(D-F)) rotated angle(D-F)) shifted 1/2[D, F];
B := cr1 intersectionpoint (subpath (0, 4) of cr2);
draw byLine(B, F, byyellow, 0, 0);
draw byLine(C, F, byred, 1, 0);
draw byLine(F, A, byblack, 0, 0);
byLineDefine(D, B, byblue, 0, 0);
byLineDefine(D, C, byred, 0, 0);
draw byNamedLineSeq(0)(DB,DC);
draw byCircleR(F, r, byblack, 0, 0, 0);
draw byLabelsOnCircle(A, B)(F);
draw byLabelsOnPolygon(B, D, F, A)(2, 0);
draw byLabelPoint(C, angle(C-F) - 45, 2);
}
\problemNP{Е}{сли}{из точки вне круга к~кругу проведены две прямых, одна из которых \drawSizedLine{DB} касается круга, а~другая \drawSizedLine{FA,CF,DC} сечет его; прямоугольник, заключенный между всей секущей \drawSizedLine{FA,CF,DC} и~внешней частью \drawSizedLine{DC} равен квадрату касательной \drawSizedLine{DB}.}\unskip
\drawCurrentPictureInMargin
\startsubproposition[title={Случай I.}]
\startCenterAlign
Пусть \drawSizedLine{FA,CF,DC} проходит через центр.

Проведем \drawSizedLine{BF} из центра к~точке касания.

$\drawSizedLine{DB}^2 = \drawSizedLine{CF,DC}^2 - \drawSizedLine{BF}^2$ \inprop[prop:I.XLVII],\\
или $\drawSizedLine{DB}^2 = \drawSizedLine{CF,DC}^2 - \drawSizedLine{CF}^2$.

$\therefore \drawSizedLine{DB}^2 = \drawSizedLine{FA,CF,DC} \times \drawSizedLine{DC}$ \inprop[prop:II.VI].
\stopCenterAlign
\stopsubproposition

\defineNewPicture{
pair A, B, C, D, E, F;
numeric r;
path cr[];
r := 7/4u;
E := (0, 0);
cr1 := (fullcircle scaled 2r) shifted E;
D := (r, 3/2r) shifted E;
A := (dir(angle(D-E) + 120))*r;
C := cr1 intersectionpoint (A--D);
cr2 := ((fullcircle scaled abs(D-E)) rotated angle(D-E)) shifted 1/2[D, E];
B := cr1 intersectionpoint (subpath (0, 4) of cr2);
draw byLine(E, A, byyellow, 1, 0);
draw byLine(E, B, byyellow, 0, 0);
draw byLine(E, C, byblue, 1, 0);
draw byLine(D, C, byred, 0, 0);
draw byLine(C, A, byred, 1, 0);
draw byLine(E, B, byyellow, 0, 0);
byLineDefine(D, B, byblue, 0, 0);
byLineDefine(D, E, byblack, 0, 0);
draw byNamedLineSeq(0)(DB,DE);
draw byCircleR(E, r, byblack, 0, 0, 0);
draw byLabelsOnCircle(A, B)(E);
draw byLabelsOnPolygon(B, D, E, A)(2, 0);
draw byLabelPoint(C, angle(C-E) + 45, 2);
}\drawCurrentPictureInMargin
\startsubproposition[title={Случай II.}]
\startCenterAlign
Если \drawSizedLine{CA,DC} не проходит через центр,\\
проведем \drawSizedLine{EA} и~\drawSizedLine{EC}.

Тогда $\drawSizedLine{CA,DC} \times \drawSizedLine{CA} = \drawSizedLine{DE}^2 - \drawSizedLine{EC}^2$ \inprop[prop:II.VI].

То есть $\drawSizedLine{CA,DC} \times \drawSizedLine{CA} = \drawSizedLine{DE}^2 - \drawSizedLine{EB}^2$.

$\therefore \drawSizedLine{CA,DC} \times \drawSizedLine{CA} = \drawSizedLine{DB}^2$ \inprop[prop:III.XVIII].
\stopCenterAlign
\stopsubproposition

\qed
\stopProposition

\startProposition[title={Предл. XXXVII. Теорема},reference=prop:III.XXXVII]
\defineNewPicture{
pair A, B, C, D, E, F;
numeric r;
path cr[];
r := 2u;
F := (0, 0);
cr1 := (fullcircle scaled 2r) shifted F;
D := (4/3r, 4/3r) shifted F;
cr2 := ((fullcircle scaled abs(D-F)) rotated angle(D-F)) shifted 1/2[D, F];
B := cr1 intersectionpoint (subpath (0, 4) of cr2);
E := cr1 intersectionpoint (subpath (4, 8) of cr2);
A := (dir(angle(D-F) + 140))*r;
C := cr1 intersectionpoint (D--A);
byAngleDefine(D, B, F, byblue, 0);
byAngleDefine(F, E, D, byred, 0);
draw byNamedAngleResized();
draw byLine(D, C, byblack, 1, 0);
draw byLine(C, A, byblack, 0, 0);
draw byLine(D, F, byyellow, 0, 0);
byLineDefine(B, F, byred, 1, 0);
byLineDefine(E, F, byblue, 1, 0);
byLineDefine(D, B, byred, 0, 0);
byLineDefine(D, E, byblue, 0, 0);
draw byNamedLineSeq(0)(BF,DB,DE,EF);
draw byCircleR(F, r, byblack, 0, 0, 0);
draw byLabelsOnCircle(A, B, E)(F);
draw byLabelsOnPolygon(E, F, B)(2, 0);
draw byLabelsOnPolygon(B, D, E)(2, 0);
draw byLabelPoint(C, angle(C-F) + 45, 2);
}
\drawCurrentPictureInMargin
\problemNP{Е}{сли}{из точки вне круга проведены две прямых линии, одна из которых \drawSizedLine{CA,DC} сечет круг, а~другая \drawSizedLine{DB} лишь падает на него, и~если прямоугольник заключенный между  всей секущей линией \drawSizedLine{CA,DC} и~ее внешней частью \drawSizedLine{DC} равен квадрату линии падающей на круг, та \drawSizedLine{DB} является касательной к~кругу.}

\startCenterAlign
Проведем из данной точки касательную к~кругу \drawSizedLine{DE}.

И проведем из центра \drawSizedLine{DF}, \drawSizedLine{BF}, и~\drawSizedLine{EF}.

$\drawSizedLine{DE}^2 = \drawSizedLine{CA,DC} \times \drawSizedLine{DC}$ \inprop[prop:III.XXXVI],\\
но $\drawSizedLine{DB}^2 = \drawSizedLine{CA,DC} \times \drawSizedLine{DC}$ (\hypstr).

$\therefore \drawSizedLine{DB}^2 = \drawSizedLine{DE}^2$,\\
и $\therefore \drawSizedLine{DB} = \drawSizedLine{DE}$.

Тогда в~\drawLine{DB,DF,BF} и~\drawLine{EF,DF,DE}\\
$\drawSizedLine{BF} \mbox{ и~} \drawSizedLine{DB} = \drawSizedLine{EF} \mbox{ и~} \drawSizedLine{DE}$,\\
и \drawSizedLine{DF} общая обоим,\\
$\therefore \drawAngle{B} = \drawAngle{E}$ \inprop[prop:I.VIII],\\
но $\drawAngle{E} = \drawRightAngle$, прямому углу \inprop[prop:III.XVIII].

$\therefore \drawAngle{B} = \drawRightAngle$, прямому углу\\
и $\therefore$ \drawSizedLine{DB} касается круга \inprop[prop:III.XVI].
\stopCenterAlign

\qed
\stopProposition

\stopBook

\startBook[title={Книга IV}]

\startsupersection[title={Определения}]

\startDefinitionOnlyNumber[reference=def:IV.I]
\defineNewPicture{
pair A, B, C, D, E, F, G, H;
A := (0, 0);
B := (1/2u, u);
C := (2u, 3/2u);
D := (4/3u, -1/2u);
E := 1/2[A, B];
F := 1/2[B, C];
G := 1/2[C, D];
H := 1/2[D, A];
draw byPolygon(E,F,G,H)(byred);
byLineDefine(A, B, byblack, 0, 0);
byLineDefine(B, C, byblack, 0, 0);
byLineDefine(C, D, byblack, 0, 0);
byLineDefine(D, A, byblack, 0, 0);
draw byNamedLineSeq(-1)(AB,BC,CD,DA);
}\drawCurrentPictureInMargin[inside]
Говорят, что прямолинейная фигуга \emph{вписывается} в~другую прямолинейную фигуру, если каждый из углов вписываемой фигуры касается всех сторон той, в~которую она вписывается.
\stopDefinitionOnlyNumber

\startDefinitionOnlyNumber[reference=def:IV.II]
Про фигуру говорят, что она \emph{описывается около} фигуры, если каждая сторона описываемой фигуры касается угла той, около которой она описывается.
\stopDefinitionOnlyNumber

\defineNewPicture{
angleScale := 3/4;
pair A, B, C, D, O;
numeric r;
r := 4/5u;
A := (r, 0);
B := (0, r);
C := (-r, 0);
D := (0, -r);
O := (0, 0);
byAngleDefine(A, B, C, byred, 0);
byAngleDefine(B, C, D, byred, 0);
byAngleDefine(C, D, A, byred, 0);
byAngleDefine(D, A, B, byred, 0);
draw byNamedAngleResized();
byLineDefine(A, B, byred, 0, 0);
byLineDefine(B, C, byred, 0, 0);
byLineDefine(C, D, byred, 0, 0);
byLineDefine(D, A, byred, 0, 0);
draw byNamedLineSeq(1)(DA,CD,BC,AB);
draw byCircleR(O, r, byblack, 0, 0, 1);
}\drawCurrentPictureInMargin[inside]
\startDefinitionOnlyNumber[reference=def:IV.III]
Про прямолинейную фигуру говорят, что она \emph{вписывается} в~круг, когда каждый угол вписываемой фигуры касается окружности.
\stopDefinitionOnlyNumber

\defineNewPicture{
pair A, B, C, D, O;
numeric r;
r := 2/3u;
A := (r, r);
B := (-r, r);
C := (-r, -r);
D := (r, -r);
O := (0, 0);
draw byPolygon(A,B,C,D)(byred);
draw byArbitraryFigure.ABCDo(A--B--C--D--cycle, byred, 0, 0);
draw byFilledCircleSector(O, r, 0, 8, white);
}\drawCurrentPictureInMargin[inside]
\startDefinitionOnlyNumber[reference=def:IV.IV]
Про фигуру говорят, что она \emph{описывается} около круга, если каждая из ее сторон касается окружности.
\stopDefinitionOnlyNumber

\vfill\pagebreak

\startDefinitionOnlyNumber[reference=def:IV.V]
\defineNewPicture{
pair A, B, C, D, E, F, O;
numeric r[];
r1 := 3/4u;
A := dir(0)*r1;
B := dir(60)*r1;
C := dir(120)*r1;
D := dir(180)*r1;
E := dir(240)*r1;
F := dir(300)*r1;
r2 := abs(1/2[A, B]);
O := (0, 0);
draw byPolygon(A,B,C,D,E,F)(byblue);
draw byArbitraryFigure.ABCDEFo(A--B--C--D--E--F--cycle, byblue, 0, 0);
draw byFilledCircleSector(O, r2, 0, 8, white);
}\drawCurrentPictureInMargin[inside]
Про круг говорят, что он \emph{вписывается} в~фигуру, когда его окружность касается каждой из сторон фигуры.
\stopDefinitionOnlyNumber


\startDefinitionOnlyNumber[reference=def:IV.VI]
\defineNewPicture[1/2]{
pair A, B, C, D, E, F, G, H, K, I;
numeric r;
A := (0, 0);
B := (3u, 7/3u);
C := (7/2u, 0);
D = whatever[A, B] = whatever[C, C shifted dir(angle(C-A)) shifted dir(angle(C-B))];
E = whatever[B, C] = whatever[A, A shifted dir(angle(A-B)) shifted dir(angle(A-C))];
F = whatever[C, A] = whatever[B, B shifted dir(angle(B-C)) shifted dir(angle(B-A))];
I = whatever[C, D] = whatever[A, E];
G = whatever[A, B] = whatever[I, I shifted ((A-B) rotated 90)];
H = whatever[B, C] = whatever[I, I shifted ((B-C) rotated 90)];
K = whatever[C, A] = whatever[I, I shifted ((C-A) rotated 90)];
r := abs(D-I);
draw byPolygon(G,H,K)(byblue);
draw byCircleR(I, r, byblack, 0, 0, -1);
}\drawCurrentPictureInMargin[inside]
Про круг говорят, что он \emph{описывается} около прямолинейной фигуры, если его окружность проходит через все углы фигуры.

\stopDefinitionOnlyNumber

\startDefinitionOnlyNumber[reference=def:IV.VII]
\defineNewPicture{
pair A, B, O;
numeric r;
r := 3/4u;
O := (0, 0);
A := dir(0)*r;
B := dir(100)*r;
draw byLine(A, B, byblack, 0, 0);
draw byCircleR(O, r, byblue, 0, 0, 0);
}\drawCurrentPictureInMargin[inside]
Говорят что прямая \emph{вписывается} в~круг, когда ее концы находятся на окружности.
\stopDefinitionOnlyNumber

\vskip 1cm

Четвертая книга посвящена решению задач, преимущественно касающихся вписыванию и~описыванию правильных многоугольников.

Правильным называют такой многоугольник стороны и~углы которого равны.
\stopsupersection

\vfill\pagebreak

\startProposition[title={Предл. I. Задача},reference=prop:IV.I]
\defineNewPicture{
pair A, B, C, E, G, H, O;
numeric r[];
path cr[];
r1 := 4/3u;
r2 := 7/4u;
O := (0, 0);
B := (r1, 0) shifted O;
C := (-r1, 0) shifted O;
cr1 := (fullcircle scaled 2r1) shifted O;
cr2 := (fullcircle scaled 2r2) shifted C;
A := cr1 intersectionpoint (subpath (0, 2) of cr2);
E := (B--C) intersectionpoint cr2;
G := (xpart(lrcorner(cr1)), ypart(lrcorner(cr2)));
H := G shifted (-r2, 0);
draw byLine(B, E, byred, 0, 0);
draw byLine(E, C, byred, 1, 0);
draw byLine(A, C, byyellow, 0, 0);
draw byLine(G, H, byblue, 0, 0);
draw byCircleR(O, r1, byyellow, 0, 0, 0);
draw byCircle.C(C, E, byblue, 0, 0, 0);
draw byLabelsOnCircle(C, B)(O);
draw byLabelsOnPolygon(C, A, E)(2, 0);
draw byLabelsOnPolygon(C, E, A)(2, -1);
draw byLabelsOnPolygon(G, H, noPoint)(0, 0);
}
\drawCurrentPictureInMargin
\problemNP{В}{данный}{круг \drawCircle{O} вписать прямую, равную данной \drawUnitLine{GH}, не большей диаметра круга.}

\startCenterAlign
Проведем диаметр \drawUnitLine{EC,BE} круга \circleO;\\
и если $\drawUnitLine{EC,BE} = \drawUnitLine{GH}$, задача решена.

Если же \drawUnitLine{EC,BE} не равна \drawUnitLine{GH},\\
$\drawUnitLine{EC,BE} > \drawUnitLine{GH}$ (\hypstr).

Сделаем $\drawUnitLine{EC} = \drawUnitLine{GH}$ \inprop[prop:I.III]\\
взяв \drawUnitLine{EC} за радиус, построим \drawCircle{C}, секущий \circleO, \\
и проведем \drawUnitLine{AC}, которая и~будет искомой прямой.

Поскольку $\drawUnitLine{AC} = \drawUnitLine{EC} = \drawUnitLine{GH}$ (\indefL[prop:I.XV], \conststr).
\stopCenterAlign

\qed
\stopProposition

\startProposition[title={Предл. II. Задача},reference=prop:IV.II]
\defineNewPicture{
pair A, B, C, D, E, F, G, H, O, d;
numeric r;
path cr;
r := 7/4u;
O := (0, 0);
cr := (fullcircle scaled 2r) shifted O;
A := (0, -r) shifted O;
G := (-r, -r) shifted O;
H := (r, -r) shifted O;
D := (0, 0);
E := (-1/2r, 4/3r);
F := (3/4r, r);
d := -1/2[ulcorner(D--E--F), lrcorner(D--E--F)] shifted (0, -2r);
D := D shifted d;
E := E shifted d;
F := F shifted d;
C := (subpath (-1, 5) of cr) intersectionpoint (A -- A shifted (dir(angleValue(F, E, D))*2r));
B := (subpath (-1, 5) of cr) intersectionpoint (A -- A shifted (dir(180-angleValue(D, F, E))*2r));
byAngleDefine(C, B, A, byblack, 0);
byAngleDefine(A, C, B, byblack, 1);
byAngleDefine(G, A, B, byyellow, 0);
byAngleDefine(B, A, C, byred, 0);
byAngleDefine(C, A, H, byblue, 0);
byAngleDefine(F, E, D, byblue, 0);
byAngleDefine(E, D, F, byred, 0);
byAngleDefine(D, F, E, byyellow, 0);
draw byNamedAngleResized();
draw byArbitraryFigure.BAC(B--A--C, byblack, 0, 1);
draw byArbitraryFigure.BAC(D--E--F--cycle, byblack, 0, 1);
draw byLine(B, C, byyellow, 0, 0);
draw byLine(G, H, byred, 0, 0);
draw byCircleR(O, r, byblack, 0, 0, 0);
draw byLabelsOnCircle(A, B, C)(O);
draw byLabelsOnPolygon(D, E, F)(0, 0);
draw byLabelsOnPolygon(H, G, noPoint)(0, 0);
}
\drawCurrentPictureInMargin
\problemNP{В}{данный}{круг \drawCircle{O} вписать треугольник, равноугольный данному.}

\startCenterAlign
Проведем касательную \drawUnitLine{GH}, к~любой точке круга \inprop[prop:III.XVII].

В точке касания сделаем $\drawAngle{CAH} = \drawAngle{E}$ \inprop[prop:I.XXIII].

Так же сделаем $\drawAngle{GAB} = \drawAngle{F}$\\
и проведем \drawUnitLine{BC}.

Поскольку $\drawAngle{CAH} = \drawAngle{E}$ (\conststr)\\
и $\drawAngle{CAH} = \drawAngle{B}$ \inprop[prop:III.XXXII].

$\therefore \drawAngle{B} = \drawAngle{E}$.

Также и~$\drawAngle{C} = \drawAngle{F}$ по той же причине.

$\therefore \drawAngle{BAC} = \drawAngle{D}$ \inprop[prop:I.XXXII].
\stopCenterAlign

\noindent И, следовательно, вписанный в~круг треугольник равноуголен данному.

\qed
\stopProposition

\startProposition[title={Предл. III. Задача},reference=prop:IV.III]
\defineNewPicture[3/5]{
pair A, B, C, D, E, F, G, H, K, L, M, N, d;
numeric r;
r := 5/4u;
d := (-4/3r, 11/5r);
D := (0, 0) shifted d;
E := (1/3r, -r) shifted d;
F := (-5/6r, -r) shifted d;
G := 4/3[F, E];
H := 4/3[E, F];
K := (0, 0);
C := (0, -r) shifted K;
A := (dir(-90+angleValue(D, F, H))*r) shifted K;
B := (dir(-90-angleValue(G, E, D))*r) shifted K;
L = whatever[C, C shifted (dir(angle(C-K) + 90))] = whatever[A, A shifted (dir(angle(A-K) + 90))];
M = whatever[B, B shifted (dir(angle(B-K) + 90))] = whatever[A, L];
N = whatever[B, M] = whatever[C, L];
byAngleDefine(D, F, H, byyellow, 0);
byAngleDefine(C, K, A, byyellow, 0);
byAngleDefine(G, E, D, byblue, 0);
byAngleDefine(B, K, C, byblue, 0);
byAngleDefine(M, L, N, byred, 0);
byAngleDefine(D, F, E, byred, 0);
byAngleDefine(L, N, M, byblack, 1);
byAngleDefine(F, E, D, byblack, 1);
byAngleDefine(N, M, L, byred, 1);
byAngleDefine(E, D, F, byred, 1);
byAngleDefine(K, A, L, byblack, 0);
byAngleDefine(L, C, K, byblack, 0);
draw byNamedAngleResized();
draw byArbitraryFigure.FDE(F--D--E, byblack, 0, 1);
byLineDefine(N, L, byblue, 0, 0);
byLineDefine(L, M, byyellow, 0, 0);
byLineDefine(M, N, byred, 1, 0);
draw byLine(G, H, byblack, 0, 0);
draw byLine(K, C, byred, 0, 0);
draw byArbitraryFigure.AKB(A--K--B, byblack, 0, 1);
draw byCircleR(K, r, byred, 0, 0, -1);
draw byNamedLineSeq(0)(NL,LM,MN);
byArbitraryFigureDefine.LAKC(L--A--K--C--cycle, byblack, 0, 1);
draw byLabelsOnPolygon(L, A, M, B, N, C)(0, 0);
draw byLabelsOnPolygon(G, E, F, H, noPoint)(0, 0);
draw byLabelsOnPolygon(F, D, E)(2, 0);
draw byLabelsOnPolygon(A, K, B)(2, 0);
}
\drawCurrentPictureInMargin
\problemNP{О}{коло}{данного круга \drawCircle[middle][1/3]{K} описать треугольник, равноугольный данному.}

\startCenterAlign
Продлим одну из сторон \drawUnitLine{GH}, данного треугольника в~обе стороны.

Из центра данного круга проведем радиус \drawUnitLine{KC}.

Сделаем $\drawAngle{CKA} = \drawAngle{DFH}$ \inprop[prop:I.XXIII]\\
и $\drawAngle{BKC} = \drawAngle{GED}$.

Через концы радиусов проведем \drawUnitLine{NL}, \drawUnitLine{LM} и~\drawUnitLine{MN}, касательные к~кругу. \inprop[prop:III.XVII]

Четыре угла
\drawFromCurrentPicture[bottom]{
startTempAngleScale(2/3);
draw byNamedAngleResized(A, C, L, CKA);
draw byNamedArbitraryFigure(LAKC);
draw byLabelsOnPolygon(L, A, K, C)(0, 0);
stopTempAngleScale;
},
взятые вместе равны четырем прямым углам \inprop[prop:I.XXXII].

Но \drawAngle{C} и~\drawAngle{A} прямые (\conststr).

$\therefore \drawAngle{L} + \drawAngle{CKA} = \drawTwoRightAngles$, двум прямым углам.

Но $\drawAngle{DFH,DFE} = \drawTwoRightAngles$ \inprop[prop:I.XIII]\\
и $\drawAngle{CKA} = \drawAngle{DFH}$ (\conststr) и~$\therefore \drawAngle{L} = \drawAngle{DFE}$.

Таким же образом можно показать, что\\
$\drawAngle{N} = \drawAngle{FED}$.

$\therefore \drawAngle{M} = \drawAngle{D}$ \inprop[prop:I.XXXII]\\
и значит, описанный около круга треугольник равноуголен данному.
\stopCenterAlign

\qed
\stopProposition

\startProposition[title={Предл. IV. Задача},reference=prop:IV.IV]
\defineNewPicture{
pair A, B, C, D, E, F, G;
numeric r;
A := (0, 0);
B := (-3/2u, -3u);
C := (3u, -3u);
D = whatever[A, A shifted (unitvector(B-A) + unitvector(C-A))] = whatever[B, B shifted (unitvector(A-B) + unitvector(C-B))];
E = whatever[D, D shifted ((A-B) rotated 90)] = whatever[A, B];
F = whatever[D, D shifted ((B-C) rotated 90)] = whatever[B, C];
G = whatever[D, D shifted ((C-A) rotated 90)] = whatever[A, C];
r := abs(E-D);
byAngleDefine(B, E, D, byred, 0);
byAngleDefine(D, F, B, byred, 0);
byAngleDefine(E, B, D, byblue, 0);
byAngleDefine(D, B, F, byyellow, 0);
byAngleDefine(G, C, D, byblack, 0);
byAngleDefine(D, C, F, byblack, 1);
byAngleDefine(E, D, F, byblack, 1);
draw byNamedAngleResized();
draw byLine(E, D, byyellow, 1, 0);
draw byLine(G, D, byred, 1, 0);
draw byLine(F, D, byblack, 1, 0);
byLineDefine(C, D, byblue, 0, 0);
byLineDefine(B, D, byblue, 1, 0);
draw byNamedLineSeq(0)(CD,BD);
byLineDefine(A, B, byyellow, 0, 0);
byLineDefine(B, C, byblack, 0, 0);
byLineDefine(C, A, byred, 0, 0);
draw byNamedLineSeq(0)(AB,BC,CA);
draw byCircle.D(D, G, byblack, 0, 0, -1);
byLineDefine(B, F, byblack, 0, 0);
byLineDefine(B, E, byyellow, 0, 0);
draw byLabelsOnPolygon(B, E, A, G, C, F)(0, 0);
draw byLabelsOnPolygon(E, D, G)(2, 0);
}
\drawCurrentPictureInMargin
\problemNP{В}{данный}{треугольник \drawFromCurrentPicture[bottom]{
startTempScale(1/4);
startAutoLabeling;
draw byNamedLineSeq(0)(CA,BC,AB);
stopAutoLabeling;
stopTempScale;
} вписать круг.}

Рассечем пополам \drawAngle{EBD,DBF} и~\drawAngle{GCD,DCF} \inprop[prop:I.IX] с~помощью \drawUnitLine{BD} и~\drawUnitLine{CD}.

Из точки, где они встречаются, проведем \drawUnitLine{FD}, \drawUnitLine{ED} и~\drawUnitLine{GD} соотвтетственно перпендикулярные \drawUnitLine{BC}, \drawUnitLine{AB} и~\drawUnitLine{CA}.

\startCenterAlign
В
\drawFromCurrentPicture{
startTempScale(3/4);
draw byNamedAngle(F, DBF);
startAutoLabeling;
draw byNamedLineSeq(0)(BD,FD,BF);
stopAutoLabeling;
stopTempScale;
}
и
\drawFromCurrentPicture{
startTempScale(3/4);
draw byNamedAngle(E, EBD);
startAutoLabeling;
draw byNamedLineSeq(0)(BE,ED,BD);
stopAutoLabeling;
stopTempScale;
}\\
$\drawAngle{DBF} = \drawAngle{EBD}$, $\drawAngle{F} = \drawAngle{E}$ и~\drawUnitLine{BD} общая.

$\therefore \drawUnitLine{ED} = \drawUnitLine{FD}$ (\inpropL[prop:I.IV], \inpropL[prop:I.XXVI]).

Так же можно показать, что $\drawUnitLine{GD} = \drawUnitLine{FD}$.

$\therefore \drawUnitLine{FD} = \drawUnitLine{ED} = \drawUnitLine{GD}$.
\stopCenterAlign

\noindent Взяв любую из этих линий за радиус, опишем \drawCircle[middle][1/2]{D} и~его окружность будет проходить через концы других двух, а~стороны данного треугольника, перпендикулярные к~трем радиусам и~проходящие через их концы, касаются круга \inprop[prop:III.XVI], который, таким образом, вписан в~данный треугольник.

\qed
\stopProposition

\startProposition[title={Предл. V. Задача},reference=prop:IV.V]
\defineNewPicture[3/4]{
def proptmp (expr a, b, c, s) =
pair A, B, C, D, E, F;
numeric r;
r := 7/4u;
F := s;
A := (dir(a)*r) shifted F;
B := (dir(b)*r) shifted F;
C := (dir(c)*r) shifted F;
D := 1/2[A, B];
E := 1/2[A, C];
byAngleDefine(A, D, F, byred, 0);
byAngleDefine(F, D, B, byblack, 0);
draw byNamedAngleResized();
draw byLine(D, F, byyellow, 0, 0);
draw byLine(E, F, byyellow, 1, 0);
draw byLine(A, F, byblack, 1, 0);
draw byLine(A, D, byblue, 0, 0);
draw byLine(D, B, byblue, 1, 0);
if (sind(b)<>-sind(c)):
byLineDefine(B, F, byblack, 0, 0);
byLineDefine(C, F, byblack, 0, 1);
draw byNamedLineSeq(0)(BF,CF);
draw byLine(B, C, byred, 0, 0);
else:
draw byLine(B, C, byblack, 0, 0);
fi;
draw byLine(C, E, byred, 0, 0);
draw byLine(E, A, byred, 1, 0);
draw byCircleR(F, r, byblack, 0, 0, 0);
draw byLabelsOnCircle(A, B, C)(F);
draw byLabelsOnPolygon(B, D, A)(2, 0);
draw byLabelsOnPolygon(A, E, C)(2, 0);
draw byLabelsOnPolygon(C, F, B)(2, 0);
enddef;
proptmp(100, 170, 10, (0, -8u));
proptmp(80, 160, -20, (0, -4u));
proptmp(100, 200, -30, (0, 0));
}
\drawCurrentPictureInMargin
\problemNP{О}{коло}{данного треугольника описать круг.}

\startCenterAlign
Сделаем $\drawUnitLine{AD} = \drawUnitLine{DB}$\\
и~$\drawUnitLine{CE} = \drawUnitLine{EA}$ \inprop[prop:I.X].

Из точек рассечения проведем\\ 
\drawUnitLine{DF} и~\drawUnitLine{EF} $\perp$ \drawUnitLine{AD} и~\drawUnitLine{CE} \inprop[prop:I.XI],\\
из точки из пересечения проведем\\
\drawUnitLine{BF}, \drawUnitLine{AF} и~\drawUnitLine{CF}\\
и~опишем круг с~любой из них в~качестве радиуса, это и~будет искомый круг.

В
\drawFromCurrentPicture{
draw byNamedAngle(FDB);
startAutoLabeling;
draw byNamedLineSeq(0)(DF,BF,DB);
stopAutoLabeling;
}
и
\drawFromCurrentPicture{
draw byNamedAngle(ADF);
startAutoLabeling;
draw byNamedLineSeq(0)(AF,DF,AD);
stopAutoLabeling;
}\\
$\drawUnitLine{DB} = \drawUnitLine{AD}$ (\conststr),\\
\drawUnitLine{DF} общая,\\
$\drawAngle{FDB} = \drawAngle{ADF}$ (\conststr).

$\therefore \drawUnitLine{AF} = \drawUnitLine{BF}$ \inprop[prop:I.IV].

Так же можно показать и~что $\drawUnitLine{CF} = \drawUnitLine{AF}$.

$\therefore \drawUnitLine{AF} = \drawUnitLine{BF} = \drawUnitLine{CF}$,\\
и, следовательно, круг с~центром в~точке их пересечения и~с любой из них в~качестве радиуса опишет данный треугольник.
\stopCenterAlign

\qed
\stopProposition

\startProposition[title={Предл. VI. Задача},reference=prop:IV.VI]
\defineNewPicture[1/5]{
pair A, B, C, D, E;
numeric r;
r := 7/4u;
E := (0, 0);
A := (0, r);
B := (-r, 0);
C := (0, -r);
D := (r, 0);
byAngleDefine(D, A, B, byyellow, 0);
byAngleDefine(C, D, A, byblack, 0);
byAngleDefine(A, E, B, byblue, 0);
byAngleDefine(D, E, A, byred, 0);
draw byNamedAngleResized();
draw byLine(B, E, byred, 1, 0);
draw byLine(D, E, byblue, 1, 0);
draw byLine(A, E, byblack, 1, 0);
draw byLine(C, E, byblack, 0, 1);
byLineDefine(A, B, byblack, 0, 0);
byLineDefine(B, C, byyellow, 0, 0);
byLineDefine(C, D, byblue, 0, 0);
byLineDefine(D, A, byred, 0, 0);
draw byNamedLineSeq(0)(AB,BC,CD,DA);
draw byCircleR(E, r, byred, 0, 0, 1);
draw byLabelsOnCircle(A, B, C, D)(E);
draw byLabelsOnPolygon(C, E, B)(2, 0);
}
\drawCurrentPictureInMargin
\problemNP{В}{данный}{круг \drawCircle{E} вписать квадрат.}

\startCenterAlign
Проведем два диаметра круга $\perp$ друг другу,\\ 
проведем \drawUnitLine{BC}, \drawUnitLine{AB}, \drawUnitLine{DA} и~\drawUnitLine{CD},\\
тогда \drawLine[middle][squareABCD]{DA,CD,BC,AB} квадрат.

Поскольку \drawAngle{A} и~\drawAngle{D} каждый в~полукруге, оба являются прямыми \inprop[prop:III.XXXI].

$\therefore \drawUnitLine{CD} \parallel \drawUnitLine{AB}$ \inprop[prop:I.XXVIII].

так же и~$\drawUnitLine{DA} \parallel \drawUnitLine{BC}$.

И, поскольку $\drawAngle{AEB} = \drawAngle{DEA}$ (\conststr),\\
и $\drawUnitLine{BE} = \drawUnitLine{AE} = \drawUnitLine{DE}$ \indef[def:I.XV],
$\therefore \drawUnitLine{AB} = \drawUnitLine{DA}$ \inprop[prop:I.IV].

Поскольку смежные углы и~стороны параллелограмма \squareABCD\ равны, все его стороны и~углы равны \inprop[prop:I.XXXIV].

И~$\therefore$ \squareABCD, вписанный в~данный круг является квадратом.
\stopCenterAlign

\qed
\stopProposition

\startProposition[title={Предл. VII. Задача},reference=prop:IV.VII]
\defineNewPicture[1/7]{
pair A, B, C, D, E, F, G, H, K;
numeric r;
r := 7/4u;
E := (0, 0);
A := (0, r);
B := (-r, 0);
C := (0, -r);
D := (r, 0);
F := (r, r);
G := (-r, r);
H := (-r, -r);
K := (r, -r);
byAngleDefine(F, G, H, byred, 0);
byAngleDefine(G, H, K, byred, 0);
byAngleDefine(H, K, F, byred, 0);
byAngleDefine(K, F, G, byred, 0);
byAngleDefine(B, E, C, byblack, 0);
byAngleDefine(E, C, H, byyellow, 0);
draw byNamedAngleResized();
draw byLine(A, C, byred, 1, 0);
draw byLine(B, D, byblue, 1, 0);
byLineDefine(F, G, byblack, 0, 0);
byLineDefine(G, H, byred, 0, 0);
byLineDefine(H, K, byblue, 0, 0);
byLineDefine(K, F, byyellow, 0, 0);
draw byNamedLineSeq(0)(KF,HK,GH,FG);
draw byCircleR(E, r, byblue, 0, 0, -1);
draw byLabelsOnPolygon(H, B, G, A, F, D, K, C)(0, 0);
draw byLabelsOnPolygon(A, E, D)(2, 0);
}
\drawCurrentPictureInMargin
\problemNP{О}{коло}{данного круга \drawCircle{E} описать квадрат.}

\startCenterAlign
Проведем два перпендикулярных друг другу диаметра данного круга \drawUnitLine{BD}, \drawUnitLine{AC}, и~через их концы проведем \drawUnitLine{HK}, \drawUnitLine{GH}, \drawUnitLine{FG} и~\drawUnitLine{KF} касательные к~кругу,\\
тогда \drawLine[middle][squareFGHK]{KF,HK,GH,FG} будет квадратом.

$\drawAngle{C} = \drawRightAngle$ прямому углу, \inprop[prop:III.XVIII]\\
так же и~$\drawAngle{E} = \drawRightAngle$ (\conststr),\\
$\therefore \drawUnitLine{HK} \parallel \drawUnitLine{BD}$.

Таким же образом можно показать что $\drawUnitLine{FG} \parallel \drawUnitLine{BD}$,\\
а~также что $\drawUnitLine{GH} \mbox{ и~} \drawUnitLine{KF} \parallel \drawUnitLine{AC}$.

$\therefore$ \squareFGHK\ параллелограмм,\\
и поскольку $\drawAngle{G} = \drawAngle{F} = \drawAngle{K} = \drawAngle{H} = \drawAngle{C}$, все они прямые \inprop[prop:I.XXXIV].

Также очевидно, что \drawUnitLine{HK}, \drawUnitLine{GH}, \drawUnitLine{FG} и~\drawUnitLine{KF} равны между собой.

$\therefore$ \squareFGHK\ квадрат.
\stopCenterAlign

\qed
\stopProposition

\startProposition[title={Предл. VIII. Задача},reference=prop:IV.VIII]
\defineNewPicture{
pair A, B, C, D, E, F, G, H, K;
numeric r;
r := 7/4u;
G := (0, 0);
A := (-r, r);
B := (-r,-r);
C := (r, -r);
D := (r, r);
E := (0, r);
F := (-r, 0);
H := (0, -r);
K := (r, 0);
draw byPolygon(E,D,K,G)(byblack);
draw byPolygon(F,G,H,B)(byred);
draw byPolygon(G,K,C,H)(byblue);
draw byLine(F, G, byyellow, 0, 0);
draw byLine(G, K, byyellow, 1, 0);
draw byLine(G, H, byblack, 0, 0);
draw byLine(E, G, byblack, 1, 0);
byLineDefine(A, B, byblack, 0, 0);
byLineDefine(A, E, byblue, 1, 0);
byLineDefine(E, D, byblue, 0, 0);
byLineDefine(D, K, byred, 1, 0);
byLineDefine(K, C, byred, 0, 0);
draw byNamedLineSeq(0)(AB,AE,ED,DK,KC);
draw byCircleR(G, r, byyellow, 0, 0, -1);
draw byLabelsOnPolygon(B, F, A, E, D, K, C, H)(0, 0);
draw byLabelsOnPolygon(F, G, E)(2, 0);
}
\drawCurrentPictureInMargin
\problemNP{В}{данный}{квадрат вписать круг.}

\startCenterAlign
Сделаем $\drawUnitLine{ED} = \drawUnitLine{AE}$,\\
и $\drawUnitLine{KC} = \drawUnitLine{DK}$.

Проведем $\drawUnitLine{FG,GK} \parallel \drawUnitLine{AE,ED}$,\\
и $\drawUnitLine{EG,GH} \parallel \drawUnitLine{DK,KC}$ \inprop[prop:I.XXXI].

$\therefore$ \drawPolygon{EDKG} параллелограмм.

И поскольку $\drawUnitLine{AE,ED} = \drawUnitLine{DK,KC}$ (\hypstr),\\
$\drawUnitLine{ED} = \drawUnitLine{DK}$.

$\therefore$ \polygonEDKG\ равносторонний \inprop[prop:I.XXXIV].

Таким же образом можно показать, что $ \drawPolygon{GKCH} = \drawPolygon{FGHB}$ равносторонние параллелограммы.

$\therefore \drawUnitLine{AE} = \drawUnitLine{GK} = \drawUnitLine{GH} = \drawUnitLine{FG}$.
\stopCenterAlign

И значит если построить круг с~центром в~точке схождения этих линий с~любой их них в~качестве радиуса, он будет вписан в~квадрат \inprop[prop:III.XVI].

\qed
\stopProposition

\startProposition[title={Предл. IX. Задача},reference=prop:IV.IX]
\defineNewPicture{
pair A, B, C, D, E;
numeric r;
r := 7/4u;
E := (0, 0);
A := (dir(45+90)*r) shifted E;
B := (dir(45+180)*r) shifted E;
C := (dir(45+270)*r) shifted E;
D := (dir(45+360)*r) shifted E;
draw byPolygon(A,C,D)(byred);
draw byPolygon(A,B,C)(byyellow);
byAngleDefine(D, A, E, byyellow, 0);
byAngleDefine(E, A, B, byblack, 0);
byAngleDefine(A, B, E, byred, 0);
byAngleDefine(E, B, C, byblue, 0);
draw byNamedAngleResized();
draw byLine(A, E, byblue, 0, 0);
draw byLine(E, C, byblue, 1, 0);
draw byLine(B, E, byblack, 0, 0);
draw byLine(E, D, byblack, 1, 0);
draw byCircleR(E, r, byblack, 0, 0, 0);
draw byLabelsOnCircle(A, B, C, D)(E);
draw byLabelsOnPolygon(B, E, A)(2, -1);
}
\drawCurrentPictureInMargin
\problemNP{О}{коло}{данного квадрата \drawPolygon{ACD,ABC} описать круг.}

\startCenterAlign
Проведем пересекающие друг друга диагонали \drawUnitLine{AE,EC} и~\drawUnitLine{BE,ED}.

Тогда, $\because$ стороны у~\drawPolygon{ACD} и~\drawPolygon{ABC} равны,\\ а~основание \drawUnitLine{AE,EC} общее,\\
$\drawAngle{DAE} = \drawAngle{EAB}$ \inprop[prop:I.VIII],\\
то есть \drawAngle{DAE,EAB} рассечена пополам.

Так же можно показать и~что \drawAngle{ABE,EBC} рассечена пополам.

Но $\drawAngle{DAE,EAB} = \drawAngle{ABE,EBC}$.

Значит и~их половины $\drawAngle{EAB} = \drawAngle{ABE}$.

$\therefore \drawUnitLine{BE} = \drawUnitLine{AE}$ \inprop[prop:I.VI].

Так же можно показать,\\
что $\drawUnitLine{AE} = \drawUnitLine{BE} = \drawUnitLine{EC} = \drawUnitLine{ED}$.
\stopCenterAlign

И если в~точке схождения этих линий с~любой из них в~качестве радиуса построить круг, он опишет данный квадрат.

\qed
\stopProposition

\startProposition[title={Предл. X. Задача},reference=prop:IV.X]
\defineNewPicture[1/2]{
pair A, B, C, D, E, F;
numeric r[];
path cr[];
r1 := 11/4u;
A := (0, 0);
cr1 := (fullcircle scaled 2r1) shifted A;
B := (r1, 0);
r2 := r1*sqrt(5)/2 - 1/2r1;
cr2 := (fullcircle scaled 2r2) shifted B;
C := (r2, 0);
D := cr1 intersectionpoint (subpath (0, 4) of cr2);
F = whatever[1/2[A, C], 1/2[A, C] shifted ((A-C) rotated 90)]
 = whatever[1/2[A, D], 1/2[A, D] shifted ((A-D) rotated 90)];
r3 := abs(F-A);
byAngleDefine(D, A, B, byblack, 1);
byAngleDefine(C, D, A, byblack, 0);
byAngleDefine(B, D, C, byyellow, 0);
byAngleDefine(D, C, B, byblue, 0);
byAngleDefine(C, B, D, byred, 0);
draw byNamedAngleResized();
draw byLine(A, D, byyellow, 0, 0);
draw byLine(B, D, byblue, 0, 0);
draw byLine(C, D, byred, 0, 0);
draw byLine(A, C, byblack, 0, 0);
draw byLine(C, B, byblack, 1, 0);
draw byCircleABC.F(A, C, D, byblue, 0, 0, 0);
draw byCircle.A(A, B, byred, 0, 0, 0);
draw byLabelsOnCircle(B)(A);
draw byLabelsOnCircle(A)(F);
draw byLabelPoint(D, angle(D-A)-15, 2);
draw byLabelsOnPolygon(A, C, D)(2, 0);
}
\drawCurrentPicture
\problemNP{П}{остроить}{равнобедренный треугольник, в~котором угол при основании вдвое больше угла в~вершине.}

\startCenterAlign
Возьмем любую прямую \drawProportionalLine{AC,CB} \\
и разделим ее так, что $\drawProportionalLine{AC,CB} \times \drawProportionalLine{CB} = \drawProportionalLine{AC}^2$ \inprop[prop:II.XI].

С \drawProportionalLine{AC,CB} в~качестве радиуса опишем \drawCircle[middle][1/5]{A} и~впишем в~него от конца радиуса $\drawProportionalLine{BD} = \drawProportionalLine{AC}$ \inprop[prop:IV.I].

Проведем \drawProportionalLine{AD}.

Тогда \drawLine{AD,BD,CB,AC} и~есть искомый треугольник.

Действительно, проведем \drawProportionalLine{CD} и~опишем \drawCircle[middle][1/3]{F} около \drawLine{AD,CD,AC} \inprop[prop:IV.V].

Поскольку $\drawProportionalLine{AC,CB} \times \drawProportionalLine{CB} = \drawProportionalLine{AC}^2 = \drawProportionalLine{BD}^2$,\\
$\therefore$ \drawProportionalLine{BD} касается \circleF\ \inprop[prop:III.XXXVII]\\
$\therefore \drawAngle{BDC} = \drawAngle{A}$ \inprop[prop:III.XXXII].

Добавим \drawAngle{CDA} к~каждому,\\
$\therefore \drawAngle{BDC} + \drawAngle{CDA} = \drawAngle{A} + \drawAngle{CDA}$.

Но $\drawAngle{BDC} + \drawAngle{CDA} \mbox{ или } \drawAngle{BDC,CDA} = \drawAngle{B}$ \inprop[prop:I.V].

Поскольку $\drawProportionalLine{AD} = \drawProportionalLine{AC,CB}$ \inprop[prop:I.V].

Следовательно $\drawAngle{B} = \drawAngle{A} + \drawAngle{CDA} = \drawAngle{C}$ \inprop[prop:I.XXXII].

$\therefore \drawProportionalLine{CD} = \drawProportionalLine{BD}$ \inprop[prop:I.VI].

$\therefore \drawProportionalLine{BD} = \drawProportionalLine{AC} = \drawProportionalLine{CD}$ (\conststr).

$\therefore \drawAngle{A} = \drawAngle{CDA}$ \inprop[prop:I.V].

$\therefore \drawAngle{B} = \drawAngle{BDC,CDA} = \drawAngle{C} = \drawAngle{A} + \drawAngle{CDA} = \mbox{ дважды } \drawAngle{A}$.

И, следовательно, каждый из углов при основании вдвое больше угла в~вершине.
\stopCenterAlign

\qed
\stopProposition

\startProposition[title={Предл. XI. Задача},reference=prop:IV.XI]
\defineNewPicture[1/5]{
pair A, B, C, D, E, O;
numeric r;
r := 9/4u;
O := (0, 0);
A := (dir(90+0/5(360))*r) shifted O;
B := (dir(90+1/5(360))*r) shifted O;
C := (dir(90+2/5(360))*r) shifted O;
D := (dir(90+3/5(360))*r) shifted O;
E := (dir(90+4/5(360))*r) shifted O;
draw byArbitraryFigure.pg(C--E--B--D, byblack, 0, 1);
draw byPolygon(A,C,D)(byyellow);
byAngleDefineExtended(A, B, C, byblack, -1)(bytransparent);
byAngleDefineExtended(D, E, A, byblack, -1)(bytransparent);
byAngleDefine.Cl(B, C, A, byblack, -1);
byAngleDefine.Dr(A, D, E, byblack, -1);
byAngleDefine.Al(B, A, C, byblack, -1);
byAngleDefine.Ar(D, A, E, byblack, -1);
byAngleDefine(C, A, D, byblack, 0);
byAngleDefineExtended(E, C, A, byblack, 1)(byyellow);
byAngleDefine(D, C, E, byblue, 0);
byAngleDefine(A, D, B, byblack, 1);
byAngleDefine(B, D, C, byred, 0);
draw byNamedAngleResized();
draw byNamedAngleDummySides(ADB,BDC);
byLineDefine(A, B, byblue, 0, 0);
byLineDefine(B, C, byred, 0, 0);
byLineDefine(C, D, byblack, 0, 0);
byLineDefine(D, E, byred, 1, 0);
byLineDefine(E, A, byyellow, 0, 0);
draw byNamedLineSeq(0)(AB,BC,CD,DE,EA);
draw byCircleR(O, r, byblue, 0, 0, 1);
draw byLabelsOnCircle(A, B, C, D, E)(O);
}
\drawCurrentPictureInMargin
\problemNP[2]{В}{данный}{круг \drawCircle[middle][1/6]{O} вписать равносторонний и~равноугольный пятиугольник.}

\startCenterAlign
Построим равнобедренный треугольник, в~котором каждый из углов при основании вдвое больше угла в~вершине \inprop[prop:IV.X].

Впишем равноугольный ему\\
треугольник \drawPolygon{ACD} в~данный круг \drawCircle[middle][1/6]{O} \inprop[prop:IV.II].

Рассечем \drawAngle{ECA,DCE} и~\drawAngle{ADB,BDC} пополам \inprop[prop:I.IX].

Проведем \drawUnitLine{BC}, \drawUnitLine{AB}, \drawUnitLine{EA} и~\drawUnitLine{DE}.

Поскольку углы \drawAngle{ECA}, \drawAngle{DCE}, \drawAngle{CAD}, \drawAngle{BDC} и~\drawAngle{ADB}\\
равны между собой, дуги, на которых они стоят, тоже равны \inprop[prop:III.XXVI].

И~$\therefore$ \drawUnitLine{CD}, \drawUnitLine{BC}, \drawUnitLine{AB}, \drawUnitLine{EA} и~\drawUnitLine{DE} стягивающие эти дуги также равны между собой \inprop[prop:III.XXIX].

И~$\therefore$ пятиугольник является равносторонним,\\
а~также равноугольным, поскольку все его углы стоят на равных дугах \inprop[prop:III.XXVII].
\stopCenterAlign

\qed
\stopProposition

\startProposition[title={Предл. XII. Задача},reference=prop:IV.XII]
\defineNewPicture{
pair B, C, D, G, H, K, L, M, F;
numeric r[];
r1 := 5/2u;
F := (0, 0);
G := (dir(90+0/5(360))*r1) shifted F;
H := (dir(90+1/5(360))*r1) shifted F;
K := (dir(90+2/5(360))*r1) shifted F;
L := (dir(90+3/5(360))*r1) shifted F;
M := (dir(90+4/5(360))*r1) shifted F;
B := 1/2[H, K];
C := 1/2[K, L];
D := 1/2[L, M];
r2 := abs(F-B);
byAngleDefine(B, F, K, byred, 0);
byAngleDefine(K, F, C, byyellow, 0);
byAngleDefine(C, F, L, byblue, 0);
byAngleDefine(L, F, D, byred, 0);
byAngleDefine(F, K, B, byyellow, 1);
byAngleDefine(C, K, F, byblack, 0);
byAngleDefine(F, C, K, byblue, 1);
byAngleDefine(L, C, F, byblue, 1);
byAngleDefine(F, L, C, byblack, 0);
byAngleDefine(D, L, F, byblack, 1);
draw byNamedAngleResized();
draw byLine(F, K, byblue, 0, 0);
draw byLine(F, C, byblack, 1, 0);
draw byLine(F, L, byblack, 0, 0);
byLineDefine(F, B, byred, 1, 0);
byLineDefine(F, D, byyellow, 1, 0);
draw byNamedLineSeq(0)(FB,FD);
byLineDefine(G, H, byblack, 0, 0);
byLineDefine(H, B, byblue, 1, 0);
byLineDefine(B, K, byblack, 0, 0);
byLineDefine(K, C, byred, 0, 0);
byLineDefine(C, L, byyellow, 0, 0);
byLineDefine(L, M, byblack, 0, 0);
byLineDefine(M, G, byblack, 0, 0);
draw byNamedLineSeq(0)(GH,HB,BK,KC,CL,LM,MG);
draw byCircleR(F, r2, byred, 0, 0, -1);
draw byLabelsOnPolygon(M, D, L, C, K, B, H, G)(2, 0);
draw byLabelsOnPolygon(B, F, D)(2, 0);
}
\drawCurrentPictureInMargin
\problemNP{О}{коло}{данного круга \drawCircle[middle][1/4]{F} описать равносторонний и~равноугольный пятиугольник.}

\startCenterAlign
Проведем пять касательных через вершины любого правильного пятиугольника вписанного в~круг \circleF\ \inprop[prop:III.XVII].

Эти пять касательных и~будут образовывать искомый пятиугольник.

Проведем
$\left\{\vcenter{
\nointerlineskip\hbox{\drawProportionalLine{FB}}
\nointerlineskip\hbox{\drawProportionalLine{FK}}
\nointerlineskip\hbox{\drawProportionalLine{FC}}
\nointerlineskip\hbox{\drawProportionalLine{FL}}
\nointerlineskip\hbox{\drawProportionalLine{FD}}
}\right.$.\\
В \drawLine{FB,FK,BK} и~\drawLine{FK,FC,KC}\\
$\drawProportionalLine{BK} = \drawProportionalLine{KC}$ \inprop[prop:I.XLVII] \\
$\drawProportionalLine{FC} = \drawProportionalLine{FB}$, и~\drawProportionalLine{FK} общая.

$\therefore \drawAngle{FKB} = \drawAngle{CKF}$ и~$\therefore \drawAngle{BFK} = \drawAngle{KFC}$ \inprop[prop:I.VIII]

$\therefore \drawAngle{FKB,CKF} = \mbox{ дважды } \drawAngle{CKF}$, и~$\drawAngle{BFK,KFC} = \mbox{ дважды } \drawAngle{KFC}$.

Таким же образом можно показать, что\\
$\drawAngle{FLC,DLF} = \mbox{ дважды } \drawAngle{FLC}$, и~$\drawAngle{CFL,LFD} = \mbox{ дважды } \drawAngle{CFL}$.

Но $\drawAngle{BFK,KFC} = \drawAngle{CFL,LFD}$ \inprop[prop:III.XXVII],\\
$\therefore$ и~их половины $\drawAngle{KFC} = \drawAngle{CFL}$, а~также $\drawAngle{FCK} = \drawAngle{LCF}$,\\
и \drawProportionalLine{FC} общая.

$\therefore \drawAngle{CKF} = \drawAngle{FLC}$ и~$\drawProportionalLine{KC} = \drawProportionalLine{CL}$,\\
$\therefore \drawProportionalLine{KC,CL} = \mbox{ дважды } \drawProportionalLine{KC}$.

Таким же образом можно показать, что $\drawProportionalLine{BK,HB} = \mbox{ дважды } \drawProportionalLine{BK}$,\\
но $\drawProportionalLine{BK} = \drawProportionalLine{KC}$.

$\therefore \drawProportionalLine{BK,HB} = \drawProportionalLine{KC,CL}$.
\stopCenterAlign

Так же можно показать, что остальные стороны равны и, следовательно, пятиугольник равносторонний, так же он и~равноугольный, поскольку

\startCenterAlign
$\drawAngle{FLC,DLF} = \mbox{ дважды } \drawAngle{FLC}$\\
и $\drawAngle{FKB,CKF} = \mbox{ дважды } \drawAngle{CKF}$,\\
и сделовательно $ \drawAngle{CKF} = \drawAngle{FLC}$,\\
$\therefore \drawAngle{FLC,DLF} = \drawAngle{FKB,CKF}$.
\stopCenterAlign

Так же можно показать, что и~другие углы описанного пятиугольника равны.

\qed
\stopProposition

\startProposition[title={Предл. XIII. Задача},reference=prop:IV.XIII]
\defineNewPicture{
pair A, B, C, D, E, F, G, H, K, L, M;
numeric r[];
r1 := 5/2u;
F := (0, 0);
A := (dir(90+0/5(360))*r1) shifted F;
B := (dir(90+1/5(360))*r1) shifted F;
C := (dir(90+2/5(360))*r1) shifted F;
D := (dir(90+3/5(360))*r1) shifted F;
E := (dir(90+4/5(360))*r1) shifted F;
G := 1/2[A, B];
H := 1/2[B, C];
K := 1/2[C, D];
L := 1/2[D, E];
M := 1/2[E, A];
r2 := abs(F-G);
byAngleDefine(G, B, F, byyellow, 0);
byAngleDefine(F, B, H, byyellow, 0);
byAngleDefine(F, H, C, byblack, 0);
byAngleDefine(H, C, F, byblue, 0);
byAngleDefine(F, C, K, byblue, 0);
byAngleDefine(C, K, F, byblack, 0);
byAngleDefine(K, D, F, byred, 0);
byAngleDefine(F, D, E, byred, 0);
byAngleDefine(D, E, F, byblack, 1);
byAngleDefine(F, E, M, byblack, 1);
draw byNamedAngleResized();
draw byLine(F, G, byblack, 0, 1);
draw byLine(F, M, byblack, 0, 1);
draw byLine(F, A, byblack, 0, 1);
draw byLine(F, H, byblue, 1, 0);
draw byLine(F, K, byblue, 0, 0);
draw byLine(F, C, byblack, 0, 0);
draw byLine(F, D, byyellow, 1, 0);
byLineDefine(F, B, byred, 0, 0);
byLineDefine(F, E, byred, 1, 0);
draw byNamedLineSeq(0)(FB,FE);
byLineDefine(A, B, byblack, 0, 1);
byLineDefine(B, H, byblack, 1, 0);
byLineDefine(H, C, byyellow, 0, 0);
byLineDefine(C, K, byyellow, 0, 0);
byLineDefine(K, D, byblack, 1, 0);
byLineDefine(D, E, byblack, 0, 1);
byLineDefine(E, A, byblack, 0, 1);
draw byNamedLineSeq(0)(AB,BH,HC,CK,KD,DE,EA);
draw byCircle.F(F, H, byyellow, 0, 0, -1);
draw byLabelsOnPolygon(C, H, B, A, E, D, K)(0, 0);
draw byLabelsOnPolygon(E, F, D)(2, -2);
}
\drawCurrentPictureInMargin
\problemNP{В}{данный}{равноугольный и~равносторонний пятиугольник вписать круг.}

Пусть
\drawLine[bottom]{EA,DE,KD,CK,HC,BH,AB}
будет данным равноугольным и~равносторонним пятиугольником, в~который требуется вписать круг.

\startCenterAlign
Сделаем $\drawAngle{HCF} = \drawAngle{FCK}$, и~$\drawAngle{FDE} = \drawAngle{KDF}$ \inprop[prop:I.IX]

Проведем \drawUnitLine{FD}, \drawUnitLine{FC}, \drawUnitLine{FB}, \drawUnitLine{FE}, и~т. д.

Поскольку в
\drawFromCurrentPicture{
startTempScale(1/2);
draw byNamedAngle(FBH,HCF);
startAutoLabeling;
draw byNamedLineSeq(0)(FB,FC,HC,BH);
stopAutoLabeling;
stopTempScale;
}
и
\drawFromCurrentPicture{
startTempScale(1/2);
draw byNamedAngle(FCK,KDF);
startAutoLabeling;
draw byNamedLineSeq(0)(FC,FD,KD,CK);
stopAutoLabeling;
stopTempScale;
}\\
$\drawUnitLine{KD,CK} = \drawUnitLine{HC,BH}$, $\drawAngle{HCF} = \drawAngle{FCK}$, и~\drawUnitLine{FC} общая обоим,\\
$\therefore \drawUnitLine{FB} = \drawUnitLine{FD}$ и~$\drawAngle{FBH} = \drawAngle{KDF}$ \inprop[prop:I.IV].

И поскольку $\drawAngle{B} = \drawAngle{D} = \mbox{ дважды } \drawAngle{KDF}$,\\
$\therefore \mbox{ дважды } \drawAngle{FBH}$, а~значит \drawAngle{B} рассечен \drawUnitLine{FB} пополам.

Так же можно показать, что \drawAngle{DEF,FEM} рассекается пополам \drawUnitLine{FE}, и~оставшийся угол многоугольника рассекается таким же образом.

Проведем \drawUnitLine{FK}, \drawUnitLine{FH}, и~т. д. перпендикулярные к~сторонам пятиугольника.

Тогда в~треугольниках
\drawFromCurrentPicture{
startTempScale(2/3);
draw byNamedAngle(H,HCF);
startAutoLabeling;
draw byNamedLineSeq(0)(FH,FC,HC);
stopAutoLabeling;
stopTempScale;
}
и
\drawFromCurrentPicture{
startTempScale(2/3);
draw byNamedAngle(FCK,K);
startAutoLabeling;
draw byNamedLineSeq(0)(CK,FC,FK);
stopAutoLabeling;
stopTempScale;
}\\
получим $\drawAngle{HCF} = \drawAngle{FCK}$ (\conststr), \drawUnitLine{FC} общая,\\
и $\drawAngle{H} = \drawAngle{K} = \drawRightAngle \mbox{ прямому углу }$.

$\therefore \drawUnitLine{FK} = \drawUnitLine{FH}$ \inprop[prop:I.XXVI]
\stopCenterAlign

Так же можно показать, что пять перпендикуляров к~сторонам пятиугольника равны между собой.

Опишем \drawCircle[middle][1/5]{F} с~одним из перпендикуляров в~качестве радиуса, это и~будет искомый вписанный круг. Поскольку, если он не касается сторон пятиугольника, но сечет их, то прямая, проведенная через конец диаметра под прямым углом будет проходить внутри круга, что, как было показано \inprop[prop:III.XVI], невозможно.

\qed
\stopProposition

\startProposition[title={Предл. XIV. Задача},reference=prop:IV.XIV]
\defineNewPicture{
pair A, B, C, D, E, F;
numeric r;
r := 9/4u;
F := (0, 0);
A := (dir(90+0/5(360))*r) shifted F;
B := (dir(90+1/5(360))*r) shifted F;
C := (dir(90+2/5(360))*r) shifted F;
D := (dir(90+3/5(360))*r) shifted F;
E := (dir(90+4/5(360))*r) shifted F;
byAngleDefine(C, F, B, byblue, 0);
byAngleDefine(D, F, C, byblack, 1);
byAngleDefine(B, C, F, byblack, 0);
byAngleDefine(F, C, D, byyellow, 0);
byAngleDefine(C, D, F, byyellow, 0);
byAngleDefine(F, D, E, byred, 0);
draw byNamedAngleResized();
draw byLine(F, A, byyellow, 1, 0);
draw byLine(F, C, byred, 1, 0);
draw byLine(F, D, byblue, 1, 0);
byLineDefine(F, B, byblack, 0, 0);
byLineDefine(F, E, byyellow, 0, 0);
draw byNamedLineSeq(0)(FB,FE);
byLineDefine(A, B, byblack, 0, 1);
byLineDefine(B, C, byblue, 0, 0);
byLineDefine(C, D, byred, 0, 0);
byLineDefine(D, E, byblack, 1, 0);
byLineDefine(E, A, byblack, 0, 1);
draw byNamedLineSeq(0)(AB,BC,CD,DE,EA);
draw byCircleR(F, r, byred, 0, 0, 1);
draw byLabelsOnCircle(A, B, C, D, E)(F);
draw byLabelsOnPolygon(A, F, E)(2, -1);
}
\drawCurrentPictureInMargin
\problemNP{О}{коло}{данного равностороннего и~равноугольного пятиугольника описать круг.}

\startCenterAlign
Рассечем пополам \drawAngle{BCF,FCD} и~\drawAngle{CDF,FDE} с~помощью \drawUnitLine{FC} и~\drawUnitLine{FD}, и~из точки их пересечения проведем \drawUnitLine{FE}, \drawUnitLine{FA} и~\drawUnitLine{FB}.

$\drawAngle{BCF,FCD} = \drawAngle{CDF,FDE}$, $\drawAngle{FCD} = \drawAngle{CDF}$,\\
$\therefore \drawUnitLine{FD} = \drawUnitLine{FC}$ \inprop[prop:I.VI].

И поскольку в
\drawFromCurrentPicture{
draw byNamedAngle(BCF);
startAutoLabeling;
draw byNamedLineSeq(0)(FB,FC,BC);
stopAutoLabeling;
}
и
\drawFromCurrentPicture{
draw byNamedAngle(FCD);
startAutoLabeling;
draw byNamedLineSeq(0)(FC,FD,CD);
stopAutoLabeling;
},\\
$\drawUnitLine{BC} = \drawUnitLine{CD}$, и~\drawUnitLine{FC} общая,\\
а~также $\drawAngle{BCF} = \drawAngle{FCD}$.

$\therefore \drawUnitLine{FB} = \drawUnitLine{FD}$ \inprop[prop:I.IV].

Так же можно показать, что\\
$\drawUnitLine{FA} = \drawUnitLine{FE} = \drawUnitLine{FB}$.

И, следовательно, $\drawUnitLine{FA} = \drawUnitLine{FB} = \drawUnitLine{FC} = \drawUnitLine{FD} = \drawUnitLine{FE}$.
\stopCenterAlign

Таким образом, круг с~центром в~месте пересечения этих пяти линий с~любой из них в~качестве радиуса будет описывать данный пятиугольник.

\qed
\stopProposition

\startProposition[title={Предл. XV. Задача},reference=prop:IV.XV]
\defineNewPicture{
pair A, B, C, D, E, F, G, H;
numeric r;
r := 9/4u;
G := (0, 0);
A := (dir(90)*r) shifted G;
B := (dir(150)*r) shifted G;
C := (dir(210)*r) shifted G;
D := (dir(270)*r) shifted G;
E := (dir(330)*r) shifted G;
F := (dir(30)*r) shifted G;
H := (dir(270)*r) shifted D;
byAngleDefine(A, G, F, byblack, -1);
byAngleDefine(B, G, A, byblack, -1);
byAngleDefine(C, G, B, byblack, -1);
byAngleDefine(D, G, C, byred, 0);
byAngleDefine(E, G, D, byblue, 0);
byAngleDefine(F, G, E, byblack, 0);
draw byNamedAngleResized();
draw byLine(D, H, byred, 0, 0);
draw byLine(A, D, byblack, 0, 0);
draw byLine(B, E, byyellow, 0, 0);
draw byLine(C, F, byblue, 0, 0);
draw byLine(A, B, byblack, 0, 0);
draw byLine(B, C, byblack, 0, 0);
draw byLine(C, D, byblack, 1, 0);
draw byLine(D, E, byred, 1, 0);
draw byLine(E, F, byblue, 1, 0);
draw byLine(F, A, byblack, 0, 0);
draw byCircle.D(D, G, byred, 0, 0, 0);
draw byCircleR(G, r, byyellow, 0, 0, 0);
byLineDefine(G, C, lineColor.CF, 0, 0);
byLineDefine(G, D, lineColor.AD, 0, 0);
byLineDefine(G, E, lineColor.BE, 0, 0);
draw byLabelsOnCircle(A, B, F)(G);
draw byLabelPoint(G, angle(A + F - 2G), 3);
draw byLabelPoint(D, angle(D-G) + 45, 2);
draw byLabelsOnPolygon(D, C, G)(2, 0);
draw byLabelsOnPolygon(G, E, D)(2, 0);
}
\drawCurrentPictureInMargin
\problemNP{В}{данный}{круг \drawCircle[middle][1/4]{G} вписать равносторонний и~равноугольный шестиугольник.}

\startCenterAlign
На любой точке окружности опишем \drawCircle[middle][1/4]{D} проходящий через центр и~проведем диаметры \drawUnitLine{AD}, \drawUnitLine{CF} и~\drawUnitLine{BE}.

Проведем \drawUnitLine{CD}, \drawUnitLine{DE}, \drawUnitLine{EF}, и~т. д., так и~получим искомый шестиугольник вписанный в~круг.

Поскольку \drawUnitLine{GD} проходит через центры кругов,
\drawFromCurrentPicture{
draw byNamedAngle(DGC);
startAutoLabeling;
draw byNamedLineSeq(0)(GC,GD,CD);
stopAutoLabeling;
}
и
\drawFromCurrentPicture{
draw byNamedAngle(EGD);
startAutoLabeling;
draw byNamedLineSeq(0)(GD,GE,DE);
stopAutoLabeling;
}
равносторонние,\\
так как $\drawAngle{DGC} = \drawAngle{EGD} = \dfrac{1}{3} \drawTwoRightAngles$ \inprop[prop:I.XXXII], но $\drawAngle{DGC,EGD,FGE} = \drawTwoRightAngles$ \inprop[prop:I.XIII].

$\therefore \drawAngle{DGC} = \drawAngle{EGD} = \drawAngle{FGE} = \dfrac{1}{3} \drawTwoRightAngles$ \inprop[prop:I.XXXII], и~вертикальные углы равны между собой \inprop[prop:I.XV],\\
и~стоят на равных дугах \inprop[prop:III.XXVI],\\
которые стагивают равные хорды \inprop[prop:III.XXIX].

И поскольку каждый из углов шестиугольника вдвое больше угла равностороннего треугольника, этот шестиугольник также равносторонний.
\stopCenterAlign

\qed
\stopProposition

\startProposition[title={Предл. XVI. Задача},reference=prop:IV.XVI]
\defineNewPicture{
pair A, B, C, D, E, F, G, H, O;
byPointLabelRemove(O);
numeric r, a;
r := 9/4u;
a := 18;
O := (0, 0);
A := (dir(a + 0*120)*r) shifted O;
C := (dir(a + 1*120)*r) shifted O;
D := (dir(a + 2*120)*r) shifted O;
B := (dir(a + 1*72)*r) shifted O;
F := (dir(a + 2*72)*r) shifted O;
G := (dir(a + 3*72)*r) shifted O;
H := (dir(a + 4*72)*r) shifted O;
draw byLine(C, F, byred, 1, 0);
byLineDefine(A, B, byblack, 1, 0);
byLineDefine(B, F, byblue, 1, 0);
byLineDefine(F, G, byblack, 0, 1);
byLineDefine(G, H, byblack, 0, 1);
byLineDefine(H, A, byblack, 0, 1);
byLineDefine(A, C, byyellow, 1, 0);
byLineDefine(C, D, byblack, 0, 1);
byLineDefine(D, A, byblack, 0, 1);
draw byNamedLineSeq(0)(AB,BF,FG,GH,HA);
draw byNamedLineSeq(0)(AC,CD,DA);
draw byCircleR(O, r, byblack, 0, 1, 1);
draw byArc.AB(O, A, B, r, byblack, 0, 0, 1, 0);
draw byArc.BC(O, B, C, r, byblue, 0, 0, 1, 0);
draw byArc.CF(O, C, F, r, byred, 0, 0, 1, 0);
draw byLabelsOnCircle(A, B, C, F)(O);
}
\drawCurrentPictureInMargin
\problemNP{В}{данный}{круг вписать равносторонний и~равноугольный пятнадцатиугольник.}

Пусть \drawUnitLine{AB} и~\drawUnitLine{BF} будут сторонами правильного пятиугольника, вписанного в~данный круг, а~\drawUnitLine{AC} стороной вписанного правильного треугольника.

\startCenterAlign
Дуга $\drawFromCurrentPicture{
startGlobalRotation(-angle(A-C));
startAutoLabeling;
draw byNamedArc(AB,BC);
stopAutoLabeling;
stopGlobalRotation;
} 
= \dfrac{1}{3} = \dfrac{5}{15} \drawCircle[middle][1/5]{O}$.

Дуга $\drawFromCurrentPicture{
startGlobalRotation(-angle(A-F));
startAutoLabeling;
draw byNamedArc(AB,BC,CF);
stopAutoLabeling;
stopGlobalRotation;
} 
= \dfrac{2}{5} = \dfrac{6}{15} \drawCircle[middle][1/5]{O}$.

Их разность $
\drawFromCurrentPicture{
startGlobalRotation(-angle(A-F));
startTempScale(1/2);
startAutoLabeling;
draw byNamedArc(CF);
stopAutoLabeling;
stopTempScale;
stopGlobalRotation;
}
 = \dfrac{1}{15} \drawCircle[middle][1/5]{O}$

$\therefore$ дуга стягиваемая $\drawUnitLine{FC} = \dfrac{1}{15}$ всей окружности.
\stopCenterAlign

А значит, если прямые, равные \drawUnitLine{FC}, вписать в~круг \inprop[prop:IV.I], получим равносторонний и~равноугольный пятнадцатиугольник вписанный в~круг.

\qed
\stopProposition
\stopBook

\startBook[title={Книга V}]

\startsupersection[title={Определения}]

\startDefinitionOnlyNumber[reference=def:V.I,ownnumber=1]
Меньшая величина называется аликвотой большей, если меньшая измеряет большую, то есть вмещается в~нее целое число раз.
\stopDefinitionOnlyNumber

\startDefinitionOnlyNumber[reference=def:V.II,ownnumber=2]
Большая величина называется кратной меньшей, если большая измеряется меньшей, то есть содержит в~себе меньшую целое число раз.
\stopDefinitionOnlyNumber

\startDefinitionOnlyNumber[reference=def:V.III,ownnumber=3]
Отношением называют зависимость однородных величин по количеству.
\stopDefinitionOnlyNumber

\startDefinitionOnlyNumber[reference=def:V.IV,ownnumber=4]
Про две величины говорят, что они имеют отношение между собой, когда они однородны и~меньшая из них может, взятая кратно, превзойти другую.
\stopDefinitionOnlyNumber

\vskip \baselineskip

\startalignment[middle]
\emph{Прочие определения будут даны по ходу изложения там, где в~них возникнет необходимость.}
\stopalignment
\stopsupersection

\vfill\pagebreak

\startsupersection[title={Аксиомы}]

\startAxiomOnlyNumber[reference=ax:V.I]
Равнократные или равных величин равны между собой.

\startalignment[middle]
$\eqalign{
\mbox{Если } A &= B \mbox{, то}\cr
\mbox{дважды } A &= \mbox{дважды } B \mbox{,}\cr
\mbox{то есть, }2A &= 2B \mbox{;}\cr
3A &= 3B \mbox{;}\cr
4A &= 4B \mbox{;}\cr
\mbox{и т. д. } & \mbox{и т. д.}\cr
\mbox{и } \frac{1}{2} A &= \frac{1}{2} B \mbox{;}\cr
\frac{1}{3} A &= \frac{1}{3} B \mbox{;}\cr
\frac{1}{4} A &= \frac{1}{4} B \mbox{;}\cr
\mbox{и т. д. } &\mbox{и т. д. } \cr
}$
\stopalignment
\stopAxiomOnlyNumber

\startAxiomOnlyNumber[reference=ax:V.II]
Кратное большей величины больше такого же кратного меньшей.

\startalignment[middle]
$\eqalign{
\mbox{Пусть } A &> B \mbox{,} \cr
\mbox{Тогда } 2A &> 2B \mbox{;}\cr
3A&> 3B \mbox{;} \cr
4A &> 4B \mbox{;}\cr
\mbox{и т. д. } &\mbox{и т. д. } \cr
}$
\stopalignment
\stopAxiomOnlyNumber

\pagebreak

\startAxiomOnlyNumber[reference=ax:V.III]
Величина, кратное которой больше такого же кратного другой величины, сама больше другой величины.

\startalignment[middle]
$\eqalign{
\mbox{Пусть } 2A &> 2B \mbox{,}\cr
\mbox{тогда } A &> B \mbox{;}\cr
\mbox{или пусть } 3A &> 3B \mbox{,}\cr
\mbox{тогда } A &> B \mbox{;}\cr
\mbox{или пусть } mA &> mB  \mbox{,}\cr
\mbox{тогда } A &> B \mbox{.}\cr
}$
\stopalignment
\stopAxiomOnlyNumber
\stopsupersection

\vfill\pagebreak

\startProposition[title={Предл. I. Теорема}, reference=prop:V.I]
\defineNewPicture{
byMagnitudeSymbolDefine.ab("semicircleUp", byred, 1);
byMagnitudeSymbolDefine.cd("wedgeDown", byyellow, 0);
byMagnitudeSymbolDefine.ef("sectorDown", byblue, 1);
byMagnitudeDefine.A(0, false)(5)(ab);
byMagnitudeDefine.B(0, false)(1)(ab);
byMagnitudeDefine.C(0, false)(5)(cd);
byMagnitudeDefine.D(0, false)(1)(cd);
byMagnitudeDefine.E(0, false)(5)(ef);
byMagnitudeDefine.F(0, false)(1)(ef);
}
\problemNP{Е}{сли}{будет несколько величин равнократных каждая каждой такому же количеству других величин, то сколько раз одна из первых будет кратна одной из вторых, столько же раз все первые будут кратны всем вторым.}

\startCenterAlign
Пусть \drawMagnitude{A} будет столько же раз кратна \drawMagnitude{B},
сколько и~\drawMagnitude{C} кратна \drawMagnitude{D},\\
сколько и~\drawMagnitude{E} кратна \drawMagnitude{F}.\\

Тогда очевидно, что

$\vcenter{
\nointerlineskip\hbox{\drawMagnitude{A}}
\nointerlineskip\hbox{\drawMagnitude{C}}
\nointerlineskip\hbox{\drawMagnitude{E}}
}$
столько же раз кратна
$\vcenter{
\nointerlineskip\hbox{\drawMagnitude{B}}
\nointerlineskip\hbox{\drawMagnitude{D}}
\nointerlineskip\hbox{\drawMagnitude{F}}
}$

сколько и~\drawMagnitude{A} кратна \drawMagnitude{B}.

Так как в
$\vcenter{
\nointerlineskip\hbox{\drawMagnitude{A}}
\nointerlineskip\hbox{\drawMagnitude{C}}
\nointerlineskip\hbox{\drawMagnitude{E}}
}
=
\vcenter{
\nointerlineskip\hbox{\drawMagnitude{B}}
\nointerlineskip\hbox{\drawMagnitude{D}}
\nointerlineskip\hbox{\drawMagnitude{F}}
}$ столько же величин,\\
сколько в~$\drawMagnitude{A} = \drawMagnitude{B}$.
\stopCenterAlign

То же верно для любого числа величин, что было показано для трех.

$\therefore$ Если будет несколько величин… и~т. д.
\stopProposition

\vfill\pagebreak

\startProposition[title={Предл. II. Теорема}, reference=prop:V.II]
\defineNewPicture{
byMagnitudeSymbolDefine.ab("circle", byyellow, 0);
byMagnitudeSymbolDefine.cd("sectorUp", byred, 1);
byMagnitudeSymbolDefine.e("circle", byblue, 0);
byMagnitudeSymbolDefine.f("sectorUp", byblack, 1);
byMagnitudeDefine.A(0, false)(3)(ab);
byMagnitudeDefine.B(0, false)(1)(ab);
byMagnitudeDefine.C(0, false)(3)(cd);
byMagnitudeDefine.D(0, false)(1)(cd);
byMagnitudeDefine.E(0, false)(4)(e);
byMagnitudeDefine.F(0, false)(4)(f);
}
\problemNP{Е}{сли}{первая величина столько же раз кратна второй, сколько третья четвертой, а~пятая столько же раз кратна второй, сколько шестая четвертой, то первая вместе с~пятой будут столько же раз кратны третьей, сколько третья вместе с~шестой — четвертой.}

Пусть первая величина \drawMagnitude{A} будет столько же раз кратна второй \drawMagnitude{B}, сколько третья \drawMagnitude{C} четвертой \drawMagnitude{D}.

И пусть пятая \drawMagnitude{E} будет столько же раз кратна второй \drawMagnitude{B}, сколько шестая \drawMagnitude{F} четвертой \drawMagnitude{D}.

Тогда очевидно что первая и~пятая вместе
$\vcenter{
\nointerlineskip\hbox{\drawMagnitude{A}}
\nointerlineskip\hbox{\drawMagnitude{E}}
}$,
столько же раз кратны второй \drawMagnitude{B}, сколько третья и~шестая вместе
$\vcenter{
\nointerlineskip\hbox{\drawMagnitude{C}}
\nointerlineskip\hbox{\drawMagnitude{F}}
}$,
кратны четвертой \drawMagnitude{D}, поскольку в
$\vcenter{
\nointerlineskip\hbox{\drawMagnitude{A}}
\nointerlineskip\hbox{\drawMagnitude{E}}
} = \drawMagnitude{B}$
столько же величин, сколько в
$\vcenter{
\nointerlineskip\hbox{\drawMagnitude{C}}
\nointerlineskip\hbox{\drawMagnitude{F}}
} =\drawMagnitude{D}$.

$\therefore$ Если первая величина… и~т. д.
\stopProposition

\vfill\pagebreak

\startProposition[title={Предл. III. Теорема}, reference=prop:V.III]
\defineNewPicture{
byMagnitudeSymbolDefine.a("square", byyellow, 0);
byMagnitudeSymbolDefine.be("square", byred, 0);
byMagnitudeSymbolDefine.c("rhombus", byblack, 0);
byMagnitudeSymbolDefine.df("rhombus", byblue, 0);
byMagnitudeDefine.A(0, false)(1, 2, 1)(a);
byMagnitudeDefine.B(0, false)(1)(be);
byMagnitudeDefine.C(0, false)(2, 2)(c);
byMagnitudeDefine.D(0, false)(1)(df);
byMagnitudeDefine.E(0, false)(4, 4, 4, 4)(be);
byMagnitudeDefine.F(0, false)(4, 4, 4, 4)(df);
}
\problemNP{Е}{сли}{первая величина имеет такое же отношение ко второй, какое третья имеет к~четвертой, то и~равнократные к~первой и~третьей к~равнократным второй и~четвертой, взятые в~соответствующем порядке будут иметь то же отношение.}

\startCenterAlign
Пусть у~\drawMagnitude{A} будет то же отношение к~\drawMagnitude{B}, \\
что и~у \drawMagnitude{C} к~\drawMagnitude{D}.

Возьмем\\
\drawMagnitude{E} столько же раз кратную \drawMagnitude{A},\\
сколько \drawMagnitude{F} кратную \drawMagnitude{C}.

Тогда очевидно, что\\
\drawMagnitude{E} столько же раз кратна \drawMagnitude{B},\\
сколько \drawMagnitude{F} кратна \drawMagnitude{D}.

$\because$ \drawMagnitude{E} содержит \drawMagnitude{A} содержит \drawMagnitude{B} \\
столько же раз, сколько \\
\drawMagnitude{F} содержит \drawMagnitude{C} содержит \drawMagnitude{D}.

Те же соображения применимы во всех случаях.

$\therefore$ Если первая величина… и~т. д.
\stopCenterAlign
\stopProposition

\vfill\pagebreak

\startDefinition[title={Определение V.}, reference=def:V.V,ownnumber=5]
\defineNewPicture{
byMagnitudeSymbolDefine.a("circle", byred, 0);
byMagnitudeSymbolDefine.b("square", byyellow, 0);
byMagnitudeSymbolDefine.c("rhombus", byblue, 0);
byMagnitudeSymbolDefine.d("wedgeDown", byblack, 0);
byMagnitudeDefine.A(0, false)(1)(a);
byMagnitudeDefine.B(0, false)(1)(b);
byMagnitudeDefine.C(0, false)(1)(c);
byMagnitudeDefine.D(0, false)(1)(d);
byMagnitudeDefine.Am(1, false)(2, 3, 4, 5, 6)(a);
byMagnitudeDefine.Bm(1, false)(2, 3, 4, 5, 6)(b);
byMagnitudeDefine.Cm(1, false)(2, 3, 4, 5, 6)(c);
byMagnitudeDefine.Dm(1, false)(2, 3, 4, 5, 6)(d);
}

Четыре величины, \drawMagnitude{A}, \drawMagnitude{B}, \drawMagnitude{C}, \drawMagnitude{D}, называются пропорциональными, если любые равнократные первой и~третьей, а~также любые равнократные второй и~четвертой в~виде

\sepSpace

\vbox{\hfill\hbox{
\vbox{\hbox{первой}\hbox{\drawMagnitude{Am}}\hbox{и т. д.}}
\vbox{\hbox{третьей}\hbox{\drawMagnitude{Cm}}\hbox{и т. д.}}
}\hfill\ }

\sepSpace

\vbox{\hfill\hbox{
\vbox{\hbox{второй}\hbox{\drawMagnitude{Bm}}\hbox{и т. д.}}
\vbox{\hbox{четвертой}\hbox{\drawMagnitude{Dm}}\hbox{и т. д.}}
}\hfill\ }

\sepSpace

Тогда, взяв любую пару равнократных первой и~третьей и~любую пару равнократных второй и~четвертой,

\startCenterAlign
если
$\left\{\vcenter{
\nointerlineskip\hbox{$\drawMagnitude[middle][1]{Am} >, = \mbox{или} < \drawMagnitude[middle][1]{Bm}$}
\nointerlineskip\hbox{$\drawMagnitude[middle][1]{Am} >, = \mbox{или} < \drawMagnitude[middle][2]{Bm}$}
\nointerlineskip\hbox{$\drawMagnitude[middle][1]{Am} >, = \mbox{или} < \drawMagnitude[middle][3]{Bm}$}
\nointerlineskip\hbox{$\drawMagnitude[middle][1]{Am} >, = \mbox{или} < \drawMagnitude[middle][4]{Bm}$}
\nointerlineskip\hbox{$\drawMagnitude[middle][1]{Am} >, = \mbox{или} < \drawMagnitude[middle][5]{Bm}$}
}\right.$

$\vcenter{\hbox{то}}\left\{\vcenter{
\nointerlineskip\hbox{$\drawMagnitude[middle][1]{Cm} >, = \mbox{или} < \drawMagnitude[middle][1]{Dm}$}
\nointerlineskip\hbox{$\drawMagnitude[middle][1]{Cm} >, = \mbox{или} < \drawMagnitude[middle][2]{Dm}$}
\nointerlineskip\hbox{$\drawMagnitude[middle][1]{Cm} >, = \mbox{или} < \drawMagnitude[middle][3]{Dm}$}
\nointerlineskip\hbox{$\drawMagnitude[middle][1]{Cm} >, = \mbox{или} < \drawMagnitude[middle][4]{Dm}$}
\nointerlineskip\hbox{$\drawMagnitude[middle][1]{Cm} >, = \mbox{или} < \drawMagnitude[middle][5]{Dm}$}
}\right.$
\stopCenterAlign

Другими словами, если дважды первая величина больше, равна или меньше дважды второй, то дважды третья будет больше равна или меньше дважды четвертой. Или, если дважды первая будет больше равна или меньше трижды второй, дважды третья будет больше, равна или меньше трижды четвертой и~так далее, как обозначено выше.

\startCenterAlign
Если
$\left\{\vcenter{
\nointerlineskip\hbox{$\drawMagnitude[middle][2]{Am} >, = \mbox{или} < \drawMagnitude[middle][1]{Bm}$}
\nointerlineskip\hbox{$\drawMagnitude[middle][2]{Am} >, = \mbox{или} < \drawMagnitude[middle][2]{Bm}$}
\nointerlineskip\hbox{$\drawMagnitude[middle][2]{Am} >, = \mbox{или} < \drawMagnitude[middle][3]{Bm}$}
\nointerlineskip\hbox{$\drawMagnitude[middle][2]{Am} >, = \mbox{или} < \drawMagnitude[middle][4]{Bm}$}
\nointerlineskip\hbox{$\drawMagnitude[middle][2]{Am} >, = \mbox{или} < \drawMagnitude[middle][5]{Bm}$}
}\right.$

$\vcenter{\hbox{то}}\left\{\vcenter{
\nointerlineskip\hbox{$\drawMagnitude[middle][2]{Cm} >, = \mbox{или} < \drawMagnitude[middle][1]{Dm}$}
\nointerlineskip\hbox{$\drawMagnitude[middle][2]{Cm} >, = \mbox{или} < \drawMagnitude[middle][2]{Dm}$}
\nointerlineskip\hbox{$\drawMagnitude[middle][2]{Cm} >, = \mbox{или} < \drawMagnitude[middle][3]{Dm}$}
\nointerlineskip\hbox{$\drawMagnitude[middle][2]{Cm} >, = \mbox{или} < \drawMagnitude[middle][4]{Dm}$}
\nointerlineskip\hbox{$\drawMagnitude[middle][2]{Cm} >, = \mbox{или} < \drawMagnitude[middle][5]{Dm}$}
}\right.$
\stopCenterAlign

Или же если трижды первая больше, равна или меньше дважды второй, то трижды третья больше, равна или меньше дважды четвертой. Или если трижды первая больше, равна или меньше трижды второй трижды второй, то трижды третья будет больше, равна или меньше трижды четвертой. Или если трижды первая  больше, равна или меньше четырежды второй, то трижды третья будет больше, равна или меньше четырежды четвертой и~так далее. Также

\startCenterAlign
если
$\left\{\vcenter{
\nointerlineskip\hbox{$\drawMagnitude[middle][3]{Am} >, = \mbox{или} < \drawMagnitude[middle][1]{Bm}$}
\nointerlineskip\hbox{$\drawMagnitude[middle][3]{Am} >, = \mbox{или} < \drawMagnitude[middle][2]{Bm}$}
\nointerlineskip\hbox{$\drawMagnitude[middle][3]{Am} >, = \mbox{или} < \drawMagnitude[middle][3]{Bm}$}
\nointerlineskip\hbox{$\drawMagnitude[middle][3]{Am} >, = \mbox{или} < \drawMagnitude[middle][4]{Bm}$}
\nointerlineskip\hbox{$\drawMagnitude[middle][3]{Am} >, = \mbox{или} < \drawMagnitude[middle][5]{Bm}$}
}\right.$

$\vcenter{\hbox{то}}\left\{\vcenter{
\nointerlineskip\hbox{$\drawMagnitude[middle][3]{Cm} >, = \mbox{или} < \drawMagnitude[middle][1]{Dm}$}
\nointerlineskip\hbox{$\drawMagnitude[middle][3]{Cm} >, = \mbox{или} < \drawMagnitude[middle][2]{Dm}$}
\nointerlineskip\hbox{$\drawMagnitude[middle][3]{Cm} >, = \mbox{или} < \drawMagnitude[middle][3]{Dm}$}
\nointerlineskip\hbox{$\drawMagnitude[middle][3]{Cm} >, = \mbox{или} < \drawMagnitude[middle][4]{Dm}$}
\nointerlineskip\hbox{$\drawMagnitude[middle][3]{Cm} >, = \mbox{или} < \drawMagnitude[middle][5]{Dm}$}
}\right.$
\stopCenterAlign

И так далее, для любых равнократных четырех величин взятых таким образом.

Евклид излагает это определение следующим образом:

Говорят, что величины находятся в~том же отношении: первая ко второй и~третья к~четвертой, если равнократные первой и~третьей одновременно больше, или одновременно равны, или одновременно меньше равнократных второй и~четвертой каждая каждой при какой бы то ни было кратности, если взять их в~соответственном порядке. % прямо из Мордухай-Болтовского

В дальнейшем будем выражать это определение в~общем виде так:

\startCenterAlign
$\eqalign{
\mbox{если }M \drawMagnitude{A} &>, = \mbox{ или } < m \drawMagnitude{B} \mbox{,}\cr
\mbox{когда }M \drawMagnitude{C} &>, = \mbox{или } < m \drawMagnitude{D} \mbox{.}\cr
}$
\stopCenterAlign

То мы заключаем, что первая величина \drawMagnitude{A}, так же относится ко второй \drawMagnitude{B}, как третья \drawMagnitude{C} к~четвертой \drawMagnitude{D}. Что будет далее записываться так:

\startCenterAlign
$\eqalign{
\drawMagnitude{A} : \drawMagnitude{B} & :: \drawMagnitude{C} : \drawMagnitude{D} \mbox{;} \cr
\mbox{или так: } \drawMagnitude{A} : \drawMagnitude{B} &= \drawMagnitude{C} : \drawMagnitude{D} \mbox{;} \cr
\mbox{или так: } \dfrac{\drawMagnitude{A}}{\drawMagnitude{B}} &= \dfrac{\drawMagnitude{C}}{\drawMagnitude{D}} \mbox{:} \cr
}$

и читается как, \\
\quotation{\drawMagnitude{A} относится к~\drawMagnitude{B}, как \drawMagnitude{C} к~\drawMagnitude{D}.}

И если $\drawMagnitude{A} : \drawMagnitude{B} :: \drawMagnitude{C} : \drawMagnitude{D}$ мы можем заключить, что \\
если $M \drawMagnitude{A} >, = \mbox{ или } < m \drawMagnitude{B}$, \\
то $M \drawMagnitude{C} >, = \mbox{ или } < m \drawMagnitude{D}$.
\stopCenterAlign

То есть, если первая относится ко второй, как третья к~четвертой, то если $M$ раз первая больше, равна или меньше $m$ раз второй, то $M$ раз третья будет больше, равна или меньше $m$ раз четвертой, где $M$ и~$m$ не представляют каких либо конкретных величин, но любую пару величин, также как обозначения, наподобие \drawMagnitude{A}, \drawMagnitude{D}, \drawMagnitude{B}, и~т. п. не более чем представления геометрических величин.

Учащийся должен крепко понять это определение, прежде чем двигаться далее.
\stopDefinition

\vfill\pagebreak

\startProposition[title={Предл. IV. Теорема}, reference=prop:V.IV]
\defineNewPicture{
byMagnitudeSymbolDefine.I("circle", byyellow, 0);
byMagnitudeSymbolDefine.II("square", byblack, 0);
byMagnitudeSymbolDefine.III("rhombus", byred, 0);
byMagnitudeSymbolDefine.IV("wedgeDown", byblue, 0);
}
\problemNP{Е}{сли}{первая величина имеет такое же отношение ко второй, какое третья имеет к~четвертой, то и~равнократные первой и~третьей будут иметь то же отношение к~равнократным второй и~четвертой.}

\startCenterAlign
Пусть $\drawMagnitude{I} : \drawMagnitude{II} :: \drawMagnitude{III} : \drawMagnitude{IV}$,\\
тогда $3\drawMagnitude{I} : 2\drawMagnitude{II} :: 3\drawMagnitude{III} : 2\drawMagnitude{IV}$,\\
всякое равнократное $3\drawMagnitude{I}$ и~$3\drawMagnitude{III}$ будет равнократным \drawMagnitude{I} и~\drawMagnitude{III},\\
и всякое равнократное $2\drawMagnitude{II}$ и~$2\drawMagnitude{IV}$ будет равнократным \drawMagnitude{II} и~\drawMagnitude{IV} \inprop[prop:V.III].

То есть, $M$ раз $3\drawMagnitude{I}$ и~$M$ раз $3\drawMagnitude{III}$\\
равнократны \drawMagnitude{I} и~\drawMagnitude{III},\\
и $m$ раз $2\drawMagnitude{II}$ и~$m$ раз $2\drawMagnitude{IV}$\\
равнократны $2\drawMagnitude{II}$ и~$2\drawMagnitude{IV}$.

Но $\drawMagnitude{I} : \drawMagnitude{II} :: \drawMagnitude{III} : \drawMagnitude{IV}$ (\hypstr)\\
$\therefore$ если $M 3\drawMagnitude{I} <, =, \mbox{ или } > m 2 \drawMagnitude{II}$,\\
то $M 3\drawMagnitude{III} <, =, \mbox{ или } > m 2 \drawMagnitude{IV}$ \indef[def:V.V]\\
и значит $3\drawMagnitude{I} : 2\drawMagnitude{II} :: 3\drawMagnitude{III} : 2\drawMagnitude{IV}$ \indef[def:V.V].
\stopCenterAlign

Те же соображения работают для любых равнократных первой и~третьей величин и~любых равнократных второй и~четвертой.

$\therefore$ Если первая величина… и~т.  д.
\stopProposition

\vfill\pagebreak

\startProposition[title={Предл. V. Теорема}, reference=prop:V.V]
\defineNewPicture{
byMagnitudeSymbolDefine.a("sectorDown", byblue, 1);
byMagnitudeSymbolDefine.b("semicircleDown", byyellow, 1);
byMagnitudeSymbolDefine.c("miniTriangleUp", byblack, 0);
byMagnitudeSymbolDefine.d("miniSquare", byred, 0);
byMagnitudeDefine.I(0, false)(1, 2, 1)(a, a, b);
byMagnitudeDefine.II(0, false)(1, 1)(c, d);
}
\problemNP{Е}{сли}{одна величина столько же раз кратна другой, сколько и~величина отнимаемая от первой кратна к~величине отнимаемой от второй, то остаток будет столько же раз кратен остатку, как целое целому.}

\startCenterAlign
Пусть $\drawMagnitude{I} = M' \drawMagnitude{II}$\\
и $\drawMagnitude[middle][3]{I} = M' \drawMagnitude[middle][2]{II}$.

$\therefore \drawMagnitude{I} - \drawMagnitude[middle][3]{I} = M' \drawMagnitude{II} - M' \drawMagnitude[middle][2]{II}$.

$\therefore \drawMagnitude[middle][-3]{I} = M' (\drawMagnitude{II} - \drawMagnitude[middle][2]{II})$,\\
и $\therefore \drawMagnitude[middle][-3]{I} = M' \drawMagnitude[middle][-2]{II}$.

$\therefore$ Если одна величина… и~т. д.
\stopCenterAlign
\stopProposition

\vfill\pagebreak

\startProposition[title={Предл. VI. Теорема}, reference=prop:V.VI]
\defineNewPicture{
byMagnitudeSymbolDefine.a("sectorDown", byyellow, 1);
byMagnitudeSymbolDefine.b("miniSquare", byred, 0);
byMagnitudeSymbolDefine.c("semicircleDown", byblack, 1);
byMagnitudeSymbolDefine.d("miniTriangleUp", byblue, 0);
byMagnitudeDefine.I(0, false)(1, 2, 1)(a);
byMagnitudeDefine.II(0, false)(1)(b);
byMagnitudeDefine.III(0, false)(2)(c);
byMagnitudeDefine.IV(0, false)(1)(d);
}
\problemNP[5]{Е}{сли}{две величины равнократны двум другим, и~если равнократные последним отняты от первых двух, то остатки будут либо равны вторым, либо будут им равнократны.}

\startCenterAlign
Пусть $\drawMagnitude{I} = M'\drawMagnitude{II}$\\ 
и~$\drawMagnitude{III} = M'\drawMagnitude{IV}$.

Тогда $\drawMagnitude{I} - m'\drawMagnitude{II} = M'\drawMagnitude{II} - m'\drawMagnitude{II} = (M' - m') \drawMagnitude{II}$\\
и $\drawMagnitude{III} - m'\drawMagnitude{IV} = M'\drawMagnitude{IV} - m'\drawMagnitude{IV} = (M' - m') \drawMagnitude{IV}$.

Значит, $(M' - m') \drawMagnitude{II}$ и~$(M' - m') \drawMagnitude{IV}$ равнократны \drawMagnitude{II} и~\drawMagnitude{IV} и~равны \drawMagnitude{II} и~\drawMagnitude{IV}, когда $M' - m' = 1$.

$\therefore$ Если две величины… и~т. д.
\stopCenterAlign
\stopProposition

\vfill\pagebreak

\startPropositionAZ[title={Предл. A. Теорема}, reference=prop:V.A]
\defineNewPicture{
byMagnitudeSymbolDefine.a("circle", byred, 0);
byMagnitudeSymbolDefine.b("square", byblack, 0);
byMagnitudeSymbolDefine.c("wedgeDown", byblue, 0);
byMagnitudeSymbolDefine.d("rhombus", byyellow, 0);
byMagnitudeDefine.I(0, false)(1)(a);
byMagnitudeDefine.II(0, false)(1)(b);
byMagnitudeDefine.III(0, false)(1)(c);
byMagnitudeDefine.IV(0, false)(1)(d);
byMagnitudeDefine.dI(0, false)(2)(a);
byMagnitudeDefine.dII(0, false)(2)(b);
byMagnitudeDefine.dIII(0, false)(2)(c);
byMagnitudeDefine.dIV(0, false)(2)(d);
}
\problemNP{Е}{сли}{первая величина относится ко второй так же, как третья к~четвертой, то если первая больше второй, то третья также больше четвертой,
если равна, равна, а~если меньше, то меньше.}

\startCenterAlign
Пусть $\drawMagnitude{I} : \drawMagnitude{II} :: \drawMagnitude{III} : \drawMagnitude{IV}$.

Следовательно \indef[def:V.V], если $\drawMagnitude{dI} > \drawMagnitude{dII}$, то $\drawMagnitude{dIII} > \drawMagnitude{dIV}$.

Но если $\drawMagnitude{I} > \drawMagnitude{II}$,\\
то $\drawMagnitude{dI} > \drawMagnitude{dII}$ и~$\drawMagnitude{dIII} > \drawMagnitude{dIV}$,\\
и $\therefore \drawMagnitude{III} > \drawMagnitude{IV}$.

Точно так же, если $\drawMagnitude{I} =, \mbox{ или } < \drawMagnitude{II}$, то  $\drawMagnitude{III} =, \mbox{ или } < \drawMagnitude{IV}$.

$\therefore$ Если первая величина… и~т. д.
\stopCenterAlign
\stopPropositionAZ

\startDefinition[title={Определение XIV},reference=def:V.XIV,ownnumber=14]
\def\varA{\color[byred]{A}}
\def\varB{\color[byblack]{B}}
\def\varC{\color[byblue]{C}}
\def\varD{\color[byyellow]{D}}
В геометрии используется понятие \quotation{перевернутое отношение}, когда есть четыре пропорциональных величины, утверждается, что вторая к~первой относится так же, как четвертая к~третьей.

Пусть $\varA : \varB :: \varC : \varD$, тогда, \quotation{перевернув} отношение можно заключить, что $\varB : \varA :: \varD : \varC$
\stopDefinition

\vfill\pagebreak

\startPropositionAZ[title={Предл. B. Теорема}, reference=prop:V.B]
\defineNewPicture{
byMagnitudeSymbolDefine.I("wedgeDown", byblue, 0);
byMagnitudeSymbolDefine.II("semicircleDown", byblack, 1);
byMagnitudeSymbolDefine.III("square", byred, 0);
byMagnitudeSymbolDefine.IV("rhombus", byyellow, 0);
}
\problemNP{Е}{сли}{четыре величины пропорциональны, они будут пропорциональны и~если их перевернуть.}

\startCenterAlign
Пусть $\drawMagnitude{I} : \drawMagnitude{II} :: \drawMagnitude{III} : \drawMagnitude{IV}$,\\
тогда, перевернув, получим $\magnitudeII : \magnitudeI :: \magnitudeIV : \magnitudeIII$.

Если $M \magnitudeI < m \magnitudeII$, то $M \magnitudeIII < m \magnitudeIV$ \indef[def:V.V].

Пусть $M \magnitudeI < m \magnitudeII$, то есть, $m \magnitudeII > M \magnitudeI$,\\
$\therefore M \magnitudeIII < m \magnitudeIV$, или, $m \magnitudeIV > M \magnitudeIII$.

$\therefore$ если $m \magnitudeII > M \magnitudeI$, то $m \magnitudeIV > M \magnitudeIII$.

Так же можно показать, что,\\
если $m \magnitudeII = \mbox{ или } < M \magnitudeI$,\\
то $m \magnitudeIV =, \mbox{ или } < M \magnitudeIII$.

И, следовательно \indef[def:V.V], мы заключаем,\\
что $\magnitudeII : \magnitudeI :: \magnitudeIV : \magnitudeIII$.

$\therefore$ Если четыре величины пропорциональны… и~т. д.
\stopCenterAlign
\stopPropositionAZ

\vfill\pagebreak

\startPropositionAZ[title={Предл. C. Теорема}, reference=prop:V.C]
\defineNewPicture{
byMagnitudeSymbolDefine.i("square", byblue, 0);
byMagnitudeDefine.I(0, false)(2, 2)(i);
byMagnitudeSymbolDefine.II("circle", byblack, 0);
byMagnitudeSymbolDefine.iii("rhombus", byyellow, 0);
byMagnitudeDefine.III(0, false)(2, 2)(iii);
byMagnitudeSymbolDefine.IV("wedgeUp", byred, 0);
}
\problemNP{Е}{сли}{первая величина столько же раз кратна второй, или составляет такую же ее часть, как и~третья — четвертой, то первая относится ко второй, как третья к~четвертой.}

\startCenterAlign
Пусть первая величина \drawMagnitude{I} будет столько же раз кратна второй \drawMagnitude{II}, \\
сколько третья \drawMagnitude{III} четвертой \drawMagnitude{IV}.

Тогда $\magnitudeI : \magnitudeII :: \magnitudeIII : \magnitudeIV$.

Возьмем $M \magnitudeI$, $m \magnitudeII$, $M \magnitudeIII$, $m \magnitudeIV$.

Поскольку \magnitudeI\ столько же раз кратна \magnitudeII \\
сколько \magnitudeIII кратна \magnitudeIV\ (\hypstr),\\
и $M \magnitudeI$ столько же раз кратна \magnitudeI \\
столько $M \magnitudeIII$ кратна \magnitudeIII.

$\therefore$ (предл.),\\
$M \magnitudeI$ столько же раз кратна \magnitudeII \\
сколько $M \magnitudeIII$ кратна \magnitudeIV.

Следовательно, если $M \magnitudeI$ больше раз кратна \magnitudeII, чем $m \magnitudeII$, \\
то $M \magnitudeIII$ больше раз кратна \magnitudeIV, чем $m \magnitudeIV$.

То есть, если $M \magnitudeI$ больше $m \magnitudeII$, то $M \magnitudeIII$ больше $m \magnitudeIV$.

Так же можно показать, что если $M \magnitudeI$ равна $m \magnitudeII$, то $M \magnitudeIII$ равна $m \magnitudeIV$.

И, в~общем виде, если $M \magnitudeI >, = \mbox{ или } < m \magnitudeII$\\
то и~$M \magnitudeIII >, = \mbox{ или } < m \magnitudeIV$.

$\therefore$ \indef[def:V.V],\\
$\magnitudeI : \magnitudeII :: \magnitudeIII : \magnitudeIV$.

Теперь, пусть \magnitudeII будет такой же частью от \magnitudeI \\
какая \magnitudeIV\ от \magnitudeIII.

В этом случай так же $\magnitudeII : \magnitudeI :: \magnitudeIV : \magnitudeIII$.

Поскольку \magnitudeII\ такая же часть от \magnitudeI\ что и~\magnitudeIV\ от \magnitudeIII.

Следовательно \magnitudeI\ столько же раз кратна \magnitudeII, \\
сколько \magnitudeIII\ кратна \magnitudeIV.

Следовательно, как и~в предыдущем случае, \\
$\magnitudeI : \magnitudeII :: \magnitudeIII : \magnitudeIV$.

И $\therefore \magnitudeII : \magnitudeI :: \magnitudeIV : \magnitudeIII$ \inprop[prop:V.B].\\

$\therefore$ Если первая величина… и~т. д.
\stopCenterAlign
\stopPropositionAZ

\vfill\pagebreak

\startPropositionAZ[title={Предл. D. Теорема}, reference=prop:V.D]
\defineNewPicture{
byMagnitudeSymbolDefine.i("circle", byyellow, 0);
byMagnitudeSymbolDefine.ii("square", byblack, 0);
byMagnitudeSymbolDefine.iii("rhombus", byred, 0);
byMagnitudeSymbolDefine.iv("wedgeDown", byblue, 0);
byMagnitudeSymbolDefine.va("semicircleDown", byred, 1);
byMagnitudeSymbolDefine.vb("semicircleUp", byred, 1);
byMagnitudeSymbolDefine.via("sectorDown", byblack, 1);
byMagnitudeSymbolDefine.vib("sectorUp", byblack, 1);
byMagnitudeDefine.I(0, false)(1, 2)(i);
byMagnitudeDefine.II(0, false)(1)(ii);
byMagnitudeDefine.III(0, false)(2, 2)(iii);
byMagnitudeDefine.IV(0, false)(1)(iv);
byMagnitudeDefine.V(0, false)(1, 2)(va, vb);
byMagnitudeDefine.VI(0, false)(2, 2)(via, vib);
}
\problemNP{Е}{сли}{первая величина относится ко второй так, как третья к~четвертой, и~если первая кратна второй или составляет ее часть, то и~третья к~четвертой столько же раз кратна или составляет такую же ее часть.}

\setbox0\vbox{
\unprotect
\vbox{\halign{\hfil # \hfil & \hfil # \hfil & \hfil # \hfil & \hfil # \hfil\cr
{\tfx первая} & {\tfx вторая} & {\tfx третья} & {\tfx четвертая} \cr
\drawMagnitude{I} & \drawMagnitude{II} & \drawMagnitude{III} & \drawMagnitude{IV} \cr
&\drawMagnitude{V}\ &\drawMagnitude{VI}\ & \cr
}}
\protect
}
\figureInMargin{\box0}
\startCenterAlign
Пусть $\drawMagnitude{I} : \drawMagnitude{II} :: \drawMagnitude{III} : \drawMagnitude{IV}$,\\
и если \magnitudeI\ будет кратна \magnitudeII,\\
то \magnitudeIII\ будет столько же раз кратна \magnitudeIV.

Возьмем  $\magnitudeV = \magnitudeI$.

Сколько раз \magnitudeI\ кратна \magnitudeII\\
возьмем \magnitudeVI\ столько же раз кратную \magnitudeIV, \\
тогда, поскольку $\magnitudeI : \magnitudeII :: \magnitudeIII : \magnitudeIV$\\
и мы взяли равнократные второй и~четвертой, \magnitudeI\ и~\magnitudeVI,  следовательно \inprop[prop:V.IV]\\
 $\magnitudeI : \magnitudeV :: \magnitudeIII : \magnitudeVI$, но (\conststr),\\
 $\magnitudeI = \magnitudeV \therefore$ \inprop[prop:V.A] $\magnitudeIII = \magnitudeVI$\\
 и~\magnitudeVI\ столько же раз кратна \magnitudeIV\\
 сколько \magnitudeI\ кратна \magnitudeII.

Теперь, пусть $\magnitudeII : \magnitudeI :: \magnitudeIV : \magnitudeIII$,\\
а также \magnitudeII\ составляет часть \magnitudeI;\\
тогда \magnitudeIV\ будет составлять такую же часть \magnitudeIII.

Если перевернуть, $\magnitudeI : \magnitudeII :: \magnitudeIII : \magnitudeIV$ \inprop[prop:V.B],\\
но \magnitudeII\ составляет часть \magnitudeI,
то есть, \magnitudeI\ кратна \magnitudeII.

$\therefore$ как было показано в~пердыдущем случае, \magnitudeIII\ столько же раз кратна \magnitudeIV,
то есть, \magnitudeIV\ составляет такую же часть от \magnitudeIII\\
какую \magnitudeII\ от \magnitudeI.

$\therefore$ Если первая величина относится ко второй… и~т. д.
\stopCenterAlign
\stopPropositionAZ

\vfill\pagebreak

\startProposition[title={Предл. VII. Теорема}, reference=prop:V.VII]
\defineNewPicture{
byMagnitudeSymbolDefine.I("circle", byred, 0);
byMagnitudeSymbolDefine.II("rhombus", byblue, 0);
byMagnitudeSymbolDefine.III("square", byyellow, 0);
}
\problemNP{Р}{авные}{величины имеют одно отношение к~какой-либо величине, и~эта величина имеет такое же отношение к~равным величинам.}

\startCenterAlign
Пусть $\drawMagnitude{I} = \drawMagnitude{II}$\\
и \drawMagnitude{III} будет какой-либо другой величиной.

Тогда $\magnitudeI : \magnitudeIII = \magnitudeII : \magnitudeIII$ и~$\magnitudeIII : \magnitudeI = \magnitudeIII : \magnitudeII$.

Поскольку $\magnitudeI = \magnitudeII$,\\
$\therefore M \magnitudeI = M \magnitudeII$;

$\therefore$ Если $M \magnitudeI >, = \mbox{ или } < m \magnitudeIII$,\\
то $M \magnitudeII >, = \mbox{ или } < m \magnitudeIII$,\\
и $\therefore \magnitudeI : \magnitudeIII = \magnitudeII : \magnitudeIII$ \indef[def:V.V].

Из вышеизложенного очевидно, что\\
если $m \magnitudeIII >, = \mbox{ или } < M \magnitudeI$,\\
то $m \magnitudeIII >, = \mbox{ или } < M \magnitudeII$.

$\therefore \magnitudeIII : \magnitudeI = \magnitudeIII : \magnitudeII$ \indef[def:V.V].

$\therefore$ Равные величины… и~т. д.
\stopCenterAlign
\stopProposition

\vfill\pagebreak

\startDefinition[title={Определение VII.}, reference=def:V.VII,ownnumber=7]
\defineNewPicture{
byMagnitudeSymbolDefine.i("circle", byred, 0);
byMagnitudeSymbolDefine.ii("square", byyellow, 0);
byMagnitudeSymbolDefine.iii("rhombus", byblue, 0);
byMagnitudeSymbolDefine.iv("wedgeDown", byblack, 0);
byMagnitudeDefine.I(0, false)(5)(i);
byMagnitudeDefine.II(0, false)(4)(ii);
byMagnitudeDefine.III(0, false)(5)(iii);
byMagnitudeDefine.IV(0, false)(4)(iv);
byMagnitudeSymbolDefine.Ia("wedgeDown", byred, 0);
byMagnitudeSymbolDefine.IIa("semicircleDown", byblack, 1);
byMagnitudeSymbolDefine.IIIa("square", byblue, 0);
byMagnitudeSymbolDefine.IVa("rhombus", byyellow, 0);
}

Когда из равнократных четырех величин (взятых как в~пятом определении), кратное первой больше кратного второй, а~кратное третьей не больше кратного четвертой, то говорят, что первая имеет большее отношение ко второй, чем третья к~четвертой, и~наоборот, говорят что третья имеет к~четвертой меньшее отношение, чем первая ко второй.

Если из равнократных четырех величин, наподобие тех, что сравнивались в~пятом определении, мы найдем, что $\drawMagnitude{I} > \drawMagnitude{II}$, но $\drawMagnitude{III} = \mbox{ или } > \drawMagnitude{IV}$, или же мы найдем, некоторая определенная кратность $M'$ первой и~третьей, и~другая кратность $m'$ второй и~четвертой, такие, что $M'$ раз первая величина $> m'$ раз вторая, но $M'$ раз третья $\ngtr m'$ раз четвертой, т. е. $= \mbox{ или } < m'$ раз четвертой, то про первую говоря, что ее отношение к~третьей больше, чем у~третьей к~четвертой, или что третья к~четвертой в~таких обстоятельствах имеет меньшее отношение, чем первая ко второй, однако некоторые другие равнократные могут показать, что четыре величины пропорциональны.

В дальнейшем это определение будет иметь следующий вид:

\startCenterAlign
Если $M' \drawMagnitude{Ia} > m' \drawMagnitude{IIa}$, но $M' \drawMagnitude{IIIa} = \mbox{ или } < m' \drawMagnitude{IVa}$,\\
то $\magnitudeIa : \magnitudeIIa > \magnitudeIIIa : \magnitudeIVa$.
\stopCenterAlign

В приведенном выражении $M'$ и~$m'$ представляют собой определенные кратности, в~отличие от $M$ и~$m$ в~пятом определении, которые могут быть любыми кратностями. Также нужно иметь в~виду, что \magnitudeIa, \magnitudeIIa, \magnitudeIIIa, и~подобные обозначения, всего лишь представляют геометрические величины.

Арифметически это можно представить так:

Возьмем четыре числа $\color[byred]{8}$, $\color[byblack]{7}$, $\color[byblue]{10}$, и~$\color[byyellow]{9}$.

\startCenterAlign

\unprotect
\ \hfill\vbox{
\offinterlineskip
\tabskip=0pt
\halign{\vrule height1.9ex depth0.9ex
	\hfil \color[byred] { # } \hfil &
	\hfil \color[byblack] { # } \hfil &
	\hfil \color[byblue] { # } \hfil &
	\hfil \color[byyellow] { # } \hfil \vrule \cr
	\noalign{\hrule}
	\color[byblack]{первое} &
	\color[byblack]{второе} &
	\color[byblack]{третье} &
	\color[byblack]{четвертое} \cr
	8 	& 7 		& 10 	& 9 \cr
	\noalign{\hrule}
	16 	& 14 	& 20 	& 18 \cr
	24 	& 21 	& 30 	& 27 \cr
	32 	& 28 	& 40 	& 36 \cr
	40 	& 35 	& 50 	& 45 \cr
	48 	& 42 	& 60 	& 54 \cr
	56 	& 49 	& 70 	& 63 \cr
	64 	& 56 	& 80 	& 72 \cr
	72 	& 63 	& 90 	& 81 \cr
	80 	& 70 	& 100 	& 90 \cr
	88 	& 77 	& 110 	& 99 \cr
	96 	& 84 	& 120 	& 108 \cr
	104 	& 91 	& 130 	& 117 \cr
	112 	& 98 	& 140 	& 126 \cr
	\color[byblack]{ и~т. д.} &
	\color[byblack]{ и~т. д.} &
	\color[byblack]{ и~т. д.} &
	\color[byblack]{ и~т. д.} \cr
	\noalign{\hrule}
}}\hfill\
\protect
\stopCenterAlign

Среди приведенных кратностей мы найдем что $\color[byred]{16} > \color[byblack]{14}$ и~$\color[byblue]{20} > \color[byyellow]{18}$. То есть дважды первое больше дважды второго и~дважды третье больше дважды четвертого. И~$\color[byred]{16} < \color[byblack]{21}$ и~$\color[byblue]{20} < \color[byyellow]{27}$. То есть дважды первое меньше трижды второго и~дважды третье меньше трижды четвертого. И~среди тех же кратностей мы найдем что $\color[byred]{72} > \color[byblack]{56}$ и~$\color[byblue]{90} > \color[byyellow]{72}$. То есть 9 раз первое больше 8 раз второго и~9 раз третье больше 8 раз четвертого. Можно выбрать много других равнократных, которые будут показывать, что $\color[byred]{8}$, $\color[byblack]{7}$, $\color[byblue]{10}$, и~$\color[byyellow]{9}$ пропорциональны, тогда как они таковыми не являются, поскольку мы можем найти кратное первого $>$ кратного второго, но столько же раз кратное, сколько первого, третьего $\ngtr$ столько же раз кратного, сколько второго, четвертого. Например, 9 раз первое $>$ 10 раз второго, но 9 раз третье $\ngtr$ 10 раз четвертого, то есть $\color[byred]{72} > \color[byblack]{70}$, но $\color[byblue]{90} \ngtr \color[byyellow]{90}$, или 8 раз первое $>$ 9 раз второго, но 8 раз третье $\ngtr$ 9 раз четвертого, то есть $\color[byred]{64} > \color[byblack]{63}$, но $\color[byblue]{80} \ngtr \color[byyellow]{81}$. Когда можно найти такие кратные, говорят, что первое $\color[byred]{(8)}$ имеет ко второму $\color[byblack]{(7)}$ большее отношение, чем третье $\color[byblue]{(10)}$ к~четвертому $\color[byyellow]{(9)}$, и~наоборот, третье $\color[byblue]{(10)}$ имеет к~четвертому $\color[byyellow]{(9)}$ меньшее отношение, чем первое $\color[byred]{(8)}$ ко второму $\color[byblack]{(7)}$.
\stopDefinition

\vfill\pagebreak

\startProposition[title={Предл. VIII. Теорема}, reference=prop:V.VIII]
\defineNewPicture{
byMagnitudeSymbolDefine.Ia("miniTriangleUp", byblack, 0);
byMagnitudeSymbolDefine.Ib("square", byred, 0);
byMagnitudeDefine.I(0, false)(1, 1)(Ia, Ib);
byMagnitudeSymbolDefine.II("square", byyellow, 0);
byMagnitudeSymbolDefine.III("circle", byblue, 0);
}
\problemNP{И}{з}{неравных величин к~одной и~той же величине большая имеет большее отношение, чем меньшая, и~эта величина имеет большее отношение к~меньшей, чем к~большей.}

\startCenterAlign
Возьмем две неравных величины \drawMagnitude[bottom]{I} и~\drawMagnitude{II},\\
и какую-либо другую \drawMagnitude{III}.
\stopCenterAlign

Докажем, что \magnitudeI, большая из двух неравных величин, имеет большее отношение к~\magnitudeIII, чем меньшая \magnitudeII, то есть, $\magnitudeI : \magnitudeIII > \magnitudeII : \magnitudeIII$.

\startCenterAlign
Возьмем $M' \magnitudeI$, $m' \magnitudeIII$, $M' \magnitudeII$, $m' \magnitudeIII$;\\
такие, что $M' \drawMagnitude{Ia}$ и~$M' \drawMagnitude{Ib}$ обе $>$ \magnitudeIII;\\
также возьмем $m' \magnitudeIII$ наименьшее кратное \magnitudeIII,\\
при котором $m' \magnitudeIII > M' \magnitudeII = M' \magnitudeIb$.

$\therefore M' \magnitudeII \ngtr m' \magnitudeIII$,\\
но $M' \magnitudeI > m' \magnitudeIII$,\\
поскольку $m' \magnitudeIII$, первое кратное, которое $> M' \magnitudeIb$, в~то время как $(m' - 1)\magnitudeIII$ или $m' \magnitudeIII - \magnitudeIII \ngtr M' \magnitudeIa$, и $\magnitudeIII \ngtr M' \magnitudeIa$.

$\therefore m' \magnitudeIII - \magnitudeIII + \magnitudeIII$ должна быть $< M' \magnitudeIb + M' \magnitudeIa$,
то есть, $m' \magnitudeIII$ должна быть $< M' \magnitudeI$.

$\therefore M' \magnitudeI > m' \magnitudeIII$, но, как было показано выше, $M' \magnitudeII \ngtr m' \magnitudeIII$, следовательно \indef[def:V.VII]  \magnitudeI\ имеет к~\magnitudeIII\ большее отношение, чем $\magnitudeII : \magnitudeIII$.

Теперь докажем, что у~\magnitudeIII\ большее отношение к~меньшей величине \magnitudeII, чем к~большей \magnitudeI,\\
или, $\magnitudeIII : \magnitudeII > \magnitudeIII : \magnitudeI$.

Возьмем $m' \magnitudeIII$, $M' \magnitudeII$, $m' \magnitudeIII$ и~$M' \magnitudeI$,\\
как и~в первом случае такие, что\\
$M' \magnitudeIa$ и~$M' \magnitudeIb$ будут обе $> \magnitudeIII$, и~$m' \magnitudeIII$ будет наименьшим кратным \magnitudeIII, которое становится больше $M' \magnitudeIb = M' \magnitudeII$.

$\therefore m' \magnitudeIII - \magnitudeIII \ngtr M' \magnitudeIb$,\\
и $\magnitudeIII \ngtr M' \magnitudeIa$, следовательно\\
$m' \magnitudeIII - \magnitudeIII + \magnitudeIII  < M' \magnitudeIb + M' \magnitudeIa$.

$\therefore m' \magnitudeIII < M' \magnitudeI$, и~$\therefore$ \indef[def:V.VII], \magnitudeIII\ имеет к~\magnitudeII\ большее отношение, чем \magnitudeIII\ к~\magnitudeI.

$\therefore$ Из неравных величин… и~т. д.
\stopCenterAlign

Инструмент, использованный в~этом предложении для нахождения среди кратностей величин, взятых как в~пятом определении, таких, при которых певая больше кратной второй, но такая же кратная третьей, как была взята первой, не больше такой же кратной четвертой, какая была взята второй, может быть проиллюстрирована численно так:

Число $\color[byblack]{9}$ имеет большее отношение к~$\color[byblue]{7}$, чем $\color[byyellow]{8}$ к~$\color[byblue]{7}$, то есть $\color[byblack]{9} : \color[byblue]{7} > \color[byyellow]{8} : \color[byblue]{7}$, или $\color[byred]{8} + \color[byblack]{1} : \color[byblue]{7} > \color[byyellow]{8} : \color[byblue]{7}$.

Кратное $\color[byblack]{1}$, которое первым становится больше $\color[byblue]{7}$ это $8$ раз, следовательно, мы можем умножить первое и~третье на $8$, $9$, $10$ или любое другое число, в~данном случае умножим первое и~третье на $8$ и~получим $\color[byred]{64} + \color[byblack]{8}$ и~$\color[byyellow]{64}$, теперь первое кратное $\color[byblue]{7}$, которое больше $64$ это $10$ раз, теперь, умножая второе и~четвертое на $10$ получим $\color[byblue]{70}$ и~$\color[byblue]{70}$, теперь, расположив эти кратные получим

\startCenterAlign
\unprotect
\ \hfill\vbox{
\offinterlineskip\lineskip3pt
\halign{\hfil # \hfil & \hfil # \hfil & \hfil # \hfil & \hfil # \hfil\cr
{\tfx 8 раз} & {\tfx 10 раз} & {\tfx 8 раз} & {\tfx 10 раз} \cr
{\tfx первое} & {\tfx второе} & {\tfx третье} & {\tfx четвертое} \cr
$\color[byred]{64} + \color[byblack]{8}$ & $\color[byblue]{70}$ & $\color[byyellow]{64}$ & $\color[byblue]{70}$ \cr
}}\hfill\
\protect
\stopCenterAlign

Следовательно $\color[byred]{64} + \color[byblack]{8}$, или $72$ больше чем $\color[byblue]{70}$, но $\color[byyellow]{64}$ не больше чем $\color[byblue]{70}$, $\therefore$ \indef[def:V.VII] $\color[byblack]{9}$ имеет большее отношение к~$\color[byblue]{7}$, чем $\color[byyellow]{8}$ к~$\color[byblue]{7}$.

Приведенное выше — всего лишь иллюстрация к~дальнейшей демонстрации, так как это свойство можно легко показать как для этих, так и~для других чисел следующим образом: поскольку, если предыдущее содержит последующее большее число раз, чем другое предыдущее содержит свое последующее, или  составленная дробь с~предыдущим в~качестве числителя и~последующим в~качестве знаменателя больше другой дроби, составленной из другого предыдущего в~качестве числителя и~его последующего в~качестве знаменателя, отношение первого предыдущего к~его последующему больше, чем отношение второго предыдущего к~его последующему.

Так, отношение числа $9$ к~числу $7$ больше отношения числа $8$ к~числу $7$, поскольку $\frac{9}{7}$ больше чем $\frac{8}{7}$.

Теперь, $17 : 19$ большее отношение, чем $13 : 15$, поскольку $\frac{17}{19} = \frac{17 \times 15}{19 \times 15} = \frac{255}{185}$, и~$\frac{13}{15} = \frac{13 \times 19}{15 \times 19} = \frac{247}{185}$, отсюда очевидно, что $\frac{255}{185}$ больше чем $\frac{247}{185}$, $\therefore \frac{17}{19}$ больше чем $\frac{13}{15}$, и, согласно показанному выше, $17$ имеет к~$19$ большее отношение, чем $13$ к~$15$.

В общем виде это можно выразить так:

Если $\frac{A}{B}$ больше $\frac{C}{D}$, говорят, что $A$ имеет к~$B$ большее отношение, чем $C$ к~$D$. Если $\frac{A}{B}$ равно $\frac{C}{D}$, то $A$ имеет к~$B$ такое же отношение, как $C$ к~$D$. И~если $\frac{A}{B}$ меньше чем $\frac{C}{D}$, говорят, что $A$ имеет к~$B$ меньшее отношение, чем $C$ к~$D$.

Учащийся должен в~совершенстве освоить все, до этого предложения, прежде чем двигаться дальше, чтобы хорошо понять последующие предложения этой книги. Так что мы советуем учащемуся медленно перечитать все снова, тщательно осмыслить каждый шаг в~процессе, и~особо предостерегаем от порочной практики полагаться исключительно на память. Следуя этим указаниям, он найдет части, обычно представляющие значительные трудности, совсем несложными, продолжая изучение этой книги.
\stopProposition

\vfill\pagebreak

\startProposition[title={Предл. IX. Теорема}, reference=prop:V.IX]
\defineNewPicture{
byMagnitudeSymbolDefine.I("rhombus", byblue, 0);
byMagnitudeSymbolDefine.II("circle", byred, 0);
byMagnitudeSymbolDefine.III("square", byyellow, 0);
}
\problemNP{В}{еличины,}{имеющие одинаковое отношение к~одному и~тому же, равны между собой, и~те, к~которым одно и~то же имеет равные отношения, также равны.}

\startCenterAlign
Пусть $\drawMagnitude{I} : \drawMagnitude{III} :: \drawMagnitude{II} : \drawMagnitude{III}$, тогда $\magnitudeI = \magnitudeII$.

Поскольку, если нет, пусть $\magnitudeI > \magnitudeII$,\\
тогда $\magnitudeI : \magnitudeIII > \magnitudeII : \magnitudeIII$ \inprop[prop:V.VIII],
что невозможно, поскольку противоречит гипотезе.\\
$\therefore \magnitudeI \ngtr \magnitudeII$.

Таким же образом можно показать, что\\
$\magnitudeII \ngtr \magnitudeI$,\\
$\therefore \magnitudeI = \magnitudeII$.

Теперь, пусть $\magnitudeIII : \magnitudeI :: \magnitudeIII : \magnitudeII$,\\
тогда $\magnitudeI = \magnitudeII$.

Поскольку (перевернув) $\magnitudeI : \magnitudeIII :: \magnitudeII : \magnitudeIII$,\\
следовательно, как и~впервом случае, $\magnitudeI = \magnitudeII$.

$\therefore$ Величины, имеющие одинаковое отношение... и~т. д.
\stopCenterAlign

Иначе это можно показать так:

Пусть $\color[byyellow]{A} : \color[byblue]{B} = \color[byyellow]{A} : \color[byred]{C}$, тогда $\color[byblue]{B} = \color[byred]{C}$, поскольку, в~виде дробей $\frac{\color[byyellow]{A}}{\color[byblue]{B}} = \frac{\color[byyellow]{A}}{\color[byred]{C}}$, и~числитель одного равен числителю другого, следовательно, знаменатели этих дробей равны, то есть $\color[byblue]{B} = \color[byred]{C}$.

Теперь, если $\color[byblue]{B} : \color[byyellow]{A} = \color[byred]{C} : \color[byyellow]{A}$, $\color[byblue]{B} = \color[byred]{C}$. Поскольку $\frac{\color[byblue]{B}}{\color[byyellow]{A}} = \frac{\color[byred]{C}}{\color[byyellow]{A}}$, $\color[byblue]{B} \mbox{ должна быть } = \color[byred]{C}$.
\stopProposition

\vfill\pagebreak

\startProposition[title={Предл. X. Теорема}, reference=prop:V.X]
\defineNewPicture{
byMagnitudeSymbolDefine.I("wedgeDown", byblue, 0);
byMagnitudeSymbolDefine.II("circle", byred, 0);
byMagnitudeSymbolDefine.III("square", byyellow, 0);
}
\problemNP{И}{з}{величин, имеющих отношение к~одному и~тому же, больше та, которая имеет большее отношение, а~та, к~которой одно и~то же имеет большее отношение, является меньшей из двух.}

\startCenterAlign
Пусть $\drawMagnitude{I} : \drawMagnitude{III} > \drawMagnitude{II} : \drawMagnitude{III}$, тогда $\magnitudeI > \magnitudeII$.

Поскольку, если нет, пусть $\magnitudeI = \mbox{ или } < \magnitudeII$,\\
тогда $\magnitudeI : \magnitudeIII = \magnitudeII : \magnitudeIII$ \inprop[prop:V.VII]\\
или $\magnitudeI : \magnitudeIII < \magnitudeII : \magnitudeIII$ \inprop[prop:V.VIII] и~(перевернув), что противоречит гипотезе.

$\therefore \magnitudeI \neq \mbox{ или } < \magnitudeII$,\\
и $\therefore \magnitudeI \mbox{ должна быть } >\magnitudeII$.

Теперь, пусть $\magnitudeIII : \magnitudeII > \magnitudeIII : \magnitudeI$,\\
тогда $\magnitudeII < \magnitudeI$.

Поскольку, если нет, $\magnitudeII \mbox{ должна быть } > \mbox{ или } = \magnitudeI$,\\
тогда $\magnitudeIII : \magnitudeII < \magnitudeIII : \magnitudeI$ \inprop[prop:V.VIII] и~(перевернув),\\
или $\magnitudeIII : \magnitudeII = \magnitudeIII : \magnitudeI$ \inprop[prop:V.VII], что противоречит гипотезе.

$\therefore \magnitudeII \ngtr \mbox{ или } = \magnitudeI$,\\
и $\therefore \magnitudeII \mbox{ должна быть } < \magnitudeI$.

$\therefore$ Из величин, имеющих… и~т. д.
\stopCenterAlign
\stopProposition

\vfill\pagebreak

\startProposition[title={Предл. XI. Теорема}, reference=prop:V.XI]
\defineNewPicture{
byMagnitudeSymbolDefine.I("rhombus", byblue, 0);
byMagnitudeSymbolDefine.II("square", byblue, 0);
byMagnitudeSymbolDefine.III("miniTriangleUp", byblack, 0);
byMagnitudeSymbolDefine.IV("miniCircle", byblack, 0);
byMagnitudeSymbolDefine.V("circle", byred, 0);
byMagnitudeSymbolDefine.VI("wedgeDown", byyellow, 0);
}
\problemNP{О}{тношения,}{тождественные одному и~тому же, тождественны и~друг другу.}

\startCenterAlign
Пусть $\drawMagnitude{I} : \drawMagnitude{II} = \drawMagnitude{V} : \drawMagnitude{VI}$ и~$\magnitudeV : \magnitudeVI = \drawMagnitude{III} : \drawMagnitude{IV}$,\\
тогда $\magnitudeI : \magnitudeII = \magnitudeIII : \magnitudeIV$.

Поскольку если $M \magnitudeI >, = \mbox{ или } < m \magnitudeII$,\\
то $M \magnitudeV >, = \mbox{ или } < m \magnitudeVI$,\\
и если $M \magnitudeV >, = \mbox{ или } < m \magnitudeVI$,\\
то $M \magnitudeIII >, = \mbox{ или } < m \magnitudeIV$ \indef[def:V.V].

$\therefore$ if $M \magnitudeI >, = \mbox{ или } < m \magnitudeII$, $M \magnitudeIII >, = \mbox{ или } < m \magnitudeIV$,\\
и $\therefore$ \indef[def:V.V] $\magnitudeI : \magnitudeII = \magnitudeIII : \magnitudeIV$.

$\therefore$ Отношения, тождественные между собой… и~т. д.
\stopCenterAlign
\stopProposition

\vfill\pagebreak

\startProposition[title={Предл. XII. Теорема}, reference=prop:V.XII]
\defineNewPicture{
byMagnitudeSymbolDefine.I("square", byred, 0);
byMagnitudeSymbolDefine.II("circle", byred, 0);
byMagnitudeSymbolDefine.III("semicircleDown", byblack, 1);
byMagnitudeSymbolDefine.IV("sectorUp", byblack, 1);
byMagnitudeSymbolDefine.V("rhombus", byyellow, 0);
byMagnitudeSymbolDefine.VI("wedgeDown", byyellow, 0);
byMagnitudeSymbolDefine.VII("miniCircle", byblue, 0);
byMagnitudeSymbolDefine.VIII("miniTriangleDown", byblue, 0);
byMagnitudeSymbolDefine.IX("miniTriangleUp", byblack, 0);
byMagnitudeSymbolDefine.X("miniCircle", byblack, 0);
}
\problemNP{Е}{сли}{несколько величин пропорциональны, то как одна из предыдущих будет относиться к~последующей, так и~все вместе предыдущие будут относиться ко всем вместе последующим.}

\startCenterAlign
Пусть $\drawMagnitude{I} : \drawMagnitude{II} =
\drawMagnitude{III} : \drawMagnitude{IV} =
\drawMagnitude{V} : \drawMagnitude{VI} =
\drawMagnitude{VII} : \drawMagnitude{VIII} =
\drawMagnitude{IX} : \drawMagnitude{X}$,\\
тогда $\magnitudeI : \magnitudeII =
\magnitudeI + \magnitudeIII + \magnitudeV + \magnitudeVII + \magnitudeIX :
\magnitudeII + \magnitudeIV + \magnitudeVI + \magnitudeVIII + \magnitudeX$.

Поскольку если $M \magnitudeI > m \magnitudeII$, то $M \magnitudeIII > m \magnitudeIV$,\\
и $M \magnitudeV > m \magnitudeVI$, $M \magnitudeVII > m \magnitudeVIII$,\\
а также $M \magnitudeIX > m \magnitudeX$ \indef[def:V.V].

Следовательно, если $M \magnitudeI > m \magnitudeII$,\\
то $M \magnitudeI + M \magnitudeIII + M \magnitudeV + M \magnitudeVII + M \magnitudeIX$, или $M (\magnitudeI + \magnitudeIII + \magnitudeV + \magnitudeVII + \magnitudeIX)$ будет больше, чем $m \magnitudeII + m \magnitudeIV + m \magnitudeVI + m \magnitudeVIII + m \magnitudeX$, или $m (\magnitudeII + \magnitudeIV + \magnitudeVI + \magnitudeVIII + \magnitudeX)$.
\stopCenterAlign

Так же можно показать, что если $M$ раз одно из предыдущих равно или меньше чем $m$ раз одно из последующих, $M$ раз все предыдущие будут равны или меньше, чем $m$ раз все последующие вместе. Следовательно \indef[def:V.V], как одно из предыдущих к~последующему, так и~все предыдущие ко всем последующим вместе.

$\therefore$ Если несколько величин… и~т. д.
\stopProposition

\vfill\pagebreak

\startProposition[title={Предл. XIII. Теорема}, reference=prop:V.XIII]
\defineNewPicture{
byMagnitudeSymbolDefine.I("wedgeDown", byblue, 0);
byMagnitudeSymbolDefine.II("semicircleDown", byblue, 1);
byMagnitudeSymbolDefine.III("square", byred, 0);
byMagnitudeSymbolDefine.IV("rhombus", byyellow, 0);
byMagnitudeSymbolDefine.V("sectorUp", byblack, 1);
byMagnitudeSymbolDefine.VI("circle", byblack, 0);
}
\problemNP{Е}{сли}{первая величина ко второй имеет такое же отношение, как третья к~четвертой, а~третья к~четвертой имеет большее отношение, чем пятая к~шестой, то и~первая ко второй будет иметь большее отношение, чем пятая к~шестой.}

\startCenterAlign
Пусть $\drawMagnitude{I} : \drawMagnitude{II} = \drawMagnitude{III} : \drawMagnitude{IV}$,\\
но $\magnitudeIII : \magnitudeIV > \drawMagnitude{V} : \drawMagnitude{VI}$,\\
тогда $\magnitudeI : \magnitudeII > \magnitudeV : \magnitudeVI$.

Поскольку $\magnitudeIII : \magnitudeIV > \magnitudeV : \magnitudeVI$, есть такие кратные ($M'$ и~$m'$) величин \magnitudeIII\ и~\magnitudeV, и~величин \magnitudeIV\ и~\magnitudeVI,\\
что $M' \magnitudeIII > m' \magnitudeIV$,\\
но $M' \magnitudeV \ngtr m' \magnitudeVI$ \indef[def:V.VII].

Возьмем эти кратные и~возьмем такие же кратные \magnitudeI\ и~\magnitudeII.

$\therefore$ \indef[def:V.V] если $M' \magnitudeI >, =, \mbox{ или } < m' \magnitudeII$,\\
то $M' \magnitudeIII >, =, \mbox{ и~} < m' \magnitudeIV$,\\
но $M' \magnitudeIII > m' \magnitudeIV$ (\conststr);

$\therefore M' \magnitudeI > m' \magnitudeII$,\\
но $M' \magnitudeV \ngtr m' \magnitudeVI$ (\conststr)

И, следовательно, \indef[def:V.VII],\\
$\magnitudeI : \magnitudeII > \magnitudeV : \magnitudeVI$.

$\therefore$ Если первая величина ко второй… и~т. д.
\stopCenterAlign
\stopProposition

\vfill\pagebreak

\startProposition[title={Предл. XIV. Теорема}, reference=prop:V.XIV]
\defineNewPicture{
byMagnitudeSymbolDefine.I("wedgeDown", byred, 0);
byMagnitudeSymbolDefine.II("semicircleDown", byblack, 1);
byMagnitudeSymbolDefine.III("square", byyellow, 0);
byMagnitudeSymbolDefine.IV("rhombus", byblue, 0);
}
\problemNP{Е}{сли}{первая величина имеет ко второй такое же отношение, какое третья к~четвертой, то если первая больше третьей, то и~вторая будет больше четвертой, а~если равна, равна, а~если меньше, то меньше.}

\startCenterAlign
Пусть $\drawMagnitude{I} : \drawMagnitude{II} :: \drawMagnitude{III} : \drawMagnitude{IV}$,\\
и предположим, что $\magnitudeI > \magnitudeIII$, тогда $\magnitudeII > \magnitudeIV$.

Поскольку $\magnitudeI : \magnitudeII > \magnitudeIII : \magnitudeII$ \inprop[prop:V.VIII], и, согласно гипотезе, $\magnitudeI : \magnitudeII = \magnitudeIII : \magnitudeIV$.

$\therefore \magnitudeIII : \magnitudeIV > \magnitudeIII : \magnitudeII$ \inprop[prop:V.XIII].

$\therefore \magnitudeIV < \magnitudeII$ \inprop[prop:V.X], или $\magnitudeII > \magnitudeIV$.

Теперь, пусть $\magnitudeI = \magnitudeIII$, тогда $\magnitudeII = \magnitudeIV$.

Поскольку $\magnitudeI : \magnitudeII = \magnitudeIII : \magnitudeII$ \inprop[prop:V.VII],\\
и $\magnitudeI : \magnitudeII = \magnitudeIII : \magnitudeIV$ (\hypstr);\\
$\therefore \magnitudeIII : \magnitudeII = \magnitudeIII : \magnitudeIV$ \inprop[prop:V.XI],\\
и $\therefore \magnitudeII = \magnitudeIV$ \inprop[prop:V.IX].

Теперь же, если $\magnitudeI < \magnitudeIII$, то $\magnitudeII < \magnitudeIV$,\\
поскольку $\magnitudeIII > \magnitudeI$ и~$\magnitudeIII : \magnitudeIV = \magnitudeI : \magnitudeII$.

$\therefore \magnitudeIV > \magnitudeII$, как и~в предыдущем случае,\\
то есть $\magnitudeII < \magnitudeIV$.

$\therefore$ Если первая величина имеет ко второй… и~т. д.
\stopCenterAlign
\stopProposition

\vfill\pagebreak

\startProposition[title={Предл. XV. Теорема}, reference=prop:V.XV]
\defineNewPicture{
byMagnitudeSymbolDefine.I("circle", byred, 0);
byMagnitudeSymbolDefine.II("square", byyellow, 0);
}
\problemNP{В}{еличины}{относятся друг к~другу так же, как и~их равнократные.}

\startCenterAlign
Пусть будут две величины \drawMagnitude{I} и~\drawMagnitude{II},\\
тогда $\magnitudeI : \magnitudeII :: M' \magnitudeI : M' \magnitudeII$.

$\eqalign{
\mbox{Поскольку } \magnitudeI : \magnitudeII &= \magnitudeI : \magnitudeII \cr
&= \magnitudeI : \magnitudeII \cr
&= \magnitudeI : \magnitudeII \cr
}$

$\therefore \magnitudeI : \magnitudeII :: 4 \magnitudeI : 4 \magnitudeII$. \inprop[prop:V.XII].

И, поскольку те же рассуждения применимы в~общем, получим:\\
$\magnitudeI : \magnitudeII :: M' \magnitudeI : M' \magnitudeII$.

$\therefore$ Величины относятся друг к~другу… и~т. д.
\stopCenterAlign
\stopProposition

\vfill\pagebreak

\startDefinition[title={Определение XIII},reference=def:V.XIII,ownnumber=13]
\defineNewPicture{
byMagnitudeSymbolDefine.I("circle", byyellow, 0);
byMagnitudeSymbolDefine.II("rhombus", byblack, 0);
byMagnitudeSymbolDefine.III("wedgeDown", byred, 0);
byMagnitudeSymbolDefine.IV("square", byblue, 0);
}
Отношение называют переставленным, когда есть четыре пропорциональных величины, и~утверждается, что первая относится к~третьей так же, как вторая к~четвертой, как показано в~следующем предложении:

\startCenterAlign
Пусть $\drawMagnitude{I} : \drawMagnitude{II} :: \drawMagnitude{III} : \drawMagnitude{IV}$,\\
тогда \quotation{переставив} отношение можем заключить,\\
что $\magnitudeI : \magnitudeIII :: \magnitudeII : \magnitudeIV$.
\stopCenterAlign

Здесь важно отметить, что величины \magnitudeI, \magnitudeII, \magnitudeIII, \magnitudeIV\ должны быть однородными, то есть в~таком случае мы должны сравнивать линии с~линиями, поверхности с~поверхностями, тела с~телам и~т. п. Так, учащийся должен хорошо усвоить, что линия с~поверхностью, поверхность с~телом и~другие разнородные величины не могут находиться в~положении предыдущего и~последующего.
\stopDefinition

\vfill\pagebreak

\startProposition[title={Предл. XVI. Теорема}, reference=prop:V.XVI]
\defineNewPicture{
byMagnitudeSymbolDefine.I("wedgeDown", byred, 0);
byMagnitudeSymbolDefine.II("semicircleDown", byblack, 1);
byMagnitudeSymbolDefine.III("square", byyellow, 0);
byMagnitudeSymbolDefine.IV("rhombus", byblue, 0);
}
\problemNP{Е}{сли}{четыре величины пропорциональны, то они останутся пропорциональными в~переставленном порядке.}

\startCenterAlign
Пусть $\drawMagnitude{I} : \drawMagnitude{II} :: \drawMagnitude{III} : \drawMagnitude{IV}$,\\
тогда $\magnitudeI : \magnitudeIII :: \magnitudeII : \magnitudeIV$.

Поскольку $M \magnitudeI : M \magnitudeII :: \magnitudeI : \magnitudeII$ \inprop[prop:V.XV],\\
и $M \magnitudeI : M \magnitudeII :: \magnitudeIII : \magnitudeIV$ (\hypstr и~\inpropL[prop:V.XI]),\\
а также $m \magnitudeIII : m \magnitudeIV :: \magnitudeIII : \magnitudeIV$ \inprop[prop:V.XV].

$\therefore M \magnitudeI : M \magnitudeII :: m \magnitudeIII : m \magnitudeIV$ \inprop[prop:V.XIV],\\
и $\therefore$ если $M \magnitudeI >, = \mbox { или } < m \magnitudeIII$,\\
то $M \magnitudeII >, =, \mbox{ или } < m \magnitudeIV$ \inprop[prop:V.XIV].

следовательно \indef[def:V.V],\\
 $\magnitudeI : \magnitudeIII :: \magnitudeII : \magnitudeIV$.

 $\therefore$ Если четыре величины… и~т. д.
\stopCenterAlign
\stopProposition

\vfill\pagebreak

\startDefinition[title={Определение XVI},reference=def:V.XVI,ownnumber=16]
\def\varA{\color[byred]{A}}
\def\varB{\color[byblack]{B}}
\def\varC{\color[byblue]{C}}
\def\varD{\color[byyellow]{D}}
Выделением отношения называют, когда есть четыре пропорциональных величины, и~утверждается, что избыток первой над второй относится ко второй как избыток третьей над четвертой к~четвертой.

\startCenterAlign
Пусть $\varA : \varB :: \varC : \varD$,\\
\quotation{выделив} заключим,\\
что $\varA - \varB : \varB :: \varC - \varD : \varD$.
\stopCenterAlign

Согласно описанному выше, предполагается, что $\varA$ больше $\varB$ и~$\varC$ больше $\varD$, если бы это было не так и~$\varB$ была бы больше $\varA$, и~$\varD$ больше $\varC$, $\varB$ и~$\varD$ можно было бы взять как предыдущие, а~$\varA$ и~$\varC$ как последующие, \quotation{перевернув} отношение.

\startCenterAlign
$\varB : \varA :: \varD : \varC$;\\
тогда \quotation{выделив,} заключим,\\
что $\varB - \varA : \varA :: \varD - \varC : \varC$.
\stopCenterAlign
\stopDefinition

\vfill\pagebreak

\startProposition[title={Предл. XVII. Теорема}, reference=prop:V.XVII]
\defineNewPicture{
byMagnitudeSymbolDefine.I("wedgeDown", byred, 0);
byMagnitudeSymbolDefine.II("semicircleDown", byblack, 1);
byMagnitudeSymbolDefine.III("square", byyellow, 0);
byMagnitudeSymbolDefine.IV("rhombus", byblue, 0);
}
\problemNP{Е}{сли}{величины пропорциональны присоединенные, то они останутся пропорциональны взятые раздельно, то есть, если две величины взятые вместе относятся к~одной из них также, как две другие взятые вместе к~одной из тех, оставшаяся величина из первых двух будет иметь такое же отношение к~другой такое же отношение, как оставшаяся из вторых к~другой из них.}

\startCenterAlign
Пусть $\drawMagnitude{I} + \drawMagnitude{II} : \magnitudeII :: \drawMagnitude{III} + \drawMagnitude{IV} : \magnitudeIV$,\\
тогда $\magnitudeI : \magnitudeII :: \magnitudeIII : \magnitudeIV$.

Возьмем $\magnitudeI > m \magnitudeII$ к~каждой добавим $M \magnitudeII$,\\
тогда получим $M \magnitudeI + M \magnitudeII > m \magnitudeII + M \magnitudeII$,\\
или $M (\magnitudeI + \magnitudeII) > (m + M) \magnitudeII$,\\
но, поскольку $\magnitudeI + \magnitudeII: \magnitudeII :: \magnitudeIII + \magnitudeIV: \magnitudeIV$ (\hypstr),\\
и $M (\magnitudeI + \magnitudeII) > (m + M) \magnitudeII$.

$\therefore M(\magnitudeIII + \magnitudeIV) > (m + M) \magnitudeIV$ \indef[def:V.V].

$\therefore M \magnitudeIII + M \magnitudeIV > m \magnitudeIV + M \magnitudeIV$.

$\therefore M \magnitudeIII > m \magnitudeIV$, вычтя $M \magnitudeIV$ из обеих сторон,\\
то есть когда $M \magnitudeI > m \magnitudeII$, тогда $M \magnitudeIII > m \magnitudeIV$.

Так же можно доказать, что если $M \magnitudeI = \mbox{ или } < m \magnitudeII$, то $M \magnitudeIII = \mbox{ или } < m \magnitudeIV$,\\
и $\therefore \magnitudeI : \magnitudeII :: \magnitudeIII : \magnitudeIV$ \indef[def:V.V].

$\therefore$ Если величины пропорциональны составленные… и~т. д.
\stopCenterAlign
\stopProposition

\vfill\pagebreak

\startDefinition[title={Определение XV},reference=def:V.XV,ownnumber=15]
\def\varA{\color[byred]{A}}
\def\varB{\color[byblack]{B}}
\def\varC{\color[byyellow]{C}}
\def\varD{\color[byblue]{D}}
О присоединении говорят, когда есть четыре пропорциональных величины и~утверждается, что первая вместе со второй относится ко второй, как третья вместе с~четвертой к~четвертой.

\startCenterAlign
Пусть $\varA : \varB :: \varC : \varD$

Тогда используя \quotation{присоединение}, заключим,\\
что $\varA + \varB : \varB :: \varC + \varD: \varD$.
\stopCenterAlign

 \quotation{Перевернув} отношение, можем сделать $\varB$ и~$\varD$ первой и~третьей, а~$\varA$ и~$\varC$ которой и~четвертой.

\startCenterAlign
$\varB : \varA :: \varD : \varC$.

Тогда, совершив \quotation{присоединение}, заключим\\
что $\varB + \varA : \varA :: \varD + \varC : \varC$.
\stopCenterAlign
\stopDefinition

\vfill\pagebreak

\startProposition[title={Предл. XVIII. Теорема}, reference=prop:V.XVIII]
\defineNewPicture{
byMagnitudeSymbolDefine.I("wedgeDown", byred, 0);
byMagnitudeSymbolDefine.II("semicircleDown", byblack, 1);
byMagnitudeSymbolDefine.III("square", byyellow, 0);
byMagnitudeSymbolDefine.IV("rhombus", byblue, 0);
byMagnitudeSymbolDefine.V("circle", byblack, 0);
}
\problemNP{Е}{сли}{величины пропорциональны выделенные, то они будут пропорциональны и~составленные. То есть, если первая относится ко второй, как третья к~четвертой, то первая и~вторая вместе будут ко второй, как третья и~четвертая вместе к~четвертой.}

\startCenterAlign
Пусть $\drawMagnitude{I} : \drawMagnitude{II} :: \drawMagnitude{III} : \drawMagnitude{IV}$,\\
тогда $\magnitudeI + \magnitudeII : \magnitudeII :: \magnitudeIII + \magnitudeIV : \magnitudeIV$.

Ведь если нет, то $\magnitudeI + \magnitudeII : \magnitudeII :: \magnitudeIII + \drawMagnitude{V} : \magnitudeV$,\\
полагая, что $\magnitudeV \neq \magnitudeIV$.

$\therefore \magnitudeI : \magnitudeII :: \magnitudeIII : \magnitudeV$ \inprop[prop:V.XVII]\\
но $\magnitudeI : \magnitudeII :: \magnitudeIII : \magnitudeIV$ (\hypstr).

$\therefore \magnitudeIII : \magnitudeV :: \magnitudeIII : \magnitudeIV$ \inprop[prop:V.XI].

$\therefore \magnitudeV = \magnitudeIV$ \inprop[prop:V.IX],\\
что противоречит предположению.

$\therefore \magnitudeV \mbox{ не неравна } \magnitudeIV$;\\
то есть $\magnitudeV = \magnitudeIV$.

$\therefore \magnitudeI + \magnitudeII : \magnitudeII :: \magnitudeIII + \magnitudeIV : \magnitudeIV$.

$\therefore$ Если величины пропорциональны выделенные… и~т. д.
\stopCenterAlign
\stopProposition

\vfill\pagebreak

\startProposition[title={Предл. XIX. Теорема}, reference=prop:V.XIX]
\defineNewPicture{
byMagnitudeSymbolDefine.I("wedgeDown", byred, 0);
byMagnitudeSymbolDefine.II("semicircleDown", byblack, 1);
byMagnitudeSymbolDefine.III("square", byblue, 0);
byMagnitudeSymbolDefine.IV("rhombus", byyellow, 0);
}
\problemNP{Е}{сли}{целая величина относится к~целой так же, как вычтенная из первой к~вычтенной из второй, то и~остаток будет относиться как остатку как целая к~целой.}

\startCenterAlign
Пусть $\drawMagnitude{I} + \drawMagnitude{II} : \drawMagnitude{III} + \drawMagnitude{IV} :: \magnitudeI : \magnitudeIII$,\\
тогда $\magnitudeII : \magnitudeIV :: \magnitudeI + \magnitudeII : \magnitudeIII + \magnitudeIV$.

Поскольку $\magnitudeI + \magnitudeII : \magnitudeI :: \magnitudeIII + \magnitudeIV : \magnitudeIII$ (перестав.).

$\therefore \magnitudeII : \magnitudeI :: \magnitudeIV : \magnitudeIII$ (выдел.).

Теперь $\magnitudeII : \magnitudeIV :: \magnitudeI : \magnitudeIII$ (перестав.).

Но $\magnitudeI + \magnitudeII : \magnitudeIII + \magnitudeIV :: \magnitudeI : \magnitudeIII$ (\hypstr).

$\therefore \magnitudeII : \magnitudeIV :: \magnitudeI + \magnitudeII : \magnitudeIII + \magnitudeIV$ \inprop[prop:V.XI].

$\therefore$ Если целая величина относится к~целой… и~т. д.
\stopCenterAlign
\stopProposition

\startDefinition[title={Определение XVII},reference=def:V.XVII,ownnumber=17] %у Мордухай-Болтовского это называется «переворачиванием отношения», при том, что уже есть «перевернутое отношение» и~это совсем другое, так что вместо convertendo здесь будет пока «конверсия», чтобы совсем не запутаться.
О \quotation{конверсии},  говорят, когда есть четыре пропорциональных величины и~утверждается, что первая к~избытку первой над второй относится также, как третья к~избытку третьей над четвертой. См. следующее предложение.
\stopDefinition

\vfill\pagebreak

\startPropositionAZ[title={Предл. E. Теорема}, reference=prop:V.E]
\defineNewPicture{
byMagnitudeSymbolDefine.i("circle", byblue, 0);
byMagnitudeSymbolDefine.II("sectorUp", byblack, 1);
byMagnitudeSymbolDefine.iii("square", byred, 0);
byMagnitudeSymbolDefine.IV("rhombus", byyellow, 0);
byMagnitudeDefine.I(0, true)(1, 1)(i, II);
byMagnitudeDefine.III(0, true)(1, 1)(iii, IV);
}
\problemNP{Е}{сли}{четыре величины пропорциональны, они также пропорциональны если совершить конверсию, то есть первая к~избытку над второй относится также как третья к~ее избытку над четвертой.}

\startCenterAlign
Пусть $\drawMagnitude{I} : \drawMagnitude{II} :: \drawMagnitude{III} : \drawMagnitude{IV}$,\\
тогда $\magnitudeI : \drawMagnitude{i} :: \magnitudeIII : \drawMagnitude{iii}$.

Поскольку $\magnitudeI : \magnitudeII :: \magnitudeIII : \magnitudeIV$;\\
следовательно $\magnitudei : \magnitudeII :: \magnitudeiii : \magnitudeIV$ (выдел.).

$\therefore \magnitudeII : \magnitudei :: \magnitudeIV : \magnitudeiii$ (перевер.).

$\therefore \magnitudeI : \magnitudei :: \magnitudeIII : \magnitudeiii$ (конвер.).

$\therefore$ Если четыре величины… и~т. д.
\stopCenterAlign
\stopPropositionAZ

\startDefinition[title={Определение XVIII},reference=def:V.XVIII,ownnumber=18]
Говорят, \quotation{По равенству}, когда есть более двух величин и~столько же других величин таких, что они пропорциональны, есть брать по две из каждого ряда, и~утверждается, что первая к~последней из первого ряда величин относится так же, как первая к~последней из второго ряда.
\stopDefinition

\vfill\pagebreak

\startDefinition[title={Определение XIX},reference=def:V.XIX,ownnumber=19]
\def\varA{\color[byred]{A}}
\def\varB{\color[byred]{B}}
\def\varC{\color[byyellow]{C}}
\def\varD{\color[byyellow]{D}}
\def\varE{\color[byblue]{E}}
\def\varF{\color[byblue]{F}}
\def\varL{\color[byred]{L}}
\def\varM{\color[byred]{M}}
\def\varN{\color[byyellow]{N}}
\def\varO{\color[byyellow]{O}}
\def\varP{\color[byblue]{P}}
\def\varQ{\color[byblue]{Q}}
Термин \quotation{по равенству} используется сам по себе, когда первая величина относится ко второй из первого ряда как первая ко второй из второго ряда и~когда вторая к~третьей из первого относится как вторая к~третьей из второго и~так далее, и~делается утверждение, такое, как упомянуто в~предыдущем определении.

\startCenterAlign
Таким образом, если есть два ряда величин,\\
$\varA, \varB, \varC, \varD, \varE, \varF$, первый,\\
и $\varL, \varM, \varN, \varO, \varP, \varQ$, второй,\\
таких, что $\varA : \varB :: \varL : \varM$, $\varB : \varC :: \varM : \varB$, $\varC : \varD :: \varN : \varO$, $\varD : \varE :: \varO : \varP$, $\varE : \varF :: \varP : \varQ$.

Мы используем понятие \quotation{по равенству}, чтобы заключить, что $\varA : \varF :: \varL : \varQ$.
\stopCenterAlign
\stopDefinition

\vfill\pagebreak

\startDefinition[title={Определение XX},reference=def:V.XX,ownnumber=20]
\def\varA{\color[byred]{A}}
\def\varB{\color[byred]{B}}
\def\varC{\color[byblue]{C}}
\def\varD{\color[byblue]{D}}
\def\varE{\color[byyellow]{E}}
\def\varF{\color[byyellow]{F}}
\def\varL{\color[byyellow]{L}}
\def\varM{\color[byyellow]{M}}
\def\varN{\color[byblue]{N}}
\def\varO{\color[byblue]{O}}
\def\varP{\color[byred]{P}}
\def\varQ{\color[byred]{Q}}
\quotation{По равенству в~перемешанной пропорции} говорят, когда первая величина ко второй в~первом ряду относится как последняя к~предпоследней во втором ряду, вторая к~третьей в~первом ряду как пред-предпоследняя к~предпоследней во втором, третья к~четвертой в~первом как четвертая с~конца к~пред-предпоследней во втором и~так далее, и~утверждается то же, что и~в \indefL[def:V.XVIII]. Это иллюстрируется в~\inpropL[prop:V.XXIII].

\startCenterAlign
Так, если есть два ряда величин,\\
$\varA, \varB, \varC, \varD, \varE, \varF$, первый,\\
и $\varL, \varM, \varN, \varO, \varP, \varQ$, второй,\\
таких, что $\varA : \varB :: \varP : \varQ$, $\varB : \varC :: \varO : \varP$, $\varC : \varD :: \varN : \varO$, $\varD : \varE :: \varM : \varN$, $\varE : \varF :: \varL : \varM$.

Используя понятие \quotation{по равенству в~перемешанной пропорции} мы утверждаем, что $\varA : \varF :: \varL : \varQ$
\stopCenterAlign
\stopDefinition

\vfill\pagebreak

\startProposition[title={Предл. XX. Теорема}, reference=prop:V.XX]
\defineNewPicture{
byMagnitudeSymbolDefine.I("wedgeDown", byblue, 0);
byMagnitudeSymbolDefine.II("semicircleDown", byred, 1);
byMagnitudeSymbolDefine.III("square", byyellow, 0);
byMagnitudeSymbolDefine.IV("rhombus", byblue, 0);
byMagnitudeSymbolDefine.V("sectorUp", byred, 1);
byMagnitudeSymbolDefine.VI("circle", byyellow, 0);
}
\problemNP{Е}{сли}{есть три величины и~другие три, которые, взятые попарно, будут иметь те же отношения, то если первая больше третьей, то и~четвертая будет больше шестой, если равны, то равны, а~если меньше, то меньше.}

\startCenterAlign
Пусть \drawMagnitude{I}, \drawMagnitude{II}, \drawMagnitude{III}, будут первыми тремя величинами,\\
а \drawMagnitude{IV}, \drawMagnitude{V}, \drawMagnitude{VI}, другими тремя,\\
такими, что $\magnitudeI : \magnitudeII :: \magnitudeIV : \magnitudeV$, и~$\magnitudeII : \magnitudeIII :: \magnitudeV : \magnitudeVI$.

Тогда если $\magnitudeI >, =, \mbox{ или } < \magnitudeIII$,\\
то $\magnitudeIV >, =, \mbox{ или } < \magnitudeVI$.

Исходя из гипотезы и~переставив получим\\
$\magnitudeI : \magnitudeIV :: \magnitudeII : \magnitudeV$,\\
и $\magnitudeII : \magnitudeV :: \magnitudeIII : \magnitudeVI$.

$\therefore \magnitudeI : \magnitudeIV :: \magnitudeIII : \magnitudeVI$ \inprop[prop:V.XI].

$\therefore$ если $\magnitudeI >, =, \mbox{ или } < \magnitudeIII$,\\
тогда $\magnitudeIV >, =, \mbox{ или } < \magnitudeVI$ \inprop[prop:V.XIV].

$\therefore$ Если есть три величины… и~т. д.
\stopCenterAlign
\stopProposition

\vfill\pagebreak

\startProposition[title={Предл. XXI. Теорема}, reference=prop:V.XXI]
\defineNewPicture{
byMagnitudeSymbolDefine.I("wedgeDown", byyellow, 0);
byMagnitudeSymbolDefine.II("wedgeUp", byred, 0);
byMagnitudeSymbolDefine.III("square", byblue, 0);
byMagnitudeSymbolDefine.IV("rhombus", byblue, 0);
byMagnitudeSymbolDefine.V("sectorUp", byred, 1);
byMagnitudeSymbolDefine.VI("circle", byyellow, 0);
}
\problemNP{Е}{сли}{есть три величины и~другие три, с~одинаковыми отношениями, если брать по две из каждого ряда, но в~противоположном порядке, тогда, если первая величина больше третьей, четвертая будет больше шестой, если равны, то равны, а~если меньше, то меньше.}

\startCenterAlign
Пусть \drawMagnitude{I}, \drawMagnitude{II}, \drawMagnitude{III}, будут первыми тремя величинами,\\
а \drawMagnitude{IV}, \drawMagnitude{V}, \drawMagnitude{VI}, другими тремя,\\
такими, что $\magnitudeI : \magnitudeII :: \magnitudeV : \magnitudeVI$,\\
и $\magnitudeII : \magnitudeIII :: \magnitudeIV : \magnitudeV$.

Тогда, если $\magnitudeI >, =, \mbox{ или } < \magnitudeIII$,\\
то $\magnitudeIV >, =, \mbox{ или } < \magnitudeVI$.

Допустим $\magnitudeI > \magnitudeIII$,\\
тогда, поскольку \magnitudeII\ любая другая величина,\\
$\magnitudeI : \magnitudeII > \magnitudeIII : \magnitudeII$ \inprop[prop:V.VIII].

Но $\magnitudeV : \magnitudeVI :: \magnitudeI : \magnitudeII$ (\hypstr).

$\therefore \magnitudeV : \magnitudeVI > \magnitudeIII : \magnitudeII$ \inprop[prop:V.XIII],\\
и, поскольку $\magnitudeII : \magnitudeIII :: \magnitudeIV : \magnitudeV$ (\hypstr).

$\therefore \magnitudeIII : \magnitudeII :: \magnitudeV : \magnitudeIV$ (перев.),\\
и было показано, что $\magnitudeV : \magnitudeVI > \magnitudeIII : \magnitudeII$,
$\therefore \magnitudeV : \magnitudeVI > \magnitudeV : \magnitudeIV$ \inprop[prop:V.XIII].

$\therefore \magnitudeVI < \magnitudeIV$,\\
то есть $\magnitudeIV > \magnitudeVI$.

Теперь, пусть $\magnitudeI = \magnitudeIII$; тогда $\magnitudeIV = \magnitudeVI$.\\
Поскольку $\magnitudeI = \magnitudeIII$,\\
$\magnitudeI : \magnitudeII = \magnitudeIII : \magnitudeII$ \inprop[prop:V.VII].

Но $\magnitudeI : \magnitudeII = \magnitudeV : \magnitudeVI$ (\hypstr),\\
и $\magnitudeIII : \magnitudeII = \magnitudeV : \magnitudeIV$ (\hypstr и~перев.).

$\therefore \magnitudeV : \magnitudeVI = \magnitudeV : \magnitudeIV$ \inprop[prop:V.XI].

$\therefore \magnitudeIV = \magnitudeVI$ \inprop[prop:V.IX].

И теперь, пусть $\magnitudeI < \magnitudeIII$, тогда $\magnitudeIV < \magnitudeVI$;\\
поскольку $\magnitudeIII > \magnitudeI$,\\
и было показано, что $\magnitudeIII : \magnitudeII = \magnitudeV : \magnitudeIV$,\\
и $\magnitudeII : \magnitudeI = \magnitudeVI : \magnitudeV$.

$\therefore$ как в~первом случае $\magnitudeVI > \magnitudeIV$,\\
то есть $\magnitudeIV < \magnitudeVI$.

$\therefore$  Если есть три величины… и~т. д.
\stopCenterAlign
\stopProposition

\vfill\pagebreak

\startProposition[title={Предл. XXII. Теорема}, reference=prop:V.XXII]
\defineNewPicture{
byMagnitudeSymbolDefine.I("wedgeDown", byred, 0);
byMagnitudeSymbolDefine.II("rhombus", byblue, 0);
byMagnitudeSymbolDefine.III("square", byyellow, 0);
byMagnitudeSymbolDefine.IV("rhombus", byred, 0);
byMagnitudeSymbolDefine.V("sectorUp", byblue, 1);
byMagnitudeSymbolDefine.VI("circle", byyellow, 0);
byMagnitudeSymbolDefine.Ia("wedgeDown", byblue, 0);
byMagnitudeSymbolDefine.IIa("rhombus", byblack, 0);
byMagnitudeSymbolDefine.IIIa("square", byyellow, 0);
byMagnitudeSymbolDefine.IVa("rhombus", byred, 0);
byMagnitudeSymbolDefine.Va("sectorUp", byblue, 1);
byMagnitudeSymbolDefine.VIa("circle", byblack, 0);
byMagnitudeSymbolDefine.VIIa("halfsquare", byyellow, 0);
byMagnitudeSymbolDefine.VIIIa("halfrhombusUp", byred, 0);
}
\problemNP{Е}{сли}{есть любое количество величин и~столько же других величин, которые взятые по две из каждого ряда по порядку имеют равные отношения, то первые, первая будут иметь к~последней среди первых величин такое же отношение, как первая к~последней среди вторых.\\
Это обычно обозначается словами \quotation{по равенству}.}

\startCenterAlign
Пусть будут величины \drawMagnitude{I}, \drawMagnitude{II}, \drawMagnitude{III},\\
и столько же других величин \drawMagnitude{IV}, \drawMagnitude{V}, \drawMagnitude{VI},\\
таких, что\\
$\magnitudeI : \magnitudeII :: \magnitudeIV : \magnitudeV$,\\
и $\magnitudeII : \magnitudeIII :: \magnitudeV : \magnitudeVI$,\\
тогда $\magnitudeI : \magnitudeIII :: \magnitudeIV : \magnitudeVI$.\\
\stopCenterAlign

Пусть эти величины и~их равнократные, предыдущих или последующих в~отношениях, расположены так:

\startCenterAlign
$\magnitudeI, \magnitudeII, \magnitudeIII, \magnitudeIV, \magnitudeV, \magnitudeVI,$\\
и\\
$M \magnitudeI, m \magnitudeII, N \magnitudeIII, M \magnitudeIV, m \magnitudeV, N \magnitudeVI,$\\
поскольку $\magnitudeI : \magnitudeII :: \magnitudeIV : \magnitudeV$,\\
$\therefore M \magnitudeI : m \magnitudeII :: M \magnitudeIV : m \magnitudeV$ \inprop[prop:V.IV].

По той же причине\\
$m \magnitudeII : N \magnitudeIII :: m \magnitudeV : N \magnitudeVI$;\\
и поскольку есть три величины\\
$M \magnitudeI, m \magnitudeII, N \magnitudeIII$,\\
и другие три, $M \magnitudeIV, m \magnitudeV, N \magnitudeVI$,\\
которые взятые по две из каждого ряда имеют равные отношения.

$\therefore$ если $M \magnitudeI >, = \mbox{ или } < N \magnitudeIII$\\
то $M \magnitudeIV >, = \mbox{ или } < N \magnitudeVI$ \inprop[prop:V.XX],\\
и $\therefore \magnitudeI : \magnitudeIII :: \magnitudeIV : \magnitudeVI$ \indef[def:V.V].

Теперь, допустим, есть четыре величины \drawMagnitude{Ia}, \drawMagnitude{IIa}, \drawMagnitude{IIIa}, \drawMagnitude{IVa},\\
и другие четыре \drawMagnitude{Va}, \drawMagnitude{VIa}, \drawMagnitude{VIIa}, \drawMagnitude{VIIIa},\\
которые взятые по две из каждого ряда имеют равные отношения\\
то есть $\magnitudeIa : \magnitudeIIa :: \magnitudeVa : \magnitudeVIa$,\\
$\magnitudeIIa : \magnitudeIIIa :: \magnitudeVIa : \magnitudeVIIa$,\\
и $\magnitudeIIIa : \magnitudeIVa :: \magnitudeVIIa : \magnitudeVIIIa$,\\
тогда $\magnitudeIa : \magnitudeIVa :: \magnitudeVa : \magnitudeVIIIa$,\\
поскольку $\magnitudeIa, \magnitudeIIa, \magnitudeIIIa$ это три величины,\\
и $\magnitudeVa, \magnitudeVIa, \magnitudeVIIa$ другие три,\\
которые взятые по две из каждого ряда имеют равные отношения,\\
следовательно, как и~в предыдущем случае $\magnitudeIa : \magnitudeIIIa :: \magnitudeVa : \magnitudeVIIa$,\\
но $\magnitudeIIIa : \magnitudeIVa :: \magnitudeVIIa : \magnitudeVIIIa$.

Значит опять, как в~предыдущем случае $\magnitudeIa : \magnitudeIVa :: \magnitudeVa : \magnitudeVIIIa$,\\
и так далее, сколько есть величин.

$\therefore$ Если есть любое количество величин… и~т. д.
\stopCenterAlign
\stopProposition

\vfill\pagebreak

\startProposition[title={Предл. XXIII. Теорема}, reference=prop:V.XXIII]
\defineNewPicture{
byMagnitudeSymbolDefine.I("wedgeDown", byyellow, 0);
byMagnitudeSymbolDefine.II("semicircleDown", byblue, 1);
byMagnitudeSymbolDefine.III("square", byred, 0);
byMagnitudeSymbolDefine.IV("rhombus", byyellow, 0);
byMagnitudeSymbolDefine.V("sectorUp", byblue, 1);
byMagnitudeSymbolDefine.VI("circle", byred, 0);
byMagnitudeSymbolDefine.VII("halfsquare", byblack, 0);
byMagnitudeSymbolDefine.VIII("halfrhombusUp", byblack, 0);
}
\problemNP{Е}{сли}{есть любое число величин и~столько же других, которые взятые по две в~противоположном порядке имеют равные отношения, первая к~последней из первого ряда будет иметь то же отношение, что и~первая к~последней из второго.\\
Это обычно обозначается словами \quotation{по равенству в~перемешанной пропорции}.}

\startCenterAlign
Пусть будут величины \drawMagnitude{I}, \drawMagnitude{II}, \drawMagnitude{III},\\
и другие три, \drawMagnitude{IV}, \drawMagnitude{V}, \drawMagnitude{VI},\\
которые, взятые по две в~противоположном порядке имеют равные отношения.

То есть $\magnitudeI : \magnitudeII :: \magnitudeV : \magnitudeVI$,\\
и $\magnitudeII : \magnitudeIII :: \magnitudeIV : \magnitudeV$,\\
тогда $\magnitudeI : \magnitudeIII :: \magnitudeIV : \magnitudeVI$.
\stopCenterAlign

Пусть эти величины и~их соответствующие равнократные будут расположены так:

\startCenterAlign
$\magnitudeI, \magnitudeII, \magnitudeIII, \magnitudeIV, \magnitudeV, \magnitudeVI,$\\
$M \magnitudeI, M \magnitudeII, m \magnitudeIII, M \magnitudeIV, m \magnitudeV, m \magnitudeVI,$\\
тогда $\magnitudeI : \magnitudeII :: M \magnitudeI : M \magnitudeII$ \inprop[prop:V.XV];\\
и по той же причине\\
$\magnitudeV : \magnitudeVI :: m \magnitudeV : m \magnitudeVI$;\\
но $\magnitudeI : \magnitudeII :: \magnitudeV : \magnitudeVI$ (\hypstr),\\
$\therefore M \magnitudeI : M \magnitudeII :: \magnitudeV : \magnitudeVI$ \inprop[prop:V.XI].

И поскольку $\magnitudeII : \magnitudeIII :: \magnitudeIV : \magnitudeV$ (\hypstr),\\
$\therefore M \magnitudeII : m \magnitudeIII :: M \magnitudeIV : m \magnitudeV$ \inprop[prop:V.IV].

Тогда, поскольку есть три величины\\
$M \magnitudeI, M \magnitudeII, m \magnitudeII$,\\
и другие три $M \magnitudeIV, m \magnitudeV, m \magnitudeVI$,\\
которые, взятые по две в~противоположном порядке имеют равные отношения,\\
значит, если $M \magnitudeI >, =, \mbox{ или } < m \magnitudeIII$,\\
то $M \magnitudeIV >, =, \mbox{ или } < m \magnitudeVI$ \inprop[prop:V.XXI],\\
и $\therefore \magnitudeI : \magnitudeIII :: \magnitudeIV : \magnitudeIV$ \indef[def:V.V].

Теперь, пусть будет четыре величины,\\
\magnitudeI, \magnitudeII, \magnitudeIII, \magnitudeIV,\\
и другие четыре,\\
\magnitudeV, \magnitudeVI, \drawMagnitude{VII}, \drawMagnitude{VIII},\\
которые, взятые по две в~противоположном порядке, имеют равные отношения,
а именно, $\magnitudeI : \magnitudeII :: \magnitudeVII : \magnitudeVIII$,\\
$\magnitudeII : \magnitudeIII :: \magnitudeVI : \magnitudeVII$,\\
и $\magnitudeIII : \magnitudeIV :: \magnitudeV : \magnitudeVI$,\\
получим $\magnitudeIII : \magnitudeIV :: \magnitudeV : \magnitudeVI$.\\
Поскольку $\magnitudeI, \magnitudeII, \magnitudeIII$ три величины,\\
и $\magnitudeVI, \magnitudeVII, \magnitudeVIII$, другие три,\\
которые взятые по две в~противоположном порядке имеют равные отношения,\\
значит, как в~первом случае, $\magnitudeI : \magnitudeIII :: \magnitudeVI : \magnitudeVIII$,\\
но $\magnitudeIII : \magnitudeIV :: \magnitudeV : \magnitudeVI$,\\
значит, опять как в~первом слчае,\\
$\magnitudeIII : \magnitudeIV :: \magnitudeV : \magnitudeVI$;\\
и так далее для любого количества величин.

$\therefore$ Если есть любое количество величин… и~т. д.
\stopCenterAlign
\stopProposition

\vfill\pagebreak

\startProposition[title={Предл. XXIV. Теорема}, reference=prop:V.XXIV]
\defineNewPicture{
byMagnitudeSymbolDefine.I("wedgeDown", byred, 0);
byMagnitudeSymbolDefine.II("semicircleDown", byblack, 1);
byMagnitudeSymbolDefine.III("square", byblue, 0);
byMagnitudeSymbolDefine.IV("rhombus", byyellow, 0);
byMagnitudeSymbolDefine.V("sectorUp", byred, 1);
byMagnitudeSymbolDefine.VI("circle", byblue, 0);
}
\problemNP{Е}{сли}{первая величина ко второй имеет такое же отношение, как третья к~четвертой, а~пятая ко второй, такое же как шестая к~четвертой, то первая и~пятая вместе ко второй будут иметь такое же отношение, как третья с~шестой вместе к~четвертой.}

\setbox0\vbox{
\unprotect
\vbox{
\offinterlineskip\lineskip3pt
\halign{\hfil # \hfil & \hfil # \hfil & \hfil # \hfil & \hfil # \hfil\cr
{\tfx первая} & {\tfx вторая} & {\tfx третья} & {\tfx четвертая} \cr
\drawMagnitude{I} & \drawMagnitude{II} & \drawMagnitude{III} & \drawMagnitude{IV} \cr
{\tfx пятая} & & {\tfx шестая} & \cr
\drawMagnitude{V} & & \drawMagnitude{VI} & \cr
}}
\protect
}
\figureInMargin{\box0}
\startCenterAlign
Пусть $\magnitudeI : \magnitudeII :: \magnitudeIII : \magnitudeIV$,\\
и $\magnitudeV : \magnitudeII :: \magnitudeVI : \magnitudeIV$,\\
то $\magnitudeI + \magnitudeV : \magnitudeII :: \magnitudeIII + \magnitudeVI : \magnitudeIV$.

Поскольку $\magnitudeV : \magnitudeII :: \magnitudeVI : \magnitudeIV$ (\hypstr),\\
и $\magnitudeII : \magnitudeI :: \magnitudeIV : \magnitudeIII$ (\hypstr и~перев.).

$\therefore \magnitudeV : \magnitudeI :: \magnitudeVI : \magnitudeIII$ \inprop[prop:V.XXII];\\
и, поскольку величины пропорциональны, они будут пропорциональны и~если их присоединить.

$\therefore \magnitudeI + \magnitudeV : \magnitudeV :: \magnitudeVI + \magnitudeIII : \magnitudeVI$ \inprop[prop:V.XVIII],\\
но $\magnitudeV : \magnitudeII :: \magnitudeVI : \magnitudeIV$ (\hypstr).

$\therefore \magnitudeI + \magnitudeV : \magnitudeII :: \magnitudeVI + \magnitudeIII : \magnitudeIV$ \inprop[prop:V.XXII].

$\therefore$ Если первая величина ко второй… и~т. д.
\stopCenterAlign
\stopProposition

\vfill\pagebreak

\startProposition[title={Предл. XXV. Теорема}, reference=prop:V.XXV]
\defineNewPicture{
byMagnitudeSymbolDefine.Ia("wedgeDown", byred, 0);
byMagnitudeSymbolDefine.III("semicircleDown", byblack, 1);
byMagnitudeSymbolDefine.IIa("square", byblue, 0);
byMagnitudeSymbolDefine.IV("rhombus", byyellow, 0);
}
\problemNP{Е}{сли}{четыре величины пропорциональны, то наибольшая и~наименьшая из них вместе больше других двух.}

\startCenterAlign
Пусть четыре величины $\drawMagnitude{Ia} + \drawMagnitude{III}, \drawMagnitude{IIa} + \drawMagnitude{IV}, \magnitudeIII, \mbox{ и~} \magnitudeIV$, будут пропорциональными.

$\magnitudeIa + \magnitudeIII : \magnitudeIIa + \magnitudeIV :: \magnitudeIII : \magnitudeIV$,\\
и пусть $\magnitudeIa + \magnitudeIII$ будет наибольшей из четырех, следовательно, согласно \inpropL[prop:V.A] и~\inpropL[prop:V.XIV], \magnitudeIV\ наименьшей.

Тогда $\magnitudeIa + \magnitudeIII + \magnitudeIV > \magnitudeIIa + \magnitudeIV + \magnitudeIII$,\\
поскольку $\magnitudeIa + \magnitudeIII : \magnitudeIIa + \magnitudeIV :: \magnitudeIII : \magnitudeIV$.

$\therefore \magnitudeIa : \magnitudeIIa :: \magnitudeIa + \magnitudeIII : \magnitudeIIa + \magnitudeIV$ \inprop[prop:V.XIX],\\
но $\magnitudeIa + \magnitudeIII > \magnitudeIIa + \magnitudeIV$ (\hypstr).

$\therefore \magnitudeIa > \magnitudeIIa$ \inprop[prop:V.A].

К каждой добавим $\magnitudeIII + \magnitudeIV$,\\
$\therefore \magnitudeIa + \magnitudeIII + \magnitudeIV > \magnitudeIIa + \magnitudeIII + \magnitudeIV$.

$\therefore$ Если четыре величины… и~т. д.
\stopCenterAlign
\stopProposition

\vfill\pagebreak

\startDefinition[title={Определение X},reference=def:V.X,ownnumber=10]
\def\varA{\color[byred]{A}}
\def\varB{\color[byyellow]{B}}
\def\varC{\color[byblue]{C}}
\def\vararS{\color[byred]{ar^2}}
\def\varar{\color[byyellow]{ar}}
\def\vara{\color[byblue]{a}}
Когда три величины пропорциональны, говорят, что первая имеет к~третьей двойное отношение первой ко второй.

Например, если $\varA$, $\varB$, $\varC$ пропорциональны, то есть $\varA : \varB :: \varB : \varC$, то говорят, что $\varA$, к~$\varC$ имеет двойное отношение $\varA : \varB$.

\startCenterAlign
Или $\dfrac{\varA}{\varC} = \mbox{ квадрату } \dfrac{\varA}{\varB}$.

Это яснее видно на таких количествах, как $\vararS$, $\varar$, $\vara$, поскольку $\vararS : \varar :: \varar : \vara$.

И $\dfrac{\vararS}{\vara} = r^2 = \mbox{ квадрату } \dfrac{\vararS}{\varar} = r$,\\
или $\vara$, $\varar$, $\vararS$.

Поскольку $\dfrac{\vara}{\vararS} = \dfrac{1}{r^2} = \mbox{ квадрату } \dfrac{\vara}{\varar} = \dfrac{1}{r}$.
\stopCenterAlign
\stopDefinition

\startDefinition[title={Определение XI},reference=def:V.XI,ownnumber=11]
\def\varA{\color[byred]{A}}
\def\varB{\color[byyellow]{B}}
\def\varC{\color[byblue]{C}}
\def\varD{\color[byblack]{D}}
\def\vararQ{\color[byred]{ar^3}}
\def\vararS{\color[byyellow]{ar^2}}
\def\varar{\color[byblue]{ar}}
\def\vara{\color[byblack]{a}}
Когда четыре величины пропорциональны, говорят, что первая к~четвертой имеет тройное отношение первой ко второй. Таким же образом можно получить четверное отношение и~т. д.

Например, пусть $\varA$, $\varB$, $\varC$, $\varD$, будут пропорциональны, то есть $\varA : \varB :: \varB : \varC :: \varC : \varD$, тогда говорят, что $\varA$ имеет к~$\varD$ тройное отношение $\varA$ к~$\varB$.

\startCenterAlign
Или $\dfrac{\varA}{\varD} = \mbox{ кубу } \dfrac{\varA}{\varB}$.
\stopCenterAlign

Это определение будет более понятным и~приложимым к~более чем четырем пропорциональным величинам так:

\startCenterAlign
Пусть $\vararQ$, $\vararS$, $\varar$, $\vara$ будут четырьмя пропорциональными величинами, то есть $\vararQ : \vararS :: \vararS : \varar :: \varar : \vara$,\\
тогда $\dfrac{\vararQ}{\vara} = r^3 = \mbox{ кубу } \dfrac{\vararQ}{\vararS} = r$.
\stopCenterAlign

Или пусть $ar^5$, $ar^4$, $ar^3$, $ar^2$, $ar$, $a$ будут шестью пропорциональными величинами, то есть:

\startCenterAlign
$ar^5 : ar^4 :: ar^4 : ar^3 :: ar^3 : ar^2 :: ar^2 : ar :: ar : a$,\\
тогда отношение $\dfrac{ar^5}{a} = r^5 = \mbox{ пятой степени } \dfrac{ar^5}{ar^4} = r$.
\stopCenterAlign

Или пусть $a$, $ar$, $ar^2$, $ar^3$, $ar^4$, будут пятью пропорциональными величинами $\dfrac{a}{ar^4} = \dfrac{1}{r^4} = \mbox{ четвертой степени } \dfrac{a}{ar} = \dfrac{1}{r}$.
\stopDefinition

\vfill\pagebreak

\startDefinitionAZ[title={Определение A},reference=def:V.A,ownnumber=A]
\def\varA{\color[byred]{A}}
\def\varB{\color[byred]{B}}
\def\varC{\color[byred]{C}}
\def\varD{\color[byred]{D}}
\def\varE{\color[byblue]{E}}
\def\varF{\color[byblue]{F}}
\def\varG{\color[byblue]{G}}
\def\varH{\color[byblue]{H}}
\def\varK{\color[byblue]{K}}
\def\varL{\color[byblue]{L}}
\def\varM{\color[byyellow]{M}}
\def\varN{\color[byyellow]{N}}
Чтобы понять составное отношение:

\figureInMargin{
\framed[align=middle]{
~$\varA\ \varB\ \varC\ \varD$~\\
~$\varE\ \varF\ \varG\ \varH\ \varK\ \varL$~\\
~$\varM\ \varN$~
}}
Когда есть любое количество однородных величин, говорят, что первая имеет к~последней составное отношение из отношения первой ко второй, второй к~третьей, третьей к~четвертой и~так далее до последней величины. Например, если $\varA, \varB, \varC, \varD$ будут четырьмя однородными величинами, можно сказать, что первая $\varA$ имеет к~последней $\varD$ отношение, составленное из отношений  $\varA$ к~$\varB$, $\varB$ к~$\varC$, и~$\varC$ к~$\varD$.

И если $\varA$ к~$\varB$ относится так же, как $\varE$ к~$\varF$, и~$\varB$ к~$\varC$ как $\varG$ к~$\varH$, и~$\varC$ к~$\varD$, в~то же время, как $\varK$ к~ $\varL$, то, по данному определению, можно сказать, что $\varA$ имеет к~$\varD$ составное отношение из отношений таких же, как $\varE$ к~$\varF$, $\varG$ к~$\varH$, и~$\varK$ к~$\varL$.

Таким же образом, с~теми же предпосылками, если $\varM$ к~$\varN$ имеет такое же отношение, как $\varA$ к~$\varD$, то для краткости можно сказать, что $\varM$ имеет к~$\varN$ составное отношение из отношений $\varE$ к~$\varF$, $\varG$ к~$\varH$, и~$\varK$ к~$\varL$.

Это определение может быть легче понять воспользовавшись арифметической или алгебраической иллюстрацией, поскольку, в~сущности, отношение, составленное из нескольких отношений, это не более чем отношение, в~котором предыдущее представляет собой произведение предыдущих отношений, из которых составлено составное отношение, а~последующее — произведение последующих.

\startCenterAlign
Так, отношение, составленное из отношений\\
$\color[byred]{2} : \color[byred]{3}$, $\color[byyellow]{4} : \color[byyellow]{7}$, $\color[byblue]{6} : \color[byblue]{11}$, $\color[byblack]{2} : \color[byblack]{5}$,\\
будет отношением $\color[byred]{2} \times \color[byyellow]{4} \times \color[byblue]{6} \times \color[byblack]{2} : \color[byred]{3} \times \color[byyellow]{7} \times \color[byblue]{11} \times \color[byblack]{5}$,\\
или отношением $96 : 1155$ или $32 : 385$.
\stopCenterAlign

\def\varA{\color[byred]{A}}
\def\varB{\color[byred]{B}}
\def\varC{\color[byyellow]{C}}
\def\varD{\color[byyellow]{D}}
\def\varE{\color[byblue]{E}}
\def\varF{\color[byblue]{F}}
А среди отношений однородных величин $\varA$, $\varB$, $\varC$, $\varD$, $\varE$, $\varF$, отношение $\varA : \varF$ это отношение, составленное из отношений

\startCenterAlign
$\varA : \varB$, $\varB : \varC$, $\varC : \varD$, $\varD : \varE$, $\varE : \varF$;\\
поскольку $\varA \times \varB \times \varC \times \varD \times \varE : \varB \times \varC \times \varD \times \varE \times \varF$,\\
или $\dfrac{\varA \times \varB \times \varC \times \varD \times \varE}{\varB \times \varC \times \varD \times \varE \times \varF} = \dfrac{\varA}{\varF}$, или отношение $\varA : \varF$.
\stopCenterAlign
\stopDefinitionAZ

\vfill\pagebreak

\startPropositionAZ[title={Предл. F. Теорема}, reference=prop:V.F]
\def\varA{\color[byred]{A}}
\def\varB{\color[byyellow]{B}}
\def\varC{\color[byyellow]{C}}
\def\varD{\color[byyellow]{D}}
\def\varE{\color[byred]{E}}
\def\varF{\color[byblue]{F}}
\def\varG{\color[byyellow]{G}}
\def\varH{\color[byyellow]{H}}
\def\varK{\color[byyellow]{K}}
\def\varL{\color[byblue]{L}}
\problemNP{О}{тношения,}{составленные из  одних и~тех же отношений одинаковы.}

\vskip -\baselineskip

\figureInMargin{
\framed[align=middle]{
~$\varA\ \varB\ \varC\ \varD\ \varE$~\\
~$\varF\ \varG\ \varH\ \varK\ \varL$~
}}
\startCenterAlign
Пусть $\varA : \varB :: \varF : \varG$, $\varB : \varC :: \varG : \varH$, $\varC : \varD :: \varH : \varK$ и $\varD : \varE :: \varK : \varL$.
\stopCenterAlign

Тогда отношение составленное из $\varA : \varB$, $\varB : \varC$, $\varC : \varD$, $\varD : \varE$, или отношение $\varA : \varE$, будет таким же, как составленное из $\varF : \varG$, $\varG : \varH$, $\varH : \varK$, $\varK : \varL$, или отношение $\varF : \varL$.

\startCenterAlign
Поскольку $\dfrac{\varA}{\varB} = \dfrac{\varF}{\varG}$, $\dfrac{\varB}{\varC} = \dfrac{\varG}{\varH}$, $\dfrac{\varC}{\varD} = \dfrac{\varH}{\varK}$, и $\dfrac{\varD}{\varE} = \dfrac{\varK}{\varL}$.

$\therefore \dfrac{\varA \times \varB \times \varC \times \varD}{\varB \times \varC \times \varD \times \varE} = \dfrac{\varF \times \varG \times \varH \times \varK}{\varG \times \varH \times \varK \times \varL}$,\\
и $\therefore \dfrac{\varA}{\varE} = \dfrac{\varF}{\varL}$,\\
или отношение $\varA : \varE$ такое же как $\varF : \varL$.
\stopCenterAlign

То же можно показать для любого числа отношений расположенных таким образом.

\startCenterAlign
Теперь, пусть $\varA : \varB :: \varK : \varL$, $\varB : \varC :: \varH : \varK$, $\varC : \varD :: \varG : \varH$ и $\varD : \varE :: \varF : \varG$.
\stopCenterAlign

Тогда отношение, составленное из отношений $\varA : \varB$, $\varB : \varC$, $\varC : \varD$, $\varD : \varE$, или отношение $\varA : \varE$, будет таким же, как отношение составленное из отношений $\varK : \varL$, $\varH : \varK$, $\varG : \varH$, $\varF : \varG$, или отношение $\varF : \varL$.

\startCenterAlign
Поскольку $\dfrac{\varA}{\varB} = \dfrac{\varK}{\varL}$, $\dfrac{\varB}{\varC} = \dfrac{\varH}{\varK}$, $\dfrac{\varC}{\varD} = \dfrac{\varG}{\varH}$ и $\dfrac{\varD}{\varE} = \dfrac{\varF}{\varG}$.

$\therefore \dfrac{\varA \times \varB \times \varC \times \varD}{\varB \times \varC \times \varD \times \varE} = \dfrac{\varK \times \varH \times \varG \times \varF}{\varL \times \varK \times \varH \times \varG}$ и $\therefore \dfrac{\varA}{\varE} = \dfrac{\varF}{\varL}$,\\
или отношение $\varA : \varE$ такое же как отношение $\varF : \varL$.

$\therefore$ Отношения, составленные… и~т. д.
\stopCenterAlign
\stopPropositionAZ

\vfill\pagebreak

\startPropositionAZ[title={Предл. G. Теорема}, reference=prop:V.G]
\def\varA{\color[byblue]{A}}
\def\varB{\color[byblue]{B}}
\def\varC{\color[byblue]{C}}
\def\varD{\color[byblue]{D}}
\def\varE{\color[byblue]{E}}
\def\varF{\color[byblue]{F}}
\def\varG{\color[byblue]{G}}
\def\varH{\color[byblue]{H}}
\def\varP{\color[byred]{P}}
\def\varQ{\color[byred]{Q}}
\def\varR{\color[byred]{R}}
\def\varS{\color[byred]{S}}
\def\varT{\color[byred]{T}}
\def\varV{\color[byyellow]{V}}
\def\varW{\color[byyellow]{W}}
\def\varX{\color[byyellow]{X}}
\def\varY{\color[byyellow]{Y}}
\def\varZ{\color[byyellow]{Z}}
\problemNP{Е}{сли}{несколько отношений такие же, как несколько других отношений, каждое к~каждому, то отношение, составленное из отношений, таких же, как первые, каждое к~каждому, будет таким же, как отношение, составленное из отношений, таких же как вторые, каждое к~каждому.}

\startCenterAlign
\unprotect
~\hfill\vbox{\halign{\vrule height2.2ex depth1ex\ \hskip 5pt\ # & # & # & # & # & # & # & # \hskip 20pt & # & # & # & # & # \hskip 5pt \vrule \cr
\noalign{\hrule}
$\varA$ & $\varB$ & $\varC$ & $\varD$ & $\varE$ & $\varF$ & $\varG$ & $\varH$ &
$\varP$ & $\varQ$ & $\varR$ & $\varS$ & $\varT$ \cr
$a$ & $b$ & $c$ & $d$ & $e$ & $f$ & $g$ & $h$ &
$\varV$ & $\varW$ & $\varX$ & $\varY$ & $\varZ$ \cr
\noalign{\hrule}
}}\hfill~
\protect

\unprotect
\setbox0\vbox{\halign{ # & # & # \cr
$\eqalign{
\mbox{ Если \ } \varA : \varB & :: a : b\cr
\varC : \varD & :: c : d\cr
\varE : \varF & :: e : f\cr
\mbox{ и~\ } \varG : \varH & :: g : h\cr
}$ &
$\eqalign{
\mbox{ и~\ } \varA : \varB & :: \varP : \varQ\cr
\varC : \varD & :: \varQ : \varR\cr
\varE : \varF & :: \varR : \varS\cr
\varG : \varH & :: \varS : \varT\cr
}$ &
$\eqalign{
\mbox{ и~\ } a : b & :: \varV : \varW\cr
c : d & :: \varW : \varX\cr
e : f & :: \varX : \varY\cr
f : h & :: \varY : \varZ\cr
}$ \cr
}}
\protect
\hskip-\wd0\hskip\textwidth\box0

Тогда $\varP : \varT = \varV : \varZ$.

$\eqalign{
\mbox{ Поскольку } \frac{\varP}{\varQ} = \frac{\varA}{\varB}
	& = \frac{a}{b} = \frac{\varV}{\varW} \cr
\frac{\varQ}{\varR} = \frac{\varC}{\varD}
	& = \frac{c}{d} = \frac{\varW}{\varX} \cr
\frac{\varR}{\varS} = \frac{\varE}{\varF}
	& = \frac{e}{f} = \frac{\varX}{\varY} \cr
\frac{\varS}{\varT} = \frac{\varG}{\varH}
	& = \frac{g}{h} = \frac{\varY}{\varZ} \cr
}$

И $\therefore
\dfrac{\varP \times \varQ \times \varR \times \varS}
	{\varQ \times \varR \times \varS \times \varT} =
\dfrac{\varV \times \varW \times \varX \times \varY}
	{\varW \times \varX \times \varY \times \varZ}$.

И $\therefore \dfrac{\varP}{\varT} = \dfrac{\varV}{\varZ}$,
или $\varP : \varT = \varV : \varZ$.

$\therefore$ Если несколько отношений… и~т. д.
\stopCenterAlign
\stopPropositionAZ

\vfill\pagebreak

\startPropositionAZ[title={Предл. H. Теорема}, reference=prop:V.H]
\def\varA{\color[byred]{A}}
\def\varB{\color[byred]{B}}
\def\varC{\color[byred]{C}}
\def\varD{\color[byred]{D}}
\def\varE{\color[byred]{E}}
\def\varF{\color[byred]{F}}
\def\varG{\color[byred]{G}}
\def\varH{\color[byred]{H}}
\def\varP{\color[byblue]{P}}
\def\varQ{\color[byblue]{Q}}
\def\varR{\color[byblue]{R}}
\def\varS{\color[byblue]{S}}
\def\varT{\color[byblue]{T}}
\def\varX{\color[byblue]{X}}
\problemNP{Е}{сли}{отношение, составленное из нескольких отношений будет таким же, как составленное из нескольких других отношений,
и если одно из первых отношений или отношение, составленное из нескольких из них,  такое же, как одно из вторых или составленное из вторых,
тогда составное отношение оставшихся из первых, будет таким же, как отношение составленное из оставшихся вторых.}
\figureInMargin{
\framed[align=middle]{
~$\varA\ \varB\ \varC\ \varD\ \varE\ \varF\ \varG\ \varH$~\\
~$\varP\ \varQ\ \varR\ \varS\ \varT\ \varX$~\\
}}
Пусть $\varA : \varB$, $\varB : \varC$, $\varC : \varD$, $\varD : \varE$, $\varE : \varF$, $\varF : \varG$, $\varG : \varH$, будут первыми отношениями и~$\varP : \varQ$, $\varQ : \varR$, $\varR : \varS$, $\varS : \varT$, $\varT : \varX$ вторыми отношениями; также пусть $\varA : \varH$, составленное из первых отношений будет таким же, как $\varP : \varX$, составленное вторых отношений, и~пусть отношение $\varA : \varE$, составленное из отношений $\varA : \varB$, $\varB : \varC$, $\varC : \varD$, $\varD : \varE$ будет таким же, как $\varP : \varR$, составленное из отношений $\varP : \varQ$, $\varQ : \varR$.

Тогда отношение, составленное из оставшихся отношений, то есть отношение, составленное из отношений $\varE : \varF$, $\varF : \varG$, $\varG : \varH$, то есть отношение $\varE : \varH$, будет таким же, как отношение $\varR : \varX$, составленное из оставшихся отношений $\varR : \varS$, $\varS : \varT$, $\varT : \varX$.

\startCenterAlign
Поскольку\\
$\dfrac
{\varA \times \varB \times \varC \times \varD \times \varE \times \varF \times \varG}
{\varB \times \varC \times \varD \times \varE \times \varF \times \varG \times \varH} =
\dfrac
{\varP \times \varQ \times \varR \times \varS \times \varT}
{\varQ \times \varR \times \varS \times \varT \times \varX}$.

Или\\
$\dfrac
{\varA \times \varB \times \varC \times \varD}
{\varB \times \varC \times \varD \times \varE}
\times
\dfrac
{\varE \times \varF \times \varG}
{\varF \times \varG \times \varH} =
\dfrac
{\varP \times \varQ}
{\varQ \times \varR}
\times
\dfrac
{\varR \times \varS \times \varT}
{\varS \times \varT \times \varX}$.

И\\
$\dfrac
{\varA \times \varB \times \varC \times \varD}
{\varB \times \varC \times \varD \times \varE} =
\dfrac
{\varP \times \varQ}
{\varQ \times \varR}$.

$\therefore \dfrac
{\varE \times \varF \times \varG}
{\varF \times \varG \times \varH} =
\dfrac
{\varR \times \varS \times \varT}
{\varS \times \varT \times \varX}$.

$\therefore \dfrac{\varE}{\varH} = \dfrac{\varR}{\varX}$.

$\therefore \varE : \varH = \varR : \varX$.

$\therefore$ Если отношение… и~т. д.
\stopCenterAlign
\stopPropositionAZ

\vfill\pagebreak

\startPropositionAZ[title={Предл. K. Теорема}, reference=prop:V.K]
\def\varA{\color[byred]{A}}
\def\varB{\color[byred]{B}}
\def\varC{\color[byred]{C}}
\def\varD{\color[byred]{D}}
\def\varE{\color[byred]{E}}
\def\varF{\color[byred]{F}}
\def\varG{\color[byred]{G}}
\def\varH{\color[byred]{H}}
\def\varK{\color[byred]{K}}
\def\varL{\color[byred]{L}}
\def\varM{\color[byred]{M}}
\def\varN{\color[byred]{N}}
\def\varO{\color[byyellow]{O}}
\def\varP{\color[byyellow]{P}}
\def\varQ{\color[byyellow]{Q}}
\def\varR{\color[byyellow]{R}}
\def\varS{\color[byyellow]{S}}
\def\varT{\color[byyellow]{T}}
\def\varV{\color[byyellow]{V}}
\def\varW{\color[byyellow]{W}}
\def\varX{\color[byyellow]{X}}
\def\varY{\color[byyellow]{Y}}
\def\vara{\color[byblue]{a}}
\def\varb{\color[byblue]{b}}
\def\varc{\color[byblue]{c}}
\def\vard{\color[byblue]{d}}
\def\vare{\color[byblue]{e}}
\def\varf{\color[byblue]{f}}
\def\varg{\color[byblue]{g}}
\problemNP{Е}{сли}{есть любое количество отношений и~любое количество других отношений, такие, что отношение составленное из отношений, таких же, как первые отношения, каждое к~каждому, такое же, как отношение, составленное из отношений, таких же, как вторые, каждое к~каждому, и~если есть отношение, или отношение, составленное из отношений, таких же, как несколько из первых отношений, каждое к~каждому, такое же, как одно из вторых, или отношение, составленное из отношений таких же, каждое к~каждому, как несколько из вторых, то отношение, составленное отношений, таких же, как оставшиеся первые, каждое к~каждому, будет таким же, как отношение, составленное из таких же, как оставшиеся вторые, каждое к~каждому.}

\vskip 0.5\baselineskip

\unprotect
~\hfill\vbox{\halign{\vrule height2.2ex depth1ex\ \hskip 2pt\
#&#& #&#& #&#& #&#& #&#& #&# & \hfil#\hfil
\hskip 2pt \vrule \cr
\noalign{\hrule}
& & & & & & & $h$ & $k$ & $m$ & $n$ & $s$ & \cr
$\varA$ & $\varB$, & $\varC$ & $\varD$, & $\varE$ & $\varF$, & $\varG$ & $\varH$, & $\varK$ & $\varL$, & $\varM$ & $\varN$ & $\vara\ \varb\ \varc\ \vard\ \vare\ \varf\ \varg$ \cr
$\varO$ & $\varP$, & $\varQ$ & $\varR$, & $\varS$ & $\varT$, & $\varV$ & $\varW$, & $\varX$ & $\varY$ & & & $h\ k\ l\ m\ n\ p$ \cr
& & $a$ & $b$ & $k$ & $m$ & & $e$ & $f$ & $g$ & & & \cr
\noalign{\hrule}
}}\hfill~
\protect

\vskip 0.5\baselineskip

Пусть $\varA : \varB$, $\varC : \varD$, $\varE : \varF$, $\varG : \varH$, $\varK : \varL$, $\varM : \varN$, будут первыми отношениями, а~$\varO : \varP$, $\varQ : \varR$, $\varS : \varT$, $\varV : \varW$, $\varX : \varY$, вторыми.

\startCenterAlign
$\eqalign{
\mbox{И пусть }
\varA : \varB &= \vara : \varb ,\cr
\varC : \varD &= \varb : \varc ,\cr
\varE : \varF &= \varc : \varf ,\cr
\varG : \varH &= \vard : \vare ,\cr
\varK : \varL &= \vare : \varf ,\cr
\varM : \varN &= \varf : \varg .
}$
\stopCenterAlign

Тогда, согласно определению составного отношения, отношение $\vara : \varg$ будет составлено из отношений $\vara : \varb$, $\varb : \varc$, $\varc : \vard$, $\vard : \vare$, $\vare : \varf$, $\varf : \varg$, таких же, как отношения $\varA : \varB$, $\varC : \varD$, $\varE : \varF$, $\varG : \varH$, $\varK : \varL$, $\varM : \varN$, каждое к~каждому.

\startCenterAlign
$\eqalign{
\mbox{Также, }
\varO : \varP &= h : k ,\cr
\varQ : \varR &= k : l ,\cr
\varS : \varT &= l : m ,\cr
\varV : \varW &= m : n ,\cr
\varX : \varY &= n : p .\cr
}$
\stopCenterAlign

Тогда отношение $h : p$ будет составлено из отношений $h : k$, $k : l$, $l : m$, $m : n$, $n : p$, таких же, как $\varO : \varP$, $\varQ : \varR$, $\varS : \varT$, $\varV : \varW$, $\varX : \varY$, каждое к~каждому.

\startCenterAlign
$\therefore$ согласно гипотезе,\\ 
$\vara : \varg = h : p$.
\stopCenterAlign

Теперь, пусть отношение, составленное из отношений $\varA : \varB$, $\varC : \varD$, двух первых отношения (или отношения $\vara : \varc$, поскольку $\varA : \varB = \vara : \varb$, и~$\varC : \varD = \varb : \varc$), будет таким же, как отношение $a : d$, состаавленное из отношений $a : b$, $b : c$, $c : d$, таких же, как отношения $\varO : \varP$, $\varQ : \varR$, $\varS : \varT$, три из вторых.

И пусть будут отношения $h : s$, составленное из отношений $h : k$, $k : m$, $m : n$, $n : s$, таких же, как оставшиеся из первых, а~именно $\varE : \varF$, $\varG : \varH$, $\varK : \varL$, $\varM : \varN$, и~отношение $e : g$, составленное из отношения $e : f$ , $f : g$, таких же, как оставшиеся из вторых, а~именно $\varV : \varW$, $\varX : \varY$.  Тогда отношение $h : s$  будет таким же, как отношение $e : g$, или $h : s = e : g$.

\startCenterAlign
Поскольку $\dfrac
{\varA \times \varC \times \varE \times \varG \times \varK \times \varM}
{\varB \times \varD \times \varF \times \varH \times \varL \times \varN}
=
\dfrac
{\vara \times \varb \times \varc \times \vard \times \vare \times \varf}
{\varb \times \varc \times \vard \times \vare \times \varf \times \varg}$.

И, составлением отношений, $\dfrac
{\varO \times \varQ \times \varS \times \varV \times \varX}
{\varP \times \varR \times \varT \times \varW \times \varY}
=$\\
$\dfrac
{h \times k \times l \times m \times n}
{k \times l \times m \times n \times p}$.\\

$\therefore \dfrac
{\vara \times \varb \times \varc \times \vard \times \vare \times \varf}
{\varb \times \varc \times \vard \times \vare \times \varf \times \varg}
=$\\
$\dfrac
{h \times k \times l \times m \times n}
{k \times l \times m \times n \times p}$ (\hypstr).

Или $\dfrac{\vara \times \varb}{\varb \times \varc}
\times
\dfrac
{\varc \times \vard \times \vare \times \varf}
{\vard \times \vare \times \varf \times \varg}
=
\dfrac
{h \times k \times l}
{k \times l \times m}
\times
\dfrac
{m \times n}
{n \times p}$.

Но $\dfrac{\vara \times \varb}{\varb \times \varc} = \dfrac{\varA \times \varC}{\varB \times \varD}
=
\dfrac
{\varO \times \varQ \times \varS}
{\varP \times \varR \times \varT}
=
\dfrac{a \times b \times c}{b \times c \times d}
=
\dfrac{h \times k \times l}{k \times l \times m}$.

$\therefore
\dfrac
{\varc \times \vard \times \vare \times \varf}
{\vard \times \vare \times \varf \times \varg}
=
\dfrac
{m \times n}
{n \times p}$.

И $\dfrac
{\varc \times \vard \times \vare \times \varf}
{\vard \times \vare \times \varf \times \varg}
=$\\ 
$\dfrac
{h \times k \times l \times m \times n}
{k \times l \times m \times n \times p}$ (\hypstr).

И $\dfrac
{m \times n}
{n \times p}
=
\dfrac
{e \times f}
{f \times g}$ (\hypstr),

$\therefore \dfrac
{h \times k \times l \times m \times n}
{k \times l \times m \times n \times p}
= \dfrac{e f}{f g}$.

$\therefore \dfrac{h}{s} = \dfrac{e}{g}$.

$\therefore h : s = e : g$.

$\therefore$ Если есть любое количество отношений… и~т. д.
\stopCenterAlign
\stopPropositionAZ
\stopBook

\startBook[title={Книга VI}]

\startsupersection[title={Определения}]

\startDefinitionOnlyNumber[reference=def:VI.I]
\defineNewPicture{
pair a, b, c, A, B, C, d, e, f, g, D, E ,F ,G;
a := (0, 0);
b := (u, 0);
c := (1/2u, u);
A := (a scaled 2) shifted (u, -u);
B := (b scaled 2) shifted (u, -u);
C := (c scaled 2) shifted (u, -u);
d := (0, 0);
e := (2/3u, 0);
f := (u, 2/3u);
g := (1/3u, 2/3u);
D := d scaled 2;
E := e scaled 2;
F := f scaled 2;
G := g scaled 2;
d := d shifted (0, -5/2u);
e := e shifted (0, -5/2u);
f := f shifted (0, -5/2u);
g := g shifted (0, -5/2u);
D := D shifted (u, -5/2u);
E := E shifted (u, -5/2u);
F := F shifted (u, -5/2u);
G := G shifted (u, -5/2u);
draw byPolygon(a,b,c)(byblue);
draw byPolygon(A,B,C)(byblue);
draw byPolygon(d,e,f,g)(byred);
draw byPolygon(D,E,F,G)(byred);
}\drawCurrentPictureInMargin
Про прямолинейные фигуры говорят, что они подобны, когда их углы равны в~одном порядке и~стороны при равных углах пропорциональны.
\stopDefinitionOnlyNumber

\startDefinitionOnlyNumber[reference=def:VI.II]
Говорят, что две стороны одной фигуры взаимно пропорциональны двум сторонам другой, когда одна из сторон первой фигуры относится к~другой, как одна из сторон второй к~другой.
\stopDefinitionOnlyNumber

\startDefinitionOnlyNumber[reference=def:VI.III]
Говорится, что прямая делится в~крайнем и~среднем отношении, если как целая к~большему отрезку, так и~больший отрезок к~меньшему.
\stopDefinitionOnlyNumber

\startDefinitionOnlyNumber[reference=def:VI.IV]
\defineNewPicture{
pair d[];
d1 := (0, 0);
d2 := (5/2u, 0);
d3 := (5u, 0);
d4 := (15/2u, 0);
numeric h;
h := 6/5u;
pair A, B, C, H;
A := (0, 0) shifted d1;
B := (3/2u, 0) shifted d1;
C := (3/4u, h) shifted d1;
H := (xpart(C), 0);
draw byPolygon (A,B,C)(byyellow);
draw byLine(C, H)(byyellow, 1, 0);
pair A, B, C, D, H, G;
A := (0, 0) shifted d2;
B := (1/2u, 0) shifted d2;
C := (3/2u, h) shifted d2;
D := (u, h) shifted d2;
H := (xpart(C), 0);
G := (xpart(D), 0);
draw byPolygon (A,B,C,D)(byblue);
byLineDefine(C, H, byblue, 1, 0);
draw byLine(D, G, byblue, 1, 0);
byLineDefine(A, H, byblue, 1, 0);
draw byNamedLineSeq(0)(CH,AH);
pair A, B, C, D, H;
A := (0, 0) shifted d3;
B := (2/3u, 0) shifted d3;
C := (u, h) shifted d3;
H := (xpart(C), 0);
draw byPolygon (A,B,C)(byblack);
byLineDefine(C, H, byblack, 1, 0);
byLineDefine(A, H, byblack, 1, 0);
draw byNamedLineSeq(0)(CH,AH);
pair A, B, C, D, E, H;
A := (0, 0) shifted d4;
B := (1/2u, 0) shifted d4;
C := (u, 1/2h) shifted d4;
D := (1/4u, h) shifted d4;
E := (-1/2u, 1/2h) shifted d4;
H := (xpart(D), 0);
draw byPolygon (A,B,C,D,E)(byred);
draw byLine(D, H)(byred, 1, 0);
}
Высотой фигуры называют перпендикуляр,  проведенный от вершины к~основанию или его продолжению.

\vskip\baselineskip

\noindent\ \hfill\reuseMPgraphic{\currentInstance::currentPicture}\hfill\
\stopDefinitionOnlyNumber
\stopsupersection

\vfill\pagebreak

\startProposition[title={Предл. I. Теорема},reference=prop:VI.I]
\defineNewPicture[2/5]{
pair A, B, C, D, H, G, K, L, M, N, O, P, Q;
numeric w, h;
w := 9/2u;
h := 3u;
A := (6/11w, h);
B := (5/11w, 0);
C := (6/11w, 0);
D := (7/11w, 0);
H := (3/11w, 0);
G := (4/11w, 0);
K := (8/11w, 0);
L := (9/11w, 0);
M := (0/11w, 0);
N := (1/11w, 0);
O := (2/11w, 0);
P := (10/11w, 0);
Q := (11/11w, 0);
draw byPolygon(A,M,N)(byblack);
draw byPolygon(A,O,H)(byblack);
draw byPolygon(A,G,B)(byblack);
draw byPolygon(A,B,C)(byred);
draw byPolygon(A,C,D)(byblue);
draw byPolygon(A,K,L)(byyellow);
draw byPolygon(A,P,Q)(byyellow);
draw byLine(M, B)(byblack, 0, 0);
draw byLine(B, C)(byblue, 0, 0);
draw byLine(C, D)(byred, 0, 0);
draw byLine(D, Q)(byblack, 0, 0);
draw byLabelsOnPolygon(A, Q, D, C, B, M)(0, 0);
}
\drawCurrentPictureInMargin
\problemNP{Т}{реугольники}{и параллелограммы, имеющие одну и~ту же высоту относятся друг к~другу как их основания.}

Пусть у~треугольников \drawPolygon[bottom]{ABC} и~\drawPolygon[bottom]{ACD} общая вершина, а~их основания \drawUnitLine{BC} и~\drawUnitLine{CD} лежат на одной прямой.

Продлим \drawUnitLine{BC,CD} в~обе стороны, возьмем на продолжении со стороны \drawUnitLine{CD} линии равные ей, а~со стороны \drawUnitLine{BC} линии равные ей, и~проведем линии от вершины к~их концам.

Треугольники
\drawFromCurrentPicture[bottom][trianglesAMC]{
draw byNamedPolygon(AMN,AOH,AGB,ABC);
draw byNamedLine(MB,BC);
draw byLabelsOnPolygon(A, C, M)(0, 0);
}
образованные таким образом равны между собой, поскольку равны их основания \inprop[prop:I.XXXVIII].

$\therefore$ \trianglesAMC\ и~его основание соответственно равнократны \polygonABC\ и~его основанию \drawUnitLine{BC}.

Таким же образом
\drawFromCurrentPicture[bottom][trianglesACQ]{
draw byNamedPolygon(ACD,AKL,APQ);
draw byNamedLine(CD,DQ);
draw byLabelsOnPolygon(C, A, Q)(0, 0);
}
и его основание соответственно равнократны \polygonACD\ и~его основанию \drawUnitLine{CD}.

$\therefore$ если $m$ или $6$ раз \polygonABC\ $>, = \mbox{ или } < n$ или $5$ раз \polygonACD\ то $m$ или $6$ раз \drawUnitLine{BC} $>, = \mbox{ или } < n$ или $5$ раз \drawUnitLine{CD}, $m$ и~$n$ обозначают любые кратные, взятые как в~\indefL[def:V.V]. Хотя мы показали только, что это свойство проявляется, когда $m = 6$ и~$n = 5$, но очевидно, что это свойство работает для любых кратных значений, которые можно придать $m$ и~$n$.

\startCenterAlign
$\therefore \polygonABC\ : \polygonACD\ :: \drawUnitLine{BC} : \drawUnitLine{CD}$ \indef[def:V.V]
\stopCenterAlign

Параллелограммы с~той же высотой вдвое больше треугольников, на тех же основаниях, и~пропорциональны им (часть I), и, поскольку они вдвое больше, такие параллелограммы относятся как их основания \inprop[prop:V.XV].

\qed
\stopProposition

\startProposition[title={Предл. II. Теорема},reference=prop:VI.II]
\defineNewPicture[1/2]{
pair A, B, C, D, E, Ab, Bb, Cb, Db, Eb, Ac, Bc, Cc, Dc, Ec, Fc, d[];
d1 := (0, 0);
d2 := (u, -4u);
d3 := (5/2u, -9/2u);
A := (0, 0) shifted d1;
B := (-1/3u, -7/2u) shifted d1;
C := B shifted (5/2u, 0);
D := 1/2[A, B];
E := 1/2[A, C];
Ab := (0, 0) shifted d2;
Bb := (-u, -2u) shifted d2;
Cb := Bb shifted (2u, 0);
Db := 4/3[Ab, Bb];
Eb := 4/3[Ab, Cb];
Ac := (0, 0) shifted d3;
Bc := (-3u, -5u) shifted d3;
Cc := Bc shifted (2u, 0);
Dc := 3/5[Ac, Bc];
Ec := 3/5[Ac, Cc];
Fc = whatever[Bc, Ec] = whatever[Cc, Dc];
draw byLine(D, E, byblack, 0, 0);
draw byLine(B, E, byred, 0, 0);
draw byLine(C, D, byblue, 0, 0);
byLineDefine(A, D, byred, 1, 0);
byLineDefine(D, B, byyellow, 0, 0);
byLineDefine(B, C, byblack, 1, 0);
byLineDefine(A, E, byblue, 1, 0);
byLineDefine(E, C, byyellow, 1, 0);
draw byNamedLineSeq(0)(AD,DB,BC,EC,AE);
draw byLine(Bb, Eb, byred, 0, 0);
draw byLine(Cb, Db, byblue, 0, 0);
draw byLine(Bb, Cb, byblack, 1, 0);
byLineDefine(Ab, Bb, byred, 1, 0);
byLineDefine(Bb, Db, byyellow, 0, 0);
byLineDefine(Db, Eb, byblack, 0, 0);
byLineDefine(Ab, Cb, byblue, 1, 0);
byLineDefine(Cb, Eb, byyellow, 1, 0);
draw byNamedLineSeq(0)(AbBb,BbDb,DbEb,CbEb,AbCb);
draw byLine(Bc, Fc, byyellow, 0, 0);
draw byLine(Fc, Ec, byred, 1, 0);
draw byLine(Cc, Fc, byyellow, 1, 0);
draw byLine(Fc, Dc, byblue, 1, 0);
byLineDefine(Bc, Cc, byblack, 1, 0);
byLineDefine(Bc, Dc, byred, 0, 0);
byLineDefine(Dc, Ec, byblack, 0, 0);
byLineDefine(Cc, Ec, byblue, 0, 0);
draw byNamedLineSeq(0)(BcCc,BcDc,DcEc,CcEc);
byPointLabelDefine(Ab, "A");
byPointLabelDefine(Bb, "B");
byPointLabelDefine(Cb, "C");
byPointLabelDefine(Db, "D");
byPointLabelDefine(Eb, "E");
byPointLabelDefine(Bc, "B");
byPointLabelDefine(Cc, "C");
byPointLabelDefine(Dc, "D");
byPointLabelDefine(Ec, "E");
byPointLabelDefine(Fc, "F");
draw byLabelsOnPolygon(A, E, C, B, D)(0, 0);
draw byLabelsOnPolygon(Ab, Cb, Eb, Db, Bb)(0, 0);
draw byLabelsOnPolygon(Dc, Ec, Cc, Bc)(0, 0);
draw byLabelsOnPolygon(Bc, Fc, Dc)(2, 0);
}
\drawCurrentPictureInMargin
\problemNP{Е}{сли}{в треугольнике параллельно одной из сторон \drawUnitLine{BC} проведена прямая \drawUnitLine{DE}, то она рассекает две другие стороны на пропорциональные отрезки.\\
И если прямая \drawUnitLine{DE} делит две стороны треугольника на пропорциональные сегменты, то она параллельна оставшейся стороне \drawUnitLine{BC}.}

\startsubproposition[title={Часть I.}]
\startCenterAlign
Пусть $\drawUnitLine{DE} \parallel \drawUnitLine{BC}$,\\
тогда $\drawUnitLine{DB} : \drawUnitLine{AD} :: \drawUnitLine{EC} : \drawUnitLine{AE}$.

Проведем \drawUnitLine{BE} и~\drawUnitLine{CD},\\
и $\drawLine[middle][triangleBDE]{DE,BE,DB} = \drawLine[middle][triangleCDE]{EC,CD,DE}$ \inprop[prop:I.XXXVII].

$\therefore \triangleBDE\ :
\drawLine[middle][triangleADE]{AD,AE,DE} :: \triangleCDE\ : \triangleADE$ \inprop[prop:V.VII].

Но $\triangleBDE\ : \triangleADE\ :: \drawUnitLine{DB} : \drawUnitLine{AD}$ \inprop[prop:VI.I],\\
$\therefore \drawUnitLine{DB} : \drawUnitLine{AD} :: \drawUnitLine{EC} : \drawUnitLine{AE}$ \inprop[prop:V.XI].
\stopCenterAlign
\stopsubproposition

\vfill\pagebreak

\startsubproposition[title={Часть II.}]
\startCenterAlign
Пусть $\drawUnitLine{DB} : \drawUnitLine{AD} :: \drawUnitLine{EC} : \drawUnitLine{AE}$.\\
тогда $\drawUnitLine{DE} \parallel \drawUnitLine{BC}$.

Оставим то же построение,\\
$\left.\eqalign{
\mbox{ поскольку } \drawUnitLine{DB} : \drawUnitLine{AD} &:: \triangleBDE\ : \triangleADE \cr
\mbox{ и~} \drawUnitLine{EC} : \drawUnitLine{AE} &:: \triangleCDE\ : \triangleADE \cr
}\right\}\mbox{ \inprop[prop:VI.I] }$.

Но $\drawUnitLine{DB} : \drawUnitLine{AD} :: \drawUnitLine{EC} : \drawUnitLine{AE}$ (\hypstr)

$\therefore \triangleBDE\ : \triangleADE :: \triangleCDE\ : \triangleADE$ \inprop[prop:V.XI].

$\therefore \triangleBDE\ = \triangleCDE$ \inprop[prop:V.IX].

Но они на одном основании \drawUnitLine{BC}, и~по одну его сторону\\
и $\therefore \drawUnitLine{DE} \parallel \drawUnitLine{BC}$ \inprop[prop:I.XXXIX].
\stopCenterAlign
\stopsubproposition

\qed
\stopProposition

\startProposition[title={Предл. III. Теорема},reference=prop:VI.III]
\defineNewPicture{
pair A, B, C, D, E;
numeric a;
a := 80;
A := (0, 0);
B := A shifted (dir(220)*3u);
C = whatever[A, A + dir(220+a)] = whatever[B, B shifted dir(0)];
D = whatever[A, A + dir(220+1/2a)] = whatever[B, C];
E = whatever[A, B] = whatever[C, C shifted (A-D)];
byAngleDefine(B, A, D, byyellow, 0);
byAngleDefine(D, A, C, byblack, 0);
byAngleDefine(C, A, E, byblack, 1);
byAngleDefine(A, E, C, byblue, 0);
byAngleDefine(E, C, A, byred, 0);
draw byNamedAngleResized();
draw byLine(C, A, byyellow, 0, 0);
draw byLine(A, D, byblue, 0, 0);
byLineDefine(A, E, byred, 1, 0);
byLineDefine(C, E, byblue, 1, 0);
byLineDefine(A, B, byred, 0, 0);
byLineDefine(B, D, byblack, 0, 0);
byLineDefine(D, C, byblack, 1, 0);
draw byNamedLineSeq(0)(AE,CE,DC,BD,AB);
draw byLabelsOnPolygon(B, A, E, C, D)(0, 0);
}
\drawCurrentPictureInMargin
\problemNP{П}{рямая}{\drawUnitLine{AD}, рассекающая угол треугольника пополам делит противоположную сторону на части \drawUnitLine{BD} и~\drawUnitLine{DC} пропорциональные смежным им сторонам \drawUnitLine{AB} и~\drawUnitLine{CA}.\\
И если прямая \drawUnitLine{AD}, проведенная из любого угла треугольника делит противоположную сторону \drawUnitLine{BD,DC} на части \drawUnitLine{BD} и~\drawUnitLine{DC}, пропорциональные смежным им сторонам \drawUnitLine{AB} и~\drawUnitLine{CA}, то она делит угол пополам.}

\startsubproposition[title={Часть I.}]
\startCenterAlign
Проведем $\drawUnitLine{CE} \parallel \drawUnitLine{AD}$, до \drawUnitLine{AE},\\
тогда, $\drawAngle{BAD} = \drawAngle{E}$ \inprop[prop:I.XXIX].

$\therefore \drawAngle{DAC} = \drawAngle{E}$; но $\drawAngle{DAC} = \drawAngle{C}$.

$\therefore \drawAngle{C} = \drawAngle{E}$.

$\therefore \drawUnitLine{AE} = \drawUnitLine{CA}$ \inprop[prop:I.VI],\\
и поскольку $\drawUnitLine{AD} \parallel \drawUnitLine{CE}$,\\
$\drawUnitLine{AE} : \drawUnitLine{AB} :: \drawUnitLine{DC} : \drawUnitLine{BD}$ \inprop[prop:VI.II].

Но $\drawUnitLine{AE} = \drawUnitLine{CA}$;\\
$\therefore \drawUnitLine{CA} : \drawUnitLine{AB} :: \drawUnitLine{DC} : \drawUnitLine{BD}$ \inprop[prop:V.VII].
\stopCenterAlign
\stopsubproposition

\vfill\pagebreak

\startsubproposition[title={Часть II.}]
\startCenterAlign
Оставим то же построение,\\
и $\drawUnitLine{AB} : \drawUnitLine{AE} :: \drawUnitLine{BD} : \drawUnitLine{DC}$ \inprop[prop:VI.II].

Но $\drawUnitLine{BD} : \drawUnitLine{DC} :: \drawUnitLine{AB} : \drawUnitLine{CA}$ (\hypstr).

$\therefore \drawUnitLine{AB} : \drawUnitLine{AE} :: \drawUnitLine{AB} : \drawUnitLine{CA}$ \inprop[prop:V.XI].

И $\therefore \drawUnitLine{AE} = \drawUnitLine{CA}$ \inprop[prop:V.IX],\\
и $\therefore \drawAngle{E} = \drawAngle{C}$ \inprop[prop:I.V].

Но поскольку $\drawUnitLine{AD} \parallel \drawUnitLine{CE}$, $\drawAngle{DAC} = \drawAngle{C}$,\\
и $\drawAngle{BAD} = \drawAngle{E}$ \inprop[prop:I.XXIX].

$\therefore \drawAngle{C} = \drawAngle{E}$, и~$\drawAngle{BAD} = \drawAngle{DAC}$,\\
и $\therefore$ \drawUnitLine{AD} делит \drawAngle{BAD,DAC} пополам.
\stopCenterAlign
\stopsubproposition

\qed
\stopProposition

\startProposition[title={Предл. IV. Теорема},reference=prop:VI.IV]
\defineNewPicture{
pair A, B, C, D, E, F;
B := (0, 0);
A := (5/4u, u);
C := (2u, 0);
D := (A scaled 4/3) shifted (C-B);
E := (C scaled 4/3) shifted (C-B);
F = whatever[A, B] = whatever[D, E];
byAngleDefine(A, B, C, byyellow, 0);
byAngleDefine(B, C, A, byblue, 0);
byAngleDefine(C, A, B, byblack, 1);
byAngleDefine(D, C, E, byred, 0);
byAngleDefine(C, E, D, byblack, 0);
byAngleDefine(E, D, C, byred, 1);
draw byNamedAngleResized();
draw byLine(C, A, byred, 0, 0);
draw byLine(C, D, byblue, 1, 0);
byLineDefine(A, F, byyellow, 1, 0);
byLineDefine(D, F, byyellow, 0, 0);
byLineDefine(A, B, byblue, 0, 0);
byLineDefine(D, E, byred, 1, 0);
byLineDefine(E, C, byblack, 1, 0);
byLineDefine(B, C, byblack, 0, 0);
draw byNamedLineSeq(0)(AF,DF,DE,EC,BC,AB);
draw byLabelsOnPolygon(B, A, F, D, E, C)(0, 0);
}
\drawCurrentPictureInMargin
\problemNP[3]{В}{равноугольных}{треугольниках \drawLine[bottom][triangleABC]{CA,BC,AB} и~\drawLine[bottom][triangleCDE]{CD,DE,EC}, стороны при равных углах пропорциональны, а~стороны противолежащие равным углам соответственны.} % In the original "sides which are opposite to the equal angles are homologous", what dos it mean?

Пусть два треугольника будут расположены так, что стороны \drawUnitLine{BC} и~\drawUnitLine{EC} против равных углов \drawAngle{D} и~ \drawAngle{A} будут смежными и~на одной прямой и~сами треугольники лежат по одну сторону этой прямой, а~равные углы, напротив, не будут смежными, т. е. \drawAngle{DCE} напротив \drawAngle{B}, и~\drawAngle{BCA} напротив \drawAngle{E}.

\startCenterAlign
Проведем \drawUnitLine{AF} и~\drawUnitLine{DF}.

Тогда, поскольку $\drawAngle{BCA} = \drawAngle{E}$, $\drawUnitLine{CA} \parallel \drawUnitLine{DF,DE}$ \inprop[prop:I.XXVIII].

И по той же причине $\drawUnitLine{CD} \parallel \drawUnitLine{AB,AF}$.

$\therefore$
\drawFromCurrentPicture{
startGlobalRotation(-lineAngle.CD);
startAutoLabeling;
draw byNamedLineSeq(0)(CA,AF,DF,CD);
stopAutoLabeling;
stopGlobalRotation;
}
параллелограмм.

Но $\drawUnitLine{BC} : \drawUnitLine{EC} :: \drawUnitLine{DF} : \drawUnitLine{DE}$ \inprop[prop:VI.II],\\
и поскольку $\drawUnitLine{DF} = \drawUnitLine{CA}$ \inprop[prop:I.XXXIV],\\
$\drawUnitLine{BC} : \drawUnitLine{EC} :: \drawUnitLine{CA} : \drawUnitLine{DE}$.

И, переставлением, $\drawUnitLine{BC} : \drawUnitLine{CA} :: \drawUnitLine{EC} : \drawUnitLine{DE}$ \inprop[prop:V.XVI].

Так же можно показать, что\\
$\drawUnitLine{AB} : \drawUnitLine{CD} :: \drawUnitLine{BC} : \drawUnitLine{EC}$,\\
и переставлением, что\\
$\drawUnitLine{AB} : \drawUnitLine{BC} :: \drawUnitLine{CD} : \drawUnitLine{EC}$,\\
но уже было доказано, что\\
$\drawUnitLine{BC} : \drawUnitLine{CA} :: \drawUnitLine{EC} : \drawUnitLine{DE}$\\
и значит, по равенству\\
$\drawUnitLine{AB} : \drawUnitLine{CA} :: \drawUnitLine{CD} : \drawUnitLine{DE}$ \inprop[prop:V.XXII].
\stopCenterAlign

Следовательно, стороны при равных углах пропорциональны, а~противолежащие им соответственны.

\qed
\stopProposition

\startProposition[title={Предл. V. Теорема},reference=prop:VI.V]
\defineNewPicture{
pair A, B, C, D, E, F, G, d;
B := (0, 0);
A := (2u, 5/2u);
C := (3u, 0);
byAngleDefine(B, A, C, byyellow, 0);
byAngleDefine(A, B, C, byblue, 0);
byAngleDefine(B, C, A, byred, 0);
draw byNamedAngleResized(BAC, ABC, BCA);
byLineDefine(A, B, byred, 1, 0);
byLineDefine(B, C, byblack, 1, 0);
byLineDefine(C, A, byblue, 1, 0);
draw byNamedLineSeq(0)(AB,BC,CA);
d := (0, -3u);
D := A shifted d;
E := B shifted d;
F := C shifted d;
G := (A yscaled -1) shifted d;
byAngleDefine(F, D, E, byblack, 0);
byAngleDefine(D, E, F, byyellow, 0);
byAngleDefine(E, F, D, byred, 1);
byAngleDefine(E, G, F, byblack, 0);
byAngleDefine(G, E, F, byblue, 0);
byAngleDefine(E, F, G, byred, 0);
draw byNamedAngleResized(FDE, DEF, EFD, EGF, GEF, EFG);
draw byLine(E, F, byblack, 0, 0);
byLineDefine(F, G, byyellow, 1, 0);
byLineDefine(G, E, byyellow, 0, 0);
byLineDefine(D, E, byred, 0, 0);
byLineDefine(F, D, byblue, 0, 0);
draw byNamedLineSeq(0)(FG,GE,DE,FD);
draw byLabelsOnPolygon(B, A, C)(0, 0);
draw byLabelsOnPolygon(E, D, F, G)(0, 0);
angleScale := 4/5;
}
\drawCurrentPictureInMargin
\problemNP{Е}{сли}{у двух треугольников стороны пропорциональны
$\drawUnitLine{CA} : \drawUnitLine{BC} :: \drawUnitLine{FD} : \drawUnitLine{EF}$
и
$\drawUnitLine{BC} : \drawUnitLine{AB} :: \drawUnitLine{EF} : \drawUnitLine{DE}$
то они равноугольны, а~равные углы стягиваются соответственными сторонами.
}
\startCenterAlign
Из концов \drawUnitLine{EF}, проведем \drawUnitLine{FG} и~\drawUnitLine{GE},\\
делая $\drawAngle{GEF} = \drawAngle{B}$, $\drawAngle{EFG} = \drawAngle{C}$ \inprop[prop:I.XXIII],\\
получим $\drawAngle{G} = \drawAngle{A}$ \inprop[prop:I.XXXII].

И поскольку треугольники равноугольные,\\
$\drawUnitLine{AB} : \drawUnitLine{BC} :: \drawUnitLine{GE} : \drawUnitLine{EF}$ \inprop[prop:VI.IV].

Но $\drawUnitLine{AB} : \drawUnitLine{BC} :: \drawUnitLine{DE} : \drawUnitLine{EF}$ (\hypstr).

$\therefore \drawUnitLine{DE} : \drawUnitLine{EF} :: \drawUnitLine{GE} : \drawUnitLine{EF}$, и~значит $\drawUnitLine{DE} = \drawUnitLine{GE}$ \inprop[prop:V.IX].

Так же можно показать, что $\drawUnitLine{FD} = \drawUnitLine{FG}$.

$\therefore$ у~двух треугольников с~общим основанием \drawUnitLine{EF} и~равными сторонами,  углы против равных сторон равны, т.~е.
$\drawAngle{DEF} = \drawAngle{GEF}$ и~$\drawAngle{EFD} = \drawAngle{EFG}$ \inprop[prop:I.VIII].

Но $\drawAngle{GEF} = \drawAngle{B}$ (\conststr) и~$\therefore \drawAngle{DEF} = \drawAngle{B}$.

По той же причине $\drawAngle{EFD} = \drawAngle{C}$, и~значит $\drawAngle{D} = \drawAngle{A}$ \inprop[prop:I.XXXII], т.~е. треугольники равноугольны и~очевидно, что соответственные стороны располагаются между равными углами.
\stopCenterAlign

\qed
\stopProposition

\startProposition[title={Предл. VI. Теорема},reference=prop:VI.VI]
\defineNewPicture[1/7]{
pair A, B, C, D, E, F, G, d;
A := (0, 0);
B := (2u, 5/2u);
C := (3u, 0);
byAngleDefine(A, B, C, byyellow, 0);
byAngleDefine(B, A, C, byblue, 0);
byAngleDefine(A, C, B, byred, 0);
draw byNamedAngleResized(BAC, ABC, ACB);
byLineDefine(A, B, byred, 1, 0);
byLineDefine(C, A, byblack, 1, 0);
byLineDefine(B, C, byblue, 1, 0);
draw byNamedLineSeq(0)(AB,BC,CA);
d := (0, -3u);
D := A shifted d;
E := B shifted d;
F := C shifted d;
G := (B yscaled -1) shifted d;
byAngleDefine(F, D, E, byblue, 1);
byAngleDefine(D, E, F, byyellow, 1);
byAngleDefine(E, F, D, byred, 1);
byAngleDefine(F, D, G, byblue, 0);
byAngleDefine(G, F, D, byred, 0);
byAngleDefine(D, G, F, byblack, 0);
draw byNamedAngleResized(FDE, DEF, EFD, FDG, GFD, DGF);
draw byLine(F, D, byblue, 0, 0);
byLineDefine(F, G, byyellow, 1, 0);
byLineDefine(G, D, byyellow, 0, 0);
byLineDefine(D, E, byred, 0, 0);
byLineDefine(E, F, byblack, 0, 0);
draw byNamedLineSeq(0)(FG,GD,DE,EF);
draw byLabelsOnPolygon(A, B, C)(0, 0);
draw byLabelsOnPolygon(D, E, F, G)(0, 0);
}
\drawCurrentPictureInMargin
\problemNP[3]{Е}{сли}{у двух треугольников \drawLine[bottom][triangleABC]{AB,BC,CA} и~\drawLine[bottom][triangleDEF]{DE,EF,FD} один угол одного \drawAngle{C} равен одному углу другого \drawAngle{EFD}, и~стороны, при равных углах пропорциональны, то треугольники равноугольны, и~углы против соответственных сторон у~них равны.}

\startCenterAlign
Из концов \drawUnitLine{FD}, одной из сторон \triangleDEF\ при \drawAngle{EFD}, проведем \drawUnitLine{GD} и~\drawUnitLine{FG},\\
делая $\drawAngle{GFD} = \drawAngle{C}$ и~$\drawAngle{FDG} = \drawAngle{A}$,\\
тогда $\drawAngle{G} = \drawAngle{B}$ \inprop[prop:I.XXXII].\\
И поскольку два треугольника равноугольны,\\
$\drawUnitLine{BC} : \drawUnitLine{CA} :: \drawUnitLine{FG} : \drawUnitLine{FD}$ \inprop[prop:VI.IV],\\
но $\drawUnitLine{BC} : \drawUnitLine{CA} :: \drawUnitLine{EF} : \drawUnitLine{FD}$ (\hypstr),
$\therefore \drawUnitLine{FG} : \drawUnitLine{FD} :: \drawUnitLine{EF} : \drawUnitLine{FD}$ \inprop[prop:V.XI],\\
и следовательно $\drawUnitLine{FG} = \drawUnitLine{EF}$ \inprop[prop:V.IX],\\
$\therefore \triangleDEF\ =
\drawLine[bottom][triangleDGF]{FD,FG,GD}$ во всех отношениях \inprop[prop:I.IV],\\
но $\drawAngle{GFD} = \drawAngle{A}$ (\conststr) и~$\therefore \drawAngle{EFD} = \drawAngle{A}$,\\
и поскольку также $\drawAngle{EFD} = \drawAngle{C}$,\\
$\drawAngle{E} = \drawAngle{B}$ \inprop[prop:I.XXXII].\\
И $\therefore$ \triangleABC\ и~\triangleDEF\ равноугольны с~равными углами против соответственных сторон.
\stopCenterAlign

\qed
\stopProposition

\startProposition[title={Предл. VII. Теорема},reference=prop:VI.VII]
\defineNewPicture[1/4]{
pair A, B, C, D, E, F, G, d;
A := (0, 0);
B := (3u, -1/2u);
C := (5/2u, 3u);
G := 1/3[A, C];
d := (0, -4u);
D := (A scaled 4/5) shifted d;
E := (B scaled 4/5) shifted d;
F := (C scaled 4/5) shifted d;
byAngleDefine(B, C, A, byblue, 0);
byAngleDefine(C, A, B, byred, 0);
byAngleDefine(A, B, G, byblack, 0);
byAngleDefine(G, B, C, byblack, 1);
byAngleDefine(C, G, B, byyellow, 0);
byAngleDefine(B, G, A, byred, 0);
draw byNamedAngleResized(BCA, CAB, ABG, GBC, CGB, BGA);
draw byLine(B, G, byblue, 0, 0);
byLineDefine(A, B, byyellow, 0, 0);
byLineDefine(B, C, byred, 0, 0);
byLineDefine(C, A, byblack, 0, 0);
draw byNamedLineSeq(0)(CA,BC,AB);
byAngleDefine(D, E, F, byblue, 1);
byAngleDefine(E, F, D, byyellow, 1);
byAngleDefine(F, D, E, byred, 1);
draw byNamedAngleResized(DEF, EFD, FDE);
byLineDefine(D, E, byyellow, 1, 0);
byLineDefine(E, F, byred, 1, 0);
byLineDefine(F, D, byblue, 1, 0);
draw byNamedLineSeq(0)(FD,EF,DE);
draw byLabelsOnPolygon(A, G, C, B)(0, 0);
draw byLabelsOnPolygon(D, F, E)(0, 0);
}
\drawCurrentPictureInMargin
\problemNP[2]{Е}{сли}{у двух треугольников \drawLine[middle][triangleABC]{CA,BC,AB} и~\drawLine[middle][triangleDEF]{FD,EF,DE} один угол одного \drawAngle{F} равен одному другого \drawAngle{C}, а~стороны при других пропорциональны $\drawUnitLine{BC} : \drawUnitLine{AB} :: \drawUnitLine{EF} : \drawUnitLine{DE}$, причем оставшиеся углы \drawAngle{A} и~\drawAngle{D} либо меньше, либо не меньше прямого угла, то треугольники равноугольны, а~углы между пропорциональными сторонами равны.}

\startCenterAlign
Для начала допустим, что углы \drawAngle{A} и~\drawAngle{D} оба меньше прямого угла, тогда, если предположить, что \drawAngle{ABG,GBC} и~\drawAngle{E} между пропорциональными сторонами не равны, пусть \drawAngle{ABG,GBC} будет больше и~сделаем $\drawAngle{GBC} = \drawAngle{E}$.

Поскольку $\drawAngle{C} = \drawAngle{F}$ (\hypstr). и~$\drawAngle{GBC} = \drawAngle{E}$ (\conststr)\\
$\therefore \drawAngle{CGB} = \drawAngle{D}$ \inprop[prop:I.XXXII].

$\therefore \drawUnitLine{BC} : \drawUnitLine{BG} :: \drawUnitLine{EF} : \drawUnitLine{DE}$ \inprop[prop:VI.IV],\\
но $\drawUnitLine{BC} : \drawUnitLine{AB} :: \drawUnitLine{EF} : \drawUnitLine{DE}$ (\hypstr).

$\therefore \drawUnitLine{BC} : \drawUnitLine{BG} :: \drawUnitLine{BC} : \drawUnitLine{AB}$.

$\therefore \drawUnitLine{BG} = \drawUnitLine{AB}$ \inprop[prop:V.IX],\\
и $\therefore \drawAngle{A} = \drawAngle{BGA}$ \inprop[prop:I.V].

Но \drawAngle{A} меньше прямого угла (\hypstr)\\
$\therefore$ \drawAngle{BGA} меньше прямого угла,\\
и $\therefore$ \drawAngle{CGB} должен быть больше прямого угла \inprop[prop:I.XIII], но он, как было показано $= \drawAngle{D}$ и~значит меньше прямого угла, что невозможно. $\therefore$ \drawAngle{ABG,GBC} и~\drawAngle{E} не неравны.

$\therefore$ они равны, и~поскольку $\drawAngle{C} = \drawAngle{F}$ (\hypstr).

$\therefore \drawAngle{A} = \drawAngle{D}$ \inprop[prop:I.XXXII], и~следовательно треугольники равноугольны.
\stopCenterAlign

А если предположить, что \drawAngle{A} и~\drawAngle{D} оба не меньше прямого угла, то можно доказать, как и~ранее, что треугольники равноугольны и~стороны при равных углах пропорциональны \inprop[prop:VI.IV].

\qed
\stopProposition

\startProposition[title={Предл. VIII. Теорема},reference=prop:VI.VIII]
\defineNewPicture{
pair A, B, C, D;
A := (0, 0);
B := A shifted (dir(-145)*7/2u);
C = whatever[B, B shifted (1, 0)] = whatever[A, A shifted dir(-145 - 90)];
D := (xpart(A), ypart(B));
draw byPolygon(A,B,D)(byyellow);
draw byPolygon(A,D,C)(byred);
byAngleDefine(A, B, C, byblack, 0);
byAngleDefine(B, C, A, byblue, 1);
byAngleDefine(B, D, A, byblue, 0);
byAngleDefine(D, A, B, byred, 0);
byAngleDefine(C, A, D, byyellow, 0);
draw byNamedAngleResized();
draw byNamedAngleDummySides(BCA);
draw byLine(A, D, byblack, 0, 0);
draw byLabelsOnPolygon(B, A, C, D)(0, 0);
}
\drawCurrentPictureInMargin
\problemNP[3]{Е}{сли}{из прямого угла в~прямоугольном треугольнике \drawPolygon[bottom][triangleABC]{ABD,ADC} проведен к~противоположной стороне перпендикуляр \drawUnitLine{AD}, треугольники \drawPolygon[bottom][triangleABD]{ABD} и~\drawPolygon[bottom][triangleADC]{ADC}, по сторонам перпендикуляра подобны целому треугольнику и~друг другу.}

\startCenterAlign
Поскольку $\drawAngle{DAB,CAD} = \drawAngle{D}$ \inax[ax:I.XI], и~\drawAngle{B} общий \triangleABC\ и~\triangleABD,\\
$\drawAngle{C} = \drawAngle{DAB}$ \inprop[prop:I.XXXII].

$\therefore$ \triangleABC\ и~\triangleABD\ равноугольны и~значит стороны при равных углах пропорциональны \inprop[prop:VI.IV], и, следовательно, подобны \indef[def:VI.I].

Так же можно доказать, что \triangleADC\ подобен \triangleABC, но \triangleABD, как было показано, тоже подобен \triangleABC, $\therefore$ \triangleABD\ и~\triangleADC\ подобны и~всему треугольнику и~друг другу.
\stopCenterAlign

\qed
\stopProposition

\startProposition[title={Предл. IX. Задача},reference=prop:VI.IX]
\defineNewPicture{
pair A, B, C, D, E, F;
numeric n;
n := 5;
A := (0, 0);
B := (3u, 5u);
D := A shifted (0, u);
E := ((D-A) scaled n) shifted A;
F = whatever[A, B] = whatever[D, D shifted (B-E)];
C := 1/3[D, E];
draw byLine(D, F, byred, 1, 0);
byLineDefine(B, E, byred, 0, 0);
byLineDefine(A, F, byyellow, 0, 0);
byLineDefine(A, D, byblue, 0, 0);
byLineDefine(D, C, byblue, 1, 0);
byLineDefine(C, E, byblack, 1, 0);
byLineDefine(F, B, byyellow, 1, 0);
byLineDefine(A, E, byblack, 0, 0);
draw byNamedLineSeq(0)(BE, FB, AF, AD, DC, CE);
for i := 2 step 1 until n - 1:
draw byMarkLine(i/n, byblack)(AE);
endfor;
draw byLabelsOnPolygon(B, F, A, D, C, E)(0, 0);
}
\drawCurrentPictureInMargin
\problemNP{О}{т}{данной прямой \drawSizedLine{AF,FB} отнять требуемую часть.}

\startCenterAlign
Из любого конца данной прямой проведем \drawSizedLine{AD,DC} под любым углом к~\drawSizedLine{AF,FB}.

Продлим \drawSizedLine{AD,DC}, пока вся продленная линия \drawSizedLine{AD,DC,CE} не будет содержать \drawSizedLine{AD} столько раз, какую часть требуется отнять от \drawSizedLine{AF,FB}.

Проведем \drawSizedLine{BE},\\
и проведем $\drawSizedLine{DF} \parallel \drawSizedLine{BE}$.

\drawSizedLine{AF} и~есть требуемая часть \drawSizedLine{AF,FB}.

Поскольку $\drawSizedLine{DF} \parallel \drawSizedLine{BE}$\\
$\drawSizedLine{AF} : \drawSizedLine{FB} :: \drawSizedLine{AD} : \drawSizedLine{DC,CE}$ \inprop[prop:VI.II].

И, присоединив \inprop[prop:V.XVIII],\\
$\drawSizedLine{AF,FB} : \drawSizedLine{AF} :: \drawSizedLine{AD,DC,CE} : \drawSizedLine{AD}$.

То есть \drawSizedLine{AD,DC,CE} содержит \drawSizedLine{AD} столько раз, сколько \drawSizedLine{AF,FB} содержит требуемую часть (\conststr).

$\therefore$ \drawSizedLine{AF} и~есть требуемая часть.
\stopCenterAlign

\qed
\stopProposition

\startProposition[title={Предл. X. Задача},reference=prop:VI.X]
\defineNewPicture{
pair A', D', E', C', A, B, C, D, E, F, G, L, d;
numeric a[];
A' := (0, 0);
D' := (3/2u, 0);
E' := (5/2u, 0);
C' := (7/2u, 0);
draw byLine(A', D', byblue, 0, 1);
draw byLine(D', E', byred, 0, 1);
draw byLine(E', C', byyellow, 0, 1);
d := (0, -7/2u);
a1 := 40;
a2 := -70;
A := (A' rotated a1) shifted d;
D := (D' rotated a1) shifted d;
E := (E' rotated a1) shifted d;
C := (C' rotated a1) shifted d;
L := 6/5[A, C];
F = whatever[A, A shifted dir(0)] = whatever[D, D shifted dir(a2)];
G = whatever[A, A shifted dir(0)] = whatever[E, E shifted dir(a2)];
B = whatever[A, A shifted dir(0)] = whatever[C, C shifted dir(a2)];
draw byLine(D, F, byblack, 0, 0);
draw byLine(E, G, byblack, 1, 0);
byLineDefine(C, B, byblack, 0, 1);
byLineDefine(A, D, byblue, 1, 0);
byLineDefine(D, E, byred, 1, 0);
byLineDefine(E, C, byyellow, 1, 0);
byLineDefine(C, L, byblack, 0, 0);
byLineDefine(A, F, byblue, 0, 0);
byLineDefine(F, G, byred, 0, 0);
byLineDefine(G, B, byyellow, 0, 0);
draw byNamedLineSeq(0)(noLine,CL,EC,DE,AD,AF,FG,GB,CB);
draw byLabelsOnPolygon(C, B, G, F, A, D, E, C, noPoint)(2, 0);
draw byLabelsOnPolygon(E, C, noPoint)(2, 0);
draw byLabelsOnPolygon(A', D', E', C', noPoint)(0, 0);
}
\drawCurrentPictureInMargin
\problemNP{Д}{анную}{прямую \drawSizedLine{AF,FG,GB} рассечь подобно данной рассеченной \drawSizedLine{A'D',D'E',E'C'}.}

\startCenterAlign
Из любого конца \drawSizedLine{AF,FG,GB} проведем \drawSizedLine{AD,DE,EC} под любым углом.

Возьмем \drawSizedLine{AD}, \drawSizedLine{DE} и~\drawSizedLine{EC} равные \drawSizedLine{A'D'}, \drawSizedLine{D'E'} и~\drawSizedLine{E'C'} соответственно \inprop[prop:I.II].

Проведем \drawSizedLine{CB}, и~проведем \drawSizedLine{EG} и~\drawSizedLine{DF} $\parallel$ ей.

Поскольку
$\left\{\vcenter{
\nointerlineskip\hbox{\drawSizedLine{CB}}
\nointerlineskip\hbox{\drawSizedLine{EG}}
\nointerlineskip\hbox{\drawSizedLine{DF}}}\right\}$
все $\parallel$,\\
$\drawSizedLine{GB} : \drawSizedLine{FG} :: \drawSizedLine{EC} : \drawSizedLine{DE}$ \inprop[prop:VI.II],\\
или $\drawSizedLine{GB} : \drawSizedLine{FG} :: \drawSizedLine{E'C'} : \drawSizedLine{D'E'}$ (\conststr),\\
и $\drawSizedLine{FG} : \drawSizedLine{AF} :: \drawSizedLine{DE} : \drawSizedLine{AD}$ \inprop[prop:VI.II],\\
$\drawSizedLine{FG} : \drawSizedLine{AF} :: \drawSizedLine{D'E'} : \drawSizedLine{A'D'}$ (\conststr).

И $\therefore$ данная линия \drawSizedLine{AF,FG,GB} разделена подобно \drawSizedLine{A'D',D'E',E'C'}.
\stopCenterAlign

\qed
\stopProposition

\startProposition[title={Предл. XI. Задача},reference=prop:VI.XI]
\defineNewPicture{
pair A', C', A, B, C, D, E, d;
numeric a[], l[];
a1 := 85;
a2 := 60;
l1 := 2u;
l2 := 5/2u;
A := (0, 0);
B := A shifted (dir(a1)*l1);
C := A shifted (dir(a2)*l2);
D := B shifted (dir(a1)*l2);
E = whatever[A, C] = whatever[D, D shifted (B-C)];
d := (2u, 0);
A' := A shifted d;
C' := A' shifted (dir(a1)*l2);
draw byLine(B, C, byyellow, 0, 0);
byLineDefine(D, E, byyellow, 1, 0);
byLineDefine(A, B, byblack, 0, 0);
byLineDefine(B, D, byblue, 1, 0);
byLineDefine(A, C, byred, 1, 0);
byLineDefine(C, E, byred, 0, 0);
draw byNamedLineSeq(0)(DE,CE,AC,AB,BD);
draw byLine(A', C', byblue, 0, 0);
draw byLabelsOnPolygon(E, C, A, B, D)(0, 0);
draw byLabelsOnPolygon(C', A', noPoint)(0, 0);
}
\drawCurrentPictureInMargin
\problemNP{Н}{айти}{третью пропорциональную для двух данных прямых \drawSizedLine{A'C'} и~\drawSizedLine{AB}.}

\startCenterAlign
От любого конца данной линии \drawSizedLine{AB} проведем \drawSizedLine{AC,CE} под углом.

Сделаем $\drawSizedLine{AC} = \drawSizedLine{A'C'}$, и~проведем \drawSizedLine{BC}.

Сделаем $\drawSizedLine{BD} = \drawSizedLine{A'C'}$,\\
и проведем $\drawSizedLine{DE} \parallel \drawSizedLine{BC}$  \inprop[prop:I.XXXI].

\drawSizedLine{CE} и~есть третья пропорциональная к~\drawSizedLine{AB} и~\drawSizedLine{A'C'}.

Поскольку $\drawSizedLine{BC} \parallel \drawSizedLine{DE}$,\\
$\therefore \drawSizedLine{AB} : \drawSizedLine{BD} :: \drawSizedLine{AC} : \drawSizedLine{CE}$ \inprop[prop:VI.II].

Но $\drawSizedLine{BD} = \drawSizedLine{AC} = \drawSizedLine{A'C'}$ (\conststr);\\
$\therefore \drawSizedLine{AB} : \drawSizedLine{A'C'} :: \drawSizedLine{A'C'} : \drawSizedLine{CE}$ \inprop[prop:V.VII].
\stopCenterAlign

\qed
\stopProposition

\startProposition[title={Предл. XII. Задача},reference=prop:VI.XII]
\defineNewPicture[1/2]{
pair A, Ae, B, Be, C, Ce, D, E, F, G, H;
numeric l[], a;
l1 := 4/3u;
l2 := 11/6u;
l3 := 3/2u;
A := (0, 0); Ae := (l1, 0);
B := (0, 0); Be := (l2, 0);
C := (0, 0); Ce := (l3, 0);
a := 45;
byLineDefine.A(A, Ae, byblue, 1, 0);
byLineDefine.B(B, Be, byred, 1, 0);
byLineDefine.C(C, Ce, byyellow, 1, 0);
forsuffixes i=A, B, C:
lineUseLineLabel.i := true;
endfor;
D := (0, 0);
G := D shifted (dir(a)*l1);
E := G shifted (dir(a)*l2);
H := D shifted (l3, 0);
F = whatever[D, H] = whatever[E, E shifted (G-H)];
draw byLine(G, H, byblack, 0, 1);
byLineDefine(E, F, byblack, 1, 0);
byLineDefine(D, H, byyellow, 0, 0);
byLineDefine(H, F, byblack, 0, 0);
byLineDefine(D, G, byblue, 0, 0);
byLineDefine(G, E, byred, 0, 0);
draw byNamedLineSeq(0)(EF,HF,DH,DG,GE);
draw byLabelsOnPolygon(D, G, E, F, H)(0, 0);
}
\drawCurrentPictureInMargin
\problemNP{Д}{ля}{трех данных прямых
$\left\{\vcenter{
\nointerlineskip\hbox{\drawSizedLine{A}}
\nointerlineskip\hbox{\drawSizedLine{B}}
\nointerlineskip\hbox{\drawSizedLine{C}}}\right\}$
найти четвертую пропорциональную.}

\startCenterAlign
Проведем \drawSizedLine{DG,GE} и~\drawSizedLine{DH,HF} под углом.

Возьмем $\drawUnitLine{DG} = \drawUnitLine{A}$,\\
возьмем $\drawUnitLine{GE} = \drawUnitLine{B}$,\\
возьмем $\drawUnitLine{DH} = \drawUnitLine{C}$,\\
проведем \drawUnitLine{GH},\\
и $\drawUnitLine{EF} \parallel \drawUnitLine{GH}$ \inprop[prop:I.XXXI].

\drawSizedLine{HF} и~есть четвертая пропорциональная.

Ведь, учитывая параллельные,\\
$\drawUnitLine{DG} : \drawUnitLine{GE} :: \drawUnitLine{DH} : \drawUnitLine{HF}$ \inprop[prop:VI.II].

Но
$\left\{\vcenter{
\nointerlineskip\hbox{\drawSizedLine{A}}
\nointerlineskip\hbox{\drawSizedLine{B}}
\nointerlineskip\hbox{\drawSizedLine{C}}}\right\}
=
\left\{\vcenter{
\nointerlineskip\hbox{\drawSizedLine{DG}}
\nointerlineskip\hbox{\drawSizedLine{GE}}
\nointerlineskip\hbox{\drawSizedLine{DH}}}\right\}$
(\conststr).

$\therefore \drawUnitLine{A} : \drawUnitLine{B} :: \drawUnitLine{C} : \drawUnitLine{HF}$ \inprop[prop:V.VII].
\stopCenterAlign

\qed
\stopProposition

\startProposition[title={Предл. XIII. Задача},reference=prop:VI.XIII]
\defineNewPicture{
pair Ab, Bb, Cb, A, B, C, D, O;
numeric l[], r;
path q;
l1 := 3u;
l2 := 2u;
r := 1/2*(l1 + l2);
Ab := (0, 0);
Bb := (l1, 0);
Cb := (l1 + l2, 0);
byLineDefine.A(Ab, Bb, byblue, 1, 0);
byLineDefine.B(Bb, Cb, byred, 1, 0);
lineUseLineLabel.A := true;
lineUseLineLabel.B := true;
A := (0, 0);
B := (l1, 0);
C := (l1 + l2, 0);
O := 1/2[A, C];
q := (fullcircle scaled 2r) shifted O;
D := q intersectionpoint (B--(B shifted (0, r)));
byAngleDefine(A, D, C, byblue, 0);
draw byNamedAngleResized();
byLineDefine(A, C, byblack, 0, 1);
byLineStylize(A, C, 0, 0, -1)(AC);
draw byMarkLine(1/2, byblue)(AC);
draw byLine(B, D, byblack, 0, 0);
byLineDefine(A, B, byblue, 0, 0);
byLineDefine(B, C, byred, 0, 0);
byLineDefine(A, D, byyellow, 0, 0);
byLineDefine(C, D, byyellow, 1, 0);
draw byNamedLineSeq(-1)(AB,BC,CD,AD);
draw byArc.O(O, C, A, r, byred, 0, 0, 0, 0);
draw byLabelsOnPolygon(C, B, A, noPoint)(0, 0);
draw byLabelsOnCircle(D)(O);
}
\drawCurrentPictureInMargin
\problemNP{Н}{айти}{третью пропорциональную между двумя данными прямыми линиями $\left\{\vcenter{
\nointerlineskip\hbox{\drawSizedLine{A}}
\nointerlineskip\hbox{\drawSizedLine{B}}}\right\}$.}

\startCenterAlign
Проведем любую прямую \drawSizedLine{AB,BC}.

Сделаем $\drawSizedLine{AB} = \drawSizedLine{A}$ и~$\drawSizedLine{BC} = \drawSizedLine{B}$.

Рассечем \drawSizedLine{AB,BC} пополам, и~взяв середину за центр и~половину за радиус опишем полукруг
\drawFromCurrentPicture[bottom]{
draw byNamedLineFull(A, C, 0, 0, -1)(AC);
draw byNamedArc(O);
draw byLabelsOnPolygon(C, A, noPoint)(0, 0);
},\\
проведем $\drawSizedLine{BD} \perp \drawSizedLine{AB}$:\\
\drawSizedLine{BD} и~есть искомая средняя пропорциональная.

Проведем \drawUnitLine{AD} и~\drawUnitLine{CD}.

Поскольку \drawAngle{D} прямой угол \inprop[prop:III.XXXI],\\
и \drawUnitLine{BD} это $\perp$ к~нему с~противной стороны.

$\therefore$ \drawUnitLine{BD} и~есть средняя пропорциональная между \drawUnitLine{AB} и~\drawUnitLine{BC} \inprop[prop:VI.VIII],\\
и $\therefore$ между \drawUnitLine{A} и~\drawUnitLine{B} (\conststr).
\stopCenterAlign

\qed
\stopProposition

\startProposition[title={Предл. XIV. Теорема},reference=prop:VI.XIV]
\defineNewPicture[1/4]{
pair A, B, C, D, E, F, G, H, d[];
numeric a;
d1 := (3/2u, 0);
d2 := (1/2u, -7/4u);
d3 := 2/3d1;
d4 := 3/2d2;
C := (0, 0);
E := C shifted d1;
G := C shifted d2;
B := C shifted (d1 + d2);
F := B shifted d3;
D := B shifted d4;
A := B shifted (d3 + d4);
H = whatever[A, F] = whatever[C, E];
draw byPolygon(C,E,B,G)(byyellow);
draw byPolygon(B,F,A,D)(byblue);
draw byPolygon(E,H,F,B)(byred);
draw byLine(B, G, byred, 0, 0);
draw byLine(B, F, byblack, 0, 0);
draw byLine(B, E, byblue, 0, 0);
draw byLine(B, D, byyellow, 0, 0);
draw byLabelsOnPolygon(B, G, C, E, H, F, A, D)(0, 0);
}
\drawCurrentPictureInMargin
\problemNP[3]{В}{равных}{и равноугольных параллелограммах \drawPolygon{BFAD} и~\drawPolygon{CEBG},  стороны при равных углах взаимно пропорциональны ($\drawUnitLine{BG} : \drawUnitLine{BF} :: \drawUnitLine{BD} : \drawUnitLine{BE}$). И~равноугольные параллелограммы, у~которых стороны взаимно пропорциональны, равны.}

\startCenterAlign
Пусть \drawUnitLine{BG}, \drawUnitLine{BF}, \drawUnitLine{BD} и~\drawUnitLine{BE}, будут так расположены, что \drawUnitLine{BG,BF} и~\drawUnitLine{BD,BE} составят прямые линии. Что они могут занять такое положение — очевидно (\inpropL[prop:I.XIII], \inpropL[prop:I.XIV], \inpropL[prop:I.XV]).

Достроим \drawPolygon{EHFB}. Поскольку $\drawPolygon{CEBG} = \drawPolygon{BFAD}$,\\
$\therefore \drawPolygon{CEBG} : \drawPolygon{EHFB} :: \drawPolygon{BFAD} : \drawPolygon{EHFB}$ \inprop[prop:V.VII]\\
$\therefore \drawUnitLine{BG} : \drawUnitLine{BF} :: \drawUnitLine{BD} : \drawUnitLine{BE}$ \inprop[prop:VI.I].

При том же построении:\\
$\drawUnitLine{BG} : \drawUnitLine{BF} ::
\left\{\eqalign{
\drawPolygon{CEBG} &: \drawPolygon{EHFB} \mbox{\inprop[prop:VI.I]}\cr
\drawUnitLine{BD} &: \drawUnitLine{BE} \mbox{(\hypstr)}\cr
\drawPolygon{BFAD} &: \drawPolygon{EHFB} \mbox{\inprop[prop:VI.I]}\cr
}\right.$

$\therefore \drawPolygon{CEBG} : \drawPolygon{EHFB} :: \drawPolygon{BFAD} : \drawPolygon{EHFB}$ \inprop[prop:V.XI],\\
и $\therefore \drawPolygon{CEBG} = \drawPolygon{BFAD}$ \inprop[prop:V.IX].
\stopCenterAlign

\qed
\stopProposition

\startProposition[title={Предл. XV. Теорема},reference=prop:VI.XV]
\defineNewPicture{
pair A, B, C, D, E;
D := (0, 0);
B := (3u, 0);
A := (5/4u, -3/2u);
C := A shifted 3/2(A-D);
E := A shifted 3/2(A-B);
draw byPolygon(D,A,B)(byblue);
draw byPolygon(D,A,E)(byred);
draw byPolygon(A,B,C)(byyellow);
byAngleDefine(D, A, E, byblue, 0);
byAngleDefine(B, A, C, byred, 0);
draw byNamedAngleResized();
byLineDefine(B, D, byblack, 1, 0);
byLineDefine(D, A, byyellow, 0, 0);
byLineDefine(B, A, byblack, 0, 0);
byLineDefine(A, C, byred, 0, 0);
byLineDefine(A, E, byblue, 0, 0);
draw byNamedLineSeq(0)(noLine,AE,BA,BD,DA,AC);
draw byLabelsOnPolygon(A, E, D, B, C)(0, 0);
}
\drawCurrentPictureInMargin
\problemNP[4]{В}{равных}{треугольниках, имеющих по одному равному углу $\drawAngle{DAE} = \drawAngle{BAC}$, стороны при равных углах взаимно пропорциональны\\
$\drawUnitLine{AE} : \drawUnitLine{BA} :: \drawUnitLine{AC} : \drawUnitLine{DA}$.\\
И треугольники, имеющие по одному равному углу и~взаимно пропорциональные стороны при равных углах, равны.}

\startsubproposition[title={Часть I.}]
Пусть треугольники будут расположены так, что равные углы \drawAngle{DAE} и~\drawAngle{BAC} будут вертикальны, то есть так, что \drawUnitLine{AE} и~\drawUnitLine{BA} будут на одной прямой. Таким образом и~\drawUnitLine{AC} с~\drawUnitLine{DA} будут на одной прямой \inprop[prop:I.XIV].

\startCenterAlign
Проведем \drawUnitLine{BD}, тогда\\
$\eqalign{
\drawUnitLine{DA} : \drawUnitLine{BA}
&:: \drawPolygon{DAE} : \drawPolygon{DAB} \mbox{\inprop[prop:VI.I]}\cr
&:: \drawPolygon{ABC} : \drawPolygon{DAB} \mbox{\inprop[prop:V.VII]}\cr
&:: \drawUnitLine{AC} : \drawUnitLine{DA} \mbox{\inprop[prop:VI.I]}\cr
}$

$\therefore \drawUnitLine{AE} : \drawUnitLine{BA} :: \drawUnitLine{AC} : \drawUnitLine{DA}$ \inprop[prop:V.XI].
\stopCenterAlign
\stopsubproposition

\vfill\pagebreak

\startsubproposition[title={Часть II.}]
\startCenterAlign
При том же построении.

$\drawPolygon{DAE} : \drawPolygon{DAB} :: \drawUnitLine{DA} : \drawUnitLine{BA}$ \inprop[prop:VI.I]\\
и $\drawUnitLine{AC} : \drawUnitLine{DA} :: \drawPolygon{ABC} : \drawPolygon{DAB}$ \inprop[prop:VI.I].

Но $ \drawUnitLine{AE} : \drawUnitLine{BA} :: \drawUnitLine{AC} : \drawUnitLine{DA}$, (\hypstr).

$\therefore \drawPolygon{DAE} : \drawPolygon{DAB} :: \drawPolygon{ABC} : \drawPolygon{DAB}$ \inprop[prop:V.XI].

$\therefore \drawPolygon{DAE} = \drawPolygon{ABC}$ \inprop[prop:V.IX].
\stopCenterAlign
\stopsubproposition

\qed
\stopProposition

\startProposition[title={Предл. XVI. Теорема},reference=prop:VI.XVI]
\defineNewPicture{
pair A, B, C, D, Eb, Fb, Ee, Fe, E, F, G, H, d[];
numeric l[], h[];
l1 := -3u;
l2 := -2u;
h1 := -3/4l2;
h2 := -3/4l1;
A := (0, 0);
B := (l1, 0);
G := (0, h1);
F := (l1, h1);
d1 := (0, -h2 - 1/2u);
C := (0, 0) shifted d1;
D := (l2, 0) shifted d1;
H := (0, h2) shifted d1;
E := (l2, h2) shifted d1;
d2 := (0, h1 + u);
d3 := (0, h1 + 1/2u);
Eb := (0, 0) shifted d2; Ee := (-h2, 0) shifted d2;
Fb := (0, 0) shifted d3; Fe := (-h1, 0) shifted d3;
draw byPolygon(A,B,F,G)(byred);
draw byPolygon(C,D,E,H)(byyellow);
byLineDefine(A, B, byyellow, 0, 0);
byLineDefine(A, G, byblack, 0, 0);
draw byNamedLineSeq(0)(AB,AG);
byLineDefine(C, D, byblue, 0, 0);
byLineDefine(C, H, byred, 0, 0);
draw byNamedLineSeq(0)(CD,CH);
draw byLineWithName(Ee, Eb, byred, 1, 0)(E);
draw byLineWithName(Fe, Fb, byblack, 1, 0)(F);
draw byLabelsOnPolygon(F, G, A, B)(0, 0);
draw byLabelsOnPolygon(E, H, C, D)(0, 0);
draw byLabelLine(0)(E, F);
}
\drawCurrentPictureInMargin
\problemNP{Е}{сли}{четыре прямые пропорциональны $\drawUnitLine{AB} : \drawUnitLine{CD} :: \drawUnitLine{E} : \drawUnitLine{F}$, прямоугольник $\drawUnitLine{AB} \times \drawUnitLine{F}$, заключенный между крайними равен прямоугольнику $\drawUnitLine{CD} \times \drawUnitLine{E}$, заключенному между средними.\\
И если прямоугольник заключенный между крайними равен прямоугольнику, заключенному между средними, четыре прямых пропорциональны.}

\startsubproposition[title={Часть I.}]
\startCenterAlign
Из концов \drawUnitLine{AB} и~\drawUnitLine{CD} проведем \drawUnitLine{AG} и~\drawUnitLine{CH} $\perp$ им и~$= \drawUnitLine{F}$ и~\drawUnitLine{E} соответственно.

Достроим параллелограммы \drawPolygon{ABFG} и~\drawPolygon{CDEH}.

И поскольку\\
$\drawUnitLine{AB} : \drawUnitLine{CD} :: \drawUnitLine{E} : \drawUnitLine{F}$ (\hypstr)\\
$\therefore \drawUnitLine{AB} : \drawUnitLine{CD} :: \drawUnitLine{CH} : \drawUnitLine{AG}$ (\conststr).

$\therefore \drawPolygon{ABFG} = \drawPolygon{CDEH}$ \inprop[prop:VI.XIV],\\
то есть, прямоугольник, заключенный между крайними равен прямоугольнику, заключенному между средними.
\stopCenterAlign
\stopsubproposition

\vfill\pagebreak

\startsubproposition[title={Часть II.}]
\startCenterAlign
Оставим то же построение.

Поскольку $\drawUnitLine{F} = \drawUnitLine{AG}$, $\drawPolygon{ABFG} = \drawPolygon{CDEH}$\\
и $\drawUnitLine{CH} = \drawUnitLine{E}$,\\
$\therefore \drawUnitLine{AB} : \drawUnitLine{CD} :: \drawUnitLine{CH} : \drawUnitLine{AG}$ \inprop[prop:VI.XIV]ю

Но $\drawUnitLine{CH} = \drawUnitLine{E}$,\\
и $\drawUnitLine{AG} = \drawUnitLine{F}$ (\conststr).

$\therefore \drawUnitLine{AB} : \drawUnitLine{CD} :: \drawUnitLine{E} : \drawUnitLine{F}$ \inprop[prop:V.VII].
\stopCenterAlign
\stopsubproposition

\qed
\stopProposition

\startProposition[title={Предл. XVII. Теорема},reference=prop:VI.XVII]
\defineNewPicture{
pair Ab, Ae, Bb, Be, Cb, Ce, Db, De;
pair A, B, C, D, E, F, G, H;
pair d[];
numeric l[], r;
r := 3/4;
l1 := 3u;
l2 := r*l1;
l3 := r*l2;
d1 := (0, 4/2u); d2 := (0, 3/2u); d3 := (0, 2/2u); d4 := (0, 1/2u);
Ae := (0, 0) shifted d1; Ab := (-l1, 0) shifted d1;
Be := (0, 0) shifted d2; Bb := (-l2, 0) shifted d2;
Ce := (0, 0) shifted d3; Cb := (-l3, 0) shifted d3;
De := (0, 0) shifted d4; Db := (-l2, 0) shifted d4;
d5 := (0, -l2);
A := (0, 0) shifted d5; B := (-l2, 0) shifted d5; C := (-l2, l2) shifted d5; D := (0, l2) shifted d5;
d6 := (0, -l2-l1-1/2u);
E := (0, 0) shifted d6; F := (-l3, 0) shifted d6; G := (-l3, l1) shifted d6; H := (0, l1) shifted d6;
draw byPolygon(A,B,C,D)(byred);
byLineDefine(A, B, byyellow, 0, 0);
byLineDefine(A, D, byblue, 0, 0);
draw byNamedLineSeq(0)(AB,AD);
draw byPolygon(E,F,G,H)(byyellow);
byLineDefine(E, F, byblack, 0, 0);
byLineDefine(E, H, byred, 0, 0);
draw byNamedLineSeq(0)(EF,EH);
draw byLineWithName(Ab, Ae, byred, 0, 0)(A);
draw byLineWithName(Bb, Be, byblue, 0, 0)(B);
draw byLineWithName(Cb, Ce, byblack, 0, 0)(C);
draw byLineWithName(Db, De, byyellow, 0, 0)(D);
draw byLabelLine(0)(A, B, C, D);
draw byLabelPolygon(1)(ABCD);
draw byLabelPolygon(1)(EFGH);
}
\drawCurrentPictureInMargin
\problemNP{Е}{сли}{три прямые пропорциональны $\drawUnitLine{A} : \drawUnitLine{B} :: \drawUnitLine{B} : \drawUnitLine{C}$, прямоугольник заключенный между крайними равен квадрату средней.\\
И есть прямоугольник заключенный между крайними равен квадрату на средней, то три прямые пропорциональны.}

\startsubproposition[title={Часть I.}]
\startCenterAlign
$\eqalign{
\mbox{ Предположим } \drawUnitLine{D} &= \drawUnitLine{B}, \cr
\mbox{ и~поскольку } \drawUnitLine{A} : \drawUnitLine{B} &:: \drawUnitLine{B} : \drawUnitLine{C}, \cr
\mbox{ то } \drawUnitLine{A} : \drawUnitLine{B} &:: \drawUnitLine{D} : \drawUnitLine{C}, \cr
\therefore \drawUnitLine{A} \times \drawUnitLine{C} &= \drawUnitLine{B} \times \drawUnitLine{D} \cr
}$\\
\inprop[prop:VI.XVI].

Но $\drawUnitLine{D} = \drawUnitLine{B}$,\\
$\therefore \drawUnitLine{B} \times \drawUnitLine{D} = \drawUnitLine{B} \times \drawUnitLine{B} \mbox{ или } = \drawUnitLine{B}^2$,\\
следовательно, если три прямых пропорциональны, прямоугольник, заключенный между крайними равен квадрату средней.
\stopCenterAlign
\stopsubproposition

\startsubproposition[title={Часть II.}]
\startCenterAlign
Допустим $\drawUnitLine{D} = \drawUnitLine{B}$,\\
тогда $\drawUnitLine{A} \times \drawUnitLine{C} = \drawUnitLine{D} \times \drawUnitLine{B}$.

$\therefore \drawUnitLine{A} : \drawUnitLine{B} :: \drawUnitLine{D} : \drawUnitLine{C}$ \inprop[prop:VI.XVI],\\
и $\drawUnitLine{A} : \drawUnitLine{B} :: \drawUnitLine{B} : \drawUnitLine{C}$.
\stopCenterAlign
\stopsubproposition

\qed
\stopProposition

\startProposition[title={Предл. XVIII. Задача},reference=prop:VI.XVIII]
\defineNewPicture{
pair C, D, E, F, L, d;
C := (0, 0);
D := (3/2u, 0);
E := (u, 3u);
F := (-u, 2u);
L := (5/2u, u);
draw byPolygon(D,C,F)(byyellow);
draw byPolygon(D,F,E)(byblue);
draw byPolygon(D,E,L)(byred);
byAngleDefineExtended(D, F, E, byred, 1)(byblack);
byAngleDefineExtended(D, E, L, byyellow, 1)(byred);
byAngleDefine(D, C, F, byred, 1);
byAngleDefine(F, D, C, byblue, 1);
byAngleDefine(E, D, F, byblack, 1);
byAngleDefine(L, D, E, byyellow, 1);
draw byNamedAngleResized(DFE, DEL, DCF, FDC, EDF, LDE);
draw byLine(D, F, byred, 1, 0);
byLineDefine(D, E, byyellow, 1, 0);
byLineDefine(C, D, byblack, 1, 0);
byLineDefine(C, F, byblue, 0, 1);
byLineDefine(F, E, byyellow, 0, 0);
draw byNamedLineSeq(0)(DE,FE,CF,CD);
pair A, B, H, G, K;
numeric s;
s := 5/6;
d := (0, -7/2u);
A := (C scaled s) shifted d;
B := (D scaled s) shifted d;
H := (E scaled s) shifted d;
G := (F scaled s) shifted d;
K := (L scaled s) shifted d;
draw byPolygon(B,A,G)(byyellow);
draw byPolygon(B,G,H)(byblue);
draw byPolygon(B,H,K)(byred);
byAngleDefineExtended(B, G, H, byblack, 1)(byred);
byAngleDefineExtended(B, H, K, byyellow, 1)(byred);
byAngleDefine(B, A, G, byred, 0);
byAngleDefine(G, B, A, byblue, 0);
byAngleDefine(H, B, G, byblack, 0);
byAngleDefine(K, B, H, byyellow, 0);
draw byNamedAngleResized(BGH, BHK, BAG, GBA, HBG, KBH);
byLineDefine(B, G, byred, 0, 0);
byLineDefine(A, B, byblack, 0, 0);
byLineDefine(A, G, byblue, 0, 0);
byLineDefine(G, H, byblue, 1, 0);
draw byNamedLineSeq(0)(noLine,BG,AB,AG,GH);
draw byLabelsOnPolygon(F, E, L, D, C)(0, 0);
draw byLabelsOnPolygon(G, H, K, B, A)(0, 0);
}
\drawCurrentPictureInMargin
\problemNP[2]{Н}{а}{данной прямой \drawUnitLine{AB} построить подобную данной \drawPolygon[middle][polygonCD]{DCF,DFE,DEL} прямолинейную фигуру и~так же расположенную.}

\startCenterAlign
Разобьем фигуру на треугольники прямыми \drawUnitLine{DF} и~\drawUnitLine{DE}.

На концах \drawUnitLine{AB}\\
сделаем $\drawAngle{GBA} = \drawAngle{FDC}$ и~$\drawAngle{A} = \drawAngle{C}$;\\
теперь на концах \drawUnitLine{BG}\\
сделаем $\drawAngle{G} = \drawAngle{F}$ и~$\drawAngle{HBG} = \drawAngle{EDF}$;\\
и таким же образом\\
$\drawAngle{KBH} = \drawAngle{LDE}$ и~$\drawAngle{H} = \drawAngle{E}$.

Тогда $\polygonCD = \drawPolygon[middle][polygonAB]{BAG,BGH,BHK}$.

Из построения и~\inpropL[prop:I.XXXII] очевидно, что фигуры равноугольны, а~поскольку треугольники \drawPolygon{DCF} и~\drawPolygon{BAG} равноугольны,\\
то, согласно \inpropL[prop:VI.IV],
$\drawUnitLine{AB}:\drawUnitLine{AG} :: \drawUnitLine{CD} : \drawUnitLine{CF}$\\
и $\drawUnitLine{AG}:\drawUnitLine{BG} :: \drawUnitLine{CF} : \drawUnitLine{DF}$.

И поскольку \drawPolygon{DFE} и~\drawPolygon{BGH} равноугольны,\\
$\drawUnitLine{BG}:\drawUnitLine{GH} :: \drawUnitLine{DF} : \drawUnitLine{FE}$\\
$\therefore$ по равенству,\\
$\drawUnitLine{AG}:\drawUnitLine{GH} :: \drawUnitLine{CF} : \drawUnitLine{FE}$ \inprop[prop:VI.XXII].

Тем же способом можно показать, что оставшиеся стороны фигуры пропорциональны.

$\therefore$ \inprop[prop:VI.I]\\
\polygonAB\ подобна \polygonCD\ и~расположена и~на данной прямой \drawUnitLine{AB}.
\stopCenterAlign

\qed
\stopProposition

\startProposition[title={Предл. XIX. Теорема},reference=prop:VI.XIX]
\defineNewPicture[1/5]{
pair A, B, C, D, E, F, G, d;
numeric s;
A := (5/2u, -3u);
B := (0, 0);
C := (-u, ypart(A));
d := (0, -4u);
s := 3/4;
D := (A scaled s) shifted d;
E := (B scaled s) shifted d;
F := (C scaled s) shifted d;
G := B shifted (unitvector(C-B) scaled ((abs(E-F)/abs(B-C))*abs(E-F)));
draw byPolygon(A,B,G)(byblue);
draw byPolygon(A,G,C)(byred);
byAngleDefine(A, B, C, byred, 0);
draw byNamedAngleResized(ABC);
draw byLine(A, G, byyellow, 1, 0);
byLineDefine(A, B, byyellow, 0, 0);
byLineDefine(B, G, byblack, 1, 0);
byLineDefine(G, C, byblack, 0, 0);
draw byNamedLineSeq(0)(AB,BG,GC);
draw byPolygon(D,E,F)(byyellow);
byAngleDefine(D, E, F, byblack, 0);
draw byNamedAngleResized(DEF);
byLineDefine(D, E, byred, 0, 0);
byLineDefine(E, F, byblue, 0, 0);
draw byNamedLineSeq(0)(DE,EF);
draw byLabelsOnPolygon(F, E, D)(0, 0);
draw byLabelsOnPolygon(C, G, B, A)(0, 0);
}
\drawCurrentPictureInMargin
\problemNP{П}{одобные}{треугольники \drawPolygon[bottom][triangleDEF]{DEF} и~\drawPolygon[bottom][triangleABC]{ABG,AGC} находятся друг к~другу в~двойном отношении соответственных сторон.}

Пусть \drawAngle{E} и~\drawAngle{B} будут равными углами, и~\drawUnitLine{BG,GC} и~\drawUnitLine{EF} соответственными сторонами подобных треугольников \triangleDEF\ и~\triangleABC\ и~на стороне \drawUnitLine{BG,GC}
большего из них возьмем третью пропорциональную \drawUnitLine{BG} такую, что

\startCenterAlign
$\drawUnitLine{BG,GC} : \drawUnitLine{EF} :: \drawUnitLine{EF} : \drawUnitLine{BG}$.

Проведем \drawUnitLine{AG}.\\
$\drawUnitLine{BG,GC} : \drawUnitLine{AB} :: \drawUnitLine{EF} : \drawUnitLine{DE}$ \inprop[prop:VI.IV].

$\therefore \drawUnitLine{BG,GC} : \drawUnitLine{EF} :: \drawUnitLine{AB} : \drawUnitLine{DE}$ \inprop[prop:V.XVI].

Но $\drawUnitLine{BG,GC} : \drawUnitLine{EF} :: \drawUnitLine{EF} : \drawUnitLine{BG}$ (\conststr),\\
$\therefore \drawUnitLine{EF} : \drawUnitLine{BG} :: \drawUnitLine{AB} : \drawUnitLine{DE}$.

Следовательно $\triangleDEF = \drawPolygon[bottom][triangleABG]{ABG}$, поскольку у~них стороны при равных углах \drawAngle{E} и~\drawAngle{B} взаимно пропорциональны \inprop[prop:VI.XV].

$\therefore \triangleABC : \triangleDEF :: \triangleABC : \triangleABG$ \inprop[prop:V.VII].

Но $\triangleABC : \triangleABG :: \drawUnitLine{BG,GC} : \drawUnitLine{BG}$ \inprop[prop:VI.I].

$\therefore \triangleABC : \triangleDEF :: \drawUnitLine{BG,GC} : \drawUnitLine{BG}$.

То есть, эти треугольники относятся друг к~другу в~двойном отношении соответственных сторон \drawUnitLine{EF} и~\drawUnitLine{BG,GC} \indef[def:V.XI].
\stopCenterAlign

\qed
\stopProposition

\startProposition[title={Предл. XX. Теорема},reference=prop:VI.XX]
\defineNewPicture{
pair A, B, C, D, E;
A := (-2u, 5/2u);
B := (-1/2u, 0);
C := (3/2u, ypart(B));
D := (2u, 3/2u);
E := (u, ypart(A));
draw byPolygon(B,E,A)(byblue);
draw byPolygon(B,D,E)(byred);
draw byPolygon(B,C,D)(byyellow);
byAngleDefine(B, D, E, byblue, 0);
byAngleDefine(B, C, D, byblack, 0);
byAngleDefine(B, D, C, byred, 0);
draw byNamedAngleResized(BDE, BCD, BDC);
draw byLine(B, D, byblack, 1, 0);
byLineDefine(B, E, byblack, 0, 0);
byLineDefine(B, C, byblue, 0, 0);
byLineDefine(C, D, byblue, 1, 0);
byLineDefine(D, E, byyellow, 0, 0);
draw byNamedLineSeq(0)(BE,BC,CD,DE);
pair F, G, H, K, L, d;
numeric s;
s := 3/4;
d := (0, -3u);
F := (A scaled s) shifted d;
G := (B scaled s) shifted d;
H := (C scaled s) shifted d;
K := (D scaled s) shifted d;
L := (E scaled s) shifted d;
draw byPolygon(G,L,F)(byblue);
draw byPolygon(G,K,L)(byred);
draw byPolygon(G,H,K)(byyellow);
byAngleDefine(G, K, L, byblue, 1);
byAngleDefine(G, H, K, byblack, 1);
byAngleDefine(G, K, H, byred, 1);
draw byNamedAngleResized(GKL, GHK, GKH);
draw byLine(G, K, byblack, 1, 1);
byLineDefine(G, L, byblack, 0, 1);
byLineDefine(G, H, byred, 0, 1);
byLineDefine(H, K, byred, 1, 0);
byLineDefine(K, L, byyellow, 0, 0);
draw byNamedLineSeq(0)(GL,GH,HK,KL);
draw byLabelsOnPolygon(A, E, D, C, B)(0, 0);
draw byLabelsOnPolygon(F, L, K, H, G)(0, 0);
}
\drawCurrentPictureInMargin
\problemNP{П}{одобные}{многоугольники разделяются на равное количество подобных треугольников, каждая пара которых пропорциональна многоугольникам, а~многоугольники находятся друг к~другу в~двойном отношении соответственных сторон.}

Проведя \drawUnitLine{BE}, \drawUnitLine{BD}, \drawUnitLine{GL} и~\drawUnitLine{GK}, разделим многоугольники на треугольники. Поскольку многоугольники подобны, $\drawAngle{C} = \drawAngle{H}$, и~$\drawUnitLine{BC} : \drawUnitLine{CD} :: \drawUnitLine{GH} : \drawUnitLine{HK}$.

\startCenterAlign
$\therefore$ \drawPolygon{BCD} и~\drawPolygon{GHK} подобны и~$\drawAngle{BDC} = \drawAngle{GKH}$ \inprop[prop:VI.VI],\\
но $\drawAngle{BDE,BDC} = \drawAngle{GKL,GKH}$, поскольку они являются углами подобных многоугольников. Следовательно остатки \drawAngle{BDE} и~\drawAngle{GKL} равны.

А значит $\drawUnitLine{BD} : \drawUnitLine{CD} :: \drawUnitLine{GK} : \drawUnitLine{HK}$,\\
учитывая подобие треугольников,\\
и $\drawUnitLine{CD} : \drawUnitLine{DE} :: \drawUnitLine{HK} : \drawUnitLine{KL}$,\\
учитывая подобие многоугольников.

$\therefore \drawUnitLine{BD} : \drawUnitLine{DE} :: \drawUnitLine{GK} : \drawUnitLine{KL}$,\\
по равенству \inprop[prop:V.XXII], и~поскольку пропорциональные стороны заключают равные углы, треугольники \drawPolygon{BDE} и~\drawPolygon{GKL} подобны \inprop[prop:VI.VI].

Так же можно показать, что и~треугольники \drawPolygon{BEA} и~\drawPolygon{GLF} подобны.

Но \drawPolygon{BCD} имеет к~\drawPolygon{GHK} двойное отношение \drawUnitLine{BD} к~\drawUnitLine{GK} \inprop[prop:VI.XIX],\\
и \drawPolygon{BDE} к~\drawPolygon{GKL} точно так же имеет двойное отношение \drawUnitLine{BD} к~\drawUnitLine{GK}.

$\therefore \drawPolygon{BCD} : \drawPolygon{GHK} :: \drawPolygon{BDE} : \drawPolygon{GKL}$ \inprop[prop:V.XI].

Теперь \drawPolygon{BDE} к~\drawPolygon{GKL} имеет двойное отношение \drawUnitLine{BE} к~\drawUnitLine{GL}, и~\drawPolygon{BEA} к~\drawPolygon{GLF} двойное отношение \drawUnitLine{BE} к~\drawUnitLine{GL}.

$\eqalign{
\drawPolygon{BCD} : \drawPolygon{GHK} &:: \drawPolygon{BDE} : \drawPolygon{GKL} \cr
&:: \drawPolygon{BEA} : \drawPolygon{GLF} \cr
}$.
\stopCenterAlign

И поскольку как относится одно из предыдущих к~одному из последующих, так и~все предыдущие вместе ко всем последующим вместе, подобные треугольники относятся друг к~другу как многоугольники целиком \inprop[prop:V.XII].

\startCenterAlign
Но \drawPolygon{BCD} к~\drawPolygon{GHK} имеет двойное отношение \drawUnitLine{BC} к~\drawUnitLine{GH}.

$\therefore$ \drawPolygon{BEA,BDE,BCD} к~\drawPolygon{GLF,GKL,GHK} имеет двойное отношение \drawUnitLine{BC} к~\drawUnitLine{GH}.
\stopCenterAlign

\qed
\stopProposition

\startProposition[title={Предл. XXI. Теорема},reference=prop:VI.XXI]
\defineNewPicture[1/2]{
pair Aa, Ab, Ac, Ba, Bb, Bc, Ca, Cb, Cc, d[];
numeric s[];
Aa := (0, 0);
Ab := (0, 3/2u);
Ac := (-3u, 0);
s1 := 5/6;
d1 := (0, -2u);
Ba := (Aa scaled s1) shifted d1;
Bb := (Ab scaled s1) shifted d1;
Bc := (Ac scaled s1) shifted d1;
s2 := 4/6;
d2 := d1 shifted (0, -2u * s1) ;
Ca := (Aa scaled s2) shifted d2;
Cb := (Ab scaled s2) shifted d2;
Cc := (Ac scaled s2) shifted d2;
draw byPolygon.A(Aa,Ab,Ac)(byred);
draw byPolygon.B(Ba,Bb,Bc)(byblue);
draw byPolygon.C(Ca,Cb,Cc)(byyellow);
byPointLabelRemove(Aa, Ba, Ca, Ab, Bb, Cb, Ac, Bc, Cc);
}
\drawCurrentPictureInMargin
\problemNP{Ф}{игуры}{\drawPolygon[bottom]{A} и~\drawPolygon[bottom]{B} подобные одной фигуре \drawPolygon[bottom]{C} подобны друг другу.}

Поскольку \polygonA\ и~\polygonC\ подобны, они равноугольны, и~стороны при равных углах у~них  пропорциональны \indef[def:VI.I]. И~поскольку \polygonB\ и~\polygonC\ также подобны, и~стороны при равных углах у~них пропорциональны, \polygonA\ и~\polygonB\ тоже равноугольны и~у них тоже стороны при равных углах пропорциональны \inprop[prop:V.XI], и~значит они тоже подобны.

\qed
\stopProposition

\startProposition[title={Предл. XXII. Теорема},reference=prop:VI.XXII]
\defineNewPicture[1/2]{
pair A, B, C, D, E, F, G, H, Ob, Oe, Pb, Pe;
pair K, L, Ma, Mb, Mc, Md, Na, Nb, Nc, Nd;
pair d[];
numeric s[];
s1 := 6/5;
A := (0, 0); B := (3/2u, 0); K := (u, u);
d1 := (2u, 0);
C := (A scaled s1) shifted d1;
D := (B scaled s1) shifted d1;
L := (K scaled s1) shifted d1;
d2 := (4u, 0);
Ob := (A scaled (s1*s1)) shifted d2;
Oe := (B scaled (s1*s1)) shifted d2;
E := (0, 0); F := (u, 0); Ma := (3/2u, 2/3u); Mb := (u, 4/3u); Mc := (1/5u, 3/2u); Md := (-1/4u, 5/6u);
G := (E scaled s1) shifted d1;
H := (F scaled s1) shifted d1;
Na := (Ma scaled s1) shifted d1;
Nb := (Mb scaled s1) shifted d1;
Nc := (Mc scaled s1) shifted d1;
Nd := (Md scaled s1) shifted d1;
Pb := (E scaled (s1*s1)) shifted d2;
Pe := (F scaled (s1*s1)) shifted d2;
d3 := (1/4u, -5/2u);
forsuffixes i=E,F,Ma,Mb,Mc,Md,G,H,Na,Nb,Nc,Nd,Pb,Pe:
i := i shifted d3;
endfor;
draw byPolygon.AB(A,B,K)(byyellow);
draw byPolygon.CD(C,D,L)(byred);
draw byPolygon.EF(E,F,Ma,Mb,Mc,Md)(byblue);
draw byPolygon.GH(G,H,Na,Nb,Nc,Nd)(white);
draw byLine(A, B, byblack, 0, 0);
draw byLine(C, D, byblue, 0, 0);
draw byLineWithName(Ob, Oe, byblack, 1, 0)(O);
draw byLine(E, F, byred, 0, 0);
draw byLine(G, H, byyellow, 0, 0);
draw byLineWithName(Pb, Pe, byred, 1, 0)(P);
byPointLabelRemove(Ma, Mb, Mc, Md, Na, Nb, Nc, Nd, K, L);
draw byLabelLine(1)(AB,CD,EF,GH,O,P);
}
\drawCurrentPicture
\initialIndentation{10}
\problem{Е}{сли}{четыре прямых пропорциональны $\drawUnitLine{AB} : \drawUnitLine{CD} :: \drawUnitLine{EF} : \drawUnitLine{GH}$, подобные прямолинейные фигуры подобным образом расположенные на них также пропорциональны.\\
И если четыре подобных прямолинейных фигуры подобно расположены на четырех прямых линиях, то и~линии пропорциональны.}

\startsubproposition[title={Часть I.}]
\startCenterAlign
Возьмем \drawUnitLine{O}, третью пропорциональную к~\drawUnitLine{AB} и~\drawUnitLine{CD}, и~\drawUnitLine{P}, третью пропорциональную к~\drawUnitLine{EF} и~\drawUnitLine{GH} \inprop[prop:VI.XI].

Поскольку $\drawUnitLine{AB} : \drawUnitLine{CD} :: \drawUnitLine{EF} : \drawUnitLine{GH}$ (\hypstr)\\
$\drawUnitLine{CD} : \drawUnitLine{O} :: \drawUnitLine{GH} : \drawUnitLine{P}$ (\conststr).

$\therefore$ по равенству,\\
$\drawUnitLine{AB} : \drawUnitLine{O} :: \drawUnitLine{EF} : \drawUnitLine{P}$,\\
но $\drawPolygon{AB} : \drawPolygon{CD} :: \drawUnitLine{AB} : \drawUnitLine{O}$ \inprop[prop:VI.XX],\\
и $\drawPolygon{EF} : \drawPolygon{GH} :: \drawUnitLine{EF} : \drawUnitLine{P}$.

$\therefore \drawPolygon{AB} : \drawPolygon{CD} :: \drawPolygon{EF} : \drawPolygon{GH}$ \inprop[prop:V.XI].
\stopCenterAlign
\stopsubproposition

\vfill\pagebreak

\startsubproposition[title={Часть II.}]
\startCenterAlign
Оставим то же построение.\\
$\drawPolygon{AB} : \drawPolygon{CD} :: \drawPolygon{EF} : \drawPolygon{GH}$ (\hypstr).

$\therefore \drawUnitLine{AB} : \drawUnitLine{O} :: \drawUnitLine{EF} : \drawUnitLine{P}$ (\conststr).

И $\therefore \drawUnitLine{AB} : \drawUnitLine{CD} :: \drawUnitLine{EF} : \drawUnitLine{GH}$ \inprop[prop:V.XI].
\stopCenterAlign
\stopsubproposition

\qed
\stopProposition

\startProposition[title={Предл. XXIII. Теорема},reference=prop:VI.XXIII]
\defineNewPicture{
pair A, B, C, D, E, F, G ,H, d[];
d1 := (u, 0);
d2 := d1 scaled -2;
d3 := (1/2u, 2u);
d4 := d3 scaled -2/3;
C := (0, 0);
G := C shifted d1;
B := C shifted d2;
E := C shifted d3;
D := C shifted d4;
H := C shifted d1 shifted d4;
A := C shifted d2 shifted d4;
F := C shifted d1 shifted d3;
draw byPolygon(A,B,C,D)(byyellow);
draw byPolygon(E,F,G,C)(byblue);
draw byPolygon(C,D,H,G)(byred);
byAngleDefine(B, C, D, byred, 0);
byAngleDefine(D, C, G, byyellow, 0);
byAngleDefine(G, C, E, byblack, 0);
draw byNamedAngleResized();
draw byLine(D, C, byblack, 0, 0);
draw byLine(C, E, byred, 0, 0);
draw byLine(B, C, byblue, 0, 0);
draw byLine(C, G, byyellow, 0, 0);
draw byLabelsOnPolygon(D, A, B, C, E, F, G, H)(0, 0);
}
\drawCurrentPictureInMargin
\problemNP[2]{Р}{авноугольные}{параллелограммы \drawPolygon[middle]{ABCD} и~\drawPolygon[middle]{EFGC} имеют друг к~другу составное отношение их сторон.}

Пусть две стороны при равных углах \drawUnitLine{BC} и~\drawUnitLine{CG} будут расположены так, что образуют одну прямую.

\startCenterAlign
Поскольку $\drawAngle{BCD} + \drawAngle{DCG} = \drawTwoRightAngles$,\\
и $\drawAngle{GCE} = \drawAngle{BCD}$ (\hypstr),\\
$\drawAngle{GCE} + \drawAngle{DCG} = \drawTwoRightAngles$,\\
и $\therefore$ \drawUnitLine{CE} и~\drawUnitLine{DC} образуют одну прямую \inprop[prop:I.XIV].

Достроим \drawPolygon[middle]{CDHG}.

Поскольку $\polygonABCD\ : \polygonCDHG\ :: \drawUnitLine{BC} : \drawUnitLine{CG}$ \inprop[prop:VI.I],\\
и $\polygonCDHG\ : \polygonEFGC\ :: \drawUnitLine{DC} : \drawUnitLine{CE}$ \inprop[prop:VI.I],\\
\polygonABCD\ имеет к~\polygonEFGC\ отношение, составленное из отношений \drawUnitLine{BC} к~\drawUnitLine{CG}, и~\drawUnitLine{DC} к~\drawUnitLine{CE}.
\stopCenterAlign

\qed
\stopProposition

\startProposition[title={Предл. XXIV. Теорема},reference=prop:VI.XXIV]
\defineNewPicture[1/4]{
pair A, B, C, D, E, F, G, H, K, d[];
numeric s;
d1 := (3u, 0);
d2 := (u, 3u);
s := 3/5;
A := (0, 0);
B := A shifted d1;
C := A shifted d1 shifted d2;
D := A shifted d2;
G := s[A, D];
E := s[A, B];
H := s[B, C];
K := s[D, C];
F = whatever[G, H] = whatever[E, K];
draw byLine(F, H, byblack, 0, 1);
draw byLine(E, F, byblack, 0, 1);
draw byPolygon(G,F,K,D)(byyellow);
draw byPolygon(A,G,F)(byblue);
draw byPolygon(F,C,K)(byred);
draw byLine(G, F, byred, 0, 0);
byLineDefine(A, E, byblack, 0, 1);
byLineDefine(E, B, byblack, 0, 1);
byLineDefine(B, H, byblack, 0, 1);
byLineDefine(H, C, byblack, 0, 1);
byLineDefine(D, K, byblue, 0, 0);
byLineDefine(K, C, byblue, 1, 0);
byLineDefine(D, G, byred, 1, 0);
byLineDefine(G, A, byyellow, 0, 0);
draw byNamedLineSeq(0)(GA,DG,DK,KC,HC,BH,EB,AE);
draw byLabelsOnPolygon(A, G, D, K, C, H, B, E)(0, 0);
draw byLabelsOnPolygon(H, F, E)(2, 0);
}
\drawCurrentPictureInMargin
\problemNP[2]{В}{о}{всяком параллелограмме
\drawFromCurrentPicture[middle][parallelogramABCD]{
draw byNamedLine(AE,EB,BH,HC,FH,EF);
draw byNamedPolygon(GFKD,AGF,FCK);
draw byLabelsOnPolygon(D, C, B, A)(0, 0);
}
параллелограммы
\drawFromCurrentPicture[middle][parallelogramFHCK]{
draw byNamedLine(FH,HC);
draw byNamedPolygon(FCK);
draw byLabelsOnPolygon(K, C, H, F)(0, 0);
}
и
\drawFromCurrentPicture[middle][parallelogramAEFG]{
draw byNamedLine(AE,EF);
draw byNamedPolygon(AGF);
draw byLabelsOnPolygon(G, F, E, A)(0, 0);
}
на диагонали подобны целому и~между собой.}

\startCenterAlign
Поскольку у~\parallelogramABCD\ и~\parallelogramAEFG\ есть общий угол, они равноугольны.\\
Но поскольку $\drawUnitLine{GF} \parallel \drawUnitLine{DK,KC}$\\
\drawFromCurrentPicture[bottom]{
startGlobalRotation(180-angle(A-C));
startAutoLabeling;
draw byNamedPolygon(AGF);
stopAutoLabeling;
stopGlobalRotation;
}
и
\drawFromCurrentPicture[bottom]{
startGlobalRotation(180-angle(A-C));
startAutoLabeling;
draw byNamedPolygon(GFKD,AGF,FCK);
stopAutoLabeling;
stopGlobalRotation;
}
подобны \inprop[prop:VI.IV],\\
$\therefore \drawUnitLine{GA} : \drawUnitLine{GF} :: \drawUnitLine{GA,DG} : \drawUnitLine{DK,KC}$,
а оставшиеся противоположные стороны равны им.

$\therefore$ у~\parallelogramAEFG\ и~\parallelogramABCD\ стороны при равных углах пропорциональны и~значит, они подобны.

Так же можно показать, что параллелограммы \parallelogramABCD\ и~\parallelogramFHCK\ подобны.

И так как оба \parallelogramAEFG\ и~\parallelogramFHCK\ подобны \parallelogramABCD, они подобны и~друг другу.
\stopCenterAlign

\qed
\stopProposition

\startProposition[title={Предл. XXV. Задача},reference=prop:VI.XXV]
\defineNewPicture{
pair d[];
pair A, B, C;
A := (5/3u, 2u);
B := (0, 0);
C := (9/4u, 0);
draw byPolygon(A,B,C)(byred);
pair L, E;
L := (xpart(B), -1/2ypart(A));
E := (xpart(C), ypart(L));
draw byPolygon(B,C,E,L)(byblue);
byAngleDefine(C, B, L, byred, 0);
draw byNamedAngleResized(CBL);
pair Da, Db, Dc, Dd, De;
Da := dir(0) scaled 1/2u;
Db := dir(72) scaled 2/5u;
Dc := dir(144) scaled 1/2u;
Dd := dir(-144) scaled 3/5u;
De := dir(-72) scaled 1/2u;
d1 := (3u, 3/2u);
byPointLabelRemove(Da,Db,Dc,Dd,De);
forsuffixes i=Da,Db,Dc,Dd,De:
	i := ((i rotated 36) scaled 3/2) shifted d1;
endfor;
draw byPolygon.D(Da,Db,Dc,Dd,De)(byblue);
pair F, M;
numeric a, l;
a := 	((abs(Da-Db)*distanceToLine(Dc, Da--Db))+
	(abs(Da-Dc)*distanceToLine(Dd, Da--Dc))+
	(abs(Da-Dd)*distanceToLine(De, Da--Dd)))/2;
l := a/abs(C-E);
F := C shifted (l, 0);
M := E shifted (l, 0);
draw byPolygon(C,F,M,E)(white);
byAngleDefine(F, C, E, byyellow, 0);
draw byNamedAngleResized(FCE);
draw byLine(C, E, byred, 0, 0);
draw byLine(B, C, byyellow, 0, 0);
draw byLine(C, F, byblack, 1, 0);
pair K, G, H;
numeric s;
s := (a/(abs(B-C)*abs(B-L)))**(1/2);
d2 := (2/3u, -3u);
K := (A scaled s) shifted d2;
G := (B scaled s) shifted d2;
H := (C scaled s) shifted d2;
draw byPolygon(K,G,H)(byyellow);
draw byLine(G, H, byblue, 0, 0);
draw byLabelsOnPolygon(C, F, M, E, L, B, A)(0, 0);
draw byLabelsOnPolygon(G, K, H)(0, 0);
}
\drawCurrentPictureInMargin
\problemNP[2]{П}{остроить}{прямолинейную фигуру подобную данной \hskip -11pt \drawPolygon[middle][triangleABC]{ABC} \hskip -9pt и~равную другой данной \drawPolygon[middle][polygonD]{D}.}

\startCenterAlign
На \drawUnitLine{BC} опишем $\drawPolygon[middle][polygonBCEL]{BCEL} = \triangleABC$,\\
и на \drawUnitLine{CE} опишем
$\drawFromCurrentPicture[middle][polygonCFME]{
startTempAngleScale(angleScale*1/2);
draw byNamedAngle(C);
startAutoLabeling;
draw byNamedPolygon(CFME);
stopAutoLabeling;
stopTempAngleScale;
} = \polygonD$,\\
с $\drawAngle{B} = \drawAngle{C}$ \inprop[prop:I.XLV],\\
и тогда \drawUnitLine{BC} и~\drawUnitLine{CF} будут лежать на одной прямой (\inpropL[prop:I.XXIX], \inpropL[prop:I.XIV]).

Найдем между \drawUnitLine{BC} и~\drawUnitLine{CF} среднюю пропорциональную \drawUnitLine{GH} \inprop[prop:VI.XIII],\\
и на \drawUnitLine{GH} опишем \drawPolygon[middle][triangleKGH]{KGH}, подобный \triangleABC, и~подобно расположенный.

Тогда $\triangleKGH\ = \polygonD$.

Поскольку \triangleABC\ и~\triangleKGH\ подобны\\
и $\drawUnitLine{BC} : \drawUnitLine{GH} :: \drawUnitLine{GH} : \drawUnitLine{CF}$ (\conststr),\\
$\triangleABC\ : \triangleKGH\ :: \drawUnitLine{BC} : \drawUnitLine{CF}$ \inprop[prop:VI.XX].

Но $\polygonBCEL\ : \polygonCFME\ :: \drawUnitLine{BC} : \drawUnitLine{CF}$ \inprop[prop:VI.I].

$\therefore \triangleABC\ : \triangleKGH\ :: \polygonBCEL\ : \polygonCFME$ \inprop[prop:V.XI].

Но $\triangleABC\ = \polygonBCEL$ (\conststr),\\
и $\therefore \triangleKGH\ = \polygonCFME$ \inprop[prop:V.XIV];\\
и $\polygonCFME\ = \polygonD$ (\conststr).

И, следовательно, \triangleKGH\ подобный \triangleABC\ вместе с~тем $= \polygonD$.
\stopCenterAlign

\qed
\stopProposition

\startProposition[title={Предл. XXVI. Теорема},reference=prop:VI.XXVI]
\defineNewPicture[1/4]{
pair A, B, C, D, E, F, G, H, K, d[];
d1 := (3u, 0);
d2 := (u, 3u);
A := (0, 0);
B := A shifted d1;
C := A shifted d1 shifted d2;
D := A shifted d2;
F := 3/4[A, C];
E = whatever[A, B] = whatever[F, F shifted d2];
G = whatever[A, D] = whatever[F, F shifted d1];
H := 2/3[G, F];
K = whatever[A, B] = whatever[H, H shifted d2];
draw byPolygon(F,H,K,E)(byred);
draw byPolygon(G,D,C,B,E,F)(byblue);
byAngleDefine(B, A, D, byyellow, 0);
draw byNamedAngleResized();
draw byLine(K, H, byyellow, 0, 0);
draw byLine(G, H, byyellow, 1, 0);
draw byLine(A, C, byblack, 0, 1);
byLineDefine(A, G, byred, 0, 0);
byLineDefine(G, D, byred, 1, 0);
byLineDefine(A, K, byblue, 0, 0);
byLineDefine(K, E, byblue, 1, 0);
byLineDefine(E, B, byblack, 1, 0);
draw byNamedLineSeq(0)(GD,AG,AK,KE,EB);
draw byArbitraryFigure.AHC(A..H..C, byblack, 0, 0);
byLineDefine.KAt(K, A, byblack, 0, 1);
byLineStylize(H, G, 1, 0, 1)(KAt);
byLineDefine.AGt(A, G, byblack, 0, 1);
byLineStylize(K, H, 0, 0, 1)(AGt);
byLineDefine.GHt(G, H, byblack, 0, 1);
byLineStylize(A, K, 0, 1, 1)(GHt);
byLineDefine.EAt(E, A, byblack, 0, 1);
byLineStylize(F, G, 1, 0, 1)(EAt);
draw byLabelsOnPolygon(A, G, D, C, B, E, K)(0, 0);
draw byLabelsOnPolygon(K, H, A)(2, 0);
draw byLabelsOnPolygon(E, F, A)(2, 0);
}
\drawCurrentPictureInMargin
\problemNP[3]{Е}{сли}{у подобных и~подобно расположенных параллелограммов
\drawFromCurrentPicture[middle][parallelogramAEFG]{
draw byNamedPolygon(FHKE);
draw byNamedLine(KAt,AGt,GHt);
draw byLabelsOnPolygon(A, G, F, E)(0, 0);
}
и
\drawFromCurrentPicture[middle][parallelogramABCD]{
draw byNamedPolygon(GDCBEF);
draw byNamedLine(EAt,AC);
draw byNamedLineFull(E, F, 0, 1, 1)(AGt);
draw byLabelsOnPolygon(A, D, C, B)(0, 0);
}
есть общий угол, они расположены на одной диагонали.}

\startCenterAlign
Действительно, если возможно, пусть
\drawFromCurrentPicture[middle][lineAHC]{
startGlobalRotation(180-angle(A-C));
draw byNamedArbitraryFigure(AHC);
draw byLabelsOnPolygon(A, H, C, noPoint)(0, 0);
stopGlobalRotation;
}
будет диагональю \parallelogramABCD\ и~проведем $\drawUnitLine{KH} \parallel \drawUnitLine{AG}$ \inprop[prop:I.XXXI].

Поскольку \drawLine[middle][parallelogramAKHG]{AG,GH,KH,AK} и~\parallelogramABCD\ на одной диагонали \lineAHC, и~\drawAngle{A} уних общий, они подобны \inprop[prop:VI.XXIV].

$\therefore \drawSizedLine{AG} : \drawSizedLine{AK} :: \drawSizedLine{AG,GD} : \drawSizedLine{AK,KE,EB}$;\\
но $\drawSizedLine{AG} : \drawSizedLine{AK,KE} :: \drawSizedLine{AG,GD} : \drawSizedLine{AK,KE,EB}$ (\hypstr).

$\therefore \drawSizedLine{AG} : \drawSizedLine{AK} :: \drawSizedLine{AG} : \drawSizedLine{AK,KE}$,\\
и $\therefore \drawSizedLine{AK} = \drawSizedLine{AK,KE}$ \inprop[prop:V.IX], что невозможно.

$\therefore$ \lineAHC\ не является диагональю \parallelogramABCD\ и~таким же образом можно показать, что диагональю не является никакая другая прямая, кроме \drawUnitLine{AC}.
\stopCenterAlign

\qed
\stopProposition

\startProposition[title={Предл. XXVII. Теорема},reference=prop:VI.XXVII]
\defineNewPicture[1/2]{
pair A, B, C, D, E, F, G, H, K;
numeric l, s;
l := 5u;
s := 1/5;
A := (0, 0);
B := (xpart(A), -l);
C := 1/2[A, B];
D := s[A, B];
E := (-1/2l, ypart(C));
F := (xpart(E), ypart(A));
G := (-s*l, ypart(D));
H := (xpart(G), ypart(B));
K = whatever[C, E] = whatever[G, H];
draw byPolygon(A,D,G,K,E,F)(byred);
draw byPolygon(C,D,G,K)(byblue);
draw byPolygon(B,C,K,H)(byyellow);
draw byLine(A, D, byyellow, 0, 0);
draw byLine(D, C, byred, 0, 0);
draw byLine(C, B, byblue, 0, 0);
draw byLabelsOnPolygon(E, F, A, D, C, B, H)(0, 0);
draw byLabelsOnPolygon(H, G, D)(2, 0);
}
\drawCurrentPictureInMargin
\problemNP{И}{з}{всех прямоугольников, заключенных между частями данной прямой линии, наибольшим является квадрат, описанный на половине линии.}

\startCenterAlign
Пусть \drawSizedLine{AD,DC,CB} будет данной прямой линией,\\
\drawSizedLine{AD} и~\drawSizedLine{DC,CB} неравными частями,\\
а \drawSizedLine{AD,DC} и~\drawSizedLine{CB} равными частями.

Тогда $\drawPolygon[middle]{ADGKEF,CDGK} > \drawPolygon[middle]{BCKH,CDGK}$.
\stopCenterAlign

Поскольку, как уже было показано \inprop[prop:II.V], квадрат половины линии равен прямоугольнику, заключенному между неравными частями вместе с~квадратом на части между серединой и~точкой неравного рассечения. Квадрат описанный на половине линии превосходит, таким образом, прямоугольник, заключенный между неравными частями линии.

\qed
\stopProposition

\startProposition[title={Предл. XXVIII. Задача},reference=prop:VI.XXVIII]
\defineNewPicture{
pair A, B, C, D, E, F, G, H, d;
path a;
numeric r;
A := (0, 0);
B := (4u, 0);
C := 1/2[A, B];
D := C shifted (0, 3/2u);
r := 2u;
a := (fullcircle scaled (2r)) shifted D;
E := a intersectionpoint (A--C);
F := a intersectionpoint (D--2[D, C]);
byLineDefine(D, E, byyellow, 0, 0);
byLineDefine(D, C, byred, 0, 0);
byLineDefine(C, F, byblack, 1, 0);
draw byNamedLineSeq(0)(CF,DC,DE);
draw byLine(A, E, byred, 1, 0);
draw byLine(E, C, byblue, 0, 0);
draw byLine(C, B, byblue, 1, 0);
draw byArcBE.a(D, -1/2, -4 + 1/2, r, byred, 0, 0, 0, 0);
d := (0, 2u);
G := A shifted d;
H := C shifted d;
byLineDefine.G(G, H, byyellow, 1, 0);
lineUseLineLabel.G := true;
draw byLabelsOnCircle(F)(a);
draw byLabelsOnPolygon(E, D, C, B)(2, 0);
draw byLabelsOnPolygon(A, B, noPoint)(0, 0);
draw byLabelPoint(E, angle(E-D) + 45, 2);
}
\drawCurrentPictureInMargin
\problemNP[5]{Р}{азделить}{данную прямую \drawSizedLine{AE,EC,CB} так, чтобы прямоугольник, заключенный между частями был равен данной площади, не превосходящей квадрат половины данной прямой.}

\startCenterAlign
Пусть данная площадь будет $=\drawSizedLine{G}^2$.\\
Рассечем пополам \drawSizedLine{AE,EC,CB}, или сделаем $\drawSizedLine{AE,EC} = \drawSizedLine{CB}$,\\
и если $\drawSizedLine{AE,EC}^2 = \drawSizedLine{G}^2$, то задача решена.

Но если $\drawSizedLine{AE,EC}^2 \neq \drawSizedLine{G}^2$,\\
тогда $\drawSizedLine{AE,EC} > \drawSizedLine{G}$ (\hypstr).

Проведем $\drawSizedLine{DC} \perp \drawSizedLine{AE,EC} = \drawSizedLine{G}$,\\
сделаем $\drawSizedLine{DC,CF} = \drawSizedLine{AE,EC} \mbox{ или } \drawSizedLine{CB}$,\\
с \drawSizedLine{DC,CF} в~качестве радиуса опишем круг, секущий данную прямую, проведем \drawSizedLine{DE}.

Тогда $\drawSizedLine{AE} \times \drawSizedLine{EC,CB} + \drawSizedLine{EC}^2 = \drawSizedLine{AE,EC}^2$ \inprop[prop:II.V] $= \drawSizedLine{DE}^2$.

Но $\drawSizedLine{DE}^2 = \drawSizedLine{DC}^2 + \drawSizedLine{EC}^2$ \inprop[prop:I.XLVII];

$\therefore \drawSizedLine{AE} \times \drawSizedLine{EC,CB} + \drawSizedLine{EC}^2 = \drawSizedLine{DC}^2 + \drawSizedLine{EC}^2$,\\
из обеих вычтем $\drawSizedLine{EC}^2$,\\
и $\drawSizedLine{AE} \times \drawSizedLine{EC,CB} = \drawSizedLine{DC}^2$.

Но $\drawSizedLine{DC} = \drawSizedLine{G}$ (\conststr),\\
и $\therefore$ \drawSizedLine{AE,EC,CB} рассечена так, что $\drawSizedLine{AE} \times \drawSizedLine{EC,CB} = \drawSizedLine{G}^2$.
\stopCenterAlign

\qed
\stopProposition

\startProposition[title={Предл. XXIX. Задача},reference=prop:VI.XXIX]
\defineNewPicture[1/2]{
pair A, B, C, D, E, F, G, H, d;
d := (0, -1/2u);
G := (0, 0) shifted d;
H := (3/2u, 0) shifted d;
A := (0, 0);
B := (2u, ypart(A));
C := 1/2[A, B];
D := (xpart(B), abs(G-H));
E := C shifted (-abs(C-D), 0);
F := C shifted (abs(C-D), 0);
draw byLineWithName(H, G, byblack, 0, 0)(G);
draw byLineFull(C, D, byred, 0, 0)(B, D, 1, 0, 0);
draw byLine(B, D, byred, 1, 0);
draw byLineFull(E, A, byyellow, 1, 0)(E, A, 0, 0, -1);
draw byLineFull(A, C, byblue, 0, 0)(A, C, 0, 0, -1);
draw byLineFull(C, B, byblue, 1, 0)(C, B, 0, 0, -1);
draw byLineFull(B, F, byyellow, 0, 0)(B, F, 0, 0, -1);
draw byArcBE.a(C, 0, 4, abs(C-D), byred, 0, 0, 0, 0);
draw byLabelsOnCircle(D)(a);
draw byLabelsOnPolygon(F, B, C, A, noPoint)(0, 0);
draw byLabelLine(0)(G);
}
\drawCurrentPictureInMargin
\problemNP{П}{родлить}{данную прямую \drawSizedLine{AC,CB} так, чтобы прямоугольник, заключенный между прямыми между концами данной линии и~точкой, до которой она продлена, был равен данной площади, то есть квадрату на \drawSizedLine{G}.}

\startCenterAlign
Сделаем $\drawSizedLine{AC} = \drawSizedLine{CB}$,\\
и проведем $\drawSizedLine{BD} \perp \drawSizedLine{CB} = \drawSizedLine{G}$,\\
проведем \drawSizedLine{CD},\\
и с~\drawSizedLine{CD} в~качестве радиуса, опишем круг, пересекающий продленную \drawSizedLine{AC,CB}.

Тогда $\drawSizedLine{AC,CB,BF} \times \drawSizedLine{BF} + \drawSizedLine{CB}^2 = \drawSizedLine{AC,CB}^2$ \inprop[prop:II.VI] $= \drawSizedLine{CD}^2$.

Но $\drawSizedLine{CD}^2 = \drawSizedLine{BD}^2 + \drawSizedLine{CB}^2$ \inprop[prop:I.XLVII].

$\therefore \drawSizedLine{AC,CB,BF} \times \drawSizedLine{BF} + \drawSizedLine{CB}^2 = \drawSizedLine{BD}^2 + \drawSizedLine{CB}^2$,\\
из обеих вычтем $\drawSizedLine{CB}^2$,\\
и $\therefore \drawSizedLine{AC,CB,BF} \times \drawSizedLine{BF}= \drawSizedLine{BD}^2$\\
но $\drawSizedLine{BD} = \drawSizedLine{G}$.

$\therefore \drawSizedLine{BD}^2 = \mbox{ данной площади.}$
\stopCenterAlign

\qed
\stopProposition

\startProposition[title={Предл. XXX. Задача},reference=prop:VI.XXX]
\defineNewPicture[1/4]{
pair A, B, C, D, E, F, G, H;
numeric w;
w := 3u;
A := (0, 0);
B := (w, 0);
C := (0, w);
H := (w, w);
G := 1/2[A, C] shifted (0, -abs((1/2[A, C]) - B));
D := G shifted (abs(G-A), 0);
E = whatever[D, D shifted (0, 1)] = whatever[A, B];
F = whatever[C, H] = whatever[D, E];
draw byPolygon(A,C,F,E)(byyellow);
draw byPolygon(E,F,H,B)(byblue);
draw byLine(A, E, byred, 0, 0);
draw byLine(E, B, byred, 1, 0);
byLineDefine(C, A, byblue, 0, 0);
byLineDefine(A, G, byblue, 1, 0);
byLineDefine(G, D, byyellow, 0, 0);
byLineDefine(D, E, byblack, 0, 0);
draw byNamedLineSeq(0)(CA,DE,GD,AG);
byLineDefine.AGt(A, G, byblack, 0, 1);
byLineStylize(A, D, 1, 0, -1)(AGt);
byLineDefine.GDt(G, D, byblack, 0, 1);
byLineStylize(A, E, 0, 0, -1)(GDt);
byLineDefine.DEt(D, E, byblack, 0, 1);
byLineStylize(G, E, 0, 1, -1)(DEt);
draw byLabelsOnPolygon(A, C, F, H, B, E, D, G)(0, 0);
}
\drawCurrentPictureInMargin
\problemNP{Р}{ассечь}{данную прямую \drawProportionalLine{AE,EB} в~крайнем и~среднем отношении.}

\startCenterAlign
На \drawProportionalLine{AE,EB} опишем квадрат \drawPolygon[middle][squareABHC]{ACFE,EFHB} \inprop[prop:I.XLVI].

И продлим \drawProportionalLine{CA}, так, что $\drawProportionalLine{CA,AG} \times \drawProportionalLine{AG} = \drawProportionalLine{AE,EB}^2$ \inprop[prop:VI.XXIX].

Возьмем $\drawProportionalLine{AE} = \drawProportionalLine{AG}$,\\
и проведем $\drawProportionalLine{DE} \parallel \drawProportionalLine{CA,AG}$,\\
встречающуюся с~$\drawProportionalLine{GD} \parallel \drawProportionalLine{AE,EB}$ \inprop[prop:I.XXXI].

Тогда $\drawFromCurrentPicture[middle][rectangleCFDG]{
draw byNamedPolygon(ACFE);
draw byNamedLine(AGt,GDt,DEt);
draw byLabelsOnPolygon(G, C, F, D)(0, 0);
}
= \drawProportionalLine{CA,AG} \times \drawProportionalLine{AG}$, и~$\therefore\ = \squareABHC$.

И если из обеих частей вычесть общую \drawPolygon{ACFE},\\
\drawLine{DE,GD,AG,AE}, являющаяся квадратом \drawProportionalLine{AE},\\
будет $= \drawPolygon{EFHB}$, которая $= \drawProportionalLine{AE,EB} \times \drawProportionalLine{EB}$.

То есть $\drawProportionalLine{AE}^2 = \drawProportionalLine{AE,EB} \times \drawProportionalLine{EB}$.

$\therefore \drawProportionalLine{AE,EB} : \drawProportionalLine{AE} :: \drawProportionalLine{AE} : \drawProportionalLine{EB}$,\\
и \drawProportionalLine{AE,EB} разделена в~крайнем и~среднем отношении \indef[def:VI.III].
\stopCenterAlign

\qed
\stopProposition

\startProposition[title={Предл. XXXI. Теорема},reference=prop:VI.XXXI]
\defineNewPicture{
pair A, B, C, D, E, F, G, H, K, L;
numeric a, r, l[];
a := -125;
A := (0, 0);
B := A shifted (dir(a)*2u);
C = whatever[A, A shifted dir(a+90)] = whatever[B, B shifted dir(0)];
D = whatever[A, A shifted dir(-90)] = whatever[B, C];
l1 := abs(A-B);
l2 := abs(B-C);
l3 := abs(C-A);
r := 1/4;
E := A shifted (dir(a-90)*l1*r);
F := B shifted (dir(a-90)*l1*r);
G := B shifted (dir(-90)*l2*r);
H := C shifted (dir(-90)*l2*r);
K := C shifted (dir(a + 180)*l3*r);
L := A shifted (dir(a + 180)*l3*r);
draw byPolygon.AB(A,B,F,E)(byblue);
draw byPolygon.BC(B,C,H,G)(byred);
draw byPolygon.CA(C,A,L,K)(byyellow);
draw byLine(A, D, byblack, 0, 0);
byLineDefine(A, B, byyellow, 0, 0);
byLineDefine(B, D, byblue, 1, 0);
byLineDefine(D, C, byblue, 0, 0);
byLineDefine(C, A, byred, 0, 0);
draw byNamedLineSeq(-1)(AB,BD,DC,CA);
byPointLabelRemove(H, G, L, K, F, E);
draw byLabelsOnPolygon(B, F, E, A, L, K, C, H, G)(0, 0);
draw byLabelsOnPolygon(A, D, C)(2, 0);
}
\drawCurrentPictureInMargin
\problemNP{Е}{сли}{подобные и~подобным образом описанные прямолинейные фигуры построены на сторонах прямоугольного треугольника \drawLine[bottom]{CA,DC,BD,AB}, фигура на стороне \drawProportionalLine{BD,DC} против прямого угла равна сумме фигур на двух других сторонах.}

\startCenterAlign
Из прямого угла проведем перпендикуляр \drawProportionalLine{AD} к~\drawProportionalLine{BD,DC}.

Тогда $\drawProportionalLine{BD,DC} : \drawProportionalLine{CA} :: \drawProportionalLine{CA} : \drawProportionalLine{DC}$ \inprop[prop:VI.VIII].

$\therefore
\drawFromCurrentPicture[bottom][figBC]{
startGlobalRotation(180-angle(B-C));
startAutoLabeling;
draw byNamedPolygon(BC);
stopAutoLabeling;
stopGlobalRotation;
} :
\drawFromCurrentPicture[bottom][figCA]{
startGlobalRotation(-angle(C-A));
startAutoLabeling;
draw byNamedPolygon(CA);
stopAutoLabeling;
stopGlobalRotation;
} :: \drawProportionalLine{BD,DC} : \drawProportionalLine{DC}$ \inprop[prop:VI.XX].

Но $\figBC\ :
\drawFromCurrentPicture[bottom][figAB]{
startGlobalRotation(-angle(A-B));
startAutoLabeling;
draw byNamedPolygon(AB);
stopAutoLabeling;
stopGlobalRotation;
} :: \drawProportionalLine{BD,DC} : \drawProportionalLine{BD}$ \inprop[prop:VI.XX].

Значит $\drawProportionalLine{DC} + \drawProportionalLine{BD} : \drawProportionalLine{BD,DC} :: \figAB\ + \figCA\ : \figBC$.

Но $\drawProportionalLine{DC} + \drawProportionalLine{BD} = \drawProportionalLine{BD,DC}$.

И $\therefore \figAB\ + \figCA\ = \figBC$.
\stopCenterAlign

\qed
\stopProposition

\startProposition[title={Предл. XXXII. Теорема},reference=prop:VI.XXXII]
\defineNewPicture[1/2]{
pair A, B, C, D, E;
numeric s;
B := (0, 0);
C := (5/2u, 0);
A := (u, 2u);
s := 3/5;
D := (A scaled s) shifted C;
E := (C scaled s) shifted C;
byAngleDefine(C, A, B, byyellow, 0);
byAngleDefine(A, B, C, byred, 0);
byAngleDefine(B, C, A, byblue, 0);
byAngleDefine(E, D, C, byblack, 0);
byAngleDefine(D, C, E, byblack, 1);
byAngleDefine(A, C, D, byyellow, 0);
draw byNamedAngleResized();
byLineDefine(A, B, byblue, 0, 0);
byLineDefine(B, C, byyellow, 0, 0);
byLineDefine(C, E, byyellow, 1, 0);
byLineDefine(E, D, byred, 1, 0);
byLineDefine(D, C, byblue, 1, 0);
byLineDefine(C, A, byred, 0, 0);
draw byNamedLineSeq(0)(ED,DC,noLine,CA,AB,BC,CE);
byLineDefine.ABt(A, B, byblack, 0, 1);
byLineDefine.BCt(B, C, byblack, 0, 1);
byLineDefine.CAt(C, A, byblack, 0, 1);
byLineDefine.DCt(D, C, byblack, 0, 1);
byLineDefine.CEt(C, E, byblack, 0, 1);
byLineDefine.EDt(E, D, byblack, 0, 1);
draw byLabelsOnPolygon(E, C, B, A, C, D)(0, 0);
}
\drawCurrentPictureInMargin
\problemNP{Е}{сли}{у треугольников
\drawFromCurrentPicture[bottom]{
startTempScale(1/3);
startTempAngleScale(angleScale*3/5);
draw byNamedAngle(A, B, BCA);
draw byNamedLineSeq(0)(CAt,BCt,ABt);
draw byLabelsOnPolygon(C, B, A)(0, 0);
stopTempAngleScale;
stopTempScale;
}
и~\drawFromCurrentPicture[bottom]{
startTempAngleScale(angleScale*3/5);
draw byNamedAngle(D, DCE);
draw byNamedLineSeq(0)(EDt,CEt,DCt);
draw byLabelsOnPolygon(C, D, E)(0, 0);
stopTempAngleScale;
},
две стороны пропорциональны $\drawUnitLine{AB} : \drawUnitLine{CA} :: \drawUnitLine{DC} : \drawUnitLine{ED}$ и~расположены они так, что касаются углом и~соответственные стороны параллельны, то оставшиеся стороны \drawUnitLine{BC} и~\drawUnitLine{CE} лежат на одной прямой.}

\startCenterAlign
Поскольку $\drawUnitLine{AB} \parallel \drawUnitLine{DC}$,\\
$\drawAngle{A} = \drawAngle{ACD}$ \inprop[prop:I.XXIX].

И поскольку $\drawUnitLine{CA} \parallel \drawUnitLine{ED}$,\\
$\drawAngle{ACD} = \drawAngle{D}$ \inprop[prop:I.XXIX].

$\therefore \drawAngle{A} = \drawAngle{D}$;\\
и поскольку $\drawUnitLine{AB} : \drawUnitLine{CA} :: \drawUnitLine{DC} : \drawUnitLine{ED}$ (\hypstr),\\
треугольники равноугольны \inprop[prop:VI.VI].

$\therefore \drawAngle{B} = \drawAngle{DCE}$.

Но $\drawAngle{A} = \drawAngle{ACD}$.

$\therefore \drawAngle{BCA} + \drawAngle{ACD} + \drawAngle{DCE} = \drawAngle{BCA} + \drawAngle{A} + \drawAngle{B} = \drawTwoRightAngles$ \inprop[prop:I.XXXII],\\
и $\therefore$ \drawUnitLine{BC} и~\drawUnitLine{CE} лежат на одной прямой \inprop[prop:I.XIV].
\stopCenterAlign

\qed
\stopProposition

\startProposition[title={Предл. XXXIII. Теорема},reference=prop:VI.XXXIII]
\defineNewPicture[1/2]{
pair A, B, C, D, E, F, G, H, K, L, M, N;
numeric r, a[], ba, aa, q;
path c[];
q := 8/360;
r := 9/4u;
aa := 125;
ba := 205;
a1 := 30;
a2 := 35;
G := (0, 0);
A := (dir(aa)*r) shifted G;
B := (dir(ba)*r) shifted G;
C := (dir(ba + a1)*r) shifted G;
K := (dir(ba + 2a1)*r) shifted G;
L := (dir(ba + 3a1)*r) shifted G;
byAngleDefine(B, A, C, byyellow, 0);
byAngleDefine.BC(B, G, C, byblack, 0);
byAngleDefine.CK(C, G, K, byred, 0);
byAngleDefine.KL(K, G, L, byblue, 0);
draw byNamedAngleResized(BAC, BC, CK, KL);
draw byLine(A, B, byblack, 0, 1);
draw byLine(A, C, byblack, 0, 1);
draw byLine(G, C, byblack, 0, 1);
draw byLine(G, K, byblack, 0, 1);
byLineDefine(G, B, byblack, 0, 1);
byLineDefine(G, L, byblack, 0, 1);
draw byNamedLineSeq(0)(GB,GL);
draw byArc.LB(G, L, B, r, byred, 0, 0, 0, 0);
draw byArc.BC(G, B, C, r, byblack, 0, 0, 0, 0);
draw byArc.CK(G, C, K, r, byred, 0, 0, 0, 0);
draw byArc.KL(G, K, L, r, byblue, 0, 0, 0, 0);
byCircleDefineR(G, r, byred, 0, 0, 0);
H := (0, -1/2u - 2r);
D := (dir(aa)*r) shifted H;
E := (dir(ba)*r) shifted H;
F := (dir(ba + a2)*r) shifted H;
M := (dir(ba + 2a2)*r) shifted H;
N := (dir(ba + 3a2)*r) shifted H;
byAngleDefine(E, D, F, byyellow, 1);
byAngleDefine.EF(E, H, F, byblack, 1);
byAngleDefine.FM(F, H, M, byred, 1);
byAngleDefine.MN(M, H, N, byblue, 1);
draw byNamedAngleResized(EDF, EF, FM, MN);
draw byLine(D, E, byblack, 0, 1);
draw byLine(D, F, byblack, 0, 1);
draw byLine(H, F, byblack, 0, 1);
draw byLine(H, M, byblack, 0, 1);
byLineDefine(H, E, byblack, 0, 1);
byLineDefine(H, N, byblack, 0, 1);
draw byNamedLineSeq(0)(HE,HN);
draw byArc.NE(H, N, E, r, byblue, 0, 0, 0, 0);
draw byArc.EF(H, E, F, r, byyellow, 1, 0, 0, 0);
draw byArc.FM(H, F, M, r, byred, 1, 0, 0, 0);
draw byArc.MN(H, M, N, r, byblue, 1, 0, 0, 0);
byCircleDefineR(H, r, byblue, 0, 0, 0);
draw byLabelsOnCircle(B, C, K, L)(G);
draw byLabelsOnCircle(E, F, M, N)(H);
draw byLabelsOnPolygon(B, G, L)(2, 0);
draw byLabelsOnPolygon(E, H, N)(2, 0);
}
\drawCurrentPictureInMargin
\problemNP[3]{В}{равных}{кругах \drawCircle[middle][1/4]{G} и~\drawCircle[middle][1/4]{H} углы,что в~центре, что на окружности, относятся друг к~другу как дуги, на которых они стоят $\drawAngle{BC} : \drawAngle{EF} ::
\drawFromCurrentPicture[middle][arcBC]{
startGlobalRotation(180-angle(B-C));
startAutoLabeling;
draw byNamedArc(BC);
stopAutoLabeling;
stopGlobalRotation;
} : \drawFromCurrentPicture[middle][arcEF]{
startGlobalRotation(180-angle(E-F));
startAutoLabeling;
draw byNamedArc(EF);
stopAutoLabeling;
stopGlobalRotation;
}$,  как и~секторы.}

Возьмем на окружности \circleG\ любое количество дуг
\drawFromCurrentPicture[middle][arcCK]{
startGlobalRotation(180-angle(C-K));
startAutoLabeling;
draw byNamedArc(CK);
stopAutoLabeling;
stopGlobalRotation;
},
\drawFromCurrentPicture[middle][arcKL]{
startGlobalRotation(180-angle(K-L));
startAutoLabeling;
draw byNamedArc(KL);
stopAutoLabeling;
stopGlobalRotation;
}, и~т. д. каждая $= \arcBC$, и~также на окружности \circleH\ возьмем любое количество дуг
\drawFromCurrentPicture[middle][arcFM]{
startGlobalRotation(180-angle(F-M));
startAutoLabeling;
draw byNamedArc(FM);
stopAutoLabeling;
stopGlobalRotation;
},
\drawFromCurrentPicture[middle][arcMN]{
startGlobalRotation(180-angle(M-N));
startAutoLabeling;
draw byNamedArc(MN);
stopAutoLabeling;
stopGlobalRotation;
}, и~т. д. каждая $= \arcEF$, проведем радиусы к~концам равных дуг.

Поскольку дуги \arcBC, \arcCK, \arcKL, и~т. д. равны, углы \drawAngle{BC}, \drawAngle{CK}, \drawAngle{KL}, и~т. д. также равны \inprop[prop:III.XXVII]. $\therefore$ \drawAngle{BC,CK,KL} столько же раз кратен \drawAngle{BC}, сколько \drawFromCurrentPicture[middle][arcBL]{
startGlobalRotation(180-angle(B-L));
startAutoLabeling;
draw byNamedArc(BC,CK,KL);
stopAutoLabeling;
stopGlobalRotation;
} кратна \arcBC. Точно так же \drawAngle{EF,FM,MN} столько же раз кратен \drawAngle{EF}, сколько дуга \drawFromCurrentPicture[middle][arcEN]{
startGlobalRotation(180-angle(E-N));
startAutoLabeling;
draw byNamedArc(EF,FM,MN);
stopAutoLabeling;
stopGlobalRotation;
} кратна дуге \arcEF.


\startCenterAlign
Тогда очевидно \inprop[prop:III.XXVII],\\
что если \drawAngle{BC,CK,KL} (или если $m$ раз \drawAngle{BC}) $>, =, < \drawAngle{EF,FM,MN}$ (или $n$ раз \drawAngle{EF})\\
то \arcBL (или $m$ раз \arcBC) $>, =, < \arcEN$ (или $n$ раз \arcEF).
\stopCenterAlign

$\therefore \drawAngle{BC} : \drawAngle{EF} :: \arcBC\ : \arcEF$ \indef[def:V.V], или углы в~центре относятся так же, как дуги, на которых они стоят, но углы на окружности вдвое меньше углов в~центре \inprop[prop:III.XX] и~относятся так же \inprop[prop:V.XV], и~следовательно так же, как дуги, на которых они стоят.

Очевидно, что секторы в~равных кругах и~на равных дугах равны (\inpropL[prop:I.IV], \inpropN[prop:III.XXIV], \inpropN[prop:III.XXVII], и~\indefL[def:III.IX]). Значит, если выше углы заменить на секторы, получим доказательство второй части, то есть, что в~равных кругах секторы относятся друг к~другу как дуги, на которых они стоят.

\qed
\stopProposition

\startPropositionAZ[title={Предл. A. Теорема},reference=prop:VI.A]
\defineNewPicture{
pair A, B, C, D, E, F;
A := (0, 0);
B := (2u, 0);
C := (3u, 7/4u);
F := 4/3[A, C];
D = whatever[A, B] = whatever[C, C shifted 1/2[unitvector(F-C), unitvector(B-C)]];
E = whatever[A, C] = whatever[B, B shifted (C-D)];
byAngleDefine(C, E, B, byyellow, 0);
byAngleDefine(E, B, C, byblue, 1);
byAngleDefine(C, B, D, byred, 0);
byAngleDefine(B, D, C, byyellow, 1);
byAngleDefine(B, C, E, byblack, 1);
byAngleDefine(D, C, B, byblue, 0);
byAngleDefine(F, C, D, byblack, 0);
draw byNamedAngleResized();
draw byLine(B, E, byred, 0, 0);
draw byLine(B, C, byyellow, 0, 0);
byLineDefine(A, B, byblue, 0, 0);
byLineDefine(B, D, byblue, 1, 0);
byLineDefine(D, C, byred, 1, 0);
byLineDefine(A, E, byblack, 0, 0);
byLineDefine(E, C, byblack, 1, 0);
byLineDefine(C, F, byyellow, 1, 0);
draw byNamedLineSeq(0)(noLine,DC,BD,AB,AE,EC,CF);
draw byLabelsOnPolygon(C, D, B, A, E)(2, 0);
draw byLabelsOnPolygon(E, C, noPoint)(0, 0);
}
\drawCurrentPictureInMargin
\problemNP[2]{Е}{сли}{прямая линия \drawSizedLine{DC} рассекая пополам внешний угол \drawAngle{DCB,FCD} треугольника \drawFromCurrentPicture[bottom]{
startTempScale(1/2);
startAutoLabeling;
draw byNamedLineSeq(0)(AE,EC,BC,AB);
stopAutoLabeling;
stopTempScale;
} встречается в~продолжением стороны треугольника \drawSizedLine{AB}, то вся сторона с~продолжением \drawSizedLine{AB,BD} и~ее внешняя часть \drawSizedLine{BD} будут пропорциональны сторонам \drawSizedLine{AE,EC} и~\drawSizedLine{BC}, между которыми заключен угол, смежный с~рассекаемым внешним.}

\startCenterAlign
Действительно, если провести $\drawSizedLine{BE} \parallel \drawSizedLine{DC}$,\\
$\eqalign{
\mbox{то } \drawAngle{DCB} &= \drawAngle{EBC} \mbox{, \inprop[prop:I.XXIX];}\cr
& = \drawAngle{FCD} \mbox{, (\hypstr),}\cr
& = \drawAngle{E} \mbox{, \inprop[prop:I.XXIX].}
}$

И $\therefore \drawSizedLine{EC} = \drawSizedLine{BC}$, \inprop[prop:I.VI],\\
и $\drawSizedLine{AE,EC} : \drawSizedLine{BC} :: \drawSizedLine{AE,EC} : \drawSizedLine{EC}$ \inprop[prop:V.VII].

Но также $\drawSizedLine{AB,BD} : \drawSizedLine{BD} :: \drawSizedLine{AE,EC} : \drawSizedLine{EC}$ \inprop[prop:VI.II].

И следовательно $\drawSizedLine{AB,BD} : \drawSizedLine{BD} :: \drawSizedLine{AE,EC} : \drawSizedLine{BC}$ \inprop[prop:V.XI].
\stopCenterAlign

\qed
\stopPropositionAZ

\startPropositionAZ[title={Предл. B. Теорема},reference=prop:VI.B]
\defineNewPicture{
pair A, B, C, D, E, O;
numeric r;
path c;
r := 7/4u;
O := (0, 0);
c := (fullcircle scaled 2r) shifted O;
A := (dir(75)*r) shifted O;
B := (dir(180 + 10)*r) shifted O;
C := (dir(-10)*r) shifted O;
D = whatever[B, C] = whatever[A, A shifted 1/2[unitvector(B-A), unitvector(C-A)]];
E := (subpath (0, -4) of c) intersectionpoint (A--4[A, D]);
byAngleDefine(A, B, C, byyellow, 0);
byAngleDefine(A, E, C, byblack, 0);
byAngleDefine(C, A, E, byblue, 0);
byAngleDefine(E, A, B, byred, 0);
draw byNamedAngleResized();
draw byLine(A, B, byblue, 0, 0);
draw byLine(B, D, byred, 1, 0);
draw byLine(D, C, byred, 0, 0);
draw byLine(C, A, byblack, 0, 0);
draw byLine(A, D, byyellow, 0, 0);
draw byLine(D, E, byyellow, 1, 0);
draw byLine(C, E, byblue, 1, 0);
draw byCircleABC.O(A, B, C, byyellow, 0, 0, 1/2);
draw byLabelsOnCircle(A, B, C, E)(O);
draw byLabelsOnPolygon(B, D, A)(2, 0);
}
\drawCurrentPictureInMargin
\problemNP{Е}{сли}{угол треугольника рассекается пополам прямой линией, которая также сечет его основания, прямоугольник заключенный между сторонами треугольника равен прямоугольнику, заключенному между частями основания, вместе с~квадратом прямой, рассекающей угол.}

\startCenterAlign
Проведем \drawSizedLine{AD}, делая $\drawAngle{CAE} = \drawAngle{EAB}$,\\
тогда $\drawSizedLine{AB} \times \drawSizedLine{CA} = \drawSizedLine{BD} \times \drawSizedLine{DC} + \drawSizedLine{AD}^2$.

Около \drawLine[bottom]{CA,DC,BD,AB} опишем \drawCircle[middle][1/4]{O} \inprop[prop:IV.V],\\
продлим \drawSizedLine{AD} до окружности и~проведем \drawSizedLine{CE}.

Поскольку $\drawAngle{CAE} = \drawAngle{EAB}$ (\hypstr),\\
и $\drawAngle{B} = \drawAngle{E}$ \inprop[prop:III.XXI].

$\therefore$ \drawLine[bottom]{AD,BD,AB} и~\drawLine[middle]{CA,CE,DE,AD} равноугольны \inprop[prop:I.XXXII].

$\therefore \drawSizedLine{AB} : \drawSizedLine{AD} :: \drawSizedLine{AD,DE} : \drawSizedLine{CA}$ \inprop[prop:VI.IV].

$\therefore \drawSizedLine{AB} \times \drawSizedLine{CA} = \drawSizedLine{AD} \times \drawSizedLine{AD,DE}$ \inprop[prop:VI.XVI]\\
$= \drawSizedLine{DE} \times \drawSizedLine{AD} + \drawSizedLine{AD}^2$ \inprop[prop:II.III].

Но $\drawSizedLine{DE} \times \drawSizedLine{AD} = \drawSizedLine{BD} \times \drawSizedLine{DC}$ \inprop[prop:III.XXXV].

$\therefore \drawSizedLine{AB} \times \drawSizedLine{CA} = \drawSizedLine{BD} \times \drawSizedLine{DC} + \drawSizedLine{AD}^2$.
\stopCenterAlign

\qed
\stopPropositionAZ

\startPropositionAZ[title={Предл. C. Теорема},reference=prop:VI.C]
\defineNewPicture{
pair A, B, C, D, E, O;
numeric r;
r := 9/4u;
O := (0, 0);
A := (dir(60)*r) shifted O;
B := (dir(180-5)*r) shifted O;
C := (dir(5)*r) shifted O;
D = whatever[B, C] = whatever[A, A shifted ((B-C) rotated 90)];
E := (dir(60 + 180)*r) shifted O;
byAngleDefine(A, B, C, byyellow, 1);
byAngleDefine(A, E, C, byred, 1);
byAngleDefine(B, D, A, byblue, 0);
byAngleDefine(B, C, A, byblue, 1);
byAngleDefine(E, C, B, byyellow, 0);
byAngleDefine(C, A, B, byred, 0);
draw byNamedAngleResized();
draw byLine(A, E, byblue, 0, 0);
draw byLine(C, E, byblack, 0, 0);
draw byLine(A, D, byyellow, 1, 0);
draw byLine(A, B, byblue, 1, 0);
draw byLine(B, D, byred, 1, 0);
draw byLine(D, C, byred, 0, 0);
draw byLine(C, A, byyellow, 0, 0);
draw byCircleABC.O(A, B, C, byred, 0, 0, 1/2);
draw byLabelsOnCircle(A, B, C, E)(O);
draw byLabelsOnPolygon(C, D, B)(2, 0);
}
\drawCurrentPictureInMargin
\problemNP{Е}{сли}{из любого угла треугольника провести прямую перпендикулярную основанию, прямоугольник, заключенный между сторонами треугольника будет равен прямоугольнику, заключенному между перпендикуляром и~диаметром круга, описанного около треугольника.}

\startCenterAlign
Из угла \drawAngle{A} треугольника \drawLine[bottom]{CA,DC,BD,AB}\\
проведем $\drawUnitLine{AD} \perp \drawUnitLine{BD,DC}$;\\
тогда $\drawUnitLine{AB} \times \drawUnitLine{CA} = \drawUnitLine{AD} \times $ диаметр описанного круга.

Опишем \drawCircle[middle][1/4]{O} \inprop[prop:IV.V], проведем диаметр \drawUnitLine{AE}, и~проведем \drawUnitLine{CE},\\
тогда, поскольку $\drawAngle{D} = \drawAngle{BCA,ECB}$ (\conststr и~\inpropL[prop:III.XXXI]),
и $\drawAngle{B} = \drawAngle{E}$ \inprop[prop:III.XXI].

$\therefore$ \drawLine[bottom]{AD,BD,AB} равноуголен с~\drawLine[middle]{CA,CE,AE} \inprop[prop:VI.IV].

$\therefore \drawUnitLine{AB} : \drawUnitLine{AD} :: \drawUnitLine{AE} : \drawUnitLine{CA}$.

И $\therefore \drawUnitLine{AB} \times \drawUnitLine{CA} = \drawUnitLine{AD} \times \drawUnitLine{AE}$ \inprop[prop:VI.XVI].
\stopCenterAlign

\qed
\stopPropositionAZ

\startPropositionAZ[title={Предл. D. Теорема},reference=prop:VI.D]
\defineNewPicture[1/6]{
pair A, B, C, D, E, O;
numeric r, a[];
a1 := 110;
a2 := 190;
a3 := -50;
a4 := 15;
r := 9/4u;
O := (0, 0);
A := (dir(a1)*r) shifted O;
B := (dir(a2)*r) shifted O;
C := (dir(a3)*r) shifted O;
D := (dir(a4)*r) shifted O;
E = whatever[B, D] = whatever[A, A shifted dir((a1-180)-(a3-(a1-180)))];
byAngleDefine(B, C, A, byyellow, 0);
byAngleDefine(A, C, D, byred, 1);
byAngleDefine(B, D, A, byblue, 1);
byAngleDefine(A, B, D, byyellow, 1);
byAngleDefine(E, A, B, byblue, 0);
byAngleDefine(C, A, E, byblack, 0);
byAngleDefine(D, A, C, byred, 0);
draw byNamedAngleResized();
draw byLine(A, E, byyellow, 1, 0);
draw byLine(A, C, byblue, 0, 0);
draw byLine(B, E, byred, 1, 0);
draw byLine(E, D, byred, 0, 0);
draw byLine(A, B, byblack, 1, 0);
draw byLine(B, C, byblue, 1, 0);
draw byLine(C, D, byblack, 0, 0);
draw byLine(D, A, byyellow, 0, 0);
draw byCircleABC.O(A, B, C, byred, 0, 0, 1/2);
draw byLabelsOnCircle(A, B, C, D)(O);
draw byLabelsOnPolygon(D, E, B)(2, 0);
}
\drawCurrentPictureInMargin
\problemNP[5]{П}{рямоугольник,}{заключенный между диагоналями четырехугольника, вписанного в~круг равен прямоугольникам, заключенным между противоположными сторонами вместе взятым.}

\startCenterAlign
Пусть \offsetPicture{15pt}{15pt}{\drawLine{DA,CD,BC,AB}} будет любым четырехугольником, вписанным в~\drawCircle[middle][1/6]{O}.

Проведем \drawUnitLine{BE,ED} и~\drawUnitLine{AC}, тогда $\drawUnitLine{BE,ED} \times \drawUnitLine{AC} = \drawUnitLine{AB} \times \drawUnitLine{CD} + \drawUnitLine{DA} \times \drawUnitLine{BC}$.

Сделаем $\drawAngle{EAB} = \drawAngle{DAC}$ \inprop[prop:I.XXIII],\\
$\therefore \drawAngle{EAB,CAE} = \drawAngle{CAE,DAC}$, и~$\drawAngle{BCA} = \drawAngle{D}$ \inprop[prop:III.XXI].

$\therefore \drawUnitLine[0.8cm]{DA} : \drawUnitLine[0.8cm]{ED} :: \drawUnitLine[0.8cm]{AC} : \drawUnitLine[0.8cm]{BC}$ \inprop[prop:VI.IV],\\
и $\therefore \drawUnitLine[0.75cm]{ED} \times \drawUnitLine[0.75cm]{AC} = \drawUnitLine[0.75cm]{DA} \times \drawUnitLine[0.75cm]{BC}$ \inprop[prop:VI.XVI].

Теперь, поскольку\\
$\drawAngle{EAB} = \drawAngle{DAC}$ (\conststr) и $\drawAngle{B} = \drawAngle{ACD}$ \inprop[prop:III.XXI],\\
$\therefore \drawUnitLine{AB} : \drawUnitLine{BE} :: \drawUnitLine{AC} : \drawUnitLine{CD}$ \inprop[prop:VI.IV],\\
и $\therefore \drawUnitLine[0.75cm]{BE} \times \drawUnitLine[0.75cm]{AC} = \drawUnitLine[0.75cm]{AB} \times \drawUnitLine[0.75cm]{CD}$ \inprop[prop:VI.XVI].

Но, как написано выше, $\drawUnitLine{ED} \times \drawUnitLine{AC} = \drawUnitLine{DA} \times \drawUnitLine{BC}$.

$\therefore \drawUnitLine{BE,ED} \times \drawUnitLine{AC} = \drawUnitLine{AB} \times \drawUnitLine{CD} + \drawUnitLine{DA} \times \drawUnitLine{BC}$ \inprop[prop:II.I].
\stopCenterAlign

\qed
\stopPropositionAZ
\stopBook

\stoptext
%\closeout \lettrineslist

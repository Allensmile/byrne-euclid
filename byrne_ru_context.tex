\input preamble.tex
\input preamble_be.tex

\mainlanguage[ru]

\def\inpropstr{пр.}
\def\inpoststr{пост.}
\def\indefstr{опр.}
\def\inaxstr{акс.}
\def\qedstr{ч. т. д.}

\def\mpPre{textLabels := true;}

\starttext
\setuplayout[title]
\setupheader [state=stop]
~\vfill
~\hfill\symbol[cc][cc] \symbol[cc][by] \symbol[cc][sa]\hfill~
\vfill~
\pagebreak\ \pagebreak
\setupheader[state=start]
\setuplayout[reset]

\startbook[title={Книга I}]
\startVerboseProposition[title={Предложение I. Задача}, reference=prop:I.I]

\defineNewPicture[1/2]{
	pair A, B, C;
	path P[];
	numeric r;
	r := 3/2u;
	A := (0, 0);
	B := (r, 0);
	P1 := fullcircle scaled 2r;
	P2 := fullcircle scaled 2r shifted B;
	C := P1 intersectionpoint P2;
		byLineDefine(A, B, black, 0, 0);
		byLineDefine(B, C, byred, 0, 0);
		byLineDefine(C, A, byyellow, 0, 0);
		draw byNamedLineSeq(1)(AB,CA,BC);
		draw byCircle(A, B, byblue, 0, 0, 1/2)(A);
		draw byCircle(B, A, byred, 0, 0, 1/2)(B);
		draw byLabelsOnPolygon(A, C, B)(0, -1);
}
\drawCurrentPictureInMargin
\problemNP{Н}{а}{данной ограниченной прямой \drawUnitLine{AB} построить равносторонний треугольник.}

\startCenterAlign
Опишем \offsetPicture{15pt}{0pt}{\drawFromCurrentPicture{
draw byNamedLine(AB);
draw byNamedCircle(A);
draw byLabelLineEnd(A, B, 0);
draw byLabelLineEnd(B, A, 1);
}} и \offsetPicture{15pt}{0pt}{\drawFromCurrentPicture{
draw byNamedLine(AB); 
draw byNamedCircle(B);
draw byLabelLineEnd(A, B, 1);
draw byLabelLineEnd(B, A, 0);
}}
\inpost[post:III];\\
проведем \drawUnitLine{CA} и \drawUnitLine{BC} \inpost[post:I].\\
Тогда \drawLine[bottom][triangleABC]{AB,CA,BC} равносторонний.

Поскольку $\drawUnitLine{AB} = \drawUnitLine{CA}$ \indef[def:XV];\\
и $\drawUnitLine{AB} = \drawUnitLine{BC}$ \indef[def:XV];\\
$\therefore \drawUnitLine{CA} = \drawUnitLine{BC}$ \inax[post:I];\\
и значит \triangleABC\ и есть искомый треугольник.
\stopCenterAlign

\qed
\stopVerboseProposition

\startProposition[title={Предложение II. Задача}, reference=prop:I.II]
\defineNewPicture{
pair A, B, C, D, E, F;
path P[];
numeric r[];
A := (0, 0);
B := (-3/5u, -3/5u);
C := (-2u, -1/3u);
r1 := abs(A-B);
D := (fullcircle scaled 2r1 shifted A) intersectionpoint (fullcircle scaled 2r1 shifted B);
r2 := abs(B-C);
r3 := r1 + r2;
P1 := fullcircle scaled 2r2 shifted B;
P2 := fullcircle scaled 2r3 shifted D;
E := (D -- 10[D, B]) intersectionpoint P1;
F := (D -- 10[D, A]) intersectionpoint P2;
byLineDefine(A, B, black, 1, 0);
byLineDefine(B, C, black, 0, 0);
byLineDefine(B, D, byred, 1, 0);
byLineDefine(D, A, byred, 0, 0);
byLineDefine(B, E, byyellow, 0, 0);
byLineDefine(A, F, byblue, 0, 0);
draw byNamedLineSeq(0)(AF,DA,BD,BE);
draw byNamedLineSeq(0)(AB,BC);
draw byCircle(D, E, byred, 0, 0, 1/2)(A);
draw byCircle(B, C, byblue, 0, 0, -1/2)(B);
draw byLabelsOnPolygon(E, D, A, F)(2, -1);
draw byLabelsOnPolygon(E, B, C)(2, -1);
draw byLabelLineEnd(C, B, 0);
draw byLabelLineEnd(E, B, 1);
draw byLabelLineEnd(F, A, 1);
}
\drawCurrentPictureInMargin
\problemNP{О}{т}{данной точки \drawFromCurrentPicture{
startGlobalRotation(-lineAngle.DA);
draw byNamedLineSeq(0)(DA,AF);
draw byLabelPoint(A, 90, 1);
stopGlobalRotation;
} отложить прямую, равную данной прямой \drawUnitLine{BC}.}

\startCenterAlign
Проведем \drawUnitLine{AB} \inpost[post:I], построим \drawFromCurrentPicture[bottom]{
startAutoLabeling;
startTempScale(scaleFactor*3);
startGlobalRotation(180-lineAngle.AB);
draw byNamedLineSeq(0)(AB,BD,DA);
stopGlobalRotation;
stopTempScale;
stopAutoLabeling;
} \inprop[prop:I.I],\\
продлим \drawUnitLine{BD} \inpost[post:II],\\
опишем
\drawFromCurrentPicture{
draw byNamedLine (BC); 
draw byNamedCircle(B); 
draw byLabelLineEnd(B, C, 0); 
draw byLabelLineEnd(C, B, 0);
}
\inpost[post:III], и
\drawFromCurrentPicture{
draw byNamedLine (BD, BE);
draw byNamedCircle(A);
draw byLabelLineEnd(D, E, 0); 
draw byLabelLineEnd(E, D, 1);
}
\inpost[post:III];\\
продлим \drawUnitLine{DA} \inpost[post:II],\\
тогда искомая прямая это \drawUnitLine{AF}.

Поскольку $\drawUnitLine{BE,BD} = \drawUnitLine{DA,AF}$ \indef[def:XV],\\
и $\drawUnitLine{BD} = \drawUnitLine{DA}$ (постр.),\\
$\therefore \drawUnitLine{BE} = \drawUnitLine{AF}$ \inax[post:III],\\
но \indef[def:XV] $\drawUnitLine{BC} = \drawUnitLine{BE} = \drawUnitLine{AF}$;

$\therefore \drawUnitLine{AF}$ проведенная из данной точки (\drawUnitLine{DA,AF}) равна данной прямой \drawUnitLine{BC}.
\stopCenterAlign

\qed
\stopProposition

\startProposition[title={Предложение III. Задача}, reference=prop:I.III]
\defineNewPicture{
pair A, B, C, D, E, F;
path P;
numeric r;
A := (0, 0);
r := 7/4u;
B := A shifted (r, 0);
C := A shifted (4/3r, 0);
D := A shifted dir(30)*r;
E := A shifted (7/6r, -1/6r);
F := A shifted (7/6r, -7/6r);
byLineDefine(A, B, black, 0, 0);
byLineDefine(B, C, black, 1, 0);
byLineDefine(A, D, byred, 0, 0);
draw byNamedLineSeq(0)(BC,AB,AD);
draw byLine(E, F, byblue, 0, 0);
draw byCircle(A, D, byblue, 0, 0, 0)(A);
draw byLabelsOnPolygon(B, A, D)(2, -1);
draw byLabelLineEnd(D, A, 0);
draw byLabelLineEnd(C, A, 0);
draw byLabelPoint(B, angle(B-A) + 45, 2);
draw byLabelsOnPolygon(E, F)(0, 0);
}
\drawCurrentPictureInMargin
\problemNP{О}{т}{большей \drawUnitLine{AB,BC}  из двух данных прямых, отнять прямую, равную меньшей \drawUnitLine{EF}.}

\startCenterAlign
Проведем $\drawUnitLine{AD} = \drawUnitLine{EF}$ \inprop[prop:I.II];\\
опишем 
\drawFromCurrentPicture{
draw byNamedLine (AD); 
draw byNamedCircle(A);
draw byLabelLineEnd(D, A, 0);
draw byLabelLineEnd(A, D, 0);
} \inpost[post:III],\\
тогда $\drawUnitLine{EF} = \drawUnitLine{AB}$

Поскольку $\drawUnitLine{AD} = \drawUnitLine{AB}$ \indef[def:XV],\\
и $\drawUnitLine{EF} = \drawUnitLine{AD}$ (постр.);

$\therefore \drawUnitLine{EF} = \drawUnitLine{AB}$ \inax[ax:I];
\stopCenterAlign

\qed
\stopProposition

\startProposition[title={Предложение IV. Теорема}, reference=prop:I.IV]
\defineNewPicture{
pair A, B, C, D, E, F, d;
A := (0, 0);
B := A shifted (-5/2u, -7/2u);
C := A shifted (1/3u, -5/2u);
d := (0, -4u);
D := A shifted d;
E := B shifted d;
F := C shifted d;
draw byAngleWithName(B, A, C, byyellow, 0)(A);
draw byAngleWithName(A, B, C, byblue, 0)(B);
draw byAngleWithName(B, C, A, byred, 0)(C);
byLineDefine(A, B, byred, 0, 0);
byLineDefine(B, C, black, 0, 0);
byLineDefine(C, A, byblue, 0, 0);
draw byNamedLineSeq(0)(CA,BC,AB);
draw byAngleWithName(E, D, F, byyellow, 0)(D);
draw byAngleWithName(D, E, F, byblue, 0)(E);
draw byAngleWithName(E, F, D, byred, 0)(F);
byLineDefine(D, E, byred, 0, 1);
byLineDefine(E, F, black, 0, 1);
byLineDefine(F, D, byblue, 0, 1);
draw byNamedLineSeq(0)(FD,EF,DE);
draw byLabelsOnPolygon(F, E, D)(0, 0);
draw byLabelsOnPolygon(B, A, C)(0, -1);
}
\drawCurrentPictureInMargin
\problemNP{Е}{сли}{два треугольника имеют по две стороны, равные каждая каждой, ($\drawUnitLine{AB} = \drawUnitLine{DE}$ и $\drawUnitLine{CA} = \drawUnitLine{FD}$) и по равному углу ($\drawAngle{A} = \drawAngle{D}$) содержащемуся между равными прямыми, то они будут иметь и основание равное основанию ($\drawUnitLine{BC} = \drawUnitLine{EF}$), и один треугольник будет равен другому, и остальные углы, стягиваемые равными сторонами, будут равны каждый каждому ($\drawAngle{B} = \drawAngle{E}$ and $\drawAngle{C} = \drawAngle{F}$).}

Представим, что два треугольника расположены таким образом, что вершина одного из двух равных углов \drawAngle{A} или \drawAngle{D}, совпадает с вершиной другого, и \drawUnitLine{AB} совпадает \drawUnitLine{DE}, тогда \drawUnitLine{CA} при наложении совпадет с \drawUnitLine{FD}. Следовательно \drawUnitLine{BC} совпает с \drawUnitLine{EF}, или же две прямые будут содержать пространство, что невозможно \inax[ax:X], следовательно $\drawUnitLine{BC} = \drawUnitLine{EF}$, $\drawAngle{B} = \drawAngle{E}$ и $\drawAngle{C} = \drawAngle{F}$, и поскольку треугольники \drawLine{CA,BC,AB} и \drawLine{FD,EF,DE} совпадают при наложении, они равны во всех отношениях.

\qed
\stopProposition

\startProposition[title={Предложение V. Теорема}, reference=prop:I.V]
\defineNewPicture[11/40]{
pair A, B, C, D, E;
picture q;
A := (0, 0);
B := A shifted (u, -2u);
C := B xscaled -1;
D := 9/5[A,B];
E := 9/5[A,C];
draw byAngle(B, A, C, black, 0);
draw byAngle(A, B, C, byblue, 0);
draw byAngle(B, C, A, byblue, 0);
draw byAngle(C, B, E, byyellow, 0);
draw byAngle(D, C, B, byyellow, 0);
draw byAngle(B, D, C, byred, 0);
draw byAngle(C, E, B, byred, 0);
byAngleDefine(E, B, D, black, 1);
byAngleDefine(D, C, E, black, 1);
byLineDefine(B, D, byyellow, 0, 0);
byLineDefine(C, E, byyellow, 0, 0);
byLineDefine(B, E, byblue, 0, 0);
byLineDefine(C, D, byblue, 0, 0);
byLineDefine(A, B, byred, 0, 0);
byLineDefine(A, C, byred, 0, 0);
byLineDefine(B, C, black, 0, 0);
draw byNamedLineSeq(0)(CD,noLine,BC,noLine,BE,CE,AC,AB,BD);
draw byLabelsOnPolygon(E, C, A, B, D, C, B)(0, 0);
}
\drawCurrentPictureInMargin
\problemNP[2]{У}{любого}{равнобедренного треугольника \drawLine[bottom]{BC,AC,AB} углы при основании равны между собой и по продолжении равных сторон углы под основанием будут равны между собой.}

\startCenterAlign
Продлим \drawUnitLine{AB} и \drawUnitLine{AC} \inpost[post:II],\\
возьмем $\drawUnitLine{BD} = \drawUnitLine{CE}$ \inprop[prop:I.III];\\
проведем \drawUnitLine{BE} и \drawUnitLine{CD}.

Тогда в
\drawFromCurrentPicture{
startAutoLabeling;
draw byNamedAngle(BAC);
draw byNamedLineSeq(0)(BE,CE,AC,AB);
stopAutoLabeling;
}
и
\drawFromCurrentPicture{
startAutoLabeling;
draw byNamedAngle(BAC);
draw byNamedLineSeq(0)(BD,CD,AC,AB);
stopAutoLabeling;
}\\
получим $\drawUnitLine{AB,BD} = \drawUnitLine{AC,CE}$ (конст.),\\
\drawAngle{BAC} общий обоим,\\
и $\drawUnitLine{AB} = \drawUnitLine{AC}$ (гип.)\\
$\therefore \drawAngle{BCA,DCB} = \drawAngle{ABC,CBE}$, $\drawUnitLine{BE} = \drawUnitLine{CD}$ и $\drawAngle{CEB} = \drawAngle{BDC}$ \inprop[prop:I.IV].

Так же у \drawLine{BE,CE,BC} и \drawLine{BD,CD,BC}\\
получим $\drawUnitLine{BD} = \drawUnitLine{CE}$, $\drawAngle{CEB} = \drawAngle{BDC}$ и $\drawUnitLine{BE} = \drawUnitLine{CD}$,\\
$\therefore \drawAngle{DCE,DCB} = \drawAngle{EBD,CBE}$ и $\drawAngle{DCB} = \drawAngle{CBE}$ \inprop[prop:I.IV]\\
но $\drawAngle{BCA,DCB} = \drawAngle{ABC,CBE}$, $\therefore \drawAngle{BCA} = \drawAngle{ABC}$.
\stopCenterAlign

\qed
\stopProposition


\startProposition[title={Предложение VI. Теорема}, reference=prop:I.VI]
\defineNewPicture[1/4]{
pair A, B, C, D;
A := (0, 0);
B := A shifted (7/2u, 0);
D := A shifted (7/4u, 3u);
C := 2/3[A, D];
draw byAngleWithName(B, A, C, byyellow, 0)(A);
draw byAngleWithName(A, B, D, black, 0)(B);
byLineDefine(B, C, byyellow, 0, 0);
byLineDefine(A, B, byred, 0, 0);
byLineDefine(B, D, byblue, 0, 0);
byLineDefine(C, A, black, 0, 0);
byLineDefine(C, D, black, 1, 0);
draw byNamedLine(BC);
draw byNamedLineSeq(0)(CA,CD,BD,AB);
draw byLabelsOnPolygon(A, C, D, B)(0, 0);
}
\drawCurrentPictureInMargin
\problemNP{Е}{сли}{у любого треугольника \drawLine[bottom][triangleABD]{CA,CD,BD,AB} два угла \drawAngle{A} и \drawAngle{B} равны между собой, то и стороны \drawUnitLine{CA,CD} and \drawUnitLine{BD}, стягивающие равные углы, будут равны.}

Предположим, что стороны не равны и одна из них \drawUnitLine{CA,CD} больше чем другая \drawUnitLine{BD}, тогдо отрежем от нее $\drawUnitLine{CA} = \drawUnitLine{BD}$ \inprop[prop:I.III] и проведем \drawUnitLine{BC}.

\startCenterAlign
Тогда в \drawLine[bottom]{BC,AB,CA} и \triangleABD,\\
$\drawUnitLine{CA} = \drawUnitLine{BD}$ (постр.),\\
$\drawAngle{A} = \drawAngle{B}$ (гип.)\\
и \drawUnitLine{AB} общая обоим,\\
$\therefore$ эти треугольники равны \inprop[prop:I.IV]\\
часть равна целому, что не имеет смысла;\\
$\therefore$ ни одна из сторон \drawUnitLine{CA,CD} или \drawUnitLine{BD} не больше другой,\\
$\therefore$ они равны.
\stopCenterAlign

\qed
\stopProposition

\startProposition[title={Предложение VII. Теорема}, reference=prop:I.VII]
\defineNewPicture{
pair A, B, C, D, E, F, G, H;
A := (0, 0);
B := A shifted (4u, 0);
C := A shifted (u, 3u);
D := C shifted (7/4u, 0);
E := 1/2[C, D] yscaled -0.7;
F := E shifted (0, -2u);
G := 5/4[A, E];
H := 5/4[A, F];
draw byAngleWithName(B, C, A, black, 0)(C);
draw byAngle(D, C, B, byred, 0);
draw byAngleWithName(A, D, B, byyellow, 0)(D);
draw byAngle(C, D, A, byblue, 0);
draw byAngle(B, F, H, black, 0);
draw byAngle(B, F, E, byred, 0);
draw byAngle(B, E, G, byyellow, 0);
draw byAngle(G, E, F, byblue, 0);
draw byLine(C, D, black, 1, 0);
draw byLine(E, F, black, 1, 0);
draw byLine(A, B, black, 0, 0);
byLineDefine(B, C, byblue, 0, 0);
byLineDefine(C, A, byred, 0, 0);
byLineDefine(B, D, byblue, 0, 0);
byLineDefine(D, A, byred, 0, 0);
byLineDefine(B, E, byblue, 0, 0);
byLineDefine(E, A, byred, 0, 0);
byLineDefine(B, F, byblue, 0, 0);
byLineDefine(F, A, byred, 0, 0);
byLineDefine(E, G, byred, 1, 0);
byLineDefine(F, H, byred, 1, 0);
draw byNamedLine(EG,FH);
draw byNamedLineSeq(0)(BC,CA,EA,BE);
draw byNamedLineSeq(0)(BD,DA,FA,BF);
string pointLabel.F, pointLabel.E;
pointLabel.F := "C";
pointLabel.E := "D";
draw byLabelsOnPolygon(F, A, C, D, B, F, noPoint)(2, 0);
draw byLabelsOnPolygon(A, E, B)(2, 0);
draw byLabelsOnPolygon(H, F, A)(2, 0);
}
\drawCurrentPictureInMargin
\problemNP{П}{о}{одну сторону одной и той же прямой \drawUnitLine{AB} нельзя построить два разных треугольника с равными друг другу смежными сторонами $\drawUnitLine{CA} = \drawUnitLine{DA}$ и $\drawUnitLine{BC} = \drawUnitLine{BD}$.}

Если два треугольника построены на одном основании и по одну сторону от него, то вершина одного может находиться вовне другого, внутри или на одной из его сторон.

Если такое возможно, то построим два треугольника таких, что $\left\{\eqalign{\drawUnitLine{CA}&=\drawUnitLine{BC}\cr \drawUnitLine{DA}&=\drawUnitLine{BD}\cr}\right\}$, затем проведем \drawUnitLine{CD}, тогда

\startCenterAlign
$\drawAngle{C,DCB} = \drawAngle{CDA}$ \inprop[prop:I.V]

$\therefore\drawAngle{DCB} < \drawAngle{CDA}$ и

$\left.
\eqalign{
\therefore\drawAngle{DCB} &< \drawAngle{CDA,D}\cr
\mbox{но \inprop[prop:I.V]} \drawAngle{DCB} &= \drawAngle{CDA,D}
}\right\}\mbox{что невозможно,}$
\stopCenterAlign

\noindent следовательно, смежные стороны таких двух треугольников не могут быть равны.

\qed
\stopProposition

\startProposition[title={Предложение VIII. Теорема}, reference=prop:I.VIII]
\defineNewPicture{
pair A, B, C, D, E, F, d;
A := (0, 0);
B := A shifted (-u, -4u);
C := A shifted (3/2u, -3u);
d := (0, -9/2u);
D := A shifted d;
E := B shifted d;
F := C shifted d;
draw byAngleWithName(F, D, E, black, 0)(D);
draw byAngleWithName(C, A, B, black, 0)(A);
byLineDefine(A, B, byred, 0, 0);
byLineDefine(B, C, black, 0, 0);
byLineDefine(C, A, byblue, 0, 0);
byLineDefine(D, E, byred, 0, 1);
byLineDefine(E, F, black, 0, 1);
byLineDefine(F, D, byblue, 0, 1);
draw byNamedLineSeq(0)(CA,BC,AB);
draw byNamedLineSeq(0)(FD,EF,DE);
draw byLabelsOnPolygon(C, B, A)(0, 0);
draw byLabelsOnPolygon(F, E, D)(0, 0);
}
\drawCurrentPictureInMargin
\problemNP{Е}{сли}{у двух треугольников по две попарно равных стороны ($\drawUnitLine{CA} = \drawUnitLine{FD}$ и $\drawUnitLine{AB} = \drawUnitLine{DE}$), а также равные основания ($\drawUnitLine{BC} = \drawUnitLine{EF}$), то углы
\drawFromCurrentPicture{
startAutoLabeling;
startGlobalRotation(-angleDirection.A);
draw byNamedAngle(A);
draw byNamedAngleDummySides(A);
stopGlobalRotation;
stopAutoLabeling;
} и
\drawFromCurrentPicture{
startAutoLabeling;
startGlobalRotation(-angleDirection.D);
draw byNamedAngle(D);
draw byNamedAngleDummySides(D);
stopGlobalRotation;
stopAutoLabeling;
}, заключенные между равными сторонами, равны.}

Если совместить равные основания \drawUnitLine{BC} и \drawUnitLine{EF} так, чтобы треугольники находились по одну сторону, а их равные стороны \drawUnitLine{AB} и \drawUnitLine{DE}, \drawUnitLine{CA} и \drawUnitLine{FD} были смежными, вершина одного будет совпадать с вершиной другого \inprop[prop:I.VII].

Следовательно, стороны \drawUnitLine{AB} и \drawUnitLine{CA}, будут совпадать с \drawUnitLine{DE} и \drawUnitLine{FD}, $\therefore \drawAngle{A} = \drawAngle{D}$.

\qed
\stopProposition

\startProposition[title={Предложение IX. Задача}, reference=prop:I.IX]
\defineNewPicture{
pair A, B, C, D, E, F;
A := (0, 2u);
B := (-4/3u, 0);
C := B xscaled -1;
D := A yscaled -1;
E := 5/4[A, B];
F := 5/4[A, C];
draw byAngle(B, A, D, byblue, 0);
draw byAngle(C, A, D, byyellow, 0);
byLineDefine(B, C, byyellow, 0, 0);
byLineDefine(A, D, black, 0, 0);
byLineDefine(D, B, byblue, 0, 0);
byLineDefine(C, D, byblue, 0, 0);
byLineDefine(A, B, byred, 0, 0);
byLineDefine(C, A, byred, 0, 0);
byLineDefine(B, E, byred, 1, 0);
byLineDefine(C, F, byred, 1, 0);
draw byNamedLine(BC,AD);
draw byNamedLineSeq(0)(DB,CD);
draw byNamedLineSeq(0)(BE,AB,CA,CF);
draw byLabelsOnPolygon(D, B, A, C)(0, 0);
}
\drawCurrentPictureInMargin
\problemNP{Р}{ассечь}{данный прямолинейный угол \drawAngle{BAD,CAD} пополам.}

\startCenterAlign
Возьмем $\drawUnitLine{AB} = \drawUnitLine{CA}$ \inprop[prop:I.III]

проведем \drawUnitLine{BC}, на которой построим \drawLine{CD,DB,BC} \inprop[prop:I.I],\\
проведем \drawUnitLine{AD}.

Поскольку $\drawUnitLine{AB} = \drawUnitLine{CA}$ (постр.),\\
\drawUnitLine{AD} общая обоим треугольникам\\
и $\drawUnitLine{CD} = \drawUnitLine{DB}$ (постр.),

$\therefore \drawAngle{BAD} = \drawAngle{CAD}$ \inprop[prop:I.VIII].
\stopCenterAlign

\qed
\stopProposition

\startProposition[title={Предложение X. Задача}, reference=prop:I.X]
\defineNewPicture{
pair A, B, C, D;
A := (0, 3u);
B := (-7/4u, 0);
C := B xscaled -1;
D := 1/2[B, C];
draw byAngle(B, A, D, byblue, 0);
draw byAngle(C, A, D, byyellow, 0);
draw byLine(A, D, byred, 0, 0);
byLineDefine(D, B, black, 0, 0);
byLineDefine(C, D, black, 1, 0);
byLineDefine(A, B, byyellow, 0, 0);
byLineDefine(C, A, byblue, 0, 0);
draw byNamedLineSeq(0)(AB,CA,CD,DB);
draw byLabelsOnPolygon(B, A, C, D)(0, 0);
}
\drawCurrentPictureInMargin
\problemNP{Р}{ассечь}{данную ограниченную прямую линию \drawUnitLine{DB,CD}.}

\startCenterAlign
Построим \drawLine[bottom]{AB,CA,CD,DB} \inprop[prop:I.I],\\
проведем \drawUnitLine{AD}, делая $\drawAngle{BAD} = \drawAngle{CAD}$ \inprop[prop:I.IX],

Тогда $\drawUnitLine{DB} = \drawUnitLine{CD}$ \inprop[prop:I.IV],

поскольку $\drawUnitLine{AB} = \drawUnitLine{CA}$ (постр.)\\
$\drawAngle{BAD} = \drawAngle{CAD}$\\
и \drawUnitLine{AD} общая обоим треугольникам.

Следовательно, данная линия рассечена пополам.
\stopCenterAlign

\qed
\stopProposition

\startProposition[title={Предложение XI. Задача}, reference=prop:I.XI]
\defineNewPicture[1/4]{
pair A, B, C, D, E, F;
A := (0, 3u);
B := (-7/4u, 0);
C := B xscaled -1;
D := 1/2[B, C];
E := 3/2[D, B];
F := 3/2[D, C];
draw byAngle(A, D, B, byred, 0);
draw byAngle(C, D, A, byblue, 0);
draw byLine(A, D, byyellow, 0, 0);
byLineDefine(A, B, byblue, 0, 0);
byLineDefine(C, A, byblue, 0, 0);
draw byNamedLineSeq(0)(AB,CA);
draw byLine(D, B, black, 0, 0);
draw byLine(B, E, black, 1, 0);
draw byLine(C, D, byred, 0, 0);
draw byLine(F, C, byred, 1, 0);
draw byLabelsOnPolygon(F, C, D, B, E)(2, 0);
draw byLabelsOnPolygon(B, A, C)(2, 0);
}
\drawCurrentPictureInMargin
\problemNP{И}{з}{данной точки 
\drawFromCurrentPicture{
draw byNamedLineSeq(0)(DB,CD);
draw byLabelPoint(D, 90, 1);
}
на данной прямой \drawUnitLine{DB,CD} построить перпендикуляр.}

\startCenterAlign
Возьмем любую точку 
\drawFromCurrentPicture{
draw byNamedLineSeq(0)(CD,FC);
draw byLabelPoint(C, 90, 1);
} на данной прямой,\\
отсечем $\drawUnitLine{DB} = \drawUnitLine{CD}$ \inprop[prop:I.III],\\
построим \drawLine[bottom]{AB,CA,CD,DB} \inprop[prop:I.I],\\
проведем \drawUnitLine{AD} и она будет перпендикуляром к данной прямой.

Поскольку $\drawUnitLine{AB} = \drawUnitLine{CA}$ (постр.)\\
$\drawUnitLine{CD} = \drawUnitLine{DB}$ (пост.)\\
и \drawUnitLine{AD} общая обоим треугольникам.

$\therefore \drawAngle{ADB} = \drawAngle{CDA}$ \inprop[prop:I.VIII]

$\therefore \drawUnitLine{AD} \perp \drawUnitLine{DB,CD}$ \indef[def:X]
\stopCenterAlign

\qed
\stopProposition


\startProposition[title={Предложение XII. Задача}, reference=prop:I.XII]
\defineNewPicture{
pair A, B, C, D, E, F;
path c;
numeric r, a[];
A := (0, 3u);
B := (-7/4u, 0);
C := B xscaled -1;
D := 1/2[B, C];
E := 4/3[D, B];
F := 4/3[D, C];
r := abs(A-B);
c := fullcircle scaled 2r shifted A;
a1 := xpart(c intersectiontimes (F--1/2[B, C]));
a2 := xpart(c intersectiontimes (E--1/2[B, C]));
draw byAngle(A, D, B, byyellow, 0);
draw byAngle(C, D, A, byblue, 0);
draw byLine(A, D, byred, 0, 0);
byLineDefine(A, B, byblue, 0, 0);
byLineDefine(C, A, byblue, 0, 0);
draw byNamedLineSeq(0)(AB,CA);
draw byArc(A, B, C)(r, byred, 0, 0, 0, 0)(O);
draw byArcBE(A, a2-1/4, a2, r, byred, 1, 0, 0, 0)(Ol);
draw byArcBE(A, a1, a1+1/4, r, byred, 1, 0, 0, 0)(Or);
draw byLine(D, B, black, 0, 0);
draw byLine(B, E, black, 1, 0);
draw byLine(C, D, byyellow, 0, 0);
draw byLine(F, C, byyellow, 1, 0);
draw byLabelsOnPolygon(B, A, C)(2, 0);
draw byLabelLineEnd(B, A, 0);
draw byLabelLineEnd(D, A, 0);
draw byLabelLineEnd(C, A, 0);
}
\drawCurrentPictureInMargin
\problemNP{П}{ровести}{перпендикуляр к данной неограниченной прямой \drawUnitLine{DB,CD} из данной, не находящейся на ней точки \drawFromCurrentPicture[middle][pointA]{
startTempScale(scaleFactor/2);
draw byNamedLine(AD);
draw byNamedLineSeq(0)(AB,CA);
draw byLabelsOnPolygon(B, A, C)(2, 0);
stopTempScale;
}.}

\startCenterAlign
Взяв данную точку \pointA\ в качестве центра по одну сторону прямой, и любое расстояние, позволяющее достигнуть другой стороны, построим \drawArc{O}.

Возьмем $\drawUnitLine{DB} = \drawUnitLine{CD}$ \inprop[prop:I.X],\\
проведем \drawUnitLine{AB}, \drawUnitLine{CA} и \drawUnitLine{AD}.

Тогда $\drawUnitLine{AD} \perp \drawUnitLine{DB,CD}$.

Поскольку \inprop[prop:I.VIII], раз $\drawUnitLine{DB} = \drawUnitLine{CD}$ (постр.),\\
\drawUnitLine{AD} общая обоим треугольникам,\\
и $\drawUnitLine{AB} = \drawUnitLine{CA}$ \indef[def:XV],

$\therefore \drawAngle{ADB} = \drawAngle{CDA}$,

и $\therefore \drawUnitLine{AD} \perp \drawUnitLine{DB,CD}$ \indef[def:X]
\stopCenterAlign

\qed
\stopProposition

\startProposition[title={Предложение XIII. Теорема}, reference=prop:I.XIII]
\defineNewPicture{
pair A, B, C, D, E;
A := (0, 5/2u);
B := (-7/4u, 0);
C := B xscaled -1;
D := (xpart(A), ypart(B));
E := (2/3xpart(C), 2/3ypart(A));
draw byAngle(A, D, B, byyellow, 0);
draw byAngle(E, D, A, byred, 0);
draw byAngle(C, D, E, byblue, 0);
draw byLine(A, D, black, 0, 0);
draw byLine(E, D, byyellow, 0, 0);
draw byLine(B, C, byred, 0, 0);
draw byLabelsOnPolygon(C, D, B, noPoint)(0, 0);
draw byLabelLineEnd(E, D, 0);
draw byLabelLineEnd(A, D, 0);
}
\drawCurrentPictureInMargin
\problemNP{Е}{сли}{прямая линия \drawUnitLine{ED} восставленная на другой прямой линии \drawUnitLine{BC} образует с ней углы, то это будут либо два прямых угла, либо их сумма будет равна двум прямым углам.}

\startCenterAlign
Если \drawUnitLine{ED} $\perp$ к \drawUnitLine{BC} тогда,\\
\drawAngle{ADB,EDA} и $\drawAngle{CDE} = \drawTwoRightAngles$ \indef[def:X],

но если \drawUnitLine{ED} будет не $\perp$ к \drawUnitLine{BC},\\
проведем $\drawUnitLine{AD} \perp \drawUnitLine{BC}$; \inprop[prop:I.XI]\\
$\drawAngle{ADB} +\drawAngle{CDE,EDA} = \drawTwoRightAngles$ (постр.),\\
$\drawAngle{ADB} = \drawAngle{CDE,EDA} = \drawAngle{EDA} + \drawAngle{CDE}$

$\therefore \drawAngle{ADB} + \drawAngle{CDE,EDA} = \drawAngle{ADB} + \drawAngle{EDA} + \drawAngle{CDE}$ \inax[ax:II]

$= \drawAngle{ADB,EDA} + \drawAngle{CDE} = \drawTwoRightAngles$.
\stopCenterAlign

\qed
\stopProposition

\startProposition[title={Предложение XIV. Теорема}, reference=prop:I.XIV]
\defineNewPicture[1/4]{
pair A, B, C, D, E;
A := (u, 5/2u);
B := (-7/4u, 0);
C := B xscaled -1;
D := (0, 0);
E := (xpart(C), -1/2ypart(A));
draw byAngle(B, D, A, byyellow, 0);
draw byAngle(C, D, A, byblue, 0);
draw byAngle(E, D, C, byred, 0);
draw byLine(A, D, byred, 0, 0);
draw byLine(E, D, byyellow, 0, 0);
draw byLine(B, D, byblue, 0, 0);
draw byLine(C, D, black, 0, 0);
draw byLabelsOnPolygon(E, D, B, noPoint)(0, 0);
draw byLabelLineEnd(A, D, 0);
}
\drawCurrentPictureInMargin
\problemNP{Е}{сли}{две прямые \drawUnitLine{BD} и \drawUnitLine{CD} образуют с третьей \drawUnitLine{AD} смежные углы, находясь по разные стороны от нее, и эти углы \drawAngle{BDA} и \drawAngle{CDA} равны двум прямым углам, то эти прямые будут лежать на одной прямой.}

\startCenterAlign
Действительно, пусть \drawUnitLine{ED}, а не \drawUnitLine{CD}, будет продолжением \drawUnitLine{BD},\\
тогда $\drawAngle{BDA} + \drawAngle{CDA,EDC} = \drawTwoRightAngles$

но, согласно гипотезе $\drawAngle{BDA} + \drawAngle{CDA} = \drawTwoRightAngles$

$\therefore\drawAngle{CDA,EDC} = \drawAngle{CDA}$, \inax[ax:III];\\ 
что не имеет смысла \inax[ax:IX].

$\therefore \drawUnitLine{ED}$ не является продолжением \drawUnitLine{BD}, и то же можно показать для любой другой прямой линии, за исключением \drawUnitLine{CD}, $\therefore \drawUnitLine{CD}$ является продложением \drawUnitLine{BD}.
\stopCenterAlign

\qed
\stopProposition

\startProposition[title={Предложение XV. Теорема}, reference=prop:I.XV]
\defineNewPicture{
pair A, B, C, D, E;
A := (7/4u, 3/2u);
B := A scaled -1;
C := A xscaled -1;
D := C scaled -1;
E := (A--B) intersectionpoint (C--D);
draw byAngle(B, E, C, byyellow, 0);
draw byAngle(C, E, A, byred, 0);
draw byAngle(A, E, D, black, 0);
draw byAngle(D, E, B, byblue, 0);
draw byLine(A, B, byred, 0, 0);
draw byLine(C, D, black, 0, 0);
draw byLabelsOnPolygon(C, E, A, noPoint)(0, 0);
draw byLabelPoint(B, lineAngle.AB + 90, 1);
draw byLabelPoint(D, lineAngle.CD - 90, 1);
}
\drawCurrentPictureInMargin
\problemNP[2]{Е}{сли}{две прямых линии \drawUnitLine{AB} и \drawUnitLine{CD} пересекаются, вертикальные углы \drawAngle{BEC} и \drawAngle{AED}, \drawAngle{CEA} и \drawAngle{DEB} будут равны между собой.}

\startCenterAlign
$\drawAngle{BEC} + \drawAngle{CEA} = \drawTwoRightAngles$

$\drawAngle{AED} + \drawAngle{CEA} = \drawTwoRightAngles$

$\therefore \drawAngle{BEC} = \drawAngle{AED}$.

Таким же образом можно показать, что\\
$\drawAngle{CEA} = \drawAngle{DEB}$
\stopCenterAlign

\qed
\stopProposition

\startProposition[title={Предложение XVI. Теорема}, reference=prop:I.XVI]
\defineNewPicture[1/4]{
pair A, B, C, D, E, F, G;
A := (0, 0);
B := A shifted (3/2u, 7/2u);
C := A shifted (3u, 0);
D := B shifted (3u, 0);
E = whatever[A, D] = whatever[B, C];
F := (xpart(D), ypart(A));
G := 4/3[B, C];
draw byAngleWithName(B, A, C, byblue, 0)(A);
draw byAngleWithName(C, B, A, black, 0)(B);
draw byAngle(A, E, B, byyellow, 0);
draw byAngle(D, E, C, byyellow, 0);
draw byAngle(E, C, D, black, 0);
draw byAngle(G, C, A, byred, 0);
draw byAngle(D, C, F, black, 1);
byLineDefine(C, F, black, 1, 0);
byLineDefine(C, G, black, 0, 0);
byLineDefine(B, E, byblue, 0, 0);
byLineDefine(E, C, byblue, 1, 0);
byLineDefine(A, E, byred, 0, 0);
byLineDefine(E, D, byred, 1, 0);
byLineDefine(A, B, byyellow, 1, 0);
byLineDefine(A, C, black, 0, 0);
byLineDefine(C, D, byyellow, 0, 0);
draw byNamedLineSeq(0)(AE,ED,CD);
draw byNamedLineSeq(0)(EC,CG,noLine,CF,AC,AB,BE);
draw byLabelsOnPolygon(F, A, B, E, D, C)(2, 0);
draw byLabelsOnPolygon(F, C, G, noPoint)(0, 0);
}
\drawCurrentPictureInMargin
\problemNP[2]{П}{ри}{продолжении стороны треугольника \drawLine[bottom]{BE,EC,AC,AB} внешний угол \drawFromCurrentPicture[bottom][anglesECDpDCF]{
startAutoLabeling;
draw byNamedAngleSides(ECD,DCF)(CF);
stopAutoLabeling;
} будет больше любого из противолежащих ему внутренних углов \drawAngle{B} or \drawAngle{A}.
}

\startCenterAlign
Сделаем $\drawUnitLine{BE} = \drawUnitLine{EC}$ \inprop[prop:I.X];\\
проведем \drawUnitLine{AE} и продлим до $\drawUnitLine{ED} = \drawUnitLine{AE}$;\\
проведем \drawUnitLine{CD}. В \drawLine{BE,AE,AB} и \drawLine{EC,ED,CD};\\
$\drawUnitLine{BE} = \drawUnitLine{EC}$, $\drawAngle{AEB} = \drawAngle{DEC}$ \inprop[prop:I.XV] и $\drawUnitLine{AE} = \drawUnitLine{ED}$ (постр.),\\
$\therefore \drawAngle{B} = \drawAngle{ECD}$ \inprop[prop:I.IV],\\
$\therefore \anglesECDpDCF\ > \drawAngle{ECD}$.

Так же можно показать, что при продлении \drawUnitLine{AC,CF}, $\drawAngle{GCA} > \drawAngle{A}$ \\
и, следовательно \anglesECDpDCF\ который $= \drawAngle{GCA}$ будет $> \drawAngle{A}$.
\stopCenterAlign

\qed
\stopProposition

\startProposition[title={Предложение XVII. Теорема}, reference=prop:I.XVII]
\defineNewPicture[1/4]{
pair A, B, C, D;
A := (0, 0);
B := A shifted (3/2u, 5/2u);
C := A shifted (9/4u, 0);
D := C shifted (u, 0);
draw byAngleWithName(B, A, C, byblue, 0)(A);
draw byAngleWithName(A, B, C, black, 0)(B);
draw byAngle(A, C, B, byred, 0);
draw byAngle(B, C, D, byyellow, 0);
byLineDefine(A, B, byred, 0, 0);
byLineDefine(B, C, byblue, 0, 0);
byLineDefine(A, C, black, 0, 0);
byLineDefine(C, D, black, 0, 0);
draw byNamedLineSeq(0)(noLine,BC,AB,AC,CD);
draw byLabelsOnPolygon(D, C, A, B)(0, 0);
}
\drawCurrentPictureInMargin
\problemNP{В}{о}{всяком треугольнике \drawLine[bottom]{AB,BC,AC} любые два угла взятые вместе меньше двух прямых углов.}

\startCenterAlign
Продлим \drawUnitLine{AC}, тогда\\
$\drawAngle{ACB} + \drawAngle{BCD} = \drawTwoRightAngles$

Но $\drawAngle{BCD} > \drawAngle{A}$ \inprop[prop:I.XVI]

$\therefore \drawAngle{ACB} + \drawAngle{A} < \drawTwoRightAngles$,
\stopCenterAlign

\noindent и таким же образом можно показать, что любые два других угла вместе будут меньше двух прямых углов.

\qed
\stopProposition

\startProposition[title={Предложение XVIII. Теорема}, reference=prop:I.XVIII]
\defineNewPicture[1/4]{
pair A, B, C, D;
A := (0, 0);
B := A shifted (5/2u, -1/2u);
C := A shifted (3/2u, 2u);
D := 2[C, A];
draw byAngleWithName(C, A, B, byblue, 0)(A);
draw byAngle(A, B, C, black, 0);
draw byAngle(D, B, A, byred, 0);
draw byAngleWithName(B, D, A, byyellow, 0)(D);
draw byLine(A, B, byyellow, 0, 0);
byLineDefine(A, C, byred, 0, 0);
byLineDefine(B, C, byblue, 0, 0);
byLineDefine(B, D, black, 0, 0);
byLineDefine(D, A, byred, 1, 0);
draw byNamedLineSeq(0)(AC,BC,BD,DA);
draw byLabelsOnPolygon(A, C, B, D)(0, 0);
}
\drawCurrentPictureInMargin
\problemNP{В}{о}{всяком треугольнике \drawLine{AC,BC,BD,DA}, если одна сторона \drawUnitLine{DA,AC} больше другой \drawUnitLine[0.5cm]{BC}, то противолежащий большей стороне угол будет больше противолежащего меньшей стороне угла, т. е. $\drawAngle{ABC,DBA} > \drawAngle{D}$}

\startCenterAlign
Сделаем $\drawUnitLine{AC} = \drawUnitLine{BC}$ \inprop[prop:I.III],\\ проведем \drawUnitLine{AB},

Тогда $\drawAngle{A} = \drawAngle{ABC}$ \inprop[prop:I.V];

но $\drawAngle{ABC} > \drawAngle{D}$ \inprop[prop:I.XVI]

$\therefore \drawAngle{ABC} > \drawAngle{D}$\\
и тем более $\drawAngle{ABC,DBA} > \drawAngle{D}$.
\stopCenterAlign

\qed
\stopProposition

\startProposition[title={Предложение XIX. Теорема}, reference=prop:I.XIX]
\defineNewPicture[1/4]{
pair A, B, C;
A := (0, 0);
B := A shifted (7/2u, 0);
C := A shifted (u, 3u);
draw byAngleWithName(C, A, B, byblue, 0)(A);
draw byAngleWithName(A, B, C, byred, 0)(B);
byLineDefine(A, B, black, 0, 0);
byLineDefine(B, C, byblue, 0, 0);
byLineDefine(C, A, byred, 0, 0);
draw byNamedLineSeq(0)(CA,BC,AB);
draw byLabelsOnPolygon(B, A, C)(0, 0);
}
\drawCurrentPictureInMargin
\problemNP{В}{о}{всяком треугольнике \drawLine[bottom]{CA,BC,AB} если один угол \drawAngle{A} больше другого \drawAngle{B}, то сторона \drawUnitLine{BC} противолежащая большему углу, больше стороны \drawUnitLine{CA} противолежащей меньшему.}

\startCenterAlign
Если \drawUnitLine{BC} не больше \drawUnitLine{CA} тогда\\
$\drawUnitLine{BC} =$ или $< \drawUnitLine{CA}$.

Если $\drawUnitLine{BC} = \drawUnitLine{CA}$ тогда\\
$\drawAngle{A} = \drawAngle{B}$ \inprop[prop:I.V],\\
что противоречит гипотезе.

\drawUnitLine{BC} также не меньше \drawUnitLine{CA};\\ 
поскольку, если это так, то $\drawAngle{A} < \drawAngle{B}$ \inprop[prop:I.XVIII]\\
что противоречит гипотезе:

$\therefore \drawUnitLine{BC} > \drawUnitLine{CA}$.
\stopCenterAlign

\qed
\stopProposition

\startProposition[title={Предложение XX. Теорема}, reference=prop:I.XX]
\defineNewPicture{
pair A, B, C, D;
A := (0, 0);
B := A shifted (7/2u, 0);
D := A shifted (4/3u, 3/2u);
C := ((fullcircle scaled 2arclength(D--B)) shifted D) intersectionpoint (D--10[A, D]);
draw byAngleWithName(B, C, A, byred, 0)(C);
draw byAngle(C, B, D, byblue, 0);
draw byAngle(D, B, A, byyellow, 0);
byLineDefine(B, D, byred, 0, 0);
byLineDefine(A, B, black, 0, 0);
byLineDefine(B, C, byyellow, 0, 0);
byLineDefine(C, D, byblue, 1, 0);
byLineDefine(D, A, byblue, 0, 0);
draw byNamedLineSeq(0)(BD);
draw byNamedLineSeq(0)(DA,CD,BC,AB);
draw byLabelsOnPolygon(D, C, B, A)(0, 0);
}
\drawCurrentPictureInMargin
\problemNP{Л}{юбые}{две стороны \drawUnitLine{DA} и \drawUnitLine{BD} всякого треугольника \drawLine[bottom]{DA,BD,AB} взятые вместе больше третьей стороны  \drawUnitLine{AB}.}

\startCenterAlign
Продлим \drawUnitLine{DA},\\
и сделаем $\drawUnitLine{CD} = \drawUnitLine{BD}$ \inprop[prop:I.III];\\
проведем \drawUnitLine{BC}.

Тогда, поскольку $\drawUnitLine{CD} = \drawUnitLine{BD}$ (постр.),\\
$\drawAngle{CBD} = \drawAngle{C}$ \inprop[prop:I.V]

$\therefore \drawAngle{CBD,DBA} > \drawAngle{C}$ \inax[ax:IX]

$\therefore \drawUnitLine{DA} + \drawUnitLine{CD} > \drawUnitLine{AB}$ \inprop[prop:I.XIX]

и $\therefore \drawUnitLine{DA} + \drawUnitLine{BD} > \drawUnitLine{AB}$
\stopCenterAlign

\qed
\stopProposition

\startProposition[title={Предложение XXI. Теорема}, reference=prop:I.XXI]
\defineNewPicture{
pair A, B, C, D, E;
A := (0, 0);
B := A shifted (7/2u, 0);
C := A shifted (3u, 4u);
D := 1/2[1/2[A, B], C];
E = whatever[A, D] = whatever[B, C];
draw byAngleWithName(B, D, A, byred, 0)(D);
draw byAngleWithName(B, E, D, byblue, 0)(E);
draw byAngleWithName(B, C, A, byyellow, 0)(C);
byLineDefine(B, D, byyellow, 0, 0);
byLineDefine(A, D, black, 0, 0);
byLineDefine(D, E, black, 1, 0);
byLineDefine(A, B, byblue, 1, 0);
byLineDefine(B, E, byred, 1, 0);
byLineDefine(E, C, byred, 0, 0);
byLineDefine(C, A, byblue, 0, 0);
draw byNamedLine(BD);
draw byNamedLineSeq(0)(AD,DE);
draw byNamedLineSeq(0)(CA,EC,BE,AB);
draw byLabelsOnPolygon(A, C, E, B)(0, 0);
draw byLabelsOnPolygon(A, D, E)(2, 0);
}
\drawCurrentPictureInMargin
\problemNP[2]{Е}{сли}{из любой точки \drawFromCurrentPicture{
startTempScale(1/5);
draw byNamedLineSeq(0)(AD,BD);
draw byLabelsOnPolygon(A, D, B)(2, 0);
stopTempScale;
} внутри треугольника \drawFromCurrentPicture[bottom]{
startAutoLabeling;
startTempScale(1/5);
draw byNamedLineSeq(0)(CA,EC,BE,AB);
stopTempScale;
stopAutoLabeling;
} провести прямые линии к концам стороны \drawSizedLine{AB}, эти прямые  вместе меньше двух других сторон треугольника, и будут заключать больший угол.}

\startCenterAlign
Продлим \drawSizedLine{AD},\\
$\drawSizedLine{CA} + \drawSizedLine{EC} > \drawSizedLine{AD,DE}$ \inprop[prop:I.XX],\\
добавим к каждой \drawSizedLine{BE},\\
$\drawSizedLine{CA} + \drawSizedLine{EC,BE} > \drawSizedLine{AD,DE} + \drawSizedLine{BE}$ \inax[ax:IV]

Таким же образом можно показать, что\\
$\drawSizedLine{AD,DE} + \drawSizedLine{BE} > \drawSizedLine{AD} + \drawSizedLine{BD}$,\\
$\therefore \drawSizedLine{CA} + \drawSizedLine{EC,BE} > \drawSizedLine{AD} + \drawSizedLine{BD}$,\\
что и требовалось доказать.

Далее $\drawAngle{E} > \drawAngle{C}$ \inprop[prop:I.XVI],\\
и также $\drawAngle{D} > \drawAngle{E}$ \inprop[prop:I.XVI],

$\therefore \drawAngle{D} > \drawAngle{C}$.
\stopCenterAlign

\qed
\stopProposition

\startProposition[title={Предложение XXII. Задача}, reference=prop:I.XXII]
\defineNewPicture[1/2]{
numeric r[], d;
pair A, B, C, D, E, LI, LII, LIII, LIV, LV, LVI;
path q[];
r1 := 3/2u;
r2 := 4/3u;
r3 := (2/3)*(r1+r2);
d := 1/3u;
A := (0, 0);
B := A shifted (r3, 0);
q1 := (fullcircle scaled 2r1) shifted A;
q2 := (fullcircle scaled 2r2) shifted B;
C := q1 intersectionpoint q2;
D := point 11/2 of q1;
E := point 3/4 of q2;
LI := (xpart(point 0 of q2), ypart(point 6 of q1) - 1/2d);
LII := LI shifted (-r3, 0);
LIII := LI shifted (0, -d);
LIV := LIII shifted (-r2, 0);
LV := LIII shifted (0, -d);
LVI := LV shifted (-r1, 0);
draw byCircle(A, D, byblue, 0, 0, 0)(A);
byLineDefine(A, D, byblue, 0, 0);
byLineDefine(B, E, byred, 0, 0);
byLineDefine(A, B, black, 0, 0);
byLineDefine(B, C, byyellow, 0, 0);
byLineDefine(C, A, byyellow, 1, 0);
draw byNamedLineSeq(0)(BC,CA);
draw byNamedLineSeq(0)(AD,AB,BE);
draw byLineWithName (LII, LI, black, 1, 0)(L');
draw byLineWithName (LIV, LIII, byred, 1, 0)(L'');
draw byLineWithName (LVI, LV, byblue, 1, 0)(L''');
draw byCircle(B, E, byred, 0, 0, 0)(B);
draw byLabelsOnPolygon(D, A, C)(2, 0);
draw byLabelsOnPolygon(E, B, A)(2, 0);
draw byLabelsOnPolygon(A, C, B)(2, 0);
draw byLabelLineEnd(D, A, 0);
draw byLabelLineEnd(E, B, 0);
draw byLabelLine(L', L'', L''');
}
\drawCurrentPictureInMargin
\problemNP{И}{з}{трех прямых линий $\left\{\vcenter{
\nointerlineskip\hbox{\drawSizedLine{L'}}
\nointerlineskip\hbox{\drawSizedLine{L''}}
\nointerlineskip\hbox{\drawSizedLine{L'''}}}\right.$
таких, что любые две вместе длиннее третьей, составить треугольник.}

\startCenterAlign
Предположим $\drawSizedLine{AB} = \drawSizedLine{L'}$ \inprop[prop:I.III].

$\left.\eqalign{
\mbox{Проведем } \drawSizedLine{AD} &= \drawSizedLine{L'''}\cr
\mbox{и } \drawSizedLine{BE} &= \drawSizedLine{L''}
}\right\}\mbox{\inprop[prop:I.II].}$

Взяв \drawSizedLine{AD} и \drawSizedLine{BE} как радиусы, опишем
\drawFromCurrentPicture{
draw byNamedLine(AD); draw byNamedCircle(A);
draw byLabelLineEnd(A, D, 0);
draw byLabelLineEnd(D, A, 0);
} и
\offsetPicture{12pt}{0pt}{\drawFromCurrentPicture{
draw byNamedLine(BE); draw byNamedCircle(B);
draw byLabelLineEnd(B, E, 0);
draw byLabelLineEnd(E, B, 0);
}} \inpost[post:III];\\
проведем \drawSizedLine{CA} и \drawSizedLine{BC},\\
тогда \drawLine[bottom]{CA,BC,AB} будет искомым треугольником.

$\left.\eqalign{
\mbox{Поскольку } \drawSizedLine{AB} &= \drawSizedLine{L'} \mbox{,} \cr
\drawSizedLine{BC} &= \drawSizedLine{BE} = \drawSizedLine{L''} \cr
\mbox{и } \drawSizedLine{CA} &= \drawSizedLine{AD} = \drawSizedLine{L'''} \cr
}\right\}\mbox{(постр.)}$
\stopCenterAlign

\qed
\stopProposition

\startProposition[title={Предложение XXIII. Задача}, reference=prop:I.XXIII]
\defineNewPicture{
angleScale := 3/2;
pair A, B, C, D, E, F, G, H, J, d;
A := (0, 0);
B := A shifted (7/2u, 0);
C := A shifted (3u, 11/5u);
D := 5/4[A, B];
E := 7/6[A, C];
d := (0, -3u);
F := A shifted d;
G := B shifted d;
H := C shifted d;
J := D shifted d;
draw byAngleWithName(B, A, C, byred, 0)(A);
draw byAngleWithName(G, F, H, byblue, 0)(F);
byLineDefine(B, D, black, 1, 1);
byLineDefine(C, E, byblue, 1, 1);
byLineDefine(A, B, black, 0, 1);
byLineDefine(C, A, byblue, 0, 1);
draw byLine(B, C, byred, 0, 1);
draw byNamedLineSeq(0)(CE,CA,AB,BD);
byLineDefine(G, J, black, 1, 0);
byLineDefine(F, G, black, 0, 0);
byLineDefine(G, H, byred, 0, 0);
byLineDefine(H, F, byyellow, 0, 0);
draw byNamedLineSeq(0)(noLine,GH,HF,FG,GJ);
draw byLabelsOnPolygon(D, B, A, C, E, noPoint)(0, 0);
draw byLabelsOnPolygon(noPoint, J, G, F, H, G)(2, 0);
}
\drawCurrentPictureInMargin
\problemNP{П}{ри}{данной точке \drawFromCurrentPicture{
startTempScale(scaleFactor/2);
draw byNamedLineSeq(0)(FG,HF);
draw byLabelsOnPolygon(G, F, H)(2, 0);
stopTempScale;
} на данной прямой \drawUnitLine{FG,GJ}, построить угол равный данному прямолинейному углу \drawAngle{A}.}

Проведем \drawUnitLine{BC} между любыми двумя точками сторонами данного угла.

\startCenterAlign
Построим \drawLine[bottom]{HF,GH,FG} \inprop[prop:I.XXII]\\
такой, что $\drawUnitLine{FG} = \drawUnitLine{AB}$, $\drawUnitLine{HF} = \drawUnitLine{CA}$\\
и $\drawUnitLine{GH} = \drawUnitLine{BC}$.

Тогда $\drawAngle{A} = \drawAngle{F}$ \inprop[prop:I.VIII].
\stopCenterAlign

\qed
\stopProposition

\startProposition[title={Предложение XXIV. Теорема}, reference=prop:I.XXIV]
\defineNewPicture{
pair A, B, C, D, E, F, G, d;
A := (0, 0);
B := A shifted (u, -5/2u);
C := A shifted (-u, -7/2u);
D := (xpart(C) - 3/2u, ypart(B));
d := (0, -4u);
E := A shifted d;
F := B shifted d;
G := C shifted d;
draw byAngle(B, A, C, black, 2);
draw byAngle(C, A, D, black, 3);
draw byAngle(F, E, G, black, 2);
draw byAngle(B, D, A, byblue, 0);
draw byAngle(C, D, B, byred, 0);
draw byAngle(D, C, A, byyellow, 0);
draw byAngle(A, C, B, black, 0);
byLineDefine(A, B, byblue, 0, 0);
byLineDefine(B, C, black, 1, 0);
draw byLine(C, A, byred, 0, 0);
draw byLine(B, D, black, 0, 0);
byLineDefine(A, D, byred, 1, 0);
byLineDefine(C, D, byblue, 1, 0);
draw byNamedLineSeq(0)(AB,AD,CD,BC);
byLineDefine(E, F, byblue, 0, 1);
byLineDefine(F, G, byyellow, 0, 1);
byLineDefine(G, E, byred, 0, 1);
draw byNamedLineSeq(0)(EF,FG,GE);
draw byLabelsOnPolygon(D, A, B, C)(0, 0);
draw byLabelsOnPolygon(G, E, F)(0, 0);
}
\drawCurrentPictureInMargin
\problemNP{Е}{сли}{у двух треугольников по две стороны соотверственно равны друг другу ($\drawUnitLine{AB} = \drawUnitLine{EF}$ и $\drawUnitLine{AD} = \drawUnitLine{GE}$), и угол заключенный ними в одном
\drawFromCurrentPicture[bottom]{
startAutoLabeling;
draw byNamedAngleSides(BAC,CAD)(AB,CA,AD);
stopAutoLabeling;
}
больше, чем в другом
\drawFromCurrentPicture[bottom][angleFEG]{
startAutoLabeling;
draw byNamedAngleSides(FEG)(EF, GE);
stopAutoLabeling;
},
то сторона \drawUnitLine{BD} противолежащая большему углу больше стороны, противолежащей меньшему \drawUnitLine{FG}.
}

\startCenterAlign
Сделаем $\drawFromCurrentPicture[bottom][angleBAC]{
startAutoLabeling;
draw byNamedAngleSides(BAC)(AB, CA);
stopAutoLabeling;
} = \angleFEG$ \inprop[prop:I.XXIII],\\
и $\drawUnitLine{CA} = \drawUnitLine{GE}$ \inprop[prop:I.III],\\
проведем \drawUnitLine{CD} и \drawUnitLine{BC}.

Поскольку $\drawUnitLine{CA} = \drawUnitLine{AD}$ (\inaxL[ax:I]. гип. постр.)\\
$\therefore \drawAngle{BDA,CDB} = \drawAngle{DCA}$ \inprop[prop:I.V]
но $\drawAngle{CDB} < \drawAngle{DCA}$,\\
и $\therefore \drawAngle{CDB} < \drawAngle{DCA,ACB}$,

$\therefore \drawUnitLine{BD} > \drawUnitLine{BC}$ \inprop[prop:I.XIX]

но $\drawUnitLine{BC} = \drawUnitLine{FG}$ \inprop[prop:I.IV]

$\therefore \drawUnitLine{BD} > \drawUnitLine{FG}$.
\stopCenterAlign

\qed
\stopProposition

\startProposition[title={Предложение XXV. Теорема}, reference=prop:I.XXV]
\defineNewPicture{
pair A, B, C, D, E, F, d;
A := (0, 0);
B := A shifted (u, -3u);
C := A shifted (-7/4u, -4u);
d := (0, -9/2u);
D := A shifted d;
E := ((B shifted -A) rotated -10) shifted d;
F := C shifted d;
draw byAngleWithName(B, A, C, byyellow, 0)(A);
draw byAngleWithName(E, D, F, black, 0)(D);
byLineDefine(A, B, byblue, 0, 0);
byLineDefine(B, C, black, 0, 0);
byLineDefine(C, A, byred, 0, 0);
draw byNamedLineSeq(0)(AB,BC,CA);
byLineDefine(D, E, byblue, 0, 1);
byLineDefine(E, F, byyellow, 0, 1);
byLineDefine(F, D, byred, 0, 1);
draw byNamedLineSeq(0)(DE,EF,FD);
draw byLabelsOnPolygon(A, B, C)(0, 0);
draw byLabelsOnPolygon(D, E, F)(0, 0);
}
\drawCurrentPictureInMargin
\problemNP{Е}{сли}{у двух треугольников две стороны \drawUnitLine{AB} и \drawUnitLine{CA} соответственно равны двум сторонам \drawUnitLine{DE} и \drawUnitLine{FD} другого, но основания неравны, то угол над большим основанием \drawUnitLine{BC} одного треугольника меньше угла под меньшим \drawUnitLine{EF} другого.}

\startCenterAlign
$\drawAngle{A} =\mbox{, } > \mbox{ или } < \drawAngle{D}$

\drawAngle{A} не равен \drawAngle{D}\\
поскольку если $\drawAngle{A} = \drawAngle{D}$ то $\drawUnitLine{BC} = \drawUnitLine{EF}$ \inprop[prop:I.IV]\\
что противоречит гипотезе;

\drawAngle{A} не меньше \drawAngle{D}\\
поскольку если $\drawAngle{A} < \drawAngle{D}$\\
то $\drawUnitLine{BC} < \drawUnitLine{EF}$ \inprop[prop:I.XXIV],\\
что противоречит гипотезе:

$\therefore \drawAngle{A} > \drawAngle{D}$.
\stopCenterAlign

\qed
\stopProposition

\startProposition[title={Предложение XXVI. Теорема}, reference=prop:I.XXVI]
\defineNewPicture{
pair A, B, C, D, E, F, G, d;
A := (0, 0);
B := A shifted (3u, 0);
C := A shifted (2u, 3u);
d := (0, -4u);
D := A shifted d;
E := B shifted d;
F := C shifted d;
G := 3/4[D, F];
draw byAngleWithName(B, A, C, byyellow, 0)(A);
draw byAngleWithName(C, B, A, byred, 0)(B);
draw byAngleWithName(A, C, B, black, 1)(C);
byLineDefine(A, B, byblue, 0, 0);
byLineDefine(B, C, black, 0, 0);
byLineDefine(C, A, byred, 0, 0);
draw byNamedLineSeq(0)(CA,BC,AB);
draw byAngleWithName(E, D, F, byyellow, 0)(D);
draw byAngle(G, E, D, black, 0);
draw byAngle(F, E, G, byblue, 0);
draw byAngleWithName(D, F, E, black, 1)(F);
draw byLine(E, G, byyellow, 0, 1);
byLineDefine(D, E, byblue, 0, 1);
byLineDefine(E, F, black, 0, 1);
byLineDefine(F, G, byred, 1, 1);
byLineDefine(G, D, byred, 0, 1);
draw byNamedLineSeq(0)(GD,FG,EF,DE);
draw byLabelsOnPolygon(C, B, A)(0, 0);
draw byLabelsOnPolygon(D, G, F, E)(0, 0);
}
\problemNP{Е}{сли}{у двух треугольников два угла соответственно равны двум углам другого ($\drawAngle{A} = \drawAngle{D}$ и $\drawAngle{B} = \drawAngle{GED,FEG}$), и одна сторона равна одного равна так же расположенной стороне другого, то и остальные стороны и углы соответственно равны друг другу.}
\drawCurrentPictureInMargin
\startsubproposition[title={Случай I.}]
\startCenterAlign
Пусть  \drawUnitLine{AB} и \drawUnitLine{DE}, лежащие между равными углами равны,\\
тогда $\drawUnitLine{CA} = \drawUnitLine{GD,FG}$.

Поскольку, если \drawUnitLine{GD,FG} больше,\\
сделаем $\drawUnitLine{CA} = \drawUnitLine{GD}$, проведем \drawUnitLine{EG}.

В \drawLine[bottom]{CA,BC,AB} и
\drawLine[bottom]{GD,EG,DE} получим \\
$\drawUnitLine{CA} = \drawUnitLine{GD}$, $\drawAngle{A} = \drawAngle{D}$, $\drawUnitLine{AB} = \drawUnitLine{DE}$;\\
$\therefore \drawAngle{B} = \drawAngle{GED}$ (pr. 4.)\\
но $\drawAngle{B} = \drawAngle{GED,FEG}$ (hyp.)

и следовательно $\drawAngle{GED} = \drawAngle{GED,FEG}$, что не имеет смысла, а значит ни \drawUnitLine{CA} ни \drawUnitLine{GD,FG} не больше другой, и $\therefore$ они равны;

$\therefore \drawUnitLine{BC} = \drawUnitLine{EF}$, и $\drawAngle{C} = \drawAngle{F}$ \inprop[prop:I.IV].
\stopCenterAlign
\stopsubproposition

\vfill\pagebreak

\defineNewPicture{
pair A, B, C, D, E, F, G, d;
d := (0, -4u);
A := (0, 0);
B := A shifted (3u, 0);
C := A shifted (1u, 3u);
D := A shifted d;
E := B shifted d;
F := C shifted d;
G := 3/4[D, E];
draw byAngleWithName(B, A, C, byyellow, 0)(A);
draw byAngleWithName(C, B, A, byred, 0)(B);
byLineDefine(A, B, byblue, 0, 0);
byLineDefine(B, C, black, 0, 0);
byLineDefine(C, A, byred, 0, 0);
draw byNamedLineSeq(0)(CA,AB,BC);
draw byAngleWithName(F, D, E, byyellow, 0)(D);
draw byAngleWithName(F, G, D, black, 0)(G);
draw byAngleWithName(F, E, D, byred, 0)(E);
draw byLine(F, G, byyellow, 0, 1);
byLineDefine(D, G, byblue, 0, 1);
byLineDefine(G, E, byblue, 1, 1);
byLineDefine(E, F, black, 0, 1);
byLineDefine(F, D, byred, 0, 1);
draw byNamedLineSeq(0)(FD,EF,GE,DG);
draw byLabelsOnPolygon(C, B, A)(0, 0);
draw byLabelsOnPolygon(D, F, E, G)(0, 0);
}
\drawCurrentPictureInMargin
\startsubproposition[title={Случай II.}]
\startCenterAlign
Теперь пусть $\drawUnitLine{CA} = \drawUnitLine{FD}$, лежат против равных углов \drawAngle{B} и \drawAngle{E}.\\ 
Если такое возможно, пусть $\drawUnitLine{DG,GE} > \drawUnitLine{AB}$, тогда возьмем $\drawUnitLine{DG} = \drawUnitLine{AB}$, проведем \drawUnitLine{FG}.

Тогда в \drawLine[bottom]{CA,BC,AB} и \drawLine[bottom]{FD,FG,DG} получим $\drawUnitLine{CA} = \drawUnitLine{FD}$, $\drawUnitLine{AB} = \drawUnitLine{DG}$ и $\drawAngle{A} = \drawAngle{D}$,

$\therefore \drawAngle{B} = \drawAngle{G}$ \inprop[prop:I.IV]\\
но $\drawAngle{B} = \drawAngle{E}$ (гип.)

$\therefore \drawAngle{G} = \drawAngle{E}$ что не имеет смысла \inprop[prop:I.XVI]

Следовательно, ни \drawUnitLine{AB} ни \drawUnitLine{DG,GE} не больше другой, а значит они равны. Следовательно (согласно \inpropL[prop:I.IV]) треугольники равны во всех отношениях.
\stopCenterAlign
\stopsubproposition

\qed
\stopProposition

\startProposition[title={Предложение XXVII. Теорема}, reference=prop:I.XXVII]
\defineNewPicture{
pair A, B, C, D, E, F, G, H, I, d;
A := (0, 0);
B := A shifted (8/3u, 0);
d := (0, -7/4u);
C := A shifted d;
D := B shifted d;
E := 1/3[A, B];
F := 2/3[C, D];
G := 3/2[F, E];
H := 3/2[E, F];
I := 1/2[A, C] shifted (-2u, 0);
draw byAngle(A, E, F, byyellow, 0);
draw byAngle(F, E, B, byred, 0);
draw byAngle(C, F, E, byblue, 0);
draw byAngle(E, F, D, byyellow, 0);
byLineDefine(I, A, byblue, 0, 0);
byLineDefine(A, B, byblue, 0, 0);
byLineDefine(I, C, byred, 0, 0);
byLineDefine(C, D, byred, 0, 0);
draw byNamedLineSeq(0)(CD,IC,IA,AB);
draw byLine(G, H, black, 0, 0);
draw byLabelLine(AB, CD, GH);
}
\drawCurrentPictureInMargin
\problemNP{Е}{сли}{прямая \drawUnitLine{GH}, пересекая две другие прямые \drawUnitLine{CD} и \drawUnitLine{AB}, образует накрестлежащие углы \drawAngle{CFE}, \drawAngle{FEB} и \drawAngle{EFD} и \drawAngle{AEF} равные между собой, то две пересекаемые прямые параллельны.}

Если \drawUnitLine{CD} не параллельна \drawUnitLine{AB}, то они сойдутся по продолжении.

Если это возможно, пусть они не будут параллельны, но сойдутся, если их продолжить; тогда внешний угол \drawAngle{FEB} будет больше \drawAngle{CFE} \inprop[prop:I.XVI], но они равны (гип.), что не имеет смысла. Таким же образом можно показать, что они не сойдутся с другой стороны; $\therefore$ они параллельны.

\qed
\stopProposition

\startProposition[title={Предложение XXVIII. Теорема}, reference=prop:I.XXVIII]
\defineNewPicture{
pair A, B, C, D, E, F, G, H, d;
A := (0, 0);
B := A shifted (7/2u, 0);
d := (0, -3/2u);
C := A shifted d;
D := B shifted d;
E := 9/20[A, B];
F := 11/20[C, D];
G := 7/4[F, E];
H := 4/3[E, F];
draw byAngle(G, E, A, black, 0);
draw byAngle(B, E, G, byyellow, 0);
draw byAngle(A, E, F, byred, 0);
draw byAngle(F, E, B, byblue, 0);
draw byAngle(C, F, E, byblue, 0);
draw byAngle(E, F, D, byred, 0);
draw byLine(A, B, byred, 0, 0);
draw byLine(C, D, byyellow, 0, 0);
draw byLine(G, H, black, 0, 0);
draw byLabelLine(AB, CD, GH);
}
\drawCurrentPictureInMargin
\problemNP{Е}{сли}{прямая \drawUnitLine{GH}, падающая на две прямые \drawUnitLine{AB} и \drawUnitLine{CD} образует внешний угол, равный внутреннему противолежащему с той же стороны $\drawAngle{GEA} = \drawAngle{CFE}$ или $\drawAngle{BEG} = \drawAngle{EFD}$, или внутренние углы с одной стороны \drawAngle{EFD} и \drawAngle{FEB} или \drawAngle{CFE} и \drawAngle{AEF} вместе равны двум прямым углам, то прямые параллельны.}

\startCenterAlign
Во-первых, если $\drawAngle{GEA} = \drawAngle{CFE}$,\\ 
то $\drawAngle{GEA} = \drawAngle{FEB}$ \inprop[prop:I.XV],\\
$\therefore \drawAngle{CFE} = \drawAngle{FEB} \therefore \drawUnitLine{AB} \parallel \drawUnitLine{CD}$ \inprop[prop:I.XXVII].

Во-вторых, если $\drawAngle{CFE} + \drawAngle{AEF} = \drawTwoRightAngles$,\\
то $\drawAngle{AEF} + \drawAngle{FEB} = \drawTwoRightAngles$ \inprop[prop:I.XIII],\\
$\therefore \drawAngle{CFE} + \drawAngle{AEF} = \drawAngle{AEF} + \drawAngle{FEB}$ \inax[ax:III]

$\therefore \drawAngle{CFE} = \drawAngle{FEB}$

$\therefore \drawUnitLine{AB} \parallel \drawUnitLine{CD}$
\stopCenterAlign

\qed
\stopProposition

\startProposition[title={Предложение XXIX. Теорема}, reference=prop:I.XXIX]
\defineNewPicture{
pair A, B, C, D, E, F, G, H, I, J, d[];
A := (0, 0);
B := A shifted (7/2u, 0);
d1 := (0, -2u);
C := A shifted d1;
D := B shifted d1;
E := 11/20[A, B];
F := 9/20[C, D];
G := 7/4[F, E];
H := 4/3[E, F];
d2 := (3/2u, 1/2u);
I := E shifted -d2;
J := E shifted d2;
draw byAngle(I, E, A, byblue, 0);
draw byAngle(F, E, I, byyellow, 0);
draw byAngle(B, E, F, black, 0);
draw byAngle(G, E, B, byred, 0);
draw byAngle(E, F, D, black, 0);
draw byLine(I, E, black, 0, 0);
draw byLine(E, J, black, 1, 0);
draw byLine(A, B, byyellow, 0, 0);
draw byLine(C, D, byred, 0, 0);
draw byLine(G, H, byblue, 0, 0);
draw byLabelLine(AB, CD, GH);
draw byLabelsOnPolygon(A, E, G)(2, 0);
draw byLabelPoint(I, lineAngle.IE - 90, 1);
draw byLabelPoint(J, lineAngle.EJ + 90, 1);
}
\drawCurrentPictureInMargin
\problemNP{П}{рямая}{\drawUnitLine{GH}, падающая на две параллельные прямые \drawUnitLine{AB} и \drawUnitLine{CD}, образует накрестлежащие углы, равные между собой, внешний и противолежащий с той же стороны внутренний углы, равные между собой, а также внутренние односторонние углы, равные двум прямым углам.}

Если накрестлежащие углы \drawAngle{IEA,FEI} и \drawAngle{EFD} не равны, проведем \drawUnitLine{IE}, так, чтобы $\drawAngle{FEI} = \drawAngle{EFD}$ \inprop[prop:I.XXIII].

Следовательно $\drawUnitLine{IE,EJ} \parallel \drawUnitLine{CD}$ \inprop[prop:I.XXVII] и, следовательно, две пересекающихся прямых параллельны одной и той же прямой, что невозможно \inax[ax:XII].

А значит \drawAngle{IEA,FEI} и \drawAngle{EFD} не являются неравными, то есть, они равны $\drawAngle{IEA,FEI} = \drawAngle{GEB}$ \inprop[prop:I.XV]; $\therefore \drawAngle{GEB} = \drawAngle{EFD}$, внешний угол равен внутреннему, противолежащему с той же стороны: если к обоим добавить \drawAngle{BEF}, то $\drawAngle{EFD} + \drawAngle{BEF} = \drawAngle{BEF,GEB} = \drawTwoRightAngles$ \inprop[prop:I.XIII]. Другими словами, два внутренних угла по одну сторону пересекающей прямой равны двум прямым углам.

\qed
\stopProposition

\startProposition[title={Предложение XXX. Теорема}, reference=prop:I.XXX]
\defineNewPicture{
pair A, B, C, D, E, F, G, H, I, J, K, d;
A := (0, 0);
B := A shifted (7/2u, 0);
d := (0, -u);
C := A shifted d;
D := B shifted d;
E := C shifted d;
F := D shifted d;
G := 13/20[A, B];
H := 7/20[E, F];
I := (G--H) intersectionpoint (C--D);
J := 3/2[H, G];
K := 5/4[G, H];
draw byAngleWithName(B, G, J, byyellow, 0)(G);
draw byAngleWithName(D, I, J, byblue, 0)(I);
draw byAngleWithName(F, H, J, byred, 0)(H);
draw byLine(A, B, byred, 0, 0);
draw byLine(C, D, byyellow, 0, 0);
draw byLine(E, F, byblue, 0, 0);
draw byLine(J, K, black, 0, 0);
draw byLabelLine(AB, CD, EF, JK);
draw byLabelsOnPolygon(J, G, A)(2, 0);
draw byLabelsOnPolygon(J, I, C)(2, 0);
draw byLabelsOnPolygon(J, H, E)(2, 0);
}
\drawCurrentPictureInMargin
\problemNP{П}{рямые}{\drawUnitLine{AB} и \drawUnitLine{EF} параллельные одной и той же прямой \drawUnitLine{CD}, параллельны между собой}

\startCenterAlign
Пусть \drawUnitLine{JK} пересекает $\left\{\vcenter{\nointerlineskip\hbox{\drawUnitLine{AB}}\nointerlineskip\hbox{\drawUnitLine{CD}}\nointerlineskip\hbox{\drawUnitLine{EF}}}\right\}$;

Тогда $\drawAngle{G} = \drawAngle{I} = \drawAngle{H}$ \inprop[prop:I.XXIX],

$\therefore \drawAngle{G} = \drawAngle{H}$

$\therefore \drawUnitLine{AB} \parallel \drawUnitLine{EF}$ \inprop[prop:I.XXVII].
\stopCenterAlign

\qed
\stopProposition

\stopbook
\stoptext
%\closeout \lettrineslist

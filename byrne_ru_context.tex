\input preamble.tex
\input preamble_be.tex

\mainlanguage[ru]

\def\inpropstr{пр.}
\def\inpoststr{пост.}
\def\indefstr{опр.}
\def\inaxstr{акс.}
\def\qedstr{Ч. Т. Д.}

\starttext
\startbook[title={Книга I}]
\startVerboseProposition[title={Предложение I. Задача}, reference=prop:I.I]

\defineNewPicture[1/2]{
	pair A, B, C;
	path P[];
	numeric r;
	r := 3/2u;
	A := (0, 0);
	B := (r, 0);
	P1 := fullcircle scaled 2r;
	P2 := fullcircle scaled 2r shifted B;
	C := P1 intersectionpoint P2;
		byLineDefine(A, B, black, 0, 0);
		byLineDefine(B, C, byred, 0, 0);
		byLineDefine(C, A, byyellow, 0, 0);
		draw byNamedLineSeq(1)(AB,CA,BC);
		draw byCircle(A, B, byblue, 0, 0, 1/2)(A);
		draw byCircle(B, A, byred, 0, 0, 1/2)(B);
		draw byLabelsOnPolygon(A, C, B)(0, -1);
}
\drawCurrentPictureInMargin
\problemNP{Н}{а}{данной ограниченной прямой \drawUnitLine{AB} построить равносторонний треугольник.}

\startCenterAlign
Опишем \offsetPicture{15pt}{0pt}{\drawFromCurrentPicture{
draw byNamedLine(AB);
draw byNamedCircle(A);
draw byLabelLineEnd(A, B, 0);
draw byLabelLineEnd(B, A, 1);
}} и \offsetPicture{15pt}{0pt}{\drawFromCurrentPicture{
draw byNamedLine(AB); 
draw byNamedCircle(B);
draw byLabelLineEnd(A, B, 1);
draw byLabelLineEnd(B, A, 0);
}}
\inpost[post:III];\\
проведем \drawUnitLine{CA} и \drawUnitLine{BC} \inpost[post:I].\\
Тогда \drawLine[bottom][triangleABC]{AB,CA,BC} равносторонний.

Поскольку $\drawUnitLine{AB} = \drawUnitLine{CA}$ \indef[def:XV];\\
и $\drawUnitLine{AB} = \drawUnitLine{BC}$ \indef[def:XV];\\
$\therefore \drawUnitLine{CA} = \drawUnitLine{BC}$ \inax[post:I];\\
и значит \triangleABC\ и есть искомый треугольник.
\stopCenterAlign

\qed
\stopVerboseProposition

\startProposition[title={Предложение II. Задача}, reference=prop:I.II]
\defineNewPicture{
pair A, B, C, D, E, F;
path P[];
numeric r[];
A := (0, 0);
B := (-3/5u, -3/5u);
C := (-2u, -1/3u);
r1 := abs(A-B);
D := (fullcircle scaled 2r1 shifted A) intersectionpoint (fullcircle scaled 2r1 shifted B);
r2 := abs(B-C);
r3 := r1 + r2;
P1 := fullcircle scaled 2r2 shifted B;
P2 := fullcircle scaled 2r3 shifted D;
E := (D -- 10[D, B]) intersectionpoint P1;
F := (D -- 10[D, A]) intersectionpoint P2;
byLineDefine(A, B, black, 1, 0);
byLineDefine(B, C, black, 0, 0);
byLineDefine(B, D, byred, 1, 0);
byLineDefine(D, A, byred, 0, 0);
byLineDefine(B, E, byyellow, 0, 0);
byLineDefine(A, F, byblue, 0, 0);
draw byNamedLineSeq(0)(AF,DA,BD,BE);
draw byNamedLineSeq(0)(AB,BC);
draw byCircle(D, E, byred, 0, 0, 1/2)(A);
draw byCircle(B, C, byblue, 0, 0, -1/2)(B);
draw byLabelsOnPolygon(E, D, A, F)(2, -1);
draw byLabelsOnPolygon(E, B, C)(2, -1);
draw byLabelLineEnd(C, B, 0);
draw byLabelLineEnd(E, B, 1);
draw byLabelLineEnd(F, A, 1);
}
\drawCurrentPictureInMargin
\problemNP{О}{т}{данной точки \drawFromCurrentPicture{
startGlobalRotation(-lineAngle.DA);
draw byNamedLineSeq(0)(DA,AF);
draw byLabelPoint(A, 90, 1);
stopGlobalRotation;
} отложить прямую, равную данной прямой \drawUnitLine{BC}.}

\startCenterAlign
Проведем \drawUnitLine{AB} \inpost[post:I], построим \drawFromCurrentPicture[bottom]{
startAutoLabeling;
startTempScale(scaleFactor*3);
startGlobalRotation(180-lineAngle.AB);
draw byNamedLineSeq(0)(AB,BD,DA);
stopGlobalRotation;
stopTempScale;
stopAutoLabeling;
} \inprop[prop:I.I],\\
продлим \drawUnitLine{BD} \inpost[post:II],\\
опишем
\drawFromCurrentPicture{
draw byNamedLine (BC); 
draw byNamedCircle(B); 
draw byLabelLineEnd(B, C, 0); 
draw byLabelLineEnd(C, B, 0);
}
\inpost[post:III], и
\drawFromCurrentPicture{
draw byNamedLine (BD, BE);
draw byNamedCircle(A);
draw byLabelLineEnd(D, E, 0); 
draw byLabelLineEnd(E, D, 1);
}
\inpost[post:III];\\
продлим \drawUnitLine{DA} \inpost[post:II],\\
тогда искомая прямая это \drawUnitLine{AF}.

Поскольку $\drawUnitLine{BE,BD} = \drawUnitLine{DA,AF}$ \indef[def:XV],\\
и $\drawUnitLine{BD} = \drawUnitLine{DA}$ (постр.),\\
$\therefore \drawUnitLine{BE} = \drawUnitLine{AF}$ \inax[post:III],\\
но \indef[def:XV] $\drawUnitLine{BC} = \drawUnitLine{BE} = \drawUnitLine{AF}$;

$\therefore \drawUnitLine{AF}$ проведенная из данной точки (\drawUnitLine{DA,AF}) равна данной прямой \drawUnitLine{BC}.
\stopCenterAlign

\qed
\stopProposition

\startProposition[title={Предложение III. Задача}, reference=prop:I.III]
\defineNewPicture{
pair A, B, C, D, E, F;
path P;
numeric r;
A := (0, 0);
r := 7/4u;
B := A shifted (r, 0);
C := A shifted (4/3r, 0);
D := A shifted dir(30)*r;
E := A shifted (7/6r, -1/6r);
F := A shifted (7/6r, -7/6r);
byLineDefine(A, B, black, 0, 0);
byLineDefine(B, C, black, 1, 0);
byLineDefine(A, D, byred, 0, 0);
draw byNamedLineSeq(0)(BC,AB,AD);
draw byLine(E, F, byblue, 0, 0);
draw byCircle(A, D, byblue, 0, 0, 0)(A);
draw byLabelsOnPolygon(B, A, D)(2, -1);
draw byLabelLineEnd(D, A, 0);
draw byLabelLineEnd(C, A, 0);
draw byLabelPoint(B, angle(B-A) + 45, 2);
draw byLabelsOnPolygon(E, F)(0, 0);
}
\drawCurrentPictureInMargin
\problemNP{О}{т}{большей \drawUnitLine{AB,BC}  из двух данных прямых, отнять прямую, равную меньшей \drawUnitLine{EF}.}

\startCenterAlign
Проведем $\drawUnitLine{AD} = \drawUnitLine{EF}$ \inprop[prop:I.II];\\
опишем 
\drawFromCurrentPicture{
draw byNamedLine (AD); 
draw byNamedCircle(A);
draw byLabelLineEnd(D, A, 0);
draw byLabelLineEnd(A, D, 0);
} \inpost[post:III],\\
тогда $\drawUnitLine{EF} = \drawUnitLine{AB}$

Поскольку $\drawUnitLine{AD} = \drawUnitLine{AB}$ \indef[def:XV],\\
и $\drawUnitLine{EF} = \drawUnitLine{AD}$ (постр.);

$\therefore \drawUnitLine{EF} = \drawUnitLine{AB}$ \inax[ax:I];
\stopCenterAlign

\qed
\stopProposition

\startProposition[title={Предложение IV. Теорема}, reference=prop:I.IV]
\defineNewPicture{
pair A, B, C, D, E, F, d;
A := (0, 0);
B := A shifted (-5/2u, -7/2u);
C := A shifted (1/3u, -5/2u);
d := (0, -4u);
D := A shifted d;
E := B shifted d;
F := C shifted d;
draw byAngleWithName(B, A, C, byyellow, 0)(A);
draw byAngleWithName(A, B, C, byblue, 0)(B);
draw byAngleWithName(B, C, A, byred, 0)(C);
byLineDefine(A, B, byred, 0, 0);
byLineDefine(B, C, black, 0, 0);
byLineDefine(C, A, byblue, 0, 0);
draw byNamedLineSeq(0)(CA,BC,AB);
draw byAngleWithName(E, D, F, byyellow, 0)(D);
draw byAngleWithName(D, E, F, byblue, 0)(E);
draw byAngleWithName(E, F, D, byred, 0)(F);
byLineDefine(D, E, byred, 0, 1);
byLineDefine(E, F, black, 0, 1);
byLineDefine(F, D, byblue, 0, 1);
draw byNamedLineSeq(0)(FD,EF,DE);
draw byLabelsOnPolygon(F, E, D)(0, 0);
draw byLabelsOnPolygon(B, A, C)(0, -1);
}
\drawCurrentPictureInMargin
\problemNP{Е}{сли}{два треугольника имеют по две стороны, равные каждая каждой, ($\drawUnitLine{AB} = \drawUnitLine{DE}$ и $\drawUnitLine{CA} = \drawUnitLine{FD}$) и по равному углу ($\drawAngle{A} = \drawAngle{D}$) содержащемуся между равными прямыми, то они будут иметь и основание равное основанию ($\drawUnitLine{BC} = \drawUnitLine{EF}$), и один треугольник будет равен другому, и остальные углы, стягиваемые равными сторонами, будут равны каждый каждому ($\drawAngle{B} = \drawAngle{E}$ and $\drawAngle{C} = \drawAngle{F}$).}

Представим, что два треугольника расположены таким образом, что вершина одного из двух равных углов \drawAngle{A} или \drawAngle{D}, совпадает с вершиной другого, и \drawUnitLine{AB} совпадает \drawUnitLine{DE}, тогда \drawUnitLine{CA} при наложении совпадет с \drawUnitLine{FD}. Следовательно \drawUnitLine{BC} совпает с \drawUnitLine{EF}, или же две прямые будут содержать пространство, что невозможно \inax[ax:X], следовательно $\drawUnitLine{BC} = \drawUnitLine{EF}$, $\drawAngle{B} = \drawAngle{E}$ и $\drawAngle{C} = \drawAngle{F}$, и поскольку треугольники \drawLine{CA,BC,AB} и \drawLine{FD,EF,DE} совпадают при наложении, они равны во всех отношениях.

\qed
\stopProposition

\startProposition[title={Предложение V. Теорема}, reference=prop:I.V]
\defineNewPicture[11/40]{
pair A, B, C, D, E;
picture q;
A := (0, 0);
B := A shifted (u, -2u);
C := B xscaled -1;
D := 9/5[A,B];
E := 9/5[A,C];
draw byAngle(B, A, C, black, 0);
draw byAngle(A, B, C, byblue, 0);
draw byAngle(B, C, A, byblue, 0);
draw byAngle(C, B, E, byyellow, 0);
draw byAngle(D, C, B, byyellow, 0);
draw byAngle(B, D, C, byred, 0);
draw byAngle(C, E, B, byred, 0);
byAngleDefine(E, B, D, black, 1);
byAngleDefine(D, C, E, black, 1);
byLineDefine(B, D, byyellow, 0, 0);
byLineDefine(C, E, byyellow, 0, 0);
byLineDefine(B, E, byblue, 0, 0);
byLineDefine(C, D, byblue, 0, 0);
byLineDefine(A, B, byred, 0, 0);
byLineDefine(A, C, byred, 0, 0);
byLineDefine(B, C, black, 0, 0);
draw byNamedLineSeq(0)(CD,noLine,BC,noLine,BE,CE,AC,AB,BD);
draw byLabelsOnPolygon(E, C, A, B, D, C, B)(0, 0);
}
\drawCurrentPictureInMargin
\problemNP[2]{У}{любого}{равнобедренного треугольника \drawLine[bottom]{BC,AC,AB} углы при основании равны между собой и по продолжении равных сторон углы под основанием будут равны между собой.}

\startCenterAlign
Продлим \drawUnitLine{AB} и \drawUnitLine{AC} \inpost[post:II],\\
возьмем $\drawUnitLine{BD} = \drawUnitLine{CE}$ \inprop[prop:I.III];\\
проведем \drawUnitLine{BE} и \drawUnitLine{CD}.

Тогда в
\drawFromCurrentPicture{
startAutoLabeling;
draw byNamedAngle(BAC);
draw byNamedLineSeq(0)(BE,CE,AC,AB);
stopAutoLabeling;
}
и
\drawFromCurrentPicture{
startAutoLabeling;
draw byNamedAngle(BAC);
draw byNamedLineSeq(0)(BD,CD,AC,AB);
stopAutoLabeling;
}\\
получим $\drawUnitLine{AB,BD} = \drawUnitLine{AC,CE}$ (конст.),\\
\drawAngle{BAC} общий обоим,\\
и $\drawUnitLine{AB} = \drawUnitLine{AC}$ (гип.)\\
$\therefore \drawAngle{BCA,DCB} = \drawAngle{ABC,CBE}$, $\drawUnitLine{BE} = \drawUnitLine{CD}$ и $\drawAngle{CEB} = \drawAngle{BDC}$ \inprop[prop:I.IV].

Так же у \drawLine{BE,CE,BC} и \drawLine{BD,CD,BC}\\
получим $\drawUnitLine{BD} = \drawUnitLine{CE}$, $\drawAngle{CEB} = \drawAngle{BDC}$ и $\drawUnitLine{BE} = \drawUnitLine{CD}$,\\
$\therefore \drawAngle{DCE,DCB} = \drawAngle{EBD,CBE}$ и $\drawAngle{DCB} = \drawAngle{CBE}$ \inprop[prop:I.IV]\\
но $\drawAngle{BCA,DCB} = \drawAngle{ABC,CBE}$, $\therefore \drawAngle{BCA} = \drawAngle{ABC}$.
\stopCenterAlign

\qed
\stopProposition


\startProposition[title={Предложение VI. Теорема}, reference=prop:I.VI]
\defineNewPicture[1/4]{
pair A, B, C, D;
A := (0, 0);
B := A shifted (7/2u, 0);
D := A shifted (7/4u, 3u);
C := 2/3[A, D];
draw byAngleWithName(B, A, C, byyellow, 0)(A);
draw byAngleWithName(A, B, D, black, 0)(B);
byLineDefine(B, C, byyellow, 0, 0);
byLineDefine(A, B, byred, 0, 0);
byLineDefine(B, D, byblue, 0, 0);
byLineDefine(C, A, black, 0, 0);
byLineDefine(C, D, black, 1, 0);
draw byNamedLine(BC);
draw byNamedLineSeq(0)(CA,CD,BD,AB);
draw byLabelsOnPolygon(A, C, D, B)(0, 0);
}
\drawCurrentPictureInMargin
\problemNP{Е}{сли}{у любого треугольника \drawLine[bottom][triangleABD]{CA,CD,BD,AB} два угла \drawAngle{A} и \drawAngle{B} равны между собой, то и стороны \drawUnitLine{CA,CD} and \drawUnitLine{BD}, стягивающие равные углы, будут равны.}

Предположим, что стороны не равны и одна из них \drawUnitLine{CA,CD} больше чем другая \drawUnitLine{BD}, тогдо отрежем от нее $\drawUnitLine{CA} = \drawUnitLine{BD}$ \inprop[prop:I.III] и проведем \drawUnitLine{BC}.

\startCenterAlign
Тогда в \drawLine[bottom]{BC,AB,CA} и \triangleABD,\\
$\drawUnitLine{CA} = \drawUnitLine{BD}$ (постр.),\\
$\drawAngle{A} = \drawAngle{B}$ (гип.)\\
и \drawUnitLine{AB} общая обоим,\\
$\therefore$ эти треугольники равны \inprop[prop:I.IV]\\
часть равна целому, что не имеет смысла;\\
$\therefore$ ни одна из сторон \drawUnitLine{CA,CD} или \drawUnitLine{BD} не больше другой,\\
$\therefore$ они равны.
\stopCenterAlign

\qed
\stopProposition

\startProposition[title={Предложение VII. Теорема}, reference=prop:I.VII]
\defineNewPicture{
pair A, B, C, D, E, F, G, H;
A := (0, 0);
B := A shifted (4u, 0);
C := A shifted (u, 3u);
D := C shifted (7/4u, 0);
E := 1/2[C, D] yscaled -0.7;
F := E shifted (0, -2u);
G := 5/4[A, E];
H := 5/4[A, F];
draw byAngleWithName(B, C, A, black, 0)(C);
draw byAngle(D, C, B, byred, 0);
draw byAngleWithName(A, D, B, byyellow, 0)(D);
draw byAngle(C, D, A, byblue, 0);
draw byAngle(B, F, H, black, 0);
draw byAngle(B, F, E, byred, 0);
draw byAngle(B, E, G, byyellow, 0);
draw byAngle(G, E, F, byblue, 0);
draw byLine(C, D, black, 1, 0);
draw byLine(E, F, black, 1, 0);
draw byLine(A, B, black, 0, 0);
byLineDefine(B, C, byblue, 0, 0);
byLineDefine(C, A, byred, 0, 0);
byLineDefine(B, D, byblue, 0, 0);
byLineDefine(D, A, byred, 0, 0);
byLineDefine(B, E, byblue, 0, 0);
byLineDefine(E, A, byred, 0, 0);
byLineDefine(B, F, byblue, 0, 0);
byLineDefine(F, A, byred, 0, 0);
byLineDefine(E, G, byred, 1, 0);
byLineDefine(F, H, byred, 1, 0);
draw byNamedLine(EG,FH);
draw byNamedLineSeq(0)(BC,CA,EA,BE);
draw byNamedLineSeq(0)(BD,DA,FA,BF);
string pointLabel.F, pointLabel.E;
pointLabel.F := "C";
pointLabel.E := "D";
draw byLabelsOnPolygon(F, A, C, D, B, F, noPoint)(2, 0);
draw byLabelsOnPolygon(A, E, B)(2, 0);
draw byLabelsOnPolygon(H, F, A)(2, 0);
}
\drawCurrentPictureInMargin
\problemNP{П}{о}{одну сторону одной и той же прямой \drawUnitLine{AB} нельзя построить два разных треугольника с равными друг другу смежными сторонами $\drawUnitLine{CA} = \drawUnitLine{DA}$ и $\drawUnitLine{BC} = \drawUnitLine{BD}$.}

Если два треугольника построены на одном основании и по одну сторону от него, то вершина одного может находиться вовне другого, внутри или на одной из его сторон.

Если такое возможно, то построим два треугольника таких, что $\left\{\eqalign{\drawUnitLine{CA}&=\drawUnitLine{BC}\cr \drawUnitLine{DA}&=\drawUnitLine{BD}\cr}\right\}$, затем проведем \drawUnitLine{CD}, тогда

\startCenterAlign
$\drawAngle{C,DCB} = \drawAngle{CDA}$ \inprop[prop:I.V]

$\therefore\drawAngle{DCB} < \drawAngle{CDA}$ и

$\left.
\eqalign{
\therefore\drawAngle{DCB} &< \drawAngle{CDA,D}\cr
\mbox{но \inprop[prop:I.V]} \drawAngle{DCB} &= \drawAngle{CDA,D}
}\right\}\mbox{что невозможно,}$
\stopCenterAlign

\noindent следовательно, смежные стороны таких двух треугольников не могут быть равны.

\qed
\stopProposition

\stopbook
\stoptext
%\closeout \lettrineslist

\input preamble.tex
\input preamble_be.tex

\mainlanguage[ru]

\def\inpropstr{пр.}
\def\inpoststr{пост.}
\def\indefstr{опр.}
\def\inaxstr{акс.}
\def\qedstr{ч. т. д.}

\def\mpPre{textLabels := true;}

\starttext
\setuplayout[title]
\setupheader [state=stop]
~\vfill
~\hfill\symbol[cc][cc] \symbol[cc][by] \symbol[cc][sa]\hfill~
\vfill~
\pagebreak\ \pagebreak
\setupheader[state=start]
\setuplayout[reset]

\startbook[title={Книга I}]
\startVerboseProposition[title={Предложение I. Задача}, reference=prop:I.I]

\defineNewPicture[1/2]{
	pair A, B, C;
	path P[];
	numeric r;
	r := 3/2u;
	A := (0, 0);
	B := (r, 0);
	P1 := fullcircle scaled 2r;
	P2 := fullcircle scaled 2r shifted B;
	C := P1 intersectionpoint P2;
		byLineDefine(A, B, black, 0, 0);
		byLineDefine(B, C, byred, 0, 0);
		byLineDefine(C, A, byyellow, 0, 0);
		draw byNamedLineSeq(1)(AB,CA,BC);
		draw byCircle(A, B, byblue, 0, 0, 1/2)(A);
		draw byCircle(B, A, byred, 0, 0, 1/2)(B);
		draw byLabelsOnPolygon(A, C, B)(0, -1);
}
\drawCurrentPictureInMargin
\problemNP{Н}{а}{данной ограниченной прямой \drawUnitLine{AB} построить равносторонний треугольник.}

\startCenterAlign
Опишем \offsetPicture{15pt}{0pt}{\drawFromCurrentPicture{
draw byNamedLine(AB);
draw byNamedCircle(A);
draw byLabelLineEnd(A, B, 0);
draw byLabelLineEnd(B, A, 1);
}} и \offsetPicture{15pt}{0pt}{\drawFromCurrentPicture{
draw byNamedLine(AB); 
draw byNamedCircle(B);
draw byLabelLineEnd(A, B, 1);
draw byLabelLineEnd(B, A, 0);
}}
\inpost[post:III];\\
проведем \drawUnitLine{CA} и \drawUnitLine{BC} \inpost[post:I].\\
Тогда \drawLine[bottom][triangleABC]{AB,CA,BC} равносторонний.

Поскольку $\drawUnitLine{AB} = \drawUnitLine{CA}$ \indef[def:XV];\\
и $\drawUnitLine{AB} = \drawUnitLine{BC}$ \indef[def:XV];\\
$\therefore \drawUnitLine{CA} = \drawUnitLine{BC}$ \inax[post:I];\\
и значит \triangleABC\ и есть искомый треугольник.
\stopCenterAlign

\qed
\stopVerboseProposition

\startProposition[title={Предложение II. Задача}, reference=prop:I.II]
\defineNewPicture{
pair A, B, C, D, E, F;
path P[];
numeric r[];
A := (0, 0);
B := (-3/5u, -3/5u);
C := (-2u, -1/3u);
r1 := abs(A-B);
D := (fullcircle scaled 2r1 shifted A) intersectionpoint (fullcircle scaled 2r1 shifted B);
r2 := abs(B-C);
r3 := r1 + r2;
P1 := fullcircle scaled 2r2 shifted B;
P2 := fullcircle scaled 2r3 shifted D;
E := (D -- 10[D, B]) intersectionpoint P1;
F := (D -- 10[D, A]) intersectionpoint P2;
byLineDefine(A, B, black, 1, 0);
byLineDefine(B, C, black, 0, 0);
byLineDefine(B, D, byred, 1, 0);
byLineDefine(D, A, byred, 0, 0);
byLineDefine(B, E, byyellow, 0, 0);
byLineDefine(A, F, byblue, 0, 0);
draw byNamedLineSeq(0)(AF,DA,BD,BE);
draw byNamedLineSeq(0)(AB,BC);
draw byCircle(D, E, byred, 0, 0, 1/2)(A);
draw byCircle(B, C, byblue, 0, 0, -1/2)(B);
draw byLabelsOnPolygon(E, D, A, F)(2, -1);
draw byLabelsOnPolygon(E, B, C)(2, -1);
draw byLabelsOnCircle(C)(B);
draw byLabelsOnCircle(E, F)(A);
}
\drawCurrentPictureInMargin
\problemNP{О}{т}{данной точки \drawFromCurrentPicture{
startGlobalRotation(-lineAngle.DA);
draw byNamedLineSeq(0)(DA,AF);
draw byLabelPoint(A, 90, 1);
stopGlobalRotation;
} отложить прямую, равную данной прямой \drawUnitLine{BC}.}

\startCenterAlign
Проведем \drawUnitLine{AB} \inpost[post:I], построим \drawFromCurrentPicture[bottom]{
startAutoLabeling;
startTempScale(scaleFactor*3);
startGlobalRotation(180-lineAngle.AB);
draw byNamedLineSeq(0)(AB,BD,DA);
stopGlobalRotation;
stopTempScale;
stopAutoLabeling;
} \inprop[prop:I.I],\\
продлим \drawUnitLine{BD} \inpost[post:II],\\
опишем
\drawFromCurrentPicture{
draw byNamedLine (BC); 
draw byNamedCircle(B); 
draw byLabelLineEnd(B, C, 0); 
draw byLabelLineEnd(C, B, 0);
}
\inpost[post:III], и
\drawFromCurrentPicture{
draw byNamedLine (BD, BE);
draw byNamedCircle(A);
draw byLabelLineEnd(D, E, 0); 
draw byLabelLineEnd(E, D, 1);
}
\inpost[post:III];\\
продлим \drawUnitLine{DA} \inpost[post:II],\\
тогда искомая прямая это \drawUnitLine{AF}.

Поскольку $\drawUnitLine{BE,BD} = \drawUnitLine{DA,AF}$ \indef[def:XV],\\
и $\drawUnitLine{BD} = \drawUnitLine{DA}$ (постр.),\\
$\therefore \drawUnitLine{BE} = \drawUnitLine{AF}$ \inax[post:III],\\
но \indef[def:XV] $\drawUnitLine{BC} = \drawUnitLine{BE} = \drawUnitLine{AF}$;

$\therefore \drawUnitLine{AF}$ проведенная из данной точки (\drawUnitLine{DA,AF}) равна данной прямой \drawUnitLine{BC}.
\stopCenterAlign

\qed
\stopProposition

\startProposition[title={Предложение III. Задача}, reference=prop:I.III]
\defineNewPicture{
pair A, B, C, D, E, F;
path P;
numeric r;
A := (0, 0);
r := 7/4u;
B := A shifted (r, 0);
C := A shifted (4/3r, 0);
D := A shifted dir(30)*r;
E := A shifted (7/6r, -1/6r);
F := A shifted (7/6r, -7/6r);
byLineDefine(A, B, black, 0, 0);
byLineDefine(B, C, black, 1, 0);
byLineDefine(A, D, byred, 0, 0);
draw byNamedLineSeq(0)(BC,AB,AD);
draw byLine(E, F, byblue, 0, 0);
draw byCircle(A, D, byblue, 0, 0, 0)(A);
draw byLabelsOnPolygon(B, A, D)(2, -1);
draw byLabelLineEnd(D, A, 0);
draw byLabelLineEnd(C, A, 0);
draw byLabelPoint(B, angle(B-A) + 45, 2);
draw byLabelsOnPolygon(E, F)(0, 0);
}
\drawCurrentPictureInMargin
\problemNP{О}{т}{большей \drawUnitLine{AB,BC}  из двух данных прямых, отнять прямую, равную меньшей \drawUnitLine{EF}.}

\startCenterAlign
Проведем $\drawUnitLine{AD} = \drawUnitLine{EF}$ \inprop[prop:I.II];\\
опишем 
\drawFromCurrentPicture{
draw byNamedLine (AD); 
draw byNamedCircle(A);
draw byLabelLineEnd(D, A, 0);
draw byLabelLineEnd(A, D, 0);
} \inpost[post:III],\\
тогда $\drawUnitLine{EF} = \drawUnitLine{AB}$

Поскольку $\drawUnitLine{AD} = \drawUnitLine{AB}$ \indef[def:XV],\\
и $\drawUnitLine{EF} = \drawUnitLine{AD}$ (постр.);

$\therefore \drawUnitLine{EF} = \drawUnitLine{AB}$ \inax[ax:I];
\stopCenterAlign

\qed
\stopProposition

\startProposition[title={Предложение IV. Теорема}, reference=prop:I.IV]
\defineNewPicture{
pair A, B, C, D, E, F, d;
A := (0, 0);
B := A shifted (-5/2u, -7/2u);
C := A shifted (1/3u, -5/2u);
d := (0, -4u);
D := A shifted d;
E := B shifted d;
F := C shifted d;
draw byAngleWithName(B, A, C, byyellow, 0)(A);
draw byAngleWithName(A, B, C, byblue, 0)(B);
draw byAngleWithName(B, C, A, byred, 0)(C);
byLineDefine(A, B, byred, 0, 0);
byLineDefine(B, C, black, 0, 0);
byLineDefine(C, A, byblue, 0, 0);
draw byNamedLineSeq(0)(CA,BC,AB);
draw byAngleWithName(E, D, F, byyellow, 0)(D);
draw byAngleWithName(D, E, F, byblue, 0)(E);
draw byAngleWithName(E, F, D, byred, 0)(F);
byLineDefine(D, E, byred, 0, 1);
byLineDefine(E, F, black, 0, 1);
byLineDefine(F, D, byblue, 0, 1);
draw byNamedLineSeq(0)(FD,EF,DE);
draw byLabelsOnPolygon(F, E, D)(0, 0);
draw byLabelsOnPolygon(B, A, C)(0, -1);
}
\drawCurrentPictureInMargin
\problemNP{Е}{сли}{два треугольника имеют по две стороны, равные каждая каждой, ($\drawUnitLine{AB} = \drawUnitLine{DE}$ и $\drawUnitLine{CA} = \drawUnitLine{FD}$) и по равному углу ($\drawAngle{A} = \drawAngle{D}$) содержащемуся между равными прямыми, то они будут иметь и основание равное основанию ($\drawUnitLine{BC} = \drawUnitLine{EF}$), и один треугольник будет равен другому, и остальные углы, стягиваемые равными сторонами, будут равны каждый каждому ($\drawAngle{B} = \drawAngle{E}$ and $\drawAngle{C} = \drawAngle{F}$).}

Представим, что два треугольника расположены таким образом, что вершина одного из двух равных углов \drawAngle{A} или \drawAngle{D}, совпадает с вершиной другого, и \drawUnitLine{AB} совпадает \drawUnitLine{DE}, тогда \drawUnitLine{CA} при наложении совпадет с \drawUnitLine{FD}. Следовательно \drawUnitLine{BC} совпает с \drawUnitLine{EF}, или же две прямые будут содержать пространство, что невозможно \inax[ax:X], следовательно $\drawUnitLine{BC} = \drawUnitLine{EF}$, $\drawAngle{B} = \drawAngle{E}$ и $\drawAngle{C} = \drawAngle{F}$, и поскольку треугольники \drawLine{CA,BC,AB} и \drawLine{FD,EF,DE} совпадают при наложении, они равны во всех отношениях.

\qed
\stopProposition

\startProposition[title={Предложение V. Теорема}, reference=prop:I.V]
\defineNewPicture[11/40]{
pair A, B, C, D, E;
picture q;
A := (0, 0);
B := A shifted (u, -2u);
C := B xscaled -1;
D := 9/5[A,B];
E := 9/5[A,C];
draw byAngle(B, A, C, black, 0);
draw byAngle(A, B, C, byblue, 0);
draw byAngle(B, C, A, byblue, 0);
draw byAngle(C, B, E, byyellow, 0);
draw byAngle(D, C, B, byyellow, 0);
draw byAngle(B, D, C, byred, 0);
draw byAngle(C, E, B, byred, 0);
byAngleDefine(E, B, D, black, 1);
byAngleDefine(D, C, E, black, 1);
byLineDefine(B, D, byyellow, 0, 0);
byLineDefine(C, E, byyellow, 0, 0);
byLineDefine(B, E, byblue, 0, 0);
byLineDefine(C, D, byblue, 0, 0);
byLineDefine(A, B, byred, 0, 0);
byLineDefine(A, C, byred, 0, 0);
byLineDefine(B, C, black, 0, 0);
draw byNamedLineSeq(0)(CD,noLine,BC,noLine,BE,CE,AC,AB,BD);
draw byLabelsOnPolygon(E, C, A, B, D, C, B)(0, 0);
}
\drawCurrentPictureInMargin
\problemNP[2]{У}{любого}{равнобедренного треугольника \drawLine[bottom]{BC,AC,AB} углы при основании равны между собой и по продолжении равных сторон углы под основанием будут равны между собой.}

\startCenterAlign
Продлим \drawUnitLine{AB} и \drawUnitLine{AC} \inpost[post:II],\\
возьмем $\drawUnitLine{BD} = \drawUnitLine{CE}$ \inprop[prop:I.III];\\
проведем \drawUnitLine{BE} и \drawUnitLine{CD}.

Тогда в
\drawFromCurrentPicture{
startAutoLabeling;
draw byNamedAngle(BAC);
draw byNamedLineSeq(0)(BE,CE,AC,AB);
stopAutoLabeling;
}
и
\drawFromCurrentPicture{
startAutoLabeling;
draw byNamedAngle(BAC);
draw byNamedLineSeq(0)(BD,CD,AC,AB);
stopAutoLabeling;
}\\
получим $\drawUnitLine{AB,BD} = \drawUnitLine{AC,CE}$ (конст.),\\
\drawAngle{BAC} общий обоим,\\
и $\drawUnitLine{AB} = \drawUnitLine{AC}$ (гип.)\\
$\therefore \drawAngle{BCA,DCB} = \drawAngle{ABC,CBE}$, $\drawUnitLine{BE} = \drawUnitLine{CD}$ и $\drawAngle{CEB} = \drawAngle{BDC}$ \inprop[prop:I.IV].

Так же у \drawLine{BE,CE,BC} и \drawLine{BD,CD,BC}\\
получим $\drawUnitLine{BD} = \drawUnitLine{CE}$, $\drawAngle{CEB} = \drawAngle{BDC}$ и $\drawUnitLine{BE} = \drawUnitLine{CD}$,\\
$\therefore \drawAngle{DCE,DCB} = \drawAngle{EBD,CBE}$ и $\drawAngle{DCB} = \drawAngle{CBE}$ \inprop[prop:I.IV]\\
но $\drawAngle{BCA,DCB} = \drawAngle{ABC,CBE}$, $\therefore \drawAngle{BCA} = \drawAngle{ABC}$.
\stopCenterAlign

\qed
\stopProposition


\startProposition[title={Предложение VI. Теорема}, reference=prop:I.VI]
\defineNewPicture[1/4]{
pair A, B, C, D;
A := (0, 0);
B := A shifted (7/2u, 0);
D := A shifted (7/4u, 3u);
C := 2/3[A, D];
draw byAngleWithName(B, A, D, byyellow, 0)(A);
draw byAngleWithName(A, B, D, black, 0)(B);
byLineDefine(B, C, byyellow, 0, 0);
byLineDefine(A, B, byred, 0, 0);
byLineDefine(B, D, byblue, 0, 0);
byLineDefine(C, A, black, 0, 0);
byLineDefine(C, D, black, 1, 0);
draw byNamedLine(BC);
draw byNamedLineSeq(0)(CA,CD,BD,AB);
draw byLabelsOnPolygon(A, C, D, B)(0, 0);
}
\drawCurrentPictureInMargin
\problemNP{Е}{сли}{у любого треугольника \drawLine[bottom][triangleABD]{CA,CD,BD,AB} два угла \drawAngle{A} и \drawAngle{B} равны между собой, то и стороны \drawUnitLine{CA,CD} и \drawUnitLine{BD}, стягивающие равные углы, будут равны.}

Предположим, что стороны не равны и одна из них \drawUnitLine{CA,CD} больше чем другая \drawUnitLine{BD}, тогда отрежем от нее $\drawUnitLine{CA} = \drawUnitLine{BD}$ \inprop[prop:I.III] и проведем \drawUnitLine{BC}.

\startCenterAlign
Тогда в \drawLine[bottom]{BC,AB,CA} и \triangleABD,\\
$\drawUnitLine{CA} = \drawUnitLine{BD}$ (постр.),\\
$\drawAngle{A} = \drawAngle{B}$ (гип.)\\
и \drawUnitLine{AB} общая обоим,\\
$\therefore$ эти треугольники равны \inprop[prop:I.IV]\\
часть равна целому, что невозможно;\\
$\therefore$ ни одна из сторон \drawUnitLine{CA,CD} или \drawUnitLine{BD} не больше другой,\\
$\therefore$ они равны.
\stopCenterAlign

\qed
\stopProposition

\startProposition[title={Предложение VII. Теорема}, reference=prop:I.VII]
\defineNewPicture{
pair A, B, C, D, E, F, G, H;
A := (0, 0);
B := A shifted (4u, 0);
C := A shifted (u, 3u);
D := C shifted (7/4u, 0);
E := 1/2[C, D] yscaled -0.7;
F := E shifted (0, -2u);
G := 5/4[A, E];
H := 5/4[A, F];
draw byAngleWithName(B, C, A, black, 0)(C);
draw byAngle(D, C, B, byred, 0);
draw byAngleWithName(A, D, B, byyellow, 0)(D);
draw byAngle(C, D, A, byblue, 0);
draw byAngle(B, F, H, black, 0);
draw byAngle(B, F, E, byred, 0);
draw byAngle(B, E, G, byyellow, 0);
draw byAngle(G, E, F, byblue, 0);
draw byLine(C, D, black, 1, 0);
draw byLine(E, F, black, 1, 0);
draw byLine(A, B, black, 0, 0);
byLineDefine(B, C, byblue, 0, 0);
byLineDefine(C, A, byred, 0, 0);
byLineDefine(B, D, byblue, 0, 0);
byLineDefine(D, A, byred, 0, 0);
byLineDefine(B, E, byblue, 0, 0);
byLineDefine(E, A, byred, 0, 0);
byLineDefine(B, F, byblue, 0, 0);
byLineDefine(F, A, byred, 0, 0);
byLineDefine(E, G, byred, 1, 0);
byLineDefine(F, H, byred, 1, 0);
draw byNamedLine(EG,FH);
draw byNamedLineSeq(0)(BC,CA,EA,BE);
draw byNamedLineSeq(0)(BD,DA,FA,BF);
byPointLabelDefine(F, "C");
byPointLabelDefine(E, "D");
draw byLabelsOnPolygon(F, A, C, D, B, F, noPoint)(2, 0);
draw byLabelsOnPolygon(A, E, B)(2, 0);
draw byLabelsOnPolygon(H, F, A)(2, 0);
}
\drawCurrentPictureInMargin
\problemNP{П}{о}{одну сторону одной и той же прямой \drawUnitLine{AB} нельзя построить два разных треугольника с равными друг другу смежными сторонами $\drawUnitLine{CA} = \drawUnitLine{DA}$ и $\drawUnitLine{BC} = \drawUnitLine{BD}$.}

Если два треугольника построены на одном основании и по одну сторону от него, то вершина одного может находиться вовне другого, внутри или на одной из его сторон.

Если такое возможно, то построим два треугольника таких, что $\left\{\eqalign{\drawUnitLine{CA}&=\drawUnitLine{BC}\cr \drawUnitLine{DA}&=\drawUnitLine{BD}\cr}\right\}$, затем проведем \drawUnitLine{CD}, тогда

\startCenterAlign
$\drawAngle{C,DCB} = \drawAngle{CDA}$ \inprop[prop:I.V]

$\therefore\drawAngle{DCB} < \drawAngle{CDA}$ и

$\left.
\eqalign{
\therefore\drawAngle{DCB} &< \drawAngle{CDA,D}\cr
\mbox{но \inprop[prop:I.V]} \drawAngle{DCB} &= \drawAngle{CDA,D}
}\right\}\mbox{что невозможно,}$
\stopCenterAlign

\noindent следовательно, смежные стороны таких двух треугольников не могут быть равны.

\qed
\stopProposition

\startProposition[title={Предложение VIII. Теорема}, reference=prop:I.VIII]
\defineNewPicture{
pair A, B, C, D, E, F, d;
A := (0, 0);
B := A shifted (-u, -4u);
C := A shifted (3/2u, -3u);
d := (0, -9/2u);
D := A shifted d;
E := B shifted d;
F := C shifted d;
draw byAngleWithName(F, D, E, black, 0)(D);
draw byAngleWithName(C, A, B, black, 0)(A);
byLineDefine(A, B, byred, 0, 0);
byLineDefine(B, C, black, 0, 0);
byLineDefine(C, A, byblue, 0, 0);
byLineDefine(D, E, byred, 0, 1);
byLineDefine(E, F, black, 0, 1);
byLineDefine(F, D, byblue, 0, 1);
draw byNamedLineSeq(0)(CA,BC,AB);
draw byNamedLineSeq(0)(FD,EF,DE);
draw byLabelsOnPolygon(C, B, A)(0, 0);
draw byLabelsOnPolygon(F, E, D)(0, 0);
}
\drawCurrentPictureInMargin
\problemNP{Е}{сли}{у двух треугольников по две попарно равных стороны ($\drawUnitLine{CA} = \drawUnitLine{FD}$ и $\drawUnitLine{AB} = \drawUnitLine{DE}$), а также равные основания ($\drawUnitLine{BC} = \drawUnitLine{EF}$), то углы
\drawFromCurrentPicture{
startAutoLabeling;
startGlobalRotation(-angleDirection.A);
draw byNamedAngle(A);
draw byNamedAngleDummySides(A);
stopGlobalRotation;
stopAutoLabeling;
} и
\drawFromCurrentPicture{
startAutoLabeling;
startGlobalRotation(-angleDirection.D);
draw byNamedAngle(D);
draw byNamedAngleDummySides(D);
stopGlobalRotation;
stopAutoLabeling;
}, заключенные между равными сторонами, равны.}

Если совместить равные основания \drawUnitLine{BC} и \drawUnitLine{EF} так, чтобы треугольники находились по одну сторону, а их равные стороны \drawUnitLine{AB} и \drawUnitLine{DE}, \drawUnitLine{CA} и \drawUnitLine{FD} были смежными, вершина одного будет совпадать с вершиной другого \inprop[prop:I.VII].

Следовательно, стороны \drawUnitLine{AB} и \drawUnitLine{CA}, будут совпадать с \drawUnitLine{DE} и \drawUnitLine{FD}, $\therefore \drawAngle{A} = \drawAngle{D}$.

\qed
\stopProposition

\startProposition[title={Предложение IX. Задача}, reference=prop:I.IX]
\defineNewPicture{
pair A, B, C, D, E, F;
A := (0, 2u);
B := (-4/3u, 0);
C := B xscaled -1;
D := A yscaled -1;
E := 5/4[A, B];
F := 5/4[A, C];
draw byAngle(B, A, D, byblue, 0);
draw byAngle(C, A, D, byyellow, 0);
byLineDefine(B, C, byyellow, 0, 0);
byLineDefine(A, D, black, 0, 0);
byLineDefine(D, B, byblue, 0, 0);
byLineDefine(C, D, byblue, 0, 0);
byLineDefine(A, B, byred, 0, 0);
byLineDefine(C, A, byred, 0, 0);
byLineDefine(B, E, byred, 1, 0);
byLineDefine(C, F, byred, 1, 0);
draw byNamedLine(BC,AD);
draw byNamedLineSeq(0)(DB,CD);
draw byNamedLineSeq(0)(BE,AB,CA,CF);
draw byLabelsOnPolygon(D, B, A, C)(0, 0);
}
\drawCurrentPictureInMargin
\problemNP{Р}{ассечь}{данный прямолинейный угол \drawAngle{BAD,CAD} пополам.}

\startCenterAlign
Возьмем $\drawUnitLine{AB} = \drawUnitLine{CA}$ \inprop[prop:I.III]

проведем \drawUnitLine{BC}, на которой построим \drawLine{CD,DB,BC} \inprop[prop:I.I],\\
проведем \drawUnitLine{AD}.

Поскольку $\drawUnitLine{AB} = \drawUnitLine{CA}$ (постр.),\\
\drawUnitLine{AD} общая обоим треугольникам\\
и $\drawUnitLine{CD} = \drawUnitLine{DB}$ (постр.),

$\therefore \drawAngle{BAD} = \drawAngle{CAD}$ \inprop[prop:I.VIII].
\stopCenterAlign

\qed
\stopProposition

\startProposition[title={Предложение X. Задача}, reference=prop:I.X]
\defineNewPicture{
pair A, B, C, D;
A := (0, 3u);
B := (-7/4u, 0);
C := B xscaled -1;
D := 1/2[B, C];
draw byAngle(B, A, D, byblue, 0);
draw byAngle(C, A, D, byyellow, 0);
draw byLine(A, D, byred, 0, 0);
byLineDefine(D, B, black, 0, 0);
byLineDefine(C, D, black, 1, 0);
byLineDefine(A, B, byyellow, 0, 0);
byLineDefine(C, A, byblue, 0, 0);
draw byNamedLineSeq(0)(AB,CA,CD,DB);
draw byLabelsOnPolygon(B, A, C, D)(0, 0);
}
\drawCurrentPictureInMargin
\problemNP{Р}{ассечь}{данную ограниченную прямую линию \drawUnitLine{DB,CD}.}

\startCenterAlign
Построим \drawLine[bottom]{AB,CA,CD,DB} \inprop[prop:I.I],\\
проведем \drawUnitLine{AD}, делая $\drawAngle{BAD} = \drawAngle{CAD}$ \inprop[prop:I.IX],

Тогда $\drawUnitLine{DB} = \drawUnitLine{CD}$ \inprop[prop:I.IV],

поскольку $\drawUnitLine{AB} = \drawUnitLine{CA}$ (постр.)\\
$\drawAngle{BAD} = \drawAngle{CAD}$\\
и \drawUnitLine{AD} общая обоим треугольникам.

Следовательно, данная линия рассечена пополам.
\stopCenterAlign

\qed
\stopProposition

\startProposition[title={Предложение XI. Задача}, reference=prop:I.XI]
\defineNewPicture[1/4]{
pair A, B, C, D, E, F;
A := (0, 3u);
B := (-7/4u, 0);
C := B xscaled -1;
D := 1/2[B, C];
E := 3/2[D, B];
F := 3/2[D, C];
draw byAngle(A, D, B, byred, 0);
draw byAngle(C, D, A, byblue, 0);
draw byLine(A, D, byyellow, 0, 0);
byLineDefine(A, B, byblue, 0, 0);
byLineDefine(C, A, byblue, 0, 0);
draw byNamedLineSeq(0)(AB,CA);
draw byLine(D, B, black, 0, 0);
draw byLine(B, E, black, 1, 0);
draw byLine(C, D, byred, 0, 0);
draw byLine(F, C, byred, 1, 0);
draw byLabelsOnPolygon(F, C, D, B, E)(2, 0);
draw byLabelsOnPolygon(B, A, C)(2, 0);
}
\drawCurrentPictureInMargin
\problemNP{И}{з}{данной точки 
\drawFromCurrentPicture{
draw byNamedLineSeq(0)(DB,CD);
draw byLabelPoint(D, 90, 1);
}
на данной прямой \drawUnitLine{DB,CD} построить перпендикуляр.}

\startCenterAlign
Возьмем любую точку 
\drawFromCurrentPicture{
draw byNamedLineSeq(0)(CD,FC);
draw byLabelPoint(C, 90, 1);
} на данной прямой,\\
отсечем $\drawUnitLine{DB} = \drawUnitLine{CD}$ \inprop[prop:I.III],\\
построим \drawLine[bottom]{AB,CA,CD,DB} \inprop[prop:I.I],\\
проведем \drawUnitLine{AD} и она будет перпендикуляром к данной прямой.

Поскольку $\drawUnitLine{AB} = \drawUnitLine{CA}$ (постр.)\\
$\drawUnitLine{CD} = \drawUnitLine{DB}$ (пост.)\\
и \drawUnitLine{AD} общая обоим треугольникам.

$\therefore \drawAngle{ADB} = \drawAngle{CDA}$ \inprop[prop:I.VIII]

$\therefore \drawUnitLine{AD} \perp \drawUnitLine{DB,CD}$ \indef[def:X]
\stopCenterAlign

\qed
\stopProposition


\startProposition[title={Предложение XII. Задача}, reference=prop:I.XII]
\defineNewPicture{
pair A, B, C, D, E, F;
path c;
numeric r, a[];
A := (0, 3u);
B := (-7/4u, 0);
C := B xscaled -1;
D := 1/2[B, C];
E := 4/3[D, B];
F := 4/3[D, C];
r := abs(A-B);
c := fullcircle scaled 2r shifted A;
a1 := xpart(c intersectiontimes (F--1/2[B, C]));
a2 := xpart(c intersectiontimes (E--1/2[B, C]));
draw byAngle(A, D, B, byyellow, 0);
draw byAngle(C, D, A, byblue, 0);
draw byLine(A, D, byred, 0, 0);
byLineDefine(A, B, byblue, 0, 0);
byLineDefine(C, A, byblue, 0, 0);
draw byNamedLineSeq(0)(AB,CA);
draw byArc(A, B, C)(r, byred, 0, 0, 0, 0)(O);
draw byArcBE(A, a2-1/4, a2, r, byred, 1, 0, 0, 0)(Ol);
draw byArcBE(A, a1, a1+1/4, r, byred, 1, 0, 0, 0)(Or);
draw byLine(D, B, black, 0, 0);
draw byLine(B, E, black, 1, 0);
draw byLine(C, D, byyellow, 0, 0);
draw byLine(F, C, byyellow, 1, 0);
draw byLabelsOnPolygon(B, A, C)(2, 0);
draw byLabelLineEnd(B, A, 0);
draw byLabelLineEnd(D, A, 0);
draw byLabelLineEnd(C, A, 0);
}
\drawCurrentPictureInMargin
\problemNP{П}{ровести}{перпендикуляр к данной неограниченной прямой \drawUnitLine{DB,CD} из данной, не находящейся на ней точки \drawFromCurrentPicture[middle][pointA]{
startTempScale(scaleFactor/2);
draw byNamedLine(AD);
draw byNamedLineSeq(0)(AB,CA);
draw byLabelsOnPolygon(B, A, C)(2, 0);
stopTempScale;
}.}

\startCenterAlign
Взяв данную точку \pointA\ в качестве центра по одну сторону прямой, и любое расстояние, позволяющее достигнуть другой стороны, построим \drawArc{O}.

Возьмем $\drawUnitLine{DB} = \drawUnitLine{CD}$ \inprop[prop:I.X],\\
проведем \drawUnitLine{AB}, \drawUnitLine{CA} и \drawUnitLine{AD}.

Тогда $\drawUnitLine{AD} \perp \drawUnitLine{DB,CD}$.

Поскольку \inprop[prop:I.VIII], раз $\drawUnitLine{DB} = \drawUnitLine{CD}$ (постр.),\\
\drawUnitLine{AD} общая обоим треугольникам,\\
и $\drawUnitLine{AB} = \drawUnitLine{CA}$ \indef[def:XV],

$\therefore \drawAngle{ADB} = \drawAngle{CDA}$,

и $\therefore \drawUnitLine{AD} \perp \drawUnitLine{DB,CD}$ \indef[def:X]
\stopCenterAlign

\qed
\stopProposition

\startProposition[title={Предложение XIII. Теорема}, reference=prop:I.XIII]
\defineNewPicture{
pair A, B, C, D, E;
A := (0, 5/2u);
B := (-7/4u, 0);
C := B xscaled -1;
D := (xpart(A), ypart(B));
E := (2/3xpart(C), 2/3ypart(A));
draw byAngle(A, D, B, byyellow, 0);
draw byAngle(E, D, A, byred, 0);
draw byAngle(C, D, E, byblue, 0);
draw byLine(A, D, black, 0, 0);
draw byLine(E, D, byyellow, 0, 0);
draw byLine(B, C, byred, 0, 0);
draw byLabelsOnPolygon(C, D, B, noPoint)(0, 0);
draw byLabelLineEnd(E, D, 0);
draw byLabelLineEnd(A, D, 0);
}
\drawCurrentPictureInMargin
\problemNP{Е}{сли}{прямая линия \drawUnitLine{ED} восставленная на другой прямой линии \drawUnitLine{BC} образует с ней углы, то это будут либо два прямых угла, либо их сумма будет равна двум прямым углам.}

\startCenterAlign
Если \drawUnitLine{ED} $\perp$ к \drawUnitLine{BC} тогда,\\
\drawAngle{ADB,EDA} и $\drawAngle{CDE} = \drawTwoRightAngles$ \indef[def:X],

но если \drawUnitLine{ED} будет не $\perp$ к \drawUnitLine{BC},\\
проведем $\drawUnitLine{AD} \perp \drawUnitLine{BC}$; \inprop[prop:I.XI]\\
$\drawAngle{ADB} +\drawAngle{CDE,EDA} = \drawTwoRightAngles$ (постр.),\\
$\drawAngle{ADB} = \drawAngle{CDE,EDA} = \drawAngle{EDA} + \drawAngle{CDE}$

$\therefore \drawAngle{ADB} + \drawAngle{CDE,EDA} = \drawAngle{ADB} + \drawAngle{EDA} + \drawAngle{CDE}$ \inax[ax:II]

$= \drawAngle{ADB,EDA} + \drawAngle{CDE} = \drawTwoRightAngles$.
\stopCenterAlign

\qed
\stopProposition

\startProposition[title={Предложение XIV. Теорема}, reference=prop:I.XIV]
\defineNewPicture[1/4]{
pair A, B, C, D, E;
A := (u, 5/2u);
B := (-7/4u, 0);
C := B xscaled -1;
D := (0, 0);
E := (xpart(C), -1/2ypart(A));
draw byAngle(B, D, A, byyellow, 0);
draw byAngle(C, D, A, byblue, 0);
draw byAngle(E, D, C, byred, 0);
draw byLine(A, D, byred, 0, 0);
draw byLine(E, D, byyellow, 0, 0);
draw byLine(B, D, byblue, 0, 0);
draw byLine(C, D, black, 0, 0);
draw byLabelsOnPolygon(E, D, B, noPoint)(0, 0);
draw byLabelLineEnd(A, D, 0);
}
\drawCurrentPictureInMargin
\problemNP{Е}{сли}{две прямые \drawUnitLine{BD} и \drawUnitLine{CD} образуют с третьей \drawUnitLine{AD} смежные углы, находясь по разные стороны от нее, и эти углы \drawAngle{BDA} и \drawAngle{CDA} равны двум прямым углам, то эти прямые будут лежать на одной прямой.}

\startCenterAlign
Действительно, пусть \drawUnitLine{ED}, а не \drawUnitLine{CD}, будет продолжением \drawUnitLine{BD},\\
тогда $\drawAngle{BDA} + \drawAngle{CDA,EDC} = \drawTwoRightAngles$

но, согласно гипотезе $\drawAngle{BDA} + \drawAngle{CDA} = \drawTwoRightAngles$

$\therefore\drawAngle{CDA,EDC} = \drawAngle{CDA}$, \inax[ax:III];\\ 
что не имеет смысла \inax[ax:IX].

$\therefore \drawUnitLine{ED}$ не является продолжением \drawUnitLine{BD}, и то же можно показать для любой другой прямой линии, за исключением \drawUnitLine{CD}, $\therefore \drawUnitLine{CD}$ является продложением \drawUnitLine{BD}.
\stopCenterAlign

\qed
\stopProposition

\startProposition[title={Предложение XV. Теорема}, reference=prop:I.XV]
\defineNewPicture{
pair A, B, C, D, E;
A := (7/4u, 3/2u);
B := A scaled -1;
C := A xscaled -1;
D := C scaled -1;
E := (A--B) intersectionpoint (C--D);
draw byAngle(B, E, C, byyellow, 0);
draw byAngle(C, E, A, byred, 0);
draw byAngle(A, E, D, black, 0);
draw byAngle(D, E, B, byblue, 0);
draw byLine(A, B, byred, 0, 0);
draw byLine(C, D, black, 0, 0);
draw byLabelsOnPolygon(C, E, A, noPoint)(0, 0);
draw byLabelPoint(B, lineAngle.AB + 90, 1);
draw byLabelPoint(D, lineAngle.CD - 90, 1);
}
\drawCurrentPictureInMargin
\problemNP[2]{Е}{сли}{две прямых линии \drawUnitLine{AB} и \drawUnitLine{CD} пересекаются, вертикальные углы \drawAngle{BEC} и \drawAngle{AED}, \drawAngle{CEA} и \drawAngle{DEB} будут равны между собой.}

\startCenterAlign
$\drawAngle{BEC} + \drawAngle{CEA} = \drawTwoRightAngles$

$\drawAngle{AED} + \drawAngle{CEA} = \drawTwoRightAngles$

$\therefore \drawAngle{BEC} = \drawAngle{AED}$.

Таким же образом можно показать, что\\
$\drawAngle{CEA} = \drawAngle{DEB}$
\stopCenterAlign

\qed
\stopProposition

\startProposition[title={Предложение XVI. Теорема}, reference=prop:I.XVI]
\defineNewPicture[1/4]{
pair A, B, C, D, E, F, G;
A := (0, 0);
B := A shifted (3/2u, 7/2u);
C := A shifted (3u, 0);
D := B shifted (3u, 0);
E = whatever[A, D] = whatever[B, C];
F := (xpart(D), ypart(A));
G := 4/3[B, C];
draw byAngleWithName(B, A, C, byblue, 0)(A);
draw byAngleWithName(C, B, A, black, 0)(B);
draw byAngle(A, E, B, byyellow, 0);
draw byAngle(D, E, C, byyellow, 0);
draw byAngle(E, C, D, black, 0);
draw byAngle(G, C, A, byred, 0);
draw byAngle(D, C, F, black, 1);
byLineDefine(C, F, black, 1, 0);
byLineDefine(C, G, black, 0, 0);
byLineDefine(B, E, byblue, 0, 0);
byLineDefine(E, C, byblue, 1, 0);
byLineDefine(A, E, byred, 0, 0);
byLineDefine(E, D, byred, 1, 0);
byLineDefine(A, B, byyellow, 1, 0);
byLineDefine(A, C, black, 0, 0);
byLineDefine(C, D, byyellow, 0, 0);
draw byNamedLineSeq(0)(AE,ED,CD);
draw byNamedLineSeq(0)(EC,CG,noLine,CF,AC,AB,BE);
draw byLabelsOnPolygon(F, A, B, E, D, C)(2, 0);
draw byLabelsOnPolygon(F, C, G, noPoint)(0, 0);
}
\drawCurrentPictureInMargin
\problemNP[2]{П}{ри}{продолжении стороны треугольника \drawLine[bottom]{BE,EC,AC,AB} внешний угол \drawFromCurrentPicture[bottom][anglesECDpDCF]{
startAutoLabeling;
draw byNamedAngleSides(ECD,DCF)(CF);
stopAutoLabeling;
} будет больше любого из противолежащих ему внутренних углов \drawAngle{B} or \drawAngle{A}.
}

\startCenterAlign
Сделаем $\drawUnitLine{BE} = \drawUnitLine{EC}$ \inprop[prop:I.X];\\
проведем \drawUnitLine{AE} и продлим до $\drawUnitLine{ED} = \drawUnitLine{AE}$;\\
проведем \drawUnitLine{CD}. В \drawLine{BE,AE,AB} и \drawLine{EC,ED,CD};\\
$\drawUnitLine{BE} = \drawUnitLine{EC}$, $\drawAngle{AEB} = \drawAngle{DEC}$ \inprop[prop:I.XV] и $\drawUnitLine{AE} = \drawUnitLine{ED}$ (постр.),\\
$\therefore \drawAngle{B} = \drawAngle{ECD}$ \inprop[prop:I.IV],\\
$\therefore \anglesECDpDCF\ > \drawAngle{ECD}$.

Так же можно показать, что при продлении \drawUnitLine{AC,CF}, $\drawAngle{GCA} > \drawAngle{A}$ \\
и, следовательно \anglesECDpDCF\ который $= \drawAngle{GCA}$ будет $> \drawAngle{A}$.
\stopCenterAlign

\qed
\stopProposition

\startProposition[title={Предложение XVII. Теорема}, reference=prop:I.XVII]
\defineNewPicture[1/4]{
pair A, B, C, D;
A := (0, 0);
B := A shifted (3/2u, 5/2u);
C := A shifted (9/4u, 0);
D := C shifted (u, 0);
draw byAngleWithName(B, A, C, byblue, 0)(A);
draw byAngleWithName(A, B, C, black, 0)(B);
draw byAngle(A, C, B, byred, 0);
draw byAngle(B, C, D, byyellow, 0);
byLineDefine(A, B, byred, 0, 0);
byLineDefine(B, C, byblue, 0, 0);
byLineDefine(A, C, black, 0, 0);
byLineDefine(C, D, black, 0, 0);
draw byNamedLineSeq(0)(noLine,BC,AB,AC,CD);
draw byLabelsOnPolygon(D, C, A, B)(0, 0);
}
\drawCurrentPictureInMargin
\problemNP{В}{о}{всяком треугольнике \drawLine[bottom]{AB,BC,AC} любые два угла взятые вместе меньше двух прямых углов.}

\startCenterAlign
Продлим \drawUnitLine{AC}, тогда\\
$\drawAngle{ACB} + \drawAngle{BCD} = \drawTwoRightAngles$

Но $\drawAngle{BCD} > \drawAngle{A}$ \inprop[prop:I.XVI]

$\therefore \drawAngle{ACB} + \drawAngle{A} < \drawTwoRightAngles$,
\stopCenterAlign

\noindent и таким же образом можно показать, что любые два других угла вместе будут меньше двух прямых углов.

\qed
\stopProposition

\startProposition[title={Предложение XVIII. Теорема}, reference=prop:I.XVIII]
\defineNewPicture[1/4]{
pair A, B, C, D;
A := (0, 0);
B := A shifted (5/2u, -1/2u);
C := A shifted (3/2u, 2u);
D := 2[C, A];
draw byAngleWithName(C, A, B, byblue, 0)(A);
draw byAngle(A, B, C, black, 0);
draw byAngle(D, B, A, byred, 0);
draw byAngleWithName(B, D, A, byyellow, 0)(D);
draw byLine(A, B, byyellow, 0, 0);
byLineDefine(A, C, byred, 0, 0);
byLineDefine(B, C, byblue, 0, 0);
byLineDefine(B, D, black, 0, 0);
byLineDefine(D, A, byred, 1, 0);
draw byNamedLineSeq(0)(AC,BC,BD,DA);
draw byLabelsOnPolygon(A, C, B, D)(0, 0);
}
\drawCurrentPictureInMargin
\problemNP{В}{о}{всяком треугольнике \drawLine{AC,BC,BD,DA}, если одна сторона \drawUnitLine{DA,AC} больше другой \drawUnitLine[0.5cm]{BC}, то противолежащий большей стороне угол будет больше противолежащего меньшей стороне угла, т. е. $\drawAngle{ABC,DBA} > \drawAngle{D}$}

\startCenterAlign
Сделаем $\drawUnitLine{AC} = \drawUnitLine{BC}$ \inprop[prop:I.III],\\ проведем \drawUnitLine{AB},

Тогда $\drawAngle{A} = \drawAngle{ABC}$ \inprop[prop:I.V];

но $\drawAngle{ABC} > \drawAngle{D}$ \inprop[prop:I.XVI]

$\therefore \drawAngle{ABC} > \drawAngle{D}$\\
и тем более $\drawAngle{ABC,DBA} > \drawAngle{D}$.
\stopCenterAlign

\qed
\stopProposition

\startProposition[title={Предложение XIX. Теорема}, reference=prop:I.XIX]
\defineNewPicture[1/4]{
pair A, B, C;
A := (0, 0);
B := A shifted (7/2u, 0);
C := A shifted (u, 3u);
draw byAngleWithName(C, A, B, byblue, 0)(A);
draw byAngleWithName(A, B, C, byred, 0)(B);
byLineDefine(A, B, black, 0, 0);
byLineDefine(B, C, byblue, 0, 0);
byLineDefine(C, A, byred, 0, 0);
draw byNamedLineSeq(0)(CA,BC,AB);
draw byLabelsOnPolygon(B, A, C)(0, 0);
}
\drawCurrentPictureInMargin
\problemNP{В}{о}{всяком треугольнике \drawLine[bottom]{CA,BC,AB} если один угол \drawAngle{A} больше другого \drawAngle{B}, то сторона \drawUnitLine{BC} противолежащая большему углу, больше стороны \drawUnitLine{CA} противолежащей меньшему.}

\startCenterAlign
Если \drawUnitLine{BC} не больше \drawUnitLine{CA} тогда\\
$\drawUnitLine{BC} =$ или $< \drawUnitLine{CA}$.

Если $\drawUnitLine{BC} = \drawUnitLine{CA}$ тогда\\
$\drawAngle{A} = \drawAngle{B}$ \inprop[prop:I.V],\\
что противоречит гипотезе.

\drawUnitLine{BC} также не меньше \drawUnitLine{CA};\\ 
поскольку, если это так, то $\drawAngle{A} < \drawAngle{B}$ \inprop[prop:I.XVIII]\\
что противоречит гипотезе:

$\therefore \drawUnitLine{BC} > \drawUnitLine{CA}$.
\stopCenterAlign

\qed
\stopProposition

\startProposition[title={Предложение XX. Теорема}, reference=prop:I.XX]
\defineNewPicture{
pair A, B, C, D;
A := (0, 0);
B := A shifted (7/2u, 0);
D := A shifted (4/3u, 3/2u);
C := ((fullcircle scaled 2arclength(D--B)) shifted D) intersectionpoint (D--10[A, D]);
draw byAngleWithName(B, C, A, byred, 0)(C);
draw byAngle(C, B, D, byblue, 0);
draw byAngle(D, B, A, byyellow, 0);
byLineDefine(B, D, byred, 0, 0);
byLineDefine(A, B, black, 0, 0);
byLineDefine(B, C, byyellow, 0, 0);
byLineDefine(C, D, byblue, 1, 0);
byLineDefine(D, A, byblue, 0, 0);
draw byNamedLineSeq(0)(BD);
draw byNamedLineSeq(0)(DA,CD,BC,AB);
draw byLabelsOnPolygon(D, C, B, A)(0, 0);
}
\drawCurrentPictureInMargin
\problemNP{Л}{юбые}{две стороны \drawUnitLine{DA} и \drawUnitLine{BD} всякого треугольника \drawLine[bottom]{DA,BD,AB} взятые вместе больше третьей стороны  \drawUnitLine{AB}.}

\startCenterAlign
Продлим \drawUnitLine{DA},\\
и сделаем $\drawUnitLine{CD} = \drawUnitLine{BD}$ \inprop[prop:I.III];\\
проведем \drawUnitLine{BC}.

Тогда, поскольку $\drawUnitLine{CD} = \drawUnitLine{BD}$ (постр.),\\
$\drawAngle{CBD} = \drawAngle{C}$ \inprop[prop:I.V]

$\therefore \drawAngle{CBD,DBA} > \drawAngle{C}$ \inax[ax:IX]

$\therefore \drawUnitLine{DA} + \drawUnitLine{CD} > \drawUnitLine{AB}$ \inprop[prop:I.XIX]

и $\therefore \drawUnitLine{DA} + \drawUnitLine{BD} > \drawUnitLine{AB}$
\stopCenterAlign

\qed
\stopProposition

\startProposition[title={Предложение XXI. Теорема}, reference=prop:I.XXI]
\defineNewPicture{
pair A, B, C, D, E;
A := (0, 0);
B := A shifted (7/2u, 0);
C := A shifted (3u, 4u);
D := 1/2[1/2[A, B], C];
E = whatever[A, D] = whatever[B, C];
draw byAngleWithName(B, D, A, byred, 0)(D);
draw byAngleWithName(B, E, D, byblue, 0)(E);
draw byAngleWithName(B, C, A, byyellow, 0)(C);
byLineDefine(B, D, byyellow, 0, 0);
byLineDefine(A, D, black, 0, 0);
byLineDefine(D, E, black, 1, 0);
byLineDefine(A, B, byblue, 1, 0);
byLineDefine(B, E, byred, 1, 0);
byLineDefine(E, C, byred, 0, 0);
byLineDefine(C, A, byblue, 0, 0);
draw byNamedLine(BD);
draw byNamedLineSeq(0)(AD,DE);
draw byNamedLineSeq(0)(CA,EC,BE,AB);
draw byLabelsOnPolygon(A, C, E, B)(0, 0);
draw byLabelsOnPolygon(A, D, E)(2, 0);
}
\drawCurrentPictureInMargin
\problemNP[2]{Е}{сли}{из любой точки \drawFromCurrentPicture{
startTempScale(1/5);
draw byNamedLineSeq(0)(AD,BD);
draw byLabelsOnPolygon(A, D, B)(2, 0);
stopTempScale;
} внутри треугольника \drawFromCurrentPicture[bottom]{
startAutoLabeling;
startTempScale(1/5);
draw byNamedLineSeq(0)(CA,EC,BE,AB);
stopTempScale;
stopAutoLabeling;
} провести прямые линии к концам стороны \drawSizedLine{AB}, эти прямые  вместе меньше двух других сторон треугольника, и будут заключать больший угол.}

\startCenterAlign
Продлим \drawSizedLine{AD},\\
$\drawSizedLine{CA} + \drawSizedLine{EC} > \drawSizedLine{AD,DE}$ \inprop[prop:I.XX],\\
добавим к каждой \drawSizedLine{BE},\\
$\drawSizedLine{CA} + \drawSizedLine{EC,BE} > \drawSizedLine{AD,DE} + \drawSizedLine{BE}$ \inax[ax:IV]

Таким же образом можно показать, что\\
$\drawSizedLine{AD,DE} + \drawSizedLine{BE} > \drawSizedLine{AD} + \drawSizedLine{BD}$,\\
$\therefore \drawSizedLine{CA} + \drawSizedLine{EC,BE} > \drawSizedLine{AD} + \drawSizedLine{BD}$,\\
что и требовалось доказать.

Далее $\drawAngle{E} > \drawAngle{C}$ \inprop[prop:I.XVI],\\
и также $\drawAngle{D} > \drawAngle{E}$ \inprop[prop:I.XVI],

$\therefore \drawAngle{D} > \drawAngle{C}$.
\stopCenterAlign

\qed
\stopProposition

\startProposition[title={Предложение XXII. Задача}, reference=prop:I.XXII]
\defineNewPicture[1/2]{
numeric r[], d;
pair A, B, C, D, E, LI, LII, LIII, LIV, LV, LVI;
path q[];
r1 := 3/2u;
r2 := 4/3u;
r3 := (2/3)*(r1+r2);
d := 1/3u;
A := (0, 0);
B := A shifted (r3, 0);
q1 := (fullcircle scaled 2r1) shifted A;
q2 := (fullcircle scaled 2r2) shifted B;
C := q1 intersectionpoint q2;
D := point 11/2 of q1;
E := point 3/4 of q2;
LI := (xpart(point 0 of q2), ypart(point 6 of q1) - 1/2d);
LII := LI shifted (-r3, 0);
LIII := LI shifted (0, -d);
LIV := LIII shifted (-r2, 0);
LV := LIII shifted (0, -d);
LVI := LV shifted (-r1, 0);
draw byCircle(A, D, byblue, 0, 0, 0)(A);
byLineDefine(A, D, byblue, 0, 0);
byLineDefine(B, E, byred, 0, 0);
byLineDefine(A, B, black, 0, 0);
byLineDefine(B, C, byyellow, 0, 0);
byLineDefine(C, A, byyellow, 1, 0);
draw byNamedLineSeq(0)(BC,CA);
draw byNamedLineSeq(0)(AD,AB,BE);
draw byLineWithName (LII, LI, black, 1, 0)(L');
draw byLineWithName (LIV, LIII, byred, 1, 0)(L'');
draw byLineWithName (LVI, LV, byblue, 1, 0)(L''');
draw byCircle(B, E, byred, 0, 0, 0)(B);
draw byLabelsOnPolygon(D, A, C)(2, 0);
draw byLabelsOnPolygon(E, B, A)(2, 0);
draw byLabelsOnPolygon(A, C, B)(2, 0);
draw byLabelsOnCircle(D)(A);
draw byLabelsOnCircle(E)(B);
draw byLabelLine(L', L'', L''');
}
\drawCurrentPictureInMargin
\problemNP{И}{з}{трех прямых линий $\left\{\vcenter{
\nointerlineskip\hbox{\drawSizedLine{L'}}
\nointerlineskip\hbox{\drawSizedLine{L''}}
\nointerlineskip\hbox{\drawSizedLine{L'''}}}\right.$
таких, что любые две вместе длиннее третьей, составить треугольник.}

\startCenterAlign
Предположим $\drawSizedLine{AB} = \drawSizedLine{L'}$ \inprop[prop:I.III].

$\left.\eqalign{
\mbox{Проведем } \drawSizedLine{AD} &= \drawSizedLine{L'''}\cr
\mbox{и } \drawSizedLine{BE} &= \drawSizedLine{L''}
}\right\}\mbox{\inprop[prop:I.II].}$

Взяв \drawSizedLine{AD} и \drawSizedLine{BE} как радиусы, опишем
\drawFromCurrentPicture{
draw byNamedLine(AD); draw byNamedCircle(A);
draw byLabelLineEnd(A, D, 0);
draw byLabelLineEnd(D, A, 0);
} и
\offsetPicture{12pt}{0pt}{\drawFromCurrentPicture{
draw byNamedLine(BE); draw byNamedCircle(B);
draw byLabelLineEnd(B, E, 0);
draw byLabelLineEnd(E, B, 0);
}} \inpost[post:III];\\
проведем \drawSizedLine{CA} и \drawSizedLine{BC},\\
тогда \drawLine[bottom]{CA,BC,AB} будет искомым треугольником.

$\left.\eqalign{
\mbox{Поскольку } \drawSizedLine{AB} &= \drawSizedLine{L'} \mbox{,} \cr
\drawSizedLine{BC} &= \drawSizedLine{BE} = \drawSizedLine{L''} \cr
\mbox{и } \drawSizedLine{CA} &= \drawSizedLine{AD} = \drawSizedLine{L'''} \cr
}\right\}\mbox{(постр.)}$
\stopCenterAlign

\qed
\stopProposition

\startProposition[title={Предложение XXIII. Задача}, reference=prop:I.XXIII]
\defineNewPicture{
angleScale := 3/2;
pair A, B, C, D, E, F, G, H, J, d;
A := (0, 0);
B := A shifted (7/2u, 0);
C := A shifted (3u, 11/5u);
D := 5/4[A, B];
E := 7/6[A, C];
d := (0, -3u);
F := A shifted d;
G := B shifted d;
H := C shifted d;
J := D shifted d;
draw byAngleWithName(B, A, C, byred, 0)(A);
draw byAngleWithName(G, F, H, byblue, 0)(F);
byLineDefine(B, D, black, 1, 1);
byLineDefine(C, E, byblue, 1, 1);
byLineDefine(A, B, black, 0, 1);
byLineDefine(C, A, byblue, 0, 1);
draw byLine(B, C, byred, 0, 1);
draw byNamedLineSeq(0)(CE,CA,AB,BD);
byLineDefine(G, J, black, 1, 0);
byLineDefine(F, G, black, 0, 0);
byLineDefine(G, H, byred, 0, 0);
byLineDefine(H, F, byyellow, 0, 0);
draw byNamedLineSeq(0)(noLine,GH,HF,FG,GJ);
draw byLabelsOnPolygon(D, B, A, C, E, noPoint)(0, 0);
draw byLabelsOnPolygon(noPoint, J, G, F, H, G)(2, 0);
}
\drawCurrentPictureInMargin
\problemNP{П}{ри}{данной точке \drawFromCurrentPicture{
startTempScale(scaleFactor/2);
draw byNamedLineSeq(0)(FG,HF);
draw byLabelsOnPolygon(G, F, H)(2, 0);
stopTempScale;
} на данной прямой \drawUnitLine{FG,GJ}, построить угол равный данному прямолинейному углу \drawAngle{A}.}

Проведем \drawUnitLine{BC} между любыми двумя точками сторонами данного угла.

\startCenterAlign
Построим \drawLine[bottom]{HF,GH,FG} \inprop[prop:I.XXII]\\
такой, что $\drawUnitLine{FG} = \drawUnitLine{AB}$, $\drawUnitLine{HF} = \drawUnitLine{CA}$\\
и $\drawUnitLine{GH} = \drawUnitLine{BC}$.

Тогда $\drawAngle{A} = \drawAngle{F}$ \inprop[prop:I.VIII].
\stopCenterAlign

\qed
\stopProposition

\startProposition[title={Предложение XXIV. Теорема}, reference=prop:I.XXIV]
\defineNewPicture{
pair A, B, C, D, E, F, G, d;
A := (0, 0);
B := A shifted (u, -5/2u);
C := A shifted (-u, -7/2u);
D := (xpart(C) - 3/2u, ypart(B));
d := (0, -4u);
E := A shifted d;
F := B shifted d;
G := C shifted d;
draw byAngle(B, A, C, black, 2);
draw byAngle(C, A, D, black, 3);
draw byAngle(F, E, G, black, 2);
draw byAngle(B, D, A, byblue, 0);
draw byAngle(C, D, B, byred, 0);
draw byAngle(D, C, A, byyellow, 0);
draw byAngle(A, C, B, black, 0);
byLineDefine(A, B, byblue, 0, 0);
byLineDefine(B, C, black, 1, 0);
draw byLine(C, A, byred, 0, 0);
draw byLine(B, D, black, 0, 0);
byLineDefine(A, D, byred, 1, 0);
byLineDefine(C, D, byblue, 1, 0);
draw byNamedLineSeq(0)(AB,AD,CD,BC);
byLineDefine(E, F, byblue, 0, 1);
byLineDefine(F, G, byyellow, 0, 1);
byLineDefine(G, E, byred, 0, 1);
draw byNamedLineSeq(0)(EF,FG,GE);
draw byLabelsOnPolygon(D, A, B, C)(0, 0);
draw byLabelsOnPolygon(G, E, F)(0, 0);
}
\drawCurrentPictureInMargin
\problemNP{Е}{сли}{у двух треугольников по две стороны соотверственно равны друг другу ($\drawUnitLine{AB} = \drawUnitLine{EF}$ и $\drawUnitLine{AD} = \drawUnitLine{GE}$), и угол заключенный ними в одном
\drawFromCurrentPicture[bottom]{
startAutoLabeling;
draw byNamedAngleSides(BAC,CAD)(AB,CA,AD);
stopAutoLabeling;
}
больше, чем в другом
\drawFromCurrentPicture[bottom][angleFEG]{
startAutoLabeling;
draw byNamedAngleSides(FEG)(EF, GE);
stopAutoLabeling;
},
то сторона \drawUnitLine{BD} противолежащая большему углу больше стороны, противолежащей меньшему \drawUnitLine{FG}.
}

\startCenterAlign
Сделаем $\drawFromCurrentPicture[bottom][angleBAC]{
startAutoLabeling;
draw byNamedAngleSides(BAC)(AB, CA);
stopAutoLabeling;
} = \angleFEG$ \inprop[prop:I.XXIII],\\
и $\drawUnitLine{CA} = \drawUnitLine{GE}$ \inprop[prop:I.III],\\
проведем \drawUnitLine{CD} и \drawUnitLine{BC}.

Поскольку $\drawUnitLine{CA} = \drawUnitLine{AD}$ (\inaxL[ax:I]. гип. постр.)\\
$\therefore \drawAngle{BDA,CDB} = \drawAngle{DCA}$ \inprop[prop:I.V]
но $\drawAngle{CDB} < \drawAngle{DCA}$,\\
и $\therefore \drawAngle{CDB} < \drawAngle{DCA,ACB}$,

$\therefore \drawUnitLine{BD} > \drawUnitLine{BC}$ \inprop[prop:I.XIX]

но $\drawUnitLine{BC} = \drawUnitLine{FG}$ \inprop[prop:I.IV]

$\therefore \drawUnitLine{BD} > \drawUnitLine{FG}$.
\stopCenterAlign

\qed
\stopProposition

\startProposition[title={Предложение XXV. Теорема}, reference=prop:I.XXV]
\defineNewPicture{
pair A, B, C, D, E, F, d;
A := (0, 0);
B := A shifted (u, -3u);
C := A shifted (-7/4u, -4u);
d := (0, -9/2u);
D := A shifted d;
E := ((B shifted -A) rotated -10) shifted d;
F := C shifted d;
draw byAngleWithName(B, A, C, byyellow, 0)(A);
draw byAngleWithName(E, D, F, black, 0)(D);
byLineDefine(A, B, byblue, 0, 0);
byLineDefine(B, C, black, 0, 0);
byLineDefine(C, A, byred, 0, 0);
draw byNamedLineSeq(0)(AB,BC,CA);
byLineDefine(D, E, byblue, 0, 1);
byLineDefine(E, F, byyellow, 0, 1);
byLineDefine(F, D, byred, 0, 1);
draw byNamedLineSeq(0)(DE,EF,FD);
draw byLabelsOnPolygon(A, B, C)(0, 0);
draw byLabelsOnPolygon(D, E, F)(0, 0);
}
\drawCurrentPictureInMargin
\problemNP{Е}{сли}{у двух треугольников две стороны \drawUnitLine{AB} и \drawUnitLine{CA} соответственно равны двум сторонам \drawUnitLine{DE} и \drawUnitLine{FD} другого, но основания неравны, то угол над большим основанием \drawUnitLine{BC} одного треугольника меньше угла под меньшим \drawUnitLine{EF} другого.}

\startCenterAlign
$\drawAngle{A} =\mbox{, } > \mbox{ или } < \drawAngle{D}$

\drawAngle{A} не равен \drawAngle{D}\\
поскольку если $\drawAngle{A} = \drawAngle{D}$ то $\drawUnitLine{BC} = \drawUnitLine{EF}$ \inprop[prop:I.IV]\\
что противоречит гипотезе;

\drawAngle{A} не меньше \drawAngle{D}\\
поскольку если $\drawAngle{A} < \drawAngle{D}$\\
то $\drawUnitLine{BC} < \drawUnitLine{EF}$ \inprop[prop:I.XXIV],\\
что противоречит гипотезе:

$\therefore \drawAngle{A} > \drawAngle{D}$.
\stopCenterAlign

\qed
\stopProposition

\startProposition[title={Предложение XXVI. Теорема}, reference=prop:I.XXVI]
\defineNewPicture{
pair A, B, C, D, E, F, G, d;
A := (0, 0);
B := A shifted (3u, 0);
C := A shifted (2u, 3u);
d := (0, -4u);
D := A shifted d;
E := B shifted d;
F := C shifted d;
G := 3/4[D, F];
draw byAngleWithName(B, A, C, byyellow, 0)(A);
draw byAngleWithName(C, B, A, byred, 0)(B);
draw byAngleWithName(A, C, B, black, 1)(C);
byLineDefine(A, B, byblue, 0, 0);
byLineDefine(B, C, black, 0, 0);
byLineDefine(C, A, byred, 0, 0);
draw byNamedLineSeq(0)(CA,BC,AB);
draw byAngleWithName(E, D, F, byyellow, 0)(D);
draw byAngle(G, E, D, black, 0);
draw byAngle(F, E, G, byblue, 0);
draw byAngleWithName(D, F, E, black, 1)(F);
draw byLine(E, G, byyellow, 0, 1);
byLineDefine(D, E, byblue, 0, 1);
byLineDefine(E, F, black, 0, 1);
byLineDefine(F, G, byred, 1, 1);
byLineDefine(G, D, byred, 0, 1);
draw byNamedLineSeq(0)(GD,FG,EF,DE);
draw byLabelsOnPolygon(C, B, A)(0, 0);
draw byLabelsOnPolygon(D, G, F, E)(0, 0);
}
\problemNP{Е}{сли}{у двух треугольников два угла соответственно равны двум углам другого ($\drawAngle{A} = \drawAngle{D}$ и $\drawAngle{B} = \drawAngle{GED,FEG}$), и одна сторона равна одного равна так же расположенной стороне другого, то и остальные стороны и углы соответственно равны друг другу.}
\drawCurrentPictureInMargin
\startsubproposition[title={Случай I.}]
\startCenterAlign
Пусть  \drawUnitLine{AB} и \drawUnitLine{DE}, лежащие между равными углами равны,\\
тогда $\drawUnitLine{CA} = \drawUnitLine{GD,FG}$.

Поскольку, если \drawUnitLine{GD,FG} больше,\\
сделаем $\drawUnitLine{CA} = \drawUnitLine{GD}$, проведем \drawUnitLine{EG}.

В \drawLine[bottom]{CA,BC,AB} и
\drawLine[bottom]{GD,EG,DE} получим \\
$\drawUnitLine{CA} = \drawUnitLine{GD}$, $\drawAngle{A} = \drawAngle{D}$, $\drawUnitLine{AB} = \drawUnitLine{DE}$;\\
$\therefore \drawAngle{B} = \drawAngle{GED}$ (pr. 4.)\\
но $\drawAngle{B} = \drawAngle{GED,FEG}$ (hyp.)

и следовательно $\drawAngle{GED} = \drawAngle{GED,FEG}$, что не имеет смысла, а значит ни \drawUnitLine{CA} ни \drawUnitLine{GD,FG} не больше другой, и $\therefore$ они равны;

$\therefore \drawUnitLine{BC} = \drawUnitLine{EF}$, и $\drawAngle{C} = \drawAngle{F}$ \inprop[prop:I.IV].
\stopCenterAlign
\stopsubproposition

\vfill\pagebreak

\defineNewPicture{
pair A, B, C, D, E, F, G, d;
d := (0, -4u);
A := (0, 0);
B := A shifted (3u, 0);
C := A shifted (1u, 3u);
D := A shifted d;
E := B shifted d;
F := C shifted d;
G := 3/4[D, E];
draw byAngleWithName(B, A, C, byyellow, 0)(A);
draw byAngleWithName(C, B, A, byred, 0)(B);
byLineDefine(A, B, byblue, 0, 0);
byLineDefine(B, C, black, 0, 0);
byLineDefine(C, A, byred, 0, 0);
draw byNamedLineSeq(0)(CA,AB,BC);
draw byAngleWithName(F, D, E, byyellow, 0)(D);
draw byAngleWithName(F, G, D, black, 0)(G);
draw byAngleWithName(F, E, D, byred, 0)(E);
draw byLine(F, G, byyellow, 0, 1);
byLineDefine(D, G, byblue, 0, 1);
byLineDefine(G, E, byblue, 1, 1);
byLineDefine(E, F, black, 0, 1);
byLineDefine(F, D, byred, 0, 1);
draw byNamedLineSeq(0)(FD,EF,GE,DG);
draw byLabelsOnPolygon(C, B, A)(0, 0);
draw byLabelsOnPolygon(D, F, E, G)(0, 0);
}
\drawCurrentPictureInMargin
\startsubproposition[title={Случай II.}]
\startCenterAlign
Теперь пусть $\drawUnitLine{CA} = \drawUnitLine{FD}$, лежат против равных углов \drawAngle{B} и \drawAngle{E}.\\ 
Если такое возможно, пусть $\drawUnitLine{DG,GE} > \drawUnitLine{AB}$, тогда возьмем $\drawUnitLine{DG} = \drawUnitLine{AB}$, проведем \drawUnitLine{FG}.

Тогда в \drawLine[bottom]{CA,BC,AB} и \drawLine[bottom]{FD,FG,DG} получим $\drawUnitLine{CA} = \drawUnitLine{FD}$, $\drawUnitLine{AB} = \drawUnitLine{DG}$ и $\drawAngle{A} = \drawAngle{D}$,

$\therefore \drawAngle{B} = \drawAngle{G}$ \inprop[prop:I.IV]\\
но $\drawAngle{B} = \drawAngle{E}$ (гип.)

$\therefore \drawAngle{G} = \drawAngle{E}$ что не имеет смысла \inprop[prop:I.XVI]

Следовательно, ни \drawUnitLine{AB} ни \drawUnitLine{DG,GE} не больше другой, а значит они равны. Следовательно (согласно \inpropL[prop:I.IV]) треугольники равны во всех отношениях.
\stopCenterAlign
\stopsubproposition

\qed
\stopProposition

\startProposition[title={Предложение XXVII. Теорема}, reference=prop:I.XXVII]
\defineNewPicture{
pair A, B, C, D, E, F, G, H, I, d;
A := (0, 0);
B := A shifted (8/3u, 0);
d := (0, -7/4u);
C := A shifted d;
D := B shifted d;
E := 1/3[A, B];
F := 2/3[C, D];
G := 3/2[F, E];
H := 3/2[E, F];
I := 1/2[A, C] shifted (-2u, 0);
draw byAngle(A, E, F, byyellow, 0);
draw byAngle(F, E, B, byred, 0);
draw byAngle(C, F, E, byblue, 0);
draw byAngle(E, F, D, byyellow, 0);
byLineDefine(I, A, byblue, 0, 0);
byLineDefine(A, B, byblue, 0, 0);
byLineDefine(I, C, byred, 0, 0);
byLineDefine(C, D, byred, 0, 0);
draw byNamedLineSeq(0)(CD,IC,IA,AB);
draw byLine(G, H, black, 0, 0);
draw byLabelLine(AB, CD, GH);
}
\drawCurrentPictureInMargin
\problemNP{Е}{сли}{прямая \drawUnitLine{GH}, пересекая две другие прямые \drawUnitLine{CD} и \drawUnitLine{AB}, образует накрестлежащие углы \drawAngle{CFE}, \drawAngle{FEB} и \drawAngle{EFD} и \drawAngle{AEF} равные между собой, то две пересекаемые прямые параллельны.}

Если \drawUnitLine{CD} не параллельна \drawUnitLine{AB}, то они сойдутся по продолжении.

Если это возможно, пусть они не будут параллельны, но сойдутся, если их продолжить; тогда внешний угол \drawAngle{FEB} будет больше \drawAngle{CFE} \inprop[prop:I.XVI], но они равны (гип.), что не имеет смысла. Таким же образом можно показать, что они не сойдутся с другой стороны; $\therefore$ они параллельны.

\qed
\stopProposition

\startProposition[title={Предложение XXVIII. Теорема}, reference=prop:I.XXVIII]
\defineNewPicture{
pair A, B, C, D, E, F, G, H, d;
A := (0, 0);
B := A shifted (7/2u, 0);
d := (0, -3/2u);
C := A shifted d;
D := B shifted d;
E := 9/20[A, B];
F := 11/20[C, D];
G := 7/4[F, E];
H := 4/3[E, F];
draw byAngle(G, E, A, black, 0);
draw byAngle(B, E, G, byyellow, 0);
draw byAngle(A, E, F, byred, 0);
draw byAngle(F, E, B, byblue, 0);
draw byAngle(C, F, E, byblue, 0);
draw byAngle(E, F, D, byred, 0);
draw byLine(A, B, byred, 0, 0);
draw byLine(C, D, byyellow, 0, 0);
draw byLine(G, H, black, 0, 0);
draw byLabelLine(AB, CD, GH);
}
\drawCurrentPictureInMargin
\problemNP{Е}{сли}{прямая \drawUnitLine{GH}, падающая на две прямые \drawUnitLine{AB} и \drawUnitLine{CD} образует внешний угол, равный внутреннему противолежащему с той же стороны $\drawAngle{GEA} = \drawAngle{CFE}$ или $\drawAngle{BEG} = \drawAngle{EFD}$, или внутренние углы с одной стороны \drawAngle{EFD} и \drawAngle{FEB} или \drawAngle{CFE} и \drawAngle{AEF} вместе равны двум прямым углам, то прямые параллельны.}

\startCenterAlign
Во-первых, если $\drawAngle{GEA} = \drawAngle{CFE}$,\\ 
то $\drawAngle{GEA} = \drawAngle{FEB}$ \inprop[prop:I.XV],\\
$\therefore \drawAngle{CFE} = \drawAngle{FEB} \therefore \drawUnitLine{AB} \parallel \drawUnitLine{CD}$ \inprop[prop:I.XXVII].

Во-вторых, если $\drawAngle{CFE} + \drawAngle{AEF} = \drawTwoRightAngles$,\\
то $\drawAngle{AEF} + \drawAngle{FEB} = \drawTwoRightAngles$ \inprop[prop:I.XIII],\\
$\therefore \drawAngle{CFE} + \drawAngle{AEF} = \drawAngle{AEF} + \drawAngle{FEB}$ \inax[ax:III]

$\therefore \drawAngle{CFE} = \drawAngle{FEB}$

$\therefore \drawUnitLine{AB} \parallel \drawUnitLine{CD}$
\stopCenterAlign

\qed
\stopProposition

\startProposition[title={Предложение XXIX. Теорема}, reference=prop:I.XXIX]
\defineNewPicture{
pair A, B, C, D, E, F, G, H, I, J, d[];
A := (0, 0);
B := A shifted (7/2u, 0);
d1 := (0, -2u);
C := A shifted d1;
D := B shifted d1;
E := 11/20[A, B];
F := 9/20[C, D];
G := 7/4[F, E];
H := 4/3[E, F];
d2 := (3/2u, 1/2u);
I := E shifted -d2;
J := E shifted d2;
draw byAngle(I, E, A, byblue, 0);
draw byAngle(F, E, I, byyellow, 0);
draw byAngle(B, E, F, black, 0);
draw byAngle(G, E, B, byred, 0);
draw byAngle(E, F, D, black, 0);
draw byLine(I, E, black, 0, 0);
draw byLine(E, J, black, 1, 0);
draw byLine(A, B, byyellow, 0, 0);
draw byLine(C, D, byred, 0, 0);
draw byLine(G, H, byblue, 0, 0);
draw byLabelLine(AB, CD, GH);
draw byLabelsOnPolygon(A, E, G)(2, 0);
draw byLabelPoint(I, lineAngle.IE - 90, 1);
draw byLabelPoint(J, lineAngle.EJ + 90, 1);
}
\drawCurrentPictureInMargin
\problemNP{П}{рямая}{\drawUnitLine{GH}, падающая на две параллельные прямые \drawUnitLine{AB} и \drawUnitLine{CD}, образует накрестлежащие углы, равные между собой, внешний и противолежащий с той же стороны внутренний углы, равные между собой, а также внутренние односторонние углы, равные двум прямым углам.}

Если накрестлежащие углы \drawAngle{IEA,FEI} и \drawAngle{EFD} не равны, проведем \drawUnitLine{IE}, так, чтобы $\drawAngle{FEI} = \drawAngle{EFD}$ \inprop[prop:I.XXIII].

Следовательно $\drawUnitLine{IE,EJ} \parallel \drawUnitLine{CD}$ \inprop[prop:I.XXVII] и, следовательно, две пересекающихся прямых параллельны одной и той же прямой, что невозможно \inax[ax:XII].

А значит \drawAngle{IEA,FEI} и \drawAngle{EFD} не являются неравными, то есть, они равны $\drawAngle{IEA,FEI} = \drawAngle{GEB}$ \inprop[prop:I.XV]; $\therefore \drawAngle{GEB} = \drawAngle{EFD}$, внешний угол равен внутреннему, противолежащему с той же стороны: если к обоим добавить \drawAngle{BEF}, то $\drawAngle{EFD} + \drawAngle{BEF} = \drawAngle{BEF,GEB} = \drawTwoRightAngles$ \inprop[prop:I.XIII]. Другими словами, два внутренних угла по одну сторону пересекающей прямой равны двум прямым углам.

\qed
\stopProposition

\startProposition[title={Предложение XXX. Теорема}, reference=prop:I.XXX]
\defineNewPicture{
pair A, B, C, D, E, F, G, H, I, J, K, d;
A := (0, 0);
B := A shifted (7/2u, 0);
d := (0, -u);
C := A shifted d;
D := B shifted d;
E := C shifted d;
F := D shifted d;
G := 13/20[A, B];
H := 7/20[E, F];
I := (G--H) intersectionpoint (C--D);
J := 3/2[H, G];
K := 5/4[G, H];
draw byAngleWithName(B, G, J, byyellow, 0)(G);
draw byAngleWithName(D, I, J, byblue, 0)(I);
draw byAngleWithName(F, H, J, byred, 0)(H);
draw byLine(A, B, byred, 0, 0);
draw byLine(C, D, byyellow, 0, 0);
draw byLine(E, F, byblue, 0, 0);
draw byLine(J, K, black, 0, 0);
draw byLabelLine(AB, CD, EF, JK);
draw byLabelsOnPolygon(J, G, A)(2, 0);
draw byLabelsOnPolygon(J, I, C)(2, 0);
draw byLabelsOnPolygon(J, H, E)(2, 0);
}
\drawCurrentPictureInMargin
\problemNP{П}{рямые}{\drawUnitLine{AB} и \drawUnitLine{EF} параллельные одной и той же прямой \drawUnitLine{CD}, параллельны между собой}

\startCenterAlign
Пусть \drawUnitLine{JK} пересекает $\left\{\vcenter{\nointerlineskip\hbox{\drawUnitLine{AB}}\nointerlineskip\hbox{\drawUnitLine{CD}}\nointerlineskip\hbox{\drawUnitLine{EF}}}\right\}$;

Тогда $\drawAngle{G} = \drawAngle{I} = \drawAngle{H}$ \inprop[prop:I.XXIX],

$\therefore \drawAngle{G} = \drawAngle{H}$

$\therefore \drawUnitLine{AB} \parallel \drawUnitLine{EF}$ \inprop[prop:I.XXVII].
\stopCenterAlign

\qed
\stopProposition

\startProposition[title={Предложение XXXI. Задача}, reference=prop:I.XXXI]
\defineNewPicture[1/6]{
pair A, B, C, D, E, F, d;
A := (0, 0);
B := A shifted (7/2u, 0);
d := (0, -3u);
C := A shifted d;
D := B shifted d;
E := 4/5[A, B];
F := 1/5[C, D];
draw byAngleWithName(F, E, A, byyellow, 0)(E);
draw byAngleWithName(E, F, D, byred, 0)(F);
draw byLine(E, F, black, 0, 0);
draw byLine(A, E, byred, 0, 0);
draw byLine(E, B, byred, 1, 0);
draw byLine(C, F, byblue, 0, 0);
draw byLine(F, D, byblue, 0, 0);
draw byLabelsOnPolygon(A, E, B, noPoint, D, F, C, noPoint)(0, 0);
}
\drawCurrentPictureInMargin
\problemNP{П}{ровести}{через данную точку 
\drawFromCurrentPicture[middle][pointE]{
draw byNamedLineSeq(0)(AE,EF);
draw byLabelsOnPolygon(A, E, F)(2, 0);
} 
прямую, параллельную данной прямой \drawUnitLine{CF,FD}.}

\startCenterAlign
Проведем \drawUnitLine{EF} из точки \pointE к любой точке
\drawFromCurrentPicture[middle][pointF]{
draw byNamedLineSeq(0)(FD,EF);
draw byLabelsOnPolygon(D, F, E)(2, 0);
}
на \drawUnitLine{CF,FD},

сделаем $\drawAngle{E} = \drawAngle{F}$ \inprop[prop:I.XXIII],

тогда $\drawUnitLine{AE,EB} \parallel \drawUnitLine{CF,FD}$ \inprop[prop:I.XXVII].
\stopCenterAlign

\qed
\stopProposition

\startProposition[title={Предложение XXXII. Теорема}, reference=prop:I.XXXII]
\defineNewPicture[1/6]{
pair A, B, C, D, E;
A := (0, 0);
B := A shifted (-7/4u, -3u);
C := A shifted (7/4u, -3u);
D := 4/3[B, A];
E := A shifted (unitvector(C-B) scaled 3/2u);
draw byAngle(D, A, E, byred, 0);
draw byAngle(E, A, C, black, 0);
draw byAngle(C, A, B, byblue, 0);
draw byAngleWithName(A, B, C, byyellow, 0)(B);
draw byAngleWithName(B, C, A, black, 0)(C);
draw byLine(A, E, byblue, 0, 0);
byLineDefine(A, D, black, 1, 0);
byLineDefine(A, B, black, 0, 0);
byLineDefine(B, C, byred, 0, 0);
byLineDefine(C, A, byyellow, 0, 0);
draw byNamedLineSeq(0)(CA,noLine,AD,AB,BC);
draw byLabelsOnPolygon(C, B, A, D, noPoint, E, A)(0, 0);
}
\drawCurrentPictureInMargin
\problemNP[2]{П}{о}{продлении любой стороны треугольника \drawUnitLine{AB}, the внешний угол \drawAngle{DAE,EAC} равен сумме двух внутренних и противолежащих \drawAngle{B} и \drawAngle{C}, а три внутренних угла треугольника вместе равны двум прямым углам.}

\startCenterAlign
Через точку
\drawFromCurrentPicture{
draw byNamedLineSeq(0)(AB,CA);
draw byLabelsOnPolygon(B, A, C)(2, 0);
}
проведем \\
$\drawUnitLine{AE} \parallel \drawUnitLine{BC}$ \inprop[prop:I.XXXI].

Тогда $\left\{\eqalign{\drawAngle{DAE} &= \drawAngle{B}\cr \drawAngle{EAC} &= \drawAngle{C}\cr}\right\}$ \inprop[prop:I.XXIX],

$\therefore \drawAngle{B} + \drawAngle{C} = \drawAngle{DAE,EAC}$ \inax[ax:II],

и следовательно\\
$\drawAngle{B} + \drawAngle{CAB} + \drawAngle{C} = \drawAngle{DAE,EAC,CAB} = \drawTwoRightAngles$ \inprop[prop:I.XIII]
\stopCenterAlign

\qed
\stopProposition

\startProposition[title={Предложение XXXIII. Теорема}, reference=prop:I.XXXIII]
\defineNewPicture[1/4]{
pair A, B, C, D, d[];
d1 := (5/2u, 0);
d2 := (-7/8u, -3u);
A := (0, 0);
B := A shifted d1;
C := A shifted d2;
D := C shifted d1;
draw byAngle(B, A, D, byyellow, 0);
draw byAngle(D, A, C, byred, 0);
draw byAngle(C, D, A, black, 0);
draw byAngle(A, D, B, byblue, 0);
draw byLine(A, D, black, 0, 0);
byLineDefine(A, B, byred, 0, 0);
byLineDefine(C, D, byred, 1, 0);
byLineDefine(A, C, byblue, 0, 0);
byLineDefine(B, D, byyellow, 0, 0);
draw byNamedLineSeq(0)(AB,BD,CD,AC);
draw byLabelsOnPolygon(A, B, D, C)(0, 0);
}
\drawCurrentPictureInMargin
\problemNP{П}{рямые}{\drawUnitLine{AC} и \drawUnitLine{BD}, соединяющие с одной и той же стороны равные и параллельные прямые\drawUnitLine{AB} и \drawUnitLine{CD}, сами равны и параллельны.}

\startCenterAlign
Проведем диагональ \drawUnitLine{AD}.

$\drawUnitLine{AB} = \drawUnitLine{CD}$ (гип.)\\
$\drawAngle{BAD} = \drawAngle{CDA}$ \inprop[prop:I.XXIX]\\
и \drawUnitLine{AD} общая обоим треугольникам;

$\therefore \drawUnitLine{AC} = \drawUnitLine{BD}$, и $\drawAngle{ADB} = \drawAngle{DAC}$ \inprop[prop:I.IV];

и $\therefore \drawUnitLine{AC} \parallel \drawUnitLine{BD}$ \inprop[prop:I.XXVII].
\stopCenterAlign

\qed
\stopProposition

\startProposition[title={Предложение XXXIV. Теорема}, reference=prop:I.XXXIV]
\defineNewPicture{
pair A, B, C, D, d[];
d1 := (5/2u, 0);
d2 := (-7/8u, -3u);
A := (0, 0);
B := A shifted d1;
C := A shifted d2;
D := C shifted d1;
draw byAngle(B, A, D, byblue, 0);
draw byAngle(D, A, C, byred, 0);
draw byAngle(C, D, A, byyellow, 0);
draw byAngle(A, D, B, byred, 0);
draw byAngleWithName(A, C, D, black, 0)(C);
draw byAngleWithName(D, B, A, black, 0)(B);
draw byLine(A, D, black, 0, 0);
byLineDefine(A, B, byred, 0, 0);
byLineDefine(C, D, byred, 1, 0);
byLineDefine(A, C, byyellow, 0, 0);
byLineDefine(B, D, byblue, 0, 0);
draw byNamedLineSeq(0)(AB,BD,CD,AC);
draw byLabelsOnPolygon(A, B, D, C)(0, 0);
}
\drawCurrentPictureInMargin
\problemNP{П}{ротивоположные}{стороны и углы параллелограмма равны, а диагональ \drawUnitLine{AD} делит его на две равные части.}

\startCenterAlign
Поскольку $\left\{\eqalign{\drawAngle{BAD} &= \drawAngle{CDA}\cr\drawAngle{DAC} &= \drawAngle{ADB}\cr}\right\}$ \inprop[prop:I.XXIX]\\
и \drawUnitLine{AD} общая обоим треугольникам.

$\therefore \left\{\eqalign{\drawUnitLine{AB} &= \drawUnitLine{CD}\cr \drawUnitLine{AC} &= \drawUnitLine{BD}\cr \drawAngle{B} &= \drawAngle{C}\cr}\right\}$ \inprop[prop:I.XXVI]\\
и $\drawAngle{BAD,DAC} = \drawAngle{CDA,ADB}$ \inax[ax:II]:
\stopCenterAlign

Следовательно, противоположные стороныи углы параллелограмма равны. И, поскольку треугольники \drawLine{AD,CD,AC} и \drawLine{AB,BD,AD} равны во всех отношениях \inprop[prop:I.IV], диагональ делит параллелограмм на две равные части. 

\qed
\stopProposition

\startProposition[title={Предложение XXXV. Теорема}, reference=prop:I.XXXV]
\defineNewPicture{
pair A, B, C, D, E, F, G, d[];
d1 := (7/4u, 0);
d2 := (u, -3u);
d3 := (-2u, -3u);
A := (0, 0);
B := A shifted d1;
C := A shifted d2;
D := C shifted d1;
E := C shifted -d3;
F := D shifted -d3;
G := (B--D) intersectionpoint (C--E);
draw byPolygon(A,B,G,C)(byyellow);
draw byPolygon(E,F,D,G)(byyellow);
draw byPolygon(B,E,G)(byyellow);
draw byPolygon(C,D,G)(byblue);
draw byAngleWithName(B, A, C, byred, 0)(A);
draw byAngleWithName(E, B, D, byblue, 0)(B);
draw byAngleWithName(A, E, C, black, 0)(E);
draw byAngleWithName(A, F, D, white, 0)(F);
byLineDefine(A, C, byblue, 0, 0);
byLineDefine(B, D, byred, 0, 0);
byLineDefine(C, D, black, 0, 0);
draw byNamedLineSeq(0)(AC,CD,BD);
draw byLabelsOnPolygon(C, A, B, E, F, D)(0, 0);
}
\drawCurrentPictureInMargin
\problemNP{П}{араллелограммы}{, находящиеся на одном и том же основании и между одними и теми же параллельными прямыми, равны по площади между собой.}

\startCenterAlign
Из параллельности прямых следует,\\
$\left.\eqalign{
\drawAngle{A} &= \drawAngle{B};\cr
\drawAngle{E} &= \drawAngle{F};\cr
\mbox{и } \drawUnitLine{AC} &= \drawUnitLine{BD}\cr
}\right\}
\eqalign{
&\mbox{\inprop[prop:I.XXIX]}\cr
&\mbox{\inprop[prop:I.XXIX]}\cr
&\mbox{\inprop[prop:I.XXXIV]}\cr
}$

Но
$\drawFromCurrentPicture[middle][polygonABC]{
startAutoLabeling;
draw byNamedPolygon (ABGC, BEG);
stopAutoLabeling;
draw byNamedLine (AC);
}
=
\drawFromCurrentPicture[middle][polygonEFD]{
startAutoLabeling;
draw byNamedPolygon (EFDG, BEG);
stopAutoLabeling;
draw byNamedLine (BD);
}$ \inprop[prop:I.VIII]

$\therefore
\drawFromCurrentPicture[middle][polygonAFDC]{
startAutoLabeling;
draw byNamedPolygon (ABGC, EFDG, BEG, CDG);
stopAutoLabeling;
draw byNamedLine (AC);
} - \polygonEFD =
\drawPolygon{ABGC, CDG}$,\\
и $\drawPolygon{ABGC, CDG} - \polygonABC =
\drawPolygon{EFDG, CDG}$;

$\therefore \drawPolygon{ABGC, CDG} = \drawPolygon{EFDG, CDG}$.
\stopCenterAlign

\qed
\stopProposition

\startProposition[title={Предложение XXXVI. Теорема}, reference=prop:I.XXXVI]
\defineNewPicture[1/5]{
pair A, B, C, D, E, F, G, H, J, I, d[];
numeric h;
h := 3u;
d1 := (3/2u, 0);
d2 := (2/3u, -h);
d3 := (-8/3u, -h);
d4 := (-1/2u, -h);
A := (0, 0);
B := A shifted d1;
C := A shifted d2;
D := C shifted d1;
E := C shifted -d3;
F := D shifted -d3;
G := E shifted d4;
H := F shifted d4;
I := (B--D) intersectionpoint (C--E);
J := (E--G) intersectionpoint (D--F);
draw byPolygon(A,B,I,C)(byred);
draw byPolygon(C,D,I)(byred);
draw byPolygon(I,D,J,E)(byblue);
draw byPolygon(E,F,J)(byyellow);
draw byPolygon(J,F,H,G)(byyellow);
byLineDefine(C, E, byyellow, 0, 0);
byLineDefine(D, F, black, 1, 0);
byLineDefine(C, D, black, 0, 0);
byLineDefine(E, F, byred, 0, 0);
draw byNamedLineSeq(0)(CE,EF,DF,CD);
draw byLineFull(G, H, byblue, 0, 0)(E, F, 1, 1, 0);
draw byLabelsOnPolygon(A, B, D, C)(0, 0);
draw byLabelsOnPolygon(E, F, H, G)(0, 0);
}
\drawCurrentPictureInMargin
\problemNP{П}{араллелограммы}{\drawPolygon[bottom][polygonABDC]{ABIC, CDI}~и~\drawPolygon[bottom][polygonEFHG]{EFJ, JFHG} на равных основаниях и между одними параллельными прямыми равны по площади.}

\startCenterAlign
Проведем \drawUnitLine{CE} и \drawUnitLine{DF}\\
$\drawUnitLine{CD} = \drawUnitLine{GH} = \drawUnitLine{EF}$ (\inpropL[prop:I.XXXIV], и гип.);

$\therefore \drawUnitLine{CD} = \mbox{ and } \parallel \drawUnitLine{EF}$;

$\therefore \drawUnitLine{CE} = \mbox{ and } \parallel \drawUnitLine{DF}$ \inprop[prop:I.XXXV]

И следовательно
\drawPolygon[bottom][polygonCDFE]{CDI, IDJE, EFJ}
параллелограмм,

но $\polygonABDC = \polygonCDFE = \polygonEFHG$ \inprop[prop:I.XXXV]

$\therefore \polygonABDC = \polygonEFHG$ \inax[ax:I].
\stopCenterAlign

\qed
\stopProposition

\startProposition[title={Предложение XXXVII. Теорема}, reference=prop:I.XXXVII]
\defineNewPicture{
pair A, B, C, D, E, F, G, H, I, d[];
d1 := (3/2u, 0);
d2 := (3/4u, -5/2u);
d3 := (-7/4u, -5/2u);
A := (0, 0);
B := A shifted d1;
C := A shifted d2;
D := C shifted d1;
E := C shifted -d3;
F := D shifted -d3;
G := (B--D) intersectionpoint (C--E);
H := 11/10[F, A];
I := 11/10[A, F];
draw byPolygon(A,B,C)(byblue);
draw byPolygon(B,C,G)(byyellow);
draw byPolygon(C,D,G)(byyellow);
draw byPolygon(D,G,E)(black);
draw byPolygon(E,F,D)(byred);
draw byLine(B, D, byred, 0, 0);
draw byLine(E, C, byblue, 0, 0);
byLineDefine(A, C, byred, 1, 0);
byLineDefine(F, D, byblue, 1, 0);
byLineDefine(C, D, black, 0, 0);
draw byNamedLineSeq(0)(AC,CD,FD);
draw byLine(H, I, black, 1, 0);
draw byLabelsOnPolygon(D, C, A, B, E, F)(0, 0);
draw byLabelLine(HI);
}
\drawCurrentPictureInMargin
\problemNP{Т}{реугольники}{
\drawPolygon[bottom][polygonBCD]{BCG, CDG} и~\drawPolygon[bottom][polygonCDE]{DGE, CDG}
на одном основании \drawUnitLine{CD} и между теми же параллельными равны.}

\startCenterAlign
$\left.\eqalign{\mbox{Проведем} \drawUnitLine{AC} &\parallel \drawUnitLine{BD}\cr
\drawUnitLine{FD} &\parallel \drawUnitLine{EC}}\right\}\mbox{\inprop[prop:I.XXXI]}$

Проведем \drawUnitLine{HI}.

\drawPolygon[bottom][polygonABDC]{ABC, BCG, CDG}
и
\drawPolygon[bottom][polygonEFDC]{DGE, CDG, EFD}
являются параллелограммами на одном основании и между одними параллельными прямыми и, следовательно, равными между собой. \inprop[prop:I.XXXV]

$\therefore \left\{\eqalign{\polygonABDC &= \mbox{ дважды } \polygonBCD\cr\polygonEFDC &= \mbox{ дважды } \polygonCDE\cr}\right\}$ \inprop[prop:I.XXXIV]

$\therefore \polygonBCD = \polygonCDE$.
\stopCenterAlign

\qed
\stopProposition

\startProposition[title={Предложение XXXVIII. Теорема}, reference=prop:I.XXXVIII]
\defineNewPicture{
pair A, B, C, D, E, F, G, H, J, I, d[];
numeric h;
h := 5/2u;
d1 := (3/2u, 0);
d2 := (2/3u, -h);
d3 := (-8/3u, -h);
d4 := (-1/2u, -h);
A := (0, 0);
B := A shifted d1;
C := A shifted d2;
D := C shifted d1;
E := C shifted -d3;
F := D shifted -d3;
G := E shifted d4;
H := F shifted d4;
I := (xpart(A), ypart(C));
J := (xpart(F), ypart(C));
draw byPolygon(A,B,C)(byyellow);
draw byPolygon(B,C,D)(byred);
draw byPolygon(E,F,H)(black);
draw byPolygon(E,G,H)(byblue);
draw byLine(B, D, byblue, 0, 0);
draw byLine(E, G, byred, 0, 0);
byLineDefine(A, C, byblue, 1, 0);
byLineDefine(F, H, byred, 1, 0);
byLineDefine(A, F, black, 1, 0);
draw byNamedLineSeq(0)(AC,AF,FH);
draw byLine(I, J, black, 1, 0);
draw byLabelsOnPolygon(A, B, D, C)(0, 0);
draw byLabelsOnPolygon(E, F, H, G)(0, 0);
}
\drawCurrentPictureInMargin
\problemNP{Т}{реугольники}{\drawPolygon[bottom][polygonBCD]{BCD} and \drawPolygon[bottom][polygonEGH]{EGH} на равных основаниях и между теми же параллельными прмыми равны.}

\startCenterAlign
$\left.\eqalign{\mbox{Проведем } \drawUnitLine{AC} &\parallel \drawUnitLine{BD}\cr
\mbox{и } \drawUnitLine{FH} &\parallel \drawUnitLine{EG}}\right\}\mbox{\inprop[prop:I.XXXI]}$\\
$\drawPolygon[bottom][polygonABDC]{ABC, BCD} = \drawPolygon[bottom][polygonEFHG]{EFH, EGH}$ \inprop[prop:I.XXXVI];

но $\polygonABDC = \mbox{ дважды } \polygonBCD$ \inprop[prop:I.XXXIV],\\
и $\polygonEFHG = \mbox{ дважды } \polygonEGH$ \inprop[prop:I.XXXIV],

$\therefore \polygonBCD = \polygonEGH$ \inax[ax:VII].
\stopCenterAlign

\qed
\stopProposition

\startProposition[title={Предложение XXXIX. Теорема}, reference=prop:I.XXXIX]
\defineNewPicture{
pair A, B, C, D, E, F, G;
A := (0, 0);
B := A shifted (5/2u, 0);
C := A shifted (u, -5/2u);
D := C shifted (3u, 0);
E = whatever[A, D] = whatever[B, C];
F := 5/4[C, B];
G := 6/4[C, B];
draw byPolygon(A,B,E)(byred);
draw byPolygon(A,E,C)(byyellow);
draw byPolygon(B,E,D)(black);
draw byPolygon(E,C,D)(byyellow);
draw byPolygon(F,B,D)(byblue);
byLineDefine(A, F, byred, 0, 0);
byLineDefine(D, F, byyellow, 0, 0);
byLineDefine(A, B, byblue, 0, 0);
byLineDefine(C, D, black, 0, 0);
byLineDefine(C, G, black, 1, 0);
draw byNamedLineSeq(0)(AB,AF,DF,CD,CG);
draw byLabelsOnPolygon(F, D, C, A)(2, 0);
draw byLabelsOnPolygon(A, B, F)(2, 0);
draw byLabelsOnPolygon(A, F, G, noPoint)(0, 0);
}
\drawCurrentPictureInMargin
\problemNP{Р}{авные}{треугольники
\drawPolygon[bottom][polygonADC]{AEC, ECD} и~\drawPolygon[bottom][polygonBDC]{BED, ECD}, находящиеся на одном основании \drawUnitLine{CD} и с одной и той же стороны, находятся между теми же параллельными прямыми.}

\startCenterAlign
Если \drawUnitLine{AB}, соединяющая вершины треугольников, не $\parallel \drawUnitLine{CD}$, проведем $\drawUnitLine{AF} \parallel \drawUnitLine{CD}$ \inprop[prop:I.XXXI], касающуюся \drawUnitLine{CG}.

Проведем \drawUnitLine{DF}.

Поскольку $\drawUnitLine{AF} \parallel \drawUnitLine{CD}$ (постр.)\\
$\polygonADC =
\drawPolygon[bottom][polygonFDC]{BED, ECD, FBD}$ \inprop[prop:I.XXXVII];\\
но $\polygonADC = \polygonBDC$ (hyp.);

$\therefore \polygonBDC = \polygonFDC$, часть равна целому, что не имеет смысла.

$\therefore \drawUnitLine{AF} \nparallel \drawUnitLine{CD}$; и таким же образом можно показать, что никакая другая линия, кроме \drawUnitLine{AB} не $\parallel \drawUnitLine{CD}$; $\therefore \drawUnitLine{AB} \parallel \drawUnitLine{CD}$.
\stopCenterAlign

\qed
\stopProposition

\startProposition[title={Предложение XL. Теорема}, reference=prop:I.XL]
\defineNewPicture{
pair A, B, C, D, E, F, G, H, d;
A := (0, 0);
B := A shifted (5/2u, 0);
C := A shifted (-u, -9/4u);
d := (2u, 0);
D := C shifted d;
E := B shifted (-1/2u, -9/4u);
F := E shifted d;
G := 5/4[E, B];
H := 2[B, G];
draw byPolygon(A,C,D)(byyellow);
draw byPolygon(B,E,F)(byred);
draw byPolygon(G,B,F)(byblue);
draw byLine(E, H, black, 1, 0);
byLineDefine(A, G, byred, 0, 0);
byLineDefine(F, G, byyellow, 0, 0);
byLineDefine(A, B, byblue, 0, 0);
byLineDefine(C, D, black, 0, 0);
byLineDefine(E, F, black, 0, 0);
byLineDefine(D, E, byblue, 1, 0);
draw byNamedLineSeq(0)(AB,AG,FG,EF,DE,CD);
draw byLabelsOnPolygon(G, F, E, D, C, A)(2, 0);
draw byLabelsOnPolygon(A, B, G)(2, 0);
draw byLabelsOnPolygon(A, G, H, noPoint)(0, 0);
}
\drawCurrentPictureInMargin
\problemNP{Р}{авные}{треугольники
\drawFromCurrentPicture[bottom][polygonACD]{
startAutoLabeling;
draw byNamedPolygon (ACD);
stopAutoLabeling;
draw byNamedLineFull(A, A, 1, 1, 0)(CD);
}
и
\drawFromCurrentPicture[bottom][polygonBEF]{
startAutoLabeling;
draw byNamedPolygon (BEF);
stopAutoLabeling;
draw byNamedLineFull(B, B, 1, 1, 0)(EF);
} находящиеся на равных основаниях и с одной и той же стороны, находятся между теми же параллельными.}

\startCenterAlign
Если \drawSizedLine{AB} соединяющая вершины треугольников не $\parallel \drawSizedLine{CD,DE,EF}$,\\
проведем \drawSizedLine{AG} $\parallel \drawSizedLine{CD,DE,EF}$ \inprop[prop:I.XXXI], касающуюся \drawSizedLine{EH}.\\
Проведем \drawSizedLine{FG}.

Поскольку $\drawSizedLine{AG} \parallel \drawSizedLine{CD,DE,EF}$ (постр.)\\
$\polygonACD =
\drawFromCurrentPicture[bottom][polygonGEF]{
startAutoLabeling;
draw byNamedPolygon (BEF, GBF);
stopAutoLabeling;
draw byNamedLineFull(G, G, 1, 1, 0)(EF);
}$ но $\polygonACD = \polygonBEF$

$\therefore \polygonBEF = \polygonGEF$, часть равна целому, что не имеет смысла.

$\therefore \drawSizedLine{AG} \nparallel \drawSizedLine{CD,DE,EF}$: и таким же образом можно показать, что никакая другая линия, кроме \drawSizedLine{AB} не $\parallel \drawSizedLine{CD,DE,EF}$: $\therefore \drawSizedLine{AB} \parallel \drawSizedLine{CD,DE,EF}$.
\stopCenterAlign

\qed
\stopProposition

\startProposition[title={Предложение XL. Теорема}, reference=prop:I.XLI]
\defineNewPicture[1/4]{
pair A, B, C, D, E, F, G, d;
A := (0, 0);
d := (2u, 0);
B := A shifted d;
C := B shifted (2u, 0);
D := A shifted (4/3u, -3u);
E := D shifted d;
F = whatever[B, E] = whatever[D, C];
G = whatever[A, E] = whatever[D, C];
draw byPolygon(A,B,F,G)(byyellow);
draw byPolygon(G,F,E)(byyellow);
draw byPolygon(A,G,D)(byblue);
draw byPolygon(D,E,G)(byblue);
draw byPolygon(C,F,E)(byred);
draw byLine(A, E, byred, 0, 0);
draw byLineFull(A, C, black, 1, 0)(D, E, 1, 1, 0);
draw byLineFull(D, E, black, 0, 0)(A, C, 1, 1, 0);
draw byLabelsOnPolygon(E, D, A, B, C)(0, 0);
}
\drawCurrentPictureInMargin
\problemNP[2]{Е}{сли}{параллелограмм \drawPolygon[bottom][polygonABED]{ABFG,GFE,AGD,DEG} и треугольник \drawPolygon[bottom][polygonCED]{GFE,DEG,CFE} стоят на одном основании \drawUnitLine{DE} и между теми же параллельными прямыми \drawUnitLine{AC} и \drawUnitLine{DE}, то параллелограмм вдвое больше треугольника.}

\startCenterAlign
Проведем диагональ \drawUnitLine{AE};

тогда $\drawPolygon[bottom][polygonAED]{AGD,DEG} = \polygonCED$ \inprop[prop:I.XXXVII]\\
$\polygonABED = \mbox{ дважды } \polygonAED$ \inprop[prop:I.XXXIV]

$\therefore \polygonABED = \mbox{ дважды } \polygonCED$.
\stopCenterAlign

\qed
\stopProposition

\startProposition[title={Предложение XLII. Задача}, reference=prop:I.XLII]
\defineNewPicture[1/4]{
pair A, B, C, D, E, F, G, H, I, J, d;
A := (0, 0);
d := (2u, 0);
B := A shifted d;
C := B shifted (2u, 0);
D := A shifted (u, -3u);
E := D shifted d;
F := (B--E) intersectionpoint (D--C);
G := 2[D, E];
H := B shifted (xpart(G)-xpart(E), -ypart(D)-u);
I := E shifted (xpart(G)-xpart(E), -ypart(D)-u);
J := D shifted (xpart(G)-xpart(E), -ypart(D)-u);
draw byPolygon(A,B,F,D)(byyellow);
draw byPolygon(D,E,F)(byyellow);
draw byPolygon(C,F,E)(byblue);
draw byPolygon(E,C,G)(black);
draw byAngleWithName(B, E, D, byblue, 0)(E);
draw byAngleWithName(H, I, J, byyellow, 0)(I);
angleStandalone.I := 1;
draw byLine(A, D, byred, 1, 0);
draw byLine(B, E, byred, 0, 0);
draw byLine(C, E, byyellow, 0, 0);
draw byLine(A, C, byblue, 0, 0);
draw byLine(D, E, black, 0, 0);
draw byLine(E, G, black, 1, 0);
draw byLabelsOnPolygon(G, E, D, A, B, C)(0, 0);
startAutoLabeling;
draw byNamedAngleSidesFull(I)();
stopAutoLabeling;
}
\drawCurrentPicture
\problemNP[2]{П}{остроить}{параллелограмм, равный данному треугольнику
\drawPolygon[bottom][polygonCDG]{DEF,CFE,ECG} и с данным углом \drawAngle{I}.}

\startCenterAlign
Сделаем $\drawUnitLine{DE} = \drawUnitLine{EG}$ \inprop[prop:I.X]\\
Проведем \drawUnitLine{CE}.

Сделаем $\drawAngle{E} = \drawAngle{I}$ \inprop[prop:I.XXIII]\\
Проведем $\left\{\eqalign{\drawUnitLine{AD} &\parallel \drawUnitLine{BE}\cr\drawUnitLine{AC} &\parallel \drawUnitLine{DE}\cr}\right\}$ \inprop[prop:I.XXXI]

$\drawPolygon[bottom][polygonABED]{ABFD,DEF}
= \mbox{ дважды }
\drawPolygon[bottom][polygonCED]{DEF,CFE}$ \inprop[prop:I.XLI]\\
но $\polygonCED =
\drawPolygon[bottom][polygonDCG]{ECG}$ \inprop[prop:I.XXXVIII]

$\therefore \polygonABED = \polygonCDG$.
\stopCenterAlign

\qed
\stopProposition

\startProposition[title={Предложение XLIII. Теорема}, reference=prop:I.XLIII]
\defineNewPicture[2/5]{
pair A, B, C, D, E, F, G, H, I, d[];
path q[];
d1 := (5/2u, 0);
d2 := (-u, -3u);
A := (0, 0);
B := A shifted d1;
C := A shifted d2;
D := C shifted d1;
E := 2/5[A, D];
q1 := (E shifted d1) -- (E shifted -d1);
q2 := (E shifted d2) -- (E shifted -d2);
F := q1 intersectionpoint (A--C);
G := q1 intersectionpoint (B--D);
H := q2 intersectionpoint (A--B);
I := q2 intersectionpoint (C--D);
draw byPolygon(A,E,H)(byyellow);
draw byPolygon(A,E,F)(byyellow);
draw byPolygon(H,B,G,E)(byblue);
draw byPolygon(F,C,I,E)(black);
draw byPolygon(I,D,E)(byred);
draw byPolygon(G,D,E)(byred);
draw byLabelsOnPolygon(C, F, A, H, B, G, D, I)(0, 0);
draw byLabelsOnPolygon(F, E, H)(2, 0);
}
\drawCurrentPictureInMargin
\problemNP{Д}{ополнения}{\drawPolygon[bottom][polygonHBGE]{HBGE} и \drawPolygon[bottom][polygonFCIE]{FCIE} параллелограммов на одной диагонали параллелограмма, равны между собой.}

\startCenterAlign
$\drawPolygon[bottom][polygonADC]{AEF,FCIE,IDE} = \drawPolygon[bottom][polygonABD]{AEH,HBGE,GDE}$ \inprop[prop:I.XXXIV]\\
и $\drawPolygon[bottom][polygonAEFpIDE]{AEF,IDE} = \drawPolygon[bottom][polygonAEHpGDE]{AEH,GDE}$ \inprop[prop:I.XXXIV]

$\therefore \polygonFCIE = \polygonHBGE$ \inax[ax:III]
\stopCenterAlign

\qed
\stopProposition

\startProposition[title={Предложение XLIV. Задача}, reference=prop:I.XLIV]
\defineNewPicture{
pair A, B, C, D, E, F, G, H, I, J, K, L, M, N, O, d[];
path q[];
d1 := (3u, 0);
d2 := (-3/2u, -3u);
d3 := (3/2u, 5/2u);
d4 := -d2 +1/2d1;
A := (0, 0);
B := A shifted d1;
C := A shifted d2;
D := C shifted d1;
E := 2/5[C, B];
q1 := (E shifted d1) -- (E shifted -d1);
q2 := (E shifted d2) -- (E shifted -d2);
F := q1 intersectionpoint (A--C);
G := q1 intersectionpoint (B--D);
H := q2 intersectionpoint (A--B);
I := q2 intersectionpoint (C--D);
J := A shifted d3;
K := J shifted (2(xpart(A)-xpart(H)), 0);
L := (xpart(1/3[J, K]), ypart(F)-ypart(A)+ypart(J));
M := A shifted d4;
N := F shifted d4;
O := E shifted d4;
draw byPolygon(J,K,L)(byred);
draw byPolygon(A,H,E,F)(byyellow);
draw byPolygon(E,G,D,I)(byblue);
draw byAngleWithName(A, F, E, byblue, 0)(F);
draw byAngleWithName(F, E, I, byred, 0)(E);
draw byAngleWithName(E, I, D, black, 0)(I);
draw byAngleWithName(M, N, O, byyellow, 0)(N);
angleStandalone.N := 1;
draw byLine(B, E, byred, 0, 0);
draw byLine(E, C, black, 0, 1);
byLineDefine(A, F, byred, 1, 0);
byLineDefine(F, C, black, 0, 1);
draw byLine(H, E, byblue, 1, 0);
byLineDefine(B, G, byyellow, 0, 0);
byLineDefine(A, H, byblue, 0, 0);
byLineDefine(H, B, black, 0, 0);
byLineDefine(F, E, black, 1, 0);
byLineDefine(E, G, black, 0, 0);
byLineDefine(C, D, byyellow, 1, 0);
draw byNamedLineSeq(0)(FE,EG,BG,HB,AH,AF,FC,CD);
draw byLabelsOnPolygon(K, J, L)(0, 0);
draw byLabelsOnPolygon(F, A, H, B, G, D, I, C)(0, 0);
draw byLabelsOnPolygon(F, E, H)(2, 0);
startAutoLabeling;
draw byNamedAngleSidesFull(N)();
stopAutoLabeling;
}
\drawCurrentPicture
\problemNP[2]{К}{ данной}{прямой линии \drawUnitLine{EG} приложить параллелограмм, равный данному треугольнику \drawPolygon[bottom][polygonJKL]{JKL}, и с углом, равным данному прямолинейному углу \drawAngle{N}.}

\startCenterAlign
Сделаем $\drawPolygon[bottom][polygonAHEF]{AHEF} = \polygonJKL$ с $\drawAngle{F} = \drawAngle{N}$ \inprop[prop:I.XLII]\\
и со стороной \drawUnitLine{FE} смежной и являющейся продолжением \drawUnitLine{EG}.

Продлим \drawUnitLine{AH} до $\drawUnitLine{BG} \parallel \drawUnitLine{HE}$\\
проведем \drawUnitLine{BE}, продлим ее, до продолжения \drawUnitLine{AF};\\ 
проведем $\drawUnitLine{CD} \parallel \drawUnitLine{FE,EG}$ до продолжения  \drawUnitLine{BG} и продлим \drawUnitLine{HE}.

$\polygonAHEF = \drawPolygon[bottom][polygonEGDI]{EGDI}$ \inprop[prop:I.XLIII]\\
но $\polygonAHEF = \polygonJKL$ (постр.)

$\therefore \polygonEGDI = \polygonJKL$;

и $\drawAngle{F} = \drawAngle{E} =\drawAngle{I} = \drawAngle{N}$ (\inpropL[prop:I.XXIX] и постр.)
\stopCenterAlign

\qed
\stopProposition

\startProposition[title={Предложение XLV. Задача}, reference=prop:I.XLV]
\defineNewPicture{
pair A, B, C, D, E, F, G, H, I, J, K, L, M, N, O, P, d[];
numeric a, h[], b[], s[];
a := 15;
A := (0, 0);
B := A shifted (0, 2u);
C := A shifted (4/3u, u);
D := A shifted (2u, -3/2u);
E := A shifted (-6/5u, -u);
b1 := arclength(B--C);
h1 := distanceToLine(A, B--C);
s1 := (b1 * h1)/2;
b2 := arclength(C--D);
h2 := distanceToLine(A, C--D);
s2 := (b2 * h2)/2;
b3 := arclength(D--E);
h3 := distanceToLine(A, D--E);
s3 := (b3 * h3)/2;
d1 := (0, ypart(D)-u);
d2 := (0, -b3/2) rotated -a;
d3 := (h3*(1/cosd(a)), 0);
d6 := (u, 0);
F := (-u, 0) shifted d1;
G := F shifted d3;
H := F shifted d2;
I := G shifted d2;
d4 := (2*(s2/b3)*(1/cosd(a)), 0);
J := G shifted d4;
K := J shifted d2;
d5 := (2*(s1/b3)*(1/cosd(a)), 0);
L := J shifted d5;
M := L shifted d2;
N := L shifted d6;
O := M shifted d6;
P := K shifted d6;
draw byPolygon(A,B,C)(byred);
draw byPolygon(A,C,D)(byyellow);
draw byPolygon(A,D,E)(byblue);
byLineDefine(A, C, byblue, 0, 0);
byLineDefine(A, D, byred, 0, 0);
draw byNamedLineSeq(0)(AC,AD);
draw byPolygon(F,G,I,H)(byblue);
draw byPolygon(G,J,K,I)(byyellow);
draw byPolygon(J,L,M,K)(byred);
draw byAngleWithName(G, I, H, byyellow, 0)(I);
draw byAngleWithName(J, K, I, black, 0)(K);
draw byAngleWithName(L, M, K, byblue, 0)(M);
draw byAngleWithName(N, O, P, byred, 0)(O);
angleStandalone.O := 1;
draw byLine(G, I, byred, 0, 0);
draw byLine(J, K, byblue, 0, 0);
draw byLabelsOnPolygon(A, B, C, D, E)(0, 0);
draw byLabelsOnPolygon(F, G, J, L, M, K, I, H)(0, 0);
startAutoLabeling;
draw byNamedAngleSidesFull(O)();
stopAutoLabeling;
}
\drawCurrentPictureInMargin
\problemNP[2]{П}{остроить}{параллелограмм равный данной прямолинейной фигуре \drawPolygon[middle][polygonABCDE]{ABC,ACD,ADE} в угле, равном данному прямолинейному углу \drawAngle{O}.}

\startCenterAlign
Проведем \drawUnitLine{AD} и \drawUnitLine{AC}, делящие прямолинейную фигуру на треугольники.

Построим $\drawPolygon{FGIH} = \drawPolygon{ADE}$\\
с $\drawAngle{I} = \drawAngle{O}$ \inprop[prop:I.XLII]\\
к \drawUnitLine{GI} приложим $\drawPolygon{GJKI} = \drawPolygon{ACD}$\\
с $\drawAngle{K} = \drawAngle{O}$ \inprop[prop:I.XLIV]\\
к \drawUnitLine{JK} приложим $\drawPolygon{JLMK} = \drawPolygon{ABC}$\\
с $\drawAngle{M} = \drawAngle{O}$ \inprop[prop:I.XLIV]

$\therefore \drawPolygon[middle][polygonFLMH]{FGIH,GJKI,JLMK} = \polygonABCDE$\\
и \polygonFLMH является параллелограммом. (\inpropL[prop:I.XXIX], \inpropL[prop:I.XIV], \inpropL[prop:I.XXX])\\
с $\drawAngle{M} = \drawAngle{O}$.
\stopCenterAlign

\qed
\stopProposition

\startProposition[title={Предложение XLVI. Задача}, reference=prop:I.XLVI]
\defineNewPicture{
pair A, B, C, D;
numeric d;
d := 7/2u;
A := (0, 0);
B := A shifted (d, 0);
C := A shifted (0, -d);
D := A shifted (d, -d);
draw byAngleWithName(B, A, C, black, 0)(A);
draw byAngleWithName(D, B, A, byblue, 0)(B);
draw byAngleWithName(C, D, B, byred, 0)(D);
draw byAngleWithName(A, C, D, byyellow, 0)(C);
byLineDefine(A, B, byred, 0, 0);
byLineDefine(B, D, byyellow, 0, 0);
byLineDefine(D, C, black, 0, 0);
byLineDefine(C, A, byblue, 0, 0);
draw byNamedLineSeq(0)(AB,BD,DC,CA);
draw byLabelsOnPolygon(A, B, D, C)(0, 0);
}
\drawCurrentPictureInMargin
\problemNP{Н}{а}{данной прямой \drawUnitLine{DC} построить квадрат.}

\startCenterAlign
Проведем $\drawUnitLine{CA} \perp \mbox{ and } = \drawUnitLine{DC}$ (\inpropL[prop:I.XI], \inpropL[prop:I.III])

Проведем $\drawUnitLine{AB} \parallel \drawUnitLine{DC}$, и касающуюся \drawUnitLine{BD} проведенную $\parallel \drawUnitLine{CA}$.

В
\drawFromCurrentPicture[bottom][polygonABDC]{
startTempAngleScale(angleScale*3/4);
draw byNamedAngle(A,B,C,D);
draw byNamedLineSeq(0)(AB,BD,DC,CA);
draw byLabelsOnPolygon(A, B, D, C)(0, 0);
stopTempAngleScale;
}
$\drawUnitLine{CA} = \drawUnitLine{DC}$ (постр.)\\
$\drawAngle{C} = \mbox{right angle}$ (постр.)

$\therefore \drawAngle{D} = \drawAngle{C} = \mbox{прямому углу}$ \inprop[prop:I.XXIX], и оставшиеся стороны и углы должны быть равны, \inprop[prop:I.XXXIV]

И $\therefore \polygonABDC$ является квадратом. \indef[def:XXVII]
\stopCenterAlign

\qed
\stopProposition

\startProposition[title={Предложение XLVII. Теорема}, reference=prop:I.XLVII]
\defineNewPicture[1/2]{
pair A, B, C, D, E, F, G, H, I, J, K, L, M, d[];
%byPointLabelDefine(A, "α");
%byPointLabelDefine(B, "β");
%byPointLabelDefine(C, "γ");
%byPointLabelDefine(D, "δ");
%byPointLabelDefine(E, "ε");
%byPointLabelDefine(F, "ζ");
%byPointLabelDefine(G, "η");
%byPointLabelDefine(H, "θ");
%byPointLabelDefine(I, "ι");
%byPointLabelDefine(J, "κ");
%byPointLabelDefine(K, "λ");
%byPointLabelDefine(L, "μ");
%byPointLabelDefine(M, "ν");
A := (0, 0);
B := A shifted (-7/10u, -8/7u);
C = whatever[A, A shifted ((A-B) rotated 90)] = whatever[B, B shifted dir(0)];
d1 := (B-A) rotated -90;
D := A shifted d1;
E := B shifted d1;
d2 := (A-C) rotated -90;
F := C shifted d2;
G := A shifted d2;
d3 := (C-B) rotated -90;
H := B shifted d3;
I := C shifted d3;
J = whatever[A, A shifted dir(90)];
J = whatever[B, C];
K = whatever[A, A shifted dir(90)];
K = whatever[H, I];
L = whatever[B, F];
L = whatever[A, C];
M = whatever[A, I];
M = whatever[B, C];
draw byPolygon(A,B,E,D)(black);
draw byPolygon(L,A,G,F)(byred);
draw byPolygon(C,L,F)(byred);
draw byPolygon(J,M,I,K)(byblue);
draw byPolygon (M,C,I)(byblue);
draw byPolygon(B,J,K,H)(byyellow);
draw byAngle(F, C, A, byyellow, 0);
draw byAngle(B, C, I, byyellow, 0);
draw byAngle(A, C, B, black, 0);
draw byLine(A, K, black, 1, 0);
draw byLineFull(B, F, black, 0, 0)(G, G, 1, 1, -1);
draw byLineFull(A, I, black, 0, 0)(K, K, 1, 1, 1);
byLineDefine(C, F, byblue, 1, 0);
byLineDefine(C, I, byred, 1, 0);
draw byNamedLineSeq(0)(CF,CI);
byLineDefine(A, B, byyellow, 0, 0);
byLineDefine(B, C, byred, 0, 0);
byLineDefine(C, A, byblue, 0, 0);
draw byNamedLineSeq(-1)(AB,BC,CA);
byLineDefineWithName (C, A, black, 0, 0)(CAb);
byLineStylize (M, M, 1, 0, -1) (CAb);
byLineDefineWithName (A, M, black, 0, 0)(AMb);
byLineStylize (C, C, 0, 1, -1) (AMb);
byLineDefineWithName (B, C, black, 0, 0)(BCb);
byLineStylize (L, L, 0, 1, -1) (BCb);
byLineDefineWithName (L, B, black, 0, 0)(BLb);
byLineStylize (C, C, 1, 0, -1) (BLb);
draw byLabelsOnPolygon(B, E, D, A, G, F, C, I, K, H)(0, 1);
draw byLabelsOnPolygon(A, J, C)(2, -1);
}
\drawCurrentPictureInMargin
\problemNP{В}{прямоугольном}{треугольнике \drawLine[bottom][triangleABC]{CA,BC,AB} квадрат гипотенузы \drawUnitLine{BC} равен сумме квадратов катетов \drawUnitLine{CA} и \drawUnitLine{AB}.}

\startCenterAlign
На \drawUnitLine{BC}, \drawUnitLine{CA}, \drawUnitLine{AB} построим квадраты, \inprop[prop:I.XLVI]

Проведем $\drawUnitLine{AK} \parallel \drawUnitLine{CI}$ \inprop[prop:I.XXXI]\\
также проведем \drawUnitLine{BF} and \drawUnitLine{AI}.\\
$\drawAngle{BCI} = \drawAngle{FCA}$,

К каждому добавим \drawAngle{ACB} $\therefore \drawAngle{BCI,ACB} = \drawAngle{FCA,ACB}$,\\
$\drawUnitLine{BC} = \drawUnitLine{CI}$ и $\drawUnitLine{CA} = \drawUnitLine{CF}$;

$\therefore
\drawFromCurrentPicture[middle][polygonAFC]{
draw byNamedPolygon(MCI);
draw byNamedAngle(ACB);
draw byNamedLine(CAb,AMb);
draw byLabelsOnPolygon(I,A,C)(1, 1);
}
=
\drawFromCurrentPicture[middle][polygonBLC]{
draw byNamedPolygon(CLF);
draw byNamedAngle(ACB);
draw byNamedLine(BCb,BLb);
draw byLabelsOnPolygon(B,F,C)(1, 1);
}
$.

Теперь, поскольку $\drawUnitLine{AB} \parallel \drawUnitLine{CF}$\\
$\drawPolygon[middle][polygonACFG]{LAGF,CLF} = \mbox{ дважды } \polygonBLC$,\\
и $\drawPolygon[middle][polygonJMCK]{JMIK,MCI} = \mbox{ дважды } \polygonAFC$;

$\therefore \polygonACFG = \polygonJMCK$.

Так же можно показать, что $\drawPolygon[middle][polygonABED]{ABED} = \drawPolygon[middle][polygonBJKH]{BJKH}$;

а значит $\drawPolygon[middle][polygonABEDpACFG]{ABED,LAGF,CLF} = \drawPolygon[middle][polygonBCIH]{JMIK,MCI,BJKH}$.

\stopCenterAlign

\qed
\stopProposition

\startProposition[title={Предложение XLVIII. Теорема}, reference=prop:I.XLVIII]
\defineNewPicture{
pair A, B, C, D;
numeric d;
d := 7/4u;
A := (0, 0);
B := A shifted (0, 3u);
C := A shifted (d, 0);
D := A shifted (-d, 0);
draw byAngle(B, A, C, byred, 0);
draw byAngle(D, A, B, byyellow, 0);
draw byLine(A, B, byblue, 0, 0);
byLineDefine(A, C, black, 0, 0);
byLineDefine(A, D, black, 1, 0);
byLineDefine(B, C, byred, 0, 0);
byLineDefine(B, D, byred, 1, 0);
draw byNamedLineSeq(0)(AC,AD,BD,BC);
draw byLabelsOnPolygon(D, B, C, A)(0, 0);
}
\drawCurrentPictureInMargin
\problemNP{Е}{сли}{в треугольнике квадрат одной стороны \drawUnitLine{BC} равен сумме квадратов двух других сторон \drawUnitLine{AB} и \drawUnitLine{AC}, то угол \drawAngle{DAB}, заключенный между этими двумя сторонами прямой.}

\startCenterAlign
Проведем $\drawUnitLine{AD} \perp \drawUnitLine{AB}$ и $= \drawUnitLine{AC}$ (\inpropL[prop:I.XI], \inpropL[prop:I.III])\\
также проведем \drawUnitLine{BD}.

Поскольку $\drawUnitLine{AD} = \drawUnitLine{AC}$ (постр.)\\
$\drawUnitLine{AD}^2 = \drawUnitLine{AC}^2$;

$\therefore \drawUnitLine{AD}^2 + \drawUnitLine{AB}^2 = \drawUnitLine{AC}^2 + \drawUnitLine{AB}^2$

но $\drawUnitLine{AD}^2 + \drawUnitLine{AB}^2 = \drawUnitLine{BD}^2$ \inprop[prop:I.XLVII],\\
и $\drawUnitLine{AC}^2 + \drawUnitLine{AB}^2 = \drawUnitLine{BC}^2$ (гип.)

$\therefore \drawUnitLine{BD}^2 = \drawUnitLine{BC}^2$,

$\therefore \drawUnitLine{BD} = \drawUnitLine{BC}$;

и $\therefore \drawAngle{DAB} = \drawAngle{BAC}$ \inprop[prop:I.VIII],

следовательно \drawAngle{BAC} прямой угол.
\stopCenterAlign

\qed

\stopProposition
\stopbook

\startbook[title={Книга 2}]
\startDefinition[title={Определение I},reference=def:II.I]

\defineNewPicture{
pair A, B, C, D;
numeric w, h;
w := 7/2u;
h := 3u;
A := (0, 0);
B := (w, 0);
C := (0, h);
D := (w, h);
draw byPolygon(A,B,D,C)(byblue);
byLineDefine(A, B, black, 0, 0);
byLineDefine(A, C, byred, 0, 0);
draw byNamedLineSeq(0)(AB,AC);
draw byLabelsOnPolygon(A, C, D, B)(0, 0);
}
\drawCurrentPictureInMargin
\problemNP{О}{ всяком}{прямоугольнике или прямоугольном параллелограмме говорят, что он заключается между любыми двумя своими смежными сторонами.}

Таким образом, про прямоугольный параллелограмм \drawPolygon{ABDC} можно сказать, что он заключен между сторонами \drawUnitLine{AB} и \drawUnitLine{AC}, что можно записать короче в виде $\drawUnitLine{AB} \cdot \drawUnitLine{AC}$.

Если смежные стороны равны, т. е. $\drawUnitLine{AB} = \drawUnitLine{AC}$, то $\drawUnitLine{AB} \cdot \drawUnitLine{AC}$, обозначает прямоугольник, заключенный между \drawUnitLine{AB} и \drawUnitLine{AC}, являющийся квадратом\\
и равный $\left\{\eqalign{
\drawUnitLine{AB} \cdot \drawUnitLine{AC}&\mbox{ или } \drawUnitLine{AC}^2\cr
\drawUnitLine{AB} \cdot \drawUnitLine{AC}&\mbox{ или } \drawUnitLine{AC}^2}\right.$

\stopDefinition

\vfill\pagebreak

\startDefinition[title={Определение II},reference=def:II.II]

\defineNewPicture{
pair A, B, C, D, E, F, G, H, I, d[];
d1 := (3u, 0);
d2 := (1/2u, 3u);
A := (0, 0);
B := A shifted d1;
C := A shifted d2;
D := A shifted d1 shifted d2;
E := 2/3[A, B];
F := 2/3[C, D];
G := 2/3[A, C];
H := 2/3[B, D];
I = whatever[E, F] = whatever[G, H];
draw byPolygon(A,E,I,G)(byblue);
draw byPolygon(E,B,H,I)(byyellow);
draw byPolygon(G,I,F,C)(byyellow);
draw byPolygon(I,H,D,F)(byred);
draw byLabelsOnPolygon(C, F, D, H, B, E, A, G)(0,0);
draw byLabelsOnPolygon(G, I, F)(2,0);
}
\drawCurrentPictureInMargin
\problemNP{В}{о}{ всяком параллелограмме фигура, образованная  одним из параллелограммов на на его диаметре вместе с двумя дополнениями, называется гномоном.}

Так, \drawPolygon{AEIG,EBHI,GIFC} и \drawPolygon{EBHI,GIFC,IHDF} являются гномонами.
\stopDefinition

\vfill\pagebreak

\startProposition[title={Предложение I. Теорема},reference=prop:II.I]
\defineNewPicture[1/4]{
pair B, C, D, E, G, H, K, L, M, N;
numeric w, h;
w := 7/2u;
h := 3u;
G := (0, 0);
H := (w, 0);
B := (0, h);
C := (w, h);
K := 2/5[G, H];
D := 2/5[B, C];
L := 3/4[G, H];
E := 3/4[B, C];
M := G shifted (0, -2/3u);
N := M shifted (h, 0);
draw byPolygon(G,K,D,B)(byyellow);
draw byPolygon(K,L,E,D)(byblue);
draw byPolygon(L,H,C,E)(byred);
draw byLine(K, D, black, 1, 0);
draw byLine(L, E, black, 1, 0);
byLineDefine(G, B, black, 0, 0);
draw byLineWithName(M, N, black, 0, 0)(A);
byLineDefine(H, C, black, 1, 0);
byLineDefine(G, K, byblue, 0, 0);
byLineDefine(K, L, byred, 0, 0);
byLineDefine(L, H, byyellow, 0, 0);
byLineDefine(B, D, byblue, 1, 0);
byLineDefine(D, E, byred, 1, 0);
byLineDefine(E, C, byyellow, 1, 0);
draw byNamedLineSeq(0)(GK,KL,LH,HC,EC,DE,BD,GB);
draw byLabelsOnPolygon(B, D, E, C, H, L, K, G)(0, 0);
draw byLabelLine(A);
}
\drawCurrentPictureInMargin
\problemNP{П}{рямоугольник,}{заключенный между двумя прямыми линиями, одна из которых рассечена на сколько угодно отрезков, равен сумме прямоугольников, заключенных между нерассеченной прямой и каждым из этих отрезков.\\
$\drawProportionalLine{GK,KL,LH} \cdot \drawProportionalLine{A} = \left\{\eqalign{
&\drawProportionalLine{A} \cdot \drawProportionalLine{GK}\cr
+&\drawProportionalLine{A} \cdot \drawProportionalLine{KL}\cr
+&\drawProportionalLine{A} \cdot \drawProportionalLine{LH}}\right.$}

\startCenterAlign
Проведем $\drawProportionalLine{GB} \perp \drawProportionalLine{GK,KL,LH} \mbox{ и } = \drawProportionalLine{A}$ (\inpropL[prop:I.II], \inpropL[prop:I.III]); достроим параллелограммы, то есть

проведем $\left\{\eqalign{
\drawProportionalLine{BD,DE,EC} &\parallel \drawProportionalLine{GK,KL,LH} \cr
\vcenter{
\nointerlineskip\hbox{\drawProportionalLine{KD}}
\nointerlineskip\hbox{\drawProportionalLine{LE}}
\nointerlineskip\hbox{\drawProportionalLine{HC}}} &\parallel \drawProportionalLine{GB}
}\right\}$ \inprop[prop:I.XXXI]

$\drawPolygon{GKDB,KLED,LHCE} =
\drawPolygon{GKDB} +
\drawPolygon{KLED} +
\drawPolygon{LHCE}$\\
$\drawPolygon{GKDB,KLED,LHCE} = \drawProportionalLine{GK,KL,LH} \cdot \drawProportionalLine{GB}$\\
$\polygonGKDB = \drawProportionalLine{GK} \cdot \drawProportionalLine{GB}$,
$\polygonKLED = \drawProportionalLine{KL} \cdot \drawProportionalLine{GB}$,\\
$\polygonLHCE = \drawProportionalLine{LH} \cdot \drawProportionalLine{GB}$

$\therefore \drawProportionalLine{GK,KL,LH} \cdot \drawProportionalLine{A} = \drawProportionalLine{GK} \cdot \drawProportionalLine{A} + \drawProportionalLine{KL} \cdot \drawProportionalLine{A} + \drawProportionalLine{LH} \cdot \drawProportionalLine{A}$.
\stopCenterAlign

\qed
\stopProposition

\startProposition[title={Предложение II. Теорема},reference=prop:II.II]
\defineNewPicture[1/4]{
pair A, B, C, D, E, F;
numeric w;
w := 7/2u;
A := (0, w);
B := (w, w);
C := 2/3[A, B];
D := (0, 0);
E := (w, 0);
F := 2/3[D, E];
draw byPolygon(A,C,F,D)(byred);
draw byPolygon(C,B,E,F)(byyellow);
draw byLine(C, F, black, 0, 0);
byLineDefine(A, D, black, 1, 0);
byLineDefine(B, E, black, 1, 0);
byLineDefine(A, C, byblue, 0, 0);
byLineDefine(C, B, byred, 0, 0);
draw byNamedLineSeq(0)(AD,AC,CB,BE);
draw byLabelsOnPolygon(A, C, B, E, F, D)(0, 0);
}
\drawCurrentPictureInMargin
\problemNP{Е}{сли}{прямая линия как-либо рассечена \drawProportionalLine{AC,CB}, квадрат всей линии равен сумме прямоугольников, заключенных между целой линией и каждой из ее частей.\\
$\drawProportionalLine{AC,CB}^2 = \left\{\eqalign{
& \drawProportionalLine{AC,CB} \cdot \drawProportionalLine{AC} \cr
+ & \drawProportionalLine{AC,CB} \cdot \drawProportionalLine{CB}
}\right.$
}

\startCenterAlign
Опишем \drawPolygon[bottom][polygonABED]{ACFD,CBEF} \inprop[prop:I.XLVI]

Проведем \drawProportionalLine{CF} параллельную \drawProportionalLine{AD} \inprop[prop:I.XXXI]

$\polygonABED = \drawProportionalLine{AC,CB}^2$

$\drawPolygon{ACFD} = \drawProportionalLine{CF} \cdot \drawProportionalLine{AC} = \drawProportionalLine{AC,CB} \cdot \drawProportionalLine{AC}$

$\drawPolygon{CBEF} = \drawProportionalLine{CF} \cdot \drawProportionalLine{CB} = \drawProportionalLine{AC,CB} \cdot \drawProportionalLine{CB}$

$\polygonABED = \drawPolygon{ACFD} + \drawPolygon{CBEF}$

$\therefore \drawProportionalLine{AC,CB}^2 = \drawProportionalLine{AC,CB} \cdot \drawProportionalLine{AC} + \drawProportionalLine{AC,CB} \cdot \drawProportionalLine{CB}$.
\stopCenterAlign

\qed
\stopProposition

\startProposition[title={Предложение III. Теорема},reference=prop:II.III]
\defineNewPicture{
pair A, B, C, D, E, F;
numeric w, h;
w := -4u;
h := 11/4u;
A := (0, h);
B := (w, h);
C := (w+h, h);
D := (w+h, 0);
E := (w, 0);
F := (0, 0);
draw byPolygon(A,C,D,F)(byyellow);
draw byPolygon(C,B,E,D)(byred);
byLineDefine(D, F, byred, 0, 0);
byLineDefine(B, C, byblue, 0, 0);
byLineDefine(C, D, byblue, 0, 0);
byLineDefine(D, E, byblue, 0, 0);
byLineDefine(E, B, byblue, 0, 0);
draw byNamedLineSeq(0)(CD,noLine,DF,DE,EB,BC);
draw byLabelsOnPolygon(B, C, A, F, D, E)(0, 0);
}
\drawCurrentPictureInMargin
\problemNP{Е}{сли}{прямая линия как-либо рассечена \drawProportionalLine{DE,DF}, то прямоугольник, заключенный между всей прямой и ее частью, равен квадрату этой части, вместе с прямоугольником, заключенным между частями.\\
$\drawProportionalLine{DE,DF} \cdot \drawProportionalLine{DE} = \drawProportionalLine{DE}^2 + \drawProportionalLine{DE} \cdot \drawProportionalLine{DF}$, или \\
$\drawProportionalLine{DE,DF} \cdot \drawProportionalLine{DF} = \drawProportionalLine{DF}^2 + \drawProportionalLine{DE} \cdot \drawProportionalLine{DF}$.}

\startCenterAlign
Опишем \drawPolygon{CBED} \inprop[prop:I.XLVI]

Опишем \drawPolygon{ACDF} \inprop[prop:I.XXXI]

Тогда $\drawPolygon[middle][polygonABEF]{ACDF,CBED} = \polygonCBED + \polygonACDF$, но\\
$\polygonABEF = \drawProportionalLine{DE,DF} \cdot \drawProportionalLine{DE}$ и\\
$\polygonCBED = \drawProportionalLine{DE}^2$, $\polygonACDF = \drawProportionalLine{DE} \cdot \drawProportionalLine{DF}$,

$\therefore \drawProportionalLine{DE,DF} \cdot \drawProportionalLine{DE} = \drawProportionalLine{DE}^2 + \drawProportionalLine{DE} \cdot \drawProportionalLine{DF}$:

Таким же образом можно показать, что $\drawProportionalLine{DE,DF} \cdot \drawProportionalLine{DF} = \drawProportionalLine{DF}^2 + \drawProportionalLine{DE} \cdot \drawProportionalLine{DF}$.
\stopCenterAlign

\qed
\stopProposition

\startProposition[title={Предложение IV. Теорема},reference=prop:II.IV]
\defineNewPicture[1/4]{
pair A, B, C, D, E, F, G, H, K;
numeric w;
w := 4u;
A := (0, w);
B := (w, w);
C :=2/3[A, B];
D := (0, 0);
E := (w, 0);
F := 2/3[D, E];
H := 2/3[D, A];
K := 2/3[E, B];
G = whatever[H, K] = whatever[F, C];
draw byPolygon(A,C,G,H)(byyellow);
draw byPolygon(G,K,E,F)(byyellow);
draw byPolygon(D,H,G)(byblue);
draw byPolygon(G,B,C)(byred);
draw byAngle(F, D, G, byyellow, 0);
draw byAngle(F, G, D, byred, 0);
draw byAngle(K, G, B, black, 0);
draw byAngle(G, B, K, byblue, 0);
draw byLine(H, G, byred, 1, 0);
draw byLine(G, K, byred, 0, 0);
draw byLine(C, G, byblue, 1, 0);
draw byLine(F, G, byblue, 0, 0);
byLineDefine(E, K, byblue, 0, 0);
byLineDefine(F, E, byred, 0, 0);
byLineDefine(D, F, byblue, 0, 0);
byLineDefine(K, B, byred, 0, 0);
byLineDefine(G, D, black, 0, 0);
byLineDefine(B, G, black, 1, 0);
draw byNamedLineSeq(-1)(BG,GD,DF,FE,EK,KB);
byLineDefine(A, D, black, 0, 0);
byLineStylize(B, E, 0, 0, -1)(AD);
byLineDefine(B, A, black, 0, 0);
byLineStylize(E, D, 0, 0, -1)(BA);
draw byLabelsOnPolygon(A, C, B, K, E, F, D, H)(0, 0);
draw byLabelsOnPolygon(H, G, C)(2, 0);
}
\drawCurrentPictureInMargin
\problemNP{Е}{сли}{прямая как-либо рассечена \drawProportionalLine{DF,FE}, квадрат всей прямой равен квадратам частей, с дважды взятым прямоугольником, заключенным между частями.\\
$\drawProportionalLine{DF,FE}^2 = \drawProportionalLine{DF}^2 + \drawProportionalLine{FE}^2 + \mbox{дважды} \drawProportionalLine{DF} \cdot \drawProportionalLine{FE}$
}

\startCenterAlign
Опишем \drawLine[bottom][squareABED]{AD,BA,KB,EK,FE,DF} \inprop[prop:I.XLVI]\\
проведем \drawProportionalLine{BG,GD} \inpost[post:I]\\
и $\left\{\eqalign{
\drawProportionalLine{FG,CG} &\parallel \drawProportionalLine{EK,KB} \cr
\drawProportionalLine{HG,GK} &\parallel \drawProportionalLine{DF,FE}
}\right\}$ \inprop[prop:I.XXXI]\\
$\drawAngle{GBK} = \drawAngle{FDG}$ \inprop[prop:I.V],\\
$\drawAngle{GBK} = \drawAngle{FGD}$ \inprop[prop:I.XXIX],\\
$\therefore \drawAngle{FDG} = \drawAngle{FGD}$\\
$\therefore$ согласно (\inpropL[prop:I.VI], \inpropL[prop:I.XXIX], \inpropL[prop:I.XXXIV])\\ $\drawFromCurrentPicture[middle][squareFGHD]{
draw byNamedPolygon(DHG);
draw byNamedLineFull(G, G, 1, 0, -1)(DF);
draw byNamedLineFull(D, D, 0, 1, -1)(FG);
draw byLabelsOnPolygon(D, H, G, F)(0, 0);
} \mbox{ является квадратом } = \drawProportionalLine{DF}^2$.\\
Аналогично \drawFromCurrentPicture[middle][squareKBCG]{
draw byNamedPolygon(GBC);
draw byNamedLineFull(B, B, 1, 0, -1)(GK);
draw byNamedLineFull(G, G, 0, 1, -1)(KB);
draw byLabelsOnPolygon(G, C, B, K)(0, 0);
} является квадратом $= \drawProportionalLine{GK}^2$,\\
$\drawPolygon[middle]{ACGH} = \drawPolygon[middle]{GKEF} = \drawProportionalLine{DF} \cdot \drawProportionalLine{GK}$ \inprop[prop:I.XLIII]\\
Но $\drawFromCurrentPicture[middle][squareABEDf]{
draw byNamedPolygon(GBC,DHG,ACGH,GKEF);
draw byNamedLineFull(G, G, 1, 0, -1)(DF);
draw byNamedLineFull(D, D, 0, 1, -1)(FG);
draw byNamedLineFull(B, B, 1, 0, -1)(GK);
draw byNamedLineFull(G, G, 0, 1, -1)(KB);
draw byLabelsOnPolygon(A, B, E, D)(0, 0);
} = \squareFGHD + \drawPolygon{ACGH} + \drawPolygon{GKEF} + \squareKBCG$,\\
$\therefore \drawProportionalLine{DF,FE}^2 = \drawProportionalLine{DF}^2 + \drawProportionalLine{FE}^2 + \mbox{ дважды } \drawProportionalLine{DF} \cdot \drawProportionalLine{FE}$
\stopCenterAlign

\qed
\stopProposition

\startProposition[title={Предложение V. Теорема},reference=prop:II.V]
\defineNewPicture[1/4]{
pair A, B, C, D, E, F, G, H, K, L, M;
numeric h;
h := 6u;
A := (0, -h);
B := (0, 0);
C := 1/2[A, B];
D := 2/5[B, C];
E := (1/2h, -1/2h);
F := (1/2h, 0);
G := (xpart(F), ypart(D));
H = whatever[D, G] = whatever[B, E];
K := (xpart(H), ypart(A));
L := (xpart(H), ypart(C));
M := (xpart(H), ypart(B));
draw byPolygon(B,D,H,M)(byblue);
draw byPolygon(D,C,L,H)(byyellow);
draw byPolygon(C,L,K,A)(black);
draw byPolygon(M,H,G,F)(byyellow);
draw byPolygon(H,L,E,G)(byred);
draw byLine(H, G, black, 1, 0);
draw byLine(D, H, byred, 0, 0);
byLineDefine(L, M, black, 1, 0);
byLineDefine(C, L, byred, 0, 0);
byLineDefine(B, D, byred, 0, 0);
byLineDefine(D, C, byblue, 0, 0);
byLineDefine(C, A, byyellow, 0, 0);
byLineDefine(E, B, black, 0, 0);
byLineDefine(A, K, byred, 1, 0);
byLineDefine(K, L, byyellow, 0, 0);
byLineDefine(L, E, byblue, 1, 0);
draw byNamedLineSeq(-1)(BD,DC,CA,AK,KL,LM,CL,LE,EB);
draw byLabelsOnPolygon(A, C, D, B, M, F, G, E, L, K)(0, 0);
draw byLabelsOnPolygon(M, H, G)(2, 0);
}
\drawCurrentPictureInMargin
\problemNP{Е}{сли}{прямая \drawProportionalLine{BD,DC,CA} рассечена на равные \drawProportionalLine{BD,DC} \drawProportionalLine{CA} и неравные \drawProportionalLine{BD} \drawProportionalLine{DC,CA} отрезки, прямоугольник, заключенный между неравными частями, вместе с квадратом отрезка между сечениями, равен квадрату половины всей прямой. \\
$\drawProportionalLine{BD} \cdot \drawProportionalLine{DC,CA} + \drawProportionalLine{DC}^2 = \drawProportionalLine{CA}^2 = \drawProportionalLine{BD,DC}^2$.
}

\startCenterAlign
Опишем \drawPolygon[middle][squareCBFE]{BDHM,DCLH,MHGF,HLEG} \inprop[prop:I.XLVI], проведем \drawProportionalLine{EB}\\
и $\left\{\eqalign{
\drawProportionalLine{DH,HG} & \parallel \drawProportionalLine{CL,LE} \cr
\drawProportionalLine{LM,KL} & \parallel \drawProportionalLine{BD,DC,CA} \cr
\drawProportionalLine{AK} & \parallel \drawProportionalLine{CL,LE}
}\right\}$ \inprop[prop:I.XXXI]\\
$\drawPolygon{CLKA} = \drawPolygon{DCLH,BDHM}$ \inprop[prop:I.XXXVI]\\
$\drawPolygon{MHGF} = \drawPolygon{DCLH}$ \inprop[prop:I.XLIII]\\
$\therefore \mbox{\inax[ax:II] } \drawPolygon[middle]{DCLH,BDHM,MHGF} = \drawFromCurrentPicture{
startGlobalRotation(90);
startAutoLabeling;
draw byNamedPolygon(DCLH,CLKA);
stopAutoLabeling;
stopGlobalRotation;
} = \drawProportionalLine{BD} \cdot \drawProportionalLine{DC,CA}$\\
но $\drawPolygon{HLEG} = \drawProportionalLine{DC}^2$ \inprop[prop:II.IV]\\
и $\squareCBFE = \drawProportionalLine{BD,DC}^2$ (постр.)\\
$\therefore \mbox{\inax[ax:II] } \squareCBFE = \drawFromCurrentPicture{
startGlobalRotation(90);
startAutoLabeling;
draw byNamedPolygon(DCLH,CLKA,HLEG);
stopAutoLabeling;
stopGlobalRotation;
}$

$\therefore \drawProportionalLine{BD} \cdot \drawProportionalLine{DC,CA} + \drawProportionalLine{DC}^2 = \drawProportionalLine{CA}^2 = \drawProportionalLine{BD,DC}^2$
\stopCenterAlign

\qed
\stopProposition

\startProposition[title={Предложение VI. Теорема},reference=prop:II.VI]
\defineNewPicture[1/4]{
pair A, B, C, D, E, F, G, H, K, L, M;
numeric h, s;
h := 5u;
s := 2/5h;
A := (0, h);
B := (0, 0);
C := A shifted (0, -s);
D := A shifted (0, -2s);
E := (-h+s, h-s);
F := (-h+s, 0);
G := (xpart(F), ypart(D));
H = whatever[D, G] = whatever[B, E];
K := (xpart(H), ypart(A));
L := (xpart(H), ypart(C));
M := (xpart(H), ypart(B));
draw byPolygon(B,D,H,M)(byblue);
draw byPolygon(D,C,L,H)(byyellow);
draw byPolygon(C,L,K,A)(black);
draw byPolygon(M,H,G,F)(byyellow);
draw byPolygon(H,L,E,G)(byred);
draw byLine(H, G, byblue, 1, 0);
draw byLine(D, H, byred, 0, 0);
byLineDefine(L, M, black, 1, 0);
byLineDefine(C, L, byred, 0, 0);
byLineDefine(B, D, byred, 0, 0);
byLineDefine(D, C, byblue, 0, 0);
byLineDefine(C, A, byyellow, 0, 0);
byLineDefine(E, B, black, 0, 0);
byLineDefine(A, K, byred, 1, 0);
byLineDefine(K, L, byyellow, 0, 0);
byLineDefine(L, E, black, 1, 0);
draw byNamedLineSeq(-1)(BD,DC,CA,AK,KL,LM,CL,LE,EB);
draw byLabelsOnPolygon(F, G, E, L, K, A, C, D, B, M)(0, 0);
draw byLabelsOnPolygon(L, H, D)(2, -1);
}
\drawCurrentPictureInMargin
\problemNP{Е}{сли}{прямая рассечена пополам \drawProportionalLine{DC,CA} и продлена до любой точки \drawProportionalLine{BD,DC,CA}, прямоугольник, заключенный между всей прямой и продленной частью, вместе с квадратом половины исходной линии, равен квадрату линии, составленной из половины исходной линии и продленной части.\\
$\drawProportionalLine{BD,DC,CA} \cdot \drawProportionalLine{BD} + \drawProportionalLine{DC}^2 = \drawProportionalLine{BD,DC}^2$
}

\startCenterAlign
Опишем \drawPolygon[middle][squareCBFE]{BDHM,DCLH,MHGF,HLEG} \inprop[prop:I.XLVI], проведем \drawProportionalLine{EB}

и $\left\{\eqalign{
\drawProportionalLine{HG,DH} & \parallel \drawProportionalLine{LE,CL} \cr
\drawProportionalLine{LM,KL} & \parallel \drawProportionalLine{BD,DC,CA} \cr
\drawProportionalLine{AK} & \parallel \drawProportionalLine{CL,LE}
}\right\}$ \inprop[prop:I.XXXI]

$\drawPolygon{MHGF} = \drawPolygon{DCLH} = \drawPolygon{CLKA}$ (\inpropL[prop:I.XXXVI], \inpropL[prop:I.XLIII])\\
$\therefore \drawPolygon{BDHM,DCLH,MHGF} = \drawPolygon{BDHM,DCLH,CLKA} = \drawProportionalLine{BD} \cdot \drawProportionalLine{BD,DC,CA}$;\\
но $\drawPolygon{HLEG} = \drawProportionalLine{DC}^2$ \inprop[prop:II.IV]\\
$\therefore \squareCBFE = \drawProportionalLine{HG,DH}^2 = \drawPolygon{BDHM,DCLH,CLKA,HLEG}$ (постр., \inaxL[ax:II])

$\therefore \drawProportionalLine{BD,DC,CA} \cdot \drawProportionalLine{BD} + \drawProportionalLine{DC}^2 = \drawProportionalLine{BD,DC}^2$
\stopCenterAlign

\qed
\stopProposition

\startProposition[title={Предложение VII. Теорема},reference=prop:II.VII]
\defineNewPicture[1/4]{
pair A, B, C, D, E, F, G, H, N;
numeric w;
w := 7/2u;
A := (0, w);
B := (w, w);
C := 3/5[A, B];
D := (0, 0);
E := (w, 0);
N := 3/5[D, E];
F := 3/5[E, B];
G = whatever[D, B] = whatever[N, C];
H := whatever[A, D] = whatever[F, G];
draw byPolygon(D,N,G,H)(byred);
draw byPolygon(N,E,F,G)(black);
draw byPolygon(H,G,C,A)(byyellow);
draw byPolygon(G,F,B,C)(byblue);
draw byLine(G, N, byblue, 0, 0);
draw byLine(G, F, byred, 0, 0);
draw byLine(G, H, black, 1, 0);
draw byLine(G, C, black, 1, 0);
byLineDefine(B, D, black, 0, 0);
byLineDefine(D, N, byblue, 0, 0);
byLineDefine(N, E, byred, 0, 0);
byLineDefine(E, B, byyellow, 0, 0);
draw byNamedLineSeq(-1)(BD,DN,NE,EB);
draw byLabelsOnPolygon(D, H, A, C, B, F, E, N)(0, 0);
draw byLabelsOnPolygon(H, G, C)(2, 0);
}
\drawCurrentPictureInMargin
\problemNP{Е}{сли}{прямая как-либо рассечена \drawProportionalLine{DN,NE}, то вместе квадрат всей прямой и одной из ее частей равны дважды взятому прямоугольнику, заключенному между всей прямой и этой ее частью, вместе с квадратом другой части.\\
$\drawProportionalLine{DN,NE}^2 + \drawProportionalLine{NE}^2 = 2\drawProportionalLine{DN,NE} \cdot \drawProportionalLine{NE} + \drawProportionalLine{DN}^2$
}

\startCenterAlign
Опишем \drawPolygon[middle][squareABED]{DNGH,NEFG,HGCA,GFBC}. \inprop[prop:I.XLVI], проведем \drawProportionalLine{BD} \inpost[post:I],\\
и $\left\{\eqalign{
\drawProportionalLine{GN,GC} &\parallel \drawProportionalLine{EB} \cr
\drawProportionalLine{GH,GF} &\parallel \drawProportionalLine{DN,NE}
}\right\}$\\
$\drawPolygon{HGCA} = \drawPolygon{NEFG}$ \inprop[prop:I.XLIII],\\
добавим $\drawPolygon{GFBC} = \drawProportionalLine{NE}^2$ к обоим \inprop[prop:II.IV]

$\drawPolygon{HGCA,GFBC} = \drawPolygon{NEFG,GFBC} = \drawProportionalLine{DN,NE} \cdot \drawProportionalLine{NE}$\\
$\drawPolygon{DNGH} = \drawProportionalLine{DN}^2$ \inprop[prop:II.IV]\\
$\drawPolygon{HGCA,GFBC} + \drawPolygon{NEFG,GFBC} + \drawPolygon{DNGH} = 2\drawProportionalLine{DN,NE} \cdot \drawProportionalLine{NE} + \drawProportionalLine{DN}^2 = \squareABED + \drawPolygon{GFBC}$;

$\drawProportionalLine{DN,NE}^2 + \drawProportionalLine{NE}^2 = 2\drawProportionalLine{DN,NE} \cdot \drawProportionalLine{NE} + \drawProportionalLine{DN}^2$
\stopCenterAlign

\qed
\stopProposition

\startProposition[title={Предложение VIII. Теорема},reference=prop:II.VIII]
\defineNewPicture{
pair A, B, C, D, E, F, G, H, K, L, M, N, O, P, Q, R;
numeric w, d;
w := 7/2u;
d := u;
A := (0, w + d);
B := (w, w + d);
C := (w - d, w + d);
D := (w + d, w + d);
E := (0, 0);
F := (w + d, 0);
G := (w - d, w);
H := (w - d, 0);
K := (w, w);
L := (w, 0);
M := (0, w);
N := (w + d, w);
O := (0, w - d);
P := (w + d, w - d);
Q := (w - d, w - d);
R := (w, w - d);
draw byLine(C, Q, byblue, 1, 0);
draw byLine(B, R, black, 1, 0);
draw byLine(Q, H, byblue, 0, 0);
draw byLine(R, L, byblue, 0, 0);
draw byLine(M, G, black, 1, 0);
draw byLine(O, Q, byred, 1, 0);
draw byLine(G, N, byred, 0, 0);
draw byLine(Q, P, byred, 0, 0);
draw byLine(D, E, black, 0, 0);
byLineDefine(E, H, byblue, 0, 0);
byLineDefine(H, L, byred, 0, 0);
byLineDefine(L, F, byyellow, 0, 0);
byLineDefine(F, P, byblue, 0, 0);
byLineDefine(P, N, byyellow, 0, 0);
byLineDefine(N, D, byred, 0, 0);
byLineDefine(A, D, black, 0, 0);
byLineDefine(E, A, black, 0, 0);
draw byNamedLineSeq(0)(EH,HL,LF,FP,PN,ND,AD,EA);
byLineDefine(D, F, black, 0, 0);
byLineDefine(B, L, black, 0, 1);
byLineDefine(C, H, black, 0, 1);
byLineDefine(M, N, black, 0, 1);
byLineDefine(O, P, black, 0, 1);
draw byLabelsOnPolygon(A, C, B, D, N, P, F, L, H, E, O, M)(0, 0);
}
\drawCurrentPicture
\problemNP{Е}{сли}{прямая как-либо рассечена \drawProportionalLine{EH,HL}, квадрат всей прямой вместе с любой ее частью равен четырежды прямоугольнику, заключенному между всей прямой и этой ее частью, вместе с квадратом другой части. \\
$\drawProportionalLine{EH,HL,LF}^2 = 4 \cdot \drawProportionalLine{EH,HL} \cdot \drawProportionalLine{HL} + \drawProportionalLine{EH}^2$
}

\startCenterAlign
Продлим \drawProportionalLine{EH,HL} и сделаем $\drawProportionalLine{LF} = \drawProportionalLine{HL}$

Построим \drawFromCurrentPicture{
draw byNamedLine(BL,CH,MN,OP);
startAutoLabeling;
draw byNamedLineSeq(0)(LF,HL,EH,EA,AD,DF);
stopAutoLabeling;
} \inprop[prop:I.XLVI];\\
проведем \drawProportionalLine{DE},

$\left.\eqalign{
\left.\eqalign{
\vcenter{
\nointerlineskip\hbox{\drawProportionalLine{CQ,QH}}
\nointerlineskip\hbox{\drawProportionalLine{BR,RL}}
}
}\right\} & \parallel \drawProportionalLine{FP,PN,ND}\cr
\left.\eqalign{
\vcenter{
\nointerlineskip\hbox{\drawProportionalLine{OQ,QP}}
\nointerlineskip\hbox{\drawProportionalLine{MG,GN}}
}
}\right\} & \parallel \drawProportionalLine{EH,HL,LF}\cr
}\right\}$ \inprop[prop:I.XXXI]

$\drawProportionalLine{EH,HL,LF}^2 = \drawProportionalLine{LF}^2 + \drawProportionalLine{EH,HL}^2 + 2 \cdot \drawProportionalLine{EH,HL} \cdot \drawProportionalLine{LF}$ \inprop[prop:II.IV]\\
но $\drawProportionalLine{HL}^2 + \drawProportionalLine{EH,HL}^2 = 2 \cdot \drawProportionalLine{EH,HL} \cdot \drawProportionalLine{HL} + \drawProportionalLine{EH}^2$ \inprop[prop:II.VII]

$\therefore \drawProportionalLine{EH,HL,LF}^2 = 4 \cdot \drawProportionalLine{EH,HL} \cdot \drawProportionalLine{HL} + \drawProportionalLine{EH}^2$
\stopCenterAlign

\qed
\stopProposition

\startProposition[title={Предложение IX. Теорема},reference=prop:II.IX]
\defineNewPicture[1/4]{
pair A, B, C, D, E, F, G, H;
numeric w;
w := 5u;
A := (-1/2w, 0);
B := (1/2w, 0);
C := (0, 0);
D := (1/5w, 0);
E := (0, 1/2w);
F = whatever[E, B] = (xpart(D), whatever);
G = whatever[E, C] = (whatever, ypart(F));
H := 1/2[E, F];
draw byAngleWithName(E, A, B, byyellow, 0)(A);
draw byAngleWithName(A, B, E, byblue, 0)(B);
draw byAngle(A, E, C, byyellow, 0);
draw byAngle(C, E, B, byred, 0);
draw byAngle(E, F, G, byred, 0);
draw byAngle(D, F, B, black, 0);
draw byLine(A, F, black, 0, 0);
draw byLine(F, D, byred, 1, 0);
draw byLine(G, F, byyellow, 0, -1);
draw byLine(C, G, byblue, 0, 0);
draw byLine(G, E, byblue, 1, 0);
byLineDefine(A, C, byblue, 0, 0);
byLineDefine(C, D, byyellow, 0, 0);
byLineDefine(D, B, byred, 0, 0);
byLineDefine(F, B, byyellow, 1, 0);
byLineDefine(H, F, black, 0, 0);
byLineDefine(E, H, black, 1, 0);
byLineDefine(A, E, black, 1, 0);
draw byNamedLineSeq(0)(AC,CD,DB,FB,HF,EH,AE);
draw byLabelsOnPolygon(E, F, B, D, C, A)(0, 0);
draw byLabelsOnPolygon(C, G, E)(2, 0);
}
\drawCurrentPictureInMargin
\problemNP{Е}{сли}{прямая рассечена на равные \drawProportionalLine{AC} \drawProportionalLine{CD,DB} и неравные \drawProportionalLine{AC,CD} \drawProportionalLine{DB} части, квадраты неравных частей вместе вдвое больше квадрата на половине, вместе с квадратом отрезка между сечениями.\\
$\drawProportionalLine{AC,CD}^2 + \drawProportionalLine{DB}^2 = 2 \cdot \drawProportionalLine{AC}^2 + 2 \cdot \drawProportionalLine{CD}^2$
}

\startCenterAlign
Сделаем $\drawProportionalLine{CG,GE} \perp \mbox{ и } = \drawProportionalLine{AC} \mbox{ или } \drawProportionalLine{CD,DB}$,\\
Проведем \drawProportionalLine{AE} и \drawProportionalLine{EH,HF,FB},\\
$\drawProportionalLine{FD} \parallel \drawProportionalLine{CG,GE}$, $ \drawProportionalLine{GF} \parallel \drawProportionalLine{CD,DB}$ и проведем \drawProportionalLine{AF}.

$\drawAngle{A} = \drawAngle{AEC}$ \inprop[prop:I.V] $= \frac{1}{2}\drawRightAngle$ \inprop[prop:I.XXXII]\\
$\drawAngle{B} = \drawAngle{DFB}$ \inprop[prop:I.V] $= \frac{1}{2}\drawRightAngle$ \inprop[prop:I.XXXII]\\
$\therefore \drawAngle{AEC,CEB} = \drawRightAngle$.

$\drawAngle{B} = \drawAngle{CEB} = \drawAngle{EFG} = \drawAngle{DFB}$ (\inpropL[prop:I.V], \inpropL[prop:I.XXIX]).\\
Значит, $\drawProportionalLine{FD} = \drawProportionalLine{DB}$, $\drawProportionalLine{GE} = \drawProportionalLine{GF} = \drawProportionalLine{CD}$ (\inpropL[prop:I.VI], \inpropL[prop:I.XXXIV])

$\drawProportionalLine{AF}^2 = \left\{\eqalign{
&\drawProportionalLine{AC,CD}^2 + \drawProportionalLine{FD}^2 \mbox{, или } + \drawProportionalLine{DB}^2 \cr
&= \left\{\eqalign{
&= \drawProportionalLine{AE}^2 + \drawProportionalLine{EH,HF}^2 \cr
&= 2 \cdot \drawProportionalLine{AC}^2 + 2 \cdot \drawProportionalLine{CD}^2
}\right. \mbox{ \inprop[prop:I.XLVII]}
}\right.$

$\therefore \drawProportionalLine{AC,CD}^2 + \drawProportionalLine{DB}^2 = 2 \cdot \drawProportionalLine{AC}^2 + 2 \cdot \drawProportionalLine{CD}^2$
\stopCenterAlign

\qed
\stopProposition

\startProposition[title={Предложение X. Теорема},reference=prop:II.X]
\defineNewPicture[1/4]{
pair A, B, C, D, E, F, G;
numeric w;
w := 7/2u;
A := (0, 0);
B := (w, 0);
C := 1/2[A, B];
D := 4/3[A, B];
E := (w/2, w/2);
F := (xpart(D), ypart(E));
G = whatever[E, B] = whatever[F, D];
draw byAngleWithName(E, A, C, black, 0)(A);
draw byAngle(C, E, A, byyellow, 0);
draw byAngle(B, E, C, byyellow, 0);
draw byAngle(F, E, B, byblue, 0);
draw byAngle(C, B, E, byred, 0);
draw byAngle(D, B, G, byred, 0);
draw byAngleWithName(B, G, D, byblue, 0)(G);
draw byLine(C, E, byred, 0, -1);
draw byLine(A, C, byred, 0, 0);
draw byLine(C, B, byyellow, 0, 0);
draw byLine(B, D, byblue, 0, 0);
draw byLine(E, B, black, 0, 0);
draw byLine(B, G, black, 1, 0);
byLineDefine(E, F, byyellow, 1, 0);
byLineDefine(F, D, byred, 0, 0);
byLineDefine(D, G, byred, 1, 0);
byLineDefine(A, G, black, 0, 0);
byLineDefine(A, E, byblue, 1, 0);
draw byNamedLineSeq(0)(EF,FD,DG,AG,AE);
draw byLabelsOnPolygon(A, E, F, D, G)(0, 0);
draw byLabelsOnPolygon(G, B, C, A)(2, 0);
}
\drawCurrentPictureInMargin
\problemNP{Е}{сли}{прямая линия \drawProportionalLine{AC,CB} рассечена пополам и продлена до любой точки \drawProportionalLine{AC,CB,BD}, квадрат всей линии вместе с квадратом продленной части равны дважды квадрату половины исходной линии вместе с квадратом половины вместе с продленной частью.\\
$\drawProportionalLine{AC,CB,BD}^2 + \drawProportionalLine{BD}^2 = 2 \cdot \drawProportionalLine{CB}^2 + 2 \cdot \drawProportionalLine{CB,BD}^2$
}

\startCenterAlign
Сделаем $\drawProportionalLine{CE} \perp \mbox{ и } = \drawProportionalLine{AC} \mbox{ или } \drawProportionalLine{CB}$,

проведем \drawProportionalLine{AE} и \drawProportionalLine{EB,BG},

и $\left\{\eqalign{
\drawProportionalLine{FD,DG} & \parallel \drawProportionalLine{CE} \cr
\drawProportionalLine{EF} & \parallel \drawProportionalLine{CB,BD}
}\right\}$ \inprop[prop:I.XXXI]

также проведем \drawProportionalLine{AG}.

$\drawAngle{A} = \drawAngle{CEA}$ \inprop[prop:I.V] $= \frac{1}{2}\drawRightAngle$ \inprop[prop:I.XXXII]\\
$\drawAngle{CBE} = \drawAngle{BEC}$ \inprop[prop:I.V] $= \frac{1}{2}\drawRightAngle$ \inprop[prop:I.XXXII]\\
$\therefore \drawAngle{CEA,BEC} = \drawRightAngle$.

$\drawAngle{A} = \drawAngle{CBE} = \drawAngle{BEC} = \drawAngle{FEB} = \drawAngle{G} = \frac{1}{2}\drawRightAngle$ (\inpropL[prop:I.V], \inpropN[prop:I.XXXII], \inpropN[prop:I.XXIX], \inpropN[prop:I.XXXIV]),

и $\drawProportionalLine{BD} = \drawProportionalLine{DG}$, $\drawProportionalLine{CB,BD} = \drawProportionalLine{EF} = \drawProportionalLine{FD,DG}$, (\inpropL[prop:I.VI], \inpropN[prop:I.XXXIV]).

Тогда, согласно \inprop[prop:I.XLVII]\\
$\drawProportionalLine{AG}^2 = \left\{\eqalign{
& \drawProportionalLine{AC,CB,BD}^2 + \drawProportionalLine{DG}^2 \mbox{ or } \drawProportionalLine{BD}^2 \cr
& \left\{\eqalign{
& + \drawProportionalLine{AE}^2 = 2 \cdot \drawProportionalLine{AC}^2 \cr
& + \drawProportionalLine{EB,BG}^2 = 2\cdot \drawProportionalLine{EF}^2
}\right.
}\right.$

$\therefore \drawProportionalLine{AC,CB,BD}^2 + \drawProportionalLine{BD}^2 = 2 \cdot \drawProportionalLine{CB}^2 + 2 \cdot \drawProportionalLine{CB,BD}^2$
\stopCenterAlign

\qed
\stopProposition

\startProposition[title={Предложение XI. Задача},reference=prop:II.XI]
\defineNewPicture{
pair A, B, C, D, E, F, G, H, K;
numeric w;
w := 7/2u;
A := (0, 0);
B := (w, 0);
C := (0, w);
D := (w, w);
E := 1/2[A, C];
F := E shifted (0, -abs(E-B));
G := F shifted (abs(F-A), 0);
H = whatever[A, B] = (xpart(G), whatever);
K = whatever[G, H] = whatever[C, D];
draw byPolygon(A,H,K,C)(byyellow);
draw byPolygon(H,B,D,K)(byblue);
draw byPolygon(A,H,G,F)(byblue);
byLineDefine(K, G, black, 1, 0);
byLineDefine(A, H, byred, 0, 0);
byLineDefine(H, B, byred, 1, 0);
byLineDefine(B, E, black, 0, 0);
byLineDefine(C, E, byblue, 1, 0);
byLineDefine(E, A, byblue, 0, 0);
byLineDefine(A, F, byyellow, 0, 0);
draw byNamedLineSeq(1)(BE,HB,AH);
draw byNamedLine(CE,EA,AF,KG);
draw byLabelsOnPolygon(F, A, E, C, K, D, B, H, G)(0, 0);
}
\drawCurrentPictureInMargin
\problemNP{Д}{анную}{прямую \drawProportionalLine{AH,HB} рассечь так, чтобы прямоугольник, заключенный между всей прямой и одной из ее частей был равен квадрату другой части.\\
$\drawProportionalLine{AH,HB} \cdot \drawProportionalLine{HB} = \drawProportionalLine{AH}^2$
}

\startCenterAlign
Опишем \drawPolygon{AHKC,HBDK} \inprop[prop:I.XLVI],\\
сделаем $\drawProportionalLine{EA} = \drawProportionalLine{CE}$ \inprop[prop:I.X],\\
проведем \drawProportionalLine{BE},\\
возьмем $\drawProportionalLine{EA,AF} = \drawProportionalLine{BE}$ \inprop[prop:I.III],\\
на \drawProportionalLine{AF} опишем \drawPolygon{AHGF} \inprop[prop:I.XLVI],\\
Продлим \drawProportionalLine{KG} \inpost[post:II].

Тогда \inprop[prop:II.VI] $\drawProportionalLine{CE,EA,AF} \cdot \drawProportionalLine{AF} + \drawProportionalLine{EA}^2 = \drawProportionalLine{EA,AF}^2 = \drawProportionalLine{BE}^2 = \drawProportionalLine{AH,HB}^2 + \drawProportionalLine{EA}^2 \therefore \drawProportionalLine{CE,EA,AF} \cdot \drawProportionalLine{AF} = \drawProportionalLine{AH,HB}^2$, или\\
$\drawPolygon{AHKC,AHGF} = \drawPolygon{AHKC,HBDK} \therefore \drawPolygon{AHGF} = \drawPolygon{HBDK}$

$\therefore \drawProportionalLine{AH,HB} \cdot \drawProportionalLine{HB} = \drawProportionalLine{AH}^2$

\stopCenterAlign

\qed
\stopProposition

\startProposition[title={Предложение XII. Теорема},reference=prop:II.XII]
\defineNewPicture[1/4]{
pair A, B, C, D;
numeric w, h;
w := 7/2u;
h := 3u;
A := (3/5w, 0);
B := (w, h);
C := (0, 0);
D := (w, 0);
draw byLine(B, A, byred, 0, 0);
byLineDefine(C, A, black, 0, 0);
byLineDefine(A, D, black, 1, 0);
byLineDefine(D, B, byyellow, 0, 0);
byLineDefine(B, C, byblue, 0, 0);
draw byNamedLineSeq(0)(DB,AD,CA,BC);
draw byLabelsOnPolygon(C, B, D, A)(0, 0);
}
\drawCurrentPictureInMargin
\problemNP{В}{о}{всяком тупоугольном треугольнике квадрат на стороне, стягивающей тупой угол, больше суммы квадратов на сторонах, содержащих тупой угол, на дважды прямоугольник, заключенный между какой-либо из этих сторон и продолжением этой стороны от тупого угла до перпендикуляра, падающего от противоположного острого угла.\\
$\drawProportionalLine{BC}^2 > \drawProportionalLine{CA}^2 + \drawProportionalLine{BA}^2$ на $2 \cdot \drawProportionalLine{CA} \cdot \drawProportionalLine{AD}$
}

\startCenterAlign
Согласно \inpropL[prop:II.IV]\\
$\drawProportionalLine{CA,AD}^2 = \drawProportionalLine{CA}^2 + \drawProportionalLine{AD}^2 + 2 \cdot \drawProportionalLine{CA} \cdot \drawProportionalLine{AD}^2$:

добавим $\drawProportionalLine{DB}^2$ к обоим\\
$\drawProportionalLine{CA,AD}^2 + \drawProportionalLine{DB}^2 = \drawProportionalLine{BC}^2$ \inprop[prop:I.XLVII]\\
$= 2 \cdot \drawProportionalLine{CA} \cdot \drawProportionalLine{AD} + \drawProportionalLine{CA}^2 + \left\{\eqalign{
&\drawProportionalLine{AD}^2 \cr
&\drawProportionalLine{DB}^2
}\right\}$ или\\
$+ \drawProportionalLine{BA}^2$ \inprop[prop:I.XLVII].

$\therefore \drawProportionalLine{BC}^2 = 2 \cdot \drawProportionalLine{CA} \cdot \drawProportionalLine{AD} + \drawProportionalLine{CA}^2 + \drawProportionalLine{BA}^2$: значит, $ \drawProportionalLine{BC}^2 > \drawProportionalLine{CA}^2 + \drawProportionalLine{BA}^2$ на $2 \cdot \drawProportionalLine{CA} \cdot \drawProportionalLine{AD}^2$
\stopCenterAlign

\qed
\stopProposition

\startProposition[title={Предложение XIII. Теорема},reference=prop:II.XIII]
\defineNewPicture{
pair A,B,C,D,E,F,G,H, d;
numeric w, h;
w := 3u;
h := 3u;
A := (2/5w, h);
B:= (0, 0);
C := (w, 0);
D = whatever[B, C] = (xpart(A), whatever);
d := (0, -h -4/3u);
E := (w, h) shifted d;
F := (0, 0) shifted d;
G := (2/5w, 0) shifted d;
H = whatever[F, G] = (xpart(E), whatever);
draw byLine(A, D, byyellow, 0, 0);
byLineDefine(A, B, byred, 0, 0);
byLineDefine(B, D, black, 0, 0);
byLineDefine(D, C, black, 1, 0);
byLineDefine(C, A, byblue, 0, 0);
draw byNamedLineSeq(0)(AB,BD,DC,CA);
draw byLine(E, G, byblue, 0, 0);
byLineDefine(E, F, byred, 0, 0);
byLineDefine(F, G, black, 0, 0);
byLineDefine(G, H, black, 1, 0);
byLineDefine(H, E, byyellow, 0, 0);
draw byNamedLineSeq(0)(EF,FG,GH,HE);
label.top("Первый случай", (xpart(1/2[B, C]), ypart(A) + 1/4u));
label.top("Второй случай", (xpart(1/2[F, H]), ypart(E)));
draw byLabelsOnPolygon(B, A, C, D)(0, 0);
draw byLabelsOnPolygon(F, E, H, G)(0, 0);
}
\drawCurrentPictureInMargin
\problemNP{В}{о}{всяком треугольнике квадрат стороны, стягивающей острый угол, меньше суммы квадратов сторон, содержащих этот угол, на дважды прямоугольник, заключенный между любой из этих сторон и отрезком, отсекаемым перпендикуляром из противоположного угла от этого отрезка или от продленного отрезка.\\
Первый случай.\\
$\drawProportionalLine{CA}^2 < \drawProportionalLine{BD,DC}^2 + \drawProportionalLine{AB}^2$ на $2 \cdot \drawProportionalLine{BD,DC} \cdot \drawProportionalLine{BD}$.\\
Второй случай.\\
$\drawProportionalLine{EG}^2 < \drawProportionalLine{EF}^2 + \drawProportionalLine{FG}^2$ на $2 \cdot \drawProportionalLine{FG} \cdot \drawProportionalLine{FG,GH}$.
}

\startCenterAlign
Предположим, перпендикуляр падает внутри треугольника, тогда  \inprop[prop:II.VII]\\
$\drawProportionalLine{BD,DC}^2 + \drawProportionalLine{BD}^2 = 2 \cdot \drawProportionalLine{BD,DC} \cdot \drawProportionalLine{BD} + \drawProportionalLine{DC}^2$,\\
к каждой добавим $\drawProportionalLine{AD}^2$, тогда\\
$\drawProportionalLine{BD,DC}^2 + \drawProportionalLine{BD}^2 + \drawProportionalLine{AD}^2 = 2 \cdot \drawProportionalLine{BD,DC} \cdot \drawProportionalLine{BD} + \drawProportionalLine{DC}^2 + \drawProportionalLine{AD}^2$\\
$\therefore$ \inprop[prop:I.XLVII]\\
$\drawProportionalLine{BD,DC}^2 + \drawProportionalLine{AB}^2 = 2 \cdot \drawProportionalLine{BD,DC} \cdot \drawProportionalLine{BD} + \drawProportionalLine{CA}^2$,\\
и $\therefore \drawProportionalLine{CA}^2 < \drawProportionalLine{BD,DC}^2 + \drawProportionalLine{AB}^2$ на $2 \cdot \drawProportionalLine{BD,DC} \cdot \drawProportionalLine{DC}$

Теперь предположим, что перпендикуляр падает вовне треугольника, тогда \inprop[prop:II.VII]\\
$\drawProportionalLine{FG,GH}^2 + \drawProportionalLine{FG}^2 = 2 \cdot \drawProportionalLine{FG,GH} \cdot \drawProportionalLine{FG} + \drawProportionalLine{GH}^2$,\\
к каждой добавим $\drawProportionalLine{HE}^2$, тогда\\
$\drawProportionalLine{FG,GH}^2 + \drawProportionalLine{FG}^2 + \drawProportionalLine{HE}^2= 2 \cdot \drawProportionalLine{FG,GH} \cdot \drawProportionalLine{FG} + \drawProportionalLine{GH}^2 + \drawProportionalLine{HE}^2$\\
$\therefore$ \inprop[prop:I.XLVII]\\
$\drawProportionalLine{EF} + \drawProportionalLine{FG}^2 = 2 \cdot \drawProportionalLine{FG,GH} \cdot \drawProportionalLine{FG}^2 + \drawProportionalLine{EG}^2$,\\
$\therefore \drawProportionalLine{EG}^2 < \drawProportionalLine{EF}^2 + \drawProportionalLine{FG}^2$ на $2 \cdot \drawProportionalLine{FG,GH} \cdot \drawProportionalLine{FG}$.
\stopCenterAlign

\qed
\stopProposition

\startProposition[title={Предложение XIV. Задача},reference=prop:II.XIV]
\defineNewPicture[1/3]{
path A;
pair a, b, c, d, e, f;
pair B, C, D, E, F, G, H;
numeric w, h, s, r;
a := (0, 0);
b := (u, 1/4u);
c := (2u, 1/5u);
d := (11/5u, -u);
e := (7/8u, -2u);
f := (1/8u, -3/4u);
A := a--b--c--d--e--f--cycle;
s := 0;
for i := 1 step 1 until length(A) - 1:
	s := s + 1/2(
		abs((point 0 of A) - (point i of A))
		*distanceToLine((point i + 1 of A), (point 0 of A)--(point i of A))
		);
endfor;
w := 5/2u;
h := s/w;
B := (w, 0);
C := (w, -h);
D := (0, -h);
E := (0, 0);
F := (-h, 0);
G := 1/2[B, F];
r := abs(B - G);
H := (D--(E shifted ((E-D)*10))) intersectionpoint ((fullcircle scaled 2r) shifted G);
forsuffixes i=a,b,c,d,e,f:
	i := i shifted (xpart(G)-xpart(1/2[urcorner(A),ulcorner(A)]), r - ypart(lrcorner(A)) + 1/2u);
	byPointLabelDefine(i, "");
endfor;
draw byPolygon(a,b,c,d,e,f)(byyellow);
draw byPolygon(B,C,D,E)(byred);
draw byLine(H, G, byred, 0, 0);
draw byLine(H, E, byblue, 0, 0);
draw byLine(E, D, byyellow, 0, 0);
byLineDefine(H, F, black, 0, 1);
byLineDefine(H, B, black, 0, 1);
byLineDefine(F, E, black, 1, 0);
byLineDefine(E, G, byblue, 1, 0);
byLineDefine(G, B, black, 0, 0);
draw byNamedLineSeq(-1)(FE,EG,GB,HB,HF);
draw byArc(G, B, F, r, byred, 0, 0, 0, 0)(G);
byLineDefine(B, F, black, 0, 0);
draw byLabelsOnPolygon(B, C, D, E, F, H)(0, 0);
draw byLabelsOnPolygon(H, G, B)(2, 0);
}
\drawCurrentPictureInMargin
\problemNP{П}{остроить}{квадрат, равный данной прямолинейной фигуре.\\~\\
Провести \drawSizedLine{HE} такую, что $\drawSizedLine{HE}^2 = \drawPolygon{abcdef}$
}

\startCenterAlign
Сделаем $\drawPolygon{BCDE} = \drawPolygon{abcdef}$ \inprop[prop:I.XLV],

Продлим \drawSizedLine{EG,GB} до $\drawSizedLine{FE} = \drawSizedLine{ED}$;\\
Возьмем $\drawSizedLine{FE,EG} = \drawSizedLine{GB}$ \inprop[prop:I.X],

Опишем
\drawFromCurrentPicture{
startTempScale(1/3);
draw byNamedLineFull(B, F, 0, 0, 1)(BF);
startAutoLabeling;
draw byNamedArc(G);
stopAutoLabeling;
stopTempScale;
}
\inpost[post:III]\\
и продлим до нее \drawSizedLine{ED}: проведем \drawSizedLine{HG}.

$\drawSizedLine{HF}^2 \mbox{ или } \drawSizedLine{HG}^2 = \drawSizedLine{FE} \cdot \drawSizedLine{EG,GB} + \drawSizedLine{EG}^2$ \inprop[prop:II.V],\\
но $\drawSizedLine{HG}^2 = \drawSizedLine{HE} ^2 + \drawSizedLine{EG}^2$ \inprop[prop:I.XLVII];

$\therefore \drawSizedLine{HE}^2 + \drawSizedLine{EG}^2 = \drawSizedLine{FE} \cdot \drawSizedLine{EG,GB} + \drawSizedLine{EG}^2$

$\therefore \drawSizedLine{HE}^2 = \drawSizedLine{FE} \cdot \drawSizedLine{EG,GB}$,

и $\therefore \drawSizedLine{HE}^2 = \drawPolygon{BCDE} = \drawPolygon{abcdef}$
\stopCenterAlign

\qed
\stopProposition
\stopbook

\startbook[title={Книга III}]

\startsupersection[title={Определения}]

\startDefinitionOnlyNumber[reference=def:III.I]
Равными кругами называют те, у которых равны диаметры.
\stopDefinitionOnlyNumber

\startDefinitionOnlyNumber[reference=def:III.II]
\defineNewPicture{
	pair O, A, B;
	numeric r;
	r := 3/4u;
	O := (0, 0);
	A := (-r, -r);
	B := (r, -r);
	draw byCircleR(O, r, black, 0, 0, -1)(O);
	draw byLine(A, B, black, 0, 0);
}\drawCurrentPictureInMargin
Про прямую говорят, что она касается круга, если она встречает круг, но при продолжении не пересекает его.
\stopDefinitionOnlyNumber

\startDefinitionOnlyNumber[reference=def:III.III]
\defineNewPicture{
	pair A, B, C;
	numeric r[];
	r1 := 2/3u;
	r2 := 1/2r1;
	r3 := 2/5r1;
	A := (0, 0);
	B := A shifted (dir(110) scaled (r1-r2));
	C := A shifted (dir(-130) scaled (r1+r3));
	fill (fullcircle scaled 2r1) shifted A withcolor byyellow;
	draw byCircleR(A, r1, byyellow, 0, 0, 0)(A);
	fill (fullcircle scaled 2r2) shifted B withcolor byred;
	draw byCircleR(B, r2, black, 0, 0, -1/2)(B);
	fill (fullcircle scaled 2r3) shifted C withcolor byblue;
	draw byCircleR(C, r3, byblue, 0, 0, -1/2)(C);
}\drawCurrentPictureInMargin
Про круги говорят, что они касаются друг друга, если они встречаясь, не пересекают друг друга.
\stopDefinitionOnlyNumber

\startDefinitionOnlyNumber[reference=def:III.IV]
\defineNewPicture{
	pair O, A, B, C, D, E, F;
	numeric r;
	r := 3/4u;
	O := (0, 0);
	A := dir(170) scaled r;
	B := dir(-110) scaled r;
	C := A xscaled -1;
	D := B xscaled -1;
	E := 1/2[A, B];
	F := 1/2[C, D];
	draw byLine(A, B, black, 0, 0);
	draw byLine(C, D, black, 0, 0);
	byLineDefine(O, E, byred, 0, 0);
	byLineDefine(O, F, byblue, 0, 0);
	draw byNamedLineSeq(0)(OE,OF);
	draw byCircleR(O, r, black, 0, 0, 0)(O);
}\drawCurrentPictureInMargin
Про прямые говорят, что они равноотстоят от центра круга, если перпендикуляры проведенные к ним из центра круга равны.
\stopDefinitionOnlyNumber

\startDefinitionOnlyNumber[reference=def:III.V]
Про прямую, перпендикуляр к которой длиннее, говорят, что она отстоит дальше от центра круга.
\stopDefinitionOnlyNumber

\vfill\pagebreak

\startDefinitionOnlyNumber[reference=def:III.VI]
\defineNewPicture{
	pair A, B;
	numeric r;
	r := 3/4u;
	A := (0, 0);
	B := (0, -1/8u);
	draw byFilledCircleSegment(A, r, 1/2, 4 - 1/2, byred)(A);
	draw byFilledCircleSegment(B, r, 4 - 1/2, 8 + 1/2, byblue)(B);
}\drawCurrentPictureInMargin
Сегментом круга называется фигура, заключающаяся между прямой и частью кругя, отсекаемой этой прямой.
\stopDefinitionOnlyNumber

\startDefinitionOnlyNumber[reference=def:III.VII]
\defineNewPicture{
	pair O, A, B, C;
	numeric r, b, e;
	r := u;
	b := 1/2;
	e := 4-1/2;
	O := (0, 0);
	A := O shifted (point b of fullcircle scaled 2r);
	B := O shifted (point e of fullcircle scaled 2r);
	C := B shifted ((unitvector(B-O) rotated -90) scaled r);
	draw byAngleWithName(C, B, A, byred, 0)(B);
	draw byArcBE(O, b, e, r, black, 0, 0, 0, 0)(O);
	draw byLine(A, B, black, 0, 0);
	draw byLine(B, C, black, 0, 0);
	draw byLine(O, B, black, 1, 0);
}\drawCurrentPictureInMargin
Углом сегмента называется угол между основанием сегмента и перпендикуляром к отрезку, соединяющему центр круга с одним из концов основания сегмента.
\stopDefinitionOnlyNumber

\startDefinitionOnlyNumber[reference=def:III.VIII]
\defineNewPicture{
	pair O, A, B, C, D;
	numeric r, b, e;
	r := u;
	b := -1;
	e := 5;
	O := (0, 0);
	A := dir(100) scaled r;
	B := dir(30) scaled r;
	C := point b of fullcircle scaled 2r;
	D := point e of fullcircle scaled 2r;
	draw byAngleWithName(C, A, D, byblue, 0)(A);
	draw byAngleWithName(C, B, D, byyellow, 0)(B);
	draw byArcBE(O, b, e, r, black, 0, 0, 0, 0)(O);
	draw byLine(C, A, black, 0, 0);
	draw byLine(C, B, black, 0, 0);
	draw byLine(D, A, black, 0, 0);
	draw byLine(D, B, black, 0, 0);
	draw byLine(C, D, black, 0, 0);
}\drawCurrentPictureInMargin
Углом в сегменте называют угол, заключающийся между прямыми, проведенными из какой-либо точки на окружности сегмента к концам основания сегмента.
\stopDefinitionOnlyNumber

\startDefinitionOnlyNumber[reference=def:III.IX]
\defineNewPicture{
	pair O, A, B, C;
	numeric r, b, e;
	r := 2/3u;
	b := -3/2;
	e := 9/2;
	O := (0, 0);
	A := dir(80) scaled r;
	B := point b of fullcircle scaled 2r;
	C := point e of fullcircle scaled 2r;
	draw byAngleWithName(C, A, B, byblue, 0)(A);
	draw byArcBE(O, b, e, r, black, 1, 0, 0, 0)(Op);
	draw byArcBE(O, e, b + 8, r, black, 0, 0, 0, 0)(Om);
	draw byLine(C, A, black, 0, 0);
	draw byLine(B, A, black, 0, 0);
}\drawCurrentPictureInMargin
Про угол говорят, что он опирается на дугу, если заключающие его прямые отсекают эту дугу.
\stopDefinitionOnlyNumber

\startDefinitionOnlyNumber[reference=def:III.X]
\defineNewPicture{
	pair O;
	numeric r, b, e;
	r := 2/3u;
	b := 1;
	e := 3;
	O := (0, 0);
	draw byFilledCircleSector(O, r, b, e, byyellow)(O);
	draw byArcBE(O, e, b + 8, r, black, 0, 0, -1, 0)(O);
}\drawCurrentPictureInMargin
Сектором круга называют фигуру, заключающуюся между двумя радиусами и дугой между ними.
\stopDefinitionOnlyNumber

\startDefinitionOnlyNumber[reference=def:III.XI]
\defineNewPicture{
	pair M, N, A, B, C, D, E, F;
	numeric r[], b, e;
	r1 := 3/2u;
	r2 := u;
	b := 1;
	e := 3;
	M := (0, 0);
	N := (0, -1/3u);
	A := (point b of fullcircle scaled 2r1) shifted M;
	B := (point 1/3[b,e] of fullcircle scaled 2r1) shifted M;
	C := (point e of fullcircle scaled 2r1) shifted M;
	D := (point b of fullcircle scaled 2r2) shifted N;
	E := (point 1/3[b,e] of fullcircle scaled 2r2) shifted N;
	F := (point e of fullcircle scaled 2r2) shifted N;
	draw byPolygon(A,B,C)(byred);
	draw byPolygon(D,E,F)(byred);
	draw byArcBE(M, b, e, r1, black, 0, 0, 0, 1)(M);
	draw byArcBE(N, b, e, r2, black, 0, 0, 0, 1)(N);
}\drawCurrentPictureInMargin
Подобными сегментами называют сегменты, углы в которых равны.
\stopDefinitionOnlyNumber

\startDefinitionOnlyNumber[reference=def:III.XII]
\defineNewPicture{
	pair O;
	numeric r[];
	r1 := 1/3u;
	r2 := 1/2u;
	r3 := 3/4u;
	O := (0, 0);
	draw byFilledCircleSegment(O, r3, 0, 8, byred)(OI);
	draw byFilledCircleSegment(O, r2, 0, 8, white)(OII);
	draw byCircleR(O, r2, black, 0, 0, 0)(OII);
	draw byFilledCircleSegment(O, r1, 0, 8, byblue)(OIII);
}\drawCurrentPictureInMargin
Круги, имеющие один центр называют концентрическими.
\stopDefinitionOnlyNumber
\stopsupersection

\vfill\pagebreak

\startProposition[title={Предложение I. Задача},reference=prop:III.I]
\defineNewPicture{
pair A, B, C, D, E, F, G;
numeric r, a;
r := 9/4u;
F := (0, 0);
A := F shifted (dir(170)*r);
B := F shifted (dir(-95)*r);
D := 1/2[A, B];
C := F shifted (dir(angle(A-B) - 90)*r);
E := F shifted (dir(angle(A-B) +90)*r);
G := F shifted (dir(-45)*1/2r);
a := -angle(G-D);
forsuffixes i=A, B, D, C, E, F, G:
	i := i rotated a;
endfor;
draw byAngle(A, D, F, byblue, 0);
draw byAngle(F, D, G, byyellow, 0);
draw byAngle(G, D, B, black, 0);
draw byLine(D, G)(byblue, 1, 0);
draw byLine(A, D)(byred, 0, 0);
draw byLine(D, B)(byred, 1, 0);
draw byLine(E, C)(black, 0, 0);
draw byMarkLine(1/2, black)(EC);
byLineDefine(A, G, byblue, 0, 0);
byLineDefine(B, G, black, 1, 0);
draw byNamedLineSeq(0)(AG,BG);
draw byCircleR(F, r, byblue, 0, 0, 0)(F);
draw byLabelsOnPolygon(A, G, B)(2, 0);
draw byLabelsOnPolygon(E, D, A)(2, 0);
draw byLabelsOnPolygon(E, F, C)(2, -4);
draw byLabelsOnCircle(A, B, C, E)(F);
}
\drawCurrentPictureInMargin
\problemNP{Н}{айти}{центр данного круга \drawCircle[middle][1/4]{F}.}

\startCenterAlign
Проведем внутри круга любую прямую \drawUnitLine{AD,DB},\\
сделаем $\drawUnitLine{AD} = \drawUnitLine{DB}$, проведем $\drawUnitLine{EC} \perp \drawUnitLine{AD,DB}$; рассечем пополам \drawUnitLine{EC}, точка рассечения и есть центр.\\
Действительно, пусть это не так, тогда пусть другая точка будет центром, и проведем от нее \drawUnitLine{AG}, \drawUnitLine{DG} и \drawUnitLine{BG}.\\
Поскольку в \drawLine{AG,DG,AD} и \drawLine{DB,DG,BG} $\drawUnitLine{AG} = \drawUnitLine{BG}$ (гип. и \indefL[def:XV]),\\ 
$\drawUnitLine{AD} = \drawUnitLine{DB}$ (постр.) и \drawUnitLine{DG} общая обоим,\\
$\drawAngle{ADF,FDG} = \drawAngle{GDB}$ \inprop[prop:I.VIII], и, сдедовательно, являются прямыми углами;\\
но $\drawAngle{FDG,GDB} = \drawRightAngle$ (постр.), $\drawAngle{GDB} = \drawAngle{FDG,GDB}$ \inax[ax:XI]\\ 
что невозможно;
\stopCenterAlign

\noindent значит, выбранная точка не центр круга, и так эе можно показать, что никакая другая точка, не на \drawUnitLine{EC} не является центром, значит, центр находится на \drawUnitLine{EC}, и, следовательно, точка, где \drawUnitLine{EC} рассекается пополам и есть центр.

\qed
\stopProposition

\startProposition[title={Предложение II. Теорема},reference=prop:III.II]
\defineNewPicture{
pair A, B, D, E, F;
numeric r;
r := 9/4u;
D := (0, 0);
A := (dir(185) scaled r) shifted D;
B := (dir(-70) scaled r) shifted D;
E := 3/5[A, B];
F := (dir(angle(E-D)) scaled r) shifted D;
draw byAngleWithName(D, A, B, byblue, 0)(A);
draw byAngleWithName(A, E, D, byyellow, 0)(E);
draw byAngleWithName(A, B, D, black, 0)(B);
draw byLine(D, E, black, 0, 0);
draw byLine(E, F, black, 1, 0);
draw byLine(A, B, byred, 0, 0);
byLineDefine(A, D, byyellow, 0, 0);
byLineDefine(B, D, byblue, 0, 0);
draw byNamedLineSeq(0)(AD,BD);
draw byCircleR(D, r, byred, 0, 0, 0)(D);
draw byLabelsOnPolygon(B, F, A, D)(0, 0);
draw byLabelsOnPolygon(A, E, F)(2, 0);
}
\drawCurrentPictureInMargin
\problemNP[2]{П}{рямая}{\drawSizedLine{AB}, соединяющая две точки на окружности круга \drawCircle[middle][1/4]{D}, целиком находится внутри круга.}

\startCenterAlign
Найдем центр \circleD\ \inprop[prop:III.I];\\
из центра проведем \drawSizedLine{DE} к любой точке \drawSizedLine{AB}, пересекающую окружность;\\
проведем \drawSizedLine{AD} и \drawSizedLine{BD}.

Тогда $\drawAngle{A} = \drawAngle{B}$ \inprop[prop:I.V]\\
но $\drawAngle{E} > \drawAngle{A} \mbox{ или } > \drawAngle{B}$ \inprop[prop:I.XVI]

$\therefore \drawSizedLine{AD} > \drawSizedLine{DE}$ \inprop[prop:I.XIX]\\
но $\drawSizedLine{AD} = \drawSizedLine{DE,EF}$,

$\therefore \drawSizedLine{DE,EF} > \drawSizedLine{DE}$;

$\therefore \drawSizedLine{DE} < \drawSizedLine{DE,EF}$;

$\therefore$ любая точка на \drawSizedLine{AB} находится внутри круга.
\stopCenterAlign

\qed
\stopProposition

\startProposition[title={Предложение III. Теорема},reference=prop:III.III]
\defineNewPicture[1/2]{
pair A, B, C, D, E, F;
numeric r;
r := 9/4u;
E := (0, 0);
A := (dir(-90 - 60) scaled r) shifted E;
B := (dir(-90 + 60) scaled r) shifted E;
C := (dir(90) scaled r) shifted E;
D := (dir(-90) scaled r) shifted E;
F = whatever[A, B] = whatever[C, D];
draw byAngleWithName(E, A, F, byblue, 0)(A);
draw byAngle(A, F, E, black, 0);
draw byAngle(E, F, B, byyellow, 0);
draw byAngleWithName(F, B, E, byred, 0)(B);
draw byLine(C, E, black, 1, 0);
draw byLine(E, F, black, 0, 0);
draw byLine(F, D, black, 1, 0);
draw byLine(A, F, byred, 0, 0);
draw byLine(F, B, byred, 1, 0);
byLineDefine(A, E, byyellow, 0, 0);
byLineDefine(E, B, byblue, 0, 0);
draw byNamedLineSeq(0)(AE,EB);
draw byCircleR(E, r, byblue, 0, 0, 0)(E);
draw byLabelsOnCircle(A, B)(E);
draw byLabelsOnPolygon(D, F, A)(2, -2);
draw byLabelsOnPolygon(A, E, C)(2, 0);
}
\drawCurrentPictureInMargin
\problemNP{Е}{сли}{некоторая прямая \drawUnitLine{EF}, проходящая через центр круга \drawCircle[middle][1/6]{E}, рассекает пополам хорду \drawUnitLine{AF,FB}, не проходящую через центр, то эта прямая ей перпендикулярна; а если перпендикулярна, то рассекает хорду пополам.}

\startCenterAlign
Проведем \drawUnitLine{AE} и \drawUnitLine{EB} к центру круга.

В \drawLine[bottom][triangleAEF]{AE,EF,AF} и \drawLine[bottom][triangleFEB]{EF,EB,FB}\\
$\drawUnitLine{AE} = \drawUnitLine{EB}$, \drawUnitLine{EF} общая,\\
и $\drawUnitLine{AF} = \drawUnitLine{FB}, \therefore \drawAngle{AFE} = \drawAngle{EFB}$ \inprop[prop:I.VIII]\\
и $\therefore \drawUnitLine{EF} \perp \drawUnitLine{AF,FB}$ \inprop[prop:I.X]

Теперь, пусть $\drawUnitLine{EF} \perp \drawUnitLine{AF,FB}$\\
Тогда в  \triangleAEF\ and \triangleFEB\\
$\drawAngle{A} = \drawAngle{B}$ \inprop[prop:I.V]\\
$\drawAngle{AFE} = \drawAngle{EFB}$ (гип.)\\
и $\drawUnitLine{AE} = \drawUnitLine{EB}$

$\therefore \drawUnitLine{AF} = \drawUnitLine{FB}$ \inprop[prop:I.XXVI]

и $\therefore \drawUnitLine{EF}$ рассекает \drawUnitLine{AF,FB} пополам.
\stopCenterAlign

\qed
\stopProposition

\startProposition[title={Предложение IV. Теорема},reference=prop:III.IV]
\defineNewPicture{
pair A, B, C, D, E, F;
numeric r;
r := 9/4u;
F := (0, 0);
A := (dir(-175)*r) shifted F;
B := (dir(-140)*r) shifted F;
C := (dir(-50)*r) shifted F;
D := (dir(-10)*r) shifted F;
E = whatever[A, C] = whatever[B, D];
draw byAngle(F, E, D, byblue, 0);
draw byAngle(D, E, C, byyellow, 0);
draw byLine(E, F, black, 1, 0);
draw byLine(B, D, byred, 0, 0);
draw byLine(A, C, black, 0, 0);
draw byCircleR(F, r, byblue, 0, 0, 0)(F);
draw byLabelsOnCircle(A, B, C, D)(F);
draw byLabelLineEnd(F, E, 0);
draw byLabelsOnPolygon(C, E, B)(2, 0);
}
\drawCurrentPictureInMargin
\problemNP{Е}{сли}{в круге две прямые, не проходящие через центр, пересекаются, они не делят друг друга пополам.}

Если одна из прямых проходит через центр, очевидно, она ее не может рассекать пополам другая прямая, не проходящая через центр.

Но если ни одна из прямых \drawUnitLine{AC} или \drawUnitLine{BD} не проходит через центр, проведем \drawUnitLine{EF} из центра к точке их пересечения.

\startCenterAlign
Если \drawUnitLine{AC} делится пополам, \drawUnitLine{EF} $\perp$ ей \inprop[prop:III.III]\\
$\therefore \drawAngle{FED,DEC} = \drawRightAngle$\\
и если \drawUnitLine{BD} делится пополам, $\drawUnitLine{EF} \perp \drawUnitLine{BD}$ \inprop[prop:III.III]\\
$\therefore \drawAngle{FED} = \drawRightAngle$;

и $\therefore \drawAngle{FED} = \drawAngle{FED,DEC}$;\\
часть равна целому, что невозможно.

$\therefore$ \drawUnitLine{AC} и \drawUnitLine{BD} не~делят друг друга пополам.
\stopCenterAlign

\qed
\stopProposition

\startProposition[title={Предложение V. Теорема},reference=prop:III.V]
\defineNewPicture{
pair M, N, E, F, G, C;
numeric r[], s;
path c[];
r1 := 2u;
r2 := 2u;
s := u;
M := (1/2s, 0);
N := (-1/2s, 0);
c1 := (fullcircle scaled 2r1) shifted M;
c2 := (fullcircle scaled 2r2) shifted N;
E := 1/2[M, N];
C := (subpath(0, 4) of c1) intersectionpoint (subpath(0, 4) of c2);
G := point 7/2 of c2;
F := c1 intersectionpoint (E--G);
byLineDefine(C, E, byyellow, 0, 0);
byLineDefine(E, F, black, 0, 0);
byLineDefine(F, G, black, 1, 0);
draw byNamedLineSeq(0)(CE,EF,FG);
draw byArcBE(M, 4, 0, r1, byred, 0, 0, 0, 0)(Ma);
draw byArcBE(N, 4, 0, r2, byblue, 0, 0, 0, 0)(Na);
draw byArcBE(N, 4, 8, r2, byblue, 0, 0, 0, 0)(Nb);
draw byArcBE(M, 4, 8, r1, byred, 0, 0, 0, 0)(Mb);
draw byLabelLineEnd(C, E, 0);
draw byLabelLineEnd(G, E, 0);
draw byLabelsOnPolygon(C, E, F)(2, 0);
draw byLabelsOnPolygon(C, F, E)(2, -1);
}
\drawCurrentPictureInMargin
\problemNP{Е}{сли}{два круга секут друг друга \drawArc{Ma,Na,Nb,Mb} их центры не совпадают.}

Допустим это возможно, и два пересекающихся круга имеют общий центр. Из предпологаемого центра проведем \drawUnitLine{CE} к точке пересечения и \drawUnitLine{EF,FG} пересекающую окружности.

\startCenterAlign
Тогда $\drawUnitLine{CE} = \drawUnitLine{EF}$ \inprop[prop:I.XV]\\
и $\drawUnitLine{CE} = \drawUnitLine{EF,FG}$ \inprop[prop:I.XV]

$\therefore \drawUnitLine{EF} = \drawUnitLine{EF,FG}$\\
часть равна целому, что невозможно.

$\therefore$ круги, пересекающиеся в любой точке не могут иметь общего центра.
\stopCenterAlign

\qed
\stopProposition

\startProposition[title={Предложение VI. Теорема},reference=prop:III.VI]
\defineNewPicture{
pair M, N, B, C, E, F;
numeric r[], a;
path c[];
a := 80;
r1 := 9/4u;
r2 := 7/4u;
M := (0, 0);
N := M shifted (dir(a)*(r1-r2));
c1 := (fullcircle scaled 2r1) shifted M;
c2 := (fullcircle scaled 2r2) shifted N;
F := 1/2[M, N];
C :=c1 intersectionpoint (M--(M shifted (dir(a)*2r1)));
B := point -3/2 of c1;
E := c2 intersectionpoint (F--B);
byLineDefine(C, F, byyellow, 0, 0);
byLineDefine(F, E, byblue, 1, 0);
byLineDefine(E, B, byblue, 0, 0);
draw byNamedLineSeq(0)(CF,FE,EB);
draw byCircle(M, C, byred, 0, 0, 0)(M);
draw byCircle(N, C, black, 0, 0, -1)(N);
draw byLabelsOnCircle(B, C)(M);
draw byLabelsOnPolygon(E, F, C)(2, 0);
draw byLabelPoint(E, angle(B-F)+45, 2);
}
\drawCurrentPictureInMargin
\problemNP{Е}{сли}{два круга \drawFromCurrentPicture{draw byNamedCircle(M,N);} касаются друг друга, то у них не один и тот же центр.}

Действительно, пусть это возможно, и у кругов будет один центр. Из предпологаемого центра проведем \drawUnitLine{FE,EB} и \drawUnitLine{CF} к точке касания.

\startCenterAlign
Тогда $\drawUnitLine{CF} = \drawUnitLine{FE}$ \inprop[prop:I.XV]\\
и $\drawUnitLine{CF} = \drawUnitLine{FE,EB}$ \inprop[prop:I.XV]

$\therefore \drawUnitLine{FE} = \drawUnitLine{FE,EB}$;
\stopCenterAlign
\noindent часть равна целому, что невозможно. Следовательно, выбранная точка не является центром обоих кругов и таким же образом можно показать, что так же и никакая другая.

\qed
\stopProposition

\startProposition[title={Предложение VII. Теорема},reference=prop:III.VII]
\defineNewPicture[1/2]{
pair A, B, C, D, E, F, G, H;
numeric r;
r := 2u;
E := (0, 0);
A := E shifted (dir(90)*r);
D := E shifted (dir(-90)*r);
F := 2/3[E, D];
B := E shifted (dir(90-70)*r);
C := E shifted (dir(-5)*r);
H := E shifted (dir(-170)*r);
G := E shifted (dir(90+70)*r);
draw byLine(F, B, byred, 0, 0);
draw byLine(F, C, byblue, 0, 0);
draw byLine(F, A, black, 0, 0);
draw byLine(F, D, byyellow, 0, 0);
draw byLine(F, G, byblue, 1, 0);
draw byCircleR(E, r, byblue, 0, 0, 0)(E);
draw byLabelsOnCircle(A, B, C, D, G)(E);
draw byLabelsOnPolygon(D, F, G)(2, 0);
draw byLabelsOnPolygon(F, E, A)(2, 0);
}
\drawCurrentPictureInMargin
\problemNP[2]{Е}{сли}{из точки в круге \drawFromCurrentPicture{
startTempScale(4/9);
draw byNamedCircle(E);
draw byLabelPoint(F, 0, 0);
stopTempScale;
}, не являющейся центром к окружности проведены прямые линии
$\left\{\vcenter{
\nointerlineskip\hbox{\drawUnitLine{FA}, \drawUnitLine{FB}}
\nointerlineskip\hbox{\drawUnitLine{FC}, \drawUnitLine{FD}}
}\right.$
наибольшая из них та \drawUnitLine{FA}, что проходит через центр, а меньшая — та, что является оставшейся частью диаметра \drawUnitLine{FD}. \\
Из остальных, та \drawUnitLine{FB}, что ближе к проходящей через центр больше той \drawUnitLine{FC}, что проходит дальше.\\
Линии \drawUnitLine{FG} и \drawUnitLine{FB} под равными углами к линии, проходящей через центр и по разные стороны от нее равны между собой и из той же точки нельзя провести третью линию той же длины к окружности.}

\defineNewPicture[1/2]{
pair A, B, C, D, E, F, G, H;
numeric r;
r := 2u;
E := (0, 0);
A := E shifted (dir(90)*r);
D := E shifted (dir(-90)*r);
F := 2/3[E, D];
B := E shifted (dir(20)*r);
C := E shifted (dir(-5)*r);
G := E shifted (dir(15)*r);
H := E shifted (dir(-170)*r);
draw byAngle(B, E, C, black, 0);
draw byAngle(C, E, F, byyellow, 0);
draw byLine(F, B, byred, 0, 0);
draw byLine(E, C, byblue, 1, 0);
draw byLine(F, C, byblue, 0, 0);
draw byLine(E, B, byred, 1, 0);
draw byLine(A, E, black, 1, 0);
draw byLine(E, F, black, 0, 0);
draw byLine(F, D, byyellow, 0, 0);
draw byCircleR(E, r, byblue, 0, 0, 0)(E);
draw byLabelsOnCircle(A, B, C, D)(E);
draw byLabelsOnPolygon(D, F, E, A)(2, 0);
}
\drawCurrentPictureInMargin
\startsubproposition[title={Часть I}]
Из центра круга проведем \drawUnitLine{EB} и \drawUnitLine{EC}, тогда $\drawUnitLine{AE} = \drawUnitLine{EB}$ \inprop[prop:I.XV] $\drawUnitLine{EF,AE} = \drawUnitLine{EF} + \drawUnitLine{EB} > \drawUnitLine{FB}$ \inprop[prop:I.XX]. Таким же образом можно показать, что \drawUnitLine{EF,AE} больше \drawUnitLine{FC}, или любой другой линии, проведенной из той же точки к окружности. 

Теперь, согласно \inprop[prop:I.XX] $\drawUnitLine{EF} + \drawUnitLine{FC} > \drawUnitLine{EC} = \drawUnitLine{FD} + \drawUnitLine{EF}$, вычтем \drawUnitLine{EF} из обеих; $\therefore \drawUnitLine{FC} > \drawUnitLine{FD}$ (акс.), и так же можно показать, что \drawUnitLine{FD} меньше любой другой линии, проведенной из той же точки к окружности. 

Теперь, в \drawLine[middle][triangleEFB]{FB,EF,EB} и \drawLine[middle][triangleEFC]{FC,EF,EC}, \drawUnitLine{EF} общая, $\drawAngle{BEC,CEF} > \drawAngle{CEF}$, и $\drawUnitLine{EB} > \drawUnitLine{EC} \therefore \drawUnitLine{FB} > \drawUnitLine{FC}$ \inprop[prop:I.XXIV] и так же можно показать, что \drawUnitLine{FB} больше любой другой линии, проведенной из той же точки к окружности и проходящей дальше \drawUnitLine{EF,AE}.
\stopsubproposition

\defineNewPicture[1/2]{
pair A, B, C, D, E, F, G, H, M;
numeric r;
r := 2u;
E := (0, 0);
A := E shifted (dir(90)*r);
D := E shifted (dir(-90)*r);
F := 2/3[E, D];
B := E shifted (dir(90-70)*r);
H := E shifted (dir(-170)*r);
G := E shifted (dir(90+70)*r);
M = whatever[E, H] = whatever[F, G];
draw byAngle(B, F, E, byyellow, 0);
draw byAngle(G, F, E, byred, 0);
draw byLine(F, B, byred, 0, 0);
draw byLine(E, B, byred, 1, 0);
draw byLine(E, M, byyellow, 0, 0);
draw byLine(M, H, byyellow, 1, 0);
draw byLine(F, M, byblue, 0, 0);
draw byLine(M, G, byblue, 1, 0);
draw byLine(A, E, black, 1, 0);
draw byLine(E, F, black, 0, 0);
draw byLine(F, D, black, 0, 0);
draw byCircleR(E, r, byblue, 0, 0, 0)(E);
draw byLabelsOnCircle(B, G, H)(E);
draw byLabelsOnPolygon(D, F, M, H)(2, 0);
draw byLabelsOnPolygon(M, E, A)(2, 0);
}
\drawCurrentPictureInMargin
\startsubproposition[title={Часть II}]
\startCenterAlign
Если $\drawAngle{GFE} = \drawAngle{BFE}$, то $\drawUnitLine{FM,MG} = \drawUnitLine{FB}$,\\
Если нет, возьмем $\drawUnitLine{FM} = \drawUnitLine{FB}$, проведем \drawUnitLine{EM,MH},\\
тогда в \drawLine[middle][triangleEFM]{EM,EF,FM} и \drawLine[middle][triangleEFB]{FB,EF,EB}, \drawUnitLine{EF} общая,\\
$\drawAngle{GFE} = \drawAngle{BFE}$ и $\drawUnitLine{FB} = \drawUnitLine{FM}$\\
$\therefore \drawUnitLine{EB} = \drawUnitLine{EM}$ \inprop[prop:I.IV]\\
$\therefore \drawUnitLine{EB} = \drawUnitLine{EM,MH} = \drawUnitLine{EM}$\\
часть равна целому, что невозможно:
\stopCenterAlign

$\therefore \drawUnitLine{FB} = \drawUnitLine{FM,MG}$; и никакая другая линия равная \drawUnitLine{FB}, проведенная из той же точки к окружности, поскольку, будь она ближе к проходящей через центр, она была бы больше, а дальше — меньше.
\stopsubproposition

\qed
\stopProposition

\startProposition[title={Предложение VIII. Теорема},reference=prop:III.VIII]
\problemNP{П}{редложение}{разделено на три части.}

\startsubproposition[title={I.}]
\defineNewPicture[1/4]{
pair M, D, A, E, F;
numeric r;
r := 7/4u;
M := (0, 0);
D := M shifted (dir(90)*3/2r);
A := (dir(-90)*r) shifted M;
E := (dir(-140)*r) shifted M;
F := (dir(-170)*r) shifted M;
draw byAngle(D, M, F, byyellow, 0);
draw byAngle(F, M, E, black, 0);
draw byLine(D, E, byred, 0, 0);
draw byLine(M, A, black, 1, 0);
draw byLine(M, E, byred, 1, 0);
draw byLine(M, F, byblue, 1, 0);
byLineDefine(D, M, black, 0, 0);
byLineDefine(D, F, byblue, 0, 0);
draw byNamedLineSeq(0)(DM,DF);
draw byCircle(M, E, black, 0, 0, 0)(M);
draw byLabelsOnCircle(F, E, A)(M);
draw byLabelsOnPolygon(F, D, M, A)(2, 0);
}
\drawCurrentPictureInMargin
Tсли из точки вне круга провести прямые линии \drawProportionalLine{DM,MA}, \drawProportionalLine{DE} и \drawProportionalLine{DF} к окружности, из тех, что падают на вогнутую часть окружности, наибольшей будет та \drawUnitLine{DM,MA}, что проходит через центр, а та, что ближе к ней \drawUnitLine{DE} будет длиннее той, что дальше \drawUnitLine{DF}.

\startCenterAlign
Проведем \drawUnitLine{MF} и \drawUnitLine{ME} к центру.\\
Тогда \drawUnitLine{DM,MA} проходящая через центр будет наибольшей,\\
поскольку, раз $\drawUnitLine{MA} = \drawUnitLine{ME}$, если к обеим добавить \drawUnitLine{DM}, $\drawUnitLine{DM,MA} = \drawUnitLine{DM} + \drawUnitLine{ME}$;\\
но $> \drawUnitLine{DE}$ \inprop[prop:I.XX]\\
$\therefore$ \drawUnitLine{DM,MA} больше любой другой линии, проведенной из той же точки к вогнутой части окржности.\\
Теперь в \drawLine{DM,MF,DF} и \drawLine{DM,ME,DE}, $\drawUnitLine{MF} = \drawUnitLine{ME}$, и \drawUnitLine{DM} общая,\\
но $\drawAngle{DMF,FME} > \drawAngle{DMF}$, $\therefore \drawUnitLine{DE} > \drawUnitLine{DF}$ \inprop[prop:I.XXIV];\\
Так же можно показать, что $\drawUnitLine{DE} >$ любой другой линии, более далекой от \drawUnitLine{DM,MA}.
\stopCenterAlign
\stopsubproposition

\vfill\pagebreak

\startsubproposition[title={II.}]
\defineNewPicture{
pair M, D, G, H, K;
numeric r;
r := 7/4u;
M := (0, 0);
D := M shifted (dir(90)*2r);
G := (dir(90)*r) shifted M;
H := (dir(130)*r) shifted M;
K := (dir(110)*r) shifted M;
draw byLine(G, M, black, 0, 0);
draw byLine(H, M, byblue, 0, 0);
draw byLine(D, K, byred, 1, 0);
draw byLine(K, M, byred, 0, 0);
byLineDefine(D, G, black, 1, 0);
byLineDefine(D, H, byblue, 1, 0);
draw byNamedLineSeq(0)(DG,DH);
draw byCircle(M, G, black, 0, 0, 0)(M);
draw byLabelsOnPolygon(K, H, M)(2, -2);
draw byLabelsOnPolygon(G, K, M)(2, -2);
draw byLabelsOnPolygon(D, G, K)(2, -2);
draw byLabelsOnPolygon(H, D, G)(2, 0);
draw byLabelsOnPolygon(G, M, H)(2, 0);
}
\drawCurrentPictureInMargin
Из линий, падающий на выпуклую часть окружности наименьшей \drawUnitLine{DG} является та, которая при продолжении проходит через центр, а линия, ближняя к наименьшей будет меньше дальней.

\startCenterAlign
Действительно, поскольку $\drawUnitLine{KM} + \drawUnitLine{DK} > \drawUnitLine{DG,GM}$ \inprop[prop:I.XX]\\
and $\drawUnitLine{KM} = \drawUnitLine{GM}$,\\
$\therefore \drawUnitLine{DK} > \drawUnitLine{DG}$ \inax[ax:V]

Значит, поскольку $\drawUnitLine{HM} + \drawUnitLine{DH} > \drawUnitLine{KM} + \drawUnitLine{DK}$ \inprop[prop:I.XXI],\\
и $\drawUnitLine{HM} = \drawUnitLine{KM}$,

$\therefore \drawUnitLine{DK} < \drawUnitLine{DH}$. Так же и для прочих.
\stopCenterAlign
\stopsubproposition

\startsubproposition[title={III.}]
\defineNewPicture{
pair M, D, B, G, H, N, O;
numeric r;
r := 7/4u;
M := (0, 0);
D := M shifted (dir(90)*2r);
G := (dir(90)*r) shifted M;
B := (dir(90 - 20)*r) shifted M;
H := (dir(90 + 45)*r) shifted M;
N := (dir(90 - 45)*r) shifted M;
O = whatever[D, N] = whatever[M, B];
draw byAngle(H, D, M, byyellow, 0);
draw byAngle(M, D, N, byblue, 0);
draw byLine(B, O, byred, 0, 0);
draw byLine(O, N, byyellow, 0, 0);
draw byLine(D, M, black, 0, 0);
draw byLine(H, M, byblue, 0, 0);
draw byLine(B, M, black, 1, 0);
byLineDefine(N, M, byyellow, 1, 0);
byLineDefine(D, H, byblue, 1, 0);
byLineDefine(D, O, byred, 1, 0);
draw byNamedLineSeq(0)(DH,DO,NM);
draw byCircle(M, H, black, 0, 0, 0)(M);
draw byLabelsOnPolygon(H, D, O, N)(2, 0);
draw byLabelsOnPolygon(N, M, H)(2, 0);
draw byLabelsOnPolygon(G, H, M)(2, -2);
draw byLabelsOnPolygon(M, B, G)(2, -2);
draw byLabelsOnPolygon(M, N, B)(2, -2);
}
\drawCurrentPictureInMargin
Кроме того, линии, образующие равные углы с проходящей через центр равны между собой, падают ли они на вогнутую или выпуклую часть окружности, и нельзя провести третьей линии из той же точки той же длины к окружности.

\startCenterAlign
Действительно, пусть $\drawUnitLine{DO,ON} > \drawUnitLine{DH}$, взяв $\drawAngle{HDM} = \drawAngle{MDN}$;\\
сделаем $\drawUnitLine{DO} = \drawUnitLine{DH}$, и проведем \drawUnitLine{BM,BO}.

Тогда в \drawLine{DM,DO,BO,BM} и \drawLine{HM,DH,DM} получим $\drawUnitLine{DO} = \drawUnitLine{DH}$\\
и общую \drawUnitLine{DM}, а также $\drawAngle{MDN} = \drawAngle{HDM}$,

$\therefore \drawUnitLine{BM,BO} = \drawUnitLine{HM}$ \inprop[prop:I.IV]

Но $\drawUnitLine{HM} = \drawUnitLine{BM}$;

$\therefore \drawUnitLine{BM} = \drawUnitLine{BM,BO}$, что невозможно.

$\therefore \drawUnitLine{DH} \mbox{ не } = \drawUnitLine{DO}$, ни какой либо часть \drawUnitLine{DO,ON}, $\therefore \drawUnitLine{DO,ON} \mbox{ не } > \drawUnitLine{DH}$.

Так же и $\drawUnitLine{DH} \mbox{ не } > \drawUnitLine{DO,ON}$, они $\therefore  =$ друг другу.
\stopCenterAlign
Так же и любая другая линия, проведенная из той же точки к окружности, должна будет лежать по одну сторону с одной из этих линий и быть ближе или дальше, чем они от линии, проходящей через центр и не может, следовательно, быть равной им.

\qed
\stopsubproposition
\stopProposition

\startProposition[title={Предложение IX. Теорема},reference=prop:III.IX]
\defineNewPicture[1/4]{
pair D, A, B, C, F, L, H;
numeric r;
r := 7/4u;
D := (0, 0);
A := (dir(170)*r) shifted D;
B := (dir(-90)*r) shifted D;
C := (dir(-45)*r) shifted D;
L := (dir(45)*r) shifted D;
H := (dir(45 + 180)*r) shifted D;
F := 2/4[D, L];
draw byLine(D, A, byyellow, 1, 0);
draw byLine(D, B, byyellow, 0, 0);
draw byLine(D, C, byblue, 0, 0);
draw byLine(D, F, black, 0, 0);
draw byLine(F, L, byred, 1, 0);
draw byLine(D, H, byred, 0, 0);
draw byCircleR(D, r, byblue, 0, 0, 0)(D);
draw byLabelsOnCircle(A, B, C, H, L)(D);
draw byLabelsOnPolygon(A, D, F, L)(2, 0);
}
\drawCurrentPictureInMargin
\problemNP{Е}{сли}{внутри круга \drawCircle[middle]{D} взята точка, из которой к окружности может быть проведено больше двух равных прямых линии \drawUnitLine{DA}, \drawUnitLine{DB}, \drawUnitLine{DC}, эта точка является центром круга.}

Действительно, если рассматриваемая точка \drawFromCurrentPicture{
draw byNamedLineSeq(0)(DB,DC);
draw byLabelsOnPolygon(B, D, C)(2, 0);
}, в которой встречается больше двух равных прямых линий, не центр, то какая- то другая \drawFromCurrentPicture{
startTempScale(1/2);
startGlobalRotation(-lineAngle.DF);
draw byNamedLineSeq(0)(DF,FL);
draw byLabelsOnPolygon(D, F, L)(2, 0);
stopGlobalRotation;
stopTempScale;
} должна быть центром, проведем между этими двумя точками \drawUnitLine{DF} и продлим в обе стороны до окружности.

Теперь, поскольку более чем две равных прямых линии проведено к окружности из точки, не являющейся центром, две из них должны лежать по одну сторону диаметра \drawUnitLine{DH,DF,FL}, и, поскольку из точки \drawFromCurrentPicture[middle][pointD]{
draw byNamedLine(DB);
draw byNamedLineSeq(0)(DH,DC);
draw byLabelsOnPolygon(H, D, C)(2, 0);
}, не являющейся центром проямые линии проведены к окружности, наибольшая из них \drawUnitLine{DF,FL}, та, что проходит через центр, а \drawUnitLine{DC}, которая ближе к \drawUnitLine{DF,FL}, $> \drawUnitLine{DB}$ расположенной дальше \inprop[prop:III.VIII], ер $\drawUnitLine{DC} = \drawUnitLine{DB}$ (гип.), что невозможно.

То же можно показать для любой другой точки, кроме \pointD, которая, таким образом, должна быть центром круга.

\qed
\stopProposition

\startProposition[title={Предложение X. Теорема},reference=prop:III.X]
\defineNewPicture[1/5]{
pair P, G, H, B, d, dd;
pair Pd, Gd, Hd, Bd, Pdd;
numeric r, t[];
path cr[], crd[];
r := 7/4u;
d := (0, -9/4r);
P := (0, 0);
cr1 := ((fullcircle scaled 7/3r xscaled 4/5) rotated 45) shifted P;
cr2 := ((fullcircle scaled 7/3r xscaled 4/5) rotated -45) shifted P;
H := (subpath (0, 2) of cr1) intersectionpoint cr2;
B := (subpath (0, -2) of cr1) intersectionpoint cr2;
G := (subpath (-2, -4) of cr1) intersectionpoint cr2;
Pd := P shifted d;
crd1 := (fullcircle scaled 2r) shifted Pd;
dd := (0, -3/2r);
crd2 := crd1 shifted dd;
Pdd := Pd shifted dd;
t1 := xpart(crd2 intersectiontimes (subpath (-2, -4) of crd1));
t2 := xpart(crd2 intersectiontimes (subpath (0, -2) of crd1));
Bd := point t1 of crd2;
Hd := point t2 of crd2;
Gd := point -2 of crd1;
crd2 := subpath (t1, t2 + 8) of crd2;
crd2 := crd2 .. (point -2 of crd1) .. cycle;
draw byLine(P, B, byyellow, 0, 0);
draw byLine(P, G, black, 0, 0);
draw byLine(P, H, byblue, 0, 0);
draw byArbitraryFigure(cr1, byred, 0, 0)(fI);
draw byArbitraryFigure(cr2, byblue, 0, 0)(fII);
draw byLine(Pdd, Gd, black, 0, 0);
byLineDefine(Pdd, Bd, byyellow, 0, 0);
byLineDefine(Pdd, Hd, byblue, 0, 0);
draw byNamedLineSeq(0)(PddBd,PddHd);
draw byArbitraryFigure(crd1, byred, 0, 0)(fdI);
draw byArbitraryFigure(crd2, byblue, 0, 0)(fdII);
byCircleDefineR(P, r, byred, 0, 0, 0)(PI);
byCircleDefineR(P, r, byblue, 0, 0, 0)(PII);
draw byLabelsOnCircle(G, B, H)(PI);
draw byLabelsOnPolygon(G, P, H)(2, 0);
byPointLabelDefine(Gd, "G");
byPointLabelDefine(Hd, "H");
byPointLabelDefine(Bd, "B");
byPointLabelDefine(Dd, "D");
byPointLabelDefine(Pdd, "P");
draw byLabelLineEnd(Gd, Pdd, 0);
draw byLabelPoint(Bd, 180, 2);
draw byLabelPoint(Hd, 0, 2);
draw byLabelsOnPolygon(Hd, Pdd, Bd)(2, 0);
}
\drawCurrentPictureInMargin
\problemNP{К}{руг}{\drawFromCurrentPicture{draw byNamedCircle(PII);} не сечет круга \drawFromCurrentPicture[middle][circlePI]{draw byNamedCircle(PI);} более чем в двух точках.}

Действительно, будь такое возможно, пусть они пересекаются в трех точках, из центра \drawCircle[middle][1/5]{PII} проведем \drawUnitLine{PG}, \drawUnitLine{PB} и \drawUnitLine{PH} к точкам пересечения.

$\therefore \drawUnitLine{PG} = \drawUnitLine{PB} = \drawUnitLine{PH}$ \indef[def:XV], но, поскольку круги пересекаются, у них не один центр \inprop[prop:III.V]:

$\therefore$ рассматриваемая точка не центр \circlePI , и $\therefore$, поскольку \drawUnitLine{PG}, \drawUnitLine{PB} и \drawUnitLine{PH} проведены не из центра, они не равны (\inpropL[prop:III.VII], \inpropN[prop:III.VIII]), но выше было показано, что они равны, что невозможно, круги, стало быть, не пересекаются в трех точках.

\qed
\stopProposition

\startProposition[title={Предложение XI. Теорема},reference=prop:III.XI]
\defineNewPicture[1/2]{
pair M, N, A, D, F, G, H, K;
numeric r[];
path cr[];
r1 := 9/4u;
r2 := 2/3r1;
M := (0, 0);
N := M shifted (0, +r1-r2);
cr1 := (fullcircle scaled 2r1) shifted M;
cr2 := (fullcircle scaled 2r2) shifted N;
A := (0, r1) shifted M;
F := 1/2[M,N] shifted (dir(-20)*1/3r2);
G := 1/2[M,N] shifted (dir(-20 + 180)*1/3r2);
D := cr2 intersectionpoint (F--10[F, G]);
H := cr1 intersectionpoint (F--10[F, G]);
K := cr1 intersectionpoint (F--10[G, F]);
draw byPolygon(A,F,G)(byyellow);
draw byLine(A, G, byred, 0, 0);
draw byLine(A, F, byblue, 1, 0);
draw byLine(H, D, byyellow, 0, 0);
draw byLine(D, G, byyellow, 1, 0);
draw byLine(G, F, black, 0, 0);
draw byLine(F, K, byblue, 0, 0);
byPointLabelDefine(M, "F");
byPointLabelDefine(N, "G");
draw byCircle(M, A, black, 0, 0, 0)(M);
draw byCircle(N, A, byblue, 0, 0, -1)(N);
draw byLabelsOnPolygon(K, F, G, D)(2, 0);
draw byLabelsOnPolygon(A, D, N)(2, -1); 
draw byLabelsOnCircle(A, H)(M);
}
\drawCurrentPictureInMargin
\problemNP{Е}{сли}{два круга \drawCircle[middle][1/4]{N} и \drawCircle[middle][1/5]{M} касаются между собой изнутри, прямая, соединяющая их центры, при продлении проходит через точку касания.}

\startCenterAlign
Действительно, если такое возможно, соединим центры с помощью \drawSizedLine{GF} и продлим в обе стороны.\\
из точки касания проведем \drawSizedLine{AG} к сентру \circleN,\\
из той же точки касания проведем \drawSizedLine{AF} к центру \circleM.\\
Поскольку в
\drawFromCurrentPicture{
draw byNamedPolygon(AFG);
draw byNamedLineFull(A, A, 1, 1, 0)(GF);
}
$\drawSizedLine{GF} + \drawSizedLine{AG} > \drawSizedLine{AF}$ \inprop[prop:I.XX],\\
и $\drawSizedLine{AF} = \drawSizedLine{HD,DG,GF}$, как радиусы\circleM,\\
но $\drawSizedLine{GF} + \drawSizedLine{AG} > \drawSizedLine{HD,DG,GF}$;\\
вычтем \drawSizedLine{GF} общую обеим,\\
и $\drawSizedLine{AG} > \drawSizedLine{HD,DG}$;\\
но $\drawSizedLine{AG} = \drawSizedLine{DG}$, как радиусы \circleN,\\
и $\therefore \drawSizedLine{DG} > \drawSizedLine{HD,DG}$ часть больше целого, что невозможно.

\stopCenterAlign

Центры, следовательно, расположены так, что линия, соединяющая их не может проходить через какую-либо точку, кроме точки касания.

\qed
\stopProposition

\startProposition[title={Предложение XII. Теорема},reference=prop:III.XII]
\defineNewPicture[1/4]{
pair M, N, A, C, D, F, G;
numeric r[];
path cr[];
r1 := 3/2u;
r2 := 2u;
M := (0, 0);
N := (0, -r1-r2);
cr1 := (fullcircle scaled 2r1) shifted M;
cr2 := (fullcircle scaled 2r2) shifted N;
A := M shifted (0, -r1);
F := M shifted (dir(185)*1/2r1);
G := N shifted (dir(175)*1/2r2);
C := cr1 intersectionpoint (F--G);
D := cr2 intersectionpoint (F--G);
byLineDefine(F, C, byred, 0, 0);
byLineDefine(C, D, black, 0, 0);
byLineDefine(D, G, byblue, 0, 0);
byLineDefine(A, F, byyellow, 1, 0);
byLineDefine(A, G, byyellow, 0, 0);
draw byNamedLineSeq(0)(FC,CD,DG,AG,AF);
draw byCircle(M, A, byblue, 0, 0, -1/2)(M);
draw byCircle(N, A, byred, 0, 0, -1/2)(N);
byPointLabelDefine(M, "F");
byPointLabelDefine(N, "G");
draw byLabelsOnPolygon(C, F, A)(2, 0);
draw byLabelsOnPolygon(A, G, D)(2, 0);
draw byLabelsOnPolygon(A, C, F)(2, 0);
draw byLabelsOnPolygon(G, D, D)(2, 0);
draw byLabelPoint(A, angle(F-A)-45, 2);
}
\drawCurrentPictureInMargin
\problemNP{Е}{сли}{два круга \drawCircle{M} и \drawCircle{N} касаются друг друг извне, прямая \drawUnitLine{FC,CD,DG} соединяющая их центры проходит через точку касания.}

Действительно, если такое возможно, соединим центры с помощью \drawUnitLine{FC,CD,DG}, не проходящей через точку касания, из точки касания проведем \drawUnitLine{AF} и \drawUnitLine{AG} к центрам.

\startCenterAlign
Поскольку $\drawUnitLine{AF} + \drawUnitLine{AG} > \drawUnitLine{FC,CD,DG}$ \inprop[prop:I.XX]\\
и $\drawUnitLine{FC} = \drawUnitLine{AF}$ \inprop[prop:I.XV],\\
и $\drawUnitLine{DG} = \drawUnitLine{AG}$ \inprop[prop:I.XV],

$\therefore \drawUnitLine{FC} + \drawUnitLine{DG} > \drawUnitLine{FC,CD,DG}$, часть больше целого, что невозможно.
\stopCenterAlign

Центры, следовательно, расположены так, что линия, соединяющая их не может проходить через какую-либо точку, кроме точки касания.

\qed
\stopProposition

\startProposition[title={Предложение XIII. Теорема},reference=prop:III.XIII]
\problemNP{К}{руг}{не касается круга более чем в одной точке, как внутри, так и снаружи.}

\defineNewPicture{
pair M, N, F, G, H, B, D;
numeric r[];
path cr[];
r1 := 7/4u;
r2 := 3/4r1;
M := (0, 0);
N := (dir(120)*(r1-r2)) shifted M;
cr1 := (fullcircle scaled 2r1) shifted M;
cr2 := (fullcircle scaled 2r2) shifted N;
t1 := xpart(cr1 intersectiontimes (M--10[M, N]));
t2 := xpart(cr2 intersectiontimes (M--10[M, N]));
cr2 := (subpath (t2 + 2/3, t2 - 2/3 + 8) of cr2) .. tension 3/2 .. cycle;
B := point (t1 - 1/2) of cr1;
D := point (t1 + 1/2) of cr1;
G := 3/4[B, 1/2[M, N]];
H := 5/4[B, 1/2[M, N]];
draw byLine(D, G, black, 0, 0);
byLineDefine(D, H, byred, 0, 0);
byLineDefine(B, G, byblue, 1, 0);
byLineDefine(G, H, byblue, 0, 0);
draw byNamedLineSeq(0)(BG,GH,DH);
draw byCircleR(M, r1, byyellow, 0, 0, 1)(M);
draw byArbitraryFigure(cr2, byblue, 0, 0)(fI);
byCircleDefineR(M, r1, byyellow, 0, 0, 0)(M);
byCircleDefineR(N, r2, byblue, 0, 0, 0)(N);
byPointLabelDefine(M, "H");
byPointLabelDefine(N, "G");
draw byLabelsOnCircle(D, B)(M);
draw byLabelsOnPolygon(B, G, H, D)(2, 0);
}

\startsubproposition[title={Случай I.}]
\drawCurrentPictureInMargin
Действительно, если это возможно, пусть \drawCircle{M} и \drawCircle{N} касаются друг друга внутри в двух точках, проведем \drawUnitLine{GH} соединяющую их центры, и продлим до одной из точек касания \inprop[prop:III.XI];

\startCenterAlign
проведем \drawUnitLine{DH} и \drawUnitLine{DG},

Но $\drawUnitLine{BG} = \drawUnitLine{DG}$ \indef[def:XV],\\
$\therefore$ если добавить к каждой \drawUnitLine{GH},\\
$\drawUnitLine{GH,BG} = \drawUnitLine{GH} + \drawUnitLine{DG}$;

но $\drawUnitLine{GH,BG} = \drawUnitLine{DH}$ \indef[def:XV],\\
и $\therefore \drawUnitLine{GH} + \drawUnitLine{DG} = \drawUnitLine{DH}$;

но $\therefore \drawUnitLine{GH} + \drawUnitLine{DG} > \drawUnitLine{DH}$ \inprop[prop:I.XX],\\
что невозможно.
\stopCenterAlign
\stopsubproposition

\vfill\pagebreak

\defineNewPicture{
pair P, Q, A, C;
numeric r[], t[];
path cr[];
r1 := 7/4u;
P := (0, 0);
Q := P shifted (1/4r1, 0);
cr3 := (fullcircle scaled 2r1) shifted P;
cr4 := (fullcircle scaled 2r1) shifted Q;
t3 := xpart(cr3 intersectiontimes (subpath (0, 4) of cr4));
t4 := xpart(cr3 intersectiontimes (subpath (4, 8) of cr4));
t5 := xpart(cr4 intersectiontimes (subpath (0, 4) of cr3));
t6 := xpart(cr4 intersectiontimes (subpath (4, 8) of cr3));
A := point t3 of cr3;
C := point t4 of cr3;
draw byLine(A, C, byred, 0, 0);
draw byMarkLine(3/7, black)(AC);
draw byMarkLine(4/7, black)(AC);
draw byArcBE(P, t3, t4, r1, byblue, 0, 0, 0, 0)(PI);
draw byArcBE(P, t4, t3 + 8, r1, black, 0, 0, 0, 0)(PII);
draw byArcBE(Q, t5, t6 + 8, r1, byblue, 0, 0, 0, 0)(QI);
draw byArcBE(Q, t5, t6, r1, black, 0, 0, 0, 0)(QII);
}

\startsubproposition[title={Случай II.}]
\drawCurrentPictureInMargin
Если точки касания на концах прямой линии, соединяющей центры, такая прямая должна быть рассечена пополам в двух разных точках, соответствующих центрам, поскольку она является диаметром обоих кругов, что невозможно.
\stopsubproposition

\defineNewPicture{
pair G, H, A, C;
numeric r[];
path cr[];
r1 := 7/4u;
r2 := 3/4r1;
H := (0, 0);
G := H shifted (0, r1 + r2);
cr5 := (fullcircle scaled 2r1) shifted H;
cr6 := (fullcircle scaled 2r2) shifted G;
A := 1/2[point 2 of cr5, point 6 of cr6];
C := A shifted (1/3r2, 0);
cr5 := (subpath (2 + 2/4, 2 - 2/4 + 8) of cr5) .. tension 2 .. cycle;
cr6 := (subpath (6 + 2/4, 6 - 2/4 + 8) of cr6) .. tension 2 .. cycle;
byLineDefine(H, A, byred, 0, 0);
byLineDefine(A, G, byred, 1, 0);
byLineDefine(C, H, byblue, 1, 0);
byLineDefine(C, G, black, 0, 0);
draw byNamedLineSeq(0)(HA,AG,CG,CH);
draw byArbitraryFigure(cr5, byyellow, 0, 0)(fII);
draw byArbitraryFigure(cr6, byblue, 0, 0)(fIII);
byCircleDefineR(H, r1, byyellow, 0, 0, 0)(H);
byCircleDefineR(G, r2, byblue, 0, 0, 0)(G);
draw byLabelsOnPolygon(H, G, C)(0, 0);
draw byLabelPoint(A, angle(G-A)+45, 3);
}
\startsubproposition[title={Случай III.}]\drawCurrentPictureInMargin
Теперь, если такое возможно, пусть круги \drawCircle{H} и \drawCircle{G} касаются друг друга снаружи в двух точках. Проведем \drawUnitLine{HA,AG}, соединяющую центры, и проходящую через одну из точек касания, проведем также \drawUnitLine{CH} и \drawUnitLine{CG}.

\startCenterAlign
$\drawUnitLine{CH} = \drawUnitLine{HA}$ \indef[def:XV];\\
и $\drawUnitLine{AG} = \drawUnitLine{CG}$ \indef[def:XV];

$\therefore \drawUnitLine{CG} + \drawUnitLine{CH} = \drawUnitLine{HA,AG}$;

но $\drawUnitLine{CG} + \drawUnitLine{CH} > \drawUnitLine{HA,AG}$ \inprop[prop:I.XX], что невозможно.
\stopCenterAlign
\stopsubproposition

Следовательно, ни в каком случае круги не касаются друг друга в двух точках.

\qed
\stopProposition

\startProposition[title={Предложение XIV. Теорема},reference=prop:III.XIV]
\defineNewPicture{
pair A, B, C, D, E, F, G;
numeric r;
r := 9/4u;
E := (0, 0);
A := (dir(90-20)*r) shifted E;
B := (dir(90-130)*r) shifted E;
C := (dir(90+20)*r) shifted E;
D := (dir(90+130)*r) shifted E;
F = whatever[A, B] = whatever[E, E shifted ((A-B) rotated 90)];
G = whatever[C, D] = whatever[E, E shifted ((C-D) rotated 90)];
draw byAngleWithName(E, F, A, byyellow, 0)(F);
draw byAngleWithName(E, G, C, black, 1)(G);
draw byLine(E, A, black, 0, 0);
draw byLine(E, C, byblue, 0, 0);
byLineDefine(E, F, black, 1, 0);
byLineDefine(E, G, byblue, 1, 0);
draw byNamedLineSeq(0)(EF,EG);
draw byLine(A, F, byred, 0, 0);
draw byLine(F, B, byred, 1, 0);
draw byLine(C, G, byyellow, 0, 0);
draw byLine(G, D, byyellow, 1, 0);
draw byCircleR(E, r, byblue, 0, 0, 0)(E);
draw byLabelsOnCircle(A, B, C, D)(E);
draw byLabelsOnPolygon(F, E, G)(2, 0);
draw byLabelLineEnd(G, E, 0);
draw byLabelLineEnd(F, E, 0);
}
\drawCurrentPictureInMargin
\problemNP{Р}{авные}{прямые
$\left(\vcenter{\nointerlineskip\hbox{\drawProportionalLine{AF,FB}}\nointerlineskip\hbox{\drawProportionalLine{CG,GD}}}\right)$
в круге равноотстоят от центра и равноотстоящие от центра прямые равны.}

\startCenterAlign
Из центра \drawCircle[middle][1/4]{E} проведем \drawProportionalLine{EF} $\perp$ to \drawProportionalLine{AF,FB} и $\drawProportionalLine{EG} \perp \drawProportionalLine{CG,GD}$, соединим \drawProportionalLine{EA} и \drawProportionalLine{EC}.

Тогда $\drawProportionalLine{CG} = \frac{1}{2} \drawProportionalLine{CG,GD}$ \inprop[prop:III.III]\\
и $\drawProportionalLine{AF} = \frac{1}{2} \drawProportionalLine{AF,FB}$ \inprop[prop:III.III]

поскольку $\drawProportionalLine{CG,GD} = \drawProportionalLine{AF,FB}$ (гип.)\\
$\therefore \drawProportionalLine{CG} = \drawProportionalLine{AF}$,\\
и $\drawProportionalLine{EA} = \drawProportionalLine{EC}$ \indef[def:XV]\\
$\therefore \drawProportionalLine{EA}^2 = \drawProportionalLine{EC}^2$;

но поскольку \drawAngle{F} прямой\\
$\drawProportionalLine{EA}^2 = \drawProportionalLine{EF}^2 + \drawProportionalLine{AF}^2$ \inprop[prop:I.XLVII]\\
и $\drawProportionalLine{EC}^2 = \drawProportionalLine{EG}^2 + \drawProportionalLine{CG}^2$ по той же причине,\\
$\therefore \drawProportionalLine{EF}^2 + \drawProportionalLine{AF}^2 = \drawProportionalLine{EG}^2 + \drawProportionalLine{CG}^2$

$\therefore \drawProportionalLine{EF}^2 = \drawProportionalLine{EG}^2$

$\therefore \drawProportionalLine{EF} = \drawProportionalLine{EG}$
\stopCenterAlign

Также, если прямые \drawProportionalLine{AF,FB} и \drawProportionalLine{CG,GD} равноудалены от центра, то еcть перпендикуляры \drawProportionalLine{EF} и \drawProportionalLine{EG} равны, $\drawProportionalLine{AF,FB} = \drawProportionalLine{CG,GD}$.

\startCenterAlign
Поскольку, как и в предыдущем случае,\\
$\drawProportionalLine{EG}^2 + \drawProportionalLine{CG}^2 = \drawProportionalLine{AF}^2 + \drawProportionalLine{EF}^2$\\
но $\drawProportionalLine{EG}^2 = \drawProportionalLine{EF}^2$

$\therefore \drawProportionalLine{CG} = \drawProportionalLine{AF}$, и удвоенные $\drawProportionalLine{AF,FB} = \drawProportionalLine{CG,GD}$ тоже равны.
\stopCenterAlign

\qed
\stopProposition

\startProposition[title={Предложение XV. Теорема},reference=prop:III.XV]
\problemNP{В}{круге}{наибольшая прямая — диаметр, а из других более близкая к центру больше более удаленной.}

\startsubproposition[title={Случай I.}]
\defineNewPicture[1/2]{
pair A, D, E, F, G, M, N;
numeric r;
r := 9/4u;
E := (0, 0);
A := (r, 0) shifted E;
D := (-r, 0) shifted E;
F := (dir(90-30)*r) shifted E;
G := (dir(90+30)*r) shifted E;
M := (dir(90-60)*r) shifted E;
N := (dir(90+60)*r) shifted E;
draw byAngle(N, E, G, byred, 0);
draw byAngle(G, E, F, byyellow, 0);
draw byAngle(F, E, M, byred, 0);
draw byLine(E, M, byyellow, 1, 0);
draw byLine(E, N, byyellow, 0, 0);
draw byLine(E, F, byblue, 1, 0);
draw byLine(E, G, black, 1, 0);
draw byLine(D, E, byred, 0, 0);
draw byLine(E, A, black, 0, 0);
draw byLine(F, G, byred, 1, 0);
draw byLine(M, N, byblue, 0, 0);
draw byCircleR(E, r, black, 0, 0, 0)(E);
draw byLabelsOnCircle(A, D, M, N, F, G)(E);
draw byLabelsOnPolygon(A, E, D)(2, 0);
}
\drawCurrentPictureInMargin
\startCenterAlign
Диаметр \drawUnitLine{DE,EA} $>$ любой прямой \drawUnitLine{MN}.

Действительно, проведем \drawUnitLine{EN} и \drawUnitLine{EM}.

Тогда $\drawUnitLine{EM} = \drawUnitLine{EA}$\\
и $\drawUnitLine{EN} = \drawUnitLine{DE}$,\\
$\therefore \drawUnitLine{EN} + \drawUnitLine{EM} = \drawUnitLine{DE,EA}$\\
но $\drawUnitLine{EN} + \drawUnitLine{EM} > \drawUnitLine{MN}$ \inprop[prop:I.XX]

$\therefore \drawUnitLine{DE,EA} > \drawUnitLine{MN}$.
\stopCenterAlign

Теперь покажем, что та прямая что ближе больше той, что дальше.

Пусть это будут \drawUnitLine{MN} и \drawUnitLine{FG}, расположенные по одну сторону от центра и не пересекающиеся.

\startCenterAlign
Проведем \drawUnitLine{EN}, \drawUnitLine{EM}, \drawUnitLine{EG} и \drawUnitLine{EF}.

В
\drawFromCurrentPicture{
draw byNamedAngle(NEG,GEF,FEM);
startAutoLabeling;
draw byNamedLineSeq(0)(MN,EM,EN);
stopAutoLabeling;
}
и
\drawFromCurrentPicture{
draw byNamedAngle(GEF);
startAutoLabeling;
draw byNamedLineSeq(0)(FG,EF,EG);
stopAutoLabeling;
},\\
$\drawUnitLine{EN} \mbox{ и } \drawUnitLine{EM} = \drawUnitLine{EG} \mbox{ и } \drawUnitLine{EF}$;\\
но $\drawAngle{NEG,GEF,FEM} > \drawAngle{GEF}$,

$\therefore \drawUnitLine{MN} > \drawUnitLine{EF}$ \inprop[prop:I.XXIV]
\stopCenterAlign
\stopsubproposition

\vfill\pagebreak

\startsubproposition[title={Случай II.}]
\defineNewPicture{
pair A, B, C, D, E, F, G, M, N, K, L, H;
numeric r;
r := 9/4u;
E := (0, 0);
A := (r, 0) shifted E;
D := (-r, 0) shifted E;
F := (dir(90-30)*r) shifted E;
G := (dir(90+30)*r) shifted E;
B := (dir(-90-60)*r) shifted E;
C := (dir(-90+60)*r) shifted E;
M := (dir(90-60)*r) shifted E;
N := (dir(90+60)*r) shifted E;
H := 1/2[B, C];
K := 1/2[F, G];
L := 1/2[M, N];
draw byLine(E, L, byyellow, 1, 0);
draw byLine(L, K, byred, 1, 0);
draw byLine(E, H, byblue, 1, 0);
draw byLine(F, G, byyellow, 0, 0);
draw byLine(M, N, blue, 0, 0);
draw byLine(B, C, byred, 0, 0);
draw byLine(D, A, black, 0, 0);
draw byCircleR(E, r, black, 0, 0, 0)(E);
draw byLabelsOnCircle(A, D, B, C, F, G, M, N)(E);
draw byLabelLineEnd(K, H, 0);
draw byLabelLineEnd(H, K, 0);
draw byLabelPoint(E, angle(E-H)-45, 2);
draw byLabelPoint(L, angle(E-H)-45, 2);
}
\drawCurrentPictureInMargin
Теперь возьмем \drawUnitLine{BC} и \drawUnitLine{FG} расоложенные по разные стороны от центра или пересекающиеся. Проведем из центра \drawUnitLine{EL,LK} и $\drawUnitLine{EH} \perp \drawUnitLine{FG} \mbox{ и } \drawUnitLine{BC}$,

\startCenterAlign
сделаем $\drawUnitLine{EH} = \drawUnitLine{EL}$,\\
и проведем $\drawUnitLine{MN} \perp \drawUnitLine{EL,LK}$.

Поскольку \drawUnitLine{BC} и \drawUnitLine{MN} равноудалены от центра, $\drawUnitLine{BC} = \drawUnitLine{MN}$ \inprop[prop:III.XIV];\\
но $\drawUnitLine{MN} > \drawUnitLine{FG}$ \inprop[prop:III.XV],

$\therefore \drawUnitLine{BC} > \drawUnitLine{FG}$.
\stopCenterAlign
\stopsubproposition

\qed
\stopProposition

\startProposition[title={Предложение XVI. Теорема},reference=prop:III.XVI]
\defineNewPicture[1/5]{
pair A, B, C, D, E, F, G, H, K;
numeric r;
r :=2u;
D := (0, 0);
A := (0, -r);
B := (0, r);
C := (dir(190)*r) shifted D;
E := (4/3r, -r);
F := (4/3r, -1/3r);
G := 11/12[A, F];
H := (dir(angle(G-D))*r) shifted D;
K := (-r, -r);
draw byAngle(C, A, D, byyellow, 0);
draw byAngle(D, A, G, byblue, 0);
draw byAngle(G, A, E, byred, 0);
draw byAngleWithName(D, C, A, black, 0)(C);
draw byAngleWithName(A, G, D, black, 1)(G);
draw byLine(A, C, byred, 0, 0);
draw byLine(D, C, byblue, 0, 0);
draw byLine(D, H, byblue, 1, 0);
draw byLine(H, G, black, 1, 0);
draw byLine(A, G, byred, 1, 0);
draw byLine(G, F, black, 1, 0);
draw byLine(B, D, byyellow, 1, 0);
draw byLine(D, A, black, 0, 0);
draw byLineFull(E, K, byyellow, 0, 0)(E, K, 0, 0, -1);
draw byCircle(D, A, byblue, 0, 0, 0)(D);
draw byLabelsOnCircle(B, C)(D);
draw byLabelsOnPolygon(E, A, K, noPoint)(0, -1);
draw byLabelsOnPolygon(F, G, A)(2, 0);
draw byLabelsOnPolygon(C, D, B)(2, 0);
draw byLabelPoint(H, angle(G-H)+45, 2);
}
\drawCurrentPictureInMargin
\problemNP{П}{рямая}{\drawUnitLine{EK}, проведенная под прямым углом к диаметру круга \drawUnitLine{BD,DA} проходит вне круга.\\
И если любая прямая линия \drawUnitLine{AG} проведена из точки по ту же сторону от перпендикуляра к точке касания, она сечет круг.}

\startsubproposition[title={Часть I.}]
Если это возможно, пусть \drawUnitLine{AC}, пересекающая окружность в еще одном месте будет $\perp \drawUnitLine{DA}$, и проведем \drawUnitLine{DC}.

\startCenterAlign
Тогда, поскольку $\drawUnitLine{DA} = \drawUnitLine{DC}$,\\
$\drawAngle{CAD} = \drawAngle{C}$ \inprop[prop:I.V],\\
и $\therefore$ оба угла острые \inprop[prop:I.XVII]\\
но $\drawAngle{CAD} = \drawRightAngle$ (гип.), что невозможно,

следовательно \drawUnitLine{AC} проведенная $\perp \drawUnitLine{DA}$ не пересекает окружность в другом месте.
\stopCenterAlign
\stopsubproposition

\startsubproposition[title={Часть II.}]
Пусть $\drawUnitLine{EK} \perp \drawUnitLine{DA}$ и пусть \drawUnitLine{AG} проведена из точки \drawFromCurrentPicture{
draw byNamedLineSeq(0)(AG,HG,DH);
draw byLabelsOnPolygon(D, G, A)(2, 0);
} между \drawUnitLine{EK} и кругом, и, если возможно, не сечет круг.

\startCenterAlign
Поскольку $\drawAngle{DAG,GAE} = \drawRightAngle$,\\
$\therefore \drawAngle{DAG}$ острый угол;\\
допустим $\drawUnitLine{DH,HG} \perp \drawUnitLine{AG}$, проведенной из центра круга, должна падать на сторону острого угла \drawAngle{DAG}.

$\therefore$ \drawAngle{G}, который должен быть прямым $> \drawAngle{DAG}$,

$\therefore \drawUnitLine{DA} > \drawUnitLine{DH,HG}$;

но $\drawUnitLine{DH} = \drawUnitLine{DA}$,
\stopCenterAlign

и $\therefore \drawUnitLine{DH} > \drawUnitLine{DH,HG}$, часть больше целого, что невозможно. Следовательно, точка не находится вовне круга и, следовательно, прямая \drawUnitLine{AG} сечет круг.
\stopsubproposition

\qed
\stopProposition

\startProposition[title={Предложение XVII. Задача},reference=prop:III.XVII]
\defineNewPicture{
pair A, B, D, E, F;
numeric r[], a;
path cr[];
r1 := 6/4u;
r2 := 9/4u;
E := (0, 0);
cr1 := (fullcircle scaled 2r1) shifted E;
cr2 := (fullcircle scaled 2r2) shifted E;
A := (dir(50)*r2) shifted E;
D := (dir(50)*r1) shifted E;
F := cr2 intersectionpoint (D--D shifted (dir(angle(A-E) - 90)*r2));
B := cr1 intersectionpoint (E--F);
a := angle(B-E);
forsuffixes i=A, B, D, F:
i := ((i shifted -E) rotated -a) shifted E;
endfor;
draw byAngleWithName(A, B, E, byyellow, 0)(B);
draw byAngleWithName(F, D, E, byyellow, 0)(D);
draw byAngleWithName(F, E, A, byblue, 0)(E);
draw byLine(A, B, byblue, 0, 0);
draw byLine(F, D, byblue, 1, 0);
byLineDefine(A, D, byred, 0, 0);
byLineDefine(D, E, byred, 1, 0);
byLineDefine(F, B, black, 0, 0);
byLineDefine(B, E, black, 1, 0);
draw byNamedLineSeq(0)(AD,DE,BE,FB);
draw byCircleR(E, r1, byred, 0, 0, -1)(EI);
draw byCircleR(E, r2, byyellow, 0, 0, 0)(EII);
draw byLabelsOnCircle(A, F)(EII);
draw byLabelsOnPolygon(F, E, A)(2, 0);
draw byLabelPoint(D, angle(A-E)+45, 2);
draw byLabelPoint(B, angle(F-E)-45, 2);
}
\drawCurrentPictureInMargin
\problemNP{П}{ровести}{касательную к данному кругу \drawCircle[middle][1/5]{EI} из данной точки.}

Если данная точка расположена на окружности \drawFromCurrentPicture[bottom]{
draw byNamedLineSeq(0)(AB,BE);
draw byLabelsOnPolygon(E, B, A)(2, 0);
}, ясно, что прямая $\drawUnitLine{AB} \perp \drawUnitLine{BE}$ радиусу и будет искомой касательной \inprop[prop:III.XVI].

Но если точка \drawFromCurrentPicture[bottom]{
draw byNamedLineSeq(0)(DE,AD,AB);
draw byLabelsOnPolygon(E, A, B)(2, 0);
} расположена вовне,

\startCenterAlign
проведем из нее \drawUnitLine{AD,DE} к центру, секущую \circleEI, 
и проведем $\drawUnitLine{FD} \perp \drawUnitLine{DE}$,\\
опишем \drawCircle[middle][1/6]{EII} концентрический с \circleEI\ с радиусом $= \drawUnitLine{AD,DE}$,\\
проведем \drawUnitLine{BE,FB} к центру из точки, где \drawUnitLine{FD} падает на окружность \circleEII,\\
проведем $\drawUnitLine{AB} \perp \drawUnitLine{BE,FB}$ из точки, где та сечет \circleEI,\\
Тогда \drawUnitLine{AB} и будет искомой касательной.\\
Поскольку в \drawLine[bottom]{FD,FB,BE,DE} и \drawLine[bottom]{BE,DE,AD,AB} $\drawUnitLine{AD,DE} = \drawUnitLine{FB,BE}$, \drawAngle{E} общий,\\
и $\drawUnitLine{DE} = \drawUnitLine{BE}$,\\
$\therefore \mbox{ \inprop[prop:I.IV] } \drawAngle{B} = \drawAngle{D} = \drawRightAngle$,\\
$\therefore \drawUnitLine{FD}$ касательная к \circleEI.
\stopCenterAlign

\qed
\stopProposition

\startProposition[title={Предложение XVIII. Теорема},reference=prop:III.XVIII]
\defineNewPicture{
pair B, C, D, F, G;
numeric r;
r := 7/4u;
F := (0, 0);
C := (r, 0);
G := (r, 6/5r);
D := 6/5[C, G];
B := (dir(angle(G-F))*r) shifted F;
draw byCircle(F, C, byyellow, 0, 0, 0)(F);
draw byAngleWithName(F, C, G, byred, 0)(C);
draw byAngleWithName(C, G, F, byyellow, 0)(G);
byLineDefine(F, B, byred, 0, 0);
byLineDefine(B, G, byred, 1, 0);
byLineDefine(C, D, byblue, 1, 0);
byLineDefine(F, C, byblue, 0, 0);
draw byNamedLineSeq(0)(BG,FB,FC,CD);
draw byLabelsOnCircle(C)(F);
draw byLabelsOnPolygon(C, F, G)(2, 0);
draw byLabelPoint(B, angle(G-F)+45, 2);
draw byLabelPoint(G, angle(D-C)-90, 1);
draw byLabelPoint(D, angle(D-C)-90, 1);
}
\drawCurrentPictureInMargin
\problemNP{Е}{сли}{прямая \drawUnitLine{CD} касается круга, прямая \drawUnitLine{FC}, проведенная из центра к точке касания перпендикулярна ей.}

\startCenterAlign
Действительно, если возможно, пусть \drawUnitLine{FB,BG} будем $\perp \drawUnitLine{CD}$,\\
тогда, поскольку $\drawAngle{G} = \drawRightAngle$, \drawAngle{C} острый \inprop[prop:I.XVII]

$\therefore \drawUnitLine{FC} > \drawUnitLine{FB,BG}$ \inprop[prop:I.XIX];

но $\drawUnitLine{FC} = \drawUnitLine{FB}$,\\
и $\therefore \drawUnitLine{FB} > \drawUnitLine{FB,BG}$, часть больше целого, что невозможно.

$\therefore \drawUnitLine{FB,BG}$ не $\perp \drawUnitLine{CD}$;
\stopCenterAlign

и так же можно показать, что никакая другая прямая, кроме \drawUnitLine{FC} не перпендикулярна \drawUnitLine{CD}.

\qed
\stopProposition

\startProposition[title={Предложение XIX. Теорема},reference=prop:III.XIX]
\defineNewPicture{
pair A, C, E, F, G;
numeric r;
r := 7/4u;
G := (0, 0);
A := (-r, 0) shifted G;
C := (r, 0) shifted G;
E := (r, 6/5r) shifted G;
F := (-1/7r, 1/2r) shifted G;
draw byAngle(A, C, F, byblue, 0);
draw byAngle(F, C, E, byyellow, 0);
draw byLine(A, C, byyellow, 0, 0);
draw byLine(C, E, byblue, 0, 0);
draw byLine(C, F, byred, 1, 0);
draw byCircleR(G, r, byred, 0, 0, 0)(G);
draw byLabelsOnCircle(A, C)(G);
draw byLabelLineEnd(F, C, 0);
draw byLabelLineEnd(E, C, 0);
}
\drawCurrentPictureInMargin
\problemNP{Е}{сли}{прямая \drawUnitLine{CE} касается круга, прямая  \drawUnitLine{AC}, перпендикулярная ей, проведенная из точки касания, проходит через центр круга.}

Действительно, если центр не находится на \drawUnitLine{AC}, проведем \drawUnitLine{CF} к редполагаемому центру из точки касания.

\startCenterAlign
Поскольку $\drawUnitLine{CF} \perp \drawUnitLine{CE}$ \inprop[prop:III.XVIII]\\
$\therefore \drawAngle{FCE} = \drawRightAngle$, прямому углу;\\
но $\drawAngle{ACF,FCE} = \drawRightAngle$ (гип.),

и $\therefore \drawAngle{FCE} = \drawAngle{ACF,FCE}$, часть равна целому, что невозможно.
\stopCenterAlign

Следовательно, предполагаемая точка не центр, и то же можно показать для любой другой точки, не лежащей на \drawUnitLine{AC}.

\qed
\stopProposition

\startProposition[title={Предложение XX. Теорема},reference=prop:III.XX]
\problemNP{У}{гол,}{при центре круга вдвое больше угла при окружности, когда их основание лежит на одной дуге.}\unskip

\defineNewPicture{
pair A, E, F, C;
r := 7/4u;
E := (0, 0);
A := (dir(80)*r) shifted E;
F := (dir(80 + 180)*r) shifted E;
C := (dir(-30)*r) shifted E;
draw byAngle(C, A, F, byyellow, 0);
draw byAngle(C, E, F, byblue, 0);
draw byAngleWithName(E, C, A, byred, 0)(C);
draw byLine(A, C, black, 0, 1);
draw byLine(E, C, black, 0, 0);
draw byLine(E, F, byred, 1, 0);
draw byLine(E, A, byred, 0, 0);
draw byCircleR(E, r, byblue, 0, 0, 0)(E);
draw byLabelsOnCircle(F, A, C)(E);
draw byLabelsOnPolygon(F, E, A)(2, 0);
}
\drawCurrentPictureInMargin
\startsubproposition[title={Случай I.}]
\startCenterAlign
Пусть центр круга будет на \drawUnitLine{EF,EA}\\
стороне \drawAngle{CAF}.

Поскольку $\drawUnitLine{EC} = \drawUnitLine{EA}$,\\
$\drawAngle{CAF} = \drawAngle{C}$ \inprop[prop:I.V].

Но $\drawAngle{CEF} = \drawAngle{CAF} + \drawAngle{C}$,\\
или $\drawAngle{CEF} = \mbox{ дважды } \drawAngle{CAF}$ \inprop[prop:I.XXXII].
\stopCenterAlign
\stopsubproposition

\defineNewPicture{
pair A, E, F, C, B;
r := 7/4u;
E := (0, 0);
A := (dir(80)*r) shifted E;
F := (dir(80 + 180)*r) shifted E;
B := (dir(185)*r) shifted E;
C := (dir(-30)*r) shifted E;
draw byAngle(C, A, F, byyellow, 0);
draw byAngle(C, E, F, byblue, 0);
draw byAngleWithName(E, C, A, byyellow, 0)(C);
draw byAngle(B, A, F, byred, 0);
draw byAngle(B, E, F, black, 0);
draw byAngleWithName(E, B, A, byred, 0)(B);
draw byLine(A, C, black, 0, 1);
draw byLine(E, C, black, 0, 1);
draw byLine(A, B, black, 0, 1);
draw byLine(E, B, black, 0, 1);
draw byLine(A, F, black, 0, 0);
draw byCircleR(E, r, byblue, 0, 0, 0)(E);
draw byLabelsOnCircle(F, A, B, C)(E);
draw byLabelsOnPolygon(B, E, A)(2, 0);
}
\drawCurrentPictureInMargin
\startsubproposition[title={Случай II.}]
\startCenterAlign
Пусть центр будет в \drawAngle{BAF,CAF}, углу на окружности;

проведем \drawUnitLine{AF} из угла через центр круга;\\
тогда $\drawAngle{B} = \drawAngle{BAF}$, и $\drawAngle{C} = \drawAngle{CAF}$, вследствие равенства сторон \inprop[prop:I.V].

Значит $\drawAngle{BAF} + \drawAngle{B} + \drawAngle{CAF} + \drawAngle{C} = \mbox{ дважды } \drawAngle{BAF,CAF}$.

Но $\drawAngle{BEF} = \drawAngle{BAF} + \drawAngle{B}$,\\
и $\drawAngle{CEF} = \drawAngle{CAF} + \drawAngle{C}$,\\
$\therefore \drawAngle{BEF,CEF} = \mbox{ дважды } \drawAngle{BAF,CAF}$.
\stopCenterAlign
\stopsubproposition

\defineNewPicture{
pair E, C, F, G, D;
r := 7/4u;
E := (0, 0);
F := (dir(80 + 180)*r) shifted E;
C := (dir(-30)*r) shifted E;
D := (dir(30)*r) shifted E;
G := (dir(30 + 180)*r) shifted E;
draw byAngle(G, E, F, byblue, 0);
draw byAngle(F, E, C, byyellow, 0);
draw byAngle(G, D, F, black, 0);
draw byAngle(F, D, C, byred, 0);
draw byLine(D, C, black, 0, 1);
draw byLine(D, F, black, 0, 1);
draw byLine(E, C, black, 0, 1);
draw byLine(E, F, black, 0, 1);
draw byLine(D, G, byred, 0, 0);
draw byCircleR(E, r, byblue, 0, 0, 0)(E);
draw byLabelsOnCircle(D, G, C, F)(E);
draw byLabelsOnPolygon(G, E, D)(2, 0);
}
\drawCurrentPictureInMargin
\startsubproposition[title={Случай III.}]
\startCenterAlign
Пусть центр будет вовне \drawAngle{FDC}\\
проведем диаметр \drawUnitLine{DG}.

Поскольку $\drawAngle{GEF,FEC} = \mbox{ дважды } \drawAngle{GDF,FDC}$;\\
и $\drawAngle{GEF} = \mbox{ дважды } \drawAngle{GDF}$ (случай I.);

$\therefore \drawAngle{FEC} = \mbox{ дважды } \drawAngle{FDC}$.
\stopCenterAlign
\stopsubproposition

\qed
\stopProposition

\startProposition[title={Предложение XXI. Теорема},reference=prop:III.XXI]
\problemNP{В}{круге}{углы в одном сегменте равны между собой.}

\defineNewPicture{
pair A, B, D, E, F;
numeric r;
r := 2u;
F := (0, 0);
B := (dir(-90 - 50)*r) shifted F;
D := (dir(-90 + 50)*r) shifted F;
A := (dir(90 + 25)*r) shifted F;
E := (dir(90 - 35)*r) shifted F;
draw byAngleWithName(B, A, D, byred, 0)(A);
draw byAngleWithName(B, E, D, byblue, 0)(E);
draw byAngleWithName(B, F, D, byyellow, 0)(F);
byLineDefine(B, F, byblue, 0, 0);
byLineDefine(D, F, byred, 0, 0);
draw byNamedLineSeq(0)(BF,DF);
draw byLine(B, D, black, 1, 0);
draw byLine(B, A, black, 0, 1);
draw byLine(B, E, black, 0, 1);
draw byLine(D, A, black, 0, 1);
draw byLine(D, E, black, 0, 1);
draw byCircleR(F, r, byblue, 0, 0, 0)(F);
draw byLabelsOnCircle(B, D, E, A)(F);
draw byLabelsOnPolygon(B, F, D)(2, 0);
}\drawCurrentPictureInMargin
\startsubproposition[title={Случай I.}]
Пусть сегмент будет больше половины круга, проведем \drawUnitLine{DF} и \drawUnitLine{BF} к центру.

\startCenterAlign
$\drawAngle{E} = \mbox{ дважды } \drawAngle{A} \mbox{ или дважды } = \drawAngle{E}$ \inprop[prop:III.XX];

$\therefore \drawAngle{A} = \drawAngle{E}$
\stopCenterAlign
\stopsubproposition

\defineNewPicture{
pair A, B, D, E, F, G;
numeric r;
path cr;
r := 2u;
F := (0, 0);
cr := (fullcircle scaled 2r) shifted F;
B := (dir(90 + 85)*r) shifted F;
D := (dir(90 - 85)*r) shifted F;
A := (dir(90 + 25)*r) shifted F;
E := (dir(90 - 35)*r) shifted F;
G := (dir(-90 + 20)*r) shifted F;
draw byAngle(B, A, G, byyellow, 0);
draw byAngle(G, A, D, byred, 0);
draw byAngle(B, E, G, byblue, 0);
draw byAngle(G, E, D, black, 0);
draw byFilledCircleSegment(F, r, xpart(cr intersectiontimes (F -- 2[F, B])), xpart(cr intersectiontimes (F -- 2[F, G])), byblue)(BG);
draw byFilledCircleSegment(F, r, xpart(cr intersectiontimes (F -- 2[F, G])), 8 + xpart(cr intersectiontimes (F -- 2[F, D])), byyellow)(GD);
draw byLine(G, A, byblue, 0, 0);
draw byLine(G, E, byred, 0, 0);
draw byLine(B, D, black, 1, 0);
draw byLine(B, A, black, 0, 1);
draw byLine(B, E, black, 0, 1);
draw byLine(D, A, black, 0, 1);
draw byLine(D, E, black, 0, 1);
draw byCircleR(F, r, byblue, 0, 0, 0)(F);
draw byLabelsOnCircle(B, D, E, A, G)(F);
}
\drawCurrentPictureInMargin
\startsubproposition[title={Случай II.}]
Пусть сегмент будет меньше половины круга, проведем диаметр \drawUnitLine{GA}, также проведем \drawUnitLine{GE}.

\startCenterAlign
$\drawAngle{BAG} = \drawAngle{BEG}$ и $\drawAngle{GAD} = \drawAngle{GED}$ (случай I.)

$\therefore \drawAngle{BAG,GAD} = \drawAngle{BEG,GED}$.
\stopCenterAlign
\stopsubproposition

\qed
\stopProposition

\startProposition[title={Предложение XXII. Теорема},reference=prop:III.XXII]
\defineNewPicture{
pair A, B, C, D, E;
numeric r;
r := 7/4u;
E := (0, 0);
A := (dir(80)*r) shifted E;
B := (dir(10)*r) shifted E;
C := (dir(-100)*r) shifted E;
D := (dir(150)*r) shifted E;
draw byAngle(D, A, C, byred, 0);
draw byAngle(C, A, B, byblue, 0);
draw byAngle(A, B, D, byyellow, 0);
draw byAngle(D, B, C, byred, 0);
draw byAngle(B, C, A, black, 0);
draw byAngle(A, C, D, byyellow, 0);
draw byAngle(C, D, B, byblue, 0);
draw byAngle(B, D, A, black, 0);
draw byLine(A, B, black, 0, 1);
draw byLine(B, C, black, 0, 1);
draw byLine(C, D, black, 0, 1);
draw byLine(D, A, black, 0, 1);
draw byLine(A, C, byred, 0, 0);
draw byLine(B, D, black, 0, 0);
draw byCircleR(E, r, byred, 0, 0, 1/2)(E);
draw byLabelsOnCircle(A, B, C, D)(E);
}
\drawCurrentPictureInMargin
\problemNP[2]{П}{ротивоположные}{углы \drawAngle{DAC,CAB} и \drawAngle{BCA,ACD} или \drawAngle{CDB,BDA} и \drawAngle{ABD,DBC} четырехугольника вписанного в круг, вместе равны двум прямым углам.}

\startCenterAlign
Проведем диагонали \drawUnitLine{AC} и \drawUnitLine{BD},\\
поскольку углы в одной дуге равны\\
$\drawAngle{CDB} = \drawAngle{CAB}$,\\
и $\drawAngle{DAC} = \drawAngle{DBC}$;

добавим к каждому \drawAngle{BCA,ACD}.\\
$\drawAngle{DAC,CAB} + \drawAngle{BCA,ACD} = \drawAngle{BCA,ACD} + \drawAngle{CDB} + \drawAngle{DBC} = \drawTwoRightAngles$ \inprop[prop:I.XXXII].

Так же можно показать, что\\
$\drawAngle{CDB,BDA} + \drawAngle{ABD,DBC} = \drawTwoRightAngles$.
\stopCenterAlign

\qed
\stopProposition

\startProposition[title={Предложение XXIII. Теорема},reference=prop:III.XXIII]
\defineNewPicture{
pair A, B, C, D, M, N;
numeric r[], t[];
path cr[];
r1 := 14/6u;
r2 := 15/6u;
M := (0, 0);
N := (0, 1/3r1);
cr1 := (fullcircle scaled 2r1) shifted M;
cr2 := (fullcircle scaled 2r2) shifted N;
t1 := xpart(cr1 intersectiontimes (subpath (-2, 2) of cr2));
t2 := xpart(cr1 intersectiontimes (subpath (2, 6) of cr2));
t3 := xpart(cr2 intersectiontimes (subpath (-2, 2) of cr1));
t4 := xpart(cr2 intersectiontimes (subpath (2, 6) of cr1));
A := point t1 of cr1;
B := point t2 of cr1;
C := point 3/2 of cr1;
D := cr2 intersectionpoint (1/2[A,C]--2[A,C]);
draw byAngleWithName(A, C, B, byyellow, 0)(C);
draw byAngleWithName(A, D, B, byblue, 0)(D);
draw byLineFull(A, D, byred, 0, 0)(B, D, 1, 0, 0);
draw byLineFull(B, C, byblue, 0, 0)(A, C, 1, 0, 0);
draw byLineFull(B, D, byyellow, 0, 0)(A, D, 1, 0, 0);
draw byLineFull(A, B, black, 0, 0)(A, B, 0, 0, 1);
draw byArc(M, A, B, r1, byred, 0, 0, 0, 1)(M);
draw byArc(N, A, B, r2, byblue, 0, 0, 0, 1)(N);
draw byLabelsOnPolygon(A, B, noPoint)(0, 0);
draw byLabelsOnPolygon(B, D, A)(2, 0);
draw byLabelPoint(C, (angle(C-M)-90+angle(D-A))/2, 1);
}
\drawCurrentPictureInMargin
\problemNP{Н}{а}{одной прямой и по одну ее сторону нельзя построить два подобных и неравных сегмента круга.}

\startCenterAlign
Действительно, если это возможно, построим два подобных сегмента\\
\drawFromCurrentPicture[bottom]{
draw byNamedLine(AB);
draw byNamedArcExact(N);
draw byLabelsOnPolygon(A, B, noPoint)(0, 0);
}
и
\drawFromCurrentPicture[bottom]{
draw byNamedLine(AB);
draw byNamedArcExact(M);
draw byLabelsOnPolygon(A, B, noPoint)(0, 0);
},

проведем прямую \drawUnitLine{AD}, секущую оба,

проведем \drawUnitLine{BC} и \drawUnitLine{BD}.

Поскольку сегменты подобны,\\
$\drawAngle{C} = \drawAngle{D}$ \inprop[prop:III.X],

но $\drawAngle{C} > \drawAngle{D}$ \inprop[prop:III.XVI], что невозможно.

Следовательно, никакая точка одного сегмента не расположена вовне другого сегмента и значит сегменты совпадают.
\stopCenterAlign

\qed
\stopProposition

\startProposition[title={Предложение XXIV. Теорема},reference=prop:III.XXIV]
\defineNewPicture{
pair A, B, C, D, M, N, d;
numeric r, t[];
path cr[];
r := 5/2u;
t1 := 2-1;
t2 := 2+1;
d := (0, -3/2u);
M := (0, 0);
N := M shifted d;
cr1 := ((subpath (t1, t2) of fullcircle) scaled 2r) shifted M;
cr2 := ((subpath (t1, t2) of fullcircle) scaled 2r) shifted N;
A := point length(cr1) of cr1;
B := point 0 of cr1;
C := point length(cr2) of cr2;
D := point 0 of cr2;
draw byFilledCircleSegment (M, r, t1, t2, byred)(M);
draw byLineFull(A, B, black, 0, 0)(A, B, 0, 0, -1);
draw byArc(M, B, A, r, byblue, 0, 0, 1/2, 1)(M);
draw byFilledCircleSegment (N, r, t1, t2, byyellow)(N);
draw byLineFull(C, D, byblue, 0, 0)(C, D, 0, 0, -1);
draw byArc(N, D, C, r, byred, 0, 0, 1/2, 1)(N);
draw byLabelsOnPolygon(B, A, noPoint)(0, 0);
draw byLabelsOnPolygon(D, C, noPoint)(0, 0);
}
\drawCurrentPictureInMargin
\problemNP{П}{одобные}{сегменты
\drawFromCurrentPicture[bottom]{
draw byNamedFilledCircleSegment(M);
draw byNamedLine(AB);
draw byNamedArcExact(M);
draw byLabelsOnPolygon(B, A, noPoint)(0, 0);
}
и
\drawFromCurrentPicture[bottom]{
draw byNamedFilledCircleSegment(N);
draw byNamedLine(CD);
draw byNamedArcExact(N);
draw byLabelsOnPolygon(D, C, noPoint)(0, 0);
}
кругов на равных прямых \drawUnitLine{AB} и \drawUnitLine{CD} равны между собой.}

\startCenterAlign
Поскольку, если
\drawFromCurrentPicture[bottom]{
draw byNamedFilledCircleSegment(N);
draw byLabelsOnPolygon(D, C, noPoint)(0, 0);
}
наложить на
\drawFromCurrentPicture[bottom]{
draw byNamedFilledCircleSegment(M);
draw byLabelsOnPolygon(B, A, noPoint)(0, 0);
},\\
так, что \drawUnitLine{CD} совпадет с \drawUnitLine{AB},\\
концы \drawUnitLine{CD} будут на концах \drawUnitLine{AB}\\
и \drawArc{M} по одну сторону с \drawArc{N};\\
поскольку $\drawUnitLine{CD} = \drawUnitLine{AB}$,\\
\drawUnitLine{CD} будет полностью совпадать с \drawUnitLine{AB};
\stopCenterAlign

\noindent равные сегменты на одной прямой и по одну ее сторону также совпадают \inprop[prop:III.XXIII], и, следовательно, равны.

\qed
\stopProposition

\startProposition[title={Предложение XXV. Задача},reference=prop:III.XXV]
\defineNewPicture{
pair A, B, C, D, E, F, O;
numeric r;
r := 7/4u;
O := (0, 0);
A := (dir(-20)*r) shifted O;
B := (dir(85)*r) shifted O;
C := (dir(180)*r) shifted O;
D := 1/2[A, B];
E := 1/2[B, C];
F = whatever[D, D shifted ((A-B) rotated 90)] = whatever[E, E shifted ((B-C) rotated 90)];
byLineDefine(D, F, byred, 0, 0);
byLineDefine(E, F, byyellow, 0, 0);
draw byNamedLineSeq(0)(DF,EF);
draw byLine(A, B, black, 0, 0);
draw byLine(B, C, byblue, 0, 0);
draw byArcBE(O, -6/5, 5, r, byblue, 0, 0, 1/2, 0)(O);
draw byLabelsOnCircle(A, B, C)(O);
draw byLabelLineEnd(D, F, 0);
draw byLabelLineEnd(E, F, 0);
draw byLabelsOnPolygon(D, F, E)(2, 0);
}
\drawCurrentPictureInMargin
\problemNP{К}{данному}{сегменту круга пристроить круг, сегментом которого он является.}

\startCenterAlign
Из любой точки сегмента проведем \drawUnitLine{BC} и \drawUnitLine{AB},\\
рассечем обе пополам и из точек рассечения\\
проведем $\drawUnitLine{EF} \perp \drawUnitLine{BC}$\\
и $\drawUnitLine{DF} \perp \drawUnitLine{AB}$\\
где они пересекаются, там и находится центр круга..
\stopCenterAlign

Поскольку \drawUnitLine{BC} кончающаяся на окружности рассекается перпендикуляром \drawUnitLine{EF}, который проходит через центр \inprop[prop:III.I], так же и \drawUnitLine{DF} проходит через центр, следовательно, их пересечение и есть центр.

\qed
\stopProposition

\startProposition[title={Предложение XXVI. Теорема},reference=prop:III.XXVI]
\defineNewPicture{
pair A, B, C, D, E, F, G, H, d;
numeric r, t[];
path cr[];
r := 7/4u;
t1 := 5;
t2 := 7;
G := (0, 0);
cr1 := (fullcircle scaled 2r) shifted G;
A := point 3/2 of cr1;
B := point t1 of cr1;
C := point t2 of cr1;
d := (0, -5/2r);
H := G shifted d;
cr2 := (fullcircle scaled 2r) shifted H;
D := point 5/2 of cr2;
E := point t1 of cr2;
F := point t2 of cr2;
draw byAngleWithName(B, A, C, byred, 0)(A);
draw byAngleWithName(B, G, C, byyellow, 0)(G);
draw byFilledCircleSegment (G, r, t1, t2, byyellow)(G);
draw byLine(B, C, black, 0, 0);
draw byLine(A, B, black, 0, 1);
draw byLine(A, C, black, 0, 1);
byLineDefine(B, G, byblue, 0, 0);
byLineDefine(C, G, byred, 0, 0);
draw byNamedLineSeq(0)(BG,CG);
draw byArc(G, C, B, r, byblue, 0, 0, 0, 0)(G);
draw byArc(G, B, C, r, black, 0, 0, 0, 0)(Gb);
byCircleDefineR(G, r, byblue, 0, 0, 0)(G);
draw byAngleWithName(E, D, F, byblue, 0)(D);
draw byAngleWithName(E, H, F, black, 0)(H);
draw byFilledCircleSegment (H, r, t1, t2, byyellow)(H);
draw byLine(E, F, black, 1, 0);
draw byLine(D, E, black, 0, 1);
draw byLine(D, F, black, 0, 1);
byLineDefine(E, H, byblue, 1, 0);
byLineDefine(F, H, byred, 1, 0);
draw byNamedLineSeq(0)(EH,FH);
draw byArc(H, F, E, r, byred, 0, 0, 0, 0)(H);
draw byArc(H, E, F, r, black, 1, 0, 0, 0)(Hb);
byCircleDefineR(H, r, byred, 0, 0, 0)(H);
draw byLabelsOnCircle(A, B, C)(G);
draw byLabelsOnCircle(D, E, F)(H);
draw byLabelsOnPolygon(B, G, C)(2, 0);
draw byLabelsOnPolygon(E, H, F)(2, 0);
}
\drawCurrentPictureInMargin
\problemNP{В}{равных}{кругах \drawCircle[middle][1/4]{G} и \drawCircle[middle][1/4]{H} дуги \drawArc{Gb} и \drawArc{Hb}, на которые опираются равные, что при центре, что при окружности, углы, равны.}

\startCenterAlign
Поскольку, пусть $\drawAngle{G} = \drawAngle{H}$ при центре,\\
проведем \drawUnitLine{BC} и \drawUnitLine{EF}.

Тогда, поскольку $\circleG = \circleH$,\\
у \drawLine[bottom]{BG,CG,BC} и \drawLine[bottom]{EH,FH,EF}\\
$\drawUnitLine{BG} = \drawUnitLine{CG} = \drawUnitLine{EH} = \drawUnitLine{FH}$,\\
и $\drawAngle{G} = \drawAngle{H}$,\\
$\therefore \drawUnitLine{BC} = \drawUnitLine{EF}$ \inprop[prop:I.IV].

Но $\drawAngle{A} = \drawAngle{D}$ \inprop[prop:III.XX];\\
$\therefore$
\drawFromCurrentPicture{
startTempScale(1/4);
draw byNamedArc(G);
draw byNamedLine(BC);
draw byLabelsOnCircle(B, C)(G);
stopTempScale;
}
и
\drawFromCurrentPicture{
startTempScale(1/4);
draw byNamedArc(H);
draw byNamedLine(EF);
draw byLabelsOnCircle(E, F)(H);
stopTempScale;
}
подобны \indef[def:III.XI];\\
и равны \inprop[prop:III.XXIV]
\stopCenterAlign

А если равные сегменты вычесть из равных кругов, оставшиеся сегменты будут равны,

\startCenterAlign
то есть $
\drawFromCurrentPicture{
draw byNamedFilledCircleSegment(G);
draw byNamedArcLabel(Gb);
}
=
\drawFromCurrentPicture{
draw byNamedFilledCircleSegment(H);
draw byNamedArcLabel(Hb);
}
$ \inax[ax:III];\\
и $\therefore \drawArc{Gb} = \drawArc{Hb}$.
\stopCenterAlign

И если равные углы будут при окружности, очевидно, что углы в центре, будучи вдвое больше углов при окружности, также равны, и, следовательно, дуги, на которые они опираются, тоже равны.

\qed
\stopProposition

\startProposition[title={Предложение XXVII. Теорема},reference=prop:III.XXVII]
\defineNewPicture[1/2]{
pair A, B, C, D, E, F, G, H, K, d;
numeric r, t[];
path cr[];
r := 7/4u;
t1 := 5;
t2 := 7;
t3 := 13/2;
G := (0, 0);
cr1 := (fullcircle scaled 2r) shifted G;
A := point 3/2 of cr1;
B := point t1 of cr1;
C := point t2 of cr1;
K := point t3 of cr1;
d := (0, -5/2r);
H := G shifted d;
cr2 := (fullcircle scaled 2r) shifted H;
D := point 3/2 of cr2;
E := point t1 of cr2;
F := point t3 of cr2;
draw byAngle(B, A, K, byyellow, 0);
draw byAngle(B, G, K, byyellow, 0);
draw byAngle(K, A, C, byblue, 0);
draw byAngle(K, G, C, byblue, 0);
draw byLine(A, K, black, 0, 1);
draw byLine(A, B, black, 0, 1);
draw byLine(A, C, black, 0, 1);
byLineDefine(G, B, black, 0, 1);
byLineDefine(G, C, black, 0, 1);
draw byNamedLineSeq(0)(GB,GC);
draw byLine(G, K, black, 0, 1);
draw byArc(G, C, B, r, byred, 0, 0, 0, 0)(G);
draw byArc(G, B, K, r, black, 0, 0, 0, 0)(GbI);
draw byArc(G, K, C, r, byred, 1, 0, 0, 0)(GbII);
byCircleDefineR(G, r, byred, 0, 0, 0)(G);
draw byAngleWithName(E, D, F, byred, 0)(D);
draw byAngleWithName(E, H, F, byred, 0)(H);
draw byLine(D, F, black, 0, 1);
draw byLine(D, E, black, 0, 1);
byLineDefine(H, E, black, 0, 1);
byLineDefine(H, F, black, 0, 1);
draw byNamedLineSeq(0)(HE,HF);
draw byArc(H, F, E, r, byblue, 0, 0, 0, 0)(H);
draw byArc(H, E, F, r, black, 1, 0, 0, 0)(Hb);
byCircleDefineR(H, r, byblue, 0, 0, 0)(H);
draw byLabelsOnCircle(A, B, C, K)(G);
draw byLabelsOnCircle(D, E, F)(H);
draw byLabelsOnPolygon(B, G, C)(2, 0);
draw byLabelsOnPolygon(E, H, F)(2, 0);
}
\drawCurrentPictureInMargin
\problemNP[2]{В}{равных}{кругах \drawCircle[middle][1/3]{G} и \drawCircle[middle][1/3]{H}, углы \drawAngle{BAK} и \drawAngle{D}, опирающиеся на равные дуги, равны между собой, будь они при центрах или при окружностях.}

\startCenterAlign
Поскольку, если такое возможно, пусть один из них\\
\drawAngle{D} будет больше другого \drawAngle{BAK},\\
сделаем $\drawAngle{BAK,KAC} = \drawAngle{D}$

$\therefore \drawArc{GbI,GbII} = \drawArc{Hb}$ \inprop[prop:III.XXVI]

но $\drawArc{GbI} = \drawArc{Hb}$ (гип.)\\
$\therefore \drawArc{GbI} = \drawArc{GbI,GbII}$ часть равны целому, что невозможно;

$\therefore$ ни один из углов не больше другого,

и $\therefore$ они равны.
\stopCenterAlign

\qed
\stopProposition

\startProposition[title={Предложение XXVIII. Теорема},reference=prop:III.XXVIII]
\defineNewPicture[1/2]{
pair A, B, D, E, K, L, d;
numeric r, t[];
path cr[];
r := 7/4u;
t1 := 5;
t2 := 7;
K := (0, 0);
cr1 := (fullcircle scaled 2r) shifted K;
A := point t1 of cr1;
B := point t2 of cr1;
d := (0, -5/2r);
L := K shifted d;
cr2 := (fullcircle scaled 2r) shifted L;
D := point t1 of cr2;
E := point t2 of cr2;
draw byAngleWithName(A, K, B, byred, 0)(K);
draw byLine(A, B, byred, 0, 0);
byLineDefine(A, K, black, 0, 0);
byLineDefine(B, K, byblue, 0, 0);
draw byNamedLineSeq(0)(AK,BK);
draw byArc(K, B, A, r, byyellow, 0, 0, 0, 0)(K);
draw byArc(K, A, B, r, byblue, 0, 0, 0, 0)(Kb);
byCircleDefineR(K, r, byyellow, 0, 0, 0)(K);
draw byAngleWithName(D, L, E, byyellow, 0)(L);
draw byLine(D, E, byred, 1, 0);
byLineDefine(D, L, black, 1, 0);
byLineDefine(E, L, byblue, 1, 0);
draw byNamedLineSeq(0)(DL,EL);
draw byArc(L, E, D, r, black, 0, 0, 0, 0)(L);
draw byArc(L, D, E, r, byred, 0, 0, 0, 0)(Lb);
byCircleDefineR(L, r, black, 0, 0, 0)(L);
draw byLabelsOnCircle(A, B)(K);
draw byLabelsOnCircle(D, E)(L);
draw byLabelsOnPolygon(A, K, B)(2, 0);
draw byLabelsOnPolygon(D, L, E)(2, 0);
}
\drawCurrentPictureInMargin
\problemNP{В}{равных}{кругах \drawCircle[middle][1/3]{K} и \drawCircle[middle][1/3]{L} равные хорды \drawUnitLine{AB} и \drawUnitLine{DE} отсекают равные дуги.}

\startCenterAlign
Из центров равных кругов,\\
проведем \drawUnitLine{AK}, \drawUnitLine{BK} и \drawUnitLine{DL}, \drawUnitLine{EL}.

Поскольку $\circleK = \circleL$\\
$\drawUnitLine{AK}, \drawUnitLine{BK} = \drawUnitLine{DL}, \drawUnitLine{EL}$,\\
а также $\drawUnitLine{AB} = \drawUnitLine{DE}$ (гип.)

$\therefore \drawAngle{K} = \drawAngle{L}$

$\therefore \drawArc{Kb} = \drawArc{Lb}$ \inprop[prop:III.XXVI]

и $\therefore \drawArc[bottom][1/3]{K} = \drawArc[bottom][1/3]{L}$ \inax[ax:III]
\stopCenterAlign

\qed
\stopProposition

\startProposition[title={Предложение XXIX. Теорема},reference=prop:III.XXIX]
\defineNewPicture[1/2]{
pair A, B, D, E, K, L, d;
numeric r, t[];
path cr[];
r := 7/4u;
t1 := 5;
t2 := 7;
K := (0, 0);
cr1 := (fullcircle scaled 2r) shifted K;
A := point t1 of cr1;
B := point t2 of cr1;
d := (0, -5/2r);
L := K shifted d;
cr2 := (fullcircle scaled 2r) shifted L;
D := point t1 of cr2;
E := point t2 of cr2;
draw byAngleWithName(A, K, B, byred, 0)(K);
draw byLine(A, B, byred, 0, 0);
byLineDefine(A, K, black, 0, 0);
byLineDefine(B, K, byblue, 0, 0);
draw byNamedLineSeq(0)(AK,BK);
draw byArc(K, B, A, r, byyellow, 0, 0, 0, 0)(K);
draw byArc(K, A, B, r, byblue, 0, 0, 0, 0)(Kb);
byCircleDefineR(K, r, byyellow, 0, 0, 0)(K);
draw byAngleWithName(D, L, E, byyellow, 0)(L);
draw byLine(D, E, byred, 1, 0);
byLineDefine(D, L, black, 1, 0);
byLineDefine(E, L, byblue, 1, 0);
draw byNamedLineSeq(0)(DL,EL);
draw byArc(L, E, D, r, black, 0, 0, 0, 0)(L);
draw byArc(L, D, E, r, byred, 0, 0, 0, 0)(Lb);
byCircleDefineR(L, r, black, 0, 0, 0)(L);
draw byLabelsOnCircle(A, B)(K);
draw byLabelsOnCircle(D, E)(L);
draw byLabelsOnPolygon(A, K, B)(2, 0);
draw byLabelsOnPolygon(D, L, E)(2, 0);
}
\drawCurrentPictureInMargin
\problemNP{В}{равных}{кругах \drawCircle[middle][1/3]{K} и \drawCircle[middle][1/3]{L}, хорды \drawUnitLine{AB} и \drawUnitLine{DE}, стягивающие равные дуги, равны.}

\startCenterAlign
Если дуги равны половинам окружностей, то предложение очевидно, а если нет, 

пусть \drawUnitLine{AK}, \drawUnitLine{BK} и \drawUnitLine{DL}, \drawUnitLine{EL} проведены из центров;

поскольку $\drawArc{Kb} = \drawArc{Lb}$ (гип.)\\
и $\drawAngle{K} = \drawAngle{L}$ \inprop[prop:III.XXVII];

но $\drawUnitLine{AK} \mbox{ и } \drawUnitLine{BK} = \drawUnitLine{DL} \mbox{ и } \drawUnitLine{EL}$

$\therefore \drawUnitLine{AB} = \drawUnitLine{DE}$ \inprop[prop:III.XXVII],

а это и есть хорды, стягивающие равные дуги.
\stopCenterAlign

\qed
\stopProposition

\startProposition[title={Предложение XXX. Задача},reference=prop:III.XXX]
\defineNewPicture{
pair A, B, C, D, E;
numeric r, t[];
path cr;
r := 9/4u;
t1 := -1/2;
t2 := 4 + 1/2;
t3 := 2;
E := (0, 0);
cr := (fullcircle scaled 2r) shifted E;
A := point t2 of cr;
B := point t1 of cr;
D := point t3 of cr;
C := 1/2[A, B];
draw byAngle(A, C, D, byblue, 0);
draw byAngle(D, C, B, byred, 0);
draw byLine(D, A, byblue, 0, 0);
draw byLine(D, C, byyellow, 0, 0);
draw byLine(D, B, byblue, 1, 0);
draw byLine(A, C, black, 0, 0);
draw byLine(C, B, black, 1, 0);
draw byArc(E, B, D, r, byred, 1, 0, 0, 0)(Er);
draw byArc(E, D, A, r, byred, 0, 0, 0, 0)(El);
draw byLabelsOnCircle(A, B, D)(Er);
draw byLabelLineEnd(C, D, 0);
}
\drawCurrentPictureInMargin
\problemNP{Р}{ассечь}{данную дугу \drawArc[middle][1/5]{El,Er} пополам.}

\startCenterAlign
Проведем \drawUnitLine{AC,CB};\\
сделаем $\drawUnitLine{AC} = \drawUnitLine{CB}$,\\
проведем $\drawUnitLine{DC} \perp \drawUnitLine{AC,CB}$, она и будет рассекать дугу.

Проведем \drawUnitLine{DA} и \drawUnitLine{DB}.

$\drawUnitLine{AC} = \drawUnitLine{CB}$ (постр.)\\
\drawUnitLine{DC} общая,\\
и $\drawAngle{ACD} = \drawAngle{DCB}$ (постр.)

$\therefore \drawUnitLine{DA} = \drawUnitLine{DB}$ \inprop[prop:I.IV]\\
$
\drawFromCurrentPicture{
startGlobalRotation(-arcAngle.El);
startAutoLabeling;
draw byNamedArc(El);
stopAutoLabeling;
stopGlobalRotation;
}
=
\drawFromCurrentPicture{
startGlobalRotation(-arcAngle.Er);
startAutoLabeling;
draw byNamedArc(Er);
stopAutoLabeling;
stopGlobalRotation;
}
$ \inprop[prop:III.XXVII],\\
и значит данная дуга рассечена.
\stopCenterAlign

\qed
\stopProposition

\startProposition[title={Предложение XXXI. Теорема},reference=prop:III.XXXI]
\problemNP{В}{круге}{угол заключенный в полукруге прямой, в сегменте больше полукруга острый, в сегменте меньше полукруга тупой.}

\defineNewPicture{
pair A, B, C, E;
numeric r;
r := 7/4u;
E := (0, 0);
A := (dir(110)*r) shifted E;
B := (dir(180)*r) shifted E;
C := (dir(0)*r) shifted E;
draw byAngleWithName(A, B, C, byred, 0)(B);
draw byAngleWithName(B, C, A, byblue, 0)(C);
draw byAngle(E, A, B, byyellow, 0);
draw byAngle(C, A, E, black, 0);
draw byLine(A, E, byred, 0, 0);
draw byLine(A, B, black, 0, 1);
draw byLine(A, C, black, 0, 1);
draw byLine(B, E, byblue, 0, 0);
draw byLine(E, C, black, 0, 0);
draw byCircleR(E, r, black, 0, 0, 0)(E);
draw byLabelsOnCircle(A, B, C)(E);
draw byLabelsOnPolygon(C, E, B)(2, 0);
}
\drawCurrentPictureInMargin
\startsubproposition[title={Случай I.}]
\startCenterAlign
Угол \drawAngle{EAB,CAE} в полукруге прямой.

Проведем \drawUnitLine{AE} и \drawUnitLine{BE,EC}\\
$\drawAngle{B}=\drawAngle{EAB}$ и $\drawAngle{C} = \drawAngle{CAE}$ \inprop[prop:I.V]\\
$\drawAngle{C} + \drawAngle{B} = \drawAngle{EAB,CAE} = \mbox{ половина } \drawTwoRightAngles= \drawRightAngle$ \inprop[prop:I.XXXII]
\stopCenterAlign
\stopsubproposition

\defineNewPicture{
pair A, B, C, E, D;
numeric r;
r := 7/4u;
E := (0, 0);
A := (dir(110)*r) shifted E;
B := (dir(180 + 30)*r) shifted E;
C := (dir(0 + 30)*r) shifted E;
D := (dir(0 - 30)*r) shifted E;
draw byAngle(B, A, D, byblue, 0);
draw byAngle(D, A, C, byred, 0);
draw byLine(A, D, black, 0, 1);
draw byLine(A, B, black, 0, 1);
draw byLine(B, D, black, 0, 1);
draw byLine(A, C, byblue, 0, 0);
draw byLine(B, C, byred, 0, 0);
draw byCircleR(E, r, byblue, 0, 0, 0)(E);
draw byLabelsOnCircle(A, B, C, D)(E);
}
\drawCurrentPictureInMargin
\startsubproposition[title={Случай II.}]
\startCenterAlign
Угол \drawAngle{BAD} в сегменте больше полукруга острый.

Проведем диаметр \drawUnitLine{BC} и \drawUnitLine{AC}\\
$\therefore \drawAngle{BAD,DAC} = \mbox{ прямому углу}$\\
$\therefore$ \drawAngle{BAD} острый.
\stopCenterAlign
\stopsubproposition

\vfill\pagebreak

\defineNewPicture{
pair A, B, C, E, D;
numeric r;
r := 7/4u;
E := (0, 0);
A := (dir(140)*r) shifted E;
B := (dir(-110)*r) shifted E;
C := (dir(5)*r) shifted E;
D := (dir(80)*r) shifted E;
draw byAngleWithName(A, D, C, byred, 0)(D);
draw byAngleWithName(C, B, A, byyellow, 0)(B);
draw byLine(A, B, byred, 0, 0);
draw byLine(B, C, byblue, 0, 0);
draw byLine(C, D, black, 0, 1);
draw byLine(D, A, black, 0, 1);
draw byCircleR(E, r, byblue, 0, 0, 0)(E);
draw byLabelsOnCircle(A, B, C, D)(E);
}
\drawCurrentPictureInMargin
\startsubproposition[title={Случай III.}]
\startCenterAlign
Угол \drawAngle{D} в сегменте меньше полукруга тупой.

Возьмем на противоположной стороне окружности любую точку, к которой проведем \drawUnitLine{BC} и \drawUnitLine{AB}.\\
Поскольку $\drawAngle{B} + \drawAngle{D} = \drawTwoRightAngles$ \inprop[prop:III.XXII]\\
но $\drawAngle{B} < \drawRightAngle$ (случай II.),\\
$\therefore$ \drawAngle{D} тупой.
\stopCenterAlign
\stopsubproposition

\qed
\stopProposition

\startProposition[title={Предложение XXXII. Теорема},reference=prop:III.XXXII]
\defineNewPicture{
pair A, B, C, D, E, F, O;
numeric r;
r := 9/4u;
O := (0, 0);
A := (0, r) shifted O;
B := (0, -r) shifted O;
C := (dir(-20)*r) shifted O;
D := (dir(30)*r) shifted O;
E := (-r, -r) shifted O;
F := (r, -r) shifted O;
draw byAngle(A, B, E, black, 1);
draw byAngle(D, B, A, byblue, 0);
draw byAngle(F, B, D, byyellow, 0);
draw byAngleWithName(B, A, D, byyellow, 0)(A);
draw byAngle(A, D, B, black, 0);
draw byAngle(C, D, B, black, 1);
draw byAngleWithName(D, C, B, byred, 0)(C);
draw byLine(A, D, black, 0, 1);
draw byLine(D, C, black, 0, 1);
draw byLine(C, B, black, 0, 1);
draw byLine(D, B, byred, 0, 0);
draw byLineFull(E, F, byblue, 0, 0)(E, F, 0, 0, 1);
draw byLine(A, B, black, 0, 0);
draw byCircleR(O, r, byred, 0, 0, 0)(O);
draw byLabelsOnCircle(A, B, C, D)(O);
draw byLabelsOnPolygon(E, F, noPoint)(0, 0);
}
\drawCurrentPictureInMargin
\problemNP{Е}{сли}{прямая \drawUnitLine{EF} касается круга, и из точки касания проведена прямая \drawUnitLine{DB}, секущая круг, угол \drawAngle{FBD} между этой прямой и касательной равен углу \drawAngle{A} в накрестлежащем сегменте круга.}

Если хорда проходит через центр, то очевидно, что углы равны, поскольку и тот и тот прямые. (\inpropL[prop:III.XVI], \inpropL[prop:III.XIX])

Если же нет, проведем $\drawUnitLine{AB} \perp \drawUnitLine{EF}$ из точки касания, которая будет проходить через центр \inprop[prop:III.XIX]

\startCenterAlign
$\therefore \drawAngle{ADB} = \drawAngle{ABE} $ \inprop[prop:III.XXXI]

$\drawAngle{A} + \drawAngle{DBA} = \drawAngle{ABE}= \drawAngle{DBA,FBD}$ \inprop[prop:I.XXXII]

$\therefore \drawAngle{A} = \drawAngle{FBD}$ (акс.).

Теперь $\drawAngle{ABE,DBA,FBD} = \drawTwoRightAngles = \drawAngle{A} + \drawAngle{C}$ \inprop[prop:III.XXII]

$\therefore \drawAngle{ABE,DBA} = \drawAngle{C}$, (акс.), который и будет углом в накрестлежащем сегменте.
\stopCenterAlign

\qed
\stopProposition

\startProposition[title={Предложение XXXIII. Задача},reference=prop:III.XXXIII]
\defineNewPicture{
pair A, B, D, E, G, K;
numeric r;
r := 7/4u;
G := (0, 0);
A := (0, -r) shifted G;
B := (dir(10)*r) shifted G;
E := (0, r) shifted G;
D := (r, -r) shifted G;
K := (-r, -r) shifted G;
draw byAngle(E, A, K, black, 1);
draw byAngle(B, A, E, byred, 0);
draw byAngle(D, A, B, byyellow, 0);
draw byAngleWithName(G, B, A, byred, 0)(B);
draw byLine(A, B, black, 0, 0);
draw byLine(G, B, byyellow, 0, 0);
draw byLine(A, G, byblue, 0, 0);
draw byLine(G, E, byblue, 1, 0);
draw byLineFull(K, D, byred, 0, 0)(K, D, 0, 0, 1);
draw byCircle(G, B, byblue, 0, 0, 0)(G);
byAngleDefineWithName(E, A, K, black, 1)(givenRight);
byAngleDefineWithName(B, A, K, black, 1)(givenObtuse);
byAngleDefineWithName(D, A, B, black, 1)(givenAcute);
angleStandalone.givenRight := 2;
angleStandalone.givenObtuse := 2;
angleStandalone.givenAcute := 2;
draw byLabelsOnCircle(A, B)(G);
draw byLabelsOnPolygon(A, G, E)(2, 0);
draw byLabelsOnPolygon(K, D, noPoint)(0, 0);
}
\drawCurrentPictureInMargin
\problemNP{Н}{а}{данной прямой \drawUnitLine{AB} описать сегмент круга, вмещающий угол, равный данному \drawAngle{givenRight}, \drawAngle{givenObtuse}, \drawAngle{givenAcute}.}

Если данный угол прямой, то рассечем отрезок и опишем на нем полукруг, который, очевидно, будет содержать прямой угол \inprop[prop:III.XXXI].

Если же данный угол острый или тупой, сделаем с данной прямой на одном из ее концов,

\startCenterAlign
$\drawAngle{DAB} = \drawAngle{givenAcute}$,

проведем $\drawUnitLine{AG} \perp \drawUnitLine{KD}$\\
и сделаем $\drawAngle{B} = \drawAngle{BAE}$,

опишем \drawCircle[middle][1/5]{G} с \drawUnitLine{AG} или \drawUnitLine{GB} в качестве радиуса, ведь они равны.

\drawUnitLine{KD} касается \circleG\ \inprop[prop:III.XVI]

$\therefore$ \drawUnitLine{AB} делит круг на два сегмента, вмещающие углы равные \drawAngle{EAK,BAE} и \drawAngle{DAB}, соответственно равные \drawAngle{givenObtuse} и \drawAngle{givenAcute} \inprop[prop:III.XXXII]
\stopCenterAlign

\qed
\stopProposition

\startProposition[title={Предложение XXXIV. Задача},reference=prop:III.XXXIV]
\defineNewPicture{
pair A, B, F, E, O;
numeric r;
path cr;
r := 7/4u;
O := (0, 0);
cr := (fullcircle scaled 2r) shifted O;
A := point 1 of cr;
B := point -2 of cr;
E := (r, -r) shifted O;
F := (-r, -r) shifted O;
draw byFilledCircleSegment(O, r, -2, 1, byyellow)(O);
draw byAngleWithName(F, B, A, byblue, 0)(B);
draw byLine(A, B, black, 0, 0);
draw byLineFull(E,F, byred, 0, 0)(E, F, 0, 0, -1);
draw byCircleR(O, r, byblue, 0, 0, 0)(O);
byAngleDefineWithName(F, B, A, byred, 0)(givenAngle);
angleStandalone.givenAngle := 2;
draw byLabelsOnCircle(A)(O);
draw byLabelsOnPolygon(E, B, F, noPoint)(0, 0);
}
\drawCurrentPictureInMargin
\problemNP{О}{т}{данного круга \drawCircle{O} отсечь сегмент, вмещабщий данный прямолинейный угол \drawAngle{givenAngle}.}

\startCenterAlign
Проведем \drawUnitLine{EF} \inprop[prop:III.XVII], касательную к кругу в любой точке;

в точке касания сделаем $\drawAngle{B} = \drawAngle{givenAngle}$, данному углу;

тогда \drawFromCurrentPicture[middle][segmentO]{
draw byNamedFilledCircleSegment(O);
draw byLabelsOnCircle(A, B)(O);
} содержит угол $=$ данному углу.

Поскольку \drawUnitLine{EF} касательная,\\
и \drawUnitLine{AB} сечет ее,\\
угол в $\segmentO\ = \drawAngle{B}$ \inprop[prop:III.XXXII],

но $\drawAngle{B} = \drawAngle{givenAngle}$ (постр.)
\stopCenterAlign

\qed
\stopProposition

\startProposition[title={Предложение XXXV. Теорема},reference=prop:III.XXXV]
\defineNewPicture{
pair A, B, C, D, E;
numeric r;
r := 3/2u;
E := (0, 0);
A := (dir(50)*r) shifted E;
B := (dir(100)*r) shifted E;
C := (dir(50 + 180)*r) shifted E;
D := (dir(100 + 180)*r) shifted E;
draw byLine(A, E, black, 0, 0);
draw byLine(E, C, black, 1, 0);
draw byLine(D, E, byblue, 0, 0);
draw byLine(E, B, byblue, 1, 0);
draw byCircleR(E, r, byyellow, 0, 0, 0)(E);
draw byLabelsOnCircle(A, B, C, D)(E);
draw byLabelsOnPolygon(C, E, B)(2, 0);
}
\problemNP{Е}{сли}{в круге две хорды
$\left\{\vcenter{
\nointerlineskip\hbox{\drawSizedLine{AE,EC}}
\nointerlineskip\hbox{\drawSizedLine{DE,EB}}}\right\}$
секут друг друга, прямоугольник, заключенный между частями одного равен прямоугольнику, заключенному между частями другого.}
\drawCurrentPictureInMargin
\startsubproposition[title={Случай I.}]
Если данные прямые проходят через центр, они секут друг друга посередине, следовательно прямоугольники заключенные между их частями являются квадратами, и значит, равны.
\stopsubproposition

\defineNewPicture{
pair A, B, C, D, E, H;
numeric r;
r := 3/2u;
E := (0, 0);
A := (dir(30)*r) shifted E;
B := (dir(170)*r) shifted E;
C := (dir(30 + 180)*r) shifted E;
D := (dir(-50)*r) shifted E;
H = whatever[A, C] = whatever[B, D];
draw byLine(D, E, byred, 0, 0);
draw byLine(E, B, byyellow, 0, 0);
draw byLine(B, H, byblue, 0, 0);
draw byLine(H, D, byblue, 1, 0);
draw byLine(A, E, black, 0, 0);
draw byLine(E, H, byred, 1, 0);
draw byLine(H, C, black, 1, 0);
draw byCircleR(E, r, byred, 0, 0, 0)(E);
draw byLabelsOnCircle(B, D, C, A)(E);
draw byLabelsOnPolygon(D, H, C)(2, 0);
draw byLabelsOnPolygon(B, E, A)(2, 0);
}
\drawCurrentPictureInMargin
\startsubproposition[title={Случай II.}]
\startCenterAlign
Пусть \drawSizedLine{HC,EH,AE} проходит через центр, а \drawSizedLine{BH,HD} нет, проведем \drawSizedLine{EB} и \drawSizedLine{DE}.\\
Тогда $\drawSizedLine{BH} \times \drawSizedLine{HD} = \drawSizedLine{EB}^2 - \drawSizedLine{EH}^2$ \inprop[prop:II.VI], or $\drawSizedLine{BH} \times \drawSizedLine{HD} = \drawSizedLine{HC,EH}^2 - \drawSizedLine{EH}^2$.\\
$\therefore \drawSizedLine{BH} \times \drawSizedLine{HD} = \drawSizedLine{EH,AE} \times \drawSizedLine{AE}$ \inprop[prop:II.V].
\stopCenterAlign
\stopsubproposition

\defineNewPicture{
pair A, B, C, D, E, F, K, L;
numeric r;
path cr;
r := 3/2u;
F := (0, 0);
cr := (fullcircle scaled 2r) shifted F;
A := (dir(00)*r) shifted F;
B := (dir(170)*r) shifted F;
C := (dir(-150)*r) shifted F;
D := (dir(-50)*r) shifted F;
E = whatever[A, C] = whatever[B, D];
K := cr intersectionpoint (F -- 10[E, F]);
L := cr intersectionpoint (F -- 10[F, E]);
draw byLine(K, E, byred, 1, 0);
draw byLine(E, L, byred, 0, 0);
draw byLine(B, E, byblue, 0, 0);
draw byLine(E, D, byblue, 1, 0);
draw byLine(A, E, black, 0, 0);
draw byLine(E, C, black, 1, 0);
draw byCircleR(F, r, byblue, 0, 0, 0)(F);
draw byLabelsOnCircle(B, D, K, L, A, C)(F);
draw byLabelsOnPolygon(B, E, K)(2, 0);
}
\drawCurrentPictureInMargin
\startsubproposition[title={Случай III.}]
Пусть ни одна из прямых не проходит через центр, проведем через точку их пересечения диаметр \drawSizedLine{KE,EL},

\startCenterAlign
и $\drawSizedLine{KE} \times \drawSizedLine{EL} = \drawSizedLine{BE} \times \drawSizedLine{ED}$ (случай II.),\\
а также $\drawSizedLine{KE} \times \drawSizedLine{EL} = \drawSizedLine{AE} \times \drawSizedLine{EC}$ (случай II.);\\
$\therefore \drawSizedLine{BE} \times \drawSizedLine{ED} = \drawSizedLine{AE} \times \drawSizedLine{EC}$.
\stopCenterAlign
\stopsubproposition

\qed
\stopProposition

\stopbook

\stoptext
%\closeout \lettrineslist

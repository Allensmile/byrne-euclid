\definepapersize[custom]
 [width=145mm,height=200mm]
\setuppapersize[custom][custom]
\setuppagenumbering[alternative=doublesided]
\setuplayout[backspace=12mm,
	width=76mm,	height=172mm,
	header=5mm,
	headerdistance=8mm,
	topspace=10mm,
	footer=0mm,
	margin=52mm]
\definelayout[title][backspace=15mm,
	width=115mm,	height=172mm,
	header=5mm,
	headerdistance=8mm,
	topspace=10mm,
	footer=0mm]	
	
\switchtobodyfont[10pt]
\setupbodyfont[ebgaramond-be]

\input preamble.tex
\input preamble_be.tex


\def\mpPre{textLabels := true;byDefaultAngleOptionalColor := bytransparent;angleScale := 4/3;}

\starttext

% Heath's translation of the Elements https://archive.org/stream/euclid_heath_2nd_ed/3_euclid_heath_2nd_ed is used as a reference, as well as this online edition https://mathcs.clarku.edu/~djoyce/elements/Euclid.html

\startProposition[title={Prop. I. Theor.}, reference=prop:XI.I]
\defineNewPicture[1/4]{
	numeric w, r;
	w := 3cm;
	r := 1/3w;
	byPair(A, B, C, D);
	A := (-r, 0);
	B := (0, 0);
	D := (r, 0);
	C := B shifted (dir(30)*r);
	byPointXYZDefine(P, -1/2w, 0, -1/2w);
	byPointXYZDefine(Q, 1/2w, 0, -1/2w);
	byPointXYZDefine(R, 1/2w, 0, 1/2w);
	byPointXYZDefine(S, -1/2w, 0, 1/2w);
	byPointLabelRemove(P,Q,R,S);
	byCircleDefineR(B, r, byblue, 0, 0, 0)(B);
	circleSpaceRotation.B := (0, 0, 0);
	byArcDefine (B, C, A)(r, byblue, 0, 0, -1, 0) (BAC);
	byArcDefine (B, D, C)(r, byred, 0, 0, -1, 0) (BDC);
	byArcDefine (B, A, D)(r, byblue, 1, 0, -1, 0) (BAD);
	byLineDefine(A, B, byblue, 0, 0);
	byLineDefine(B, C, byblue, 1, 0);
	bySetProjection(15, 30, 0);
	draw byNamedArcExact(BAD);
	draw byPolygon(P, Q, R, S)(byyellow);
	draw byNamedArcExact(BAC,BDC);
	draw byLine(B, D, byred, 0, 0);
	draw byNamedLineSeq(0)(AB,BC);
	draw byLabelsOnPolygon(D, B, A)(2, 0);
	draw byLabelLineEnd(C, B, 0);
	draw byLabelLineEnd(A, B, 0);
	draw byLabelLineEnd(D, B, 0);
}
\drawCurrentPictureInMargin
\problemNP{A}{ part}{of a straight line cannot be in the plane of reference and a part in a plane more elevated.}

For, if possible, let a part \drawUnitLine{AB} of the straight line \drawUnitLine{AB,BC} be in the plane of reference \drawPolygon{PQRS}, and a part \drawUnitLine{BC} be in a plane more elevated.

There then will be in \drawPolygon{PQRS} some straight line continuous with \drawUnitLine{AB} in a straight line. Let it be \drawUnitLine{BD}. 

Therefore \drawUnitLine{AB} is a common segment of the two straight lines \drawUnitLine{AB,BC} and \drawUnitLine{AB,BD}, which is impossible, inasmuch as, if we describe a circle \drawArc{BAC,BDC,BAD} with centre \drawPointL{B} and radius \drawUnitLine{AB}, then the diameters will cut off unequal circumferences \drawArc{BAC} and \drawArc{BAC,BDC} of the circle.

Therefore, a part of a straight line cannot be in the plane of reference and a part in a plane more elevated. 

\qed
\stopProposition

\startProposition[title={Prop. II. Theor.}, reference=prop:XI.II]
\defineNewPicture{
	pair A, B, C, D, E, F, G, H, K;
	A := (0, 3cm);
	B := (3cm, 0);
	C := (0, 0);
	D := (3cm, 3cm);
	E = whatever[A, B] = whatever[C, D];
	F := 2/5[E, C];
	G := 2/5[E, B];
	H := 1/3[C, B];
	K := 2/3[C, B];
	byLineDefine(A, E, byblue, 0, 0);
	byLineDefine(E, G, byblue, 1, 0);
	byLineDefine(G, B, byblue, 1, 0);
	byLineDefine(D, E, byred, 0, 0);
	byLineDefine(E, F, byred, 1, 0);
	byLineDefine(F, C, byred, 1, 0);
	byLineDefine(C, H, byblack, 0, 0);
	byLineDefine(H, K, byblack, 0, 0);
	byLineDefine(K, B, byblack, 0, 0);
	byLineDefine(F, H, byyellow, 0, 0);
	byLineDefine(G, K, byyellow, 0, 0);
	byLineDefine(F, G, byblack, 1, 0);
	draw byNamedLine(FG,FH,GK);
	draw byNamedLineSeq(0)(AE,EG, GB,KB,HK,CH,FC,EF,DE);
	draw byLabelsOnPolygon(E,G,B,K,H,C,F)(0, 0);
	draw byLabelLineEnd(A, E, 0);
	draw byLabelLineEnd(D, E, 0);
}
\drawCurrentPictureInMargin
\problemNP{I}{f}{two straight lines cut one another, they are in one plane, and every triangle is in one plane.}

For let the two straight lines \drawUnitLine{AB} and \drawUnitLine{CD} cut one another at the point \drawPointL{E}.

I say that \drawUnitLine{AB} and \drawUnitLine{CD} are in one plane, and every triangle is in one plane.

Take the points \drawPointL{F} and \drawPointL{G} at random on \drawUnitLine{EC} and  \drawUnitLine{EB}, let \drawUnitLine{CB} and \drawUnitLine{FG} be joined, and let \drawUnitLine{FH} and \drawUnitLine{GK} be drawn across.

I say first that the triangle \drawLine{EB,BC,CE} lies in one plane.

For, if part of the triangle \drawLine{EB,BC,CE}, either \drawLine{FH,HC,CF} or \drawLine{GB,BK,KG}, is in the plane of reference, and the rest in another, then a part also of one of the straight lines \drawUnitLine{EC} or \drawUnitLine{EB} is in the plane of reference, and a part in another.

But, if the part \drawLine{FG,GB,BC,CF} of the triangle \drawLine{EB,BC,CE} is in the plane of reference, and the rest in another, then a part also of both the straight lines \drawUnitLine{EC} and \drawUnitLine{EB} is in the plane of reference and a part in another, which was proved absurd. \inprop[prop:XI.I]

Therefore the triangle \drawLine{EB,BC,CE} lies in one plane. 

But, in whatever plane the triangle \drawLine{EB,BC,CE} is, each of the straight lines \drawUnitLine{EC} and \drawUnitLine{EB} also is, and in whatever plane each of the straight lines \drawUnitLine{EC} and \drawUnitLine{EB} is, \drawUnitLine{AB} and \drawUnitLine{CD} also are. \inprop[prop:XI.I]

Therefore the straight lines \drawUnitLine{AB} and \drawUnitLine{CD} are in one plane; and every triangle is in one plane.

\qed
\stopProposition

\startProposition[title={Prop. III. Theor.}, reference=prop:XI.III]
\defineNewPicture{
	numeric w, h;
	w := 4cm;
	h := 3cm;
	byPointXYZDefine(A, 1/2w, 0, 0);
	byPointXYZDefine(A', 1/2w, h, 0);
	byPointXYZDefine(A'', -1/2w, 0, 0);
	byPointXYZDefine(A''', -1/2w, h, 0);
	byPointXYZDefine(C, 0, 0, -1/2w);
	byPointXYZDefine(C', 0, h, -1/2w);
	byPointXYZDefine(C'', 0, 0, 1/2w);
	byPointXYZDefine(C''', 0, h, 1/2w);
	byPointXYZDefine(B, 0, h, 0);
	byPointXYZDefine(D, 0, 0, 0);
	byPointXYZDefine(F, 0, 1/2h, -1/8w);
	byPointXYZDefine(E, 1/8w, 1/2h, 0);
	bySetProjection(-25, 60, -90);
	draw byPolygon(B, C''', C'', D)(byred);
	draw byPolygon(D, B, A''', A'')(byyellow);
	draw byPolygon(D, A, A', B)(byyellow);
	draw byPolygon(C, D, B, C')(byred);
	draw byLine(D, B, byblack, 0, 0);
	draw byArbitraryCurve (B, F, D)(byblack, 1, 0)(BFD);
	draw byArbitraryCurve (B, E, D)(byblue, 1, 0)(BED);
	byPointLabelRemove(A',A'',A''',C',C'',C''');
	draw byLabelsOnPolygon(C, C', B, A', A, D)(0, 0);
	draw byLabelLineEnd(E, F, 0);
	draw byLabelLineEnd(F, E, 0);
}
\drawCurrentPictureInMargin
\problemNP{I}{f}{two planes cut one another, their common section is a straight line.}

For let the two planes \drawPolygon{DBA'''A'',DAA'B} and \drawPolygon{BC'''C''D,CDBC'} cut one another, and let the line \drawUnitLine{DB} be their common section.

I say that the line \drawUnitLine{DB} is a straight line.

For, if not, join the straight line 
\drawFromCurrentPicture[middle][lineBED]{
startAutoLabeling;
draw byNamedArbitraryCurve(BED);
stopAutoLabeling;
}
from \drawPointL{B} to \drawPointL{D} in the plane \drawPolygon{DBA'''A'',DAA'B}, and the straight line 
\drawFromCurrentPicture[middle][lineBFD]{
startAutoLabeling;
draw byNamedArbitraryCurve(BFD);
stopAutoLabeling;
} 
in the plane \drawPolygon{BC'''C''D,CDBC'}.

Then the two straight lines \lineBED\ and \lineBFD\ have the same ends and clearly enclose an area, which is absurd.

$\therefore$ \lineBED\ and \lineBFD\ are not straight lines.

Similarly we can prove that neither will there be any other straight line joined from \drawPointL{B} to \drawPointL{D} except \drawUnitLine{DB}, the common section of the planes \drawPolygon{DBA'''A'',DAA'B} and \drawPolygon{BC'''C''D,CDBC'}.

\qed
\stopProposition

\startProposition[title={Prop. IV. Theor.}, reference=prop:XI.IV]
\defineNewPicture{
	numeric w, h, d;
	w := 3cm;
	h := 3cm;
	d := 4cm;
	byPair(A, B, C, D, E, G, H);
	A := (-1/2w, 1/2h);
	B := (1/2w, -1/2h);
	C := (1/2w, 1/2h);
	D := (-1/2w, -1/2h);
	E = whatever[A, B] = whatever[C, D];
	G := 1/2[A, D];
	H = whatever[E, G] = whatever[B, C];
	byPointXYZDefine(F, 0, 0, d);
	bySetProjection(-60, 0, 35);
	byLineDefine(A, E, byred, 0, 0);
	byLineDefine(E, B, byred, 1, 0);
	byLineDefine(C, E, byblue, 0, 0);
	byLineDefine(E, D, byblue, 1, 0);
	byLineDefine(E, F, byblack, 0, 1);
	byLineDefine(G, E, byyellow, 0, 0);
	byLineDefine(E, H, byyellow, 1, 0);
	byLineDefine(A, F, byred, 0, 2);
	byLineDefine(B, F, byred, 1, 2);
	byLineDefine(C, F, byblue, 0, 2);
	byLineDefine(D, F, byblue, 1, 2);
	byLineDefine(G, F, byyellow, 0, 2);
	byLineDefine(H, F, byyellow, 1, 2);
	byLineDefine(A, G, byblack, 1, 1);
	byLineDefine(G, D, byblack, 1, 2);
	byLineDefine(C, H, byblack, 1, 2);
	byLineDefine(H, B, byblack, 1, 1);
	draw byAngle(D, A, E, byred, 0);
	draw byAngle(F, A, D, byyellow, 0);
	draw byAngle(A, E, G, byblue, 0);
	draw byNamedLine(AG,CH,EF);
	draw byNamedLineSeq(0)(AF, CF, CE, AE);
	draw byAngle(G, E, F, byblack, 1);
	draw byAngle(H, E, F, byblack, 0);
	draw byAngle(B, E, H, byblue, 0);
	draw byAngle(E, B, C, byred, 1);
	draw byAngle(F, B, C, byyellow, 1);
	draw byNamedLine(GD,HB);
	draw byNamedLineSeq(0)(GF,GE,EH,HF);
	draw byNamedLineSeq(0)(DF,ED,EB,BF);
	draw byLabelsOnPolygon(F, A, E)(2, 0); 
	draw byLabelsOnPolygon(D, G, F, C, H, B, E)(0, 0); 
}
\drawCurrentPictureInMargin
\problemNP{I}{f}{a straight line be set up at right angles to two straight lines which cut one another, at their common point of section, it will also be at right angles to the plane passing through them.}

For let a straight line \drawUnitLine{EF} be set up at right angles to the two straight lines \drawUnitLine{AB} and \drawUnitLine{CD} at \drawPointL{E}, the point at which the lines cut one another.

I say that \drawUnitLine{EF} is also at right angles to the plane passing through \drawUnitLine{AB} and \drawUnitLine{CD}.

Cut off \drawUnitLine{AE}, \drawUnitLine{EB}, \drawUnitLine{CE}, and \drawUnitLine{ED} equal to one another. Draw any straight line \drawUnitLine{GH} across through \drawPointL{E} at random. Join \drawUnitLine{AD} and \drawUnitLine{CB}, and join \drawUnitLine{FA}, \drawUnitLine{FG}, \drawUnitLine{FD}, \drawUnitLine{FC}, \drawUnitLine{FH}, and \drawUnitLine{FB} from a point \drawPointL{F} taken at random on \drawUnitLine{EF}. (\inpropL[prop:XI.II] \inpropN[prop:I.III])

Now, since the two straight lines \drawUnitLine{AE} and \drawUnitLine{ED} equal the two straight lines \drawUnitLine{CE} and \drawUnitLine{EB} and contain equal angles, therefore the base \drawUnitLine{AD} equals the base \drawUnitLine{CB}, and the triangle \drawLine{AE,ED,DA} equals the triangle \drawLine{EC,CB,BE}, so that $\drawAngle{GAE} = \drawAngle{HBE}$. (\inpropL[prop:I.XV] \inpropN[prop:I.IV])

But $\drawAngle{AEG} = \drawAngle{BEH}$, $\therefore$ \drawLine{AE,EG,GA} and \drawLine{BE,EH,HB} are two triangles which have two angles equal to two angles respectively, and one side equal to one side, namely that adjacent to the equal angles, that is to say, $\drawUnitLine{AE} = \drawUnitLine{EB}$. Therefore they also have the remaining sides equals to the remaining sides, that is, $\drawUnitLine{GE} = \drawUnitLine{EH}$, and $\drawUnitLine{AG} = \drawUnitLine{BH}$. (\inpropL[prop:I.XV] \inpropN[prop:I.XXVI])

And, since $\drawUnitLine{AE} = \drawUnitLine{EB}$, while \drawUnitLine{FE} is common and at right angles, therefore the base \drawUnitLine{FA} equals the base \drawUnitLine{FB}. 

For the same reason, $\drawUnitLine{FC} = \drawUnitLine{FD}$. \inprop[prop:I.IV]

And, since $\drawUnitLine{AD} = \drawUnitLine{CB}$, and $\drawUnitLine{FA} = \drawUnitLine{FB}$, the two sides \drawUnitLine{FA} and \drawUnitLine{AD} equal the two sides \drawUnitLine{FB} and \drawUnitLine{BC} respectively, and the base \drawUnitLine{FD} was proved equal to the the base \drawUnitLine{FC}, therefore the angle \drawAngle{FAD} also equals the angle \drawAngle{FBC}. \inprop[prop:I.VIII]

And since, again, \drawUnitLine{AG} was proved equal to \drawUnitLine{BH}, and further, $\drawUnitLine{FA} = \drawUnitLine{FB}$, the two sides \drawUnitLine{FA} and \drawUnitLine{AG} equal the two sides \drawUnitLine{FB} and \drawUnitLine{BH}, and the angle \drawAngle{FAG} was proved equal to the angle \drawAngle{HBF}, therefore the base $\drawUnitLine{FG} = \drawUnitLine{FH}$. \inprop[prop:I.IV]

Again, since \drawUnitLine{GE} was proved equal to \drawUnitLine{EH}, and \drawUnitLine{EF} is common, the two sides \drawUnitLine{GE} and \drawUnitLine{EF} equal the two sides \drawUnitLine{HE} and \drawUnitLine{EF}, and the base \drawUnitLine{FG} equals the base \drawUnitLine{FH}, $\therefore \drawAngle{GEF} = \drawAngle{HEF}$. \inprop[prop:I.VIII]

Therefore each of the angles \drawAngle{GEF} and \drawAngle{HEF} is right.

$\therefore$ \drawUnitLine{FE} is at right angles to \drawUnitLine{GH} drawn at random through \drawPointL{E}.

Similarly we can prove that \drawUnitLine{FE} also makes right angles with all the straight lines which meet it and are in the plane of reference. 

But a straight line is at right angles to a plane when it makes right angles with all the straight lines which meet it and are in that same plane, therefore \drawUnitLine{FE} is at right angles to the plane of reference. \indef[def:XI.III]

But the plane of reference is the plane through the straight lines \drawUnitLine{AB} and \drawUnitLine{CD}.

Therefore \drawUnitLine{FE} is at right angles to the plane through \drawUnitLine{AB} and \drawUnitLine{CD}.

\qed
\stopProposition

\startProposition[title={Prop. V. Theor.}, reference=prop:XI.V]
\defineNewPicture[1/4]{
	numeric w, h, d;
	w := 4cm;
	h := 3cm;
	d := 3cm;
	byPair(B, D, E, F, Pnw, Pne, Pse, Psw);
	Pnw := (0, 0);
	Pne := (w, 0);
	Pse := (w, -h);
	Psw := (0, -h);
	B := (1/4w, -1/2h);
	D := B shifted (1/2w, 0);
	E := (3/4w, -3/4h);
	F := (3/4w, -1/4h);
	byPointXYZDefine(A, xpart(B), ypart(B), d);
	byPointXYZDefine(C, xpart(F), ypart(F), 1/2d);
	byPointXYZDefine(A', xpart(F), ypart(F), d);
	byPointXYZDefine(A'', xpart(F), ypart(F), -1/2d);
	byPointXYZDefine(A''', xpart(B), ypart(B), -1/2d);
	byPointLabelRemove(Pnw, Pne, Pse, Psw, A', A'', A''');
	bySetProjection(-60, 0, 60);
	draw byPolygon(B, F, A'', A''')(byblue);
	draw byPolygon(Pnw, Pne, Pse, Psw)(byyellow);
	draw byPolygon(A, A', F, B)(byblue);
	byLineDefine(A, B, byblack, 0, 0);
	byLineDefine(B, C, byred, 0, 0);
	byLineDefine(B, D, byblue, 0, 0);
	byLineDefine(B, E, byblue, 1, 0);
	byLineDefine(B, F, byred, 1, 0);
	draw byNamedLine(BC, BD, BF);
	draw byNamedLineSeq(0)(AB, BE);
	draw byAngle(A, B, C, byred, 0);
	draw byAngle(C, B, F, byred, 1);
	%draw byAngle(A, B, D, byblack, -1);
	%draw byAngle(A, B, E, byblack, -1);
	draw byLabelsOnPolygon(B, A, C, F, D, E)(0, 0);
}
\drawCurrentPictureInMargin
\problemNP{I}{f}{a straight line be set up at right angles to three straight lines which meet one another, at their common point of section, the three straight are in one plane.}

For let a straight line \drawUnitLine{AB} be set up at right angles to the three straight lines \drawUnitLine{BC}, \drawUnitLine{BD} and \drawUnitLine{BE} at their point of meeting at \drawPointL{B}.

I say that \drawUnitLine{BC}, \drawUnitLine{BD}, and \drawUnitLine{BE} are in one plane. 

For suppose that they are not, but, if possible, let \drawUnitLine{BD} and \drawUnitLine{BE} be in the plane of reference \drawPolygon[middle][planeOfReference]{PnwPnePsePsw} and \drawUnitLine{BC} in one more elevated. Produce the plane \drawPolygon[middle][otherPlane]{AA'FB,BFA''A'''} through \drawUnitLine{AB} and \drawUnitLine{BC}. \inprop[prop:XI.III]

\otherPlane\ intersects \planeOfReference\ in a straight line. Let the intersection be \drawUnitLine{BF}. Therefore the three straight lines \drawUnitLine{AB}, \drawUnitLine{BC}, and \drawUnitLine{BF} are in one plane \otherPlane, namely that drawn through \drawUnitLine{AB} and \drawUnitLine{BC}. 

Now, since \drawUnitLine{AB} is at right angles to each of the straight lines \drawUnitLine{BD} and \drawUnitLine{BE}, therefore \drawUnitLine{AB} is also at right angles to the plane \planeOfReference\ through \drawUnitLine{BD} and \drawUnitLine{BE}. \inprop[prop:XI.IV]

But the plane \planeOfReference\ through \drawUnitLine{BD} and \drawUnitLine{BE} is the plane of reference, therefore \drawUnitLine{AB} is at right angles to the plane of reference. 

Thus \drawUnitLine{AB} also makes right angles with all the straight lines which meet it and lie in the plane of reference. \indef[def:XI.III]

But \drawUnitLine{BF}, which is the plane of reference, meets it, therefore the angle \drawAngle{FBA} is right. And, by hypothesis, the angle \drawAngle{ABC} is also right, therefore the angle $\drawAngle{FBA} = \drawAngle{ABC}$, and they lie in one plane, which is impossible.

Therefore the straight line \drawUnitLine{BC} is not in a more elevated plane. Therefore the three straight lines \drawUnitLine{BC}, \drawUnitLine{BD}, and \drawUnitLine{BE} are in one plane.

\qed
\stopProposition

\startProposition[title={Prop. VI. Theor.}, reference=prop:XI.VI]
\defineNewPicture{
	numeric w, h, d;
	w := 4cm;
	h := 3cm;
	d := 2cm;
	byPair(B, D, E, Pnw, Pne, Pse, Psw);
	Pnw := (0, 0);
	Pne := (w, 0);
	Pse := (w, -h);
	Psw := (0, -h);
	B := (1/4w, -1/6h);
	D := B shifted (1/2w, 0);
	E := D shifted (0, -d);
	byPointXYZDefine(A, xpart(B), ypart(B), d);
	byPointXYZDefine(C, xpart(D), ypart(D), d);
	byPointLabelRemove(Pnw, Pne, Pse, Psw);
	bySetProjection(-75, 0, -60);
	draw byPolygon(Pnw, Pne, Pse, Psw)(white);
	draw byAngle(A, B, D, byblue, 0);
	draw byAngle(B, D, E, byyellow, 0);
	draw byAngle(C, D, B, byblack, 0);
	draw byAngle(C, D, E, byblack, 1);
	byLineDefine(A, B, byblack, 0, 0);
	byLineDefine(C, D, byblack, 1, 0);
	byLineDefine(B, D, byblue, 0, 0);
	byLineDefine(D, E, byblue, 1, 0);
	byLineDefine(B, E, byred, 0, 0);
	byLineDefine(A, D, byred, 1, 0);
	byLineDefine(A, E, byyellow, 0, 0);
	draw byNamedLineSeq(0)(CD, BD);
	draw byAngle(E, D, A, byred, 0);
	draw byNamedLineSeq(0)(AD, DE);
	draw byNamedLineSeq(0)(AB, BE, AE);
	draw byAngle(A, B, E, byred, 1);
	draw byLabelsOnPolygon(C, E, B, A)(0, 0);
	draw byLabelsOnPolygon(E, D, B)(2, 0);
}
\drawCurrentPictureInMargin
\problemNP{I}{f}{two straight lines be at right angles to the same plane, the straight lines are parallel.}

For let the two straight lines \drawUnitLine{AB} and \drawUnitLine{CD} be at right angles to the plane of reference \drawPolygon[middle][planeOfReference]{PnwPnePsePsw}.

I say that $\drawUnitLine{AB} \parallel \drawUnitLine{CD}$.

Let them meet the plane of reference \planeOfReference\ at the points \drawPointL{B} and \drawPointL{D}. 

Join the straight line \drawUnitLine{BD}. Draw \drawUnitLine{DE} in \planeOfReference $\perp \drawUnitLine{BD}$, and make $\drawUnitLine{DE} = \drawUnitLine{AB}$. (\inpropL[prop:I.XI], \inpropL[prop:I.III])

Now, since \drawUnitLine{AB} is at right angles to \planeOfReference, it also makes right angles with all the straight lines which meet it and lie in the plane of reference.  \indef[def:XI.III]

But each of the straight lines \drawUnitLine{BD} and \drawUnitLine{BE} lies in \planeOfReference\ and meets \drawUnitLine{AB}, therefore each of the angles \drawAngle{ABD} and \drawAngle{ABE} is right. For the same reason each of the angles \drawAngle{CDB} and  \drawAngle{CDE} is also right. 

And since in \drawLine{AD,DB,BA} and \drawLine{DE,EB,BD} $\drawUnitLine{AB} = \drawUnitLine{DE}$, and \drawUnitLine{BD} is common, therefore the two sides \drawUnitLine{AB} and \drawUnitLine{BD} equal the two sides \drawUnitLine{ED} and \drawUnitLine{DB}. And they include right angles \drawAngle{ABD} and \drawAngle{BDE}, therefore the base $\drawUnitLine{AD} = \drawUnitLine{BE}$. \inprop[prop:I.IV]

And, since in \drawLine{AE,EB,BA} and \drawLine{DA,AE,ED} $\drawUnitLine{AB} = \drawUnitLine{DE}$ while $\drawUnitLine{AD} = \drawUnitLine{BE}$, the two sides \drawUnitLine{AB} and \drawUnitLine{BE} equal the two sides \drawUnitLine{ED} and \drawUnitLine{DA}, and \drawUnitLine{AE} is their common base, therefore the angle \drawAngle{ABE} equals the angle \drawAngle{EDA}. \inprop[prop:I.VIII]

But the angle \drawAngle{ABE} is right, therefore the angle \drawAngle{EDA} is also right. Therefore \drawUnitLine{ED} is at right angles to \drawUnitLine{DA}. 

But it is also at right angles to each of the straight lines \drawUnitLine{BD} and \drawUnitLine{DC}, therefore \drawUnitLine{ED} is set up at right angles to the three straight lines \drawUnitLine{BD}, \drawUnitLine{DA}, and \drawUnitLine{DC} at their intersection. Therefore the three straight lines \drawUnitLine{BD}, \drawUnitLine{DA}, and \drawUnitLine{DC} lie in one plane. \inprop[prop:XI.V]

But in whatever plane \drawUnitLine{DB} and \drawUnitLine{DA} lie, \drawUnitLine{AB} also lies, for every triangle lies in one plane. \inprop[prop:XI.II]

Therefore the straight lines \drawUnitLine{AB}, \drawUnitLine{BD}, and \drawUnitLine{DC} are in one plane. And each of the angles \drawAngle{ABD} and \drawAngle{BDC} is right, therefore \drawUnitLine{AB} is parallel to \drawUnitLine{CD}. \inprop[prop:I.XXVIII]

\qed
\stopProposition

\startProposition[title={Prop. VII. Theor.}, reference=prop:XI.VII]
\defineNewPicture[1/4]{
	numeric w, h, d;
	w := 4cm;
	h := 3cm;
	d := 2cm;
	byPair(A, B, C, D, E, F, H, Pnw, Pne, Pse, Psw);
	Pnw := (0, 0);
	Pne := (w, 0);
	Pse := (w, -h);
	Psw := (0, -h);
	A := (1/4w, -1/4h);
	B := A shifted (1/2w, 0);
	C := A shifted (0, -1/2h);
	D := B shifted (0, -1/2h);
	E := 1/3[A, B];
	F := 3/4[C, D];
	H := 1/2[E, F];
	byPointXYZDefine(G, xpart(H), ypart(H), 1/4d);
	byPointXYZDefine(E', xpart(E), ypart(E), d);
	byPointXYZDefine(F', xpart(F), ypart(F), d);
	byPointXYZDefine(E'', xpart(E), ypart(E), -d);
	byPointXYZDefine(F'', xpart(F), ypart(F), -d);
	byPointLabelRemove(Pnw, Pne, Pse, Psw, E', F', E'', F'');
	bySetProjection(-45, 0, -75);
	draw byPolygon(E, F, F'', E'')(byyellow);
	draw byPolygon(Pnw, Pne, Pse, Psw)(white);
	byLineDefine(A, E, byblue, 0, 0);
	byLineDefine(E, B, byblue, 0, 0);
	byLineDefine(C, F, byblue, 1, 0);
	byLineDefine(F, D, byblue, 1, 0);
	byLineDefine(E, F, byred, 1, 0);
	draw byNamedLine(EB, FD);
	draw byPolygon(E, F, F', E')(byyellow);
	draw byArbitraryCurve (E, G, F)(byred, 0, 0)(EGF);
	draw byNamedLineSeq(0)(AE, EF, CF);
	draw byLabelsOnPolygon(E, G, F)(2, 0);
	draw byLabelsOnPolygon(A, E, B, D, F, C)(0, 0);
}
\drawCurrentPictureInMargin
\problemNP{I}{f}{two straight lines are parallel and points be taken at random on each of them, the straight line joining the points is in the same plane with the parallel straight lines.}

Let \drawUnitLine{AB} and \drawUnitLine{CD} be two parallel straight lines, and let points \drawPointL{E} and \drawPointL{F} be taken at random on them respectively.

I say that the straight line joining the points \drawPointL{E} and \drawPointL{F} lies in the same plane \drawPolygon[middle][planeOfReference]{PnwPnePsePsw} with the parallel straight lines.

For suppose it is not, but, if possible, let it be in a more elevated plane. Draw a plane \drawPolygon[middle][planeEGF]{EFF'E',EFF''E''} through \drawFromCurrentPicture[middle][lineEGF]{
startAutoLabeling;
draw byNamedArbitraryCurve(EGF);
stopAutoLabeling;
}. Its intersection with the plane of reference is a straight line. Let it be \drawUnitLine{EF}. \inprop[prop:XI.III]

$\therefore$ the two straight lines \lineEGF\ and \drawUnitLine{EF} enclose an area, which is impossible. $\therefore$ the straight line joined from \drawPointL{E} to \drawPointL{F} is not in a plane more elevated. $\therefore$ the straight line joined from \drawPointL{E} to \drawPointL{F} lies in the plane through the parallel straight lines \drawUnitLine{AB} and \drawUnitLine{CD}.

$\therefore$, if two straight lines are parallel and points are taken at random on each of them, then the straight line joining the points is in the same plane with the parallel straight lines. 

\qed
\stopProposition

\startProposition[title={Prop. VIII. Theor.}, reference=prop:XI.VIII]
\defineNewPicture{
	numeric w, h, d;
	w := 4cm;
	h := 3cm;
	d := 2cm;
	byPair(B, D, E, Pnw, Pne, Pse, Psw);
	Pnw := (0, 0);
	Pne := (w, 0);
	Pse := (w, -h);
	Psw := (0, -h);
	B := (1/4w, -1/6h);
	D := B shifted (1/2w, 0);
	E := D shifted (0, -d);
	byPointXYZDefine(A, xpart(B), ypart(B), d);
	byPointXYZDefine(C, xpart(D), ypart(D), d);
	byPointXYZDefine(A', xpart(B), ypart(B), 6/5d);
	byPointXYZDefine(C', xpart(D), ypart(D), 6/5d);
	byPointLabelRemove(Pnw, Pne, Pse, Psw, A', C');
	bySetProjection(-75, 0, -60);
	draw byPolygon(Pnw, Pne, Pse, Psw)(white);
	draw byPolygon(B, D, C', A')(byyellow);
	draw byAngle(A, B, D, byblue, 0);
	draw byAngle(B, D, E, byyellow, 0);
	draw byAngle(C, D, B, byblack, 0);
	draw byAngle(C, D, E, byblack, 1);
	byLineDefine(A, B, byblack, 0, 0);
	byLineDefine(C, D, byblack, 1, 0);
	byLineDefine(B, D, byblue, 0, 0);
	byLineDefine(D, E, byblue, 1, 0);
	byLineDefine(B, E, byred, 0, 0);
	byLineDefine(A, D, byred, 1, 0);
	byLineDefine(A, E, byyellow, 1, 0);
	draw byNamedLineSeq(0)(CD, BD);
	draw byAngle(E, D, A, byred, 0);
	draw byNamedLineSeq(0)(AD, DE);
	draw byNamedLineSeq(0)(AB, BE, AE);
	draw byAngle(A, B, E, byred, 1);
	draw byLabelsOnPolygon(C, E, B, A)(0, 0);
	draw byLabelsOnPolygon(E, D, B)(2, 0);
}
\drawCurrentPictureInMargin
\problemNP{I}{f}{two straight lines be parallel, and one of them be at right angles to any plane, then the remaining one will also be at right angles to the same plane.}

Let \drawUnitLine{AB} and \drawUnitLine{CD} be two parallel straight lines, and let one of them, \drawUnitLine{AB}, be at right angles to the plane of reference \drawPolygon[middle][planeOfReference]{PnwPnePsePsw}.

I say that the remaining one, \drawUnitLine{CD}, is also at right angles to the same plane.

Let \drawUnitLine{AB} and \drawUnitLine{CD} meet the plane of reference at the points \drawPointL{B} and \drawPointL{D}. Join \drawUnitLine{BD}. Then \drawUnitLine{AB}, \drawUnitLine{CD}, and \drawUnitLine{BD} lie in one plane. \inprop[prop:XI.VII]

Draw \drawUnitLine{DE} in \planeOfReference\ $\perp \drawUnitLine{BD}$, make $\drawUnitLine{DE} = \drawUnitLine{AB}$, and join \drawUnitLine{BE}, \drawUnitLine{AE}, and \drawUnitLine{AD}. (\inpropL[prop:I.XI], \inpropL[I.III])

Now, since $\drawUnitLine{AB} \perp \planeOfReference$, therefore \drawUnitLine{AB} is also at right angles to all the straight lines which meet it and lie in \planeOfReference. Therefore each of the angles \drawAngle{ABD} and \drawAngle{ABE} is right. \indef[def:XI.III]

And, since the straight line \drawUnitLine{BD} falls on the parallels \drawUnitLine{AB} and \drawUnitLine{CD}, $\therefore \drawAngle{ABD} + \drawAngle{CDB} \drawTwoRightAngles$.

But the angle \drawAngle{ABD} is right, therefore the angle \drawAngle{CDB} is also right. $\therefore \drawUnitLine{CD} \perp \drawUnitLine{BD}$. \inprop[prop:I.XXIX]

And since in \drawLine{AD,DB,BA} and \drawLine{DE,EB,BD} $\drawUnitLine{AB} = \drawUnitLine{DE}$, and \drawUnitLine{BD} is common, the two sides \drawUnitLine{AB} and \drawUnitLine{BD} equal the two sides \drawUnitLine{ED} and \drawUnitLine{DB}, and  $\drawAngle{ABD} = \drawAngle{EDB}$, for each is right, therefore the base $\drawUnitLine{AD} = \drawUnitLine{BE}$. \inprop[prop:I.IV]

And since in \drawLine{DE,EB,BD} and \drawLine{ED,DA,AE}$\drawUnitLine{AB} = \drawUnitLine{DE}$, and $\drawUnitLine{BE} = \drawUnitLine{AD}$, the two sides \drawUnitLine{AB} and \drawUnitLine{BE} equal the two sides \drawUnitLine{ED} and \drawUnitLine{DA} respectively, and \drawUnitLine{AE} is their common base, $\therefore \drawAngle{ABE} = \drawAngle{EDA}$. \inprop[prop:I.VIII]

But the angle \drawAngle{ABE} is right, therefore the angle \drawAngle{EDA} is also right. $\therefore \drawUnitLine{ED} \perp \drawUnitLine{AD}$. But it is also $\perp \drawUnitLine{DB}$. $\therefore$ \drawUnitLine{ED} is also at right angles to the plane through \drawUnitLine{BD} and \drawUnitLine{DA}. \inprop[prop:XI.IV]

$\therefore$ \drawUnitLine{ED} also makes right angles with all the straight lines which meet it and lie in the plane through \drawUnitLine{BD} and \drawUnitLine{DA}. But \drawUnitLine{DC} lies in the plane \drawPolygon{BDC'A'} through \drawUnitLine{BD} and \drawUnitLine{DA} inasmuch as \drawUnitLine{AB} and \drawUnitLine{BD} lie in the plane \drawPolygon{BDC'A'} through \drawUnitLine{BD} and \drawUnitLine{DA}, and \drawUnitLine{DC} also lies in the plane \drawPolygon{BDC'A'} in which \drawUnitLine{AB} and \drawUnitLine{BD} lie.

$\therefore \drawUnitLine{ED} \perp \drawUnitLine{DC}$, so that $\drawUnitLine{CD} \perp \drawUnitLine{DE}$. But $\drawUnitLine{CD} \perp \drawUnitLine{BD}$. $\therefore$ \drawUnitLine{CD} is set up at right angles to the two straight lines \drawUnitLine{DE} and \drawUnitLine{DB} so that \drawUnitLine{CD} is also at right angles to the plane through \drawUnitLine{DE} and \drawUnitLine{DB}. \inprop[prop:XI.IV]

But the plane through \drawUnitLine{DE} and \drawUnitLine{DB} is the plane of reference \planeOfReference, $\therefore \drawUnitLine{CD} \perp \planeOfReference$.

Therefore, if two straight lines are parallel, and one of them is at right angles to any plane, then the remaining one is also at right angles to the same 

\qed
\stopProposition

\startProposition[title={Prop. IX. Theor.}, reference=prop:XI.IX]
\defineNewPicture{
	numeric w, h, d;
	w := 4cm;
	h := 3cm;
	d := 2cm;
	byPair(Pnw, Pne, Pse, Psw);
	Pnw := (0, 0);
	Pne := (w, 0);
	Pse := (w, -h);
	Psw := (0, -h);
	byPointXYZDefine(A, 1/4w, -1/4h, 0);
	byPointXYZDefine(B, 3/4w, -1/4h, 0);
	byPointXYZDefine(C, 1/4w, -3/4h, 0);
	byPointXYZDefine(D, 3/4w, -3/4h, 0);
	byPointXYZDefine(E, 1/4w, -1/2h, d);
	byPointXYZDefine(F, 3/4w, -1/2h, d);
	byPointXYZEmpty(G, H, K, Eab', Fab', Ecd', Fcd', A', B', C', D', H', K');
	pointXYZ.G := 1/2[pointXYZ.E, pointXYZ.F];
	pointXYZ.H := 1/2[pointXYZ.A, pointXYZ.B];
	pointXYZ.K := 1/2[pointXYZ.C, pointXYZ.D];
	pointXYZ.Eab' := 5/4[pointXYZ.A, pointXYZ.E];
	pointXYZ.Fab' := 5/4[pointXYZ.B, pointXYZ.F];
	pointXYZ.A' := 5/4[pointXYZ.E, pointXYZ.A];
	pointXYZ.B' := 5/4[pointXYZ.F, pointXYZ.B];
	pointXYZ.Ecd' := 5/4[pointXYZ.C, pointXYZ.E];
	pointXYZ.Fcd' := 5/4[pointXYZ.D, pointXYZ.F];
	pointXYZ.C' := 5/4[pointXYZ.E, pointXYZ.C];
	pointXYZ.D' := 5/4[pointXYZ.F, pointXYZ.D];
	pointXYZ.H' := (redpart(pointXYZ.H), greenpart(pointXYZ.H), bluepart(pointXYZ.G));
	pointXYZ.K' := (redpart(pointXYZ.K), greenpart(pointXYZ.K), bluepart(pointXYZ.G));
	byPointLabelRemove(Pnw, Pne, Pse, Psw, Eab', Fab', Ecd', Fcd', A', B', C', D', H', K');
	bySetProjection(-75, 0, -70);
	draw byPolygonWithName(H, H', G)(byblue)(planeHK'');
	draw byPolygonWithName(E, Ecd', Fcd', F)(byred)(planeGK'');
	draw byPolygonWithName(A', A, B, B')(byyellow)(planeGH');
	draw byPolygonWithName(C', C, D, D')(byred)(planeGK');
	draw byPolygonWithName(Pnw, Pne, Pse, Psw)(white)(refPlane);
	draw byPolygonWithName(Eab', Fab', B, A)(byyellow)(planeGH);
	byLineDefine(A, H, byred, 0, 0);
	byLineDefine(H, B, byred, 0, 0);
	byLineDefine(C, K, byyellow, 0, 0);
	byLineDefine(K, D, byyellow, 0, 0);
	byLineDefine(E, G, byblue, 0, 0);
	byLineDefine(G, F, byblue, 0, 0);
	byLineDefine(H, G, byblack, 0, 0);
	byLineDefine(G, K, byblack, 1, 0);
	draw byNamedLine(AH, HB);
	draw byPolygonWithName(H, G, K)(byblue)(planeHK');
	draw byNamedLine(HG);
	draw byPolygonWithName(E, F, D, C)(byred)(planeGK);
	draw byNamedLine(CK, KD, EG, GF);
	draw byPolygonWithName(K, K', G)(byblue)(planeHK);
	draw byNamedLine(GK);
	draw byLabelsOnPolygon(A, E, G, F, D, K, C)(0, 0);
	draw byLabelsOnPolygon(B, H)(4, 0);
}
\drawCurrentPictureInMargin
\problemNP{S}{traight}{lines which are parallel to the same straight line and are not in the same plane with it are also parallel to one another.}

For let each of the straight lines \drawUnitLine{AB}, \drawUnitLine{CD} be parallel to \drawUnitLine{EF}, not being in the same plane \drawPolygon{refPlane} with it. I say that $\drawUnitLine{AB} \parallel \drawUnitLine{CD}$.

For let a point \drawPointL{G} be taken at random on \drawUnitLine{EF}, and from it let there be drawn \drawUnitLine{GH}, in the plane \drawPolygon{planeGH,planeGH'} through \drawUnitLine{EF}, \drawUnitLine{AB}, at right angles to \drawUnitLine{EF}, and \drawUnitLine{GK} in the plane \drawPolygon{planeGK,planeGK',planeGK''} through \drawUnitLine{FE}, \drawUnitLine{CD} again at right angles to \drawUnitLine{EF}.

Now, since \drawUnitLine{EF} is at right angles to each of the straight lines \drawUnitLine{GH}, \drawUnitLine{GK}, $\therefore \drawUnitLine{EF} \perp$ to the plane \drawPolygon[middle][planeHK]{planeHK,planeHK',planeHK''} through \drawUnitLine{GH}, \drawUnitLine{GK}. \inprop[prop:XI.IV]

And $\drawUnitLine{EF} \parallel \drawUnitLine{AB}$, $\therefore \drawUnitLine{AB} \perp \planeHK$. \inprop[prop:XI.VIII]

For the same reason  $\drawUnitLine{CD} \perp \planeHK$. $\therefore$ each of the straight lines \drawUnitLine{AB}, \drawUnitLine{CD} $\perp \planeHK$.

But, if two straight lines be at right angles to the same plane, the straight lines are parallel \inprop[prop:XI.VI].

Therefore $\drawUnitLine{AB} \parallel \drawUnitLine{CD}$.

\qed
\stopProposition

\stoptext